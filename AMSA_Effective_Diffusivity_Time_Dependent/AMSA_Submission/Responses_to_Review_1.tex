\documentclass[amsa]{article}
\usepackage{mymacros}

% \setlength{\textwidth}{6.5in}
% \setlength{\textheight}{9.0in}
% \setlength{\oddsidemargin}{0in}
% \setlength{\evensidemargin}{0in}
% %\setlength{\topmargin}{-0.5in}


\begin{document}

\title{Responses to Review 1 for\\
  ``Spectral analysis and computation\\
  of effective diffusivities in 
  space-time periodic incompressible flows''}
\author{N. B. Murphy}

\maketitle





In order to make the substantial revision of our submitted manuscript
more transparent, we now provide a synopsis of the modifications. The 
results of the manuscript remain the same. However, virtually the
entire manuscript has been modified. The responses to the referee's
questions/comments are given below.   


Section 1 was split into three subsections. Section 2 was split into
three subsections and the body of Sections 2.2 and 2.3 were
significantly modified. Section 3 was streamlined and placed on more
general footing by adding Appendix E.2. Section 4 has been
substantially modified. A conclusion section was added in Section
5.


Appendices A and B are virtually the same.
Appendix C has been
almost completely rewritten. In particular, the Hilbert spaces and
their dense subsets have been redefined to have elements that are
spatially mean-zero. To precisely characterize the fluid velocity
field, in Appendix C.1 a new Hilbert space $\Hs_{\Vc}^{0,2}$ was
defined. Moreover, in Appendix C.1 the negative inverse Laplacian
$(-\Delta)^{-1}$ and the associated Green's function are now discussed in
detail, as its properties are needed 
in the proof of a new lemma, Lemma 2, and the proof of Theorem 1
etc. Appendix C.2 has been substantially modified. In particular, we
have moved some of the proof of what is now Theorem 6 to a new lemma,
Lemma 3, in Appendix C.2. Lemmas 2 and 3 are now used heavily in the
proof of Corollary 4 in Appendix C.2. The proof of what is now Lemma 5 
has been only slightly modified. In Appendix D a good portion of the
old proof of Theorem 6 has been moved to Lemma 3, as mentioned
above. We have added to Appendix E a subsection, Appendix
E.2. However, the introductory paragraph of Appendix E and Appendix
E.1, which comprised all of Appendix E in the submitted manuscript,
are virtually the same. Appendix E.2 was added to streamline our
results given in Section 3 and to place them on more general footing,
as mentioned above.   



We now address the reviewer's questions/comments.



\begin{itemize}
\item[1.]
  Throughout the paper, the sentence “the domain $D(M)$ of an operator
  $M$ is densely defined in the underlying Hilbert space” is a bit
  confusing. I think something like “the domain $D(M)$ is dense” or
  “$M$ is densely defined” is clearer. Or if these are not exactly
  what you mean, define the first sentence at the beginning.
%
\item
  The phrase "densely defined" has been eliminated and the suggested
  changes have been utilized.
%
\item[2.]
  p.8, below eq.(8), can you comment a little bit more on the
  conditions to have uniform $(\sup_{x\in\mathbb{R}^d})$ convergence?
%
\item
  The result in eq. (8) is stated in reference [55], \textit{Majda and
    Kramer, Physics Reports 213, 237 (1999).} Therein, the proof of
  this result is referenced to the article \textit{A.J. Majda,
    Lectures on turbulent diffusion, Lecture Notes at Princeton
    University, 1990.} We have tried to obtain an copy of this article
  without success, as it is not readily available to the public. We
  have not tried to reproduce the result and instead have referenced
  [55].  
%
\item[3.]
  p.8, below eq.(10), make the definition of $\Tc$ and $\Vc$ explicit,
  namely $V = [0, 2\pi]^d$ and $\Tc = [0, 2\pi]$ or other
  choices. Make sure these are consistent with the Fourier series used
  in p.12 eq.(19).
%
\item
  Below eq.(10), we have given the examples that you have
  suggested. These are consistent with the Fourier series used in
  Section 3. 
%
\item[4.]
  p.9, line 3, in the definition of $(-\Delta)^{-1}$, specify which
  boundary condition is associated to the  problem, namely periodic
  boundary condition with zero average. This operator is used
  throughout the paper; it worths to make this clarification.
%
\item
  We have made the suggested clarification after the statement of
  Theorem 1 in Appendix C1 of the revised paper. 
%
\item[5.] p.18, line 5 in paragraph 3, the sentence “whenever $\{f_n\}$
  and $\{\Phi f_n\}$ exist” sounds strange.
%
\item
  We have fixed this sentence.
%
\item[6.] p.19, eq.(33), it is better to refer to a specific theorem
  with specific definition of $Q(\lambda)$, and assumptions on $\Sigma$ (or the
  operator $\Phi$).
%
\item We have made the reference to the spectral theorem, the operator
  $\Phi$, and its spectrum $\Sigma$ more specific.
%
\item[7.]  p.19, eq.(34)(35) are defined for $\lambda\in\Sigma$ only, right? Are
  they extended somehow to the whole line/plane, or is $\lambda\in\Sigma$ assumed to
  be a line/plane?
%
\item Yes, these equations are defined for $\lambda\in\Sigma$ only. We have made
  this more explicit in the paper. Also the spectrum $\Sigma$ of a
  self-adjoint operator is real-valued, $\Sigma\subseteq\mathbb{R}$. We have made
  this more explicit in the paper as well.
%
\item[8.]  p.23, in C.1., last line,
  $\langle\nabla\phi\cdot\nabla\varphi\rangle=\int_0^1\int_{[0,1]^d}\nabla\phi\cdot\nabla\varphi dxdt$ does not define an inner
  product, since if $\varphi$ is a function of the $t$ variable only, then
  this product is zero but $\varphi$ is not necessary the zero of
  $\Hs$. Also, for eq.(46), specify with respect to which norms of
  $\tilde{\As}_{\Tc}$ and $\Hs^{1}_{\Vc}$ are the completion done.
%  
\item The Sobolev space $\Hs_{\Vc}^{1}$ is now denoted by
  $\Hs_{\Vc}^{1,2}$. The problem that you pointed out has been fixed
  by defining $\Hs_{\Vc}^{1,2}$ to have spatially mean-zero
  elements. Otherwise, non-zero constant $\psi$ satisfies
  $\int_{\Vc}|\bnabla\psi|^2\,\d\vecx=0$.  With this change,
  $\langle\nabla\phi\cdot\nabla\varphi\rangle=\int_{\Tc\times\Vc}\nabla\phi\cdot\nabla\varphi\, \d\vecx\,\d t$ \emph{does} define an inner 
  product on the function space $\tilde{\As}_{\Tc}\otimes\Hs_{\Vc}^{1,2}$. We have
  discussed this in detail in the revision. Also, the function space
  $\tilde{\As}_{\Tc}$ is \emph{not} a complete Hilbert space and is
  instead an everywhere dense subset of the Hilbert space
  $\Hs_{\Tc}$. We have clarified this in the paper and also specified
  on which function space and with what norm the completion of the
  Sobolev space $\Hs^{1,2}_{\Vc}$ is done. 
%
\item[9.]  p.24, line 1, why is $\|\partial_t\psi\|_1<\infty$?
%
\item The norm $\|\cdot\|_1$ is now denoted $\|\cdot\|_{1,2}$. Since
  $\langle|\partial_t\psi|^2\rangle_{\Tc}<\infty$ for all $\psi\in\tilde{\As}_{\Tc}$ and $\langle|\bnabla
  \psi|^2\rangle_{\Vc}<\infty$ for all $\psi\in\Hs^{1,2}_{\Vc}$, we have
  $\|\partial_t\psi\|_{1,2}^2=\langle|\bnabla \partial_t\psi|^2\rangle<\infty$ for all
  $\psi\in\Fs$, where $\Fs=\tilde{\As}_{\Tc}\otimes\Hs^{1,2}_{\Vc}$. To avoid
  confusion we have removed this sentence, as it is not used in the
  paper.  
%
\item[10.]  p.25, I am not sure about the derivation to the bound in
  eq.(49). $\Fs$ is defined through a completion, shouldn't it be the
  Sobolev space $H^1(\Tc\times\Vc)$ of $L^2$ integrable functions on the
  torus with $L^2$ integrable (weak) derivatives? In that case,
  Sobolev embedding indicates that a $L^\infty$ bound would fail in $d\geq2$.
%  
\item The issue with the derivation to the bound in eq.(49) of the
  submitted paper has been fixed in the revision and the result has
  been generalized, now given in Lemma 2 of the revised paper. 
%
\item[11.] The above estimate is used also in p.26, eq.(50), p.29,
  line 6 below eq.(56), etc. Please make sure the above question is
  clarified.
\item The question in item 10 has been clarified in Lemma 2.
%
\item[12.] p.30, below eq.(62), the relation $\Jb_k =\bsig\Eb_k$ is
  not “local”. In fact, this relation is a partial differential
  equation. It worths to mention this though the seemingly “local”
  formula is a nice analog with the corresponding constitutional
  equation in the Maxwell’s equations.
%
\item This point has been clarified in the paper.
%
\item[13.]  p.31, eq.(65), indeed $\Ab=\bGamma\Sb\bGamma=\bGamma\Sb$
  since on $\Hc$, the vector fields are already curl free. However, it
  is not clear to me that $\bGamma\Sb\bGamma=\Sb\bGamma$ (and hence
  $\bGamma\Sb = \Sb\bGamma$), because it is not clear to me that the
  range of $\Sb$ is contained in $\Hc$. I don't see where $\bGamma\Sb
  = \Sb\bGamma$ is needed, in any case.
%  
\item These equalities were intended in a weak sense that has been
  clarified in equation (A-52) of the revised paper.
%
\item[14.] p.32, Lemma 4 and its proof, double check the indices in
  $\sigma^*_{jk}$. For instance, on p. 33, last sentence in paragraph 1, is
  $\langle\vecJ_j\bcdot\vecE_k\rangle=\langle\vecJ_j\bcdot\vece_k\rangle$ equal to $\sigma^*_{jk}$
  or $\sigma^*_{kj}$? The same question for the fourth line in eq.(71). If
  the latter (instead the current version), then in eq.(70), there
  should be no ``transpose.'' 
%
\item Since $\langle\vecJ_j\rangle=\bsig^*\langle\vecE_j\rangle=\bsig^*\vece_j$, we have
  $\langle\vecJ_j\bcdot\vecE_k\rangle=\langle\vecJ_j\bcdot\vece_k\rangle=\bsig^*\vece_j\bcdot\vece_k=\sigma^*_{jk}$.
  This and the formula $\Dm^*_{jk}=\varepsilon\delta_{jk}+\langle u_j\chi_k\rangle$ imply that
  $\sigma^*_{jk}=\langle\vecJ_j\bcdot\vece_k\rangle=\langle u_k\chi_j\rangle+\varepsilon\delta_{jk}+\langle\Hm_{jk}\rangle
  =\Dm^*_{kj}+\langle\Hm_{jk}\rangle$, yielding $\bsig^*=[\Dm^*]^T+\langle\Hm\rangle$.
%
\item[15.] p.34, last 3 lines in paragraph 2, for $f\in\Hs^1_{\Vc}$, $\partial
  f/\partial x_j$ could be defined as stated here, except that the limit of
  of $\{\partial f_n/\partial x_j\}$ should be sought for in $\|\cdot\|$ topology rather
  than $\|\cdot\|_1$ (which itself involves a derivative, according to the
  definition in eq.(45)). Also, it is not clear that $\{\partial^2 f_n/\partial
  x_j^2\}$  will have a limit, since $\Hs^1_{\Vc}$ is taken as the
  completion w.r.t. a norm that only concerns first order
  derivatives.
%
\item We have updated these statements according to your suggestions
  and moved them to a more appropriate place below equation (A-15) of
  the revised paper, where the Sobolev space is defined.
%
\item[16.]  Examine the consequence of item 8 above on the norms
  appearing in Appendix D.
%
\item We have fixed the problem that you pointed out in item 8. Please
  see the corresponding response for details.

\noindent Here are list of typos I found:

%
\item[17.] p.7, last line $\phi(0,x)$ should be $\phi^\delta$.
%
\item We have fixed this typo.
%
\item[18.] p.23, line 4 in section C.1., in $\Vc=\times_{j=1}^d$, $\times$
  should be $\otimes$? 
%
\item We have fixed this typo.
%  
\item[19.] p.31, line 1, $\Hc$ should be $\Fs$?
%
\item This line was removed in the revision.
%
\item[20.] p.32, line 3 from bottom of paragraph 1, ``$\Sb=\ldots$ is
  self-adjoint'' should be ``$-\imath\Sb=\ldots$ is self-adjoint''?
%  
\item This line was removed in the revision.
%  
\end{itemize}


\end{document}
