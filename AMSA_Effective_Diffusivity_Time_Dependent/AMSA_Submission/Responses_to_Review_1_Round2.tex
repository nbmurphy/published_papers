\documentclass[amsa]{article}

\usepackage{mymacros}

\usepackage[utf8]{inputenc}
\usepackage{amsmath}


\setlength{\textwidth}{5.35in}
\setlength{\textheight}{9in}
\setlength{\oddsidemargin}{0in}
\setlength{\evensidemargin}{0in}
\setlength{\topmargin}{-1in}


\begin{document}

\title{Second Response to Reviewer 1 for\\
  ``Spectral analysis and computation\\
  of effective diffusivities in 
  space-time periodic incompressible flows''}
\author{N. B. Murphy}

\maketitle

We have made several changes to the manuscript to address the
questions of the reviewer. Here, we provide an overview of the changes
made. Below, we provide more detailed answers to the questions of the
reviewer. Our first comment regards item 1. Since the paper is already
63 pages in length, in the interest of brevity, the authors have
decided  to defer the lengthy treatment of general, \emph{unbounded}
flows to a future article. Consequently, the results of Lemma 2 and
Theorem 1 of Appendix C.1 as well as Corollary 4 of Appendix C.2 of
the current manuscript have been restricted to treat general
\emph{bounded} flows. Accordingly, Appendix C.1 has been significantly
rewritten and Appendix C.2 has been only slightly augmented to reflect
the changes made to Appendix C.1. We have also made various, very minor
changes throughout the manuscript to improve readability. Our second
comment regards item 2. We have decided to keep our notation for the
``effective conductivity.'' Please see our more detailed responses to
these two items below.

------------

The authors have revised their manuscript very carefully and answered
all of my questions in the previous report. I believe the current
manuscript is ready for publication, after the following points are
clarified. The first one is a bit technical, and the second one is a
very minor comment.  


\begin{itemize}
\item[1]  p.33, for eq. (A-27), I think if we apply (A-23) directly to
the function $\vecu\bcdot\bnabla f$, it will require the H{\"o}lder
components of (A-27) to satisfy $1/p_1+1/q_1=1/2$, because in
$\|\cdot\|_{1,2}$ the square of the function is integrated. More precisely,
%
\begin{align}
  \|(-\Delta)^{-1}(\vecu\bcdot\bnabla) f\|^2_{1,2}
  =\int_{\Tc}\int_{\Vc}|\bnabla((-\Delta)^{-1}(\vecu\bcdot\bnabla)f|^2
  \stackrel{\text{(A-23)}}{=}
  C\int_{\Tc}\|\vecu\bcdot\bnabla f\|^2_{L^2_x}
\end{align}
%
and if $1/p+1/p^\prime=1/2$
%
\begin{align}
  \|\vecu\bcdot\bnabla f\|_{L^2_x}
  \leq \|\vecu\bcdot\bnabla f\|_{L^p_x}\|\vecu\bcdot\bnabla f\|_{L^{p^\prime}_x}
\end{align}
%
For this reason, the requirement (A-31) cannot be made since
$1/p_1+1/q_1=1/2$ already forces $p_1\geq2$ and $q_1\geq2$.



I still have the feeling that a requirement of $r$ where $r$ is the
component $\vecu\in L^r_t(L^r_x)$ necessarily depends on the dimension
$d$. Essentially, (A-25) and (A-26) are $W_x^{1,2}$ estimates (I don’t
see how the time integral of absolute values of functions would help
to improve the estimates) for $m$, the solution of the problem
%
\begin{align}
  -\Delta m = h:= \vecu\cdot\nabla f.
\end{align}
%
We essentially require $\nabla f\in L_x^2$. So if $u\in L^r_x$, then $h\in L^p_x$
with $p=2r/(r+2)$, i.e.
%
\begin{align}
  1/2+1/r=1/p.
\end{align}
%
Then elliptic regularity (Calder{\'o}n-Zygmund) says $m\in W^{2,p}$. In
view of Sobolev embedding, if we want $m\in W^{1,2}$, a natural
sufficient condition would be (assume $d\geq2$)
%
\begin{align}
  1/p^*=1/p-1/d\leq1/2 ~\text{ i.e. }~ p^*\geq2,
\end{align}
%
which requires $r\geq d$.
%


\item \textbf{In response to the comments made in the first paragraph of item
  1, regarding equations (1) and (2).}

  Denote by $\|\cdot\|_p$ the $L^p(\Tc\times\Vc)$--norm. Integration by
  parts and periodicity, the
  triangle inequality $|\langle h\rangle|\leq\langle|h|\rangle$, Young's inequality
  $\|(-\Delta)^{-1}\psi\|_p\leq C\|\psi\|_p$, and H{\"o}lder's   inequality
  $\|f\,g\|_1\leq\|f\|_{p_1}\,\|g\|_{q_1}$ with conjugate exponents 
  satisfying $(1/p_1)+(1/q_1)=1$ and $1\leq p_1,\,q_1\leq\infty$, yield equation
  (A-27):  
%
\begin{align}
  \|(-\Delta)^{-1}(\vecu\bcdot\bnabla) f\|^2_{1,2}
  &=|\langle \bnabla[(-\Delta)^{-1}(\vecu\bcdot\bnabla)f]\bcdot
      \bnabla[(-\Delta)^{-1}(\vecu\bcdot\bnabla)f]\rangle|
  \\\notag
  &=|\langle [(-\Delta)^{-1}(\vecu\bcdot\bnabla f)]
      \;(\vecu\bcdot\bnabla f)\rangle|
   ~\text{ (integration by parts)}
  \\\notag
  &\leq\langle |(-\Delta)^{-1}(\vecu\bcdot\bnabla f)|
      \;|\vecu\bcdot\bnabla f|\rangle
   ~\text{ (triangle inequality)}
  \\\notag
  &=\| |(-\Delta)^{-1}(\vecu\bcdot\bnabla f)|
      \;|\vecu\bcdot\bnabla f| \|_1
    ~\text{ (definition of } \|\cdot\|_1)
  \\\notag
  &\leq \| (-\Delta)^{-1}(\vecu\bcdot\bnabla f)\|_{p_1}\;
     \| \vecu\bcdot\bnabla f \|_{q_1}
     ~\text{ (H{\"o}lder's Inequality) }
  \\\notag   
  &\leq C\,\| \vecu\bcdot\bnabla f\|_{p_1}\;
     \| \vecu\bcdot\bnabla f \|_{q_1}
     ~\text{ (Young's Inequality) }
  \notag  
\end{align}
%

\textbf{In response to the remaining comments made in the second
  paragraph of item 1.}

As mentioned above, we have decided to defer our treatment of
general \emph{unbounded} flows to a future article, restricting our
results of the current manuscript to general \emph{bounded} flows.


  
%
\item[2]
Regarding the formula (A-60), I had a question about whether it should
be
%
\begin{align}
  \sigma_{jk}^*=\langle \vecJ_j\cdot\vece_k\rangle
  ~\text{ or }~
  \sigma_{jk}^*=\langle \vecJ_k\cdot\vece_j\rangle
\end{align}
%
According to (A-49), I would say $\sigma_{jk}^*$ being the element at the
$j$th row and $k$th element, should be $\vece_j\cdot(\bsig^*\vece_k)$,
should be given by $\langle\vece_j\cdot\vecJ_k\rangle$ (the latter choice
above). Then I will get 
%
\begin{align}
  \text{ LHS of (A-61) }
  =\sigma_{kj}^*
  =
  \cdots\stackrel{\text{4th row}}{=} \langle\chi_ju_k\rangle+\varepsilon\delta_{jk}+\langle H_{kj}\rangle
  =D^*_{kj}+\langle H_{kj}\rangle
\end{align}
%
(both the authors' and my choices of $A=(A_{jk})$ are consistent for
$\bsig^*$ and for $H$) and finally
%
\begin{align}
  \bsig^*=[D^*]+\langle H\rangle.  
\end{align}
%



\item The calculation in equation (A-54) of the revised manuscript
  seems correct. If we instead define $\sigma_{jk}^*=\langle \vecJ_k\cdot\vece_j\rangle$
  then the indices $j$ and $k$ interchange in all equalities on the
  right hand side of the first one, i.e., $\sigma^*_{jk}=\,$, yielding
  $\sigma^*_{jk}=\Dm^*_{jk}+\langle\Hm_{kj}\rangle$, so that
  $\sigma^*=\Dm^*+\langle\Hm^T\rangle=\Dm^*-\langle\Hm\rangle$. We have decided to keep our
  notation $\sigma_{jk}^*=\langle \vecJ_j\cdot\vece_k\rangle$ to be consistent with
  the analytic continuation method for representing transport in
  composite materials. We have 
  changed the subscript   notation of the manuscript text in between
  and including equations   (A-39)--(A-42) to reduce confusion.
\end{itemize}


\end{document}
