\documentclass[amsa]{article}
\usepackage{mymacros}

% \setlength{\textwidth}{6.5in}
% \setlength{\textheight}{9.0in}
% \setlength{\oddsidemargin}{0in}
% \setlength{\evensidemargin}{0in}
% %\setlength{\topmargin}{-0.5in}


\begin{document}

\title{Responses to Review 2 for\\
  ``Spectral analysis and computation\\
  of effective diffusivities in 
  space-time periodic incompressible flows''}
\author{N. B. Murphy}

\maketitle

In order to make the substantial revision of our submitted manuscript
more transparent, we now provide a synopsis of the modifications. The 
results of the manuscript remain the same. However, virtually the
entire manuscript has been modified. The responses to the referee's
questions/comments are given below.   


Section 1 was split into three subsections. Section 2 was split into
three subsections and the body of Sections 2.2 and 2.3 were
significantly modified. Section 3 was streamlined and placed on more
general footing by adding Appendix E.2. Section 4 has been
substantially modified. A conclusion section was added in Section
5.


Appendices A and B are virtually the same.
Appendix C has been
almost completely rewritten. In particular, the Hilbert spaces and
their dense subsets have been redefined to have elements that are
spatially mean-zero. To precisely characterize the fluid velocity
field, in Appendix C.1 a new Hilbert space $\Hs_{\Vc}^{0,2}$ was
defined. Moreover, in Appendix C.1 the negative inverse Laplacian
$(-\Delta)^{-1}$ and the associated Green's function are now discussed in
detail, as its properties are needed 
in the proof of a new lemma, Lemma 2, and the proof of Theorem 1
etc. Appendix C.2 has been substantially modified. In particular, we
have moved some of the proof of what is now Theorem 6 to a new lemma,
Lemma 3, in Appendix C.2. Lemmas 2 and 3 are now used heavily in the
proof of Corollary 4 in Appendix C.2. The proof of what is now Lemma 5 
has been only slightly modified. In Appendix D a good portion of the
old proof of Theorem 6 has been moved to Lemma 3, as mentioned
above. We have added to Appendix E a subsection, Appendix
E.2. However, the introductory paragraph of Appendix E and Appendix
E.1, which comprised all of Appendix E in the submitted manuscript,
are virtually the same. Appendix E.2 was added to streamline our
results given in Section 3 and to place them on more general footing,
as mentioned above.   



We now address the reviewer's questions/comments.


\begin{itemize}
\item[1.] Overall, the manuscript is difficult to follow. The text
  would need more structure. The reader is sometimes lost in a very
  long section, and does not know what has been addressed, what
  remains to be done,...  It would be nice to split Sections 1 and 2
  into several subsections, and to describe, at the beginning of each
  of them, what will be done in that subsection. The reader needs a
  better guideline.   
%
\item  In order to give the reader a better guide to the paper, in the
  revised manuscript we have split various sections into subsections
  with introductory paragraphs. We have also given more prompts about
  where the current section is leading.  
%
\item[2.] In order to convey their message, I think that the authors
  should work out a simple example (for instance, a purely diffusive
  problem in a time-independent setting) very early in their
  manuscript, in an informal manner. The different quantities at hand
  in this work (the corrector problem, the eigenvalue problem (A) 
  below, the spectral measure $\mu$, the operators $Q(\lambda),\ldots$) would
  appear, and their relation one to each other would be very clear. It
  would help the reader understand the overall strategy in a simple
  case.   
%
\item We have worked out a specific case in Appendix E.1 and also made
  its connection between the eigenvalue problem, the spectral measure,
  and the operator $Q(\lambda)$ clearer in the paragraph containing equations
  (22) and (23), early on in the revised manuscript. 
 
%
\item[3.] The velocity field $u$ is assumed to be divergence-free (see
  page 6). What remains true if this is not the case?  
%
\item In the introduction, we have alerted the reader to two papers
  that treat the problem of homogenizing an advection-diffusion
  equation with a compressible flow. 1) R. M. McLaughlin and
  M. G. Forest, Phys. Fluids, 11(4):880--892, 1999. 2)
  G. C. Papanicolaou, Surveys in Applied Mathematics, chapter
  Diffusion in Random Media, pages 205-253, 1995. 
%
\item[4.] The fact that the authors have to call to spectral theory
  for \emph{unbounded} operators seems to be related to the fact that
  the problem is time-dependent (see e.g. page 10, “Due to the
  time-dependence . . . ”). However, this link is never clearly
  explained. In Appendix B page 21, it is explained that the operator
  $\partial_t$ is unbounded. But this also holds for the operator $\Delta$, which
  appears in the time-independent setting. I hence do not understand
  the particularity of the time-dependent setting.  
%
\item It is true that both of the operators $\partial_t$ and $\Delta$ are
  unbounded on $L^2(\Tc)$ and $L^2(\Vc)$, respectively. However, it is
  the operator $(-\Delta)^{-1}$ that appears in the spectral theory of
  effective diffusivity, which is smoothing and compact. For this
  reason, only the unboundedness of the operator $\partial_t$ must be
  addressed in the theory.
 
%
\item[5.] Top of page 12, the authors introduce an eigenvalue problem,
  %
  \begin{align}
    A\varphi_l=\imath\lambda_l\varphi_l. 
  \end{align}
  %
  The reason for introducing this problem is not explained. At that
  stage, the reader is concerned with a right-hand side problem,
  namely the corrector problem (9). Of course, when one knows the
  eigenelements of an operator, one knows how to solve a right-hand
  side problem (this seems to be the meaning of the line just above
  Eq. (80) page 38), but this is not the standard manner to
  proceed. Please comments on this choice and why the authors focus on
  eigenvalue problems. Furthermore, the link between this eigenvalue
  problem and the spectral measure $\mu$ is not clearly explained (or
  maybe it is in the Appendices, but this is somehow too late for the
  reader . . . ). 
%
\item This information was in the Appendices. We have moved this 
  information to equations (19)--(23) in the revised manuscript, 
  which clearly explains the link between this eigenvalue   problem
  and the spectral measure $\mu$. 
 
\item[6.] In terms of mathematical content, Section 3 seems to be much
  simpler than the previous sections: the authors consider an
  eigenvalue problem, they have a basis for the Hilbert space, so they
  can write the problem as an algebraic eigenvalue problem, in a space
  of dimension infinite but countable. This infinite matrix problem is
  next truncated. Maybe it would be worth explaining at the beginning
  of Section 3 this overall strategy.
%
\item We have added an introductory paragraph to Section 3 that
  explains this overall strategy.
%
\item[7.] What is the reason for solving the eigenvalue problem
  introduced at the beginning of Section 3 on a Fourier basis? I
  understand that this eigenvalue problem is complemented by periodic
  boundary conditions, but it is unclear to me that a Fourier method
  is the method of choice. 
%
\item We have made the following comment below equation (30) in the
  revised manuscript. Using the orthonormal trigonometric basis
  functions $\phi_{\ell,\,\veck}(t,\vecx)=\exp[\imath(\ell t+\veck\bcdot\vecx)]$
  leads to an exact representation of the spectral measure weights
  $m_{jk}(l)=\langle u_j,\varphi_l\rangle\,\overline{\langle u_k,\varphi_l\rangle}$ which involves only a
  finite number of terms. Of course, we could have used a different
  orthonormal basis. However, the spectral weights would then be given
  by an infinite series. In other words, the velocity field is
  represented by six Fourier terms in equation (28) of the revised
  manuscript, which would require infinitely many terms in other
  orthogonal basis.
%
\item[8.] In (15), the integral over $\lambda$ seems to imply that the
  eigenvalues form a continuum. In contrast, the index $l$ in the
  eigenvalue problem introduced at the beginning of Section 3 seems to
  imply that the eigenvalues are countable. Please clarify.
%  
\item We have clarified this in equations (19)--(23) in the revised
  manuscript. 
%
\item[9.] In Section 4, I understand that the numerical parameter $M$
  stands for the index at which the infinite matrix problem is
  truncated. It is however not precised how many eigenvalues are
  computed. Where do we stop in terms of the index $l$ in Eq. (A)
  above? This is an important point as the homogenized quantities read
  (see Eqs. (11), (12) and (15)) as an integral over $\lambda$.
%
\item In the revised manuscript we have replaced the numerical
  parameter $M$ with $N$, as $M$ also represents the self-adjoint
  operator $M=-\imath A$. We compute \emph{all} of the eigenvalues of the
  symmetric matrix $\Cm^{-1/2}\Bm\Cm^{-1/2}$. In our computations, we
  used for the steady case $N=150$, yielding matrices of size
  $(2N+1)^2-1=90,600$, while in the dynamic case we used $N=20$,
  yielding matrices of size $(2N+1)[(2N+1)^2-1]=68,880$. This
  information is given in the two paragraphs below equation (37) of
  the revised manuscript.   
%
\end{itemize}  
  
Here are also some additional comments:

\begin{itemize}
% 
\item[1.] In the abstract, the authors write that they consider
  \emph{periodic} velocity fields, in the time-dependent
  setting. Before that, there is a sentence on the state of the art in 
  the time-independent setting. In that sentence, make precise that
  the velocity-field is also \emph{periodic}. 
%  
\item The abstract mentions ``Two different formulations for integral
  representations for $\Dm^*$ were developed of the case of
  \emph{time-independent} fluid velocity fields$\ldots$'' The abstract also
  mentions ``Here, we extend both of these approaches to the case of
  \emph{space-time periodic} velocity fields$\ldots$'' The two different
  formulations are presented in 1) M. Avellaneda and
  A. Majda. Comm. Math. Phys., 138:339--391, 1991. and
  2) G. A. Pavliotis, PhD thesis, Rensselaer 
  Polytechnic Institute Troy, New York, 2002. The first paper provides 
  integral representations for $\Dm^*$ involving a \emph{stochastic}
  time-independent fluid velocity field. The second paper provides
  integral representations for $\Dm^*$ involving a \emph{periodic} 
  time-independent fluid velocity field. For the sake of brevity in
  the abstract, this distinction is not made. Although, this
  distinction is made in the introduction.
%
\item[2.] Bottom of page 6: the authors define $\overline{\varphi}$, the
  complex conjugate of $\varphi$ and next write that all quantities
  considered in that section are real-valued. This definition could
  thus be postponed to a later location in the manuscript.
%  
\item We have mentioned this definition throughout the revised
  manuscript. Although, the authors feel that it is also appropriate
  to provide the definition used throughout the paper at its first
  appearance. We have made this clear below equation (2) in the
  revised manuscript to avoid confusion.
%  
\item[3.] Page 8, three lines below Eq. (8): what do the authors mean
  by ``For fixed $0<\delta\ll1$ $\ldots$''as $\delta$ is meant to go to 0 (to converge
  to the homogenized problem)?
%
\item We have removed this part of the sentence to avoid confusion.
%
\item[4.] Bottom of page 8: the authors consider the operator
  $(-\Delta)^{-1}$ but do not make precise the boundary conditions they
  consider.
%  
\item We made the boundary conditions precise below the statement of
  Theorem 1 in the revised manuscript.
%  
\end{itemize}


\end{document}
