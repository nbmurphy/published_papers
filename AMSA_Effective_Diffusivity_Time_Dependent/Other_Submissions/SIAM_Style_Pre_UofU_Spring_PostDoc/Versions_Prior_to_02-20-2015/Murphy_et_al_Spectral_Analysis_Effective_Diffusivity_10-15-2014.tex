%\documentclass[final,leqno,onefignum,onetabnum]{siamltex1213}
\documentclass[leqno,onefignum,onetabnum]{siamltex1213}

\usepackage{graphicx,amssymb,amsmath,amsfonts,mathrsfs}

\newcommand{\ph}{\hat{\phi}}
\newcommand{\pt}{\tilde{\phi}} 
\newcommand{\pc}{\check{\phi}}
\newcommand{\gh}{\hat{\gamma}}
\newcommand{\Dh}{\hat{\Delta}}
\newcommand{\dha}{\hat{\delta}}
\newcommand{\qh}{\hat{q}}
\newcommand{\xh}{\hat{x}} 
\newcommand{\HM}{\mathcal{H}_{\text{max}}}
\newcommand{\Hm}{\mathcal{H}_{\text{min}}}
\newcommand{\sech}{\rm \hspace{0.7mm}sech}
\newcommand{\I}{\mathrm{i}}
\newcommand{\e}{\mathrm{e}}
\renewcommand{\d}{\mathrm{d}}
\newcommand{\hh}{\hat{h}}
\newcommand{\mh}{m_r}
\newcommand{\mt}{m_i}
\newcommand{\Sigc}{\Sigma_{\text{cont}}}
\newcommand{\Sigp}{\Sigma_{\text{pp}}}

\newcommand{\Mb}{\mathbf{M}}
\newcommand{\Xb}{\mathbf{X}}
\newcommand{\Tb}{\mathbf{T}}
\newcommand{\Hb}{\mathbf{H}}
\newcommand{\Kb}{\mathbf{K}}
\newcommand{\Jb}{\mathbf{J}}
\newcommand{\Ib}{\mathbf{I}}
\newcommand{\Sb}{\mathbf{S}}
\newcommand{\Rb}{\mathbf{R}}
\newcommand{\Ab}{\mathbf{A}}
\newcommand{\Bb}{\mathbf{B}}
\newcommand{\Cb}{\mathbf{C}}
\newcommand{\Db}{\mathbf{D}}
\newcommand{\Eb}{\mathbf{E}}
\newcommand{\Qb}{\mathbf{Q}}
\newcommand{\Nb}{\mathbf{N}}
\newcommand{\Ob}{\mathbf{0}}
\newcommand{\Vb}{\mathbf{V}}
\newcommand{\Wb}{\mathbf{W}}
\newcommand{\Gb}{\mathbf{G}}
\newcommand{\Ub}{\mathbf{U}}
\newcommand{\Zb}{\mathbf{Z}}



\newcommand{\Kc}{\mathcal{K}}
\newcommand\Kbc{\mbox{\boldmath${\mathcal{K}}$}}


\newcommand{\Tc}{\mathcal{T}}
\newcommand{\Vc}{\mathcal{V}}
\newcommand{\Hc}{\mathcal{H}}
\newcommand{\Fc}{\mathcal{F}}
\newcommand{\Ec}{\mathcal{E}}
\newcommand{\Sc}{\mathcal{S}}
\newcommand{\Mc}{\mathcal{M}}
\newcommand{\Nc}{\mathcal{N}}
\newcommand{\Uc}{\mathcal{U}}



\newcommand{\Hs}{\mathscr{H}}
\newcommand{\As}{\mathscr{A}}
\newcommand{\Ds}{\mathscr{D}}
\newcommand{\Fs}{\mathscr{F}}
\newcommand{\Ss}{\mathscr{S}}


\newcommand\bsig{\mbox{\boldmath${\sigma}$}}
\newcommand\beps{\mbox{\boldmath${\epsilon}$}}
\newcommand\bxi{\mbox{\boldmath${\xi}$}}
\newcommand\bmu{\mbox{\boldmath${\mu}$}}
\newcommand\balpha{\mbox{\boldmath${\alpha}$}}
\newcommand\brho{\mbox{\boldmath${\rho}$}}
\newcommand\bDelta{\mbox{\boldmath${\Delta}$}}
\newcommand\bkappa{\mbox{\boldmath${\kappa}$}}
\newcommand\bGamma{\mbox{\boldmath${\Gamma}$}}
\newcommand\bUpsilon{\mbox{\boldmath${\Upsilon}$}}
\newcommand\bLambda{\mbox{\boldmath${\Lambda}$}}
\newcommand\Bnabla{\mbox{\boldmath${\nabla}$}}




% Temporary Notation
\newcommand{\Et}{\tilde{\mathbf{E}}}
\newcommand\bphi{\mbox{\boldmath${\phi}$}}
\newcommand{\Wt}{\tilde{\mathbf{W}}}
\newcommand{\Dt}{\tilde{\mathbf{D}}}





\title{Spectral analysis and computation of effective diffusivities
  for time-dependent periodic flows  
  %\thanks{}
      } 

% \author{TeX Production\thanks{Society for Industrial and
% Applied Mathematics, Philadelphia, Pennsylvania. 
% (\email{tex@siam.org}). Questions, comments, or corrections
% to this document may be directed to that email address.}}

\author{
N. B. Murphy\footnotemark[1]\ \footnotemark[3]\ \footnotemark[4]
\and J. Xin\footnotemark[1]\ \footnotemark[3]
\and E. Cherkaev\footnotemark[2]\ \footnotemark[4]
\and J. Zhu\footnotemark[2]\ \footnotemark[4]%
}


% !!!!!!!!! NEED TO FIX THIS BEFORE SUBMISSION !!!!!!!!!!!
%
% \renewcommand{\thefootnote}{\fnsymbol{footnote}}

% \footnotetext[1]{
%   Department of Mathematics, 340 Rowland Hall, University of
%   California at Irvine, Irvine, CA 92697-3875, USA}
% \footnotetext[2]{University of Utah, Department of Mathematics, 155 S 1400 E
%   RM 233, Salt Lake City, UT 84112-009, USA}
% \footnotetext[3]{Support in common for N. B. Murphy and J. Xin}
% \footnotetext[4]{Support in common for N. B. Murphy, E. Cherkaev, and J. Zhu}

% \renewcommand{\thefootnote}{\arabic{footnote}}
      

       




\begin{document}
\maketitle
\slugger{siap}{xxxx}{xx}{x}{x--x}%slugger should be set to mms, siap,
                                 %sicomp, sicon, sidma, sima, simax,
                                 %sinum, siopt, sisc, or sirev 

\begin{abstract}
The enhancement in diffusive transport of particles or tracers by
incompressible, turbulent flow fields is a challenging problem with
theoretical and practical importance in many areas of science and
engineering, ranging from the transport of mass, heat, and pollutants
in geophysical flows to turbulent combustion and stellar
convection. The long time, large scale behavior of such systems is
equivalent to an enhanced diffusive process with an effective
diffusivity tensor $\Db^*$. Based on an analytic continuation method
developed for random composite materials, a rigorous integral
representation for $\Db^*$ was developed for the case of a random,
\emph{time-independent} fluid velocity field, involving a spectral
measure of a self-adjoint random operator acting on
\emph{vector-fields}. An alternate approach yielded an integral
representation for $\Db^*$ involving a spectral measure of a
self-adjoint operator acting on \emph{scalar-fields}, for the case of
a periodic, \emph{time-independent} fluid velocity field. Here, we
adapt and extend both of these approaches to the case of a periodic,
\emph{time-dependent} fluid velocity field, with possibly chaotic
dynamics, providing integral representations for $\Db^*$ involving
spectral measures of the underlying self-adjoint operators. We prove
that the two approaches are equivalent and that their correspondence
follows from a one-to-one isometry between the underlying Hilbert
spaces. Moreover, we establish a direct correspondence between the
effective parameter problem for $\Db^*$ and that arising in the
analytic continuation method for composites. We also develop novel
Fourier methods that provide the mathematical foundation for rigorous
computation of $\Db^*$. Our numerical computations are in excellent
agreement with known theoretical results.           
\end{abstract}

\begin{keywords}
advective diffusion, effective diffusivity, eddy diffusivity, spectral
measure, multiscale homogenization, turbulence, residual diffusion
\end{keywords}

\begin{AMS}
% OPERATOR THEORY: Special classes of linear operators: Hermitian and
% normal operators (spectral measures, functional calculus, etc.)
47B15,
% NUMERICAL ANALYSIS: Numerical linear algebra: Eigenvalues, eigenvectors
65C60,
% CLASSICAL THERMODYNAMICS, HEAT TRANSFER: Basic methods: Homogenization
%80M40,
% PARTIAL DIFFERENTIAL EQUATIONS: Representations of solutions:
% Integral representations of solutions 
35C15,
%FLUID MECHANICS: Incompressible inviscid fluids: None of the above,
%but in this section 
76B99
%FLUID MECHANICS: Basic methods in fluid mechanics: Spectral methods
76M22
%FLUID MECHANICS: Basic methods in fluid mechanics: Homogenization
76M50
%FLUID MECHANICS: Turbulence: Turbulent transport, mixing
76F25
%FLUID MECHANICS: Diffusion and convection: None of the above, but in
%this section 
76R99   	
% GEOPHYSICS: Hydrology, hydrography, oceanography
%86A05   	
\end{AMS}


\pagestyle{myheadings}
\thispagestyle{plain}
\markboth{N. B. Murphy et al.}{Spectral analysis of time-dependent flows}

\section{Introduction}
The long time, large scale motion of diffusing particles or
tracers being advected by an incompressible flow field is equivalent
to an enhanced diffusive process~\cite{Taylor:PRSL:196} with an
effective diffusivity tensor $\mathbf{D}^*$. Describing the associated
transport properties is a challenging problem with a broad range of
scientific and engineering applications, such as stellar
convection~\cite{Knobloch:1992ApJ,Press:1981:ApJ,canut98,canut98b,canut00},
turbulent
combustion~\cite{Aslanyan:BF00790149,Bilger:05:10.1016,Tabaczynski:10.1007},
and solute transport in porous
media~\cite{Bhattacharya:AAP:1999:951,Bhattacharya:1989:ASD,Whitaker:AIC690130308,Gupta:WRCR3940,Koch:1988:965,Lester:PRL:111.174101,Koch:JFM:7961001}.
Time-dependent flows can have fluid velocity fields with chaotic
dynamics, which gives rise to turbulence that greatly enhances the
mixing, dispersion, and large scale transport of diffusing scalars.   



In the climate
system~\cite{Csanady:1973:9789027702609,Griffies:2003:10.1007},
turbulence plays a key role in transporting mass, heat, momentum,
energy, nutrients, pollutants, and salt in geophysical
flows~\cite{Moffatt:RPP:621}. Turbulence enhances the dispersion of
atmospheric gases~\cite{Espinosa:MET1292} such as
ozone~\cite{Holton:JGRC2495,Pitari:JGR:1986,Plumb:JAS:1979,Plumb:JAS:1987}
and
pollutants~\cite{Bilger:10.1175,Beychok:1994:9780964458802,Samson:1988:88009978},
as well as atmosphere-ocean transfers of carbon dioxide and other
climatically important trace gas
fluxes~\cite{Zappa:2007:67613,Banerjee:10.1007}.  Longitudinal
dispersion of passive scalars in oceanic flows can be enhanced by
horizontal turbulence due to shearing of tidal currents, wind drift,
or
waves~\cite{Young:JPO:1982:515,Kullenberg:1972:TUS1529,Bowden:JFM:1965}.   
Chaotic motion of 
time-dependent fluid velocity fields cause instabilities in large
scale ocean currents, generating geostrophic
eddies~\cite{Ferrari:JPO:1501} which dominate the kinetic energy of
the ocean~\cite{Ferrari:ARFM:253}. Geostrophic
eddies greatly enhance~\cite{Ferrari:JPO:1501} the meridional mixing
of heat, carbon and other climatically important tracers, typically
more than one order of magnitude greater than the mean flow of the
ocean~\cite{Souza:OS:317}. Eddies also impact heat and salt budgets
through lateral fluxes and can extend the area of high biological
productivity offshore by both eddy chlorophyll advection and eddy
nutrient pumping~\cite{Chaigneau:JGR:C11025}. In sea ice, which 
couples the atmosphere to the polar
oceans~\cite{Washington:1986:9780935702521}, the transport of vast ice
floes can also be enhanced by eddie
fluxes~\cite{Watanabe:2009JPO4010}. Thermal transport through sea ice
can be enhanced by a brine velocity
field~\cite{Lytle:JGR-8853,Trodahl:2001:TCS}, which itself depends on
the brine microstructure and its
connectivity~\cite{Golden:S-2238,Golden:GRL:L16501}, and plays an
important role in many biogeochemical processes in polar ecosystems
and the climate~\cite{Golden:NAMS:2009}.




It has been noted in various atmospheric
contexts~\cite{Plumb:JAS:1979,Plumb:JAS:1987} that eddy-induced,
skew-diffusive tracer fluxes, directed normal to the tracer
gradient~\cite{Middleton:JPO:5840223}, are generally equivalent to
antisymmetric components in the effective 
diffusivity tensor $\Db^*$, while the symmetric part of $\Db^*$
represents irreversible diffusive
effects~\cite{Redi:JPO:1982:1154,Solomon:OGR:1971:233,Griffies:JPO:1998}
directed down the tracer gradient. The mixing of eddy fluxes is
typically non-divergent and unable to affect the evolution of the mean
flow~\cite{Middleton:JPO:5840223}, and do not alter the tracer
moments~\cite{Griffies:JPO:1998}. In this sense, the 
mixing is non-dissipative, reversible, and sometimes referred to as
stirring~\cite{Eckart:JMR:1948,Griffies:JPO:1998}.  




Due to the computational intensity of detailed climate
models~\cite{Griffies:2003:10.1007,Washington:1986:9780935702521,Neelin:2010:CCCM},
a coarse resolution is necessary in numerical simulations and
\emph{parameterization} is used to help resolve sub-grid
processes, such as turbulent 
entrainment-mixing processes in clouds~\cite{Lu:JGR:D50094},
atmospheric boundary layer turbulence~\cite{Bretherton:JOC:5655449},
atmosphere-surface exchange over the
sea~\cite{Fairall:1996:JGRC6562} and sea
ice~\cite{Sorensen:TC:2014,Andreas:2010:QJ618,Andreas:JH:2010}, and
eddies in the ocean~\cite{McDougall:2001:book,Gent:JPO:1995}. In this
way, only the effective or averaged behavior of these sub-grid
processes are included in the model. Here, we study the effective
behavior of advection enhanced diffusion by time-dependent fluid
velocity fields, with possibly chaotic dynamics, which gives rise to
such a parameterization, namely, the effective diffusivity tensor
$\Db^*$ of the flow. 



In recent decades, a broad range of mathematical techniques have been
developed~\cite{McLaughlin:SIAM_JAM:780,Bensoussan:Book:1978,Biferale:PF:2725,Fannjiang:SIAM_JAM:333,Mauri:1991:3:743,Pavliotis:PHD_Thesis,Pavliotis:CMS:2007:507,Clark:1998:364,Holmes:1995:94481954,Hornung:1997:9780387947860,Majda:Kramer:1999:book,Majda:1994:10.1088}
which reduce the analysis of enhanced diffusive transport by complex
velocity fields with rapidly varying structures in both space and time, to
solving averaged or \emph{homogenized} equations that do not have
rapidly varying data, and involve an effective parameter. Motivated
by~\cite{Papanicolaou:RF-835}, it was
shown~\cite{McLaughlin:SIAM_JAM:780} that the homogenized behavior of
the advection-diffusion equation with a random, time-independent,
incompressible, mean-zero fluid velocity field, is given by an inhomogeneous
diffusion equation involving the symmetric part of an effective
diffusivity tensor $\Db^*$. Moreover, a rigorous representation of
$\Db^*$ was given in terms of an auxiliary ``cell problem'' involving
a curl-free random field~\cite{McLaughlin:SIAM_JAM:780}. We stress
that the effective diffusivity tensor $\Db^*$ is not symmetric in
general. However, only its symmetric part appears in the homogenized
equation for this formulation of the effective transport properties of
advection enhanced diffusion~\cite{McLaughlin:SIAM_JAM:780}.   



The incompressibility condition of the velocity field was
used~\cite{Avellaneda:PRL-753,Avellaneda:CMP-339} to transform the
cell problem~\cite{McLaughlin:SIAM_JAM:780} into the quasi-static
limit of Maxwell's equations~\cite{Jackson-1999,Golden:CMP-473}, which
describe the transport properties of an electromagnetic wave in a
composite medium~\cite{MILTON:2002:TC}. The analytic continuation
method (ACM) for representing transport in
composites~\cite{Golden:CMP-473} provides Stieltjes integral
representations for the bulk transport coefficients of the media, such
as electrical conductivity and permittivity, magnetic permeability,
and thermal conductivity~\cite{MILTON:2002:TC}. This 
method is based on the spectral theorem~\cite{Stone:64,Reed-1980} and
a resolvent formula for, say, the electric field, involving a random
self-adjoint operator~\cite{Golden:CMP-473,Murphy:JMP:063506} or
matrix~\cite{Murphy:CMS:Submitted}. Based on~\cite{Golden:CMP-473},
the cell problem was transformed into a resolvent formula involving a
self-adjoint random operator, acting on the Hilbert space of
curl-free \emph{vector
  fields}~\cite{Avellaneda:PRL-753,Avellaneda:CMP-339}. This, in turn,  
led to a Stieltjes integral representation for the symmetric part of
the effective diffusivity tensor $\Db^*$, involving the P\'{e}clet
number of the flow and a \emph{spectral measure} of the
operator~\cite{Avellaneda:PRL-753,Avellaneda:CMP-339}.



The mathematical framework developed in~\cite{McLaughlin:SIAM_JAM:780}
was adapted~\cite{Pavliotis:PHD_Thesis} to the case of a periodic,
time-independent, incompressible fluid velocity field with non-zero
mean. The velocity field is modeled as a superposition of a
large-scale mean flow with small-scale periodically oscillating 
fluctuations. It was shown~\cite{Pavliotis:PHD_Thesis} that, depending
on the strength of the fluctuations relative to the mean flow, the
effective diffusivity tensor $\Db^*$ can be constant or a function of
both space and time. When $\Db^*$ is constant, only its symmetric part
appears in the homogenized equation as an enhancement in the
diffusivity. However, when $\Db^*$ is a function of space and time,
its antisymmetric part also plays a key role in the homogenized
equation. In particular, the symmetric part of $\Db^*$ appears as an
enhancement in the diffusivity, while both the symmetric and
antisymmetric parts of $\Db^*$ contribute to an effective drift in the
homogenized equation. The effective drift due to the antisymmetric
part is purely sinusoidal, thus divergence
free~\cite{Pavliotis:PHD_Thesis}. Based
on~\cite{Bhattacharya:AAP:1999:951}, the cell problem was
transformed~\cite{Pavliotis:PHD_Thesis} into a resolvent formula
involving a self-adjoint operator, acting on the Sobelov
space~\cite{McOwen:2003:PDE,Folland:95} of spatially periodic scalar
fields, which is also a Hilbert space. This, in turn, led to a
discrete Stieltjes integral representation for both the symmetric and
antisymmetric parts of $\Db^*$, involving the P\'{e}clet number of the
flow and a \emph{spectral measure} of the operator.   


Here, we
adapt and extend both of these approaches to the case of a periodic,
\emph{time-dependent} fluid velocity field, with possibly chaotic
dynamics, providing integral representations for $\Db^*$ involving
spectral measures of the underlying self-adjoint operators. We prove
that the two approaches are equivalent and that their correspondence
follows from a one-to-one isometry between the underlying Hilbert
spaces. Moreover, we establish a direct correspondence between the
effective parameter problem for $\Db^*$ and that arising in the
analytic continuation method for composites. We also develop novel
Fourier methods that provide the mathematical foundation for rigorous
computation of $\Db^*$. Our numerical computations are in excellent
agreement with known theoretical results.           
FINISH THIS PARAGRAPH WHEN THE REST OF THE PAPER IS FINISHED.






\appendix
\section{Title of appendix} 

\bibliographystyle{plain}
\bibliography{murphy}




\end{document}


% LocalWords:  siap xxxx mms sicomp sicon sidma sima simax sinum siopt sisc
% LocalWords:  sirev inviscid hydrography SIAM'S eddie entrainment clet
