\documentclass{article}
\usepackage[dvips]{epsfig,graphics,color,graphicx,lscape}
\usepackage{amsfonts}
\usepackage{amssymb,amsmath}
\usepackage{graphicx}


 

\setlength{\textwidth}{6.5in}
\setlength{\textheight}{9.0in}

%\setlength{\oddsidemargin}{-0.25in}
\setlength{\oddsidemargin}{0in}
%\setlength{\evensidemargin}{0.25in}
\setlength{\topmargin}{-.50in} 


% Murphy's short cut commands
\newcommand{\Real}{\mbox{Re}\,}
\newcommand{\Imag}{\mbox{Im}\,}

\newcommand{\e}{\mathrm{e}}
\newcommand{\Hb}{\mathbf{H}}
\newcommand\bnabla{\mbox{\boldmath${\nabla}$}}
\newcommand{\vecxi}{\mbox{\boldmath${\xi}$}}
\newcommand\Kbc{\mbox{\boldmath${\mathcal{K}}$}}
\newcommand{\Hc}{\mathcal{H}}
\newcommand{\Ac}{\mathcal{A}}
\newcommand{\Tc}{\mathcal{T}}
\newcommand{\Vc}{\mathcal{V}}
\newcommand{\Fc}{\mathcal{F}}
\newcommand{\Ec}{\mathcal{E}}
\newcommand{\Sm}{\mathsf{S}}
\newcommand{\Am}{\mathsf{A}}
\newcommand{\Dm}{\mathsf{D}}
\newcommand{\vecx}{\mathbf{x}}
\newcommand{\vecu}{\mathbf{u}}
\newcommand{\veca}{\mathbf{a}}


\begin{document}

\title{Effective diffusivity: a dynamical systems approach}
\author{N. B. Murphy and J. Xin}
\date{}
\maketitle
\begin{abstract}
A dynamical systems approach to the effective parameter problem for
advection-diffusion is presented.

\end{abstract}
\bigskip
 
\section{The cell problem as a dynamical system }
%
Consider the cell problem associated with the advection-diffusion
equation
%
\begin{align}\label{eq:Periodic_Cell_Prob}
  (\partial_t+\vecu\cdot\bnabla-\varepsilon\Delta)\chi_j(t,\vecx)=u_j(t,\vecx)
\end{align}
%
with velocity field
%
\begin{align}\label{eq:velocity_field_delta}
\vecu(t,\vecx)%=(\cos y + \cos t \, \sin y, \cos x + \cos t \sin x)
       =(\cos y,\cos x)+\theta\cos t \,(\sin y,\sin x).       
\end{align}
%
The components $\Dm^*_{jk}$, $j,k=1,\ldots,d$, of the effective diffusivity
tensor $\Dm^*$ are given 
by         
%
\begin{align}\label{eq:Djk}
  \Dm^*_{jk}=\varepsilon\delta_{jk}+\langle u_j,\chi_k\rangle_2,
\end{align}
%
where $\langle f,h\rangle=\langle f\; \overline{h}\rangle$ denotes the $L^2(\Tc\times\Vc)$
inner-product over the period cell $\Tc\times\Vc$, $\langle\cdot\rangle$ denotes space-time
average over $\Tc\times\Vc$, and $\overline{h}$ denotes complex conjugation
of the function $h$. We stress that $u_j$ and $\chi_k$ are
\emph{real-valued}. Inserting the formula for $u_j$
in~\eqref{eq:Periodic_Cell_Prob} into equation~\eqref{eq:Djk} yields 
%
\begin{align}\label{eq:Eff_Diffusivity_Sobolev}
  \Sm^*_{jk}=\varepsilon(\delta_{jk}+\langle\chi_j,\chi_k\rangle_1),
  \qquad
  \Am^*_{jk}=\langle A\chi_j,\chi_k\rangle_1,
  \quad
  A=(-\Delta)^{-1}(\partial_t+\vecu\cdot\bnabla).
\end{align}
%
Here $\Sm^*_{jk}$ and $\Am^*_{jk}$ are the components of the symmetric
$\Sm^*$ and antisymmetric $\Am^*$ parts of $\Dm^*$ and
$\langle f,h\rangle_1=\langle\bnabla f\cdot\bnabla h\rangle$.



Since $\chi(t,\cdot)\in\Hc^1_{\Vc}$ for each $t\in\Tc$ and the orthogonal set
$\{\e^{\imath (mx+ny)}\}_{m,n\in\mathbb{Z}}$, is complete in $L^2(\Vc)\supset\Hc^1_{\Vc}$, we can
represent $\chi_j(t,\vecx)$ by  
%
\begin{align}\label{eq:x_Expansion}
  \chi_j(t,\vecx)=\sum_{m,n}a^{\,j}_{m,n}(t)\,\e^{\imath (mx+ny)},
  \quad
  a^{\,j}_{m,n}(t)=\langle\chi_j(t,\vecx),\e^{\imath (mx+ny)}\rangle_{\Vc},
  \quad
  a^{\,j}_{m,n}(0)=a^{\,j}_{m,n}(2\pi),
\end{align}
%
where $\langle\cdot\rangle_{\Vc}$ denotes spatial averaging over the spatial period $\Vc$.
Inserting the formula for $\chi_j$ in~\eqref{eq:x_Expansion} into
equation~\eqref{eq:Periodic_Cell_Prob} yields  
%
\begin{align}\label{eq:Cell_Problem_Fourier}
  \sum_{m,n}\e^{\imath (mx+ny)}[\partial_t+\imath mu_1+\imath nu_2+\varepsilon(m^2+n^2)]a^{\,j}_{m,n}=u_j,
\end{align}
%
where we have written $\vecu=(u_1,u_2)$. Writing
$\cos{x}=(\e^{\imath x}+\e^{-\imath x})/2$ and $\sin{x}=(\e^{\imath x}-\e^{-\imath
  x})/(2\imath)$, for example, equation \eqref{eq:Cell_Problem_Fourier} can
be written, for $j=1$, as 
%
\begin{align}\label{eq:Cell_Problem_Fourier_exp}
  \sum_{m,n}\e^{\imath (mx+ny)}[
    \partial_ta^{\,1}_{m,n}
    +\frac{\imath}{2}(&m(a^{\,1}_{m,n-1}+a^{\,1}_{m,n+1})+n(a^{\,1}_{m-1,n}+a^{\,1}_{m+1,n}))
    \\
    +\frac{\theta\cos{t}}{2}(&m(a^{\,1}_{m,n-1}-a^{\,1}_{m,n+1})+n(a^{\,1}_{m-1,n}-a^{\,1}_{m+1,n}))
    +\varepsilon(m^2+n^2)a^{\,1}_{m,n}
    \notag\\
    -\frac{1}{2}&\delta_{0,m}(\delta_{1,n}+\delta_{-1,n})
          -\frac{\theta\cos{t}}{2\imath}\delta_{0,m}(\delta_{1,n}-\delta_{-1,n})]=0,
    \notag
\end{align}
%
where $\delta_{l,m}$ is the Kronecker delta.
Since $u_2(t,x,y)=u_1(t,y,x)$, the formula for
$a^{\,2}_{m,n}$ follows from interchanging $\delta_{l,m}$ with $\delta_{l,n}$ in
\eqref{eq:Cell_Problem_Fourier_exp}, $l=-1,0,1$. By the completeness of the orthogonal set
$\{\e^{\imath(mx+ny)}\}$, equation \eqref{eq:Cell_Problem_Fourier_exp} and
its analogue for $a^{\,2}_{m,n}$ can be written as the following two
(infinite) linear, coupled dynamical systems 
%
\begin{align}\label{eq:Cell_Problem_ODE}
  \partial_ta^{\,1}_{m,n}
    &+\frac{\imath}{2}(m(a^{\,1}_{m,n-1}+a^{\,1}_{m,n+1})+n(a^{\,1}_{m-1,n}+a^{\,1}_{m+1,n}))
    \\
    &+\frac{\theta\cos{t}}{2}(m(a^{\,1}_{m,n-1}-a^{\,1}_{m,n+1})+n(a^{\,1}_{m-1,n}-a^{\,1}_{m+1,n}))
    +\varepsilon(m^2+n^2)a^{\,1}_{m,n}, 
    \notag\\
    &=\frac{1}{2}\delta_{0,m}(\delta_{1,n}+\delta_{-1,n})
          +\frac{\theta\cos{t}}{2\imath}\delta_{0,m}(\delta_{1,n}-\delta_{-1,n}),
    \notag\\
 \partial_ta^{\,2}_{m,n}
    &+\frac{\imath}{2}(m(a^{\,2}_{m,n-1}+a^{\,2}_{m,n+1})+n(a^{\,2}_{m-1,n}+a^{\,2}_{m+1,n}))
    \notag\\
    &+\frac{\theta\cos{t}}{2}(m(a^{\,2}_{m,n-1}-a^{\,2}_{m,n+1})+n(a^{\,2}_{m-1,n}-a^{\,2}_{m+1,n}))
    +\varepsilon(m^2+n^2)a^{\,2}_{m,n}
    \notag\\
    &=\frac{1}{2}\delta_{0,n}(\delta_{1,m}+\delta_{-1,m})
          +\frac{\theta\cos{t}}{2\imath}\delta_{0,n}(\delta_{1,m}-\delta_{-1,m}),
    \notag          
\end{align}
%
with  boundary condition $a^{\,j}_{m,n}(0)=a^{\,j}_{m,n}(2\pi)$, $j=1,2$,
$m,n\in\mathbb{Z}$. When $\theta=0$, the velocity field in
\eqref{eq:velocity_field_delta} is time-independent. In this case the
function $\chi_j$, hence the Fourier coefficients $a^{\,j}_{m,n}$ are
also time-independent, and equation \eqref{eq:Cell_Problem_ODE}
reduces to the following algebraic system of equations
%
\begin{align}\label{eq:Cell_Problem_LinAlg}
    \frac{\imath}{2}(m(a^{\,1}_{m,n-1}+a^{\,1}_{m,n+1})+n(a^{\,1}_{m-1,n}+a^{\,1}_{m+1,n}))      
    +\varepsilon(m^2+n^2)a^{\,1}_{m,n}
    &=\frac{1}{2}\delta_{0,m}(\delta_{1,n}+\delta_{-1,n})          
    \\
    \frac{\imath}{2}(m(a^{\,2}_{m,n-1}+a^{\,2}_{m,n+1})+n(a^{\,2}_{m-1,n}+a^{\,2}_{m+1,n}))    
    +\varepsilon(m^2+n^2)a^{\,2}_{m,n}    
    &=\frac{1}{2}\delta_{0,n}(\delta_{1,m}+\delta_{-1,m}).       
    \notag          
\end{align}
%





By restricting the indices, $-M\leq \ell,m,n\leq M$, and imposing the boundary
conditions
%
\begin{align}
  a^{\,l}_{m,n}=0 \quad \text{ if } \ \ \max(|m|,|n|)>M,
\end{align}
%
the infinite systems of equations in~\eqref{eq:Cell_Problem_ODE}
and~\eqref{eq:Cell_Problem_LinAlg} become finite sets of
equations. The bijective mapping
% 
\begin{align}\label{eq:Bijections} 
  \Theta_s(m,n)&=(M+m+1)+(M+n)(2M+1), 
\end{align}
%
maps the finite sets of equations to matrix equations
%
\begin{align}
  \label{eq:Dynamic_system}
  &\partial_t\veca^{\,j}(t)=(A+\theta\cos{t}\,B+\varepsilon C)\veca^{\,j}(t)
             +\vecxi_1^{\,j}+\theta\cos{t}\,\vecxi_2^{\,j},
   \quad
   \veca^{\,j}(0)=\veca^{\,j}(2\pi)
   \\
   \label{eq:Algebraic_system}
  &(A+\varepsilon C)\veca^{\,j}=\vecxi_1^{\,j}
\end{align}
%
Equation~\eqref{eq:Dynamic_system} is a linear, inhomogeneous
dynamical system of equations, while
equation~\eqref{eq:Algebraic_system} is a linear system of algebraic
equations. In equation~\eqref{eq:Algebraic_system}, the rows and
columns of the matrices $A$ and $C$ corresponding to the
$a^{\,j}_{0,0}$ component of the unknown vector $\veca^{\,j}$,
consist entirely of zero elements, and can be removed without loss of 
generality. Consequently, for each $0<\varepsilon<\infty$, this algebraic system
in~\eqref{eq:Algebraic_system} can be directly solved using techniques
of linear algebra, to determine the Fourier coefficients of $\chi_j$ in
$\veca^{\,j}$. Moreover, the matrix $(A+\varepsilon C)$ is \emph{sparse}, so
that iterative methods are applicable.   


 

We now discuss how the symmetric $\Sm^*$ and antisymmetric $\Am^*$ parts
of the effective diffusivity tensor $\Dm^*$ are determined from the
Fourier coefficients $a^{\,j}_{m,n}$ of $\chi_j$. Inserting the formula
for $\chi_j$ in~\eqref{eq:x_Expansion} into the formula for $\Sm^*_{jk}$ in
equation~\eqref{eq:Eff_Diffusivity_Sobolev} and using the
orthogonality of  the set $\{\e^{\imath(mx+ny)}\}$ yields 
%
\begin{align}
 \Sm^*_{jk}/\varepsilon-\delta_{jk}= \langle\chi_j,\chi_k\rangle_1=\Real\sum_{m,n}(m^2+n^2)\,\langle a^{\,j}_{m,n}\,\overline{a^{\,k}_{m,n}}\rangle_{\Tc}.
  %\langle\chi_j,\chi_k\rangle_1=\sum_{m,n}(m^2+n^2)\,\text{Re}\langle\overline{a^{\,j}_{m,n}}\,a^{\,k}_{m,n}\rangle_t,
\end{align} 
%
Here, $\langle\cdot\rangle_{\Tc}$ denotes time averaging over the temporal period
$\Tc$ and, since $\chi_j$ 
is real-valued, we have $\langle\chi_j,\chi_k\rangle_1=\Real\langle\chi_j,\chi_k\rangle_1$.
As discussed above, when $\theta=0$, the Fourier coefficients
$a^{\,j}_{m,n}$ are time-independent, so that
$\langle a^{\,j}_{m,n}\,\overline{a^{\,k}_{m,n}}\rangle_{\Tc}=a^{\,j}_{m,n}\,\overline{a^{\,k}_{m,n}}$. Note
that
$\langle
A\chi_j,\chi_k\rangle_1=\langle\nabla(-\Delta)^{-1}(\partial_t+\bnabla\cdot\vecu)\chi_j\cdot\nabla\chi_k\rangle=\langle(\partial_t+\bnabla\cdot\vecu)\chi_j,\chi_k\rangle_2$. Consequently,
inserting the formula
for $\chi_j$ in~\eqref{eq:x_Expansion} into the formula for $\Am^*_{jk}$ in
equation~\eqref{eq:Eff_Diffusivity_Sobolev} and using the
orthogonality of  the set $\{\e^{\imath(mx+ny)}\}$ yields 
%
\begin{align}
  \langle A\chi_j,\chi_k\rangle_1=\sum_{m,n}\,\langle\overline{\gamma_{m,n}^{\theta,j}}\,a^{\,k}_{m,n}\rangle_{\Tc},\quad
  %\langle A\chi_j,\chi_k\rangle_1=\sum_{m,n}\,\text{Re}\langle\overline{\gamma_{m,n}^{\theta,j}}\,a^{\,k}_{m,n}\rangle_t,\quad
  \gamma_{m,n}^{\theta,j}=\partial_ta^{\,j}_{m,n}
    &+\frac{\imath}{2}(m(a^{\,j}_{m,n-1}+a^{\,j}_{m,n+1})+n(a^{\,j}_{m-1,n}+a^{\,j}_{m+1,n}))
    \\
    &+\frac{\theta\cos{t}}{2}(m(a^{\,j}_{m,n-1}-a^{\,j}_{m,n+1})+n(a^{\,j}_{m-1,n}-a^{\,j}_{m+1,n})).
  \notag
\end{align}
%
When $\theta=0$, the Fourier coefficients $a^{\,j}_{m,n}$ are 
time-independent, so that
$\langle\overline{\gamma^{\theta,j}_{m,n}}\,a^{\,k}_{m,n}\rangle_t=\overline{\gamma^{\theta,j}_{m,n}}\,a^{\,k}_{m,n}$ and
$\gamma_{m,n}^{\theta,j}=(m(a^{\,j}_{m,n-1}+a^{\,j}_{m,n+1})+n(a^{\,j}_{m-1,n}+a^{\,j}_{m+1,n}))/(-2\imath)$.












\end{document}
