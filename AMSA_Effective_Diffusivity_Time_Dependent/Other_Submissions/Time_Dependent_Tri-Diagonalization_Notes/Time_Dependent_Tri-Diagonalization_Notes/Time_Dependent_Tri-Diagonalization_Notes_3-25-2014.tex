\documentclass{article}
\usepackage[dvips]{epsfig,graphics,color,graphicx}
\usepackage{amsfonts}
\usepackage{amssymb, amsmath}
\usepackage{graphicx}
\usepackage[dvips]{epsfig,graphics,color}
\newcommand{\no}{\nonumber}
\newcommand{\ra}{\to}
\newcommand{\eps}{\mbox{$\epsilon$}}
\newcommand{\be}{\begin{equation}}
\newcommand{\ee}{\end{equation}}
\newcommand{\ba}{\begin{eqnarray}}
\newcommand{\ea}{\end{eqnarray}}
\newcommand{\lam}{\mbox{$\lambda$}}
\newtheorem{theorem}{Theorem}[section]
 %\newcommand{\bth}{\begin{theorem}}
%\newcommand{\eth}{\end{theorem}}
\newtheorem{lemma}{Lemma}[section]
   \newcommand{\blem}{\begin{lemma}}
   \newcommand{\elem}{\end{lemma}}
\newtheorem{proposition}{Proposition}[section]
   \newcommand{\bprop}{\begin{proposition}}
   \newcommand{\eprop}{\end{proposition}}
\newtheorem{definition}{Definition}[section]
\newtheorem{corollary}{Corollary}[section]
\newtheorem{algorithm}{Algorithm}[section]
\newtheorem{example}{Example}[section]
\newtheorem{assumption}{Assumption}[section]
\newtheorem{remark}{Remark}[section]
\renewcommand{\theequation}{\arabic{section}.\arabic{equation}}
\renewcommand{\thetheorem}{\arabic{section}.\arabic{theorem}}
\renewcommand{\thelemma}{\arabic{section}.\arabic{lemma}}
\renewcommand{\theproposition}{\arabic{section}.\arabic{proposition}}
\renewcommand{\thedefinition}{\arabic{section}.\arabic{definition}}
\renewcommand{\thecorollary}{\arabic{section}.\arabic{corollary}}
\renewcommand{\thealgorithm}{\arabic{section}.\arabic{algorithm}}
\renewcommand{\thefigure}{\arabic{section}.\arabic{figure}}
\newcommand{\proof}{\vspace{1ex}\noindent{\em Proof}: \ }

% Murphy's short cut commands
\newcommand{\e}{\mathrm{e}}
\newcommand{\I}{\mathrm{i}}
\newcommand{\Hb}{\mathbf{H}}
\newcommand\bkappa{\mbox{\boldmath${\kappa}$}}
\newcommand\balpha{\mbox{\boldmath${\alpha}$}}
\newcommand\Kbc{\mbox{\boldmath${\mathcal{K}}$}}


\begin{document}

\title{Decomposing parabolic eigenvalue problem}
\author{J. Xin and N. B. Murphy }
\date{}
\maketitle
\begin{abstract}
A decomposition of a time periodic parabolic eigenvalue problem and an
application to our effective diffusivity problem.

\end{abstract}
\bigskip

\section{An Example}
Consider the parabolic eigenvalue problem:
\be
\psi_t -\Delta_x \, \psi + \cos (t) \; b(x) \, \psi = \lam \, \psi, \label{eig1}
\ee
subject to $2\pi$ periodic boundary conditions in $x$ and $t$. 
\medskip

Substituting $\cos t = (e^{it} + e^{-it})/2$, and $\psi = \sum_{n} \psi_n (x) e^{i n t}$ in (\ref{eig1}), 
we have:
\be
\sum_n \,e^{int}\, (i n \, \psi_n\, - \Delta_x \psi_n) + {1\over 2} b(x)\, \sum_{n} (e^{i(n+1)t} + e^{i(n-1) t}) \psi_n 
= \lam \sum_n \psi_n e^{int}, \no
\ee
or:
\be
 \sum_n \,e^{int}\, (i n \, \psi_n\, - \Delta_x \psi_n)  + {1\over 2} b(x)\, \sum_{n} (\psi_{n-1} + \psi_{n+1}) \, e^{int} 
= \lam \sum_n \psi_n e^{int}. \no
\ee
Extracting the $n$-th mode on both sides gives:
\be
(-\Delta_x + i\, n)\, \psi_n +{1\over 2} \, b(x)\, (\psi_{n-1} + \psi_{n+1}) = \lam \, \psi_n, \;\; n \in Z, \label{eig2}
\ee
which can be put in a tri-diagonal matrix acting on $[\cdots, \psi_{n-1}, \psi_n, \psi_{n+1}, \cdots ]'$,
  with $b(x)/2$ on the off-diagonals, 
$-\Delta_x + i\, n $ on the diagonals.

Let us find a similar derivation for the eigenvalue problem of the 
advection-diffusion operator, with advection field being the time periodic cell flow:
\be \label{eig3}
\vec{u}%=(\cos y + \cos t \, \sin y, \cos x + \cos t \sin x)
       =(\cos y,\cos x)+\delta\cos t \,(\sin y,\sin x)
       :=\vec{u}_1(\vec{x})+\delta\cos{t}\,\vec{u}_2(\vec{x}). 
\ee

Question: Is it possible to carry out a similar decomposition in $x$ and $y$ and fully reduce the 
differential system like (\ref{eig2}) into an algebraic system ?

%
\subsection{An Application to our effective diffusivity problem}
%
Consider the eigenvalue problem $A\psi=\lambda\psi$ involving the
integro-differential operator $A=-\Delta^{-1}(\partial_t+\vec{u}\cdot\vec{\nabla})$, 
introduced in equation (45) of our (attached) effective-diffusivity
paper, with $\vec{u}\mapsto-\vec{u}$. Here $A$ is an anti-symmetric (normal)
operator and the incompressible velocity field
$\vec{u}(t,\vec{x})=\vec{u}_1(\vec{x})+\delta\cos{t}\,\vec{u}_2(\vec{x})$
is given in equation \eqref{eig3} above. The equation $A\psi=\lambda\psi$ may be
rewritten as     
%
\begin{align}\label{eq:Eig_prob}
  (\partial_t+\vec{u}\cdot\vec{\nabla})\psi=-\lambda\Delta\psi.
\end{align}
%
Now consider the separation of variables ansatz
$\psi(t,\vec{x})=\sum_n\psi_n(\vec{x})\,\e^{\I nt}$ and write 
$\cos{t}=(\e^{\I t}+\e^{-\I t})/2$. Plugging this into equation
\eqref{eq:Eig_prob} 
yields 
%
\begin{align}
  \sum_n(\I n + \vec{u}_1\cdot\vec{\nabla}+\lambda\Delta)\psi_n(\vec{x})\e^{\I nt}
 +\sum_n\frac{\delta}{2}(\e^{\I(n+1)t}+\e^{\I(n-1)t})\vec{u}_2\cdot\vec{\nabla}\psi_n(\vec{x})=0,
\end{align}
%
or:
%
\begin{align}
  \sum_n\left[(\I n + \vec{u}_1\cdot\vec{\nabla}+\lambda\Delta)\psi_n(\vec{x})
 +\frac{\delta}{2}\vec{u}_2\cdot\vec{\nabla}(\psi_{n-1}(\vec{x})+\psi_{n+1}(\vec{x}))\right]\e^{\I nt}=0.
\end{align}
%
By the completeness of the orthogonal set $\{\e^{\I nt}\}_n$ we
have for all $n\in\mathbb{Z}$ that
%
\begin{align}\label{eq:Eig_prob_shift}
  (\I n + \vec{u}_1\cdot\vec{\nabla})\psi_n(\vec{x})
 +\frac{\delta}{2}\vec{u}_2\cdot\vec{\nabla}(\psi_{n-1}(\vec{x})+\psi_{n+1}(\vec{x}))=-\lambda\Delta\psi_n(\vec{x}),
\end{align}
%
which is subject to the condition $\langle\vec{\nabla}\psi_n\cdot\vec{\nabla}\psi_m\rangle=\delta_{nm}$, where
$\delta_{nm}$ is the Kronecker delta and $\langle\cdot\rangle$ denotes space averaging
over a period cell. Since $\vec{u}_i$ is incompressible, there exists
an anti-symmetric matrix $\Hb_i$ such that
$\vec{u}_i=\vec{\nabla}\cdot\Hb_i$. This allows us to write
$\vec{u}_i\cdot\vec{\nabla}=\vec{\nabla}\cdot\Hb_i\vec{\nabla}$, which is an anti-symmetric
operator.    



When $\delta=0$, the velocity field $\vec{u}$ is time-independent and the
operator $A$, which arises from the cell problem, becomes
$A=\Delta^{-1}(\vec{u}_1\cdot\vec{\nabla})$. In this case, the eigenvalue problem in
\eqref{eq:Eig_prob} becomes 
%
\begin{align}\label{eq:Eig_prob_steady}
  \vec{\nabla}\cdot\Hb_1\vec{\nabla}\psi=\lambda\Delta\psi.
\end{align}
%
Discretizing this equation leads to a generalized eigenvalue
problem involving \emph{sparse} matrices. This matrix formulation has
all the desired properties of the associated abstract Hilbert space
formulation. (I will be adding the details of this to our paper soon.)
From this matrix problem, we obtain a discrete approximation of the
Stieltjes--Radon integral representation for the symmetric $\bkappa^*$
and anti-symmetric $\balpha^*$ parts of the effective diffusivity
tensor $\Kbc^*$, displayed in equation (35) of our (attached) paper.  


\vspace{1em}
\noindent\textbf{Question:} Can this approach be extended to cast
equation \eqref{eq:Eig_prob_shift} as a sequence of generalized
eigenvalue problems, which can be numerically solved to obtain the
eigenvalues and eigenvectors associated with this spatial operator? If
so, this would allow us to solve a sequence of smaller eigenvalue
problems of size $N^2$ instead of solving an $N^3$ problem (two
spatial dimensions and a time dimension). 






\section{Three-Dim Steady Cellular Flows}
The 3 dimensional (3D) steady cellular flows are:
\be
B =
(\Phi_x(x,y)W'(z),\Phi_y(x,y)W'(z),k\Phi(x,y)W(z)), \label{3dcell}
\ee
with $-\Delta \Phi = k \Phi.$   
\medskip

A special case is $k=2$, then
\be
B(x,y,z) =
(-\sin{x}\cos{y}\cos{z},-\cos{x}\sin{y}\cos{z},2\cos{x}\cos{y}\sin{z}).\label{3dcell-1}
\ee
\medskip

Question: Is the effective diffusivity problem easier for (\ref{3dcell-1}) than (\ref{eig3})? 
\medskip

Effective diffusivity in (\ref{3dcell}) is unknown, see \cite{SXZ_13} for 
related KPP problem, and \cite{RZ_07} for an upper bound. Extrapolating the KPP scaling in \cite{SXZ_13} on 
the flow (\ref{3dcell-1}), effective diffusivity at small molecular diffusivity $\epsilon$ scales like $O(\epsilon^{p})$, $p \approx 0.26$. 
In 2D, $p=1/2$, see \cite{fannjiang}, also Fig.3 in \cite{Biferale_95}.    


\begin{thebibliography}{99}

\bibitem{Biferale_95}L. Biferale, A. Crisanti, M. Vergassola, A. Vulpiani,
{\em Eddy diffusivities in scalar transport}, Phys. Fluids 7(11), pp. 2725 --2734, 1995.

\bibitem{fannjiang} A. Fannjiang and G. Papanicolaou,
{\em Convection enhanced diffusion for periodic flows}, SIAM J.
Appl. Math., {\bf 54} (1992), pp. 333-408.

\bibitem{RZ_07}L. Ryzhik and A. Zlatos,
{\em KPP pulsating front speed-up by flows}, Comm. Math. Sci., {\bf
5} (2007), pp. 575-593.

\bibitem{SXZ_13}L. Shen, J. Xin and A. Zhou, {\em Finite Element Computation of KPP
Front Speeds in 3D Cellular and ABC Flows}, Math Model. Natural Phenom., 8(3), 2013, pp. 182-197. 




\end{thebibliography}

\end{document}
