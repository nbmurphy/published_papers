\documentclass{article}
\usepackage[dvips]{epsfig,graphics,color,graphicx}
\usepackage{amsfonts}
\usepackage{amssymb, amsmath}
\usepackage{graphicx}
\usepackage[dvips]{epsfig,graphics,color}
\newcommand{\no}{\nonumber}
\newcommand{\ra}{\to}
\newcommand{\eps}{\mbox{$\epsilon$}}
\newcommand{\be}{\begin{equation}}
\newcommand{\ee}{\end{equation}}
\newcommand{\ba}{\begin{eqnarray}}
\newcommand{\ea}{\end{eqnarray}}
\newcommand{\lam}{\mbox{$\lambda$}}
\newtheorem{theorem}{Theorem}[section]
 %\newcommand{\bth}{\begin{theorem}}
%\newcommand{\eth}{\end{theorem}}
\newtheorem{lemma}{Lemma}[section]
   \newcommand{\blem}{\begin{lemma}}
   \newcommand{\elem}{\end{lemma}}
\newtheorem{proposition}{Proposition}[section]
   \newcommand{\bprop}{\begin{proposition}}
   \newcommand{\eprop}{\end{proposition}}
\newtheorem{definition}{Definition}[section]
\newtheorem{corollary}{Corollary}[section]
\newtheorem{algorithm}{Algorithm}[section]
\newtheorem{example}{Example}[section]
\newtheorem{assumption}{Assumption}[section]
\newtheorem{remark}{Remark}[section]
\renewcommand{\theequation}{\arabic{section}.\arabic{equation}}
\renewcommand{\thetheorem}{\arabic{section}.\arabic{theorem}}
\renewcommand{\thelemma}{\arabic{section}.\arabic{lemma}}
\renewcommand{\theproposition}{\arabic{section}.\arabic{proposition}}
\renewcommand{\thedefinition}{\arabic{section}.\arabic{definition}}
\renewcommand{\thecorollary}{\arabic{section}.\arabic{corollary}}
\renewcommand{\thealgorithm}{\arabic{section}.\arabic{algorithm}}
\renewcommand{\thefigure}{\arabic{section}.\arabic{figure}}
\newcommand{\proof}{\vspace{1ex}\noindent{\em Proof}: \ }

% Murphy's short cut commands
\newcommand{\e}{\mathrm{e}}
\newcommand{\I}{\mathrm{i}}
\newcommand{\Hb}{\mathbf{H}}
\newcommand\bkappa{\mbox{\boldmath${\kappa}$}}
\newcommand\balpha{\mbox{\boldmath${\alpha}$}}
\newcommand\Kbc{\mbox{\boldmath${\mathcal{K}}$}}
\newcommand{\Hc}{\mathcal{H}}
\newcommand{\Ac}{\mathcal{A}}
\newcommand{\Tc}{\mathcal{T}}
\newcommand{\Vc}{\mathcal{V}}
\newcommand{\Fc}{\mathcal{F}}



\begin{document}

\title{Decomposing parabolic eigenvalue problem}
\author{J. Xin and N. B. Murphy }
\date{}
\maketitle
\begin{abstract}
A decomposition of a time periodic parabolic eigenvalue problem and an
application to our effective diffusivity problem.

\end{abstract}
\bigskip
 
\section{An Example}
Consider the parabolic eigenvalue problem:
\be
\psi_t -\Delta_x \, \psi + \cos (t) \; b(x) \, \psi = \lam \, \psi, \label{eig1}
\ee
subject to $2\pi$ periodic boundary conditions in $x$ and $t$. 
\medskip

Substituting $\cos t = (e^{it} + e^{-it})/2$, and $\psi = \sum_{\ell} \psi_\ell (x) e^{i \ell t}$ in (\ref{eig1}), 
we have:
\be
\sum_\ell \,e^{i\ell t}\, (i \ell \, \psi_\ell\, - \Delta_x \psi_\ell) + {1\over 2} b(x)\, \sum_{\ell} (e^{i(\ell+1)t} + e^{i(\ell-1) t}) \psi_\ell 
= \lam \sum_\ell \psi_\ell e^{i\ell t}, \no
\ee
or:
\be
 \sum_\ell \,e^{i\ell t}\, (i \ell \, \psi_\ell\, - \Delta_x \psi_\ell)  + {1\over 2} b(x)\, \sum_{\ell} (\psi_{\ell-1} + \psi_{\ell+1}) \, e^{i\ell t} 
= \lam \sum_\ell \psi_\ell e^{i\ell t}. \no
\ee
Extracting the $\ell$-th mode on both sides gives:
\be
(-\Delta_x + i\, \ell)\, \psi_\ell +{1\over 2} \, b(x)\, (\psi_{\ell-1} + \psi_{\ell+1}) = \lam \, \psi_\ell, \;\; \ell \in Z, \label{eig2}
\ee
which can be put in a tri-diagonal matrix acting on $[\cdots, \psi_{\ell-1}, \psi_\ell, \psi_{\ell+1}, \cdots ]'$,
  with $b(x)/2$ on the off-diagonals, 
$-\Delta_x + i\, \ell $ on the diagonals.

Let us find a similar derivation for the eigenvalue problem of the 
advection-diffusion operator, with advection field being the time periodic cell flow:
\be \label{eig3}
\vec{u}(t,\vec{x})%=(\cos y + \cos t \, \sin y, \cos x + \cos t \sin x)
       =(\cos y,\cos x)+\delta\cos t \,(\sin y,\sin x)
       :=\vec{u}_1(\vec{x})+\delta\cos{t}\,\vec{u}_2(\vec{x}). 
\ee

Question: Is it possible to carry out a similar decomposition in $x$ and $y$ and fully reduce the 
differential system like (\ref{eig2}) into an algebraic system ?

%
\subsection{An Application to our effective diffusivity problem}
%
Consider the eigenvalue problem $A\psi_j=\lambda_j\psi_j$, $j=1,2,3,\ldots$, involving
the integro-differential operator $A=-\Delta^{-1}(\partial_t+\vec{u}\cdot\vec{\nabla})$, 
introduced in equation (45) of our (attached) effective-diffusivity
paper, with $\vec{u}\mapsto-\vec{u}$. Here $A$ is an anti-symmetric (normal)
operator and the incompressible velocity field
$\vec{u}(t,\vec{x})$
%$=\vec{u}_1(\vec{x})+\delta\cos{t}\,\vec{u}_2(\vec{x})$
is given in equation \eqref{eig3} above. The equation $A\psi_j=\lambda_j\psi_j$ may be
rewritten as     
%
\begin{align}\label{eq:Eig_prob}
  (\partial_t+\vec{u}\cdot\vec{\nabla})\psi_j=-\lambda_j\Delta\psi_j.
\end{align}
%
The eigenfunctions $\psi_j$ satisfy the following orthogonality condition
%$\langle\vec{\nabla}\psi_\ell^{\,j}\cdot\vec{\nabla}\psi_m^k\rangle=\delta_{jk}$,
%
\begin{align}\label{eq:Orthogonal}
  \langle\vec{\nabla}\psi_j\cdot\vec{\nabla}\psi_k\rangle=\delta_{jk},
\end{align}
%
where $\delta_{jk}$ is the Kronecker delta and $\langle\cdot\rangle$ denotes space-time
averaging over the period cell $\Tc\times\Vc$, with $\Tc=[0,2\pi]$ and
$\Vc=[0,2\pi]\times[0,2\pi]$.




The eigenfunction $\psi_j$ satisfies $\psi_j\in\Fc\subset\Ac(\Tc)\otimes\Hc^1(\Vc)$, 
i.e. it is mean-zero, absolutely continuous in time for $t\in\Tc$
and is in the Sobolev space $\Hc^1(\Vc)$ for
$\vec{x}\in\Vc$. Since the orthogonal set
$\{\e^{\I\ell t}\}_{\ell\in\mathbb{Z}}$ is dense in $\Ac(\Tc)$, we may represent
$\psi_j$ by
%
\begin{align}\label{eq:t_Expansion}
  \psi_j(t,\vec{x})=\sum_\ell\psi_\ell^{\,j}(\vec{x})\,\e^{\I \ell t},
\end{align}
%
where $\psi_\ell^{\,j}\in\Hc^1(\Vc)$. Write $\cos{t}=(\e^{\I t}+\e^{-\I t})/2$ and insert
this and \eqref{eq:t_Expansion} into equation \eqref{eq:Eig_prob},
yielding   
%
\begin{align}
  \sum_\ell(\I \ell + \vec{u}_1\cdot\vec{\nabla}+\lambda_j\Delta)\psi_\ell^{\,j}(\vec{x})\e^{\I \ell t}
 +\frac{\delta}{2}\sum_\ell(\e^{\I(\ell+1)t}+\e^{\I(\ell-1)t})\,\vec{u}_2\cdot\vec{\nabla}\psi_\ell^{\,j}(\vec{x})=0,
\end{align}
%
or:
%
\begin{align}
  \sum_\ell\left[(\I \ell + \vec{u}_1\cdot\vec{\nabla}+\lambda_j\Delta)\psi_\ell^{\,j}(\vec{x})
 +\frac{\delta}{2}\vec{u}_2\cdot\vec{\nabla}(\psi^{\,j}_{\ell-1}(\vec{x})+\psi^{\,j}_{\ell+1}(\vec{x}))\right]\e^{\I \ell t}=0.
\end{align}
%
By the completeness of the orthogonal set $\{\e^{\I \ell t}\}$ we
have, for all $\ell\in\mathbb{Z}$, that
%
\begin{align}\label{eq:Eig_prob_shift}
  (\I \ell + \vec{u}_1\cdot\vec{\nabla})\psi_\ell^{\,j}(\vec{x})
 +\frac{\delta}{2}\vec{u}_2\cdot\vec{\nabla}(\psi_{\ell-1}^{\,j}(\vec{x})+\psi_{\ell+1}^{\,j}(\vec{x}))
 =-\lambda_j\Delta\psi_\ell^{\,j}(\vec{x}).
\end{align}
%




The system of partial differential equations in
\eqref{eq:Eig_prob_shift} can be reduced to a system of algebraic
equations as follows. Recall that
$\vec{u}_1(\vec{x})=(\cos{y},\cos{x})$ and
$\vec{u}_2(\vec{x})=(\sin{y},\sin{x})$, which implies that
%
\begin{align}\label{eq:material_Derivative_psi}
  (\vec{u}_1\cdot\vec{\nabla})\psi_\ell^{\,j}(\vec{x})
          &=\cos{y}\,\partial_x\psi_\ell^{\,j}(\vec{x})+\cos{x}\,\partial_y\psi_\ell^{\,j}(\vec{x})\\
  (\vec{u}_2\cdot\vec{\nabla})\psi_\ell^{\,j}(\vec{x})
          &=\sin{y}\,\partial_x\psi_\ell^{\,j}(\vec{x})+\sin{x}\,\partial_y\psi_\ell^{\,j}(\vec{x})\notag
\end{align}
%
Since $\psi_\ell^{\,j}\in\Hc^1(\Vc)$ and the orthogonal set $\{\e^{\I (mx+ny)}\}$,
$m,n\in\mathbb{Z}$, is dense in this space, we can represent
$\psi_\ell^{\,j}(\vec{x})$ by
%
\begin{align}\label{eq:x_Expansion}
  \psi_\ell^{\,j}(\vec{x})=\sum_{m,n}c^{\,j}_{\ell,m,n}\,\e^{\I (mx+ny)}
\end{align}
%
Write $\cos{x}=(\e^{\I x}+\e^{-\I x})/2$ and
$\sin{x}=(\e^{\I x}-\e^{-\I x})/(2\I)$, for example, and insert this
and \eqref{eq:x_Expansion} into equation
\eqref{eq:material_Derivative_psi}, yielding
%
\begin{align}
  (\vec{u}_1&\cdot\vec{\nabla})\psi_\ell^{\,j}\\
    &=\frac{1}{2}\sum_{m,n}c^{\,j}_{\ell,m,n}\left[
        \I m\,\e^{\I mx}(\e^{\I(n+1)y}+\e^{\I(n-1)y})
        +\I n\,\e^{\I ny}(\e^{\I(m+1)x}+\e^{\I(m-1)x})
                       \right]
    \notag\\
%    
  (\vec{u}_2&\cdot\vec{\nabla})\psi_\ell^{\,j}\notag\\
    &=\frac{1}{2\I}\sum_{m,n}c^{\,j}_{\ell,m,n}\left[
        \I m\,\e^{\I mx}(\e^{\I(n+1)y}-\e^{\I(n-1)y})
        +\I n\,\e^{\I ny}(\e^{\I(m+1)x}-\e^{\I(m-1)x})
                       \right]
                       \notag
\end{align}
%
or:
%
\begin{align}\label{eq:Algebraic_advection}
  (\vec{u}_1\cdot\vec{\nabla})\psi_\ell^{\,j}
    &=\frac{\I}{2}\sum_{m,n}
    [m(c^{\,j}_{\ell,m,n-1}+c^{\,j}_{\ell,m,n+1})+n(c^{\,j}_{\ell,m-1,n}+c^{\,j}_{\ell,m+1,n})]\e^{\I (mx+ny)}
   \notag\\
  (\vec{u}_2\cdot\vec{\nabla})\psi_\ell^{\,j}
    &=\frac{1}{2}\sum_{m,n}
    [m(c^{\,j}_{\ell,m,n-1}-c^{\,j}_{\ell,m,n+1})+n(c^{\,j}_{\ell,m-1,n}-c^{\,j}_{\ell,m+1,n})]\e^{\I (mx+ny)} .
\end{align}
%
We also have 
%
\begin{align}\label{eq:Algebraic_diffusion}
  -\Delta\psi_\ell^{\,j}=\sum_{m,n}c^{\,j}_{\ell,m,n}(m^2+n^2)\e^{\I (mx+ny)}
\end{align}
%
By the completeness of the orthogonal set $\{\e^{\I (mx+ny)}\}$, 
inserting equations \eqref{eq:Algebraic_advection} and
\eqref{eq:Algebraic_diffusion} into equation
\eqref{eq:Eig_prob_shift} yields
%
\begin{align}
\I \ell+\frac{\I}{2}[&m(c^{\,j}_{\ell,m,n-1}+c^{\,j}_{\ell,m,n+1})+n(c^{\,j}_{\ell,m-1,n}+c^{\,j}_{\ell,m+1,n})]
\notag\\
+\frac{\delta}{4}[&m(c^{\,j}_{\ell-1,m,n-1}-c^{\,j}_{\ell-1,m,n+1})+n(c^{\,j}_{\ell-1,m-1,n}-c^{\,j}_{\ell-1,m+1,n}
\notag\\
+&m(c^{\,j}_{\ell+1,m,n-1}-c^{\,j}_{\ell+1,m,n+1})+n(c^{\,j}_{\ell+1,m-1,n}-c^{\,j}_{\ell+1,m+1,n}]
\notag\\
=&\lambda_j(m^2+n^2)c^{\,j}_{\ell,m,n},
\end{align}
%
which is an infinite system of algebraic equations for the unknown
Fourier coefficients $c^{\,j}_{\ell,m,n}$ associated with the
eigenfunctions $\psi^{\,j}(t,\vec{x})$ and eigenvalues $\lambda_j$,
$j\in\mathbb{N}$, $\ell,m,n\in\mathbb{Z}$.




We now discuss how the orthogonality condition
$\langle\vec{\nabla}\psi_j\cdot\vec{\nabla}\psi_k\rangle=\delta_{jk}$ in \eqref{eq:Orthogonal} is
transformed by the Fourier expansion of the
eigenfunctions $\psi^{\,j}(t,\vec{x})$. This expansion implies the
expansion of $\vec{\nabla}\psi^{\,j}(t,\vec{x})$ as follows 
%
\begin{align}
  \psi^{\,j}(t,\vec{x})=\sum_{\ell,m,n}c^{\,j}_{\ell,m,n}\e^{\I (\ell t+mx+ny)}\Rightarrow
  \vec{\nabla}\psi^{\,j}(t,\vec{x})=\sum_{\ell,m,n}c^{\,j}_{\ell,m,n}\,(m,n)\,\e^{\I (\ell t+mx+ny)}.
\end{align}
%
Therefore, by the orthogonality relation 
%
\begin{align}  
      \left\langle
      \e^{\I (\ell t+mx+ny)}\,\e^{\I (\ell^\prime t+m^\prime x+n^\prime y)}
      \right\rangle
      =
      (2\pi)^3\delta_{\ell,\ell^\prime}\delta_{m,m^\prime}\delta_{n,n^\prime}
\end{align}
%
we have that the orthogonality relation in \eqref{eq:Orthogonal} is
transformed to  
%
\begin{align}
  \delta_{jk}=\langle\vec{\nabla}\psi_j\cdot\vec{\nabla}\psi_k\rangle
      = (2\pi)^3\sum_{\ell,m,n}(m^2+n^2)c^{\,j}_{\ell,m,n}c^{\,k}_{\ell,m,n}
\end{align}
%


Since $\vec{u}_i$ is incompressible, there exists
an anti-symmetric matrix $\Hb_i$ such that
$\vec{u}_i=\vec{\nabla}\cdot\Hb_i$. This allows us to write
$\vec{u}_i\cdot\vec{\nabla}=\vec{\nabla}\cdot\Hb_i\vec{\nabla}$, which is an anti-symmetric
operator. When $\delta=0$, the velocity field $\vec{u}$ is time-independent
and the operator $A$, which arises from the cell problem, becomes
$A=\Delta^{-1}(\vec{u}_1\cdot\vec{\nabla})$. In this case, the eigenvalue problem in
\eqref{eq:Eig_prob} becomes 
%
\begin{align}\label{eq:Eig_prob_steady}
  \vec{\nabla}\cdot\Hb_1\vec{\nabla}\psi=\lambda\Delta\psi.
\end{align}
%
Discretizing this equation leads to a generalized eigenvalue
problem involving \emph{sparse} matrices. This matrix formulation has
all the desired properties of the associated abstract Hilbert space
formulation. (I will be adding the details of this to our paper soon.)
From this matrix problem, we obtain a discrete approximation of the
Stieltjes--Radon integral representation for the symmetric $\bkappa^*$
and anti-symmetric $\balpha^*$ parts of the effective diffusivity
tensor $\Kbc^*$, displayed in equation (35) of our (attached) paper.  









\section{Three-Dim Steady Cellular Flows}
The 3 dimensional (3D) steady cellular flows are:
\be
B =
(\Phi_x(x,y)W'(z),\Phi_y(x,y)W'(z),k\Phi(x,y)W(z)), \label{3dcell}
\ee
with $-\Delta \Phi = k \Phi.$   
\medskip

A special case is $k=2$, then
\be
B(x,y,z) =
(-\sin{x}\cos{y}\cos{z},-\cos{x}\sin{y}\cos{z},2\cos{x}\cos{y}\sin{z}).\label{3dcell-1}
\ee
\medskip

Question: Is the effective diffusivity problem easier for (\ref{3dcell-1}) than (\ref{eig3})? 
\medskip

Effective diffusivity in (\ref{3dcell}) is unknown, see \cite{SXZ_13} for 
related KPP problem, and \cite{RZ_07} for an upper bound. Extrapolating the KPP scaling in \cite{SXZ_13} on 
the flow (\ref{3dcell-1}), effective diffusivity at small molecular diffusivity $\epsilon$ scales like $O(\epsilon^{p})$, $p \approx 0.26$. 
In 2D, $p=1/2$, see \cite{fannjiang}, also Fig.3 in \cite{Biferale_95}.    


\begin{thebibliography}{99}

\bibitem{Biferale_95}L. Biferale, A. Crisanti, M. Vergassola, A. Vulpiani,
{\em Eddy diffusivities in scalar transport}, Phys. Fluids 7(11), pp. 2725 --2734, 1995.

\bibitem{fannjiang} A. Fannjiang and G. Papanicolaou,
{\em Convection enhanced diffusion for periodic flows}, SIAM J.
Appl. Math., {\bf 54} (1992), pp. 333-408.

\bibitem{RZ_07}L. Ryzhik and A. Zlatos,
{\em KPP pulsating front speed-up by flows}, Comm. Math. Sci., {\bf
5} (2007), pp. 575-593.

\bibitem{SXZ_13}L. Shen, J. Xin and A. Zhou, {\em Finite Element Computation of KPP
Front Speeds in 3D Cellular and ABC Flows}, Math Model. Natural Phenom., 8(3), 2013, pp. 182-197. 




\end{thebibliography}

\end{document}
