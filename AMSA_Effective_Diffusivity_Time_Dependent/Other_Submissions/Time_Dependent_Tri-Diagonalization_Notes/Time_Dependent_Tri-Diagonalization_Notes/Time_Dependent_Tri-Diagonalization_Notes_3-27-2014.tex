\documentclass{article}
\usepackage[dvips]{epsfig,graphics,color,graphicx}
\usepackage{amsfonts}
\usepackage{amssymb, amsmath}
\usepackage{graphicx}
\usepackage[dvips]{epsfig,graphics,color}
\newcommand{\no}{\nonumber}
\newcommand{\ra}{\to}
\newcommand{\eps}{\mbox{$\epsilon$}}
\newcommand{\be}{\begin{equation}}
\newcommand{\ee}{\end{equation}}
\newcommand{\ba}{\begin{eqnarray}}
\newcommand{\ea}{\end{eqnarray}}
\newcommand{\lam}{\mbox{$\lambda$}}
\newtheorem{theorem}{Theorem}[section]
 %\newcommand{\bth}{\begin{theorem}}
%\newcommand{\eth}{\end{theorem}}
\newtheorem{lemma}{Lemma}[section]
   \newcommand{\blem}{\begin{lemma}}
   \newcommand{\elem}{\end{lemma}}
\newtheorem{proposition}{Proposition}[section]
   \newcommand{\bprop}{\begin{proposition}}
   \newcommand{\eprop}{\end{proposition}}
\newtheorem{definition}{Definition}[section]
\newtheorem{corollary}{Corollary}[section]
\newtheorem{algorithm}{Algorithm}[section]
\newtheorem{example}{Example}[section]
\newtheorem{assumption}{Assumption}[section]
\newtheorem{remark}{Remark}[section]
\renewcommand{\theequation}{\arabic{section}.\arabic{equation}}
\renewcommand{\thetheorem}{\arabic{section}.\arabic{theorem}}
\renewcommand{\thelemma}{\arabic{section}.\arabic{lemma}}
\renewcommand{\theproposition}{\arabic{section}.\arabic{proposition}}
\renewcommand{\thedefinition}{\arabic{section}.\arabic{definition}}
\renewcommand{\thecorollary}{\arabic{section}.\arabic{corollary}}
\renewcommand{\thealgorithm}{\arabic{section}.\arabic{algorithm}}
\renewcommand{\thefigure}{\arabic{section}.\arabic{figure}}
\newcommand{\proof}{\vspace{1ex}\noindent{\em Proof}: \ }

% Murphy's short cut commands
\newcommand{\e}{\mathrm{e}}
\newcommand{\I}{\mathrm{i}}
\newcommand{\Hb}{\mathbf{H}}
\newcommand\bkappa{\mbox{\boldmath${\kappa}$}}
\newcommand\balpha{\mbox{\boldmath${\alpha}$}}
\newcommand\Kbc{\mbox{\boldmath${\mathcal{K}}$}}
\newcommand{\Hc}{\mathcal{H}}
\newcommand{\Ac}{\mathcal{A}}
\newcommand{\Tc}{\mathcal{T}}
\newcommand{\Vc}{\mathcal{V}}
\newcommand{\Fc}{\mathcal{F}}
\newcommand{\Ec}{\mathcal{E}}



\begin{document}

\title{Decomposing parabolic eigenvalue problem}
\author{J. Xin and N. B. Murphy }
\date{}
\maketitle
\begin{abstract}
A decomposition of a time periodic parabolic eigenvalue problem and an
application to our effective diffusivity problem.

\end{abstract}
\bigskip
 
\section{An Example}
Consider the parabolic eigenvalue problem:
\be
\psi_t -\Delta_x \, \psi + \cos (t) \; b(x) \, \psi = \lam \, \psi, \label{eig1}
\ee
subject to $2\pi$ periodic boundary conditions in $x$ and $t$. 
\medskip

Substituting $\cos t = (e^{it} + e^{-it})/2$, and $\psi = \sum_{\ell} \psi_\ell (x) e^{i \ell t}$ in (\ref{eig1}), 
we have:
\be
\sum_\ell \,e^{i\ell t}\, (i \ell \, \psi_\ell\, - \Delta_x \psi_\ell) + {1\over 2} b(x)\, \sum_{\ell} (e^{i(\ell+1)t} + e^{i(\ell-1) t}) \psi_\ell 
= \lam \sum_\ell \psi_\ell e^{i\ell t}, \no
\ee
or:
\be
 \sum_\ell \,e^{i\ell t}\, (i \ell \, \psi_\ell\, - \Delta_x \psi_\ell)  + {1\over 2} b(x)\, \sum_{\ell} (\psi_{\ell-1} + \psi_{\ell+1}) \, e^{i\ell t} 
= \lam \sum_\ell \psi_\ell e^{i\ell t}. \no
\ee
Extracting the $\ell$-th mode on both sides gives:
\be
(-\Delta_x + i\, \ell)\, \psi_\ell +{1\over 2} \, b(x)\, (\psi_{\ell-1} + \psi_{\ell+1}) = \lam \, \psi_\ell, \;\; \ell \in Z, \label{eig2}
\ee
which can be put in a tri-diagonal matrix acting on $[\cdots, \psi_{\ell-1}, \psi_\ell, \psi_{\ell+1}, \cdots ]'$,
  with $b(x)/2$ on the off-diagonals, 
$-\Delta_x + i\, \ell $ on the diagonals.

Let us find a similar derivation for the eigenvalue problem of the 
advection-diffusion operator, with advection field being the time periodic cell flow:
\be \label{eig3}
\vec{u}(t,\vec{x})%=(\cos y + \cos t \, \sin y, \cos x + \cos t \sin x)
       =(\cos y,\cos x)+\delta\cos t \,(\sin y,\sin x)
       :=\vec{u}_1(\vec{x})+\delta\cos{t}\,\vec{u}_2(\vec{x}). 
\ee

Question: Is it possible to carry out a similar decomposition in $x$ and $y$ and fully reduce the 
differential system like (\ref{eig2}) into an algebraic system ?

%
\subsection{An Application to our effective diffusivity problem}
%
Consider the eigenvalue problem $A\psi_l=\imath\lambda_l\psi_l$, $\imath=\sqrt{-1}$,
$\lambda_l\in\mathbb{R}$, $l=1,2,3,\ldots$, involving the integro-differential operator
$A=(-\Delta)^{-1}(\partial_t+\vec{u}\cdot\vec{\nabla})$,  
introduced in equation (45) of our (attached) effective-diffusivity
paper, with $\vec{u}\mapsto-\vec{u}$. Here $A$ is an anti-symmetric (normal)
operator and the incompressible velocity field
$\vec{u}(t,\vec{x})$
%$=\vec{u}_1(\vec{x})+\delta\cos{t}\,\vec{u}_2(\vec{x})$
is given in equation \eqref{eig3} above. The equation $A\psi_l=\imath\lambda_l\psi_l$ may be
rewritten as     
%
\begin{align}\label{eq:Eig_prob}
  (\partial_t+\vec{u}\cdot\vec{\nabla})\psi_l=-\imath\lambda_l\Delta\psi_l.
\end{align}
%
The eigenfunctions $\psi_l$ satisfy the following orthogonality condition
%$\langle\vec{\nabla}\psi_\ell^{\,l}\cdot\vec{\nabla}\psi_m^k\rangle=\delta_{jk}$,
%
\begin{align}\label{eq:Orthogonal}
  \langle\psi_l,\psi_i\rangle_1=\langle\vec{\nabla}\psi_l\cdot\vec{\nabla}\psi_i\rangle=\delta_{li},
\end{align}
%
where $\delta_{li}$ is the Kronecker delta and $\langle\cdot\rangle$ denotes space-time
averaging over the period cell $\Tc\times\Vc$, with $\Tc=[0,2\pi]$ and
$\Vc=[0,2\pi]\times[0,2\pi]$.




The eigenfunction $\psi_l$ is $\Tc\times\Vc$ periodic, mean-zero, and
$\psi_l\in\Ac(\Tc)\otimes\Hc^1(\Vc)$, i.e. it is absolutely continuous in time
for $t\in\Tc$, and is in the Sobolev space $\Hc^1(\Vc)$ for
$\vec{x}\in\Vc$. We denote the class of such functions by $\Fc$
%
\begin{align}\label{eq:F}
  \Fc=\{f\in\Ac(\Tc)\otimes\Hc^1(\Vc)\,|\, \langle f\rangle=0  \text{ and is periodic on }
  \, \Tc\times\Vc\}. 
\end{align}
%
Since the orthogonal set $\{\e^{\imath\ell t}\}$,
${\ell\in\mathbb{Z}}$, is dense in $\Ac(\Tc)$, we may represent $\psi_l$ by 
%
\begin{align}\label{eq:t_Expansion}
  \psi_l(t,\vec{x})=\sum_\ell\psi_\ell^{\,l}(\vec{x})\,\e^{\imath \ell t},
\end{align}
%
where $\psi_\ell^{\,l}\in\Hc^1(\Vc)$. Write $\cos{t}=(\e^{\imath t}+\e^{-\imath t})/2$ and insert
this and \eqref{eq:t_Expansion} into equation \eqref{eq:Eig_prob},
yielding   
%
\begin{align}
  \sum_\ell(\imath \ell + \vec{u}_1\cdot\vec{\nabla}+\imath\lambda_l\Delta)\psi_\ell^{\,l}(\vec{x})\e^{\imath \ell t}
 +\frac{\delta}{2}\sum_\ell(\e^{\imath(\ell+1)t}+\e^{\imath(\ell-1)t})\,\vec{u}_2\cdot\vec{\nabla}\psi_\ell^{\,l}(\vec{x})=0,
\end{align}
%
or:
%
\begin{align}
  \sum_\ell\left[(\imath \ell + \vec{u}_1\cdot\vec{\nabla}+\imath\lambda_l\Delta)\psi_\ell^{\,l}(\vec{x})
 +\frac{\delta}{2}\vec{u}_2\cdot\vec{\nabla}(\psi^{\,l}_{\ell-1}(\vec{x})+\psi^{\,l}_{\ell+1}(\vec{x}))\right]\e^{\imath \ell t}=0.
\end{align}
%
By the completeness in $L^2(\Tc)$ of the orthogonal set $\{\e^{\imath \ell t}\}$ we
have, for all $\ell\in\mathbb{Z}$, 
%
\begin{align}\label{eq:Eig_prob_shift}
  (\imath \ell + \vec{u}_1\cdot\vec{\nabla})\psi_\ell^{\,l}(\vec{x})
 +\frac{\delta}{2}\vec{u}_2\cdot\vec{\nabla}(\psi_{\ell-1}^{\,l}(\vec{x})+\psi_{\ell+1}^{\,l}(\vec{x}))
 =-\imath\lambda_l\Delta\psi_\ell^{\,l}(\vec{x}).
\end{align}
%




The system of partial differential equations in
\eqref{eq:Eig_prob_shift} can be reduced to a system of algebraic
equations as follows. Recall that
$\vec{u}_1(\vec{x})=(\cos{y},\cos{x})$ and
$\vec{u}_2(\vec{x})=(\sin{y},\sin{x})$, which implies that
%
\begin{align}\label{eq:material_Derivative_psi}
  (\vec{u}_1\cdot\vec{\nabla})\psi_\ell^{\,l}(\vec{x})
          &=\cos{y}\,\partial_x\psi_\ell^{\,l}(\vec{x})+\cos{x}\,\partial_y\psi_\ell^{\,l}(\vec{x})\\
  (\vec{u}_2\cdot\vec{\nabla})\psi_\ell^{\,l}(\vec{x})
          &=\sin{y}\,\partial_x\psi_\ell^{\,l}(\vec{x})+\sin{x}\,\partial_y\psi_\ell^{\,l}(\vec{x})\notag
\end{align}
%
Since $\psi_\ell^{\,l}\in\Hc^1(\Vc)$ and the orthogonal set $\{\e^{\imath (mx+ny)}\}$,
$m,n\in\mathbb{Z}$, is dense in this space, we can represent
$\psi_\ell^{\,l}(\vec{x})$ by
%
\begin{align}\label{eq:x_Expansion}
  \psi_\ell^{\,l}(\vec{x})=\sum_{m,n}c^{\,l}_{\ell,m,n}\,\e^{\imath (mx+ny)}
\end{align}
%
Write $\cos{x}=(\e^{\imath x}+\e^{-\imath x})/2$ and
$\sin{x}=(\e^{\imath x}-\e^{-\imath x})/(2\imath)$, for example, and insert this
and \eqref{eq:x_Expansion} into equation
\eqref{eq:material_Derivative_psi}, yielding
%
\begin{align}
  (\vec{u}_1&\cdot\vec{\nabla})\psi_\ell^{\,l}\\
    &=\frac{1}{2}\sum_{m,n}c^{\,l}_{\ell,m,n}\left[
        \imath m\,\e^{\imath mx}(\e^{\imath(n+1)y}+\e^{\imath(n-1)y})
        +\imath n\,\e^{\imath ny}(\e^{\imath(m+1)x}+\e^{\imath(m-1)x})
                       \right]
    \notag\\
%    
  (\vec{u}_2&\cdot\vec{\nabla})\psi_\ell^{\,l}\notag\\
    &=\frac{1}{2\imath}\sum_{m,n}c^{\,l}_{\ell,m,n}\left[
        \imath m\,\e^{\imath mx}(\e^{\imath(n+1)y}-\e^{\imath(n-1)y})
        +\imath n\,\e^{\imath ny}(\e^{\imath(m+1)x}-\e^{\imath(m-1)x})
                       \right]
                       \notag
\end{align}
%
or:
%
\begin{align}\label{eq:Algebraic_advection}
  (\vec{u}_1\cdot\vec{\nabla})\psi_\ell^{\,l}
    &=\frac{\imath}{2}\sum_{m,n}
    [m(c^{\,l}_{\ell,m,n-1}+c^{\,l}_{\ell,m,n+1})+n(c^{\,l}_{\ell,m-1,n}+c^{\,l}_{\ell,m+1,n})]\e^{\imath (mx+ny)}
   \notag\\
  (\vec{u}_2\cdot\vec{\nabla})\psi_\ell^{\,l}
    &=\frac{1}{2}\sum_{m,n}
    [m(c^{\,l}_{\ell,m,n-1}-c^{\,l}_{\ell,m,n+1})+n(c^{\,l}_{\ell,m-1,n}-c^{\,l}_{\ell,m+1,n})]\e^{\imath (mx+ny)} .
\end{align}
%
We also have 
%
\begin{align}\label{eq:Algebraic_diffusion}
  -\Delta\psi_\ell^{\,l}=\sum_{m,n}c^{\,l}_{\ell,m,n}(m^2+n^2)\e^{\imath (mx+ny)}
\end{align}
%
By the completeness of the orthogonal set $\{\e^{\imath (mx+ny)}\}$, 
inserting equations \eqref{eq:Algebraic_advection} and
\eqref{eq:Algebraic_diffusion} into equation
\eqref{eq:Eig_prob_shift} yields
%
\begin{align}
\imath \ell\,c^{\,l}_{\ell,m,n}+\frac{\imath}{2}
[&m(c^{\,l}_{\ell,m,n-1}+c^{\,l}_{\ell,m,n+1})+n(c^{\,l}_{\ell,m-1,n}+c^{\,l}_{\ell,m+1,n})]
\notag\\
+\frac{\delta}{4}[&m(c^{\,l}_{\ell-1,m,n-1}-c^{\,l}_{\ell-1,m,n+1})+n(c^{\,l}_{\ell-1,m-1,n}-c^{\,l}_{\ell-1,m+1,n})
\notag\\
+&m(c^{\,l}_{\ell+1,m,n-1}-c^{\,l}_{\ell+1,m,n+1})+n(c^{\,l}_{\ell+1,m-1,n}-c^{\,l}_{\ell+1,m+1,n})]
\notag\\
=\imath\lambda_l(&m^2+n^2)c^{\,l}_{\ell,m,n},
\end{align}
%
which is an infinite system of algebraic equations for the unknown
Fourier coefficients $c^{\,l}_{\ell,m,n}$ associated with the
eigenfunctions $\psi_l(t,\vec{x})$ and eigenvalues $\imath\lambda_l$,
$l\in\mathbb{N}$ and $\ell,m,n\in\mathbb{Z}$. Recalling that $\psi_l$ is
mean-zero $\langle\psi_l\rangle=0$, we have that $\ell^2+m^2+n^2>0$.




We now discuss how the orthogonality condition
$\langle\psi_l,\psi_i\rangle_1=\delta_{li}$ in \eqref{eq:Orthogonal} is
transformed by the Fourier expansion of the
eigenfunctions $\psi_l(t,\vec{x})$. This expansion of $\psi_l(t,\vec{x})$ 
implies that for $\vec{\nabla}\psi_l(t,\vec{x})$ as follows 
%
\begin{align}\label{eq:Eigenfunction_Grad_expansion}
  \psi_l(t,\vec{x})=\sum_{\ell,m,n}c^{\,l}_{\ell,m,n}\e^{\imath (\ell t+mx+ny)}\Rightarrow
  \vec{\nabla}\psi_l(t,\vec{x})=\sum_{\ell,m,n}c^{\,l}_{\ell,m,n}\,(m,n)\,\e^{\imath (\ell t+mx+ny)}.
\end{align}
%
Therefore, by the orthogonality relation 
%
\begin{align}\label{eq:Trig_orthogonal}  
      \left\langle
      \e^{\imath (\ell t+mx+ny)},\,\e^{\imath (\ell^\prime t+m^\prime x+n^\prime y)}
      \right\rangle_2
      =
      (2\pi)^3\delta_{\ell,\ell^\prime}\delta_{m,m^\prime}\delta_{n,n^\prime},
\end{align}
%
where $\langle\cdot,\cdot\rangle_2$ denotes the $L^2(\Tc\times\Vc)$ inner-product, we have that
the orthogonality relation in \eqref{eq:Orthogonal} is transformed to  
%
\begin{align}\label{eq:Ortho_Clmn}
  \delta_{li}=\langle\vec{\nabla}\psi_l\cdot\vec{\nabla}\psi_i\rangle
      = (2\pi)^3\sum_{m,n}(m^2+n^2)\overline{c^{\,l}_{0,m,n}}c^{\,i}_{0,m,n}
\end{align}
%


Recall that $\sum_ii^{-p}$ converges for all $p>1$. Consequently, from
equation \eqref{eq:Ortho_Clmn} we see that the square modulus of the
Fourier coefficients $c^{\,l}_{0,m,n}$ must have the asymptotic behavior 
$|c^{\,l}_{0,m,n}|^2\sim o((m^2+n^2)^{-3/2})$ as $m,n\to\pm\infty$. Since
$\psi_l(\cdot,\vec{x})\in\Ac(\Tc)$, i.e. $\partial_t\psi_l(\cdot,\vec{x})\in L^2(\Tc)$, we also
have $|c^{\,l}_{\ell,m,n}|^2\sim o(\ell^{-3})$ as $\ell\to\pm\infty$. Since
$\partial_t\vec{\nabla}\psi_l\in L^2(\Tc\times\Vc)$ we may generalize 
both of these statements by the following
%
\begin{align}\label{eq:Clmn_assymptotics}
  |c^{\,l}_{\ell,m,n}|^2\sim o(\ell^{-3}(m^2+n^2)^{-3/2}), \text{ as } \,\ell,m,n\to\pm\infty.
\end{align}
%



We now use the special nature of the velocity field in \eqref{eig3}
and the Fourier expansion of the eigenfunctions in
\eqref{eq:Eigenfunction_Grad_expansion} to show that the
Radon--Stieltjes integral representation for the symmetric $\bkappa^*$
and anti-symmetric $\balpha^*$ parts of the effective diffusivity
tensor $\Kbc^*$ depend only on the Fourier coefficients
$c^{\,l}_{\ell,m,n}$ for $\ell,m,n\in\{-1,0,1\}$. Recall that the Hilbert space
$\Hc$ underlying this problem is
%
\begin{align}\label{eq:Hilbert_Space}
  \Hc=\{f\in L^2(\Tc)\otimes\Hc^1(\Vc) \,|\, \langle f\rangle=0 \text{ and is periodic on}\; \Tc\times\Vc\}, 
\end{align}
% 
with $A$ maximal normal, i.e. $\imath A$ self-adjoint, on $\Fc\subset\Hc$ defined
in \eqref{eq:F}. Since the eigenfunctions $\psi_l$ and $\psi_i$ of $A$
associated with distinct eigenvalues, $\lambda_l\neq\lambda_i$, are orthonormal,
$\langle\psi_l,\psi_i\rangle_1=\langle\vec{\nabla}\psi_l\cdot\vec{\nabla}\psi_i\rangle=\delta_{li}$, the span of all
eigenfunctions constitutes a closed linear manifold $\Ec$ in
$\Fc$. Let $\Ec^\perp$ be its (closed) orthogonal complement in $\Hc$
so that [Stone]
%$\Hc=\Ec\oplus\Ec^\perp$
\begin{align}\label{eq:Hilbert_Space_Decomposition}
  \Hc=\Ec\oplus\Ec^\perp.
\end{align}
%



Recall the cell problem
% 
\begin{align}\label{eq:Cell_Problem}
  (\varepsilon-A)\chi_j=g_j, \quad g_j=-\Delta^{-1}u_j,
\end{align}
%
where $u_j$ is the $j^{\text{th}}$ component of the velocity field
$\vec{u}$. Since $\chi_j,g_j\in\Fc$ and $\Fc\subset\Ec\oplus\Ec^\perp$ they have the
following representations 
%
\begin{align}\label{eq:chi_g_expansions}
  \chi_j=\sum_l\langle\psi_l,\chi_j\rangle_1\,\psi_l +\chi_j^\perp, \quad  g_j=\sum_l\langle\psi_l,g_j\rangle_1\,\psi_l +g_j^\perp,
\end{align}
%
where $\psi_l\in\Ec$ and $\chi_j^\perp,g_j^\perp\in\Ec^\perp$. Using $A\psi_l=\imath\lambda_l\psi_l$ and the
orthonormality of the set $\{\psi_l\}$, inserting
\eqref{eq:chi_g_expansions} into the cell problem
\eqref{eq:Cell_Problem} yields
%
\begin{align}\label{eq:chi_g_cell}
  \sum_l[(\varepsilon-\imath\lambda_l)\langle\psi_l,\chi_j\rangle_1-\langle\psi_l,g_j\rangle_1]\psi_l+(\varepsilon+A)\chi_j^\perp-g_j^\perp=0.
\end{align}
%
By the orthonormality of the set $\{\psi_l\}$ and since
$\langle A\chi_j^\perp,\psi_l\rangle_1=-\langle\chi_j^\perp,A\psi_l\rangle_1=-\imath\lambda_l\langle\chi_j^\perp,\psi_l\rangle_1=0$, taking the
inner-product of both sides of \eqref{eq:chi_g_cell} with $\psi_l$ yields 
%
\begin{align}\label{eq:Coefficients}
  \langle\psi_l,\chi_j\rangle_1=\frac{\langle\psi_l,g_j\rangle_1}{(\varepsilon-\imath\lambda_l)}\,.
\end{align}
%




Recall that the components $\kappa^*_{jk}$ and $\alpha^*_{jk}$ of $\bkappa^*$
and $\balpha^*$ are given by 
%
\begin{align}
  \kappa^*_{jk}=\varepsilon(\delta_{jk}+\langle\chi_j,\chi_k\rangle_1), \quad
  \alpha^*_{jk}=\langle A\chi_j,\chi_k\rangle_1.
\end{align}
%
From equations \eqref{eq:chi_g_expansions} and \eqref{eq:Coefficients}
and the orthonormality of the set $\{\psi_l\}$, we have
%
\begin{align}\label{eq:Stieltjes-Radon_Rep}
  \langle\chi_j,\chi_k\rangle_1-\langle\chi_j^\perp,\chi_k^\perp\rangle_1&=\sum_l\overline{\langle\psi_l,\chi_j\rangle}_1\langle\psi_l,\chi_k\rangle_1
         =\sum_l\frac{\overline{\langle\psi_l,g_j\rangle}_1\langle\psi_l,g_k\rangle_1}{\varepsilon^2+\lambda_l^2}
         \notag\\
  \langle A\chi_j,\chi_k\rangle_1-\langle A\chi_j^\perp,\chi_k^\perp\rangle_1&=\sum_l(-\imath\lambda_l)\overline{\langle\psi_l,\chi_j\rangle}_1\langle\psi_l,\chi_k\rangle_1
         =\sum_l\frac{\overline{\langle\psi_l,g_j\rangle}_1\langle\psi_l,g_k\rangle_1}{\varepsilon^2+\lambda_l^2}      
\end{align}
%
The right hand sides of the formulas in equation
\eqref{eq:Stieltjes-Radon_Rep} are Radon--Stieltjes integrals
associated with a \emph{discrete} measure. The terms $\langle\chi_j^\perp,\chi_k^\perp\rangle_1$ and
$\langle A\chi_j^\perp,\chi_k^\perp\rangle_1$ also have Radon--Stieltjes integral representations
with respect to \emph{continuous} measures, and provides the standard
decomposition of the \emph{spectral measure} into its discrete and
continuous components, in the general setting [Stone].




We now show that the special nature of the velocity field in
\eqref{eig3} and the Fourier expansion of the eigenfunctions $\psi_l$ in  
\eqref{eq:Eigenfunction_Grad_expansion} allow the spectral weights
$\langle\psi_l,g_j\rangle$ in equation \eqref{eq:Stieltjes-Radon_Rep} to be given in
terms of the Fourier coefficients $c^{\,l}_{\ell,m,n}$ for the reduced
index set $\ell,m,n\in\{-1,0,1\}$. First note that, since
$u_j(t,\cdot)\in\Hc^1(\Vc)\subset L^2(\Vc)$,
%
\begin{align}\label{eq:H1_to_L2}
  \langle\psi_l,g_j\rangle_1=\langle\vec{\nabla}\psi_l\cdot\vec{\nabla}(-\Delta)^{-1}u_j\rangle
         =\langle\psi_l,(-\Delta)(-\Delta)^{-1}u_j\rangle_2
         =\langle\psi_l,u_j\rangle_2.
\end{align}
%
Writing
$\cos{x}=(\e^{\imath x}+\e^{-\imath x})/2$ and
$\sin{x}=(\e^{\imath x}-\e^{-\imath x})/(2\imath)$, for example, from equation
\eqref{eig3} we have that 
%
\begin{align}
  u_1(t,x,y)&=\cos{y}+\cos{t}\sin{y}\\
           &=\frac{1}{2}\left(\e^{\imath y}+\e^{-\imath y}\right)
            +\frac{1}{4\imath}\left(\e^{\imath t}+\e^{-\imath t})(\e^{\imath y}-\e^{-\imath y}\right)
            \notag\\
           &=\frac{1}{2}\left(\e^{\imath y}+\e^{-\imath y}\right)
            +\frac{1}{4\imath}\left(\e^{\imath(t+y)}-\e^{\imath(t-y)}+\e^{\imath(-t+y)}-\e^{\imath(-t-y)}\right), 
\end{align}
%
and $u_2(t,x,y)=u_1(t,y,x)$. This, equation \eqref{eq:H1_to_L2}, and the
orthogonality relation in \eqref{eq:Trig_orthogonal} imply that
%
\begin{align}
  \langle\psi_l,g_1\rangle_1=(2\pi)^3\left[\frac{1}{2}\left(c^{\,l}_{0,0,1}+c^{\,l}_{0,0,-1}\right)
               +\frac{1}{4\imath}\left(c^{\,l}_{1,0,1}-c^{\,l}_{1,0,-1}+c^{\,l}_{-1,0,1}-c^{\,l}_{-1,0,-1}\right) 
               \right]
               \notag\\
  \langle\psi_l,g_2\rangle_1=(2\pi)^3\left[\frac{1}{2}\left(c^{\,l}_{0,1,0}+c^{\,l}_{0,-1,0}\right)
               +\frac{1}{4\imath}\left(c^{\,l}_{1,1,0}-c^{\,l}_{1,-1,0}+c^{\,l}_{-1,1,0}-c^{\,l}_{-1,-1,0}\right) 
               \right]
\end{align}
%




Since $\vec{u}_i$ is incompressible, there exists
an anti-symmetric matrix $\Hb_i$ such that
$\vec{u}_i=\vec{\nabla}\cdot\Hb_i$. This allows us to write
$\vec{u}_i\cdot\vec{\nabla}=\vec{\nabla}\cdot\Hb_i\vec{\nabla}$, which is an anti-symmetric
operator. When $\delta=0$, the velocity field $\vec{u}$ is time-independent
and the operator $A$, which arises from the cell problem, becomes
$A=\Delta^{-1}(\vec{u}_1\cdot\vec{\nabla})$. In this case, the eigenvalue problem in
\eqref{eq:Eig_prob} becomes 
%
\begin{align}\label{eq:Eig_prob_steady}
  \vec{\nabla}\cdot\Hb_1\vec{\nabla}\psi=\lambda\Delta\psi.
\end{align}
%
Discretizing this equation leads to a generalized eigenvalue
problem involving \emph{sparse} matrices. This matrix formulation has
all the desired properties of the associated abstract Hilbert space
formulation. (I will be adding the details of this to our paper soon.)
From this matrix problem, we obtain a discrete approximation of the
Radon--Stieltjes integral representation for the symmetric $\bkappa^*$
and anti-symmetric $\balpha^*$ parts of the effective diffusivity
tensor $\Kbc^*$, displayed in equation (35) of our (attached) paper.  









\section{Three-Dim Steady Cellular Flows}
The 3 dimensional (3D) steady cellular flows are:
\be
B =
(\Phi_x(x,y)W'(z),\Phi_y(x,y)W'(z),k\Phi(x,y)W(z)), \label{3dcell}
\ee
with $-\Delta \Phi = k \Phi.$   
\medskip

A special case is $k=2$, then
\be
B(x,y,z) =
(-\sin{x}\cos{y}\cos{z},-\cos{x}\sin{y}\cos{z},2\cos{x}\cos{y}\sin{z}).\label{3dcell-1}
\ee
\medskip

Question: Is the effective diffusivity problem easier for (\ref{3dcell-1}) than (\ref{eig3})? 
\medskip

Effective diffusivity in (\ref{3dcell}) is unknown, see \cite{SXZ_13} for 
related KPP problem, and \cite{RZ_07} for an upper bound. Extrapolating the KPP scaling in \cite{SXZ_13} on 
the flow (\ref{3dcell-1}), effective diffusivity at small molecular diffusivity $\epsilon$ scales like $O(\epsilon^{p})$, $p \approx 0.26$. 
In 2D, $p=1/2$, see \cite{fannjiang}, also Fig.3 in \cite{Biferale_95}.    


\begin{thebibliography}{99}

\bibitem{Biferale_95}L. Biferale, A. Crisanti, M. Vergassola, A. Vulpiani,
{\em Eddy diffusivities in scalar transport}, Phys. Fluids 7(11), pp. 2725 --2734, 1995.

\bibitem{fannjiang} A. Fannjiang and G. Papanicolaou,
{\em Convection enhanced diffusion for periodic flows}, SIAM J.
Appl. Math., {\bf 54} (1992), pp. 333-408.

\bibitem{RZ_07}L. Ryzhik and A. Zlatos,
{\em KPP pulsating front speed-up by flows}, Comm. Math. Sci., {\bf
5} (2007), pp. 575-593.

\bibitem{SXZ_13}L. Shen, J. Xin and A. Zhou, {\em Finite Element Computation of KPP
Front Speeds in 3D Cellular and ABC Flows}, Math Model. Natural Phenom., 8(3), 2013, pp. 182-197. 




\end{thebibliography}

\end{document}
