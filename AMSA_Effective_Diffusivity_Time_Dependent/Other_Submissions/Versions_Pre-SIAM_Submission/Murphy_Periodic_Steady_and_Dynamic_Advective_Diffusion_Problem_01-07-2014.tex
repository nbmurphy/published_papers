\documentclass[11pt]{amsart}
%\usepackage{latexsym, amssymb, enumerate, amsmath}
\usepackage{graphicx,amssymb,amsmath,amsfonts,mathrsfs}

\setlength{\textwidth}{6.5in}
\setlength{\textheight}{9.0in}
\setlength{\oddsidemargin}{0in}
\setlength{\evensidemargin}{0in}
\setlength{\topmargin}{-0.5in}

\renewcommand{\topfraction}{0.85}
\renewcommand{\textfraction}{0.1}
\renewcommand{\floatpagefraction}{0.55}%0.75


\newcommand{\ph}{\hat{\phi}}
\newcommand{\pt}{\tilde{\phi}} 
\newcommand{\pc}{\check{\phi}}
\newcommand{\gh}{\hat{\gamma}}
\newcommand{\Dh}{\hat{\Delta}}
\newcommand{\dha}{\hat{\delta}}
\newcommand{\qh}{\hat{q}}
\newcommand{\xh}{\hat{x}} 
\newcommand{\HM}{\mathcal{H}_{\text{max}}}
\newcommand{\Hm}{\mathcal{H}_{\text{min}}}
\newcommand{\sech}{\rm \hspace{0.7mm}sech}
\newcommand{\I}{\mathrm{i}}
\newcommand{\e}{\mathrm{e}}
\renewcommand{\d}{\mathrm{d}}
\newcommand{\hh}{\hat{h}}
\newcommand{\mh}{m_r}
\newcommand{\mt}{m_i}

\newcommand{\Mb}{\mathbf{M}}
\newcommand{\Xb}{\mathbf{X}}
\newcommand{\Tb}{\mathbf{T}}
\newcommand{\Hb}{\mathbf{H}}
\newcommand{\Kb}{\mathbf{K}}
\newcommand{\Jb}{\mathbf{J}}
\newcommand{\Ib}{\mathbf{I}}
\newcommand{\Sb}{\mathbf{S}}
\newcommand{\Rb}{\mathbf{R}}
\newcommand{\Ab}{\mathbf{A}}
\newcommand{\Bb}{\mathbf{B}}
\newcommand{\Eb}{\mathbf{E}}
\newcommand{\Qb}{\mathbf{Q}}

\newcommand{\Kc}{\mathcal{K}}
\newcommand\Kbc{\mbox{\boldmath${\mathcal{K}}$}}


\newcommand{\Tc}{\mathcal{T}}
\newcommand{\Vc}{\mathcal{V}}

\newcommand{\Hs}{\mathscr{H}}
\newcommand{\As}{\mathscr{A}}
\newcommand{\Ds}{\mathscr{D}}


\newcommand\bsig{\mbox{\boldmath${\sigma}$}}
\newcommand\beps{\mbox{\boldmath${\epsilon}$}}
\newcommand\bxi{\mbox{\boldmath${\xi}$}}
\newcommand\bmu{\mbox{\boldmath${\mu}$}}
\newcommand\balpha{\mbox{\boldmath${\alpha}$}}
\newcommand\brho{\mbox{\boldmath${\rho}$}}
\newcommand\bDelta{\mbox{\boldmath${\Delta}$}}
\newcommand\bkappa{\mbox{\boldmath${\kappa}$}}
\newcommand\bGamma{\mbox{\boldmath${\Gamma}$}}


\newtheorem{thm}{Theorem}[section]
\newtheorem{prop}[thm]{Proposition}
\newtheorem{lem}[thm]{Lemma}
\newtheorem{cor}[thm]{Corollary}

    %\theoremstyle{definition}

\newtheorem{defn}[thm]{Definition}
\newtheorem{notation}[thm]{Notation}
\newtheorem{example}[thm]{Example}
\newtheorem{conj}[thm]{Conjecture}
\newtheorem{prob}[thm]{Problem}

    %\theoremstyle{remark}

\newtheorem{rem}[thm]{Remark}
    % Use the standard latex environments for theorems, etc. Here is one
          % possible method of declaring them: It numbers all results by the
          % section, and uses a common numbering system for the different
          % environmentts.

\begin{document}

\title{Spectral theory of advective diffusion \\
  by dynamic and steady periodic flows}


% AUTHORS 
%\author{N. B. Murphy, A. Gully, E. Cherkaev, and K. M. Golden}
\author{N. B. Murphy$^\ast$}
\address{$^*$Department of Mathematics, 340 Rowland Hall, University of
  California at Irvine, Irvine, CA 92697-3875, USA}
\email{nbmurphy@math.uci.edu}

\author{J. Xin$^\dag$}
\address{$^{\dag}$Department of Mathematics, 340 Rowland Hall, University of
  California at Irvine, Irvine, CA 92697-3875, USA} 
\email{jxin@math.uci.edu}

\author{J. Zhu$^\star$}
\address{$^\star$University of Utah, Department of Mathematics, 155 S 1400 E
  RM 233, Salt Lake City, UT 84112-009, USA}
\email{zhu@math.utah.edu}

\author{E. Cherkaev$^\ddagger$}
\address{$^\ddagger$University of Utah, Department of Mathematics, 155 S 1400 E
  RM 233, Salt Lake City, UT 84112-009, USA} 
\email{elena@math.utah.edu}

\maketitle
\vspace{-3ex}
\begin{center}
  Department of Mathematics, University of California at Irvine
\end{center}

%\vspace{3ex}


\begin{abstract}
%
The analytic continuation method for representing transport in
composites provides integral representations for the
effective coefficients of two-phase random media. Here we adapt this 
method to characterize the effective thermal transport properties of
advective diffusion by periodic flows. Our novel approach yields
integral representations for the symmetric and 
anti-symmetric parts of the effective diffusivity. These
representations hold for dynamic and steady incompressible flows, and
involve the spectral measure of a self-adjoint or normal linear
operator. In the case of a steady fluid velocity field, the spectral
measure is associated with a Hermitian Hilbert-Schmidt integral
operator, and in the case of dynamic flows, it is associated with an
unbounded integro-differential operator. We utilize the integral
representations to obtain asymptotic behavior of the effective
diffusivity as the molecular diffusivity tends to zero, for model
flows. Our analytical results are supported by numerical computations
of the spectral measures and effective diffusivities.     
%
\end{abstract}

\section{Introduction}\label{sec:Introduction}
%
The long time, large scale behavior of a diffusing particle   
or tracer being advected by an incompressible velocity field 
is equivalent to an enhanced diffusive process \cite{Taylor:PRSL:196} 
with an effective diffusivity tensor $\Kbc^*$.
Determining the effective transport properties of advection enhanced
diffusion is a challenging problem with theoretical and practical 
importance in many fields of science and engineering,
ranging from turbulent combustion to mass, heat, and salt transport in
geophysical flows \cite{Moffatt:RPP:621}. A broad range of
mathematical techniques have been developed that reduce the analysis
of complex fluid flows, with rapidly varying structures in space and
time, to solving averaged or \textit{homogenized} equations that do
not have rapidly varying data, and involve an effective parameter.




Homogenization of the advection-diffusion equation for thermal
transport by random, time-independent fluid velocity fields was
treated in \cite{McLaughlin:SIAM_JAM:780}. This 
reduced the analysis of turbulent diffusion to solving a
diffusion equation involving a homogenized temperature and a
(constant) effective diffusivity tensor $\Kbc^*$. An important
consequence of this analysis is that $\Kbc^*$ is given in terms  
of a \emph{curl-free} stationary stochastic process which satisfies a
steady state diffusion equation, involving a skew-symmetric random
matrix $\Hb$ \cite{Avellaneda:CMP-339,Avellaneda:PRL-753}. By adapting
the analytic continuation method (ACM) of homogenization theory for
composites \cite{Golden:CMP-473}, it was shown that the result in
\cite{McLaughlin:SIAM_JAM:780} leads to an integral
representation for the symmetric part $\bkappa^*$ of $\Kbc^*$,
involving a spectral measure of a self-adjoint random
operator \cite{Avellaneda:CMP-339,Avellaneda:PRL-753}. This integral
representation of $\bkappa^*$ was generalized to the time-dependent
case in \cite{Avellaneda:PRE:3249,Biferale:PF:2725}. Remarkably, this 
method has also been extended to flows with incompressible
\emph{nonzero} effective drift \cite{Pavliotis:PHD_Thesis}, flows
where particles diffuse according to linear collisions
\cite{Pavliotis:IMAJAM:951}, and solute transport in porous media
\cite{Bhattacharya:AAP:1999:951}. All these approaches yield integral
representations of the symmetric and, when appropriate, the
anti-symmetric part $\balpha^*$ of $\Kbc^*$. 




Homogenization of the advection-diffusion equation for periodic or
cellular, incompressible flow fields was treated in
\cite{Fannjiang:SIAM_JAM:333,Fannjiang:1997:1033}. As in the case of
random flows, the effective diffusivity tensor
$\Kbc^*$ is given in terms of a \emph{curl-free} vector field, which
satisfies a diffusion equation involving a skew-symmetric
matrix $\Hb$. Here, we demonstrate that the ACM can
be adapted to this periodic setting to provide integral
representations for both the symmetric $\bkappa^*$ and 
anti-symmetric $\balpha^*$ parts of $\Kbc^*$, for both cases of
steady and time-dependent flows. These integral representations
involve a self-adjoint or normal linear operator and the
(non-dimensional) molecular diffusivity $\varepsilon$. In the case of steady
fluid velocity fields, the spectral measure is associated with a
Hermitian Hilbert-Schmidt integral operator involving the Green's
function of the Laplacian on a rectangle. While in the case of dynamic 
flows, the spectral measure is associated with a Hermitian operator
which is the sum of that for steady flows and an unbounded
integro-differential operator.      
 

We utilize the analytic structure of the  integral
representation for $\Kbc^*$ to obtain its asymptotic behavior for
model flows, as the molecular diffusivity $\varepsilon$ tends to zero. This is
the high P\'{e}clet number regime that is important for the
understanding of transport processes in real fluid flows, where the
molecular diffusivity is often quite small in comparison. In
particular, FINISH THIS PARAGRAPH WHEN WE HAVE CONCRETE RESULTS.
necessary and sufficient conditions for steady periodic flow
fields $\kappa^*\sim\epsilon^{1/2}$, generically, for steady flows and $\kappa^*\sim O(1)$ for
``chaotic'' time-dependent flows. To make this manuscript more self
contained, we have include an appendix in Section \ref{sec:Appendix}
which contains many of the technical detials underlying this work.

%
\section{Mathematical Methods}\label{sec:Mathematical_Methods} 
%
In this section, we formulate the effective parameter problem for
enhanced diffusive transport by advective, periodic flows. We provide
integral representations for the symmetric $\bkappa^*$ and
anti-symmetric $\balpha^*$ parts of the effective diffusivity tensor 
$\Kbc^*$, which hold for both steady and dynamic flows. The effective
parameter problem
\cite{McLaughlin:SIAM_JAM:780,Fannjiang:SIAM_JAM:333,Biferale:PF:2725}
for such transport processes  is reviewed in Section
\ref{sec:Eff_Trans}. Parallels existing between this problem of
homogenization theory \cite{Bensoussan:Book:1978} and the ACM for
representing transport in composites \cite{Golden:CMP-473}, are put
into correspondence in Section \ref{sec:ACM}. In particular, an
abstract Hilbert space framework is provided in Section 
\ref{sec:Hilbert_Space} which places these different effective
parameter problems on common mathematical footing. Within this Hilbert
space setting, we derive in Section \ref{sec:Integral_Rep} integral
representations for $\bkappa^*$ and $\balpha^*$, involving the
molecular diffusivity $\varepsilon$ and a \emph{spectral measure} of a
self-adjoint or normal linear operator. These integral representations
are employed in Section \ref{sec:Assymptotics} to obtain the asymptotic
behavior of the components of $\Kbc^*$ in the scaling regime, where
$\varepsilon\ll1$.  



\subsection{Effective transport by
  advective-diffusion} \label{sec:Eff_Trans}  
%
Consider the advection enhanced diffusive transport of a passive
tracer $\phi(t,\vec{x}\,)$, $t>0$, $\vec{x}\in\mathbb{R}^d$, as described by
the advection-diffusion equation 
%
\begin{align}\label{eq:ADE}
  \partial_t\phi=\kappa_0\Delta \phi+\vec{\nabla}\cdot(\vec{v}\phi), \quad
  \phi(0,\vec{x})=\phi_0(\vec{x}),
  %\qquad
  %\vec{\nabla}\cdot\vec{v}=0
\end{align}
% 
with initial density $\phi_0(\vec{x})$ given. Here, $\partial_t$ denotes partial
differentiation with respect to time $t$, $\Delta=\vec{\nabla}\cdot\vec{\nabla}=\nabla^2$ is
the Laplacian, and $\kappa_0>0$ is the molecular diffusivity. The fluid
velocity field $\vec{v}=\vec{v}(t,\vec{x})$ in \eqref{eq:ADE} is
assumed to be incompressible and mean-zero 
%
\begin{align}\label{eq:incompressible}
  \vec{\nabla}\cdot\vec{v}=0, \quad
  \langle\vec{v}\,\rangle=0,
  %\quad
  %\langle|\vec{v}|^2\rangle<\infty.
\end{align}
%
where we denote by $0$ the null element of all linear spaces in
question. The bounded sets $\Tc\subset\mathbb{R}$ and     
$\Vc\subset\mathbb{R}^d$, with $t\in\Tc$ and $\vec{x}\in\Vc$, define the
space-time period cell ($(d+1)$--torus) $\Tc\otimes\Vc$. In equation
\eqref{eq:incompressible}, we denote by $\langle\cdot\rangle$ spatial averaging over
$\Vc$ in the case of a time-independent velocity field,
$\vec{v}=\vec{v}(\vec{x})$, and when the velocity field is
time-dependent, $\vec{v}=\vec{v}(t,\vec{x})$, $\langle\cdot\rangle$ denotes space-time
averaging over $\Tc\otimes\Vc$. 



We non-dimensionalize equation
\eqref{eq:ADE} as follows. Let $\ell$ and $\tilde{t}$ be typical length
and time scales associated with the problem of interest. Mapping to
the non-dimensional variables $t\mapsto t/\tilde{t}$ and $x_j\mapsto x_j/\ell$, one
finds that $\phi$ satisfies the advection-diffusion equation in
\eqref{eq:ADE} with a non-dimensional molecular diffusivity
$\varepsilon=\tilde{t}\kappa_0/\ell^{\,2}$ and velocity field
$\vec{u}=\tilde{t}\,\vec{v}/\ell$, where $x_j$ is the $j^{\,\text{th}}$
component of the vector $\vec{x}$. 




For $d$-dimensional, mean-zero, incompressible flows $\vec{u}$, there
is a (non-dimensional) skew-symmetric 
matrix $\Hb(t,\vec{x})$ such that (see Section \ref{eq:flow_matrix}
for details) 
%$\Hb^{\,T}=-\Hb$, such that $\vec{u}=\vec{\nabla}\cdot\Hb$. 
% %
\begin{align}\label{eq:u_DH}
 \vec{u}=\vec{\nabla}\cdot\Hb, \qquad  \Hb^{\,T}=-\Hb.
\end{align}
% %
Using this representation of the velocity field $\vec{u}$, which also
satisfies \eqref{eq:incompressible}, equation \eqref{eq:ADE} can be
written as a diffusion equation,  
%
\begin{align}\label{eq:ADE_Divergence}
  \partial_t\phi%&=\varepsilon\Delta \phi+\vec{u}\cdot\vec{\nabla}\phi\\
    %&=\varepsilon\vec{\nabla}\cdot\vec{\nabla}\phi+(\vec{\nabla}\cdot\Hb)\cdot\vec{\nabla}\phi\\
    %&=\vec{\nabla}\cdot[\varepsilon I+\Hb]\vec{\nabla}\phi\\
    %&=\vec{\nabla}\cdot\bkappa\vec{\nabla}\phi
    =\vec{\nabla}\cdot\bkappa\vec{\nabla}\phi, \quad
    %\bkappa=\varepsilon I+\Hb,
    \phi(0,\vec{x})=\phi_0(\vec{x}),
    \qquad
    \bkappa=\varepsilon\Ib+\Hb,
\end{align}
%
where $\bkappa(t,\vec{x})=\varepsilon\Ib+\Hb(t,\vec{x})$ can be viewed as a local
diffusivity tensor with coefficients
%
\begin{align}\label{eq:kappa_coeff}
  \kappa_{jk}=\varepsilon\delta_{jk}+H_{jk},\quad j,k=1,\cdots,d.
\end{align}
%
Here, $\delta_{jk}$ is the Kronecker delta and we denote by $\Ib$ the
identity operator on all linear spaces in question.    






We are interested in the dynamics of $\phi$ in \eqref{eq:ADE_Divergence}
for \emph{large} length and time scales, and when the initial density
$\phi_0$ is slowly varying relative to the velocity field
$\vec{u}$. Anticipating that $\phi$ will have diffusive dynamics, we
re-scale space and time by $\vec{x}\mapsto\vec{x}/\delta$ and $t\mapsto t/\delta^2$,
respectively.  For periodic diffusivity coefficients in
\eqref{eq:ADE_Divergence} which are uniformly elliptic but not
necessarily symmetric, it can be shown \cite{Fannjiang:SIAM_JAM:333}
that, as $\delta\to0$, the associated solution $\phi^\delta(t,\vec{x})$ of
\eqref{eq:ADE_Divergence} converges to $\bar{\phi}(t,\vec{x})$,
which satisfies the following diffusion equation involving a (constant)
effective diffusivity tensor $\Kbc^*$ (see Section
\ref{sec:Multiscal_Method} for details)    
%
\begin{align}\label{eq:phi_bar}
  \partial_t\bar{\phi}=\vec{\nabla}\cdot\Kbc^*\vec{\nabla}\bar{\phi}, \quad
  \bar{\phi}(0,\vec{x})=\phi_0(\vec{x}).
\end{align}
%




The components $\Kc^*_{jk}=\Kbc^*\vec{e}_j\cdot\vec{e}_k$ of the tensor
$\Kbc^*$ are obtained by solving the cell problem
\cite{Fannjiang:SIAM_JAM:333} 
% 
\begin{align}\label{eq:Cell_Problem}
  \partial_t\chi_k=\vec{\nabla}\cdot\bkappa(\vec{\nabla}\chi_k+\vec{e}_k\,), \quad
    %=\vec{\nabla}\cdot[(\varepsilon\Ib+\Hb)(\vec{\nabla}\chi_k+\vec{e}_k\,)], \quad
  \langle\vec{\nabla}\chi_k\rangle=0,
%  \quad   k=1,\ldots,d,
\end{align}
%
for each standard basis vector $\vec{e}_k$, $k=1,\ldots,d$, where
$\chi_k=\chi_k(t,\vec{x}\,;\vec{e}_k)$. Equation \eqref{eq:Cell_Problem}
also holds \cite{Fannjiang:SIAM_JAM:333} when the velocity field is
time-independent $\vec{u}=\vec{u}(\vec{x})$, however, in this case
$\chi_k$ is time-independent and $\partial_t\chi_k=0$. The symmetric $\bkappa^*$ and
anti-symmetric $\balpha^*$ parts of $\Kbc^*$ are defined by
%
\begin{align}\label{eq:Symm_Anti-Symm}
  \Kbc^*=\bkappa^*+\balpha^*,\qquad
  \bkappa^*=\frac{1}{2}(\Kbc^*+[\Kbc^*]^{\,T}), \quad
  \balpha^*=\frac{1}{2}(\Kbc^*-[\Kbc^*]^{\,T}).
\end{align}
%
The components $\kappa^*_{jk}$ and $\alpha^*_{jk}$, $j,k=1,\ldots,d$, of $\bkappa^*$
and $\balpha^*$ are given by (see Section \ref{sec:Multiscal_Method}) 
%
\begin{align}\label{eq:Eff_Diffusivity}
 \kappa^*_{jk}=\varepsilon(\delta_{jk}+\langle\vec{\nabla}\chi_j\cdot\vec{\nabla}\chi_k\rangle), \qquad
 \alpha^*_{jk}=\langle\Sb\vec{\nabla}\chi_j\cdot\vec{\nabla}\chi_k\rangle, \qquad
 \Sb=\Hb-(\Delta^{-1})\Tb, \quad \Tb=\partial_t\Ib.
\end{align}
%
Here, the inverse operation $(\Delta^{-1})$ is based on convolution with the
Green's function for the Laplacian $\Delta$ \cite{Stakgold:BVP:2000}, and
the $\Ib$ in the operator $\Tb=\partial_t\Ib$ is to remind us that it
operates componentwise. Due to the fact that the vector field
$\vec{\nabla}\chi_j$ is \emph{real-valued}, we have that
$\langle\vec{\nabla}\chi_j\cdot\vec{\nabla}\chi_k\rangle=\langle\vec{\nabla}\chi_k\cdot\vec{\nabla}\chi_j\rangle$. From equation
\eqref{eq:Eff_Diffusivity} this clearly implies that the tensor   
$\bkappa^*$ is symmetric, $\kappa^*_{jk}=\kappa^*_{kj}$. Moreover, equation
\eqref{eq:Eff_Diffusivity} demonstrates that the effective transport
of the tracer $\phi$ in the principle directions $\vec{e}_k$, $k=1,\ldots,d$,
is always \emph{enhanced} by the presence of an incompressible velocity
field, $\Kc^*_{kk}=\kappa^*_{kk}\geq\varepsilon$. Similarly, the skew-symmetry of the
operator $\Sb:\mathbb{R}^d\mapsto\mathbb{R}^d$ (see Section
\ref{sec:Symmetries_Commute}) implies that $\balpha^*$ is also
skew-symmetric,
$\alpha^*_{jk}=\langle\Sb\vec{\nabla}\chi_j\cdot\vec{\nabla}\chi_k\rangle=-\langle\Sb\vec{\nabla}\chi_k\cdot\vec{\nabla}\chi_j\rangle=-\alpha^*_{kj}\,$. In
particular,    
%
\begin{align}\label{eq:Sb_Skew}
  \alpha^*_{kk}=\langle\Sb\vec{\nabla}\chi_k\cdot\vec{\nabla}\chi_k\rangle=-\langle\Sb\vec{\nabla}\chi_k\cdot\vec{\nabla}\chi_k\rangle=0.  
\end{align}
%
  




In Section \ref{sec:ACM} we discuss the properties of $\Sb$ in more
detial, and recast equations \eqref{eq:Cell_Problem}
and \eqref{eq:Eff_Diffusivity} into a form which parallels the ACM for
characterizing effective transport in composite media
\cite{Golden:CMP-473}. In particular, we provide an abstract Hilbert
space formulation of the effective parameter problem in Section
\ref{sec:Hilbert_Space}, which yields a resolvent representation of the
vector field $\vec{\nabla}\chi_j$ involving an anti-symmetric (normal)
integro-differential operator $\Ab$ which is related to $\Sb$. In Section 
\ref{sec:Integral_Rep}, we employ this mathematical framework to
provide integral representations for $\bkappa^*$ and $\balpha^*$ in
terms of a \emph{spectral measure} associated with $\Ab$.   







\subsection{The ACM for advection enhanced diffusion by periodic
  flows} \label{sec:ACM}   
%
The ACM for representing transport in composites gives a Hilbert space
formulation of the effective parameter problem and provides an
integral representation for the effective transport coefficients of
composite media, involving a \emph{spectral measure} of a self-adjoint
operator which depends only on the composite geometry
\cite{Golden:CMP-473,Murphy:JMP:063506,MILTON:2002:TC}. Here 
we establish a correspondence between this effective parameter problem
and that for enhanced diffusive transport by advective velocity
fields. In Section \ref{sec:Hilbert_Space}, we formulate the Hilbert
space framework associated with advective diffusion, and employ it to
obtain a resolvent representation of the vector field $\vec{\nabla}\chi_k$ in
\eqref{eq:Cell_Problem}. In Section \ref{sec:Integral_Rep} we utalize
this mathematical framework to obtain integral representations for 
$\bkappa^*$ and $\balpha^*$ involving a spectral measure which depends
only on the fluid velocity field $\vec{u}$.    




Toward this goal, we rewrite the first formula in equation
\eqref{eq:Cell_Problem} in a more suggestive, divergence
form. Under the commutability condition
$\vec{\nabla}\Delta^{-1}\partial_t=\Delta^{-1}\partial_t\vec{\nabla}$ (see Section
\ref{sec:Symmetries_Commute}), we rewrite \cite{Fannjiang:SIAM_JAM:333}
$\partial_t\chi_k=\Delta\Delta^{-1}\partial_t\chi_k=\vec{\nabla}\cdot(\Delta^{-1}\Tb)\vec{\nabla}\chi_k$, where
$\Tb=\partial_t\Ib$, and we define the vector field
$\vec{E}_k=\vec{\nabla}\chi_k+\vec{e}_k$ and the operator
$\bsig=\bkappa-(\Delta^{-1})\Tb=\varepsilon\Ib+\Sb$.  In the case of steady fluid velocity
fields, we have $\bsig=\bkappa=\varepsilon\Ib+\Hb$. With these definitions,
equation \eqref{eq:Cell_Problem} may be written as
$\vec{\nabla}\cdot\bsig\vec{E}_k=0$, $\langle\vec{E}_k\rangle=\vec{e}_k$, which is
equivalent to    
%
\begin{align}\label{eq:Maxwells_Equations}    
  \vec{\nabla}\cdot\vec{J}_k=0, \quad
  \vec{\nabla}\times\vec{E}_k=0, \quad
  \vec{J}_k=\bsig\vec{E}_k,\quad
  \langle\vec{E}_k\rangle=\vec{e}_k,\qquad
  \bsig%=\bkappa-(\Delta^{-1})\Tb.
       %=\varepsilon\Ib+\Hb-(\Delta^{-1})\Tb.
       =\varepsilon\Ib+\Sb
\end{align}
%




The formulas in \eqref{eq:Maxwells_Equations} are precisely the
electrostatic version of Maxwell's equations for a conductive medium
\cite{Golden:CMP-473}, where $\vec{E}_k$ and $\vec{J}_k$ are the local
electric field and current density, respectively, and $\bsig$ is the
local conductivity tensor of the medium. In the ACM for composites,
the effective conductivity tensor $\bsig^*$ is defined as
% 
\begin{align}\label{eq:sigma*}
  \langle\vec{J}_k\rangle=\bsig^*\langle\vec{E}_k\rangle.
\end{align}
%
The linear constitutive relation $\vec{J}_k=\bsig\vec{E}_k$ in
\eqref{eq:Maxwells_Equations} relates the local intensity and flux,
while that in \eqref{eq:sigma*} relates the mean intensity and
flux. Due to the skew-symmetry of $\Sb$, the intensity-flux
relationship in \eqref{eq:Maxwells_Equations} is similar to that of a
Hall medium \cite{Isichenko:JNS:1991:375}. We demonstrate in Section
\ref{sec:Hilbert_Space} that the definition of the effective parameter
in \eqref{eq:sigma*} is equivalent to that in
\eqref{eq:Eff_Diffusivity}. This places these different effective
parameter problems on common mathematical footing, and follows by
adapting the Hilbert space formulation of the ACM to treat the 
effective transport properties of advective diffusion.        





\subsubsection{Hilbert space and resolvent
  representation} \label{sec:Hilbert_Space}   
%
In this section we explore the mathematical properties of the
skew-symmetric operator $\Sb$ introduced in equation
\eqref{eq:Eff_Diffusivity}, and provide an abstract Hilbert space
formulation of the effective parameter problem for advective
diffusion. We utalize this mathematical framework and equation
\eqref{eq:Cell_Problem}  to obtain a resolvent representation of the
vector field $\vec{\nabla}\chi_k$, involving an anti-symmetric operator $\Ab$
which is closely related to $\Sb$. Using the results of this section,
we derive in Section \ref{sec:Integral_Rep} integral representations
for $\bkappa^*$ and $\balpha^*$, involving a \emph{spectral measure}
associated with $\Ab$.       



Consider the Hilbert spaces $L^2_d(\Tc)=\otimes_{i=1}^dL^2(\Tc)$ and
$L^2_d(\Vc)=\otimes_{i=1}^dL^2(\Vc)$ (over the complex field $\mathbb{C}$) of 
square integrable vector valued functions. Now consider the associated
Hilbert spaces $\Hs_{\,\Tc}\subset L^2_d(\Tc)$ and $\Hs_{\,\Vc}\subset L^2_d(\Vc)$
of periodic vector valued functions with temporal periodicity $T$ on
the interval $\Tc=(0,T)$ and spatial periodicities $V_j$, $j=1,\ldots,d$,
on the $d$-dimensional region $\Vc=(0,V_1)\times\cdots\times(0,V_d)$, respectively,
as well as their direct product
$\Hs_{\Tc\Vc}=\Hs_{\,\Tc}\otimes\Hs_{\,\Vc}$,
%which is also a Hilbert space, 
%
\begin{align}\label{eq:Hilbert_Spaces}
  \Hs_{\Tc\Vc}=\Hs_{\,\Tc}\otimes\Hs_{\,\Vc}, \quad
  \Hs_{\,\Tc}=\{ 
     \vec{\xi}\in L^2_d(\Tc)\;|\;
     \vec{\xi}(0)=\vec{\xi}(T) 
                        \}, \quad
  \Hs_{\,\Vc}=\{ 
     \vec{\xi}\in L^2_d(\Vc)\;|\;
     \vec{\xi}(0)=\vec{\xi}(\vec{V}) 
                        \}, 
\end{align}
%
where we have defined $\vec{V}=(V_1,\ldots,V_d)$. Denote by $\langle\cdot,\cdot\rangle$ the
sesquilinear inner-product associated with the Hilbert space
$\Hs_{\Tc\Vc}$, which is defined by
$\langle\vec{\xi},\vec{\zeta}\,\rangle=\langle\vec{\xi}\cdot\vec{\zeta}\,\rangle$  and
$\langle\vec{\xi},\vec{\zeta}\,\rangle=\overline{\langle\vec{\zeta},\vec{\xi}\,\rangle}$, where $\bar{a}$ 
is the complex conjugate of $a\in\mathbb{C}$. By the Helmholtz theorem 
\cite{Denaro:2003:0271,Bhatia:IEE:1077}, the 
Hilbert space $\Hs_{\,\Vc}$ in \eqref{eq:Hilbert_Spaces} can be
decomposed into orthogonal subspaces of curl-free $\Hs_\times$,
divergence-free $\Hs_\bullet$, and constant $\Hs_{\,0}$ vector fields, with
associated orthogonal projectors $\bGamma_\times=\vec{\nabla}(\Delta^{-1})\vec{\nabla}\cdot$, 
$\bGamma_\bullet=-\vec{\nabla}\times(\Delta^{-1})\vec{\nabla}\times$, and  $\bGamma_0=\langle\cdot\rangle$ 
%, respectively,
\cite{Fannjiang:SIAM_JAM:333,MILTON:2002:TC}    
%
\begin{align}\label{eq:Helmholtz}
  &\Hs_{\,\Vc}=\Hs_\times\oplus\Hs_\bullet\oplus\Hs_{\,0},\qquad
  \Ib=\bGamma_\times+\bGamma_\bullet+\bGamma_0, \\  
  \Hs_\times=\{\vec{\xi}\;|\;\vec{\nabla}\times\vec{\xi}&=0 \text{ weakly}\}, \quad
  \Hs_\bullet
      =\{\vec{\xi}\;|\;\vec{\nabla}\cdot\vec{\xi}=0 \text{ weakly}\},   \quad
  \Hs_{\,0}
      =\{\vec{\xi}\;|\;\vec{\xi}=\langle\vec{\xi}\,\rangle\},
     \notag  
\end{align}
%








We are primarily concerned with fluid velocity fields $\vec{u}$ such
that $0<\Kc^*_{kk}<\infty$ for all $0<\varepsilon<\infty$. Consequently, from equation
\eqref{eq:Eff_Diffusivity} we have that the (weakly)
curl-free vector field $\vec{\nabla}\chi_k$ is bounded in the norm $\|\cdot\|$
induced by the $\Hs_{\Tc\Vc}$-inner-product, $\|\vec{\nabla}\chi_k\|<\infty$, so that  
$\vec{\nabla}\chi_k\in\Hs_{\,\Tc}\otimes\Hs_\times\subset\Hs_{\Tc\Vc}$. Defining the (weakly)
divergence-free vector field $\vec{J}_k=\bsig\vec{E}_k$ in
\eqref{eq:Maxwells_Equations} as a member of a subset of
$\Hs_{\Tc\Vc}$ is technically difficult, due to the
\emph{unboundedness} of the operator $\bsig=\bkappa-(\Delta^{-1})\Tb$ on
this space. We now explore the properties of this operator in more
detail. 





Since $\Vc$ is a bounded domain, $(\Delta^{-1})$ is a compact operator
\cite{Stakgold:BVP:2000} and is consequently bounded in the operator
norm $\|\cdot\|$ induced by the $\Hs_{\Tc\Vc}$-inner-product
\cite{Reed-1980,Stone:64,Stakgold:BVP:2000}.  We have already assumed
for the convergence $\phi^\delta\to\bar{\phi}$, as $\delta\to0$, that the flow matrix 
$\Hb(t,\vec{x})$ is periodic on $\Tc\otimes\Vc$. We will also assume that it
is (component-wise) mean-zero and bounded in operator norm, and that its
component-wise time derivative $\Tb\Hb$ is also bounded on $\Hs_{\Tc\Vc}$
%
\begin{align}\label{eq:Bounded_H}
  \langle\Hb\rangle=0, \quad \|\Hb\|<\infty, \quad \|\Tb\Hb\|<\infty.
  %\quad \text{ on } \Hs_{\Tc\Vc}. 
\end{align}
%
This implies that $\bkappa=\varepsilon\Ib+\Hb$ is also bounded for all
$0<\varepsilon<\infty$. Consequently, in the case of a time-independent velocity 
field $\vec{u}$, where $\bsig=\bkappa$, the operator $\bsig$ is
bounded. This and $\|\vec{\nabla}\chi_k\|<\infty$ implies that
$\vec{J}_k\in\Hs_\bullet$. Therefore, in the case of a time-dependent velocity
field, under the assumptions of \eqref{eq:Bounded_H}, the
unboundedness of $\bsig$ on $\Hs_{\Tc\Vc}$  is due to the
unboundedness of $\Tb$ on $\Hs_{\,\Tc}$.   



The unboundedness of $\Tb$ on $\Hs_{\,\Tc}$ can be understood by
considering the orthonormal set of functions
$\{\vec{\psi}_n\}\subset\Hs_{\,\Tc}$ with components $(\vec{\psi}_n)_j$, $j=1,\ldots,d$,
defined by  
%
\begin{align}\label{eq:Orthonormal}
  (\vec{\psi}_n)_j(t,\vec{x})=\beta\sin((n+j)\pi t/T), \quad
  \beta=\sqrt{2/(Td)},
  \qquad
  \langle\vec{\psi}_n\cdot\vec{\psi}_m\rangle=\delta_{nm}, \quad
  n,m\in\mathbb{N}.
\end{align}
%
The components $(\Tb\vec{\psi}_n)_j$, $j=1,\ldots,d$, of the vector
$\Tb\vec{\psi}_n$ and its norm $\|\Tb\vec{\psi}_n\|$ are given by
%$(\partial_t\vec{\psi}_n)_j(t,\vec{x})=b_{nj}\cos((n+j)\pi t/T)$, $b_{nj}=(n+j)a\pi/T$.
%
\begin{align}\label{eq:Orthonormal_Diff}
  (\partial_t\vec{\psi}_n)_j(t,\vec{x})=[\beta(n+j)\pi/T]\cos((n+j)\pi t/T),\quad
  \|\Tb\vec{\psi}_n\|^2%=(T/2)\sum_jb_{nj}^2
               =\frac{1}{d}\sum_j[(n+j)\pi/T]^2.
\end{align}
%
%The $\Hs$-norm of $\partial_t\vec{\psi}_n$ is given by
%$\|\partial_t\vec{\psi}_n\|^2=(T/2)\sum_jb_{nj}^2=\sum_j[(n+j)\pi/T]^2$. 
Therefore, the norm of the members of the set $\{\Tb\vec{\psi}_n\}$
grows arbitrarily large as $n\to\infty$. This clearly demonstrates the
unboundedness of the operator $\Tb$ on $\Hs_{\,\Tc}$.





The above analysis demonstrates that the domain $D(\Tb)$ of the
unbounded operator $\Tb$ is defined only on a proper subset of $\Hs_{\,\Tc}$,
i.e. $D(\Tb)\subset\Hs_{\,\Tc}$. However, $D(\Tb)$ can be defined as a \emph{dense}
subset of $\Hs_{\,\Tc}$ such that $\Tb$ is bounded
\cite{Reed-1980,Stone:64}. Toward this goal, consider the class
$\As_{\Tc}$ of all functions $\xi\in L^2(\Tc)$ such that $\xi(t)$ is
\emph{absolutely continuous} \cite{Royden:1988:RA} on the interval
$\Tc$ and has a derivative $\xi^{\,\prime}(t)$ belonging to $L^2(\Tc)$,
i.e. \cite{Stone:64,Royden:1988:RA}   
%
\begin{align}\label{eq:AC_L2}
  \As_{\Tc}=
     \left\{
       \xi\in L^2(\Tc) \ \Big| \ \xi(t)=c+\int_0^tg(\tau)d\tau,
       \quad  g\in L^2(\Tc)
     \right\},
\end{align}
%
where the constant $c$ and function $g(t)$ are
arbitrary. Now, consider the set $\tilde{\As}_{\Tc}$ of all
functions $\xi\in\As_{\Tc}$ that satisfy the periodic initial condition
$\xi(0)=\xi(T)$, i.e. functions $\xi$ satisfying the properties of 
equation \eqref{eq:AC_L2} with $\int_0^Tg(\tau)d\tau=0$. To illustrate some
important ideas later in this work, we also consider the
set $\hat{\As}_{\Tc}$ of all functions $\xi\in\As_{\Tc}$ that satisfy the
Dirichlet initial condition $\xi(0)=\xi(T)=0$, i.e. functions $\xi$
satisfying the properties of equation \eqref{eq:AC_L2} with $c=0$ and
$\int_0^Tg(\tau)d\tau=0$. More concisely, 
%
\begin{align}
  \tilde{\As}_{\Tc}=\{\xi\in\As_{\Tc} \,|\, \xi(0)=\xi(T)\}, \quad
  \hat{\As}_{\Tc}=\{\xi\in\As_{\Tc} \,|\, \xi(0)=\xi(T)=0\}.
\end{align}
%
These function spaces satisfy
$\hat{\As}_{\Tc}\subset\tilde{\As}_{\Tc}\subset\As_{\Tc}$ and are each everywhere
dense in $L^2(\Tc)$ \cite{Stone:64}. 






It follows that $\Ds_{\Tc}=\otimes_{i=1}^d\tilde{\As}_{\Tc}$ is everywhere
dense in the Hilbert space $\Hs_{\,\Tc}$. As an operator acting on the
function  space $\Ds_{\Tc}\otimes\Hs_{\,\Vc}$, $\bsig=\bkappa-(\Delta^{-1})\Tb$ is
bounded. This, $\vec{\nabla}\chi_k\in\Hs_{\,\Tc}\otimes\Hs_\times$, and equation
\eqref{eq:Cell_Problem} suggests that we consider the curl-free
mean-zero vector field $\vec{\nabla}\chi_k$ as a member of the function space 
$\Hs\subset\Ds_{\Tc}\otimes\Hs_{\,\Vc}$,   
%
\begin{align}\label{eq:Function_Space}
  \Hs=\{\vec{\xi}\in\Ds_{\Tc}\otimes\Hs_\times \,|\, \langle\vec{\xi}\,\rangle=0\},  
\end{align}
%
which will be used extensively in the rest of this manuscript. We
stress that $\Hs$ is \emph{not} a Hilbert space, and is instead a
dense subset of the Hilbert space $\Hs_{\,\Tc}\otimes\Hs_\times$. We will
henceforth assume that $\vec{\nabla}\chi_k\in\Hs$. In the case of a
time-independent velocity field $\vec{u}$ we set $\Ds_{\Tc}=\emptyset$ in 
\eqref{eq:Function_Space}, so that $\vec{\xi}\in\Hs$ implies that
$\vec{\xi}\in\Hs_\times$ with $\langle\vec{\xi}\,\rangle=0$. As $\bsig$ is bounded on $\Hs$
and we have $\vec{\nabla}\chi_k\in\Hs$, the divergence-free vector field
$\vec{J}_k=\bsig\vec{\nabla}\chi_k$ is also bounded $\|\vec{J}_k\|<\infty$, thus
$\vec{J}_k\in\Hs_{\Tc}\otimes\Hs_\bullet$.   








We now employ the Hilbert space framework formulated above to
demonstrate that the definition of the effective parameter in equation 
\eqref{eq:sigma*} is equivalent to that in
\eqref{eq:Eff_Diffusivity}. This puts the effective parameter
problem of enhanced diffusive transport by advective velocity fields
and that of transport in composite media on common mathematical
footing. By the mutual orthogonality of the Hilbert spaces $\Hs_\times$ and
$\Hs_\bullet$ in \eqref{eq:Helmholtz}, $\vec{\nabla}\chi_k\in\Hs$ and
$\vec{J}_k\in\Hs_{\Tc}\otimes\Hs_\bullet$ imply that $\langle\vec{J}_j\cdot\vec{\nabla}\chi_k\rangle=0$ 
for every $j,k=1,\ldots,d$. As $\vec{E}_k=\vec{\nabla}\chi_k+\vec{e}_k$, we therefore
have $\langle\vec{J}_j\cdot\vec{e}_k\rangle=\langle\vec{J}_j\cdot\vec{E}_k\rangle$. This is trivially
satisfied in the case of a time-independent velocity field $\vec{u}$, 
since in this case $\bsig=\bkappa$ is bounded so that
$\vec{J}_j\in\Hs_\bullet$ for $\vec{\nabla}\chi_k\in\Hs$. In either case, equation
\eqref{eq:sigma*} then implies that the components
$\sigma^*_{jk}=\bsig^*\vec{e}_j\cdot\vec{e}_k=\langle\bsig\vec{E}_j\cdot\vec{e}_k\rangle$ of
the effective tensor $\bsig^*$ can be expressed as
$\sigma^*_{jk}=\langle\bsig\vec{E}_j\cdot\vec{E}_k\rangle$, with $\bsig=\varepsilon\Ib+\Sb$ and
$\Sb=\Hb-(\Delta^{-1})\Tb$. Consequently, since
$\Sb\vec{e}_j=\Hb\vec{e}_j$, $\langle\Hb\rangle=0$ (component-wise), and
$\langle\vec{\nabla}\chi_k\rangle=0$, we have from equation \eqref{eq:Eff_Diffusivity}
that  
%
\begin{align}\label{eq:Reduction}
  \sigma^*_{jk}%=\bsig^*\vec{e}_j\cdot\vec{e}_k 
       %=\langle\bsig\vec{E}_j\cdot\vec{e}_k\rangle
       %=\langle\bsig\vec{E}_j\cdot\vec{E}_k\rangle
       =\langle[(\varepsilon\Ib+\Sb)]\vec{E}_j\cdot\vec{E}_k\rangle
       =\varepsilon\langle\vec{E}_j\cdot\vec{E}_k\rangle+\langle\Sb\vec{E}_j\cdot\vec{E}_k\rangle
       =\kappa^*_{jk}+\alpha^*_{jk}
       =\Kc_{jk}^*.
       %=\varepsilon(d\langle\vec{E}_j\cdot\vec{E}_k\rangle+\langle\Sb\vec{E}_j\cdot\vec{E}_k\rangle.
       %=\kappa^*_{kk}.
\end{align}
%
This puts the effective parameter problem for $\Kbc^*$ in
\eqref{eq:Cell_Problem} and \eqref{eq:Eff_Diffusivity}, and that for
$\bsig^*$ of the ACM \cite{Golden:CMP-473} in
\eqref{eq:Maxwells_Equations} and \eqref{eq:sigma*} on common
mathematical footing, for both cases of time-independent and
time-dependent velocity fields $\vec{u}$.  




We now derive the following resolvent representation for $\vec{\nabla}\chi_k$,
involving the orthogonal projection $\bGamma_\times=\vec{\nabla}\Delta^{-1}\vec{\nabla}\cdot$
onto curl-free fields in \eqref{eq:Helmholtz}, 
% 
\begin{align}\label{eq:Resolvent_Rep}
  \vec{\nabla}\chi_k=(\varepsilon\Ib+\Ab)^{-1}\vec{g}_k
           =\I(-\I\varepsilon\Ib-\Mb)^{-1}\vec{g}_k, \quad
  \Ab=\bGamma\Sb\bGamma, \quad
  \Mb=\I\Ab, \quad
  \vec{g}_k=-\bGamma\Hb\vec{e}_k,
\end{align}
%
where $\I=\sqrt{-1}\,$ and we have defined $\bGamma=\bGamma_\times$ for
notational simplicity. Equation \eqref{eq:Resolvent_Rep} follows from
applying the integro-differential operator $\vec{\nabla}\Delta^{-1}$ to
$\vec{\nabla}\cdot\bsig\vec{E}_k=0$ in equation \eqref{eq:Maxwells_Equations},
with $\vec{E}_k=\vec{\nabla}\chi_k+\vec{e}_k$ and $\bsig=\varepsilon\Ib+\Sb$, yielding  
%
\begin{align}\label{eq:Pre_Resolvent}
  \bGamma(\varepsilon\Ib+\Sb)\vec{\nabla}\chi_k=-\bGamma\Hb\vec{e}_k,
\end{align}
%
since $\bGamma\vec{e}_k=0$ and $\Sb\vec{e}_k=\Hb\vec{e}_k$.
The equivalence of equations \eqref{eq:Resolvent_Rep} and
\eqref{eq:Pre_Resolvent} can be seen by noting that
$\vec{\nabla}\chi_k\in\Hs$ implies $\bGamma\vec{\nabla}\chi_k=\vec{\nabla}\chi_k$,
and then writing $1=-\I^{\,2}$. We stress that, even though the
imaginary unit $\I$ was introduced in \eqref{eq:Resolvent_Rep}, the
representation of $\vec{\nabla}\chi_k$ in this equation is
\emph{real-valued}. 



In Section \ref{sec:Symmetries_Commute} we show that $\Ab$ acts as an
anti-symmetric linear operator on the Hilbert space $\Hs_{\Tc\Vc}$, 
and is therefore an example of a \emph{normal} operator
\cite{Stone:64}. Consequently, due to the sesquilinearity of the
$\Hs_{\Tc\Vc}$-inner-product, $\Mb=\I\Ab$ is a \emph{symmetric}
operator \cite{Reed-1980,Stone:64}. Moreover, on the function space 
$\Hs$, $\Ab$ is a \emph{maximal} normal operator and
$\Mb$ is consequently \emph{self-adjoint} \cite{Stone:64}. In Section
\ref{sec:Integral_Rep} we examine these properties of $\Ab$ and $\Mb$ 
in more detial and demonstrate how equation \eqref{eq:Resolvent_Rep}
and the spectral theory of such operators leads to an integral
representation for $\Kbc^*$.     








\subsubsection{Integral representation of the effective diffusivity
  for steady and dynamic flows}\label{sec:Integral_Rep}
%
In this section, we employ the Hilbert space formulation of the
effective parameter problem discussed in Section
\ref{sec:Hilbert_Space}, to provide an integral representation for
$\Kbc^*$ involving a \emph{spectral measure} associated with the
(maximal) normal operator $\Ab=\bGamma\Sb\bGamma$ on
$\Hs$, or equivalently the self-adjoint operator
$\Mb=\I\Ab$. This integral representation follows from the spectral
theorem for such operators \cite{Reed-1980,Stone:64} and the resolvent 
formula for $\vec{\nabla}\chi_k$ in equation \eqref{eq:Resolvent_Rep}. There
are significant differences in the theory between the case of steady 
flows, where $\Sb=\Hb$ is a \emph{bounded} linear operator, and the
case of dynamic flows, where $\Sb=\Hb-(\Delta^{-1})\Tb$ is 
\emph{unbounded}, as discussed in Section \ref{sec:Hilbert_Space}. It
is therefore natural to start our discussion with a more detailed look
into this distinction, in the present context.      





Since $\bGamma$ is an orthogonal projector from $\Hs_{\,\Vc}$ to
$\Hs_\times$, it is bounded by unity in operator norm $\|\bGamma\|\leq1$ on
$\Hs_{\,\Vc}$ and $\|\bGamma\|=1$ on $\Hs_\times$. Therefore, in the case of
steady flows, the operator $\Ab=\bGamma\Hb\bGamma$ is bounded on
$\Hs_{\,\Vc}$. Since $\|\Mb\|=\|\I\Ab\|=\|\Ab\|$ the domain of these
operators are identical, $D(\Mb)=D(\Ab)$, and for simplicity we focus  
on the operator $\Mb$. 



However, in the
case of dynamic flows, the 
operator $\Ab=\bGamma\Sb\bGamma$ is unbounded on $\Hs_{\,\Vc}$. The
(Hilbert space) adjoint $\Ab^*$ of the operator $\Ab$ is defined by
$\langle\Ab\vec{\xi},\vec{\zeta}\,\rangle=\langle\vec{\xi},\Ab^*\vec{\zeta}\,\rangle$, for all  

In
Section \ref{sec:Symmetries_Commute} we show that $\Ab$ acts as an
anti-symmetric operator on the Hilbert space $\Hs_{\Tc\Vc}$,
i.e. $\langle\Ab\vec{\xi},\vec{\zeta}\,\rangle=-\langle\vec{\xi},\Ab\vec{\zeta}\,\rangle$ for all
$\vec{\xi},\vec{\zeta}\in\Hs_{\Tc\Vc}$ such that
$|\langle\Ab\vec{\xi},\vec{\zeta}\,\rangle|,|\langle\vec{\xi},\Ab\vec{\zeta}\,\rangle|<\infty$. Therefore, by
the the sesquilinearity of the inner-product, this implies that
$\Mb=\I\Ab$ is symmetric
$\langle\Mb\vec{\xi},\vec{\zeta}\,\rangle=\langle\vec{\xi},\Mb\vec{\zeta}\,\rangle$ for all such
$\vec{\xi},\vec{\zeta}\in\Hs_{\Tc\Vc}$. In the case of steady flows, these
symmetry properties hold for all $\vec{\xi},\vec{\zeta}\in\Hs_{\Tc\Vc}$, since 
$\Ab$ is bounded. Consequently, $\Ab$ and $\Mb$ act as an
anti-symmetric and symmetric operator on the entire space
$\Hs_{\,\Vc}$.


  As discussed in Section
\ref{sec:Hilbert_Space}, the operator $\Sb$, hence $\Ab$ is defined
only on a proper subset of $\Hs_{\,\Vc}$. However, $\Ab$ can be
extended to a \emph{closed} linear operator
\cite{Reed-1980,Stone:64} on a dense subset of $\Hs_{\,\Vc}$.



projector from $\Hs_{\,\Vc}$ to $\Hs_\times$ and therefore has unit
operator norm $\|\bGamma\|=1$ on $\Hs_\times$ 
\cite{Reed-1980,Stone:64}. By equation \eqref{eq:Bounded_H} $\Hb$ is
also bounded in operator norm on $\Hs_\times$.  Therefore, in this
time-independent case, $\Ab$ is a \emph{bounded} linear operator on
$\Hs_\times$ with operator norm $\|\Ab\|\leq\|\bGamma\|\|\Hb\|\|\bGamma\|=\|\Hb\|<\infty$.
To illustrate the
differences in the theory between the cases of a time-dependent and
time-independent velocity field $\vec{u}$, we first discuss the
time-independent case





In the case of a
time-independent velocity field $\vec{u}$, $\Sb=\Hb$, so that and the
operator $\Ab$ in \eqref{eq:Resolvent_Rep} is given by
$\Ab=\bGamma\Hb\bGamma$ and by \eqref{} is bounded.







 Since $\Ab$ is
bounded, its (Hilbert space) adjoint $\Ab^*$ is also bounded with
$\|\Mb\|=\|\Mb^*\|$ \cite{Reed-1980} and they consequently have common
domains,  
%
\begin{align}\label{eq:Domain_M}
  D(\Mb)=D(\Mb^*),
\end{align}
%
which are the entire space, $D(\Mb)=D(\Mb^*)=\Hs_{\Tc\Vc}$. 
%
\begin{align}\label{eq:Symmetric_M}
  \langle\Mb\vec{\xi}\cdot\vec{\zeta}\,\rangle=\langle\vec{\xi}\cdot\Mb\vec{\zeta}\,\rangle,
  \, \text{ for all } \; \vec{\xi},\vec{\zeta}\in D(\Mb).
\end{align}
%
By definition \cite{Reed-1980}, the two properties \eqref{eq:Domain_M}
and \eqref{eq:Symmetric_M} together imply that the operator $\Mb$ is
\emph{self-adjoint}, i.e. $\Mb=\Mb^*$. Conversely, the
Hellinger--Toeplitz theorem \cite{Reed-1980} states that if the
operator $\Mb$ satisfies equation \eqref{eq:Symmetric_M} 
for \emph{every} $\vec{\xi},\vec{\zeta}\in\Hs$, then $\Mb$ is bounded. This
suggests that when $\Mb$ is unbounded on $\Hs$, it is defined as a 
self-adjoint operator only on a proper (dense) subset of $\Hs$. 




In the case of a time-dependent velocity field $\vec{u}$, in Section
\ref{sec:Hilbert_Space} we showed that the operator $\Tb$, hence
$\Sb=\Hb-\Delta^{-1}\partial_t\Ib$ is unbounded on $\Hs_{\Tc\Vc}$.  Hence $\Mb$ is unbounded on $\Hs$. 




 
For such a domain, simple integration by parts and the sesquilinearity 
of the inner-product $\langle\cdot,\cdot\rangle$ shows that, as an operator on $D(\Tb)$,
$\Tb$ is symmetric, i.e. it satisfies \eqref{eq:Symmetric_M} with
$\Tb$ in place of $\Mb$. Although, in general \cite{Reed-1980}, the
domain $D(\Tb^*)$ of its adjoint $\Tb^*$ does not satisfy the property
displayed in equation \eqref{eq:Domain_M} and in such circumstances,
$\Tb$ is \emph{not} self-adjoint on $D(\Tb)$. Only for self-adjoint
linear operators does the spectral theorem hold \cite{Reed-1980},
which provides the existence of the promised integral representation
for $\bkappa^*$, involving a spectral measure of $\Mb$. It is therefore
necessary that we find a domain $D(\Mb)$ for which $\Mb$ is a
self-adjoint operator.   



Moreover, on this domain, $\Tb$ can be
extended to a \emph{closed} linear operator
\cite{Reed-1980,Stone:64}.
Toward this goal, and to illustrate these ideas, we consider the
operator $\I\partial_t$ with three different domains, which are everywhere
dense in $L^2(\Tc)$ \cite{Stone:64}. First,  Let the
operators $\hat{\Bb}$, $\tilde{\Bb}$, and $\Bb$ be identified as
$\I\partial_t$ with domains $\hat{\As}_{\Tc}$, $\tilde{\As}_{\Tc}$, and
$\As_{\Tc}$, respectively. Then \cite{Stone:64}, $\Bb$ is a closed,
linear, symmetric, operator with adjoint $\Bb^*\equiv\hat{\Bb}$, and the
operator $\tilde{\Bb}$ is a \emph{self-adjoint} extension of $\Bb$. 



Since the set $\tilde{\As}_{\Tc}$ is everywhere dense in $L^2(\Tc)$
and consists of periodic functions $\xi(t)$ satisfying $\xi(0)=\xi(T)$,
the set $\As_{\Tc}=\otimes_{i=1}^d\tilde{\As}_{\Tc}$ is everywhere dense in
$\Hs_{\,\Tc}$. Moreover, since $\tilde{\Bb}=\I\partial_t$ with domain
$\tilde{\As}_{\Tc}$ is self-adjoint, the operator $\Tb=\I\partial_t\Ib$ with
domain $D(\Tb)=\As_{\Tc}$ is self-adjoint. This is seen by noting
that, for all $\vec{\xi},\vec{\zeta}\in\As_{\Tc}$,
$\Tb\vec{\xi}=(\tilde{\Bb}\xi_1,\ldots,\tilde{\Bb}\xi_d)$, for example, and  
%
\begin{align}
  \langle\Tb\vec{\xi}\cdot\vec{\zeta}\rangle=\sum_j\langle\tilde{\Bb}\xi_j,\zeta_j\rangle_2
                    =\sum_j\langle\xi_j,\tilde{\Bb}\zeta_j\rangle_2
                    =\langle\vec{\xi}\cdot\Tb\vec{\zeta}\rangle,
\end{align}
%
where $\langle\cdot,\cdot\rangle_2$ denotes the $L^2(\Tc)$ inner-product.
It follows that the operator $\Mb$ defined in equation
\eqref{eq:Resolvent_Rep} is self-adjoint on the space $\Hs_{\,t}\subset\Hs$,
which is everywhere dense in $\Hs$, given by
%
\begin{align}\label{eq:Ht_2}
  \Hs_{\,t}=\{\vec{\xi}\in\As_{\Tc}\otimes\Hs_\times: \langle\vec{\xi}\,\rangle=0\}.
\end{align}
%



We are now ready to provide an integral representation
for the effective diffusivity tensor $\bkappa^*$. This follows from the
spectral theorem of operational calculus in Hilbert space
\cite{Reed-1980,Stone:64}, which states that there is a
one-to-one correspondence between the self-adjoint operator $\Mb$ and
a family $\{\Qb(\lambda)\}$, $-\infty<\lambda<\infty$, of \emph{projection} operators - the
resolution of the identity - which satisfies $\lim_{\lambda\to-\infty}\Qb(\lambda)=0$ and
$\lim_{\lambda\to\infty}\Qb(\lambda)=\Ib$. Moreover, let $\vec{\xi},\vec{\zeta}\in\Hs_{\,t}$ and
consider the following functions $\mu_{\xi\zeta}(\lambda)=\langle\Qb(\lambda)\vec{\xi}\cdot\vec{\zeta}\,\rangle$
and $\mu_{\xi\xi}(\lambda)=\langle\Qb(\lambda)\vec{\xi}\cdot\vec{\xi}\,\rangle=\|\Qb(\lambda)\vec{\xi}\,\|^2$ of 
\emph{bounded variation} with associated Radon
measures $\mu_{\xi\zeta}(d\lambda)$ and $\mu_{\xi\xi}(d\lambda)$ \cite{Stone:64}
%
\begin{align}\label{eq:Bounded_Variation}
  \mu_{\xi\zeta}(d\lambda)=\langle\Qb(d\lambda)\vec{\xi}\cdot\vec{\zeta}\,\rangle, \quad
  \mu_{\xi\xi}(d\lambda)=\|\Qb(d\lambda)\vec{\xi}\,\|^2.
\end{align}
%
Let $F(\lambda)$ be an arbitrary complex-valued function and denote by
$\mathscr{D}(F)$ the set of all $\vec{\xi}\in\Hs_{\,t}$ such that
$F\in L^2(\mu_{\xi\xi})$, the class of square $\mu_{\xi\xi}$-integrable
functions. Then $\mathscr{D}(F)$ is a linear manifold and there exists
a linear transformation $\Mb(F)$ with domain $\mathscr{D}(F)$ defined
in terms of the Radon-Stieltjes integrals \cite{Stone:64}
%
\begin{align}\label{eq:Spectral_Theorem}
  \langle\Mb(F)\vec{\xi}\cdot\vec{\zeta}\,\rangle=\int_{-\infty}^\infty F(\lambda)\,\mu_{\xi\zeta}(d\lambda), \qquad
  &\forall \, \vec{\xi}\in\mathscr{D}(F), \ \vec{\zeta}\in\Hs_{\,t}
  \\
  \langle\Mb(F)\vec{\xi}\cdot\Mb(G)\vec{\zeta}\,\rangle=\int_{-\infty}^\infty F(\lambda)\bar{G}(\lambda)\,\mu_{\xi\zeta }(d\lambda), \quad
  &\forall \, \vec{\xi}\in\mathscr{D}(F), \ \vec{\zeta}\in\mathscr{D}(G),
  \notag
\end{align}
%
where $\bar{G}$ denotes the complex conjugate of the function $G$, and
the operator $\Mb(G)$ and set $\mathscr{D}(G)$ involving $G$ and
$\vec{\zeta}$ are defined analogously to that for $F$ and $\vec{\xi}$. A
Radon-Stieltjes integral representation of the functional
$\|\Mb(F)\vec{\xi}\,\|^2$ follows from the second equation in
\eqref{eq:Spectral_Theorem} with $G=F$ and $\vec{\xi}=\vec{\zeta}$, and
involves measure $\mu_{\xi\xi}(d\lambda)$ in \eqref{eq:Bounded_Variation}
\cite{Stone:64}.    


An integral representation for the components $\kappa^*_{jk}$, $j,k=1,\ldots,d$,
of $\bkappa^*$ follows from equations \eqref{eq:Eff_Diffusivity} and
\eqref{eq:Resolvent_Rep}, and the second formula in
\eqref{eq:Spectral_Theorem} with 
%
\begin{align}\label{eq:F_G}
  F(\lambda)=G(\lambda)=\I(-\I\varepsilon-\lambda)^{-1}, \quad
  \vec{\xi}=\vec{g}_j, \quad
  \vec{\zeta}=\vec{g}_k, \quad
  \vec{g}_j=\bGamma\Hb\vec{e}_j,
  \quad  j,k=1,\ldots,d.
\end{align}
%
More specifically, the orthogonality of the projection operators
$\bGamma_\times=\bGamma$ and $\bGamma_0$ in \eqref{eq:Helmholtz} implies
that the vector field
$\vec{g}_j(t,\vec{x})=\bGamma\Hb(t,\vec{x})\vec{e}_j$ is curl-free and
mean-zero. Moreover, by equation \eqref{eq:Bounded_H}, $\vec{g}_j$
and $\partial_t\vec{g}_j$ are bounded in $\Hs$-norm, with $\|\vec{g}_j\|\leq\|\Hb\|$
and $\|\partial_t\vec{g}_j\|\leq\|\partial_t\Hb\|$, so that
$\vec{g}_j\in\Hs_{\,t}$ for all $j=1,\ldots,d$. Also, since
$\I\varepsilon\not\in\mathbb{R}$ for all $\varepsilon>0$ and the measure $\mu_{\xi\xi}(d\lambda)$
is of bounded mass \cite{Stone:64} 
%$\mu^0_{\xi\xi}=\int\mu_{\xi\xi}(d\lambda)=\|\vec{\xi}\,\|^2\leq\|\Hb\|^2<\infty$,
%
\begin{align}\label{eq:Mass}
  \mu^0_{\xi\xi}=\int_{-\infty}^\infty\mu_{\xi\xi}(d\lambda)
        =\int_{-\infty}^\infty\langle\Qb(d\lambda)\vec{\xi}\cdot\vec{\xi}\,\rangle   
       =\|\vec{\xi}\,\|^2
       =\langle\Hb^T\bGamma\Hb\vec{e}_k\cdot\vec{e}_k\rangle
       \leq\|\Hb\|^2<\infty,
\end{align}
%
we have that $F(\lambda)$ in \eqref{eq:F_G} satisfies $F\in L^2(\mu_{\xi\xi})$, hence 
$\vec{g}_j\in\mathscr{D}(F)$, for all $\varepsilon>0$ and $j=1,\ldots,d$. Therefore, the 
following Radon-Stieltjes integral representation for the components
$\kappa^*_{jk}$ of $\bkappa^*$, involving the components $\mu_{jk}(d\lambda)$ of
the matrix-valued measure $\bmu(d\lambda)$, holds for all $\varepsilon>0$
%
\begin{align}\label{eq:Integral_Rep_kappa*}
  \kappa^*_{jk}=\varepsilon\left(\delta_{jk}+\int_{-\infty}^\infty\frac{\mu_{jk}(d\lambda)}{\varepsilon^2+\lambda^2}\right), \quad
         \mu_{jk}(d\lambda)=\langle\Qb(d\lambda)\vec{g}_j\cdot\vec{g}_k\rangle,
  \quad  j,k=1,\ldots,d,
\end{align}
%
where we have used the notation $\mu_{jk}(d\lambda)=\mu_{\xi\zeta}(d\lambda)$ for
$\vec{\xi}=\vec{g}_j$ and $\vec{\zeta}=\vec{g}_k$.


We conclude this section with a few remarks regarding the integral
representation in \eqref{eq:Integral_Rep_kappa*}. The Radon measure
$\mu_{jk}(d\lambda)$ is a \emph{spectral measure} associated with the
self-adjoint linear operator $\Mb$ in the $(\vec{g}_j,\vec{g}_k)$
state \cite{Reed-1980}. Since $\Qb(\lambda)$ is a projection operator, the
diagonal components of $\bmu(d\lambda)$ are \emph{positive} measures, 
$\mu_{kk}(d\lambda)=\|\Qb(d\lambda)\vec{g}_k\|^2$. In the case of a time-independent
velocity field $\vec{u}$, where $\Mb=\bGamma\Hb\bGamma$, the range of
integration in equation \eqref{eq:Integral_Rep_kappa*} is given by
$-\|\Mb\|\leq\lambda\leq\|\Mb\|$, with $\|\Mb\|\leq\|\Hb\|<\infty$, by \eqref{eq:Bounded_H}.  A
key feature of the integral representation for $\bkappa^*$ in
\eqref{eq:Integral_Rep_kappa*} is that parameter information in $\varepsilon$ is
\emph{separated} from the geometry and dynamics of the velocity field,
which is encapsulated in the measure $\bmu$. In Section 
\ref{sec:Assymptotics} we employ the integral representation for
$\bkappa^*$ in \eqref{eq:Integral_Rep_kappa*} to obtain asymptotic
behavior of $\kappa^*_{jk}$ as $\varepsilon\to0$. 









\section{Asymptotic analysis of effective
  diffusivity} \label{sec:Assymptotics} 
In two dimensions, $d=2$, the matrix $\Hb$ is determined by a stream
function $H(t,\vec{x})$ 
%
\begin{align}\label{eq:u_H}  
  \Hb=\left[
  \begin{array}{cc}
    0  & H\\
    -H & 0
  \end{array}
  \right],
  \qquad
  \vec{u}=[\partial_{x_1}H, \ \partial_{x_2}H].
\end{align}
%


\section{Numerical Results}\label{sec:Num_Results}
%
Since we are focusing on flows which are periodic on the spatial
region $\Vc$, it is convenient to consider the Fourier
representation of such a vector field $\vec{\xi}(t,\vec{x})$, 
%
\begin{align}
  \vec{\xi}(t,\vec{x})
    =\sum_{\vec{k}\in\mathbb{Z}^d}
       \hat{Y}(t,\vec{k})\e^{\I\vec{k}\cdot\vec{x}},
  \qquad
  \hat{Y}(t,\vec{k})
    =\frac{1}{(2\pi)^d}\int_{\Vc}
       \hat{Y}(t,\vec{x})\e^{-\I\vec{k}\cdot\vec{x}} \d\vec{x}.
\end{align}
%
%where $|\Vc|=(2\pi)^d$ is the Lebesgue measure of the region
%$\Vc$.
The associated action of the above projection operators on a function
$\vec{\xi}\in\Hs$ is given by \cite{Fannjiang:SIAM_JAM:333} 
%
\begin{align}\label{eq:Projections}
  \bGamma_0\vec{\xi}(t,\vec{x})
    =\langle\vec{\xi}(t,\vec{x})\rangle_x
    =\hat{Y}(t,0),
    \qquad
  \bGamma_\times\vec{\xi}(t,\vec{x})
    =\sum_{\vec{k}\neq0}
       \frac{\vec{k}(\vec{k}\cdot\hat{Y}(t,\vec{k}))}{|\vec{k}|^2}
       \e^{\I\vec{k}\cdot\vec{x}},
    \\
  \bGamma_\bullet\vec{\xi}(t,\vec{x})
    =\sum_{\vec{k}\neq0}
       \frac{\vec{k}\times(\vec{k}\times\hat{Y}(t,\vec{k}))}{|\vec{k}|^2}
       \e^{\I\vec{k}\cdot\vec{x}}
    =\sum_{\vec{k}\neq0}\left(I-
       \frac{\vec{k}\vec{k}\cdot}{|\vec{k}|^2}\right) 
       \hat{Y}(t,\vec{k})  \e^{\I\vec{k}\cdot\vec{x}},
       \notag
\end{align}
%
where $\langle\cdot\rangle_x$ denotes spatial averaging over $\Vc$. From
equation \eqref{eq:Projections} it is clear that
$\bGamma_\times+\bGamma_\bullet+\bGamma_0=\Ib$. 

%\newpage

% redefine the command that creates the equation no.
  \setcounter{equation}{1}  % reset equation counter
  \setcounter{section}{0}  % reset section counter
  \renewcommand{\theequation}{A-\arabic{equation}} 
\renewcommand{\thesection}{A-\arabic{section}}
%
\section{Appendix} 
\label{sec:Appendix}
%
\subsection{The flow matrix $\Hb$}\label{eq:flow_matrix}
%
\subsection{Multiple scale method}\label{sec:Multiscal_Method}
%
In this section we provide the details of the multiple scale method
\cite{McLaughlin:SIAM_JAM:780,Papanicolaou:1981:36:8,Papanicolaou:RF-835,Bensoussan:Book:1978}
which leads to equations
\eqref{eq:phi_bar}--\eqref{eq:Eff_Diffusivity}. We assume that
equation \eqref{eq:ADE} has already been non-dimensionalized so that
$\kappa_0\mapsto\varepsilon$ and $\vec{v}\mapsto\vec{u}$. The key assumption of the method is
that the initial density $\phi_0$ in \eqref{eq:ADE} is slowly
varying relative to the velocity field $\vec{u}$, which introduces a
small parameter $\delta\ll1$ such that  
% 
\begin{align}\label{eq:IC}
  \phi(0,\vec{x})=\phi_0(\delta\vec{x}).
\end{align}
%





The variable changes $\vec{x}\mapsto\vec{y}=\vec{x}/\delta$ and
$t\mapsto\tau=t/\delta^{\,2}$, along with equations \eqref{eq:incompressible} and
\eqref{eq:IC}, transforms equation \eqref{eq:ADE} into
\cite{McLaughlin:SIAM_JAM:780}    
%
\begin{align}\label{eq:ADE_delta}
  \partial_t\phi^\delta(t,\vec{x})=\varepsilon\Delta\phi^\delta(t,\vec{x})
                 +\delta^{\,-1}\,\vec{u}(\tau,\vec{y})\cdot\vec{\nabla}\phi^\delta(t,\vec{x}), \quad
      \phi^\delta(0,\vec{x})=\phi_0(\vec{x}).
\end{align}
%
We now expand $\phi^\delta$ in powers of $\delta$ \cite{McLaughlin:SIAM_JAM:780} 
%
\begin{align}\label{eq:Expand}
  \phi^\delta(t,\vec{x})=\bar{\phi}(t,\vec{x})
                 +\delta\phi^{(1)}(t,\vec{x},\tau,\vec{y})
                 +\delta^{\,2}\phi^{(2)}(t,\vec{x},\tau,\vec{y})+\cdots.
\end{align}
%
Writing
%
\begin{align*}
  \partial_t\phi^{(i)}=[\partial_t+\delta^{-2}\partial_\tau]\phi^{(i)}, \quad
  \vec{\nabla}\phi^{(i)}=[\vec{\nabla}_x+\delta^{-1}\vec{\nabla}_y]\phi^{(i)}, \quad
  \Delta\phi^{(i)}=[\Delta_x+2\delta^{-1}\vec{\nabla}_x\cdot\vec{\nabla}_y+\delta^{-2}\Delta_y]\phi^{(i)},
\end{align*}
%
for the functions $\phi^{(i)}$, $i=1,2,\ldots$, of the fast $(\tau,\vec{y})$ and
slow $(t,\vec{x})$ variables, we find that
%
\begin{align}
  &\partial_t\phi^\delta=\delta^{-2}[\partial_\tau\bar{\phi}]
      +\delta^{-1}[\partial_\tau\phi^{(1)}]
      +\delta^0[\partial_t\bar{\phi}+\partial_\tau\phi^{(2)}]+O(\delta),
      \\
  &\vec{\nabla}\phi^\delta=\delta^{-2}[0]
            +\delta^{-1}[\vec{\nabla}_y\bar{\phi}]
            +\delta^0[\vec{\nabla}_x\bar{\phi}+\vec{\nabla}_y\phi^{(1)}]
            +\delta^1[\vec{\nabla}_x\phi^{(1)}+\vec{\nabla}_y\phi^{(2)}]+O(\delta^2),
            \notag\\
  &\Delta\phi^\delta=\delta^{-2}[\Delta_y\bar{\phi}]
      +\delta^{-1}[2\vec{\nabla}_x\cdot\vec{\nabla}_y\bar{\phi}+\Delta_y\phi^{(1)}]
      +\delta^0[\Delta_x\bar{\phi}+2\vec{\nabla}_x\cdot\vec{\nabla}_y\phi^{(1)}+\Delta_y\phi^{(2)}]+O(\delta).
      \notag
\end{align}
%
Inserting this into equation \eqref{eq:ADE_delta} and setting the
coefficients associated with the various powers of $\delta$ to zero,
yields a sequence of problems.






Due to the dependence of $\bar{\phi}(t,\vec{x})$ on only the
slow variables, the coefficients of $\delta^{-2}$ vanish. Equating the
coefficients of $\delta^{-1}$ and $\delta^0$ to zero we, respectively, obtain 
%
\begin{align}
  \label{eq:Mult_Meth_EQs_1}
  &\partial_\tau\phi^{(1)}-\varepsilon\Delta_y\phi^{(1)}-\vec{u}\cdot\vec{\nabla}_y\phi^{(1)}=\vec{u}\cdot\vec{\nabla}_x\bar{\phi},
  \\
  \label{eq:Mult_Meth_EQs_2}
  &\partial_\tau\phi^{(2)}-\vec{u}\cdot\vec{\nabla}_y\phi^{(2)}-\varepsilon\Delta_y\phi^{(2)}
  =-\partial_t\bar{\phi}+\vec{u}\cdot\vec{\nabla}_x\phi^{(1)}+\varepsilon[\Delta_x\bar{\phi}+2\vec{\nabla}_x\cdot\vec{\nabla}_y\phi^{(1)}].
\end{align}
%
By the linearity of equation \eqref{eq:Mult_Meth_EQs_1}, we may
separate the fast and slow variables by writing
%\cite{McLaughlin:SIAM_JAM:780,Biferale:PF:2725}
\cite{McLaughlin:SIAM_JAM:780}
%
\begin{align}\label{eq:Linearity}
  \phi^{(1)}(t,\vec{x},\tau,\vec{y})
    =%\hat{\phi}^{(1)}(t,\vec{x})+
    \vec{\chi}(\tau,\vec{y})\cdot\vec{\nabla}_x\bar{\phi}(t,\vec{x}).
\end{align}
%
When
%$\hat{\phi}^{(1)}$ is a homogeneous solution of
%\eqref{eq:Mult_Meth_EQs_1} and
the components $\chi_k$, $k=1,\ldots,d$, of $\vec{\chi}$ satisfy   
%
\begin{align}\label{eq:Cell_Prob_Appendix}
  \partial_\tau\chi_k-\varepsilon\Delta_y\chi_k-\vec{u}\cdot\vec{\nabla}_y\chi_k=\vec{u}\cdot\vec{e}_k,
\end{align}
%
equation \eqref{eq:Mult_Meth_EQs_1} is automatically satisfied
\cite{McLaughlin:SIAM_JAM:780}. Equation \eqref{eq:Cell_Prob_Appendix}
along with \eqref{eq:u_DH} is equivalent to the cell problem
\eqref{eq:Cell_Problem}, where the distinction of fast variables was
dropped for notational simplicity. In order for
$\phi^{(1)}(t,\vec{x},\tau,\vec{y})$ in \eqref{eq:Linearity} to be periodic  
in $(\tau,\vec{y})$ for each fixed $(t,\vec{x})$, we must have that the
functions $\chi_k(\tau,\vec{y})$, $k=1,\ldots,d$, are periodic and
$\langle\chi_k\rangle=0$ (IS THIS WHY $\langle\chi_k\rangle=0$?). Here, $\langle\cdot\rangle$ denotes space-time
averaging with respect to the \emph{fast variables}. This and the
fundamental theorem of calculus implies that $\langle\vec{\nabla}_y\chi_k\rangle=0$. 



Due to the incompressibility of the velocity field
$\vec{\nabla}_y\cdot\vec{u}(\tau,\vec{y})=0$ and the \emph{a priori} fast variable
periodicity of the functions $\phi^{(i)}$, $i=1,2$,  the
fundamental theorem of calculus and the divergence theorem shows that
the average of the left-hand-sides of equations
\eqref{eq:Mult_Meth_EQs_1} and \eqref{eq:Mult_Meth_EQs_2} are
zero. For the equations to have solutions, the average of the
right-hand-sides must also vanish.
%\cite{Biferale:PF:2725} (Fredholm Algernative).
The resulting solvability conditions are $\langle\vec{u}\rangle=0$
and the following equation which governs the large-scale (slow
variable) dynamics   
%
\begin{align}\label{eq:Avg_EQ_phi}
  \partial_t\bar{\phi}=\varepsilon\Delta_x\bar{\phi}+\langle\vec{u}\cdot\vec{\nabla}_x\phi^{(1)}\rangle.
\end{align}
%
Here, we have used that $\bar{\phi}$ is a \emph{constant} with respect to
the fast variables and, by the divergence theorem and the fast
variable periodicity of $\phi^{(1)}$, we have
$\langle\vec{\nabla}_y\cdot\vec{\nabla}_x\phi^{(1)}\rangle=0$. The convergence of $\phi^\delta$ to $\bar{\phi}$ as
$\delta\to0$ is in $L^2$ \cite{Fannjiang:SIAM_JAM:333},  
%
\begin{align}
  \lim_{\delta\to0}\left[\,\sup_{0\leq t\leq t_0}
    \int\left|\phi^\delta(t,\vec{x})-\bar{\phi}(t,\vec{x})\right|^2\d\vec{x}
    \;\right]=0,
\end{align}
%
for all $t_0<\infty$, where we have used the notation $\d\vec{x}=\d x_1\cdots \d
x_d$ for the product Lebesgue measure.





Inserting equation \eqref{eq:Linearity} into \eqref{eq:Avg_EQ_phi}
yields equation \eqref{eq:phi_bar} with the components
$\Kc^*_{jk}=\Kbc^*\vec{e}_j\cdot\vec{e}_k$ of the effective diffusivity
tensor $\Kbc^*$ given by 
%
\begin{align}\label{eq:Eff_Diffusivity_Appendix}
  \Kc^*_{jk}=\varepsilon\delta_{jk}+\langle u_j\chi_k\rangle.
\end{align}
%
By inserting the representation for $u_j$ in
\eqref{eq:Cell_Prob_Appendix} into equation
\eqref{eq:Eff_Diffusivity_Appendix}, the functional $\langle u_j\chi_k\rangle$ can be
represented in terms of $\vec{\nabla}_y\chi_j$ and the \emph{skew-symmetric}
operator $\Sb=\Hb-\Delta_y^{-1}\Tb$, where the inverse operation $\Delta_y^{-1}$
is based on convolution with the Green's function for the Laplacian
$\Delta_y$, $\Tb=\partial_\tau\Ib$, and the $\Ib$ in this definition is to remind us
that the derivative $\partial_\tau$ operates componentwise. Indeed, writing  
$\partial_\tau\chi_j
%=\Delta_y\Delta_y^{-1}\partial_\tau\chi_j
=\vec{\nabla}_y\cdot(\Delta_y^{-1}\Tb)\vec{\nabla}_y\chi_j$,
$\Delta_y\chi_j=\vec{\nabla}_y\cdot\vec{\nabla}_y\chi_j$, and $\vec{u}=\vec{\nabla}_y\cdot\Hb$ in
\eqref{eq:u_DH}, we have    
%
\begin{align}\label{eq:Functional_Rep}
  \langle u_j\chi_k\rangle&=\langle[\partial_\tau\chi_j-\varepsilon\Delta_y\chi_j-\vec{u}\cdot\vec{\nabla}_y\chi_j]\chi_k\rangle
       \\
       &=\langle\vec{\nabla}_y\cdot[(\Delta_y^{-1}\Tb-\varepsilon\Ib-\Hb)\vec{\nabla}_y\chi_j]\chi_k\rangle
       \notag\\
       &=\langle[(\Hb-\Delta_y^{-1}\Tb+\varepsilon\Ib)\vec{\nabla}_y\chi_j]\cdot\vec{\nabla}_y\chi_k\rangle
       \notag\\
       &=\langle\Sb\vec{\nabla}_y\chi_j\cdot\vec{\nabla}_y\chi_k\rangle+\varepsilon\langle\vec{\nabla}_y\chi_j\cdot\vec{\nabla}_y\chi_k\rangle.
       \notag
\end{align}
%
Equations \eqref{eq:Eff_Diffusivity_Appendix} and
\eqref{eq:Functional_Rep} are equivalent to equations
\eqref{eq:Symm_Anti-Symm} and \eqref{eq:Eff_Diffusivity}, where the
distinction of fast variables was dropped for notational simplicity.





The above analysis shows that the main part of the study of effective, 
diffusive transport enhanced by periodic, incompressible flows, is the
study of equation \eqref{eq:Cell_Prob_Appendix}, from which the
effective diffusivity tensor $\Kbc^*$ emerges. In Section
\ref{sec:Hilbert_Space}, we use the analytical structure of the cell
problem \eqref{eq:Cell_Prob_Appendix} to derive a resolvent
representation for $\vec{\nabla}_y\chi_k$, involving an anti-symmetric
integro-differential operator $\Ab$ which is related to 
$\Sb=\Hb-\Delta^{-1}\partial_t\Ib$. In Section \ref{sec:Integral_Rep}, 
we employ this representation for $\vec{\nabla}_y\chi_k$ and the spectral
theorem, to provide integral representations for $\bkappa^*$ and
$\balpha^*$ involving a \emph{spectral measure} associated with the
operator $\Ab$ acting on a suitable Hilbert space.     
   

\subsection{Symmetries and commutivity}\label{sec:Symmetries_Commute}
%
asdf

\subsection{Existance and Uniqueness}\label{sec:Existance!}
%
Before we discuss how the Hilbert space framework presented above
leads to an  integral representation for $\Kbc^*$, we first discuss
some key differences in the theory between the cases of steady 
and dynamic velocity fields $\vec{u}$. These differences are reflected
in the measure underlying this integral representation for $\bkappa^*$
and stem from the \emph{unboundedness} of the operator $\partial_t$ on the
Hilbert space $\Hs_{\,\Tc}$ \cite{Reed-1980,Stone:64}. For steady
$\vec{u}$, in general, equation \eqref{eq:sigma*} reduces to
\eqref{eq:Eff_Diffusivity} for  diagonal components of the effective
parameter.  However, for dynamic $\vec{u}$, this is not true in
general. The details are as follows. For dynamic $\vec{u}$, the
operator $\bsig$ in \eqref{eq:Maxwells_Equations} can be written as
$\bsig=\varepsilon\Ib+\Sb$, where  $\Sb=\Hb-\Delta^{-1}\partial_t\Ib$ is skew-symmetric 
$\langle\Sb\vec{\xi},\vec{\zeta}\,\rangle=-\langle\vec{\xi},\Sb\vec{\zeta}\,\rangle$ for all
$\vec{\xi},\vec{\zeta}\in\Hs$ such that
$|\langle\partial_t\vec{\xi},\vec{\zeta}\,\rangle|,|\langle\vec{\xi},\partial_t\vec{\zeta}\,\rangle|<\infty$ (see Section
\ref{sec:Appendix} for details).  
This property of the operator $\Sb$ implies that
%$\langle\Sb\vec{E}_k\cdot\vec{E}_k\rangle=-\langle\Sb\vec{E}_k\cdot\vec{E}_k\rangle=0$.
%
\begin{align}\label{eq:Sb_Skew_2}
  \langle\Sb\vec{\xi}\cdot\vec{\xi}\,\rangle=-\langle\Sb\vec{\xi}\cdot\vec{\xi}\,\rangle=0,
  \qquad
  \Sb=\Hb-\Delta^{-1}\partial_t\Ib,
\end{align}
%
for all such $\vec{\xi}\in\Hs$. In this dynamic setting, equation
\eqref{eq:Sb_Skew} does not hold for every $\vec{\xi}\in\Hs$, as the
unbounded operator $\partial_t$ is defined only on a proper (dense) subset of
the Hilbert space $\Hs_{\,\Tc}$ \cite{Reed-1980}, and it may be that
$|\langle\partial_t\vec{\xi},\vec{\xi}\,\rangle|=\infty$. In the case of a steady velocity field
we have $\Sb\equiv\Hb$ and, by equation \eqref{eq:Bounded_H} and the Cauchy
Schwartz inequality, $|\langle\Sb\vec{\xi},\vec{\xi}\,\rangle|\leq\|\Hb\|\|\vec{\xi}\,\|^2<\infty$ for
all $\vec{\xi}\in\Hs$, so equation \eqref{eq:Sb_Skew} holds for all
$\vec{\xi}\in\Hs$.   









Another immediate consequence of equation \eqref{eq:Sb_Skew}, for
steady $\vec{u}$, is the coercivity of the bilinear functional
$\Phi(\vec{\xi},\vec{\zeta}\,)=\langle\bsig\vec{\xi}\cdot\vec{\zeta}\,\rangle$ for all $\varepsilon>0$. By equation
\eqref{eq:Bounded_H}, this functional is also bounded in the case of
steady $\vec{u}$ for all $\varepsilon<\infty$. Therefore, the Lax-Milgram theorem
\cite{McOwen:2003:PDE} provides the existence and uniqueness of a
solution $\vec{\nabla}\chi_k\in\Hs$ satisfying the cell problem
\eqref{eq:Cell_Problem}, or equivalently equation
\eqref{eq:Maxwells_Equations}, in this time-independent case. The
details are as follows. 




The distributional form of equation \eqref{eq:Cell_Problem}, written
as $\vec{\nabla}\cdot\bsig\vec{E}_k=0$, is given by
$\langle\bsig(\vec{\nabla}\chi_k+\vec{e}_k)\cdot\vec{\nabla}\zeta\rangle=0$, where $\zeta$ is a compactly
supported, infinitely differentiable function on $\Tc\otimes\Vc$, and we
stress that $\vec{\nabla}\zeta$ is \emph{curl-free}. Motivated by this, we
consider the following variational problem: find $\vec{\nabla}\chi_k\in\Hs$ such
that   
%
\begin{align}\label{eq:Variational}
  \langle\bsig(\vec{\nabla}\chi_k+\vec{e}_k)\cdot\vec{\xi}\,\rangle=0, \text{ for all }
  \vec{\xi}\in\Hs.
\end{align}
%
In order to directly apply the Lax-Milgram Theorem, we rewrite
equation \eqref{eq:Variational} as 
%
\begin{align}  \label{eq:Bilinear_functional_E} 
   &\Phi(\vec{\nabla}\chi_k,\vec{\xi}\,)
     =\langle\bsig\vec{\nabla}\chi_k\cdot\vec{\xi}\,\rangle
     =-\langle\bsig\vec{e}_k\cdot\vec{\xi}\,\rangle
     =f(\vec{\xi}\,). 
\end{align}
%
By equation \eqref{eq:Sb_Skew} $\Phi$ is coercive, i.e.
%$\Phi(\vec{\xi},\vec{\xi})=\langle[(\varepsilon\Ib+\Sb)]\vec{\xi}\cdot\vec{\xi}\rangle=\varepsilon\|\vec{\xi}\|^2>0$ 
%
\begin{align}\label{eq:Phi_Coercive}
  \Phi(\vec{\xi},\vec{\xi}\,)=\langle[(\varepsilon\Ib+\Sb)]\vec{\xi}\cdot\vec{\xi}\,\rangle=\varepsilon\|\vec{\xi}\,\|^2>0,
   \text{ for all } \vec{\xi}\in\Hs
\end{align}
%
such that $\|\vec{\xi}\,\|\neq0$ and $\varepsilon>0$, where $\|\cdot\|$
is the norm induced by the inner-product $\langle\cdot,\cdot\rangle$. Recall that
$\Sb=\Hb$ in this time-independent case. This, equation 
\eqref{eq:Bounded_H}, the triangle inequality,
and the Cauchy-Schwartz inequality imply that $\Phi$ is also bounded for
all $\varepsilon<\infty$
%for all $\vec{\xi},\vec{\zeta}\in\Hs$
% $\Phi(\vec{\xi},\vec{\zeta})\leq(\varepsilon+\|H\|)\|\vec{\xi}\|\|\vec{\zeta}\|$
%
\begin{align}\label{eq:Phi_Bounded}
  \Phi(\vec{\xi},\vec{\zeta})\leq(\varepsilon+\|H\|)\|\vec{\xi}\,\|\|\vec{\zeta}\,\|<\infty,
  \text{ for all } \vec{\xi}\in\Hs.
\end{align}
%
For the same reasons, the linear functional $f(\vec{\xi}\,)$ in
\eqref{eq:Bilinear_functional_E} is bounded for all
$\vec{\xi}\in\Hs$. Therefore, the Lax-Milgram theorem
\cite{McOwen:2003:PDE} provides the existence of a unique
$\vec{\nabla}\chi_k\in\Hs$ satisfying \eqref{eq:Cell_Problem} in this
time-independent case. 


In the time-dependent case, equation \eqref{eq:Sb_Skew} hence
\eqref{eq:Phi_Coercive} does not hold for all $\vec{\xi}\in\Hs$. Moreover, 
the operator $\partial_t$ hence $\bsig$ is not bounded on $\Hs$
\cite{Reed-1980,Stakgold:BVP:2000}, so \eqref{eq:Phi_Bounded} does not
hold. Consequently, the Lax-Milgram theorem cannot be directly
applied, and alternate techniques  
\cite{Friedman:1969:PDE,Friedman:1969:PDE:Parabolic} must be used to
prove the existence and uniqueness of a solution $\vec{\nabla}\chi_k\in\Hs$ 
satisfying the cell problem \eqref{eq:Cell_Problem}. This discussion
illustrates key differences in the analytic structure of the effective
parameter problem for $\bkappa^*$, between the cases of steady and
dynamic velocity fields $\vec{u}$, which stem from the unboundedness
of the operator $\partial_t$ on $\Hs_{\,\Tc}$, hence $\bsig$ on $\Hs$. In Section
\ref{sec:Integral_Rep}, we will discuss other consequences of this
boundedness/unboundedness property of the operator $\bsig$, and
demonstrate that it leads to significant differences in the spectral
measure underlying an  integral representation of $\bkappa^*$.     






\medskip

{\bf Acknowledgements.}
% We gratefully acknowledge support from the Division of Mathematical
% Sciences and the Office of Polar Programs at the U.S. 
% National Science Foundation (NSF) through Grants
% DMS-1009704, ARC-0934721, and DMS-0940249. We are also grateful for 
% support from the Office of Naval Research (ONR) through
% Grants N00014-13-10291 and N00014-12-10861. Finally, we would like to 
% thank the NSF Math Climate Research Network (MCRN) for their support
% of this work. 


\medskip

\bibliographystyle{plain}
\bibliography{murphy}
\end{document}

% LocalWords:  McMaster RM jk sig eps def Maxwells Milgram coercivity diag jX
% LocalWords:  mh Cond mu kk PtI Fs Es Hashin Shtrikman extremized Decomp chi
% LocalWords:  Acknowledgements DMS ONR MCRN murphy
