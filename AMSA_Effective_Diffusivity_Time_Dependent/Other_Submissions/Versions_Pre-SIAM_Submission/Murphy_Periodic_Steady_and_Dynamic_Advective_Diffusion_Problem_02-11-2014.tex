\documentclass[11pt]{amsart}
%\usepackage{latexsym, amssymb, enumerate, amsmath}
\usepackage{graphicx,amssymb,amsmath,amsfonts,mathrsfs}

\setlength{\textwidth}{6.5in}
\setlength{\textheight}{9.0in}
\setlength{\oddsidemargin}{0in}
\setlength{\evensidemargin}{0in}
\setlength{\topmargin}{-0.5in}

\renewcommand{\topfraction}{0.85}
\renewcommand{\textfraction}{0.1}
\renewcommand{\floatpagefraction}{0.55}%0.75


\newcommand{\ph}{\hat{\phi}}
\newcommand{\pt}{\tilde{\phi}} 
\newcommand{\pc}{\check{\phi}}
\newcommand{\gh}{\hat{\gamma}}
\newcommand{\Dh}{\hat{\Delta}}
\newcommand{\dha}{\hat{\delta}}
\newcommand{\qh}{\hat{q}}
\newcommand{\xh}{\hat{x}} 
\newcommand{\HM}{\mathcal{H}_{\text{max}}}
\newcommand{\Hm}{\mathcal{H}_{\text{min}}}
\newcommand{\sech}{\rm \hspace{0.7mm}sech}
\newcommand{\I}{\mathrm{i}}
\newcommand{\e}{\mathrm{e}}
\renewcommand{\d}{\mathrm{d}}
\newcommand{\hh}{\hat{h}}
\newcommand{\mh}{m_r}
\newcommand{\mt}{m_i}

\newcommand{\Mb}{\mathbf{M}}
\newcommand{\Xb}{\mathbf{X}}
\newcommand{\Tb}{\mathbf{T}}
\newcommand{\Hb}{\mathbf{H}}
\newcommand{\Kb}{\mathbf{K}}
\newcommand{\Jb}{\mathbf{J}}
\newcommand{\Ib}{\mathbf{I}}
\newcommand{\Sb}{\mathbf{S}}
\newcommand{\Rb}{\mathbf{R}}
\newcommand{\Ab}{\mathbf{A}}
\newcommand{\Bb}{\mathbf{B}}
\newcommand{\Eb}{\mathbf{E}}
\newcommand{\Qb}{\mathbf{Q}}
\newcommand{\Nb}{\mathbf{N}}
\newcommand{\Ob}{\mathbf{O}}
\newcommand{\Vb}{\mathbf{V}}
\newcommand{\Wb}{\mathbf{W}}

\newcommand{\Kc}{\mathcal{K}}
\newcommand\Kbc{\mbox{\boldmath${\mathcal{K}}$}}


\newcommand{\Tc}{\mathcal{T}}
\newcommand{\Vc}{\mathcal{V}}
\newcommand{\Hc}{\mathcal{H}}
\newcommand{\Fc}{\mathcal{F}}
\newcommand{\Sc}{\mathcal{S}}

\newcommand{\Hs}{\mathscr{H}}
\newcommand{\As}{\mathscr{A}}
\newcommand{\Ds}{\mathscr{D}}
\newcommand{\Fs}{\mathscr{F}}
\newcommand{\Ss}{\mathscr{S}}


\newcommand\bsig{\mbox{\boldmath${\sigma}$}}
\newcommand\beps{\mbox{\boldmath${\epsilon}$}}
\newcommand\bxi{\mbox{\boldmath${\xi}$}}
\newcommand\bmu{\mbox{\boldmath${\mu}$}}
\newcommand\balpha{\mbox{\boldmath${\alpha}$}}
\newcommand\brho{\mbox{\boldmath${\rho}$}}
\newcommand\bDelta{\mbox{\boldmath${\Delta}$}}
\newcommand\bkappa{\mbox{\boldmath${\kappa}$}}
\newcommand\bGamma{\mbox{\boldmath${\Gamma}$}}
\newcommand\bUpsilon{\mbox{\boldmath${\Upsilon}$}}
\newcommand\bLambda{\mbox{\boldmath${\Lambda}$}}


\newtheorem{theorem}{Theorem}[section]
\newtheorem{lemma}{Lemma}[section]
\newtheorem{corollary}{Corollary}[section]


% \newtheorem{theorem}{Theorem}[section]
% \newtheorem{prop}{Proposition}[section]
% \newtheorem{lemma}{Lemma}[section]
% \newtheorem{cor}{Corollary}[section]

%     %\theoremstyle{definition}

% \newtheorem{defn}[thm]{Definition}
% \newtheorem{notation}[thm]{Notation}
% \newtheorem{example}[thm]{Example}
% \newtheorem{conj}[thm]{Conjecture}
% \newtheorem{prob}[thm]{Problem}

%     %\theoremstyle{remark}

%\newtheorem{rem}[thm]{Remark}
    % Use the standard latex environments for theorems, etc. Here is one
          % possible method of declaring them: It numbers all results by the
          % section, and uses a common numbering system for the different
          % environmentts.

\begin{document}

\title{Spectral theory of advective diffusion \\
  by dynamic and steady periodic flows}


% AUTHORS 
%\author{N. B. Murphy, A. Gully, E. Cherkaev, and K. M. Golden}
\author{N. B. Murphy$^\ast$}
\address{$^*$Department of Mathematics, 340 Rowland Hall, University of
  California at Irvine, Irvine, CA 92697-3875, USA}
\email{nbmurphy@math.uci.edu}

\author{J. Xin$^\dag$}
\address{$^{\dag}$Department of Mathematics, 340 Rowland Hall, University of
  California at Irvine, Irvine, CA 92697-3875, USA} 
\email{jxin@math.uci.edu}

\author{J. Zhu$^\star$}
\address{$^\star$University of Utah, Department of Mathematics, 155 S 1400 E
  RM 233, Salt Lake City, UT 84112-009, USA}
\email{zhu@math.utah.edu}

\author{E. Cherkaev$^\ddagger$}
\address{$^\ddagger$University of Utah, Department of Mathematics, 155 S 1400 E
  RM 233, Salt Lake City, UT 84112-009, USA} 
\email{elena@math.utah.edu}

\maketitle
\vspace{-3ex}
\begin{center}
  Department of Mathematics, University of California at Irvine
\end{center}

%\vspace{3ex}


\begin{abstract}
%
The analytic continuation method for representing transport in
composites provides integral representations for the
effective coefficients of two-phase random media. Here we adapt this 
method to characterize the effective thermal transport properties of
advective diffusion by periodic flows. Our novel approach yields
integral representations for the symmetric and 
anti-symmetric parts of the effective diffusivity. These
representations hold for dynamic and steady incompressible flows, and
involve the spectral measure of a self-adjoint or normal linear
operator. In the case of a steady fluid velocity field, the spectral
measure is associated with a Hermitian Hilbert-Schmidt integral
operator, and in the case of dynamic flows, it is associated with an
unbounded integro-differential operator. We utilize the integral
representations to obtain asymptotic behavior of the effective
diffusivity as the molecular diffusivity tends to zero, for model
flows. Our analytical results are supported by numerical computations
of the spectral measures and effective diffusivities.     
%
\end{abstract}

\section{Introduction}\label{sec:Introduction}
%
The long time, large scale behavior of a diffusing particle   
or tracer being advected by an incompressible velocity field 
is equivalent to an enhanced diffusive process \cite{Taylor:PRSL:196} 
with an effective diffusivity tensor $\Kbc^*$.
Determining the effective transport properties of advection enhanced
diffusion is a challenging problem with theoretical and practical 
importance in many fields of science and engineering,
ranging from turbulent combustion to mass, heat, and salt transport in
geophysical flows \cite{Moffatt:RPP:621}. A broad range of
mathematical techniques have been developed that reduce the analysis
of complex fluid flows, with rapidly varying structures in space and
time, to solving averaged or \textit{homogenized} equations that do
not have rapidly varying data, and involve an effective parameter.




Homogenization of the advection-diffusion equation for thermal
transport by random, time-independent fluid velocity fields was
treated in \cite{McLaughlin:SIAM_JAM:780}. This 
reduced the analysis of turbulent diffusion to solving a
diffusion equation involving a homogenized temperature and a
(constant) effective diffusivity tensor $\Kbc^*$. An important
consequence of this analysis is that $\Kbc^*$ is given in terms  
of a \emph{curl-free} stationary stochastic process which satisfies a
steady state diffusion equation, involving a skew-symmetric random
matrix $\Hb$ \cite{Avellaneda:CMP-339,Avellaneda:PRL-753}. By adapting
the analytic continuation method (ACM) of homogenization theory for
composites \cite{Golden:CMP-473}, it was shown that the result in
\cite{McLaughlin:SIAM_JAM:780} leads to an integral
representation for the symmetric part $\bkappa^*$ of $\Kbc^*$,
involving a spectral measure of a self-adjoint random
operator \cite{Avellaneda:CMP-339,Avellaneda:PRL-753}. This integral
representation of $\bkappa^*$ was generalized to the time-dependent
case in \cite{Avellaneda:PRE:3249,Biferale:PF:2725}. Remarkably, this 
method has also been extended to flows with incompressible
\emph{nonzero} effective drift \cite{Pavliotis:PHD_Thesis}, flows
where particles diffuse according to linear collisions
\cite{Pavliotis:IMAJAM:951}, and solute transport in porous media
\cite{Bhattacharya:AAP:1999:951}. All these approaches yield integral
representations of the symmetric and, when appropriate, the
anti-symmetric part $\balpha^*$ of $\Kbc^*$. 




Homogenization of the advection-diffusion equation for periodic or
cellular, incompressible flow fields was treated in
\cite{Fannjiang:SIAM_JAM:333,Fannjiang:1997:1033}. As in the case of
random flows, the effective diffusivity tensor
$\Kbc^*$ is given in terms of a \emph{curl-free} vector field, which
satisfies a diffusion equation involving a skew-symmetric
matrix $\Hb$. Here, we demonstrate that the ACM can
be adapted to this periodic setting to provide integral
representations for both the symmetric $\bkappa^*$ and 
anti-symmetric $\balpha^*$ parts of $\Kbc^*$, for both cases of
steady and time-dependent flows. These integral representations
involve a self-adjoint or normal linear operator and the
(non-dimensional) molecular diffusivity $\varepsilon$. In the case of steady
fluid velocity fields, the spectral measure is associated with a
Hermitian Hilbert-Schmidt integral operator involving the Green's
function of the Laplacian on a rectangle. While in the case of dynamic 
flows, the spectral measure is associated with a Hermitian operator
which is the sum of that for steady flows and an unbounded
integro-differential operator.      
 

We utilize the analytic structure of the  integral
representation for $\Kbc^*$ to obtain its asymptotic behavior for
model flows, as the molecular diffusivity $\varepsilon$ tends to zero. This is
the high P\'{e}clet number regime that is important for the
understanding of transport processes in real fluid flows, where the
molecular diffusivity is often quite small in comparison. In
particular, FINISH THIS PARAGRAPH WHEN WE HAVE CONCRETE RESULTS.
necessary and sufficient conditions for steady periodic flow
fields $\kappa^*\sim\epsilon^{1/2}$, generically, for steady flows and $\kappa^*\sim O(1)$ for
``chaotic'' time-dependent flows. To make this manuscript more self
contained, we have include an appendix in Section \ref{sec:Appendix}
which contains many of the technical details underlying this work.

%
\section{Mathematical Methods}\label{sec:Mathematical_Methods} 
%
In this section, we formulate the effective parameter problem for
enhanced diffusive transport by advective, periodic flows, which leads
to a functional representation of the effective diffusivity tensor
$\Kbc^*$. We provide integral representations for the symmetric
$\bkappa^*$ and anti-symmetric $\balpha^*$ parts of $\Kbc^*$, which
hold for both steady and dynamic flows. The effective parameter
problem
\cite{McLaughlin:SIAM_JAM:780,Fannjiang:SIAM_JAM:333,Biferale:PF:2725} 
for such transport processes is reviewed in Section
\ref{sec:Eff_Trans}, and parallels existing between this problem
of homogenization theory \cite{Bensoussan:Book:1978} and the ACM for
representing transport in composites \cite{Golden:CMP-473} are put
into correspondence. In particular, an abstract Hilbert space
framework is provided in Section \ref{sec:Hilbert_Space} which places
these different effective parameter problems on common mathematical
footing. Within this Hilbert space setting, we derive in Section
\ref{sec:Integral_Rep} integral representations for $\bkappa^*$ and
$\balpha^*$, involving the molecular diffusivity $\varepsilon$ and a
\emph{spectral measure} of a self-adjoint (or equivalently a maximal
normal) linear operator. These integral representations are employed
in Section \ref{sec:Assymptotics} to obtain the asymptotic behavior of
the components of $\Kbc^*$ in the scaling regime, where $\varepsilon\ll1$.  



\subsection{Effective transport by
  advective-diffusion} \label{sec:Eff_Trans}  
%
In this section, we review the effective parameter problem for
advection enhanced diffusion. For completness and to streamline the
presentation of this theory, many of the mathematical details are
given in Section \ref{sec:Multiscal_Method} of the appendix.
Consider the advection enhanced diffusive transport of a passive
tracer $\phi(t,\vec{x}\,)$, $t>0$, $\vec{x}\in\mathbb{R}^d$, as described by
the advection-diffusion equation 
%
\begin{align}\label{eq:ADE}
  \partial_t\phi=\kappa_0\Delta \phi+\vec{\nabla}\cdot(\vec{v}\phi), \quad
  \phi(0,\vec{x})=\phi_0(\vec{x}),
  %\qquad
  %\vec{\nabla}\cdot\vec{v}=0
\end{align}
% 
with initial density $\phi_0(\vec{x})$ given. Here, $\partial_t$ denotes partial
differentiation with respect to time $t$, $\Delta=\vec{\nabla}\cdot\vec{\nabla}=\nabla^2$ is
the Laplacian, and $\kappa_0>0$ is the molecular diffusivity. The fluid
velocity field $\vec{v}=\vec{v}(t,\vec{x})$ in \eqref{eq:ADE} is
assumed to be periodic, incompressible, and mean-zero, %and bounded 
%
\begin{align}\label{eq:incompressible}
  \vec{\nabla}\cdot\vec{v}=0, \quad
  \langle\vec{v}\,\rangle=0,
  %\quad
  %\langle|\vec{v}|^2\rangle<\infty.
\end{align}
%
where we denote by $0$ the null element of all linear spaces in
question. Consider the bounded sets $\Tc\subset\mathbb{R}$ and
$\Vc\subset\mathbb{R}^d$, with $t\in\Tc$ and $\vec{x}\in\Vc$, which define the  
space-time period cell ($(d+1)$--torus) $\Tc\otimes\Vc$. In equation
\eqref{eq:incompressible}, we denote by $\langle\cdot\rangle$ spatial averaging over
$\Vc$ in the case of a time-independent velocity field,
$\vec{v}=\vec{v}(\vec{x})$, and when the velocity field is
time-dependent, $\vec{v}=\vec{v}(t,\vec{x})$, $\langle\cdot\rangle$ denotes space-time
averaging over $\Tc\otimes\Vc$. In the time-dependent case, we stress that
$\langle\vec{v}\,\rangle=0$ in \eqref{eq:incompressible} means that
$\langle\vec{v}(t,\cdot)\rangle=0$ \emph{and} $\langle\vec{v}(\cdot,\vec{x})\rangle=0$ for each
$t\in\Tc$ and $\vec{x}\in\Vc$ fixed, respectively. 



We non-dimensionalize equation
\eqref{eq:ADE} as follows. Let $\ell$ and $\tilde{t}$ be typical length
and time scales associated with the problem of interest. Mapping to
the non-dimensional variables $t\mapsto t/\tilde{t}$ and $x_j\mapsto x_j/\ell$, one
finds that $\phi$ satisfies the advection-diffusion equation in
\eqref{eq:ADE} with a non-dimensional molecular diffusivity
$\varepsilon=\tilde{t}\kappa_0/\ell^{\,2}$ and velocity field
$\vec{u}=\tilde{t}\,\vec{v}/\ell$, where $x_j$ is the $j^{\,\text{th}}$
component of the vector $\vec{x}$. 




For $d$-dimensional, mean-zero, incompressible flows $\vec{u}$, there
is a real (non-dimensional) skew-symmetric 
matrix $\Hb(t,\vec{x})$ such that (see Section \ref{eq:flow_matrix}
for details) 
%$\Hb^{\,T}=-\Hb$, such that $\vec{u}=\vec{\nabla}\cdot\Hb$. 
% %
\begin{align}\label{eq:u_DH}
 \vec{u}=\vec{\nabla}\cdot\Hb, \qquad  \Hb^{\,T}=-\Hb,
\end{align}
% %
where $\Hb^{\,T}$ denotes transposition of the matrix $\Hb$.
Using this representation of the velocity field $\vec{u}$, which also
satisfies \eqref{eq:incompressible}, equation \eqref{eq:ADE} can be
written as a diffusion equation,  
%
\begin{align}\label{eq:ADE_Divergence}
  \partial_t\phi%&=\varepsilon\Delta \phi+\vec{u}\cdot\vec{\nabla}\phi\\
    %&=\varepsilon\vec{\nabla}\cdot\vec{\nabla}\phi+(\vec{\nabla}\cdot\Hb)\cdot\vec{\nabla}\phi\\
    %&=\vec{\nabla}\cdot[\varepsilon I+\Hb]\vec{\nabla}\phi\\
    %&=\vec{\nabla}\cdot\bkappa\vec{\nabla}\phi
    =\vec{\nabla}\cdot\bkappa\vec{\nabla}\phi, \quad
    %\bkappa=\varepsilon I+\Hb,
    \phi(0,\vec{x})=\phi_0(\vec{x}),
    \qquad
    \bkappa=\varepsilon\Ib+\Hb,
\end{align}
%
where $\bkappa(t,\vec{x})=\varepsilon\Ib+\Hb(t,\vec{x})$ can be viewed as a local
diffusivity tensor with coefficients
%
\begin{align}\label{eq:kappa_coeff}
  \kappa_{jk}=\varepsilon\delta_{jk}+H_{jk},\quad j,k=1,\cdots,d.
\end{align}
%
Here, $\delta_{jk}$ is the Kronecker delta and we denote by $\Ib$ the
identity operator on all linear spaces in question.    






We are interested in the dynamics of $\phi$ in \eqref{eq:ADE_Divergence}
for \emph{large} length and time scales, and when the initial density
$\phi_0$ is slowly varying relative to the velocity field
$\vec{u}$. Anticipating that $\phi$ will have diffusive dynamics, we
re-scale space and time by $\vec{x}\mapsto\vec{x}/\delta$ and $t\mapsto t/\delta^2$,
respectively.  For periodic diffusivity coefficients in
\eqref{eq:ADE_Divergence} which are uniformly elliptic but not
necessarily symmetric, it can be shown \cite{Fannjiang:SIAM_JAM:333}
that, as $\delta\to0$, the associated solution $\phi^\delta(t,\vec{x})$ of
\eqref{eq:ADE_Divergence} converges to $\bar{\phi}(t,\vec{x})$,
which satisfies the following diffusion equation involving a (constant)
effective diffusivity tensor $\Kbc^*$    
%
\begin{align}\label{eq:phi_bar}
  \partial_t\bar{\phi}=\vec{\nabla}\cdot\Kbc^*\vec{\nabla}\bar{\phi}, \quad
  \bar{\phi}(0,\vec{x})=\phi_0(\vec{x}).
\end{align}
%





The components $\Kc^*_{jk}=\Kbc^*\vec{e}_j\cdot\vec{e}_k$ of the effective
tensor $\Kbc^*$ are given by $\Kc^*_{jk}=\varepsilon\delta_{jk}+\langle u_j\chi_k\rangle$. For each
standard basis vector $\vec{e}_k$, $k=1,\ldots,d$, the function
$\chi_k=\chi_k(t,\vec{x}\,;\vec{e}_k)$ satisfies
\cite{Fannjiang:SIAM_JAM:333} the cell problem    
% 
\begin{align}\label{eq:Cell_Problem}
  \partial_t\chi_k=\vec{\nabla}\cdot\bkappa(\vec{\nabla}\chi_k+\vec{e}_k\,), \quad
    %=\vec{\nabla}\cdot[(\varepsilon\Ib+\Hb)(\vec{\nabla}\chi_k+\vec{e}_k\,)], \quad
  \langle\vec{\nabla}\chi_k\rangle=0.
%  \quad   k=1,\ldots,d,
\end{align}
%
Equation \eqref{eq:Cell_Problem} also holds
\cite{Fannjiang:SIAM_JAM:333} when the velocity field is
time-independent $\vec{u}=\vec{u}(\vec{x})$. However in this case, 
$\chi_k$ is also time-independent and $\partial_t\chi_k=0$. The symmetric $\bkappa^*$ and
anti-symmetric $\balpha^*$ parts of the effective diffusivity tensor
$\Kbc^*$ are defined by 
%
\begin{align}\label{eq:Symm_Anti-Symm}
  \Kbc^*=\bkappa^*+\balpha^*,\qquad
  \bkappa^*=\frac{1}{2}\left(\Kbc^*+[\Kbc^*]^{\,T}\right), \quad
  \balpha^*=\frac{1}{2}\left(\Kbc^*-[\Kbc^*]^{\,T}\right).
\end{align}
%
The components $\kappa^*_{jk}$ and $\alpha^*_{jk}$, $j,k=1,\ldots,d$, of $\bkappa^*$
and $\balpha^*$ can be written in terms of the following functionals
involving the \emph{real-valued} vector field $\vec{\nabla}\chi_k$
%
\begin{align}\label{eq:Eff_Diffusivity}
 \kappa^*_{jk}=\varepsilon(\delta_{jk}+\langle\vec{\nabla}\chi_j\cdot\vec{\nabla}\chi_k\rangle), \qquad
 \alpha^*_{jk}=\langle\Sb\vec{\nabla}\chi_j\cdot\vec{\nabla}\chi_k\rangle, \qquad
 \Sb=\Hb-(\bDelta^{-1})\Tb, \quad \Tb=\partial_t\Ib.
\end{align}
%
Here, $\Tb=\text{diag}(\partial_t,\ldots,\partial_t)$ operates component-wise on vector
fields, $\bDelta^{-1}=\text{diag}(\Delta^{-1},\ldots,\Delta^{-1})$ is the inverse of
the vector Laplacian, and the inverse operation $\Delta^{-1}$ is based on
convolution with the Green's function for the Laplacian $\Delta$ on $\Vc$
\cite{Stakgold:BVP:2000}.







Due to the fact that the vector field
$\vec{\nabla}\chi_j$ is \emph{real-valued}, we have that
$\langle\vec{\nabla}\chi_j\cdot\vec{\nabla}\chi_k\rangle=\langle\vec{\nabla}\chi_k\cdot\vec{\nabla}\chi_j\rangle$. From equation
\eqref{eq:Eff_Diffusivity} this clearly implies that the tensor    
$\bkappa^*$ is symmetric, $\kappa^*_{jk}=\kappa^*_{kj}\,$. Moreover, equation
\eqref{eq:Eff_Diffusivity} demonstrates that the effective transport
of the tracer $\phi$ in the principle directions $\vec{e}_k$, $k=1,\ldots,d$,
is always \emph{enhanced} by the presence of an incompressible velocity
field, $\Kc^*_{kk}=\kappa^*_{kk}\geq\varepsilon$. The equality $\Kc^*_{kk}=\kappa^*_{kk}$
follows from the skew-symmetry of $\balpha^*$, so that
$\alpha^*_{kj}=-\alpha^*_{jk}$ and $\alpha^*_{kk}=0$. This, in turn, follows from the
skew-symmetry of the operator $\Sb$ (see Section
\ref{sec:Symmetries_Commute}),
$\alpha^*_{jk}=\langle\Sb\vec{\nabla}\chi_j\cdot\vec{\nabla}\chi_k\rangle=-\langle\Sb\vec{\nabla}\chi_k\cdot\vec{\nabla}\chi_j\rangle=-\alpha^*_{kj}\,$ 
and 
%
\begin{align}\label{eq:Sb_Skew}
  \alpha^*_{kk}=\langle\Sb\vec{\nabla}\chi_k\cdot\vec{\nabla}\chi_k\rangle=-\langle\Sb\vec{\nabla}\chi_k\cdot\vec{\nabla}\chi_k\rangle=0.  
\end{align}
%
In Section \ref{sec:Hilbert_Space} we discuss the properties of the
linear operator $\Sb$ and the vector field $\vec{\nabla}\chi_j$ in more detail.





We now recast equations \eqref{eq:Cell_Problem} and
\eqref{eq:Eff_Diffusivity} into a form which parallels the effective
parameter problem for transport in composites. This allows us to
bring to bear on the effective parameter problem for advective
diffusion, the well developed mathematical techniques of the
ACM for characterizing effective transport in composite media
\cite{Golden:CMP-473,MILTON:2002:TC}. This method gives a Hilbert
space formulation of the effective parameter problem and provides an
integral representation for the effective transport coefficients of
composites, involving a \emph{spectral measure} of a self-adjoint
operator which depends only on the composite geometry
\cite{Golden:CMP-473,Murphy:JMP:063506,MILTON:2002:TC}. Here we
establish a correspondence between this effective parameter problem
and that for enhanced diffusive transport by advective velocity 
fields. In Section \ref{sec:Hilbert_Space}, we formulate the Hilbert
space framework associated with advective diffusion, and employ it to
obtain a resolvent representation of the vector field $\vec{\nabla}\chi_k$ in
\eqref{eq:Cell_Problem}. In Section \ref{sec:Integral_Rep} we utilize
this mathematical framework to obtain integral representations for 
$\bkappa^*$ and $\balpha^*$, involving a spectral measure which depends
only on the fluid velocity field $\vec{u}$.    




Toward this goal, we recast the first formula in equation
\eqref{eq:Cell_Problem} in a more suggestive, divergence form. Using
the notation from equation \eqref{eq:Eff_Diffusivity} we write 
%
\begin{align}\label{eq:Dt_T}
  \vec{\nabla}(\Delta^{-1})\partial_t=\bDelta^{-1}\Tb\vec{\nabla},
\end{align}
%
so that \cite{Fannjiang:SIAM_JAM:333}
$\partial_t\chi_k=\Delta\Delta^{-1}\partial_t\chi_k=\vec{\nabla}\cdot(\bDelta^{-1}\Tb)\vec{\nabla}\chi_k$. Define the  
vector field $\vec{E}_k=\vec{\nabla}\chi_k+\vec{e}_k$ and the operator
$\bsig=\bkappa-(\bDelta^{-1})\Tb=\varepsilon\Ib+\Sb$, where
$\bsig=\bkappa=\varepsilon\Ib+\Hb$ in the case of steady fluid velocity
fields. With these definitions, equation \eqref{eq:Cell_Problem} may
be written as  $\vec{\nabla}\cdot\bsig\vec{E}_k=0$, $\langle\vec{E}_k\rangle=\vec{e}_k$,
which is equivalent to    
%
\begin{align}\label{eq:Maxwells_Equations}    
  \vec{\nabla}\cdot\vec{J}_k=0, \quad
  \vec{\nabla}\times\vec{E}_k=0, \quad
  \vec{J}_k=\bsig\vec{E}_k,\quad
  \langle\vec{E}_k\rangle=\vec{e}_k,\qquad
  \bsig%=\bkappa-(\bDelta^{-1})\Tb.
       %=\varepsilon\Ib+\Hb-(\bDelta^{-1})\Tb.
       =\varepsilon\Ib+\Sb.
\end{align}
%
The formulas in \eqref{eq:Maxwells_Equations} are precisely the
electrostatic version of Maxwell's equations for a conductive medium
\cite{Golden:CMP-473}, where $\vec{E}_k$ and $\vec{J}_k$ are the local
electric field and current density, respectively, and $\bsig$ is the
local conductivity tensor of the medium. In the ACM for composites,
the effective conductivity tensor $\bsig^*$ is defined as
% 
\begin{align}\label{eq:sigma*}
  \langle\vec{J}_k\rangle=\bsig^*\langle\vec{E}_k\rangle.
\end{align}
%
The linear constitutive relation $\vec{J}_k=\bsig\vec{E}_k$ in
\eqref{eq:Maxwells_Equations} relates the local intensity and flux,
while that in \eqref{eq:sigma*} relates the mean intensity and
flux. Due to the skew-symmetry of $\Sb$, the intensity-flux
relationship in \eqref{eq:Maxwells_Equations} is similar to that of a
Hall medium \cite{Isichenko:JNS:1991:375}.



For the (constant) tensors $\Kbc^*$ and $\bsig^*$ to be
meaningful, the averages which define these effective quantities in
\eqref{eq:Eff_Diffusivity} and \eqref{eq:sigma*} must be well defined
and finite. For example, in order for the diagonal components
$\Kc^*_{kk}$, $k=1,\ldots,d$, of $\Kbc^*$ to be well defined and finite,
the vector field $\vec{\nabla}\chi_k$ must be Lebesgue measurable 
and square integrable on $\Tc\times\Vc$. Moreover, for the components
$\alpha^*_{jk}$, $j\neq k=1,\ldots,d$, of $\balpha^*$ to be well defined and
finite, we must also have that the operator $\Sb$ is bounded in some
sense so that $\Sb\vec{\nabla}\chi_j\cdot\vec{\nabla}\chi_k$ is Lebesgue integrable on
$\Tc\times\Vc$. In other words, we must define the vector field
$\vec{\nabla}\chi_j$ as a member of a suitable space of functions so that the
components of the tensors $\Kbc^*$ and $\bsig^*$ are well defined and
have finite values. In Section \ref{sec:Hilbert_Space} we discuss
these important details at length and prove the following theorem. 
%
\begin{theorem}\label{thm:kappa_sigma}
%
Let the components $\Kc^*_{jk}$ and $\sigma^*_{jk}$, $j,k=1,\ldots,d$, of the
effective tensors $\Kbc^*$ and $\bsig^*$ be defined as in equations
\eqref{eq:Cell_Problem}--\eqref{eq:Eff_Diffusivity} and
\eqref{eq:Maxwells_Equations}--\eqref{eq:sigma*}, respectively. Then
there exists a function space $\Fs$ on which $\bsig=\varepsilon\Ib+\Sb$ is a
bounded linear operator for all $0<\varepsilon<\infty$ and, for $\vec{\nabla}\chi_j\in\Fs$,
$\Kc^*_{jk}$ and $\sigma^*_{jk}$ are well defined and finite.
%$|\Kc^*_{jk}|,|\sigma^*_{jk}|<\infty$.
Moreover, these effective tensors are
equivalent up to transposition,  
% 
\begin{align}\label{eq:Eff_Equiv}
  \bsig^*=[\Kbc^*]^T.
\end{align}
%
In particular, the symmetric part $\bkappa^*$ of $\Kbc^*$ is equal to
that of $\bsig^*$ and the anti-symmetric part $\balpha^*$ of $\Kbc^*$
is equal to the negative of that of $\bsig^*$.
%
\end{theorem}
%



Theorem \ref{thm:kappa_sigma} places the effective parameter problems
for transport in composites and that for transport by advective
diffusion on common mathematical footing, for both cases of  
time-independent and time-dependent velocity fields $\vec{u}$. The
validity of Theorem \ref{thm:kappa_sigma} follows by adapting the
Hilbert space formulation of the ACM to treat the effective transport
properties of advective diffusion, which is the topic of Section
\ref{sec:Hilbert_Space}. This Hilbert space formulation of the
effective parameter problem also leads to integral representations for
$\bkappa^*$ and $\balpha^*$, which is the topic of Section
\ref{sec:Integral_Rep}.              





\subsection{Hilbert space and resolvent
  representation} \label{sec:Hilbert_Space}   
%
In this section we explore the mathematical properties of the
skew-symmetric operator $\Sb$ introduced in equation
\eqref{eq:Eff_Diffusivity} and construct a function space $\Fs$ such
that for $\vec{\nabla}\chi_k\in\Fs$ equation \eqref{eq:Eff_Equiv} holds and is
well defined. We do so by providing an abstract Hilbert space
formulation of the effective parameter problem for advective
diffusion. We utilize this mathematical framework and equation
\eqref{eq:Cell_Problem}  to obtain a resolvent representation of the
vector field $\vec{\nabla}\chi_k$, involving an anti-symmetric operator $\Ab$
which is closely related to $\Sb$, where we use the terms
skew-symmetric and anti-symmetric interchangeably. Using the results
of this section, we derive in Section \ref{sec:Integral_Rep} integral
representations for the symmetric $\bkappa^*$ and anti-symmetric
$\balpha^*$ parts of the effective diffusivity tensor $\Kbc^*$,
involving a \emph{spectral measure} associated with $\Ab$.       



Consider the Hilbert spaces $L^2_d(\Tc)=\otimes_{n=1}^dL^2(\Tc)$ and
$L^2_d(\Vc)=\otimes_{n=1}^dL^2(\Vc)$ (over the complex field $\mathbb{C}$)
of Lebesgue measurable, square integrable, vector-valued functions
\cite{Folland:99}, where $\Tc\subset\mathbb{R}$ and $\Vc\subset\mathbb{R}^d$. Now 
consider the associated Hilbert spaces $\Hs_{\,\Tc}\subset L^2_d(\Tc)$ and
$\Hs_{\,\Vc}\subset L^2_d(\Vc)$ of periodic vector-valued functions with
temporal periodicity $T$ on the interval $\Tc=(0,T)$ and spatial
periodicities $V_j$, $j=1,\ldots,d$, on the $d$-dimensional region
$\Vc=(0,V_1)\times\cdots\times(0,V_d)$, respectively, as well as their direct product 
$\Hs_{\Tc\Vc}$,
%
\begin{align}\label{eq:Hilbert_Spaces}
  \Hs_{\Tc\Vc}=\Hs_{\,\Tc}\otimes\Hs_{\,\Vc}, \quad
  \Hs_{\,\Tc}=\{ 
     \vec{\xi}\in L^2_d(\Tc)\;|\;
     \vec{\xi}(0)=\vec{\xi}(T) 
                        \}, \quad
  \Hs_{\,\Vc}=\{ 
     \vec{\xi}\in L^2_d(\Vc)\;|\;
     \vec{\xi}(0)=\vec{\xi}(\vec{V}) 
                        \}, 
\end{align}
%
where we have defined $\vec{V}=(V_1,\ldots,V_d)$. Denote by $\langle\cdot,\cdot\rangle$ the
sesquilinear inner-product associated with the Hilbert space
$\Hs_{\Tc\Vc}$, which is defined by
$\langle\vec{\xi},\vec{\zeta}\,\rangle=\langle\overline{\vec{\xi}\,}\cdot\vec{\zeta}\,\rangle$  with 
$\langle\vec{\xi},\vec{\zeta}\,\rangle=\overline{\langle\vec{\zeta},\vec{\xi}\,\rangle}$, where $\bar{a}$ 
denotes complex conjugation for $a\in\mathbb{C}$ and
$(\overline{\vec{\xi}})_j=\overline{\xi_j}$, $j=1,\ldots,d$. By the Helmholtz
theorem \cite{Denaro:2003:0271,Bhatia:IEE:1077}, the Hilbert space
$\Hs_{\,\Vc}$ in \eqref{eq:Hilbert_Spaces} can be decomposed into
mutually orthogonal subspaces of curl-free $\Hs_\times$, divergence-free
$\Hs_\bullet$, and constant $\Hs_{\,0}$ vector fields, with associated
orthogonal projectors $\bGamma_\times$, $\bGamma_\bullet$, and $\bGamma_0$,
respectively, 
%$\bGamma_\times=\vec{\nabla}(\bDelta^{-1})\vec{\nabla}\cdot$,
%$\bGamma_\bullet=-\vec{\nabla}\times(\bDelta^{-1})\vec{\nabla}\times$, and  $\bGamma_0=\langle\cdot\rangle$  
%, respectively,
\cite{Fannjiang:SIAM_JAM:333,MILTON:2002:TC}    
%
\begin{align}\label{eq:Helmholtz}
  &\Hs_{\,\Vc}=\Hs_\times\oplus\Hs_\bullet\oplus\Hs_{\,0},\qquad
  \Ib=\bGamma_\times+\bGamma_\bullet+\bGamma_0, \\
  \bGamma_\times&=\vec{\nabla}(\Delta^{-1})\vec{\nabla}\cdot, \quad
  \bGamma_\bullet=-\vec{\nabla}\times(\bDelta^{-1})\vec{\nabla}\times, \quad
  \bGamma_0=\langle\cdot\rangle, \quad
  \notag \\
  \Hs_\times=\{\vec{\xi}\;|\;\vec{\nabla}\times\vec{\xi}=&0 \text{ weakly}\}, \quad
  \Hs_\bullet
      =\{\vec{\xi}\;|\;\vec{\nabla}\cdot\vec{\xi}=0 \text{ weakly}\},   \quad
  \Hs_{\,0}
      =\{\vec{\xi}\;|\;\vec{\xi}=\langle\vec{\xi}\,\rangle\}.
     \notag  
\end{align}
%








We are primarily concerned with fluid velocity fields $\vec{u}$ such
that $0<\Kc^*_{kk}<\infty$ for all $0<\varepsilon<\infty$. Consequently, in view of
equation \eqref{eq:Eff_Diffusivity}, we require that the (weakly)
curl-free vector field $\vec{\nabla}\chi_k$ satisfies
$\vec{\nabla}\chi_k\in\Hs_{\,\Tc}\otimes\Hs_\times\subset\Hs_{\Tc\Vc}$, so that it is
bounded in the norm $\|\cdot\|$ induced by the $\Hs_{\Tc\Vc}$-inner-product
\cite{Folland:99}, $\|\vec{\nabla}\chi_k\|<\infty$. Defining the (weakly)
divergence-free vector field $\vec{J}_k=\bsig\vec{E}_k$ in
\eqref{eq:Maxwells_Equations} as a member of a subset of
$\Hs_{\Tc\Vc}$ is technically difficult, due to the
\emph{unboundedness} of the linear operator
$\bsig=\bkappa-(\bDelta^{-1})\Tb$ on this space. We now explore the
properties of this operator in more detail.  





Since $\Vc$ is a bounded domain, $(\Delta^{-1})$ is a compact operator
\cite{Stakgold:BVP:2000} on the Hilbert space $L^2(\Vc)$. Hence
$(\bDelta^{-1})$ is a compact operator on the Hilbert space
$\Hs_{\,\Vc}$, and is consequently bounded in the operator norm $\|\cdot\|$
induced by the $\Hs_{\Tc\Vc}$-inner-product
\cite{Reed-1980,Stone:64,Stakgold:BVP:2000}, when considered as an 
operator on $\Hs_{\Tc\Vc}$.  We have already assumed 
for the convergence $\phi^\delta\to\bar{\phi}$, as $\delta\to0$, that the flow matrix
$\Hb(t,\vec{x})$ is periodic on $\Tc\times\Vc$. We will also assume that it
is (component-wise) mean-zero and bounded in operator norm, and that
its component-wise time derivative $\Tb\Hb$ is also bounded on
$\Hs_{\Tc\Vc}$ 
%
\begin{align}\label{eq:Bounded_H}
  \langle\Hb\rangle=0, \quad \|\Hb\|<\infty, \quad \|\Tb\Hb\|<\infty.
  %\quad \text{ on } \Hs_{\Tc\Vc}. 
\end{align}
%
This implies that $\bkappa=\varepsilon\Ib+\Hb$ is also bounded for all
$0<\varepsilon<\infty$. Consequently, in the case of a time-independent velocity 
field $\vec{u}$, where $\bsig=\bkappa$, the linear operator $\bsig$ is
bounded. This and $\|\vec{\nabla}\chi_k\|<\infty$ implies that 
$\vec{J}_k\in\Hs_\bullet$. Therefore, in the case of a time-dependent velocity
field, under the assumptions of \eqref{eq:Bounded_H}, the
unboundedness of $\bsig=\bkappa-(\bDelta^{-1})\Tb$ on $\Hs_{\Tc\Vc}$
is due to the unboundedness of $\Tb$ on $\Hs_{\,\Tc}$.   



The unboundedness of $\Tb$ on $\Hs_{\,\Tc}$ can be understood by
considering the orthonormal set of functions
$\{\vec{\psi}_n\}\subset\Hs_{\,\Tc}$ with components $(\vec{\psi}_n)_j$, $j=1,\ldots,d$,
defined by  
%
\begin{align}\label{eq:Orthonormal}
  (\vec{\psi}_n)_j(t)=\beta\sin((n+j)\pi t/T), \quad
  \beta=\sqrt{2/(Td)},
  \qquad
  \langle\vec{\psi}_n\cdot\vec{\psi}_m\rangle=\delta_{nm}, \quad
  n,m\in\mathbb{N}.
\end{align}
%
The components $(\Tb\vec{\psi}_n)_j$, $j=1,\ldots,d$, of the vector
$\Tb\vec{\psi}_n$ and its norm $\|\Tb\vec{\psi}_n\|$ are given by
%$(\partial_t\vec{\psi}_n)_j(t,\vec{x})=b_{nj}\cos((n+j)\pi t/T)$, $b_{nj}=(n+j)a\pi/T$.
%
\begin{align}\label{eq:Orthonormal_Diff}
  (\partial_t\vec{\psi}_n)_j(t)=[\beta(n+j)\pi/T]\cos((n+j)\pi t/T),\quad
  \|\Tb\vec{\psi}_n\|^2%=(T/2)\sum_jb_{nj}^2
               =\frac{1}{d}\sum_j[(n+j)\pi/T]^2.
\end{align}
%
%The $\Hs$-norm of $\partial_t\vec{\psi}_n$ is given by
%$\|\partial_t\vec{\psi}_n\|^2=(T/2)\sum_jb_{nj}^2=\sum_j[(n+j)\pi/T]^2$. 
Therefore, the norm of the members of the set $\{\Tb\vec{\psi}_n\}$
grows arbitrarily large as $n\to\infty$. This clearly demonstrates the
unboundedness of the operator $\Tb$ on $\Hs_{\,\Tc}$.





The above analysis demonstrates that the domain $D(\Tb)$ of the
unbounded operator $\Tb$ is defined only on a proper subset of $\Hs_{\,\Tc}$,
i.e. $D(\Tb)\subset\Hs_{\,\Tc}$. However, $D(\Tb)$ can be defined as a \emph{dense}
subset of $\Hs_{\,\Tc}$ such that $\Tb$ is bounded
\cite{Reed-1980,Stone:64}. Toward this goal, consider the class
$\As_{\Tc}$ of all functions $\xi\in L^2(\Tc)$ such that $\xi(t)$ is
\emph{absolutely continuous} \cite{Royden:1988:RA} on the interval
$\Tc$ and has a derivative $\xi^{\,\prime}(t)$ belonging to $L^2(\Tc)$,
i.e. \cite{Stone:64,Royden:1988:RA}   
%
\begin{align}\label{eq:AC_L2}
  \As_{\Tc}=
     \left\{
       \xi\in L^2(\Tc) \ \Big| \ \xi(t)=c+\int_0^tg(\tau)d\tau,
       \quad  g\in L^2(\Tc)
     \right\},
\end{align}
%
where the constant $c$ and function $g(t)$ are
arbitrary. Now, consider the set $\tilde{\As}_{\Tc}$ of all
functions $\xi\in\As_{\Tc}$ that satisfy the periodic initial condition
$\xi(0)=\xi(T)$, i.e. functions $\xi$ satisfying the properties of 
equation \eqref{eq:AC_L2} with $\int_0^Tg(\tau)d\tau=0$. To illustrate some
important ideas later in this work, we also consider the
set $\hat{\As}_{\Tc}$ of all functions $\xi\in\As_{\Tc}$ that satisfy the
Dirichlet initial condition $\xi(0)=\xi(T)=0$, i.e. functions $\xi$
satisfying the properties of equation \eqref{eq:AC_L2} with $c=0$ and
$\int_0^Tg(\tau)d\tau=0$. More concisely, 
%
\begin{align}\label{eq:AC_BC}
  \tilde{\As}_{\Tc}=\{\xi\in\As_{\Tc} \,|\, \xi(0)=\xi(T)\}, \quad
  \hat{\As}_{\Tc}=\{\xi\in\As_{\Tc} \,|\, \xi(0)=\xi(T)=0\}.
\end{align}
%
These function spaces satisfy
$\hat{\As}_{\Tc}\subset\tilde{\As}_{\Tc}\subset\As_{\Tc}$ and are each everywhere
dense in $L^2(\Tc)$ \cite{Stone:64}. It follows that the function space
$\Fs_{\Tc}=\otimes_{n=1}^d\tilde{\As}_{\Tc}$ is everywhere dense in
$\Hs_{\,\Tc}$ and $\Fs_{\Tc}\otimes\Hs_{\,\Vc}$ is a dense subset of the
Hilbert space $\Hs_{\Tc\Vc}$. Moreover, as a linear operator acting on
the function space $\Fs_{\Tc}$, by construction, $\Tb$ is bounded in
operator norm. We are now in a position to prove Theorem
\ref{thm:kappa_sigma}. 






\textbf{Proof of Theorem \ref{thm:kappa_sigma}.}\hspace{1ex}
%
As a linear operator acting on the function space
$\Fs_{\Tc}\otimes\Hs_{\,\Vc}$, by construction,
$\bsig=\bkappa-(\bDelta^{-1})\Tb$ is bounded in operator norm. Recall
from \eqref{eq:Maxwells_Equations} that $\vec{J}_k=\bsig\vec{E}_k$
with $\vec{E}_k=\vec{\nabla}\chi_k+\vec{e}_k$. It is clear that
$\bsig\vec{e}_k=\bkappa\vec{e}_k$, and is bounded by equation
\eqref{eq:Bounded_H}. Consequently, if
$\vec{\nabla}\chi_k\in\Fs_{\Tc}\otimes\Hs_{\,\Vc}$ then  
$\vec{J}_k$ is Lebesgue measurable and also bounded in norm on 
$\Hs_{\Tc\Vc}$. We have already established that 
$\vec{\nabla}\chi_k\in\Hs_{\,\Tc}\otimes\Hs_\times$. Therefore, this and equation
\eqref{eq:Cell_Problem} suggest that we consider the curl-free,
mean-zero vector field $\vec{\nabla}\chi_k$ as a member of the function space
$\Fs\subset\Fs_{\Tc}\otimes\Hs_{\,\Vc}$,         
%
\begin{align}\label{eq:Function_Space}
  \Fs=\{\vec{\xi}\in\Fs_{\Tc}\otimes\Hs_\times \;|\; \langle\vec{\xi}\,\rangle=0\},  
\end{align}
%
which will be used extensively. We
stress that $\Fs$ is \emph{not} a Hilbert space, and is instead a
dense subset of the Hilbert space $\Hs_{\,\Tc}\otimes\Hs_\times$. We will
henceforth assume that $\vec{\nabla}\chi_k\in\Fs$. In the case of a
time-independent velocity field $\vec{u}$ we set $\Fs_{\Tc}=\emptyset$ in 
\eqref{eq:Function_Space}, so that $\vec{\xi}\in\Fs$ implies 
$\vec{\xi}\in\Hs_\times$ with $\langle\vec{\xi}\,\rangle=0$. To summarize, since $\bsig$ is
bounded on $\Fs$ and $\vec{\nabla}\chi_k\in\Fs$, we have that the
divergence-free vector field $\vec{J}_k=\bsig\vec{E}_k$ is also
bounded $\|\vec{J}_k\|<\infty$, thus $\vec{J}_k\in\Hs_{\Tc}\otimes\Hs_\bullet$.  






By the mutual orthogonality of the Hilbert spaces $\Hs_\times$ and $\Hs_\bullet$
in equation \eqref{eq:Helmholtz}, 
$\vec{\nabla}\chi_k\in\Fs$, $\vec{J}_k\in\Hs_{\Tc}\otimes\Hs_\bullet$, and Fubini's theorem
\cite{Folland:99} imply that $\langle\vec{J}_j\cdot\vec{\nabla}\chi_k\rangle=0$ for every
$j,k=1,\ldots,d$. This is trivially 
satisfied in the case of a time-independent velocity field $\vec{u}$,
since in this case $\bsig=\bkappa$ is bounded so that
$\vec{J}_j\in\Hs_\bullet$ for $\vec{\nabla}\chi_k\in\Fs$. In either case, as
$\vec{E}_k=\vec{\nabla}\chi_k+\vec{e}_k$, we have
$\langle\vec{J}_j\cdot\vec{e}_k\rangle=\langle\vec{J}_j\cdot\vec{E}_k\rangle$. Equations
\eqref{eq:Maxwells_Equations} and \eqref{eq:sigma*} then imply that
the components
$\sigma^*_{jk}=\bsig^*\vec{e}_j\cdot\vec{e}_k=\langle\bsig\vec{E}_j\cdot\vec{e}_k\rangle$ of 
the effective tensor $\bsig^*$ can be expressed as
$\sigma^*_{jk}=\langle\bsig\vec{E}_j\cdot\vec{E}_k\rangle$, with $\bsig=\varepsilon\Ib+\Sb$ and
$\Sb=\Hb-(\bDelta^{-1})\Tb$. Consequently,      
%
\begin{align}\label{eq:Reduction}
  \sigma^*_{jk} %=\langle(\varepsilon\Ib+\Sb)\vec{E}_j\cdot\vec{E}_k\rangle
       =\varepsilon\langle\vec{E}_j\cdot\vec{E}_k\rangle+\langle\Sb\vec{E}_j\cdot\vec{E}_k\rangle.      
\end{align}
%




The property  $\langle\vec{\nabla}\chi_k\rangle=0$ in \eqref{eq:Cell_Problem}, and equation
\eqref{eq:Eff_Diffusivity} together imply that 
%
\begin{align}\label{eq:Reduction_kappa}
  \varepsilon\langle\vec{E}_j\cdot\vec{E}_k\rangle=\varepsilon[\langle\vec{\nabla}\chi_j\cdot\vec{\nabla}\chi_k\rangle
                   +\langle\vec{\nabla}\chi_j\cdot\vec{e}_k\rangle+\langle\vec{e}_j\cdot\vec{\nabla}\chi_k\rangle
                   +\langle\vec{e}_j\cdot\vec{e}_k\rangle]
                   =\varepsilon(\langle\vec{\nabla}\chi_j\cdot\vec{\nabla}\chi_k\rangle+\delta_{jk})
                   =\kappa^*_{jk}. 
\end{align}
%
From the definition of $\Sb=\Hb-(\bDelta^{-1})\Tb$ in equation
\eqref{eq:Eff_Diffusivity} we have that
$\Sb\vec{e}_j=\Hb\vec{e}_j$. Consequently,
$\langle\Sb\vec{e}_j\cdot\vec{e}_k\rangle=\langle\Hb\vec{e}_j\cdot\vec{e}_k\rangle=0$, since by 
equation \eqref{eq:Bounded_H} the matrix $\Hb$ is (component-wise)
mean-zero. Also, by the definition $\vec{u}=\vec{\nabla}\cdot\Hb$ in
\eqref{eq:u_DH} and the periodicity of $\Hb$ and $\chi_k$, we also have
$\langle\Hb\vec{e}_j\cdot\vec{\nabla}\chi_k\rangle=-\langle u_j\chi_k\rangle$ via integration by
parts. Therefore, by the skew-symmetry of $\Sb$ on $\Fs$, the
symmetries $\kappa^*_{kj}=\kappa^*_{jk}$ and $\alpha^*_{kj}=-\alpha^*_{jk}$, and equations
\eqref{eq:Eff_Diffusivity}, \eqref{eq:Eff_Diffusivity_Appendix}, and
\eqref{eq:Functional_Rep}, we have 


%
\begin{align}\label{eq:Reduction_alpha}    
   \langle\Sb\vec{E}_j\cdot\vec{E}_k\rangle&=\langle\Sb\vec{\nabla}\chi_j\cdot\vec{\nabla}\chi_k\rangle
                       +\langle\Sb\vec{\nabla}\chi_j\cdot\vec{e}_k\rangle+\langle\Sb\vec{e}_j\cdot\vec{\nabla}\chi_k\rangle
                       +\langle\Sb\vec{e}_j\cdot\vec{e}_k\rangle
                       \\
                       &=\alpha^*_{jk}-\langle\vec{\nabla}\chi_j\cdot\Hb\vec{e}_k\rangle+\langle\Hb\vec{e}_j\cdot\vec{\nabla}\chi_k\rangle
                       \notag\\
                       &=\alpha^*_{jk}+\langle\chi_ju_k\rangle-\langle u_j\chi_k\rangle
                       \notag\\
                       &=\alpha^*_{jk}+[\alpha^*_{kj}+\kappa^*_{kj}-\varepsilon\delta_{kj}]-[\alpha^*_{jk}+\kappa^*_{jk}-\varepsilon\delta_{jk}]
                       \notag\\
                       &=-\alpha^*_{jk}.
                       \notag
\end{align}
%
In summary, from equations
\eqref{eq:Reduction}--\eqref{eq:Reduction_alpha} and the
symmetries $\kappa^*_{jk}=\kappa^*_{kj}$ and $\alpha^*_{jk}=-\alpha^*_{kj}$ we have that 
%
\begin{align}\label{eq:Reduction_final}
  \sigma^*_{jk}=\kappa^*_{jk}-\alpha^*_{jk}=\kappa^*_{kj}+\alpha^*_{kj}=\Kc_{kj}^*\,,       
\end{align}
%
which is equivalent to equation \eqref{eq:Eff_Equiv}. This concludes
our proof of Theorem \ref{thm:kappa_sigma} $\Box$.     




We conclude this section with a derivation of the following resolvent
formula for $\vec{\nabla}\chi_k$, involving the orthogonal projection
operator $\bGamma_\times=\vec{\nabla}(\Delta^{-1})\vec{\nabla}\cdot$ onto curl-free fields in
\eqref{eq:Helmholtz},  
% % 
\begin{align}\label{eq:Resolvent_Rep}
  \vec{\nabla}\chi_j=(\varepsilon\Ib+\Ab)^{-1}\vec{g}_j
           =(\varepsilon\Ib+\I\Mb)^{-1}\vec{g}_j, \quad
  \Ab=\bGamma\Sb\bGamma, \quad
  \Mb=-\I\Ab, \quad
  \vec{g}_j=-\bGamma\Hb\vec{e}_j,
\end{align}
%
where $\I=\sqrt{-1}\,$ and we have defined $\bGamma=\bGamma_\times$ for
notational simplicity. Equation \eqref{eq:Resolvent_Rep} follows from
applying the integro-differential operator $\vec{\nabla}(\Delta^{-1})$ to
$\vec{\nabla}\cdot\bsig\vec{E}_j=0$ in equation \eqref{eq:Maxwells_Equations},
with $\vec{E}_j=\vec{\nabla}\chi_j+\vec{e}_j$ and $\bsig=\varepsilon\Ib+\Sb$, yielding  
%
\begin{align}\label{eq:Pre_Resolvent}
  \bGamma(\varepsilon\Ib+\Sb)\vec{\nabla}\chi_j=-\bGamma\Hb\vec{e}_j,
\end{align}
%
since $\bGamma\vec{e}_j=0$ and $\Sb\vec{e}_j=\Hb\vec{e}_j$.
The equivalence of equations \eqref{eq:Resolvent_Rep} and
\eqref{eq:Pre_Resolvent} can be seen by noting that
$\vec{\nabla}\chi_j\in\Fs$ implies $\bGamma\vec{\nabla}\chi_j=\vec{\nabla}\chi_j$. We stress
that the property $\bGamma\vec{\xi}=\vec{\xi}$ for $\vec{\xi}\in\Fs$ implies
that $\Ab=\bGamma\Sb\bGamma=\bGamma\Sb$ on $\Fs$.



It is worth mentioning that taking the dot-product of both sides of
equation \eqref{eq:Pre_Resolvent} with $\vec{\nabla}\chi_k$, averaging, using
the properties $\bGamma\vec{\nabla}\chi_j=\vec{\nabla}\chi_j$ and
$\langle\bGamma\vec{\xi}\cdot\vec{\zeta}\,\rangle=\langle\vec{\xi}\cdot\bGamma\vec{\zeta}\,\rangle$ for
$\vec{\xi},\vec{\zeta}\in\Hs_{\,\Vc}$, and integrating by parts, yields
equation  \eqref{eq:Functional_Rep}. Moreover, the condition
$\langle\vec{J}_j\cdot\vec{\nabla}\chi_k\rangle=0$ is also equivalent to equation
\eqref{eq:Functional_Rep}.   



In Section \ref{sec:Symmetries_Commute} we show that $\Ab$ in
\eqref{eq:Resolvent_Rep} acts as an anti-symmetric linear operator on
the Hilbert space $\Hs_{\Tc\Vc}$,
$\langle\Ab\vec{\xi}\cdot\vec{\zeta}\rangle=\langle\vec{\xi}\cdot\Ab^*\vec{\zeta}\rangle=-\langle\vec{\xi}\cdot\Ab\vec{\zeta}\rangle$. Therefore,
$\Ab$ commutes with its (Hilbert space) adjoint $\Ab^*=-\Ab$ (not to
be confused with an effective tensor) and is therefore an example of a
\emph{normal} operator \cite{Stone:64}. Consequently, due to the
sesquilinearity of the $\Hs_{\Tc\Vc}$-inner-product, $\Mb=-\I\Ab$ acts
as a \emph{symmetric} operator, $\Mb^*=\Mb$
\cite{Reed-1980,Stone:64}. Moreover, on the function space $\Fs$,
$\Ab$ is a \emph{maximal} normal operator and $\Mb$ is
\emph{self-adjoint} \cite{Stone:64}. In Section \ref{sec:Integral_Rep}
we examine these properties of $\Ab$ and $\Mb$ in more detail and
demonstrate how equation \eqref{eq:Resolvent_Rep} and the spectral
theory of such operators lead to integral representations for the
symmetric $\bkappa^*$ and anti-symmetric $\balpha^*$ parts of $\Kbc^*$.      








\subsection{Integral representation of the effective diffusivity
  for steady and dynamic flows}\label{sec:Integral_Rep}
%
In this section, we employ the Hilbert space formulation of the
effective parameter problem discussed in Section
\ref{sec:Hilbert_Space} above, to provide integral representations for the
symmetric $\bkappa^*$ and anti-symmetric $\balpha^*$ parts of the
effective diffusivity tensor $\Kbc^*$. In the general (infinite
dimensional) setting, these integral representations involve a
\emph{spectral measure} $\d\bmu$ associated with the (maximal) normal
operator $\Ab=\bGamma\Sb\bGamma$ on $\Fs$, or equivalently the
self-adjoint operator $\Mb=-\I\Ab$, and follow from the spectral
theorem for such linear operators \cite{Reed-1980,Stone:64} and the resolvent
formula for $\vec{\nabla}\chi_k$ given in equation
\eqref{eq:Resolvent_Rep}. The derivation of these integral
representations for $\bkappa^*$ and $\balpha^*$ is the topic of
Section \ref{sec:Integral_Rep_General}.




In Section \ref{sec:Integral_Rep_Matrix} we discuss this mathematical
framework in the finite dimensional setting, where $\Ab$ is given by
an anti-symmetric matrix. The spectral analysis of $\Ab$ illuminates a
great deal of structure regarding the spectral measure $\d\bmu$ in
this matrix setting. This structure is utalized in Section
\ref{sec:Num_Results} to formulate an efficient and stable numerical
algorithm for the explicit computation of $\bkappa^*$ and $\balpha^*$
for model velocity fields $\vec{u}$, by the direct computation of
$\d\bmu$ in terms of the eigenvalues and eigenvectors of $\Ab$.   



\subsubsection{General infinite dimensional setting}\label{sec:Integral_Rep_General}
%
In the general Hilbert space setting, there are significant
differences in the theory between the case of steady flows, where
$\Sb=\Hb$ is \emph{bounded} on the Hilbert space $\Hs_{\,\Vc}$, and
the case of dynamic flows, where $\Sb=\Hb-(\bDelta^{-1})\Tb$ is
\emph{unbounded} on the Hilbert space $\Hs_{\Tc\Vc}$, as discussed in
Section \ref{sec:Hilbert_Space}. It is therefore natural to start our
discussion with a more detailed look into this distinction, in the
present context. Since $\bGamma$ is an orthogonal projector from
$\Hs_{\,\Vc}$ to $\Hs_\times$, it is bounded by unity in operator norm
$\|\bGamma\|\leq1$ on $\Hs_{\,\Vc}$ and $\|\bGamma\|=1$ on $\Hs_\times$
\cite{Reed-1980,Stone:64}. Therefore by \eqref{eq:Bounded_H}, in the
case of  steady flows, the operator $\Ab=\bGamma\Hb\bGamma$ is bounded
on the Hilbert space $\Hs_{\,\Vc}$, with $\|\Ab\|\leq\|\Hb\|<\infty$. Let's first
focus on this time-independent case. Since $\Mb=-\I\Ab$ we have 
$\|\Mb\|=\|\Ab\|$, so the domains of these two operators are identical,
$D(\Mb)=D(\Ab)$. For simplicity we focus on the operator $\Mb$ now,
re-introducing the operator $\Ab$ later. The (Hilbert space)
adjoint $\Mb^*$ of $\Mb$ is defined by
$\langle\Mb\vec{\xi},\vec{\zeta}\,\rangle=\langle\vec{\xi},\Mb^*\vec{\zeta}\,\rangle$, and is also a
bounded operator on $\Hs_{\,\Vc}$ with $\|\Mb^*\|=\|\Mb\|$
\cite{Reed-1980}. Consequently, they have identical domains,       
%
\begin{align}\label{eq:Domain_M}
  D(\Mb)=D(\Mb^*),
\end{align}
%
which are the entire space, $D(\Mb)=D(\Mb^*)=\Hs_{\,\Vc}$. In
Section \ref{sec:Symmetries_Commute} we show that $\Mb$ is symmetric,
%
\begin{align}\label{eq:Symmetric_M}
  \langle\Mb\vec{\xi}\cdot\vec{\zeta}\,\rangle=\langle\vec{\xi}\cdot\Mb\vec{\zeta}\,\rangle,
  \, \text{ for all } \; \vec{\xi},\vec{\zeta}\in D(\Mb).
\end{align}
%
By definition \cite{Reed-1980,Stone:64}, the two properties
\eqref{eq:Domain_M} and \eqref{eq:Symmetric_M} together imply that the
operator $\Mb$ is \emph{self-adjoint}, i.e. $\Mb\equiv\Mb^*$ on $D(\Mb)$.





Conversely, the Hellinger--Toeplitz theorem \cite{Reed-1980} states,
if the operator $\Mb$ satisfies equation \eqref{eq:Symmetric_M} for
\emph{every} $\vec{\xi},\vec{\zeta}\in\Hs_{\,\Vc}$, then $\Mb$ is
bounded on $\Hs_{\,\Vc}$. This suggests that, in the time-dependent
case when $\Mb$ is unbounded on the Hilbert space $\Hs_{\Tc\Vc}$, it
is defined as a self-adjoint operator only on a proper subset of
$\Hs_{\Tc\Vc}$. However, as discussed in 
Section \ref{sec:Hilbert_Space}, the domain $D(\Mb)$ can be defined as
a \emph{dense} subset of $\Hs_{\Tc\Vc}$ such that $\Mb$ is
bounded. Moreover, on this domain, $\Mb$ can be extended to a 
\emph{closed} symmetric operator \cite{Reed-1980,Stone:64}. Although
even in this case, in general \cite{Reed-1980}, the domain $D(\Mb^*)$
of the associated adjoint $\Mb^*$ does not coincide with $D(\Mb)$, and
in such circumstances $\Mb$ is \emph{not} self-adjoint on $D(\Mb)$. Only for
self-adjoint (or maximal normal) operators does the spectral theorem hold
\cite{Reed-1980}, which provides the existence of the promised
integral representation for $\Kbc^*$, involving a spectral measure
associated with $\Mb$. It is therefore necessary that we find a
domain $D(\Mb)$ on which $\Mb$ is self-adjoint.




As $\bGamma$ is bounded on $\Hs_{\,\Vc}$ and
$\Mb=-\I\bGamma\Sb\bGamma$, our discussion in Section
\ref{sec:Hilbert_Space} indicates that the unboundedness of $\Mb$ on
$\Hs_{\Tc\Vc}$ is due to the unboundedness of the underlying operator
$\Tb$ on the Hilbert space $\Hs_{\,\Tc}$. It is therefore necessary
that we find a domain $D(\Tb)$ for which $\I\Tb$ is a self-adjoint
operator. Toward this goal, and to illustrate these ideas, we consider
the operator $\I\partial_t$ with the three different domains $\As_{\Tc}$,
$\tilde{\As}_{\Tc}$, and $\hat{\As}_{\Tc}$ defined in equations
\eqref{eq:AC_L2} and \eqref{eq:AC_BC}, which are everywhere dense in
$L^2(\Tc)$ \cite{Stone:64}. Let the operators $B$, $\tilde{B}$, and
$\hat{B}$ be identified as $\I\partial_t$ with domains $\As_{\Tc}$,
$\tilde{\As}_{\Tc}$, and $\hat{\As}_{\Tc}$, respectively. Then,
$\hat{B}$ is a closed linear symmetric operator with adjoint
$\hat{B}^*\equiv B$, and the operator $\tilde{B}$ is a \emph{self-adjoint}
extension of $\hat{B}$ \cite{Stone:64}. In symbols, this means that 
$\tilde{B}=\tilde{B}^*$ on $\tilde{\As}_{\Tc}$ and
$D(\tilde{B})=D(\tilde{B}^*)=\tilde{\As}_{\Tc}$,
i.e. $\tilde{B}\equiv\tilde{B}^*$ on $\tilde{\As}_{\Tc}$.     



Since the operator $\tilde{B}=\I\partial_t$ with domain $\tilde{\As}_{\Tc}$ is
self-adjoint, it follows that the operator $\I\Tb=\I\partial_t\Ib$ with domain
$D(\Tb)=\Fs_{\Tc}=\otimes_{n=1}^d\tilde{\As}_{\Tc}$ is self-adjoint. This is
seen as follows. By noting that
$\I\Tb\vec{\xi}=(\tilde{B}\xi_1,\ldots,\tilde{B}\xi_d)$ and, for all 
$\vec{\xi},\vec{\zeta}\in\Fs_{\Tc}$ with components
$\xi_j,\zeta_j\in\tilde{\As}_{\Tc}$, $j=1,\ldots,d$, the self-adjointness of 
$\tilde{B}$ implies that $\Tb$ is symmetric, $\Tb=\Tb^*$, on $\Fs_{\Tc}$, 
%
\begin{align}\label{eq:T_symmetric}
  \langle\Tb\vec{\xi}\cdot\vec{\zeta}\rangle=\sum_j\langle\tilde{B}\xi_j,\zeta_j\rangle_2
                    =\sum_j\langle\xi_j,\tilde{B}\zeta_j\rangle_2
                    =\langle\vec{\xi}\cdot\Tb\vec{\zeta}\rangle,
\end{align}
%
where $\langle\cdot,\cdot\rangle_2$ denotes the $L^2(\Tc)$-inner-product.  Moreover, since we have
$D(\tilde{B})=D(\tilde{B}^*)=\tilde{\As}_{\Tc}$, we also have
$D(\Tb)=D(\Tb^*)=\Fs_{\Tc}$, i.e. $\Tb\equiv\Tb^*$ on
$\Fs_{\Tc}$. Consequently, $\Tb$ is a bounded self-adjoint linear
operator on the function space $\Fs_{\Tc}$.



We now summarize what we have discussed so far, and
discuss the implications thereof. We have discussed that the 
operators $(\bDelta^{-1})$ and $\bGamma$ are bounded on the Hilbert
space $\Hs_{\,\Vc}$. In Section \ref{sec:Symmetries_Commute} we show
that they are also symmetric, hence self-adjoint on $\Hs_{\,\Vc}$. Due
to the sesquilinearity of the $\Hs_{\Tc\Vc}$-inner-product, and
equations \eqref{eq:Bounded_H} and \eqref{eq:u_DH} with $\Hb^*=\Hb^T$,
the  operator $\I\Hb$ is bounded and symmetric, hence self-adjoint on
the Hilbert space $\Hs_{\Tc\Vc}$.  Consequently, the operator
$\I\bGamma\Hb\bGamma$ is also self-adjoint on $\Hs_{\Tc\Vc}$. The
differential and integral operators $\I\Tb$ and $(\bDelta^{-1})$ are 
bounded on the function space $\Fs_{\Tc}$ and Hilbert space $\Hs_{\,\Vc}$,
respectively, and they are consequently commutable operations on the
function space $\Fs_{\Tc}\times\Hs_{\,\Vc}$ \cite{Folland:99}. Moreover, as
$\I\Tb$ and $(\bDelta^{-1})$ are self-adjoint on
$\Fs_{\Tc}$ and $\Hs_{\,\Vc}$, respectively, the operator 
$\I(\bDelta^{-1})\Tb$, hence $\I\bGamma[(\bDelta^{-1})\Tb]\bGamma$ is
self-adjoint on $\Fs_{\Tc}\times\Hs_{\,\Vc}$. It is now clear that the operator
$\Mb=\I\bGamma\Sb\bGamma$, with $\Sb=\Hb-(\bDelta^{-1})\Tb$, is
self-adjoint on $\Fs_{\Tc}\times\Hs_{\,\Vc}$. Finally, since $\Mb=-\I\Ab$ is
self-adjoint on $\Fs_{\Tc}\times\Hs_{\,\Vc}$ and an operator is
self-adjoint if and only if it is a maximal normal operator
\cite{Stone:64}, we have that $\Ab$ is a maximal normal
operator on $\Fs_{\Tc}\times\Hs_{\,\Vc}$. In view of the resolvent
formulas for $\vec{\nabla}\chi_j\in\Fs$ in \eqref{eq:Resolvent_Rep}
involving $\Mb$ and $\Ab$, we will henceforth take the domain of
these operators to be $D(\Ab)=D(\Mb)=\Fs$ in
\eqref{eq:Function_Space}, which is a \emph{closed} subset of
$\Fs_{\Tc}\times\Hs_{\,\Vc}$.   





In terms of a general, maximal normal operator $\Nb$ on $\Fs$
satisfying $\Nb\Nb^*=\Nb^*\Nb$, the spectral theorem states that $\Nb$ can
be decomposed as $\Nb=\Hb_1+\I\Hb_2$, where $\Hb_1$ and $\Hb_2$ are
self-adjoint and commute on $\Fs$ \cite{Stone:64}. Moreover, there is
a one-to-one correspondence between $\Hb_n$, $n=1,2$, and a family
$\{\Qb_n(\lambda)\}$, $-\infty<\lambda<\infty$, of self-adjoint projection operators - the
resolution of the identity - with domain $\Fs$ which satisfies
$\lim_{\lambda\to-\infty}\Qb_n(\lambda)=0$, $\lim_{\lambda\to+\infty}\Qb_n(\lambda)=\Ib$, and the
$\Qb_n(\lambda)$, $n=1,2$, commute \cite{Reed-1980,Stone:64}. Consequently,
there is a one-to-one correspondence between $\Nb$ and a family
$\{\Qb(z)\}$, 
$\Qb(z)=\Qb_1(\text{Re}(z))\Qb_2(\text{Im}(z))$, $z=\lambda_1+\I\lambda_2$,
$-\infty<\lambda_1,\lambda_2<\infty$, of self-adjoint projection operators - the
\emph{complex} resolution of the identity - which satisfies $\Qb(z)\to0$
when $\text{Re}(z)\to-\infty$  and when $\text{Im}(z)\to-\infty$, and $\Qb(z)\to\Ib$
when $\text{Re}(z)\to+\infty$ and when $\text{Im}(z)\to+\infty$ \cite{Stone:64}.



The spectral theorem also provides an operational calculus in Hilbert
space which yields integral representations associated with
$\Qb(z)$-measurable functions of $\Nb$ \cite{Stone:64}. The details
are as follows. Let $\vec{\xi},\vec{\zeta}\in\Fs$ and consider the
\emph{complex-valued} function $\mu_{\xi\zeta}(z)=\langle\Qb(z)\vec{\xi}\cdot\vec{\zeta}\,\rangle$,
$\vec{\xi}\neq\vec{\zeta}$. By the sesquilinearity of the inner-product and the
self-adjointness of the projection operator $\Qb(z)$ we have
$\mu_{\zeta\xi}(z)=\overline{\mu}_{\xi\zeta}(z)$, where $\overline{\mu}_{\xi\zeta}$ denotes
the complex conjugate of $\mu_{\xi\zeta}$. Moreover, the function $\mu_{\xi\xi}$ is
real-valued and positive
$\mu_{\xi\xi}(z)=\langle\Qb(z)\vec{\xi}\cdot\vec{\xi}\,\rangle=\langle\Qb(z)\vec{\xi}\cdot\Qb(z)\vec{\xi}\,\rangle
=\|\Qb(z)\vec{\xi}\,\|^2$. We associate with these functions of
\emph{bounded variation} Radon--Stieltjes measures $\d\mu_{\xi\zeta}(z)$ and
$\d\mu_{\xi\xi}(z)$ \cite{Stone:64}   
%
\begin{align}\label{eq:Bounded_Variation}
  \d\mu_{\xi\zeta}(z)=\d\langle\Qb(z)\vec{\xi}\cdot\vec{\zeta}\,\rangle, \quad
  \vec{\xi}\neq\vec{\zeta}, \qquad
  \d\mu_{\xi\xi}(z)=\d\|\Qb(z)\vec{\xi}\,\|^2.
\end{align}
%
Let $F(z)$ be an arbitrary complex-valued function and denote by
$\mathscr{D}(F)$ the set of all $\vec{\xi}\in\Fs$ such that
$F\in L^2(\mu_{\xi\xi})$, i.e. $F(z)$ is square integrable with respect to the
measure $\d\mu_{\xi\xi}$. 
Then $\mathscr{D}(F)$ is a linear manifold and there exists a linear
transformation $F(\Nb)$ with domain $\mathscr{D}(F)$ defined in terms
of the Radon--Stieltjes integrals \cite{Stone:64} 
%
\begin{align}\label{eq:Spectral_Theorem}
  \langle F(\Nb)\vec{\xi}\cdot\vec{\zeta}\,\rangle=\int_IF(z)\,\d\mu_{\xi\zeta}(z), \qquad
  &\forall \, \vec{\xi}\in\mathscr{D}(F), \ \vec{\zeta}\in\Fs
  \\
  \langle F(\Nb)\vec{\xi}\cdot G(\Nb)\vec{\zeta}\,\rangle=\int_I\overline{F}(z)G(z)\,\d\mu_{\xi\zeta}(z),
  \quad
  &\forall \, \vec{\xi}\in\mathscr{D}(F), \ \vec{\zeta}\in\mathscr{D}(G),
  \notag
\end{align}
%
where the operator $G(\Nb)$ and function space $\mathscr{D}(G)$ are
defined analogously to that for $F$. An integral representation for
the functional $\|F(\Nb)\vec{\xi}\,\|^2$ follows from the second formula
in \eqref{eq:Spectral_Theorem} with $G=F$ and $\vec{\xi}=\vec{\zeta}$, and
involves the measure $\d\mu_{\xi\xi}$ in \eqref{eq:Bounded_Variation}
\cite{Stone:64}.  The domain of integration $I$ in
\eqref{eq:Spectral_Theorem} is the \emph{spectrum} $\Sigma(\Nb)$ of the
operator $\Nb$, $I\equiv\Sigma(\Nb)$. Since $\Nb$ is a normal operator, its norm 
$\|\Nb\|$ coincides with the spectral radius
$\|\Nb\|=\text{sup}\{|z|: z\in\Sigma(\Nb)$, so that in general
$I\subseteq(-\infty,\infty)\times(-\I\infty,\I\infty)$  \cite{Reed-1980,Stone:64}.  We will discuss the
properties of $\Sigma(\Nb)$ in detail in Section \ref{sec:Assymptotics}.  



 

The spectral theorem of equation \eqref{eq:Spectral_Theorem} for the
maximal normal operator $\Nb$ on $\Fs$ generalizes that for
self-adjoint and maximal anti-symmetric operators, with purely real
and imaginary spectrum, respectively. More specifically, the case
$F(z)=z=\lambda_1+\I\lambda_2$ corresponds to $F(\Nb)=\Hb_1+\I\Hb_2$ with
$I\subseteq(-\infty,\infty)\times(-\I\infty,\I\infty)$ and
$\Qb(z)=\Qb_1(\text{Re}(z))\Qb_2(\text{Im}(z))$, the case
$F(z)=\text{Re}(z)$ corresponds to the self-adjoint operator   
$F(\Nb)=\Hb_1$ with $I\subseteq(-\infty,\infty)$ and $\Qb(z)=\Qb_1(\text{Re}(z))$,
and the case $F(z)=\I\,\text{Im}(z)$ corresponds to the maximal
anti-symmetric operator $F(\Nb)=\I\Hb_2$ with $I\subseteq(-\I\infty,\I\infty)$ and
$\Qb(z)=\Qb_2(\text{Im}(z))$ \cite{Stone:64}. We now apply the
spectral theorem to equations \eqref{eq:Eff_Diffusivity} and
\eqref{eq:Resolvent_Rep} to provide Radon--Stieltjes integral
representations for the symmetric $\bkappa^*$ and anti-symmetric
$\balpha^*$ parts of the effective diffusivity tensor $\Kbc^*$, for
both cases of time-independent and time-dependent velocity fields
$\vec{u}$. These representations are summarized by the following
theorem.  
%
\begin{theorem}\label{thm:Integral_Reps}
%
Let $z=\I\lambda$, $\vec{g}_j=-\bGamma\Hb\vec{e}_j$ be defined as in
\eqref{eq:Resolvent_Rep}, and $\Qb(z)=\Qb_2({\rm Im}(z))=\Qb_2(\lambda)$ be 
the complex resolution of the identity associated with the maximal
anti-symmetric operator $\Ab$ defined in \eqref{eq:Resolvent_Rep}, with
domain $\Fs$ defined in \eqref{eq:Function_Space}. Define the
matrix-valued function $\bmu(\lambda)$ with complex-valued off-diagonal
components $\mu_{jk}(\lambda)=\langle\Qb_2(\lambda)\vec{g}_j\cdot\vec{g}_k\rangle$ for $j\neq k=1,\ldots,d$,
with $\mu_{kj}=\overline{\mu}_{jk}$, and positive diagonal components
$\mu_{kk}(\lambda)=\|\Qb_2(\lambda)\vec{g}_k\|^2$.  Moreover, consider the real-valued
functions   
%
\begin{align}\label{eq:Fns_Bounded_Var}
  {\rm Re}\,\mu_{jk}(\lambda)
         =\frac{1}{2}\left(\mu_{jk}(\lambda)+\overline{\mu}_{jk}(\lambda)\right), \quad
  {\rm Im}\,\mu_{jk}(\lambda)
         =\frac{1}{2\,\I}\left(\mu_{jk}(\lambda)-\overline{\mu}_{jk}(\lambda)\right).
\end{align}
%
Corresponding to each of these functions of bounded variation, 
consider the associated 
Radon--Stieltjes measures $\d\mu_{jk}(\lambda)$, $\d\mu_{kk}(\lambda)$, $\d{\rm
  Re}\,\mu_{jk}(\lambda)$, and $\d{\rm Im}\,\mu_{jk}(\lambda)$. Then, for all $0<\varepsilon<\infty$,
there exist Radon--Stieltjes integral representations for the
components $\kappa^*_{jk}$ and $\alpha^*_{jk}$, $j,k=1,\ldots,d$, of the effective
tensors $\bkappa^*$ and $\balpha^*$ defined in equation
\eqref{eq:Eff_Diffusivity}, given by          
%
\begin{align}\label{eq:Integral_Rep_kappa*}
  \kappa^*_{jk}=\varepsilon\left(\delta_{jk}+\int_{-\infty}^\infty\frac{\d{\rm Re}\,\mu_{jk}(\lambda)}{\varepsilon^2+\lambda^2}\right),
  %\kappa^*_{kk}=\varepsilon\left(1+\int_{-\infty}^\infty\frac{\d\mu_{kk}(\lambda)}{\varepsilon^2+\lambda^2}\right),
  %\quad
  %\kappa^*_{jk}=\varepsilon\int_{-\infty}^\infty\frac{\d{\rm Re}\,\mu_{jk}(\lambda)}{\varepsilon^2+\lambda^2}\,,
  \qquad
  \alpha^*_{jk}=\int_{-\infty}^\infty\frac{\lambda\,\d{\rm Im}\,\mu_{jk}(\lambda)}{\varepsilon^2+\lambda^2}\,.       
  %\quad  j\neq k=1,\ldots,d,
\end{align}
%
Here, the domain of integration $I$ is determined by the spectrum
$\Sigma(\Ab)$ of the operator $\Ab$, where $I\subseteq[-\|\Ab\|,\|\Ab\|]$ and
$\|\Ab\|\leq\|\Hb\|<\infty$ in the case of a time-independent velocity field
$\vec{u}$ \cite{Reed-1980}.     
%
\end{theorem}

\textbf{Proof of Theorem \ref{thm:Integral_Reps}.}\hspace{1ex}
%
We first note that from $\vec{\nabla}\chi_k\in\Fs$ we have 
$\vec{\nabla}\chi_k=\bGamma\vec{\nabla}\chi_k$, so that $\alpha^*_{jk}$ in equation
\eqref{eq:Eff_Diffusivity} can re-expressed as
$\alpha^*_{jk}=\langle\Sb\vec{\nabla}\chi_j\cdot\vec{\nabla}\chi_k\rangle=\langle\bGamma\Sb\bGamma\vec{\nabla}\chi_j\cdot\vec{\nabla}\chi_k\rangle  
=\langle\Ab\vec{\nabla}\chi_j\cdot\vec{\nabla}\chi_k\rangle$, where we have used that $\bGamma$ is
self-adjoint on $\Fs$. From this and \eqref{eq:Resolvent_Rep},
equation \eqref{eq:Eff_Diffusivity} can be rewritten as
%
\begin{align}\label{eq:Eff_Diff_Resolvent}
 \kappa^*_{jk}=\varepsilon\left(\delta_{jk}+\langle(\varepsilon\Ib+\Ab)^{-1}\vec{g}_j\cdot(\varepsilon\Ib+\Ab)^{-1}\vec{g}_k\rangle\right), \quad
 \alpha^*_{jk}=\langle\Ab(\varepsilon\Ib+\Ab)^{-1}\vec{g}_j\cdot(\varepsilon\Ib+\Ab)^{-1}\vec{g}_k\rangle,
% \quad
% \vec{g}_j=-\bGamma\Hb\vec{e}_j
\end{align}
%
where $\vec{g}_k=-\bGamma\Hb\vec{e}_k$. The integral representations
for $\kappa^*_{jk}$ and $\alpha^*_{jk}$ in \eqref{eq:Integral_Rep_kappa*} follow
from equations \eqref{eq:Spectral_Theorem} and
\eqref{eq:Eff_Diff_Resolvent}, and the symmetries
$\langle\vec{\nabla}\chi_j\cdot\vec{\nabla}\chi_k\rangle=\langle\vec{\nabla}\chi_k\cdot\vec{\nabla}\chi_j\rangle$ and 
$\langle\Ab\vec{\nabla}\chi_j\cdot\vec{\nabla}\chi_k\rangle=\langle\vec{\nabla}\chi_k\cdot\Ab\vec{\nabla}\chi_j\rangle$, since
$\vec{\nabla}\chi_k$ and $\Ab\vec{\nabla}\chi_k$ are real-valued. We prove the
validity of \eqref{eq:Integral_Rep_kappa*} by showing that the
conditions of the spectral theorem of equation
\eqref{eq:Spectral_Theorem} are satisfied for the functionals in
\eqref{eq:Eff_Diff_Resolvent} and then employing these symmetries.    





We first show that $\vec{g}_k\in\Fs$ for
all $k=1,\ldots,d$. Indeed, the orthogonality of the
projection operators $\bGamma_\times=\bGamma$ and $\bGamma_0$ defined in
equation \eqref{eq:Helmholtz} implies that the vector field
$\vec{g}_k(t,\cdot)=\bGamma\Hb(t,\cdot)\vec{e}_k$ is curl-free and mean-zero
for each $t\in\Tc$ fixed, and by equation \eqref{eq:Bounded_H} 
we have $\|\vec{g}_k\|\leq\|\Hb\|<\infty$. This and the periodicity of $\Hb$ implies
that $\vec{g}_k(t,\cdot)\in\Hs_\times$, and by Fubini's theorem \cite{Folland:99}
we have $\langle\vec{g}_k\rangle=0$.  By the uniform boundedness of $\bGamma$ on
$\Hs_{\,\Vc}$ and equation \eqref{eq:Bounded_H}, we also have
\cite{Folland:99} that
$\|\Tb\vec{g}_k\|=\|\Tb\bGamma\Hb\vec{e}_k\|=\|\bGamma\Tb\Hb\vec{e}_k\|\leq\|\Tb\Hb\|<\infty$.   
Therefore $\vec{g}_k(\cdot,\vec{x}),\Tb\vec{g}_k(\cdot,\vec{x})\in\Hs_{\,\Tc}$ for each
$\vec{x}\in\Vc$ fixed, which implies that
$\vec{g}_k(\cdot,\vec{x})\in\Fs_{\Tc}$. Consequently,  $\vec{g}_k\in\Fs$ for
all $k=1,\ldots,d$.



Consider the representation for $\kappa^*_{jk}$ in
\eqref{eq:Eff_Diff_Resolvent} and define the function
$F(z)=(\varepsilon+z)^{-1}$ so that, formally,
$\kappa^*_{jk}=\varepsilon(\delta_{jk}+\langle F(\Ab)\vec{g}_j\cdot F(\Ab)\vec{g}_k\rangle)$. Since
$\vec{g}_k\in\Fs$ for all $k=1,\ldots,d$, once we establish that
$\vec{g}_k\in\Ds(F)$, i.e. $F\in L^2(\mu_{kk})$, the integral representations
for $\kappa^*_{jk}$, $j,k=1,\ldots,d$, in \eqref{eq:Integral_Rep_kappa*} follow
from the second formula in \eqref{eq:Spectral_Theorem} with
$F(z)=G(z)=(\varepsilon+z)^{-1}$, $\vec{\xi}=\vec{g}_j$, and
$\vec{\zeta}=\vec{g}_k$. Since $0<\varepsilon<\infty$ and 
$z\in(-\I\infty,\I\infty)$ for the anti-symmetric operator $\Ab$, the function
$|F(z)|^2=|\varepsilon+z|^{-2}$ is bounded, and the validity of $F\in L^2(\mu_{kk})$ is
an immediate consequence of the boundedness of the (positive) measure
mass $\mu^0_{kk}=\int\d\mu_{kk}(z)<\infty$. The validity of $\mu^0_{kk}<\infty$, in turn,
is a consequence of the fact that the function
$\mu_{jk}(z)=\langle\Qb(z)\vec{g}_j\cdot\vec{g}_k\rangle$ is of \emph{bounded
  variation} when $\vec{g}_j,\vec{g}_k\in\Fs$, hence
$|\mu^0_{jk}|<\infty$ for all $j,k=1,\ldots,d$ \cite{Stone:64}. We have therefore
established that $\vec{g}_k\in\Ds(F)$ for all $k=1,\ldots,d$.     




Before we employ the symmetry
$\langle\vec{\nabla}\chi_j\cdot\vec{\nabla}\chi_k\rangle=\langle\vec{\nabla}\chi_k\cdot\vec{\nabla}\chi_j\rangle$ to derive the 
integral representations for $\kappa^*_{kk}$ and $\kappa^*_{jk}$ in
\eqref{eq:Integral_Rep_kappa*},  we note that the condition
$\mu^0_{kk}<\infty$ implies that $\vec{g}_k\in\Ds(F)$ for the function 
$F(z)=1$. This leads to an explicit representation of the mass
$\mu^0_{jk}$ in terms of $\bGamma$ and $\Hb$, and provides a bound for
$|\mu^0_{jk}|$. Indeed, taking $F(z)=1$ $(F(\Ab)=\Ib)$, 
$\vec{\xi}=\vec{g}_j$, and $\vec{\zeta}=\vec{g}_k$ in the first formula of
equation \eqref{eq:Spectral_Theorem}, the self-adjointness of $\bGamma$
and $\bGamma^{\,2}=\bGamma$ on $\Fs$ implies that   
%
\begin{align}\label{eq:Mass}
  \mu^0_{jk}=\int_I\d\mu_{jk}(z)
        =\int_I\d\langle\Qb(z)\vec{g}_j,\vec{g}_k\rangle
        =\langle\vec{g}_j,\vec{g}_k\rangle
        =\langle\bGamma\Hb\vec{e}_j\cdot\bGamma\Hb\vec{e}_k\rangle 
        =\langle\Hb^T\bGamma\Hb\vec{e}_j\cdot\vec{e}_k\rangle.     
\end{align}
%
This and equation \eqref{eq:Bounded_H} imply that
$|\mu^0_{jk}|\leq\|\Hb\|^2<\infty$ for all $j,k=1,\ldots,d$.  





We now derive the integral representations for $\kappa^*_{kk}$ and
$\kappa^*_{jk}$, $j\neq k=1,\ldots,d$, displayed in
\eqref{eq:Integral_Rep_kappa*}. For the function $F(z)=(\varepsilon+z)^{-1}$, we
have established above that $\vec{g}_k\in\Ds(F)$ for all
$k=1,\ldots,d$, so that
$\kappa^*_{jk}=\varepsilon(\delta_{jk}+\langle F(\Ab)\vec{g}_j\cdot F(\Ab)\vec{g}_k\rangle)$ is well defined
in terms of a Radon--Stieltjes integral. Specifically, the second
formula in equation \eqref{eq:Spectral_Theorem} with
$F(z)=G(z)=(\varepsilon+z)^{-1}$, $\vec{\xi}=\vec{g}_j$, $\vec{\zeta}=\vec{g}_k$,
$\d\mu_{\xi\zeta}(z):=\d\mu_{jk}(z)=\d\langle\Qb(z)\vec{g}_j\cdot\vec{g}_k\rangle
=\d\langle\Qb_2(\text{Im}(z))\vec{g}_j\cdot\vec{g}_k\rangle$, and $z=\I\lambda$
yields         
%
\begin{align}\label{eq:Int_kappa*_jk}
  \kappa^*_{jk}/\varepsilon-\delta_{jk}
%  \langle F(\Ab)\vec{g}_j\cdot F(\Ab)\vec{g}_k\rangle
               =\int_I\frac{\d\mu_{jk}(z)}{\overline{(\varepsilon+z)}(\varepsilon+z)}
               =\int_I\frac{\d\mu_{jk}(z)}{\varepsilon^2+|z|^2}
               =\int_{-\I\infty}^{\I\infty}\frac{\d\mu_{jk}(\text{Im}(z))}{\varepsilon^2+|\text{Im}(z)|^2}
               =\int_{-\infty}^\infty\frac{\d\mu_{jk}(\lambda)}{\varepsilon^2+\lambda^2}.
\end{align}
%
Equation \eqref{eq:Int_kappa*_jk} establishes the integral
representation for $\kappa^*_{kk}$ in \eqref{eq:Integral_Rep_kappa*}, since
$\mu_{kk}(z)=\|\Qb(z)\vec{g}_k\|^2$ is a positive function so that
$\text{Re}\,\mu_{kk}(z)=\mu_{kk}(z)$ and
$\d\mu_{kk}(z)=\d\|\Qb(z)\vec{g}_k\|^2$ is a \emph{positive 
  measure}. However for $j\neq k$, the function
$\mu_{jk}(z)=\langle\Qb(z)\vec{g}_j\cdot\vec{g}_k\rangle$ is complex-valued, with    
$\mu_{kj}(z)=\overline{\mu}_{jk}(z)$, so that $\d\mu_{jk}(z)$ is a
\emph{complex measure}. Since the vector field $\vec{\nabla}\chi_k$ is
real-valued, the 
functional $\langle\vec{\nabla}\chi_j\cdot\vec{\nabla}\chi_k\rangle=\langle F(\Ab)\vec{g}_j\cdot F(\Ab)\vec{g}_k\rangle$
is also real-valued, which implies that the final integral in
\eqref{eq:Int_kappa*_jk} must be representable in terms of a
\emph{signed measure} for $j\neq k$. The validity of this follows from the
symmetry $\langle\vec{\nabla}\chi_j\cdot\vec{\nabla}\chi_k\rangle=\langle\vec{\nabla}\chi_k\cdot\vec{\nabla}\chi_j\rangle$, so that
$2\langle\vec{\nabla}\chi_j\cdot\vec{\nabla}\chi_k\rangle=\langle\vec{\nabla}\chi_j\cdot\vec{\nabla}\chi_k\rangle+\langle\vec{\nabla}\chi_k\cdot\vec{\nabla}\chi_j\rangle$. Therefore, since
$\mu_{kj}(\lambda)=\overline{\mu}_{jk}(\lambda)$, by the linearity properties of
Radon--Stieltjes integrals \cite{Stone:64}, for $j\neq k=1,\ldots,d$, equation
\eqref{eq:Int_kappa*_jk} becomes 
% 
\begin{align}\label{eq:Int_kappa*_jk_Re}
  \kappa^*_{jk}/\varepsilon-\delta_{jk}
  %2\langle F(\Ab)\vec{g}_j\cdot F(\Ab)\vec{g}_k\rangle
      %=\int_{-\infty}^\infty\frac{\d\mu_{jk}(\lambda)}{\varepsilon^2+\lambda^2}+\int_{-\infty}^\infty\frac{\d\overline{\mu}_{jk}(\lambda)}{\varepsilon^2+\lambda^2}           
       =\frac{1}{2}\int_{-\infty}^\infty\frac{\d(\mu_{jk}(\lambda)+\overline{\mu}_{jk}(\lambda))}{\varepsilon^2+\lambda^2}
       =\int_{-\infty}^\infty\frac{\d\text{Re}\,\mu_{jk}(\lambda)}{\varepsilon^2+\lambda^2}\,,            
\end{align}
%
which establishes the integral representation for $\kappa^*_{jk}$ in
\eqref{eq:Integral_Rep_kappa*} for $j\neq k$. As anticipated, the
integral representation for $\kappa^*_{jk}$ in equation
\eqref{eq:Integral_Rep_kappa*} satisfies $\kappa^*_{kj}=\kappa^*_{jk}$, since
$\mu_{kj}(\lambda)=\overline{\mu}_{jk}(\lambda)$ implies that
$\text{Re}\,\mu_{kj}(\lambda)=\text{Re}\,\mu_{jk}(\lambda)$.





We now derive the integral representation for $\alpha^*_{jk}$, $j\neq k=1,\ldots,d$,
displayed in equation \eqref{eq:Integral_Rep_kappa*}. Consider the
representation for $\alpha^*_{jk}$ in \eqref{eq:Eff_Diff_Resolvent} and
define the functions $F(z)=z(\varepsilon+z)^{-1}$ and $G(z)=(\varepsilon+z)^{-1}$ so that,
formally, $\alpha^*_{jk}=\langle F(\Ab)\vec{g}_j\cdot G(\Ab)\vec{g}_k\rangle$. We have
already established that $\vec{g}_k\in\Ds(G)$ for all
$k=1,\ldots,d$. Since $z=\I\lambda$ and $\lambda,\varepsilon\in\mathbb{R}$, we have that the
function $|F(z)|^2=\lambda^2(\varepsilon^2+\lambda^2)^{-1}<1$ for all $0<\varepsilon<\infty$.  By equations
\eqref{eq:Bounded_H} and \eqref{eq:Mass}, the positive measure
$\d\mu_{kk}$ has bounded mass $\mu^0_{kk}\leq\|\Hb\|^2<\infty$, which implies that
$F\in L^2(\mu_{kk})$ hence $\vec{g}_k\in\Ds(F)$ for all
$k=1,\ldots,d$. Consequently, the functional $\alpha^*_{jk}=\langle F(\Ab)\vec{g}_j\cdot
G(\Ab)\vec{g}_k\rangle$ has a well defined 
meaning in terms of a Radon-Stieltjes integral. Specifically, the
second formula in equation \eqref{eq:Spectral_Theorem} with
$F(z)=z(\varepsilon+z)^{-1}$, $G(z)=(\varepsilon+z)^{-1}$, $\vec{\xi}=\vec{g}_j$,
$\vec{\zeta}=\vec{g}_k$, $\d\mu_{\xi\zeta}(z):=\d\mu_{jk}(z)$ defined as in equation
\eqref{eq:Int_kappa*_jk}, 
%$$:=\d\mu_{jk}(z)=\d\langle\Qb(z)\vec{g}_j\cdot\vec{g}_k\rangle
%=\d\langle\Qb_1(0)\Qb_2(\text{Im}(z))\vec{g}_j\cdot\vec{g}_k\rangle$,
and $z=\I\lambda$ yields          
%
\begin{align}\label{eq:Int_alpha*_jk}
  \alpha^*_{jk}
  %=\langle F(\Ab)\vec{g}_j\cdot G(\Ab)\vec{g}_k\rangle
     %=\int_I\frac{\overline{z}\,\d\mu_{jk}(z)}{(\varepsilon+z)\overline{(\varepsilon+z)}}
     =\int_I\frac{\overline{z}\,\d\mu_{jk}(z)}{\varepsilon^2+|z|^2}
     =\int_{-\I\infty}^{\I\infty}\frac{-\I\,\text{Im}(z)\d\mu_{jk}(\text{Im}(z))}{\varepsilon^2+|\text{Im}(z)|^2}
     =\int_{-\infty}^\infty\frac{-\I\lambda\,\d\mu_{jk}(\lambda)}{\varepsilon^2+\lambda^2}\,.
\end{align}
%




Similar to the derivation of the integral representation for
$\kappa^*_{jk}$ in \eqref{eq:Int_kappa*_jk_Re} when $j\neq k$, we use that
$\alpha^*_{jk}=\langle\Ab\vec{\nabla}\chi_j\cdot\vec{\nabla}\chi_k\rangle=\langle
F(\Ab)\vec{g}_j\cdot G(\Ab)\vec{g}_k\rangle$ is real-valued. This implies that we
have the symmetry
$\langle\Ab\vec{\nabla}\chi_j\cdot\vec{\nabla}\chi_k\rangle=\langle\vec{\nabla}\chi_k\cdot\Ab\vec{\nabla}\chi_j\rangle$, so that
$2\langle\Ab\vec{\nabla}\chi_j\cdot\vec{\nabla}\chi_k\rangle=\langle\Ab\vec{\nabla}\chi_j\cdot\vec{\nabla}\chi_k\rangle+\langle\vec{\nabla}\chi_k\cdot\Ab\vec{\nabla}\chi_j\rangle$.
This, equation \eqref{eq:Int_alpha*_jk},
$\mu_{kj}(z)=\overline{\mu}_{jk}(z)$, $z=\I\lambda$, and the
linearity properties of Radon--Stieltjes integrals \cite{Stone:64}
imply that 
% 
\begin{align}\label{eq:Int_kappa*_jk_Im}
  \alpha^*_{jk}
  %=2\langle F(\Ab)\vec{g}_j\cdot G(\Ab)\vec{g}_k\rangle
      %=\int_{-\infty}^\infty\frac{\d\mu_{jk}(\lambda)}{\varepsilon^2+\lambda^2}+\int_{-\infty}^\infty\frac{\d\overline{\mu}_{jk}(\lambda)}{\varepsilon^2+\lambda^2}           
       =\frac{1}{2}\int_{-\infty}^\infty\frac{\lambda\,\d(\I\,[\overline{\mu}_{jk}(\lambda)-\mu_{jk}(\lambda)])}{\varepsilon^2+\lambda^2}
       =\int_{-\infty}^\infty\frac{\lambda\,\d\text{Im}\,\mu_{jk}(\lambda)}{\varepsilon^2+\lambda^2}\,,            
\end{align}
%
which establishes the integral representation for $\alpha^*_{jk}$ in
\eqref{eq:Integral_Rep_kappa*}. As anticipated, the integral
representation for $\alpha^*_{jk}$ in equation
\eqref{eq:Integral_Rep_kappa*} satisfies $\alpha^*_{kj}=-\alpha^*_{jk}$, since
$\mu_{kj}(\lambda)=\overline{\mu}_{jk}(\lambda)$ implies that
$\text{Im}\,\mu_{kj}(\lambda)=-\text{Im}\,\mu_{jk}(\lambda)$. Moreover, since
$\mu_{kk}(\lambda)$ is real-valued so that $\text{Im}\,\mu_{kk}(\lambda)\equiv0$, we also
have $\alpha^*_{kk}=0$.   



We have already discussed that the domain of integration $I$ of the
integral representations in \eqref{eq:Spectral_Theorem} are determined
by the spectrum $\Sigma(\Nb)$ of the maximal normal operator $\Nb$. Hence,
the domain of integration in equation \eqref{eq:Integral_Rep_kappa*}
is determined by the spectrum $\Sigma(\Ab)$ of the operator
$\Ab$. Although, we were able to take $I\subseteq(-\infty,\infty)$ in
\eqref{eq:Integral_Rep_kappa*} because of the properties of the
functions $F(z)$, $G(z)$, and $\mu_{jk}(z)$ underlying these integral
representations (see equations \eqref{eq:Int_kappa*_jk} and
\eqref{eq:Int_alpha*_jk}). Since the spectral radius of the normal
operator $\Ab$ is given by its norm $\|\Ab\|$ \cite{Reed-1980}, $I$ can
be an unbounded set in the case of a time-dependent velocity field
$\vec{u}$ and $I\subseteq[-\|\Ab\|,\|\Ab\|]$ with $\|\Ab\|\leq\|\Hb\|<\infty$ in the case of a
time-independent velocity field. This concludes our proof of Theorem
\ref{thm:Integral_Reps} $\Box$.    





We now discuss an important corollary of Theorem
\ref{thm:Integral_Reps} that provides integral
representations for $\bkappa^*$ and $\balpha^*$ in terms of an
anti-symmetric operator $A$, which acts on \emph{scalar-valued}
functions. This formulation
\cite{Pavliotis:PHD_Thesis,Bhattacharya:AAP:1999:951} of the effective
parameter problem for $\Kbc^*$ has a more practical numerical
implementation than that involving the operator $\Ab$, which acts on
\emph{vector-valued} functions, and will be used in Section
\ref{sec:Num_Results} to compute the effective diffusivity tensor
$\Kbc^*$ for model flows. For the case of a time-independent,
continuously differentiable velocity field
$\vec{u}$, the operator $A$ is compact on a Sobolev space $\Hc^1_{\Vc}$
\cite{Bhattacharya:AAP:1999:951}, and a resolvent formula for $\chi_j$
involving $A$ has led to \cite{Pavliotis:PHD_Thesis} a discrete 
version of the integral representation for $\balpha^*$ displayed in 
\eqref{eq:Integral_Rep_kappa*}. We now show that the conditions of
Theorem \ref{thm:Integral_Reps} can be modified slightly to generalize
this result for $\balpha^*$ to the case of a time-dependent velocity
field $\vec{u}\in\otimes_{n=1}^d(\tilde{\As}_{\Tc}\otimes L^2(\Vc))$, as well as
extending the result to $\bkappa^*$. The details are as follows.     




Consider the Hilbert
%space
spaces $\Hc_{\Tc}$ and $\Hc_{\Vc}$
%$\Hc_{\Vc}=\{f\in L^2(\Vc) \ | \ f(0)=f(\vec{V})\}$
(over the complex field $\mathbb{C}$) of Lebesgue measurable, square
integrable,
%$\Vc$--periodic,
scalar-valued functions,
which are $\Tc$--periodic and $\Vc$--periodic, respectively, as well
as their direct product $\Hc_{\Tc\Vc}$,  
%
\begin{align}\label{eq:Hilbert_Spaces_scalar}
  \Hc_{\Tc\Vc}=\Hc_{\,\Tc}\otimes\Hc_{\,\Vc}, \quad
  \Hc_{\Tc}=\{f\in L^2(\Vc) \ | \ f(0)=f(T)\}, \quad
  \Hc_{\Vc}=\{f\in L^2(\Vc) \ | \ f(0)=f(\vec{V})\},
\end{align}
%
which are analogous to the Hilbert spaces $\Hs_{\Tc}$, $\Hs_{\Vc}$,
and $\Hs_{\Tc\Vc}$ of vector-valued functions defined in equation
\eqref{eq:Hilbert_Spaces}. Denote by $\langle\cdot,\cdot\rangle$ the sesquilinear
inner-product associated with the Hilbert space $\Hc_{\Tc\Vc}$, which
is defined by $\langle f,h\rangle=\langle\overline{f}\, h\rangle$ with
$\langle f,h\rangle=\overline{\langle h,f\rangle}$. Here, $\langle\cdot\rangle$ still denotes space-time
averaging over $\Tc\times\Vc$ and we denote by $\|\cdot\|$ the norm induced by
the $\Hc_{\Tc\Vc}$-inner-product. Analogous to the Hilbert space
$\Hs_\times\subset\Hs_{\Vc}$ defined in equation \eqref{eq:Helmholtz}, we
consider the Sobolev space $\Hc^1_{\Vc}\subset\Hc_{\Vc}$     
% 
\begin{align}\label{eq:Sobolev}
  \Hc^1_{\Vc}=\{f\in \Hc_{\Vc} \; | \; \langle|\vec{\nabla}f|^2\rangle_{\Vc}<\infty\}, 
\end{align}
%
which is also a Hilbert space \cite{Folland:95}. Here, $\langle\cdot\rangle_{\Vc}$
denotes spatial averaging over $\Vc$. Finally, consider the 
function space $\Fc$ with inner-product $\langle\cdot,\cdot\rangle_1$
%
\begin{align}\label{eq:Function_Space_Scalar}
  \Fc=\{f\in\tilde{\As}_{\Tc}\otimes\Hc^1_{\Vc} \; | \; \langle f\rangle=0\},  \qquad
  \langle f,g\rangle_1=\left\langle\overline{\vec{\nabla}f}\cdot\vec{\nabla}g\right\rangle,
\end{align}
%
which is analogous to $\Fs$ in \eqref{eq:Function_Space} and involves
the space $\tilde{\As}_{\Tc}$ of absolutely continuous,
$\Tc$--periodic, scalar-valued functions defined in equation \eqref{eq:AC_BC}.    
%which is a dense subset of the Hilbert space $\Hc_{\Tc}\otimes\Hc^1_{\Vc}$,
We denote by $\|\cdot\|_1$ the norm induced by the $\Fc$-inner-product,
where $h\in\Fc$ implies that $\|\partial_th\|_1<\infty$ and $\|h\|_1<\infty$. In the
case of a time-independent velocity field $\vec{u}$ we set
$\tilde{\As}_{\Tc}=\emptyset$ in \eqref{eq:Function_Space_Scalar}, so that
$h\in\Fc$ implies $h\in\Hc^1_{\Vc}$ with $\langle h\rangle_{\Vc}=0$.





In terms of the $\Fc$-inner-product $\langle\cdot,\cdot\rangle_1$, the components
$\kappa^*_{jk}$ and $\alpha^*_{jk}$, $j,k=1,\ldots,d$, of the symmetric $\bkappa^*$
and anti-symmetric $\balpha^*$ parts of the effective diffusivity
tensor $\Kbc^*$ are given by the following functionals 
\cite{Pavliotis:PHD_Thesis},   
%
\begin{align}\label{eq:Eff_Diffusivity_Sobolev}
  \kappa^*_{jk}=\varepsilon(\delta_{jk}+\langle\chi_j,\chi_k\rangle_1), \qquad
  \alpha^*_{jk}=\langle A\chi_j,\chi_k\rangle_1, \quad
  A=\Delta^{-1}(\vec{u}\cdot\vec{\nabla}-\partial_t),
\end{align}
%
which are analogous to that in \eqref{eq:Eff_Diffusivity}.
The formulas for $\kappa^*_{jk}$ and $\alpha^*_{jk}$ in equation
\eqref{eq:Eff_Diffusivity_Sobolev} follow \cite{Pavliotis:PHD_Thesis}
from $\Kc^*_{jk}=\varepsilon\delta_{jk}+\langle u_j\chi_k\rangle$ in
\eqref{eq:Eff_Diffusivity_Appendix} and the cell problem
$-\varepsilon\Delta\chi_j-(\vec{u}\cdot\vec{\nabla}-\partial_t)\chi_j=u_j$ in equation
\eqref{eq:Cell_Prob_Appendix}: 
%
\begin{align}
  \langle u_j\chi_k\rangle=\langle\Delta\Delta^{-1}u_j\chi_k\rangle
       =-\langle\vec{\nabla}\Delta^{-1}u_j\cdot\vec{\nabla}\chi_k\rangle
       =-\langle\Delta^{-1}u_j,\chi_k\rangle_1
       =\varepsilon\langle\chi_j,\chi_k\rangle_1+\langle A\chi_j,\chi_k\rangle_1,
\end{align}
%
where the periodicity of $u_j$ and $\chi_k$ was used in the second equality.
Applying the operator $-(\Delta^{-1})$ to both sides of equation
\eqref{eq:Cell_Prob_Appendix}, we obtain the following resolvent
formula for $\chi_j$ involving $A$ in \eqref{eq:Eff_Diffusivity_Sobolev},
which is analogous to equation \eqref{eq:Resolvent_Rep},   
%
\begin{align}\label{eq:Resolvent_Rep_Scalar}
  \chi_j=(\varepsilon+A)^{-1}g_j, \qquad 
  %A=\Delta^{-1}(\vec{u}\cdot\vec{\nabla}-\partial_t), \quad
  g_j=-\Delta^{-1}u_j.
\end{align}
%



In equation \eqref{eq:Anti-sym_Sobolev} of Section
\ref{sec:Symmetries_Commute} we show that the incompressibility of
$\vec{u}$ in \eqref{eq:incompressible} implies that the operator
$(\Delta^{-1})(\vec{u}\cdot\vec{\nabla})$ is anti-symmetric on
$\tilde{\As}_{\Tc}\otimes\Hc^1_{\Vc}$ \cite{Bhattacharya:AAP:1999:951}.
Moreover, for $\vec{u}\in\otimes_{n=1}^d\tilde{\As}_{\Tc}\otimes\Hc_{\Vc}$, we also have that 
it is a bounded operator on $\tilde{\As}_{\Tc}\otimes\Hc^1_{\Vc}$. Indeed,
since $\Delta=\vec{\nabla}\cdot\vec{\nabla}$, we have 
%$\|(\Delta^{-1})h\|_1^2=|\langle\vec{\nabla}(\Delta^{-1})h\cdot\vec{\nabla}(\Delta^{-1})h\rangle|=|\langle[(\Delta^{-1})h]\,h\rangle|$,
%
\begin{align}\label{eq:H1_L2}
  \|(\Delta^{-1})h\|_1^2=\langle\vec{\nabla}(\Delta^{-1})h\cdot\vec{\nabla}(\Delta^{-1})h\rangle=\langle[(-\Delta^{-1})h]\,h\rangle,
  \ \ \forall \,h\in\Fc,
\end{align}
%
which implies that
%
\begin{align}\label{eq:Bounded_A_Sobolov}
  \|(\Delta^{-1})(\vec{u}\cdot\vec{\nabla})h\|_1^2
    %=|\langle\vec{\nabla}(\Delta^{-1})(\vec{u}\cdot\vec{\nabla})h
    %   \cdot\vec{\nabla}(\Delta^{-1})(\vec{u}\cdot\vec{\nabla})h \rangle|^2
    =|\langle[(\Delta^{-1})(\vec{u}\cdot\vec{\nabla})h]\, (\vec{u}\cdot\vec{\nabla})h \rangle|
   \leq\|(\Delta^{-1})\|\,\|\vec{u}\|^2\|h\|_1^2
   <\infty,
\end{align}
%
where we have used the simplified notation $\|\vec{u}\|^2=\sum_j\|u_j\|^2$
and the fact \cite{Stakgold:BVP:2000} that $(\Delta^{-1})$ is compact,
hence bounded on $\tilde{\As}_{\Tc}\otimes\Hc^1_{\Vc}$. Since
$(\Delta^{-1})(\vec{u}\cdot\vec{\nabla})$ is bounded anti-symmetric operator on
$\tilde{\As}_{\Tc}\otimes\Hc^1_{\Vc}$, the operator
$\I(\Delta^{-1})(\vec{u}\cdot\vec{\nabla})$ is self-adjoint on
$\tilde{\As}_{\Tc}\otimes\Hc^1_{\Vc}$. We have already
established that $\I\partial_t$ is a self-adjoint operator on the function
space $\tilde{\As}_{\Tc}$ and that $\I\partial_t$ and $(\Delta^{-1})$ commute on
$\tilde{\As}_{\Tc}\otimes\Hc^1_{\Vc}$, which implies that $\I(\Delta^{-1})\partial_t$ is
self-adjoint on $\tilde{\As}_{\Tc}\otimes\Hc^1_{\Vc}$. It is now clear that
$M=-\I A$, with $A=(\Delta^{-1})(\vec{u}\cdot\vec{\nabla}-\partial_t)$ is self-adjoint on
$\tilde{\As}_{\Tc}\otimes\Hc^1_{\Vc}$. Finally, since an operator is
self-adjoint if and only if it is a maximal normal operator
\cite{Stone:64}, we have that $A$ is a maximal normal operator on
the function space $\Fc$. Using this alternative Hilbert space
formulation of the effective parameter problem for $\Kbc^*$, we have
the following corollary of Theorem \ref{thm:Integral_Reps}.
%
\begin{corollary}\label{cor:Integral_Reps}
%
Let $z=\I\lambda$, $g_j=-(\Delta^{-1})u_j$ be defined as in
\eqref{eq:Resolvent_Rep_Scalar}, and $Q(z)=Q_2({\rm Im}(z))=Q_2(\lambda)$ be
the complex resolution of the identity associated with the maximal
anti-symmetric operator $A$ defined in
\eqref{eq:Eff_Diffusivity_Sobolev}, with domain $\Fc$ defined in 
\eqref{eq:Function_Space_Scalar}.  Define the matrix-valued function
$\bmu(\lambda)$ with complex-valued off-diagonal components
$\mu_{jk}(\lambda)=\langle Q_2(\lambda)g_j,g_k\rangle_1$ for $j\neq k=1,\ldots,d$, with
$\mu_{kj}=\overline{\mu}_{jk}$, and positive diagonal components
$\mu_{kk}(\lambda)=\|Q_2(\lambda)g_k\|_1^2$.  Moreover, consider the real-valued
functions ${\rm Re}\,\mu_{jk}(\lambda)$ and ${\rm Im}\,\mu_{jk}(\lambda)$ defined in
\eqref{eq:Fns_Bounded_Var}. Corresponding to each of these functions
of bounded variation, consider the associated Radon--Stieltjes
measures $\d\mu_{jk}(\lambda)$, $\d\mu_{kk}(\lambda)$, $\d{\rm Re}\,\mu_{jk}(\lambda)$, and
$\d{\rm Im}\,\mu_{jk}(\lambda)$. Then, for all $0<\varepsilon<\infty$, the Radon--Stieltjes
integral representations in \eqref{eq:Integral_Rep_kappa*} hold for
the functionals $\kappa^*_{jk}$ and $\alpha^*_{jk}$ defined in equation
\eqref{eq:Eff_Diffusivity_Sobolev}. The domain of integration $I$ is
determined by the spectrum $\Sigma(A)$ of the operator $A$, where
$I\subseteq[-\|A\|,\|A\|]$ and $\|A\|\leq\|(\Delta^{-1})\|\|\vec{u}\|\|\vec{\nabla}\|_1<\infty$ in the case
of a time-independent velocity field $\vec{u}$.
% 
\end{corollary}
%



\textbf{Proof of Corollary \ref{cor:Integral_Reps}.}\hspace{1ex}
%
From equations \eqref{eq:Eff_Diffusivity_Sobolev} and
\eqref{eq:Resolvent_Rep_Scalar} we have the following analogue of
\eqref{eq:Eff_Diff_Resolvent}
%
\begin{align}\label{eq:Eff_Diff_Resolvent_Sobelov}
 \kappa^*_{jk}=\varepsilon\left(\delta_{jk}+\langle(\varepsilon+A)^{-1}g_j,(\varepsilon+A)^{-1}g_k\rangle_1\right), \quad
 \alpha^*_{jk}=\langle A(\varepsilon+A)^{-1}g_j,(\varepsilon+A)^{-1}g_k\rangle_1,
\end{align}
%
where $g_j=-(\Delta^{-1})u_j$. By the proof of Theorem
\ref{thm:Integral_Reps} and the properties
$\chi_j,A\chi_j:\mathbb{R}\times\mathbb{R}^d\to\mathbb{R}$, we need only to prove
that $g_j\in\Fc$ for all $j=1,\ldots,d$. By equation \eqref{eq:H1_L2} and
$\vec{u}\in\otimes_{n=1}^d\tilde{\As}_{\Tc}\otimes\Hc_{\Vc}$, for each $t\in\Tc$
fixed, we have 
%
\begin{align}
  \|g_j(t,\cdot)\|_1^2%= \|(\Delta^{-1})u_j(t,\cdot)\|_1^2
             %=|\langle\vec{\nabla}(\Delta^{-1})u_j(t,\cdot)\cdot\vec{\nabla}(\Delta^{-1})u_j(t,\cdot)\rangle|^2
             %=|\langle(\Delta^{-1})u_j(t,\cdot)\cdot\Delta(\Delta^{-1})u_j(t,\cdot)\rangle|^2
             =|\langle[(\Delta^{-1})u_j(t,\cdot)]\,u_j(t,\cdot)\rangle|
             \leq\|(\Delta^{-1})\|\,\|u_j(t,\cdot)\|^2<\infty,
\end{align}
%
Similarly, since the operators $\partial_t$ and $\Delta^{-1}$ commute on
$\tilde{\As}_{\Tc}\otimes\Hc_{\Vc}$ and $u_j(\cdot,\vec{x})\in\tilde{\As}_{\Tc}$
for each $\vec{x}\in\Vc$ fixed, we have 
%
\begin{align}
  \|\partial_tg_j(\cdot,\vec{x})\|_1=\|\partial_t(\Delta^{-1})u_j(\cdot,\vec{x})\|_1
                    =\|(\Delta^{-1})\partial_tu_j(\cdot,\vec{x})\|_1
                    \leq\|\Delta^{-1}\|\,\|\partial_tu_j(\cdot,\vec{x})\|^2<\infty.
\end{align}
%
Finally, from equation \eqref{eq:incompressible} we have that
$\langle u_j(\cdot,\vec{x})\rangle=0$ for each $\vec{x}\in\Vc$ fixed. Therefore, by 
Fubini's theorem \cite{Folland:99} we have that
$\langle g_j\rangle=0$. Consequently, $g_j\in\Fc$ for all $j=1,\ldots,d$, which 
establishes that the integral representations in equation
\eqref{eq:Integral_Rep_kappa*} hold for the functionals $\kappa^*_{jk}$ and
$\alpha^*_{jk}$ defined in \eqref{eq:Eff_Diffusivity_Sobolev}. Analogous to
\eqref{eq:Mass}, by equation \eqref{eq:H1_L2} the mass
$\mu^0_{jk}$ of the associated measure $\d\mu_{jk}$ is given by  
% 
\begin{align}\label{eq:Mass_Scalar}
  \mu^0_{jk}
        % =\int_I\d\mu_{jk}(z)
%         =\int_I\d\langle\Qb(z)\vec{g}_j,\vec{g}_k\rangle
        =\langle g_j,g_k\rangle_1
        %=\langle\vec{\nabla}(\Delta^{-1})u_j\cdot\vec{\nabla}(\Delta^{-1})u_k\rangle 
        =\langle[(-\Delta^{-1})u_j]\,u_k\rangle,   
\end{align}
%
which implies that
$|\mu^0_{jk}|\leq\|\Delta^{-1}\|\,\|u_j\|\,\|u_k\|<\infty$.  The
domain of integration $I$ in \eqref{eq:Integral_Rep_kappa*} is
determined by the spectrum $\Sigma(A)$ of the operator $A$, where
$I\subseteq[-\|A\|,\|A\|]$ and $\|A\|\leq\|(\Delta^{-1})\|\|\vec{u}\|\|\vec{\nabla}\|_1<\infty$ in the case of a
time-independent velocity field $\vec{u}$. This concludes our proof of
Corollary \ref{cor:Integral_Reps} $\Box$.  




A natural question to ask is the following. Is the formulation of the
effective parameter problem described in Theorem \ref{thm:kappa_sigma}
equivalent to that described in Corollary \ref{cor:Integral_Reps}?
The relationship between the two formulations is one of isometry, and is
summarized by the following theorem.
%
\begin{theorem}\label{thm:Formulation_Equivalence}
%
The function spaces $\Fs$ and $\Fc$ defined in equations
\eqref{eq:Function_Space} and \eqref{eq:Function_Space_Scalar} are in
one-to-one isometric correspondence. This induces a one-to-one
isometric correspondence between the domains $D(\Ab)$ and $D(A)$ of
the operators $\Ab$ and $A$ defined in equations
\eqref{eq:Resolvent_Rep} and \eqref{eq:Eff_Diffusivity_Sobolev},
respectively. Specifically, for every $f\in D(A)$ we have
$\vec{\nabla}f\in D(\Ab)$ and $\|Af\|_1=\|\Ab\vec{\nabla}f\|$, and conversely, for each
$\vec{\xi}\in D(\Ab)$ there exists unique $f\in D(A)$ such that
$\vec{\xi}=\vec{\nabla}f$  and $\|\Ab\vec{\xi}\,\|=\|Af\|_1$. The Radon--Stieltjes
measures underlying the integral representations of Theorem
\ref{thm:kappa_sigma} and Corollary \ref{cor:Integral_Reps} are equal,
$\d\langle Q_2(\lambda)g_j,g_k\rangle_1=\d\langle\Qb_2(\lambda)\vec{g}_j,\vec{g}_k\rangle$, $j,k=1,\ldots,d$,
up to null sets of measure zero, where
$\vec{g}_j=\vec{\nabla}g_j$. Moreover, the operators $\Ab$ and $A$ are
related by $\Ab\vec{\nabla}=\vec{\nabla}A$, which implies and is implied by the
weak equality $\Qb_2(\lambda)\vec{\nabla}f=\vec{\nabla}Q_2(\lambda)f$ which holds for all
$f\in D(A)\iff\vec{\nabla}f\in D(\Ab)$.   

%
\end{theorem}
%

\textbf{Proof of Theorem \ref{thm:Formulation_Equivalence}.}\hspace{1ex}
%
We use the formula $\vec{u}=\vec{\nabla}\cdot\Hb$ displayed in equation
\eqref{eq:u_DH} to write the operator $A=\Delta^{-1}(\vec{u}\cdot\vec{\nabla}-\partial_t)$
and function $g_j=-\Delta^{-1}u_j$ defined in equations
\eqref{eq:Eff_Diffusivity_Sobolev} and \eqref{eq:Resolvent_Rep_Scalar}
as $A=\Delta^{-1}(\vec{\nabla}\cdot\Hb\vec{\nabla}-\partial_t)$ and
$g_j=-\Delta^{-1}\vec{\nabla}\cdot\Hb\vec{e}_j$, respectively. Using the definition
$\bGamma=\vec{\nabla}(\Delta^{-1})\vec{\nabla}\cdot$ and the formulas
$\vec{\nabla}\Delta^{-1}\partial_t=\bDelta^{-1}\Tb\vec{\nabla}\,$,
$\;\vec{g}_j=-\bGamma\Hb\vec{e}_j$, and $\Ab=\bGamma\Hb-\bDelta^{-1}\Tb$
displayed in equations \eqref{eq:Dt_T}, \eqref{eq:Resolvent_Rep}, and
\eqref{eq:Pre_Resolvent}, respectively, we have that   
%
\begin{align}\label{eq:A_Ab_Relation}
  \vec{\nabla}A=[\bGamma\Hb-\bDelta^{-1}\Tb]\vec{\nabla}=\Ab\vec{\nabla}, \qquad
  \vec{\nabla}g_j=\vec{g}_j.
\end{align}
%
Consequently, applying the
differential opperator $\vec{\nabla}$ to both sides of the formula
$(\varepsilon+A)\chi_j=g_j$ of \eqref{eq:Resolvent_Rep_Scalar}, we obtain the
formula  $(\varepsilon\Ib+\Ab)\vec{\nabla}\chi_j=\vec{g}_j$ of equation
\eqref{eq:Pre_Resolvent}.



Since the function spaces $\Fc$ and $\Fs$ differ
only in the characterization of the spatial variable $\vec{x}$, we now
discuss the relationship between the Hilbert spaces $\Hs_\times$ and
$\Hc^1_{\Vc}$ defined in equations \eqref{eq:Helmholtz} and
\eqref{eq:Sobolev}, respectively, with inner-product induced norms
$\|\cdot\|$ and $\|\cdot\|_1$. For $f\in\Hc^1_{\Vc}\subset L^2(\Vc)$ we have 
$\Delta^{-1}\Delta f=f$ \cite{Stakgold:BVP:2000}, which implies that
$\bGamma\vec{\nabla}f=\vec{\nabla}f$ and
$\|\vec{\nabla}f\|^2=\langle\vec{\nabla}f\cdot\vec{\nabla}f\rangle=\|f\|_1^2<\infty$. Consequently, for every  
$f\in\Hc^1_{\Vc}$ we have $\vec{\nabla}f\in\Hs_\times$. Conversely,
$\vec{\xi}\in\Hs_\times$ implies that $\vec{\xi}=\bGamma\vec{\xi}=\vec{\nabla}f$, where
we have defined the scalar-valued function
$f=\Delta^{-1}\vec{\nabla}\cdot\vec{\xi}\,$. Since $\vec{\xi}=\vec{\nabla}f$, the
$\Hc^1_{\Vc}$ norm of $f$ satisfies
$\|f\|_1^2=\langle\vec{\xi}\cdot\vec{\xi}\,\rangle=\|\vec{\xi}\,\|^2<\infty$ so that
$f\in\Hc^1_{\Vc}$. Moreover, $f$ is uniquely determined by $\vec{\xi}$ up
to equivalence class, since if $f_1=\Delta^{-1}\vec{\nabla}\cdot\vec{\xi}$ and
$f_2=\Delta^{-1}\vec{\nabla}\cdot\vec{\xi}\,$ then $\bGamma\vec{\xi}=\vec{\xi}$ implies
that $\|f_1-f_2\|_1=\|\vec{\xi}-\vec{\xi}\,\|=0$. Consequently, for every  
$\vec{\xi}\in\Hs_\times$ there exists unique $f\in\Hc^1_{\Vc}$ such that
$\vec{\xi}=\vec{\nabla}f$.  In summary, the Hilbert spaces $\Hc^1_{\Vc}$ and
$\Hs_\times$ are in one-to-one isometric correspondence, which we denote by
$\Hc^1_{\Vc}\sim\Hs_\times$. This, in turn, implies that $\Fc\sim\Fs$.  



We now return to our previous notation, where $\|\cdot\|_1$ and $\|\cdot\|$
denotes the norm induced by the $\Fc$- and $\Fs$-inner-product, 
respectively. We demonstrate that the one-to-one isometry between
$\Fc$ and $\Fs$ induces a one-to-one isometry between the domains
$D(A)$ and $D(\Ab)$ of the operators $A$ and $\Ab$,
i.e. $D(A)\sim D(\Ab)$. This, in turn, follows from another on-to-one
isometry between the class of self-adjoint operators on $\Fc$, for
example, and the class of resolutions of the identity. This 
correspondence is determined directly as follows \cite{Stone:64}. Let
$X$ be a self-adjoint operator on $\Fc$ and $Q(\lambda)$ be the associated
resolution of the idenity, which is a one-to-one correspondence
\cite{Stone:64}. The domain $D(X)$ of $X$ comprises those and only
those elements $f\in\Fc$ such that the Stieltjes integral
$\int_{-\infty}^\infty\lambda^2\,\d\|Q(\lambda)f\|_1^2$ is convergent; when $f\in D(X)$ the element
$Xf$ is determined by the relations    
%
\begin{align}\label{eq:X_Q_Correspondence}
  \langle Xf,h\rangle_1=\int_{-\infty}^\infty\lambda\,\d\langle Q(\lambda)f,h\rangle_1, \qquad
  \|Xf\|_1^2=\int_{-\infty}^\infty\lambda^2\,\d\|Q(\lambda)f\|_1^2,
\end{align}
%
where $h$ is an arbitrary element in $\Fc$ \cite{Stone:64}. Since
$M=-\I A$ is 
self-adjoint on $\Fc$ and $D(A)=D(M)$, this one-to-one isometric
correspondence also holds for the maximal normal operator $A$, and a
calculation similar to that in equations \eqref{eq:Int_kappa*_jk} and
\eqref{eq:Int_kappa*_jk_Re} shows that equation
\eqref{eq:X_Q_Correspondence} holds under the mappings $X\mapsto A$,
$\d\langle Q(\lambda)f,h\rangle_1\mapsto\d\text{Re}\,\langle Q(\lambda)f,h\rangle_1$, and $Q(\lambda)\mapsto Q_2(\lambda)$. An 
analogous result holds 
for the self-adjoint operator $\Mb=-\I\Ab$ on $\Fs$ with
$D(\Ab)=D(\Mb)$.



We now demonstrate that the one-to-one isometry between the class of
self-adjoint operators and resolutions of the identity on $\Fc$ and
that for $\Fs$, along with the property $\Fc\sim\Fs$ and equation
\eqref{eq:A_Ab_Relation}, induce the one-to-one isometry
$D(A)\sim D(\Ab)$. From $\Fc\sim\Fs$, we have for every $f\in D(A)\subset\Fc$ that 
$\vec{\nabla}f\in\Fs$, so from equation \eqref{eq:A_Ab_Relation} 
%
\begin{align}\label{eq:Norm_A_Ab}
  \|Af\|_1^2=\langle Af,Af\rangle_1
        =\langle\vec{\nabla} A f\cdot\vec{\nabla} A f\rangle
        =\langle\Ab\vec{\nabla}f\cdot\Ab\vec{\nabla}f\rangle
        =\|\Ab\vec{\nabla}f\|^2.
\end{align}
%
Consequently, by the spectral theorem of equation
\eqref{eq:Spectral_Theorem}, we have 
%
\begin{align}\label{eq:A_Ab}
  \int\lambda^2\,\d\|Q_2(\lambda)f\|_1^2=\int\lambda^2\,\d\|\Qb_2(\lambda)\vec{\nabla}f\|^2,
\end{align}
%
and the convergence of the left-hand-side of \eqref{eq:A_Ab} implies
the convergence of the right-hand-side which, in turn, implies that
$\vec{\nabla}f\in D(\Ab)$ via \eqref{eq:X_Q_Correspondence}. Conversely from
$\Fc\sim\Fs$, we have that for each $\vec{\xi}\in D(\Ab)\subseteq\Fs$ there exists 
unique $f\in\Fc$ such that $\vec{\xi}=\vec{\nabla}f$, and equation
\eqref{eq:A_Ab_Relation} then implies that   
%
\begin{align}
  \|\Ab\vec{\xi}\,\|^2=\langle\Ab\vec{\nabla}f,\Ab\vec{\nabla}f\rangle
         =\langle\vec{\nabla} A f,\vec{\nabla} A f\rangle
         =\langle Af,Af\rangle_1
         =\|Af\|_1^2.
\end{align}
%
Again, the spectral theorem implies that \eqref{eq:A_Ab} holds, and
the convergence of the right-hand-side of \eqref{eq:A_Ab} implies the
convergence of the left-hand-side which, in turn, implies that
$f\in D(A)$ via \eqref{eq:X_Q_Correspondence}. In summary, for every $f\in
D(A)$ we have $\vec{\nabla}f\in D(\Ab)$ and
$\|Af\|_1^2=\|\Ab\vec{\nabla}f\|^2$. Conversely, for each $\vec{\xi}\in D(\Ab)$ there
exists unique $f\in D(A)$ such that $\vec{\xi}=\vec{\nabla}f$ and
$\|\Ab\vec{\xi}\,\|^2=\|Af\|_1^2$. Consequently, the domains $D(\Ab)$ and $D(A)$
are in one-to-one isometric correspondence, i.e. $D(\Ab)\sim D(A)$.     




We now show that this result implies, and is implied by the formula
$\vec{\nabla}Q_2(\lambda)=\Qb_2(\lambda)\vec{\nabla}$, which holds in a weak sense, where
$Q_2(\lambda)$ and $\Qb(\lambda)$ are the resolutions of the identity associated
with the operators $A$ and $\Ab$, respectively. From equation
\eqref{eq:A_Ab} and the linearity properties of  Radon--Stieltjes
integrals \cite{Stone:64}, we have that   
%
\begin{align}\label{eq:Measure_Eqality}
  0=\int_{-\infty}^\infty\lambda^2\,\d(\|Q_2(\lambda)f\|_1^2-\|\Qb_2(\lambda)\vec{\nabla}f\|^2)
   =\int_{-\infty}^\infty\lambda^2\,\d(\langle[\vec{\nabla}Q_2(\lambda)-\Qb_2(\lambda)\vec{\nabla}]f\cdot\vec{\nabla}f\rangle).
\end{align}
%
Equation \eqref{eq:Measure_Eqality} implies that for all
$f\in D(A)\iff\vec{\nabla}f\in D(\Ab)$ we have
$\d\|Q_2(\lambda)f\|_1^2=\d\|\Qb_2(\lambda)\vec{\nabla}f\|^2$, up to null sets of measure
zero. Moreover, the equality $\vec{\nabla}Q_2(\lambda)=\Qb_2(\lambda)\vec{\nabla}$ holds
weakly. Conversely, assume that $Q_2(\lambda)$ and 
$\Qb(\lambda)$ are the resolutions of the identity associated with the
operators $A$ and $\Ab$ on the function spaces $\Fc$ and $\Fs$,
respectively, which is a one-to-one correspondence \cite{Stone:64},
and that $\vec{\nabla}Q_2(\lambda)f=\Qb_2(\lambda)\vec{\nabla}f$ 
holds for every $f\in D(A)\iff\vec{\nabla}f\in D(\Ab)$. Then equation
\eqref{eq:Measure_Eqality} holds and implies equation
\eqref{eq:A_Ab}. The corresondence $D(A)\sim D(\Ab)$ and equation
\eqref{eq:X_Q_Correspondence} then imply that
$\|\Ab\vec{\nabla}f\,\|^2=\|Af\|_1^2=\|\vec{\nabla}Af\|^2$ for all
$f\in D(A)\iff\vec{\nabla}f\in D(\Ab)$, which implies that $\Ab\vec{\nabla}=\vec{\nabla}A$
weakly. Since $g_k\in D(A)$ and $\vec{g}_k\in D(\Ab)$, this 
result implies that the Radon--Stieltjes measures underlying the
integral representations of Theorem \ref{thm:kappa_sigma} and
Corollary \ref{cor:Integral_Reps} are equal,
$\d\langle Q_2(\lambda)g_j,g_k\rangle_1=\d\langle\Qb_2(\lambda)\vec{g}_j,\vec{g}_k\rangle$, $j,k=1,\ldots,d$, up
to null sets of measure zero. This concludes our proof of Theorem
\ref{thm:Formulation_Equivalence} $\Box$.   




\subsubsection{Finite dimensional matrix
  setting}\label{sec:Integral_Rep_Matrix} 
%
In this section we demonstrate how the integral formulas in
\eqref{eq:Integral_Rep_kappa*} arise when $\Ab$ is given by an
anti-symmetric matrix. This mathematical framework will be utalized in
Section \ref{sec:Num_Results} to directly compute the effective
diffusivity tensor $\Kbc^*$ for model velocity fields $\vec{u}$. This
is accomplished via the direct computation of the spectral measure
$\d\bmu$ at the heart of these integral representations, which is
given explicitely in terms of the eigenvalues and eigenvectors of
$\Ab$ in this matrix setting.  





When the operator $\Ab$ is given by a real-valued anti-symmetric
matrix $\Ab^T=-\Ab$ of size $N$, say, $\Sigma(\Ab)$ consists solely of
\emph{pure point spectrum} called 
eigenvalues $\upsilon_n$, $n\in I_n=\{1,\ldots,N\}$, with corresponding eigenvectors
$\vec{w}_n$, $\Ab\vec{w}_n=\upsilon_n\vec{w}_n$. It is well known
\cite{Horn_Johnson-1990} that the eigenvalues $\upsilon_n$ are purely
imaginary, $\upsilon_n=\I\lambda_n$ with $\lambda_n\in\mathbb{R}$, and that the set of
eigenvectors $\{\vec{w}_n\}_{n\in I_n}$ forms an orthonormal basis
\cite{Keener-2000} for $\mathbb{R}^N$,
$\vec{w}_n\cdot\vec{w}_m=\vec{w}_n^{\;\dagger}\vec{w}_m=\delta_{nm}$, where $\dagger$ 
denotes the operation of complex-conjugate-transpose. Moreover, as 
$\Ab$ is real-valued, the eigenvalues $\upsilon_n$ and eigenvectors
$\vec{w}_n$ come in complex-conjugate pairs
\cite{Horn_Johnson-1990}. Therefore, if the size $N$ of the matrix
$\Ab$ is even, then we may re-number the index set as
$I_n=\{-N/2,\ldots,-1,1,\ldots ,N/2\}$ such that $\upsilon_{-n}=\overline{\upsilon_n}=-\upsilon_n$ and
$\vec{w}_{-n}=\overline{\vec{w}_n}$, and if $N$ is odd then $\upsilon_0=0$ is
also an eigenvalue with a \emph{real-valued} eigenvector
$\vec{w}_0$. Consequently, denoting by $\Wb$ the matrix with columns
consisting of the eigenvectors $\vec{w}_n$,
$\bUpsilon=\text{diag}(\upsilon_{-N/2},\ldots,\upsilon_{N/2})$ the diagonal matrix with
eigenvalues $\upsilon_n$ on the main diagonal, and
$\bLambda=\text{diag}(\lambda_{-N/2},\ldots,\lambda_{N/2})$, we have
$\Ab=\Wb\bUpsilon\Wb^\dagger=\I\Wb\bLambda\Wb^\dagger$. Here, the matrix $\Wb$ is
unitary $\Wb^\dagger\Wb=\Wb\Wb^\dagger=\Ib$ so that the matrix
$\Mb=-\I\Ab=\Wb\bLambda\Wb^\dagger$ is a Hermitian matrix
\cite{Horn_Johnson-1990,Keener-2000}.  




This spectral characterization of the matrix $\Ab$ leads to integral
representations for $\kappa_{jk}^*$ and $\alpha_{jk}^*$, as defined by the
functionals in equation \eqref{eq:Eff_Diff_Resolvent}. Since $\upsilon_n$ is
purely imaginary and $0<\varepsilon<\infty$, the matrix $(\varepsilon\Ib+\Ab)$ has a well defined
inverse given by
$(\varepsilon\Ib+\Ab)^{-1}=\Wb(\varepsilon\Ib+\I\bLambda)^{-1}\Wb^\dagger$
\cite{Horn_Johnson-1990}. Therefore, since the (Hilbert space) adjoint
of $\Wb$ is given by $\Wb^\dagger=\Wb^{-1}$
\cite{Horn_Johnson-1990,Keener-2000}, we have  
% 
\begin{align}\label{eq:Functionals_kappa_alpha}
  %\mathcal{F}(\Ab)
  &\kappa^*_{jk}/\varepsilon-\delta_{jk}=\langle(\varepsilon\Ib+\Ab)^{-1}\vec{g}_j\cdot(\varepsilon\Ib+\Ab)^{-1}\vec{g}_k\rangle 
=\langle(\varepsilon\Ib+\I\bLambda)^{-1}\Wb^\dagger\vec{g}_j\cdot(\varepsilon\Ib+\I\bLambda)^{-1}\Wb^\dagger\vec{g}_k\rangle ,
\\
%\mathcal{G}(\Ab)
&\alpha^*_{jk}=\langle\Ab(\varepsilon\Ib+\Ab)^{-1}\vec{g}_j\cdot(\varepsilon\Ib+\Ab)^{-1}\vec{g}_k\rangle
=\langle\I\bLambda(\varepsilon\Ib+\I\bLambda)^{-1}\Wb^\dagger\vec{g}_j\cdot(\varepsilon\Ib+\I\bLambda)^{-1}\Wb^\dagger\vec{g}_k\rangle. 
\notag
\end{align}
%
It is easy to see that the functionals in equation
\eqref{eq:Functionals_kappa_alpha} have the following integral
representations, which are the discrete versions of equations
\eqref{eq:Int_kappa*_jk} and \eqref{eq:Int_alpha*_jk}. These integrals
involve a spectral measure $\d\mu_{jk}(\lambda)$ associated with the matrix
$\Ab$, that is given in terms of the spectral weights
$(\overline{\vec{w}_n^{\,\dagger}\vec{g}_j})(\vec{w}_n^{\,\dagger}\vec{g}_k)=\Qb_n\vec{g}_j\cdot\vec{g}_k$ 
which involve the mutually orthogonal, $\Qb_n\Qb_m=\Qb_n\delta_{nm}$,
Hermitian projection matrices $\Qb_n=\vec{w}_n\vec{w}_n^{\,\dagger}$,    
%
\begin{align}\label{eq:Integral_Rep_Discrete}
  \kappa^*_{jk}/\varepsilon-\delta_{jk}
  %\mathcal{F}(\Ab)  
 % =\sum_{n\in I_n} 
%  \left\langle\frac{\overline{(\vec{w}_n^{\,\dagger}\vec{g}_j)}(\vec{w}_n^{\,\dagger}\vec{g}_k)} 
%                     {(\varepsilon-\I\lambda_n)(\varepsilon+\I\lambda_n)}\right\rangle
 % =\sum_n 
%  \left\langle\frac{(\overline{\vec{w}_n^{\,\dagger}\vec{g}_j})(\vec{w}_n^{\,\dagger}\vec{g}_k)} 
%                     {\varepsilon^2+\lambda_n^2}\right\rangle                           
 =\sum_{n\in I_n} 
 \left\langle\frac{\Qb_n\vec{g}_j\cdot\vec{g}_k}{\varepsilon^2+\lambda_n^2}\right\rangle
 =\int_{-\infty}^\infty\frac{\d\mu_{jk}(\lambda)}{\varepsilon^2+\lambda^2}\,,
 \qquad 
 \alpha^*_{jk}
 %\mathcal{G}(\Ab)   
 =\int_{-\infty}^\infty\frac{-\I\lambda\,\d\mu_{jk}(\lambda)}{\varepsilon^2+\lambda^2}\,.
 %\d\mu_{jk}(\lambda)=\sum_n\delta_{\lambda_n}(\d\lambda)\Qb_n\vec{g}_j\cdot\vec{g}_k.
 %\quad
 %\Qb_n=\vec{w}_n\vec{w}_n^{\,\dagger}
 %\notag
\end{align}
%
Here, the \emph{complex measure} $\d\mu_{jk}(\lambda)$ for $j\neq k$ and
\emph{positve measure} $\d\mu_{kk}(\lambda)$ are given by 
% 
\begin{align}
  \d\mu_{jk}(\lambda)=\sum_{n\in I_n}\langle\delta_{\lambda_n}(\d\lambda)[\Qb_n\vec{g}_j\cdot\vec{g}_k]\rangle, \quad  
  \d\mu_{kk}(\lambda)=\sum_{n\in I_n}\langle\delta_{\lambda_n}(\d\lambda)|\Qb_n\vec{g}_k|^2\rangle, \qquad
  \Qb_n=\vec{w}_n\vec{w}_n^{\,\dagger}, 
\end{align}
%
where $\delta_{\lambda_n}(\d\lambda)$ is the delta measure concentrated at $\lambda_n$,
$\Qb_n$ is the projection matrix onto the eigen-space spanned by the
eigenvector $\vec{w}_n$, and
$|\vec{\xi}\,|$ denotes the $\ell^{\,2}(\mathbb{R}^N)$ norm of the vector
$\vec{\xi}\,$.   




Exactly as in equations \eqref{eq:Int_kappa*_jk_Re} and
\eqref{eq:Int_kappa*_jk_Im}, we may use the fact that the matrix
$\Ab$, vector $\vec{g}_j$, and molecular diffusivity $\varepsilon$ are
real-valued, to re-express the integrals in
\eqref{eq:Integral_Rep_Discrete} in terms of the \emph{signed
  measures} 
$\d\text{Re}\,\mu_{jk}(\lambda):=(\d\mu_{jk}(\lambda)+\d\mu_{kj}(\lambda))/2$ and
$\d\text{Im}\,\mu_{jk}(\lambda):=(\d\mu_{jk}(\lambda)-\d\mu_{kj}(\lambda))/(2\I)$. Since
$\Qb_n$ is a Hermitian projection matrix, we have
$[\Qb_n\vec{g}_k\cdot\vec{g}_j]=[\overline{\Qb}_n\vec{g}_j\cdot\vec{g}_k]$,
which implies that the dependence of the measures
$\d\text{Re}\,\mu_{jk}(\lambda)$ and $\d\text{Im}\,\mu_{jk}(\lambda)$ on $\lambda_n$ and
$\Qb_n$ are given by  
%
\begin{align}\label{eq:Signed_mu_Dis}
  &\d\text{Re}\,\mu_{jk}(\lambda)      
      =\frac{1}{2}\sum_{n\in I_n}\langle\delta_{\lambda_n}(\d\lambda)[(\Qb_n+\overline{\Qb}_n)\vec{g}_j\cdot\vec{g}_k]\rangle
      =\sum_{n\in I_n}\langle\delta_{\lambda_n}(\d\lambda)\text{Re}[\Qb_n\vec{g}_j\cdot\vec{g}_k]\rangle,
      \\
  &\d\text{Im}\,\mu_{jk}(\lambda)
      %:=\frac{1}{2\I}(\d\mu_{jk}(\lambda)-\d\mu_{kj}(\lambda))
      =\frac{1}{2\I}\sum_{n\in I_n}\langle\delta_{\lambda_n}(\d\lambda)[(\Qb_n-\overline{\Qb}_n)\vec{g}_j\cdot\vec{g}_k]\rangle
      =\sum_{n\in I_n}\langle\delta_{\lambda_n}(\d\lambda)\text{Im}[\Qb_n\vec{g}_j\cdot\vec{g}_k]\rangle.
      \notag
\end{align}
%
Moreover, since the eigenvalues $\upsilon_n=\I\lambda_n$ and eigenvectors
$\vec{w}_n$ of the matrix $\Ab$ come in complex conjugate pairs, the
integral representations in \eqref{eq:Integral_Rep_Discrete} may be
further simplified and shown \cite{Pavliotis:PHD_Thesis} to involve
the measures in \eqref{eq:Signed_mu_Dis} with a restriced index set
$\{n\in I_n:n\geq0\}$. This is clear from equations
\eqref{eq:Integral_Rep_Discrete} and \eqref{eq:Signed_mu_Dis} since
for $n\geq0$ we have $\lambda_{-n}^2=(-\lambda_n)^2=\lambda_n^2$ and from
$\vec{w}_{-n}=\overline{\vec{w}_n}$ we also have that
$\Qb_{-n}=\overline{\Qb}_n$. Together, this implies that
$\text{Re}[\Qb_n\vec{g}_j\cdot\vec{g}_k]+\text{Re}[\Qb_{-n}\vec{g}_j\cdot\vec{g}_k] 
=2\text{Re}[\Qb_n\vec{g}_j\cdot\vec{g}_k]$ and
$\lambda_n\text{Im}[\Qb_n\vec{g}_j\cdot\vec{g}_k]+\lambda_{-n}\text{Im}[\Qb_{-n}\vec{g}_j\cdot\vec{g}_k]
=2\lambda_n\text{Im}[\Qb_n\vec{g}_j\cdot\vec{g}_k]$ with
$\lambda_0\text{Im}[\Qb_0\vec{g}_j\cdot\vec{g}_k]\equiv0$.



A key feature of the integral representations for $\bkappa^*$ and
$\balpha^*$ in \eqref{eq:Integral_Rep_kappa*} is that parameter
information in $\varepsilon$ is \emph{separated} from the geometry and dynamics
of the velocity field $\vec{u}$, which are encapsulated in the
underlying spectral measure $\d\bmu$. In Section
\ref{sec:Assymptotics} we discuss in 
detail the properties of the spectrum $\Sigma(\Ab)$ of the operator $\Ab$,
which can be more exotic than pure point spectrum in the general
setting \cite{Reed-1980,Stone:64}. We show how these properties of
$\Sigma(\Ab)$ lead to useful decompositions of the measure $\d\bmu$. These
measure decompositions are employed along with the properties of the 
integrands of the integrals in \eqref{eq:Integral_Rep_kappa*} to
obtain asymptotic behavior of $\bkappa^*$ and $\balpha^*$ as the
molecular diffusivity tends to zero, $\varepsilon\to0$.     









\section{Spectral and asymptotic analysis of effective
  diffusivity} \label{sec:Assymptotics}
%
In Section \ref{sec:Integral_Rep} we derived the Radon--Stieltjes
integral representations for the symmetric $\bkappa^*$ and
anti-symmetric $\balpha^*$ parts of the effective diffusivity tensor
$\Kbc^*$ displayed in equation \eqref{eq:Integral_Rep_kappa*},
involving a spectral measure $\d\bmu$ associated with the maximal
normal operator $\Ab$ on the function space $\Fs$. There, we briefly
discussed that the domain of integration of these representations are
determined by the spectrum $\Sigma(\Ab)$ of the operator $\Ab$. In this
section we discuss the properties of $\Sigma(\Ab)$ in more detail and use
this important information to provide various decompositions of the
measure $\d\bmu$ which provides assymptotic behavior of the componets
$\kappa^*_{jk}$ and $\alpha^*_{jk}$, $j,k=1,\ldots,d$, as the molecular diffusivity
$\varepsilon\to0$.






\section{Numerical Results}\label{sec:Num_Results}
%
In this section we discuss the numerical implementation of the
integral representations in equation \eqref{eq:Integral_Rep_Discrete},
involving the signed, discrete measures displayed in
\eqref{eq:Signed_mu_Dis}. In particular, we directly compute the
spectral measure $\d\bmu(\lambda)$ associated with discretizations of
velocity fields $\vec{u}$ for model flows, to compute the symmetric
$\bkappa^*$ and anti-symmetric $\balpha^*$ parts of the associated
effective diffusivity tensor $\Kbc^*$. We explore the numerical
implementation of the different formulations of the effective
parameter problem, described by Corollary \ref{cor:Integral_Reps} and
Theorem \ref{thm:Integral_Reps} which involve the anti-symmetric
operators $A$ and $\Ab$, respectively. This analysis demonstrates that
formulation described by Corollary \ref{cor:Integral_Reps} provides a
more practical numerical implimentation than that described in Theorem
\ref{thm:Integral_Reps}, as $A$ is a \emph{sparse} matrix of size $N$,
say, and $\Ab$ is a \emph{full} matrix of size $dN$.    

%\newpage

% redefine the command that creates the equation no.
  \setcounter{equation}{1}  % reset equation counter
  \setcounter{section}{0}  % reset section counter
  \renewcommand{\theequation}{A-\arabic{equation}} 
\renewcommand{\thesection}{A-\arabic{section}}
%
\section{Appendix} 
\label{sec:Appendix}
%
\subsection{The flow matrix $\Hb$}\label{eq:flow_matrix}
%
THIS SECTION IS UNER CONSTRUCTION
%
\subsection{Multiple scale method}\label{sec:Multiscal_Method}
%
In this section we provide the details of the multiple scale method
\cite{McLaughlin:SIAM_JAM:780,Papanicolaou:1981:36:8,Papanicolaou:RF-835,Bensoussan:Book:1978}
which leads to equations
\eqref{eq:phi_bar}--\eqref{eq:Eff_Diffusivity}. We assume that
equation \eqref{eq:ADE} has already been non-dimensionalized so that
$\kappa_0\mapsto\varepsilon$ and $\vec{v}\mapsto\vec{u}$. The key assumption of the method is
that the initial density $\phi_0$ in \eqref{eq:ADE} is slowly
varying relative to the velocity field $\vec{u}$, which introduces a
small parameter $\delta\ll1$ such that  
% 
\begin{align}\label{eq:IC}
  \phi(0,\vec{x})=\phi_0(\delta\vec{x}).
\end{align}
%





The variable changes $\vec{x}\mapsto\vec{y}=\vec{x}/\delta$ and
$t\mapsto\tau=t/\delta^{\,2}$, along with equations \eqref{eq:incompressible} and
\eqref{eq:IC}, transforms equation \eqref{eq:ADE} into
\cite{McLaughlin:SIAM_JAM:780}    
%
\begin{align}\label{eq:ADE_delta}
  \partial_t\phi^\delta(t,\vec{x})=\varepsilon\Delta\phi^\delta(t,\vec{x})
                 +\delta^{\,-1}\,\vec{u}(\tau,\vec{y})\cdot\vec{\nabla}\phi^\delta(t,\vec{x}), \quad
      \phi^\delta(0,\vec{x})=\phi_0(\vec{x}).
\end{align}
%
We now expand $\phi^\delta$ in powers of $\delta$ \cite{McLaughlin:SIAM_JAM:780} 
%
\begin{align}\label{eq:Expand}
  \phi^\delta(t,\vec{x})=\bar{\phi}(t,\vec{x})
                 +\delta\phi^{(1)}(t,\vec{x},\tau,\vec{y})
                 +\delta^{\,2}\phi^{(2)}(t,\vec{x},\tau,\vec{y})+\cdots.
\end{align}
%
Writing
%
\begin{align*}
  \partial_t\phi^{(n)}=[\partial_t+\delta^{-2}\partial_\tau]\phi^{(n)}, \quad
  \vec{\nabla}\phi^{(n)}=[\vec{\nabla}_x+\delta^{-1}\vec{\nabla}_y]\phi^{(n)}, \quad
  \Delta\phi^{(n)}=[\Delta_x+2\delta^{-1}\vec{\nabla}_x\cdot\vec{\nabla}_y+\delta^{-2}\Delta_y]\phi^{(n)},
\end{align*}
%
for the functions $\phi^{(i)}$, $n=1,2,\ldots$, of the fast $(\tau,\vec{y})$ and
slow $(t,\vec{x})$ variables, we find that
%
\begin{align}
  &\partial_t\phi^\delta=\delta^{-2}[\partial_\tau\bar{\phi}]
      +\delta^{-1}[\partial_\tau\phi^{(1)}]
      +\delta^0[\partial_t\bar{\phi}+\partial_\tau\phi^{(2)}]+O(\delta),
      \\
  &\vec{\nabla}\phi^\delta=\delta^{-2}[0]
            +\delta^{-1}[\vec{\nabla}_y\bar{\phi}]
            +\delta^0[\vec{\nabla}_x\bar{\phi}+\vec{\nabla}_y\phi^{(1)}]
            +\delta^1[\vec{\nabla}_x\phi^{(1)}+\vec{\nabla}_y\phi^{(2)}]+O(\delta^2),
            \notag\\
  &\Delta\phi^\delta=\delta^{-2}[\Delta_y\bar{\phi}]
      +\delta^{-1}[2\vec{\nabla}_x\cdot\vec{\nabla}_y\bar{\phi}+\Delta_y\phi^{(1)}]
      +\delta^0[\Delta_x\bar{\phi}+2\vec{\nabla}_x\cdot\vec{\nabla}_y\phi^{(1)}+\Delta_y\phi^{(2)}]+O(\delta).
      \notag
\end{align}
%
Inserting this into equation \eqref{eq:ADE_delta} and setting the
coefficients associated with the various powers of $\delta$ to zero,
yields a sequence of problems.






Due to the dependence of $\bar{\phi}(t,\vec{x})$ on only the
slow variables, the coefficients of $\delta^{-2}$ vanish. Equating the
coefficients of $\delta^{-1}$ and $\delta^0$ to zero we, respectively, obtain 
%
\begin{align}
  \label{eq:Mult_Meth_EQs_1}
  &\partial_\tau\phi^{(1)}-\varepsilon\Delta_y\phi^{(1)}-\vec{u}\cdot\vec{\nabla}_y\phi^{(1)}=\vec{u}\cdot\vec{\nabla}_x\bar{\phi},
  \\
  \label{eq:Mult_Meth_EQs_2}
  &\partial_\tau\phi^{(2)}-\vec{u}\cdot\vec{\nabla}_y\phi^{(2)}-\varepsilon\Delta_y\phi^{(2)}
  =-\partial_t\bar{\phi}+\vec{u}\cdot\vec{\nabla}_x\phi^{(1)}+\varepsilon[\Delta_x\bar{\phi}+2\vec{\nabla}_x\cdot\vec{\nabla}_y\phi^{(1)}].
\end{align}
%
By the linearity of equation \eqref{eq:Mult_Meth_EQs_1}, we may
separate the fast and slow variables by writing
%\cite{McLaughlin:SIAM_JAM:780,Biferale:PF:2725}
\cite{McLaughlin:SIAM_JAM:780}
%
\begin{align}\label{eq:Linearity}
  \phi^{(1)}(t,\vec{x},\tau,\vec{y})
    =%\hat{\phi}^{(1)}(t,\vec{x})+
    \vec{\chi}(\tau,\vec{y})\cdot\vec{\nabla}_x\bar{\phi}(t,\vec{x}).
\end{align}
%
When
%$\hat{\phi}^{(1)}$ is a homogeneous solution of
%\eqref{eq:Mult_Meth_EQs_1} and
the components $\chi_k$, $k=1,\ldots,d$, of $\vec{\chi}$ satisfy   
%
\begin{align}\label{eq:Cell_Prob_Appendix}
  \partial_\tau\chi_k-\varepsilon\Delta_y\chi_k-\vec{u}\cdot\vec{\nabla}_y\chi_k=\vec{u}\cdot\vec{e}_k,
\end{align}
%
equation \eqref{eq:Mult_Meth_EQs_1} is automatically satisfied
\cite{McLaughlin:SIAM_JAM:780}. Equation \eqref{eq:Cell_Prob_Appendix}
along with \eqref{eq:u_DH} is equivalent to the cell problem
\eqref{eq:Cell_Problem}, where the distinction of fast variables was
dropped for notational simplicity. In order for
$\phi^{(1)}(t,\vec{x},\tau,\vec{y})$ in \eqref{eq:Linearity} to be periodic  
in $(\tau,\vec{y})$ for each fixed $(t,\vec{x})$, we must have that the
functions $\chi_k(\tau,\vec{y})$, $k=1,\ldots,d$, are periodic. This and the
fundamental theorem of calculus implies that $\langle\vec{\nabla}_y\chi_k\rangle=0$. Here,
$\langle\cdot\rangle$ denotes space-time averaging with respect to the \emph{fast
  variables}.    



Due to the incompressibility of the velocity field
$\vec{\nabla}_y\cdot\vec{u}(\tau,\vec{y})=0$ and the \emph{a priori} fast variable
periodicity of the functions $\phi^{(i)}$, $i=1,2$,  the
fundamental theorem of calculus and the divergence theorem shows that
the average of the left-hand-sides of equations
\eqref{eq:Mult_Meth_EQs_1} and \eqref{eq:Mult_Meth_EQs_2} are
zero. For the equations to have solutions, the average of the
right-hand-sides must also vanish.
%\cite{Biferale:PF:2725} (Fredholm Algernative).
The resulting solvability conditions are $\langle\vec{u}\rangle=0$
and the following equation which governs the large-scale (slow
variable) dynamics   
%
\begin{align}\label{eq:Avg_EQ_phi}
  \partial_t\bar{\phi}=\varepsilon\Delta_x\bar{\phi}+\langle\vec{u}\cdot\vec{\nabla}_x\phi^{(1)}\rangle.
\end{align}
%
Here, we have used that $\bar{\phi}$ is a \emph{constant} with respect to
the fast variables and, by the divergence theorem and the fast
variable periodicity of $\phi^{(1)}$, we have
$\langle\vec{\nabla}_y\cdot\vec{\nabla}_x\phi^{(1)}\rangle=0$. The convergence of $\phi^\delta$ to $\bar{\phi}$ as
$\delta\to0$ is in $L^2$ \cite{Fannjiang:SIAM_JAM:333},  
%
\begin{align}
  \lim_{\delta\to0}\left[\,\sup_{0\leq t\leq t_0}
    \int\left|\phi^\delta(t,\vec{x})-\bar{\phi}(t,\vec{x})\right|^2\d\vec{x}
    \;\right]=0,
\end{align}
%
for all $t_0<\infty$, where we have used the notation $\d\vec{x}=\d x_1\cdots \d
x_d$ for the product Lebesgue measure.





Inserting equation \eqref{eq:Linearity} into \eqref{eq:Avg_EQ_phi}
yields equation \eqref{eq:phi_bar} with the components
$\Kc^*_{jk}=\Kbc^*\vec{e}_j\cdot\vec{e}_k$ of the effective diffusivity
tensor $\Kbc^*$ given by 
%
\begin{align}\label{eq:Eff_Diffusivity_Appendix}
  \Kc^*_{jk}=\varepsilon\delta_{jk}+\langle u_j\chi_k\rangle.
\end{align}
%
By inserting the representation for $u_j$ in
\eqref{eq:Cell_Prob_Appendix} into equation
\eqref{eq:Eff_Diffusivity_Appendix}, the functional $\langle u_j\chi_k\rangle$ can be
represented in terms of $\vec{\nabla}_y\chi_j$ and the \emph{skew-symmetric}
operator $\Sb=\Hb-\Delta_y^{-1}\Tb$, where the inverse operation $\Delta_y^{-1}$
is based on convolution with the Green's function for the Laplacian
$\Delta_y$, $\Tb=\partial_\tau\Ib$, and the $\Ib$ in this definition is to remind us
that the derivative $\partial_\tau$ operates component-wise. Indeed, writing  
$\partial_\tau\chi_j
%=\Delta_y\Delta_y^{-1}\partial_\tau\chi_j
=\vec{\nabla}_y\cdot(\Delta_y^{-1}\Tb)\vec{\nabla}_y\chi_j$,
$\Delta_y\chi_j=\vec{\nabla}_y\cdot\vec{\nabla}_y\chi_j$, and $\vec{u}=\vec{\nabla}_y\cdot\Hb$ in
\eqref{eq:u_DH}, we have    
%
\begin{align}\label{eq:Functional_Rep}
  \langle u_j\chi_k\rangle&=\langle[\partial_\tau\chi_j-\varepsilon\Delta_y\chi_j-\vec{u}\cdot\vec{\nabla}_y\chi_j]\chi_k\rangle
       \\
       &=\langle\vec{\nabla}_y\cdot[(\Delta_y^{-1}\Tb-\varepsilon\Ib-\Hb)\vec{\nabla}_y\chi_j]\chi_k\rangle
       \notag\\
       &=\langle[(\Hb-\Delta_y^{-1}\Tb+\varepsilon\Ib)\vec{\nabla}_y\chi_j]\cdot\vec{\nabla}_y\chi_k\rangle
       \notag\\
       &=\langle\Sb\vec{\nabla}_y\chi_j\cdot\vec{\nabla}_y\chi_k\rangle+\varepsilon\langle\vec{\nabla}_y\chi_j\cdot\vec{\nabla}_y\chi_k\rangle,
       \notag
\end{align}
%
where we have used the periodicity of $\chi_k$ and $\Hb$ to obtain the
third equality. Equations \eqref{eq:Eff_Diffusivity_Appendix} and
\eqref{eq:Functional_Rep} are equivalent to equations
\eqref{eq:Symm_Anti-Symm} and \eqref{eq:Eff_Diffusivity}, where the
distinction of fast variables was dropped for notational simplicity. 





The above analysis shows that the main part of the study of effective, 
diffusive transport enhanced by periodic, incompressible flows, is the
study of equation \eqref{eq:Cell_Prob_Appendix}, from which the
effective diffusivity tensor $\Kbc^*$ emerges. In Section
\ref{sec:Hilbert_Space}, we use the analytical structure of the cell
problem \eqref{eq:Cell_Prob_Appendix} to derive a resolvent
representation for $\vec{\nabla}_y\chi_k$, involving an anti-symmetric
integro-differential operator $\Ab$ which is related to 
$\Sb=\Hb-\bDelta^{-1}\partial_t\Ib$. In Section \ref{sec:Integral_Rep}, 
we employ this representation for $\vec{\nabla}_y\chi_k$ and the spectral
theorem, to provide integral representations for $\bkappa^*$ and
$\balpha^*$ involving a \emph{spectral measure} associated with the
operator $\Ab$ acting on a suitable Hilbert space.     
   

\subsection{Symmetries and commutivity}\label{sec:Symmetries_Commute}
%
THIS SECTION IS UNER CONSTRUCTION

We now show that the incompressibility of $\vec{u}$ in
\eqref{eq:incompressible} implies that the operator
$(\Delta^{-1})(\vec{u}\cdot\vec{\nabla})$ is anti-symmetric on $\Fc$
\cite{Bhattacharya:AAP:1999:951}. Indeed, since $\Delta=\vec{\nabla}\cdot\vec{\nabla}$
and $(\Delta^{-1})$ is self-adjoint on $\Hc^1_{\Vc}$, for $f,h\in\Fc$ we have 
%
\begin{align}\label{eq:Anti-sym_Sobolev}
  \langle(\Delta^{-1})(\vec{u}\cdot\vec{\nabla})f,h\rangle_1
                            &=\langle[\vec{\nabla}(\Delta^{-1})(\vec{u}\cdot\vec{\nabla})f]\cdot\vec{\nabla}h\rangle
                                 \\                              
                              &=-\langle[(\vec{u}\cdot\vec{\nabla})f]\,h\rangle
                                 \notag\\
                               &=-\langle[\vec{\nabla}\cdot(\vec{u}f)]\,h\rangle
                                 \notag\\     
                              &=\langle f\,[(\vec{u}\cdot\vec{\nabla})h]\rangle
                                \notag\\
                              &=\langle(\Delta^{-1})\Delta f\,[(\vec{u}\cdot\vec{\nabla})h]\rangle
                                \notag\\
                              &=-\langle\vec{\nabla}f\,[\vec{\nabla}(\Delta^{-1})(\vec{u}\cdot\vec{\nabla})h]\rangle
                                \notag\\                              
                              &=-\langle f,(\Delta^{-1})(\vec{u}\cdot\vec{\nabla})h\rangle_1
                              \notag
\end{align}
%
%
\subsection{Existence and Uniqueness}\label{sec:Existance!}
%
THIS SECTION IS UNER CONSTRUCTION

Before we discuss how the Hilbert space framework presented above
leads to an  integral representation for $\Kbc^*$, we first discuss
some key differences in the theory between the cases of steady 
and dynamic velocity fields $\vec{u}$. These differences are reflected
in the measure underlying this integral representation for $\bkappa^*$
and stem from the \emph{unboundedness} of the operator $\partial_t$ on the
Hilbert space $\Hs_{\,\Tc}$ \cite{Reed-1980,Stone:64}. For steady
$\vec{u}$, in general, equation \eqref{eq:sigma*} reduces to
\eqref{eq:Eff_Diffusivity} for  diagonal components of the effective
parameter.  However, for dynamic $\vec{u}$, this is not true in
general. The details are as follows. For dynamic $\vec{u}$, the
operator $\bsig$ in \eqref{eq:Maxwells_Equations} can be written as
$\bsig=\varepsilon\Ib+\Sb$, where  $\Sb=\Hb-\bDelta^{-1}\partial_t\Ib$ is skew-symmetric 
$\langle\Sb\vec{\xi},\vec{\zeta}\,\rangle=-\langle\vec{\xi},\Sb\vec{\zeta}\,\rangle$ for all
$\vec{\xi},\vec{\zeta}\in\Fs$ such that
$|\langle\partial_t\vec{\xi},\vec{\zeta}\,\rangle|,|\langle\vec{\xi},\partial_t\vec{\zeta}\,\rangle|<\infty$ (see Section
\ref{sec:Appendix} for details).  
This property of the operator $\Sb$ implies that
%$\langle\Sb\vec{E}_k\cdot\vec{E}_k\rangle=-\langle\Sb\vec{E}_k\cdot\vec{E}_k\rangle=0$.
%
\begin{align}\label{eq:Sb_Skew_2}
  \langle\Sb\vec{\xi}\cdot\vec{\xi}\,\rangle=-\langle\Sb\vec{\xi}\cdot\vec{\xi}\,\rangle=0,
  \qquad
  \Sb=\Hb-(\bDelta^{-1})\partial_t\Ib,
\end{align}
%
for all such $\vec{\xi}\in\Fs$. In this dynamic setting, equation
\eqref{eq:Sb_Skew} does not hold for every $\vec{\xi}\in\Fs$, as the
unbounded operator $\partial_t$ is defined only on a proper (dense) subset of
the Hilbert space $\Hs_{\,\Tc}$ \cite{Reed-1980}, and it may be that
$|\langle\partial_t\vec{\xi},\vec{\xi}\,\rangle|=\infty$. In the case of a steady velocity field
we have $\Sb\equiv\Hb$ and, by equation \eqref{eq:Bounded_H} and the Cauchy
Schwartz inequality, $|\langle\Sb\vec{\xi},\vec{\xi}\,\rangle|\leq\|\Hb\|\|\vec{\xi}\,\|^2<\infty$ for
all $\vec{\xi}\in\Fs$, so equation \eqref{eq:Sb_Skew} holds for all
$\vec{\xi}\in\Fs$.   









Another immediate consequence of equation \eqref{eq:Sb_Skew}, for
steady $\vec{u}$, is the coercivity of the bilinear functional
$\Phi(\vec{\xi},\vec{\zeta}\,)=\langle\bsig\vec{\xi}\cdot\vec{\zeta}\,\rangle$ for all $\varepsilon>0$. By equation
\eqref{eq:Bounded_H}, this functional is also bounded in the case of
steady $\vec{u}$ for all $\varepsilon<\infty$. Therefore, the Lax-Milgram theorem
\cite{McOwen:2003:PDE} provides the existence and uniqueness of a
solution $\vec{\nabla}\chi_k\in\Fs$ satisfying the cell problem
\eqref{eq:Cell_Problem}, or equivalently equation
\eqref{eq:Maxwells_Equations}, in this time-independent case. The
details are as follows. 




The distributional form of equation \eqref{eq:Cell_Problem}, written
as $\vec{\nabla}\cdot\bsig\vec{E}_k=0$, is given by
$\langle\bsig(\vec{\nabla}\chi_k+\vec{e}_k)\cdot\vec{\nabla}\zeta\rangle=0$, where $\zeta$ is a compactly
supported, infinitely differentiable function on $\Tc\otimes\Vc$, and we
stress that $\vec{\nabla}\zeta$ is \emph{curl-free}. Motivated by this, we
consider the following variational problem: find $\vec{\nabla}\chi_k\in\Fs$ such
that   
%
\begin{align}\label{eq:Variational}
  \langle\bsig(\vec{\nabla}\chi_k+\vec{e}_k)\cdot\vec{\xi}\,\rangle=0, \text{ for all }
  \vec{\xi}\in\Fs.
\end{align}
%
In order to directly apply the Lax-Milgram Theorem, we rewrite
equation \eqref{eq:Variational} as 
%
\begin{align}  \label{eq:Bilinear_functional_E} 
   &\Phi(\vec{\nabla}\chi_k,\vec{\xi}\,)
     =\langle\bsig\vec{\nabla}\chi_k\cdot\vec{\xi}\,\rangle
     =-\langle\bsig\vec{e}_k\cdot\vec{\xi}\,\rangle
     =f(\vec{\xi}\,). 
\end{align}
%
By equation \eqref{eq:Sb_Skew} $\Phi$ is coercive, i.e.
%$\Phi(\vec{\xi},\vec{\xi})=\langle[(\varepsilon\Ib+\Sb)]\vec{\xi}\cdot\vec{\xi}\rangle=\varepsilon\|\vec{\xi}\|^2>0$ 
%
\begin{align}\label{eq:Phi_Coercive}
  \Phi(\vec{\xi},\vec{\xi}\,)=\langle[(\varepsilon\Ib+\Sb)]\vec{\xi}\cdot\vec{\xi}\,\rangle=\varepsilon\|\vec{\xi}\,\|^2>0,
   \text{ for all } \vec{\xi}\in\Fs
\end{align}
%
such that $\|\vec{\xi}\,\|\neq0$ and $\varepsilon>0$, where $\|\cdot\|$
is the norm induced by the inner-product $\langle\cdot,\cdot\rangle$. Recall that
$\Sb=\Hb$ in this time-independent case. This, equation 
\eqref{eq:Bounded_H}, the triangle inequality,
and the Cauchy-Schwartz inequality imply that $\Phi$ is also bounded for
all $\varepsilon<\infty$
%for all $\vec{\xi},\vec{\zeta}\in\Fs$
% $\Phi(\vec{\xi},\vec{\zeta})\leq(\varepsilon+\|H\|)\|\vec{\xi}\|\|\vec{\zeta}\|$
%
\begin{align}\label{eq:Phi_Bounded}
  \Phi(\vec{\xi},\vec{\zeta})\leq(\varepsilon+\|H\|)\|\vec{\xi}\,\|\|\vec{\zeta}\,\|<\infty,
  \text{ for all } \vec{\xi}\in\Fs.
\end{align}
%
For the same reasons, the linear functional $f(\vec{\xi}\,)$ in
\eqref{eq:Bilinear_functional_E} is bounded for all
$\vec{\xi}\in\Fs$. Therefore, the Lax-Milgram theorem
\cite{McOwen:2003:PDE} provides the existence of a unique
$\vec{\nabla}\chi_k\in\Fs$ satisfying \eqref{eq:Cell_Problem} in this
time-independent case. 


In the time-dependent case, equation \eqref{eq:Sb_Skew} hence
\eqref{eq:Phi_Coercive} does not hold for all $\vec{\xi}\in\Fs$. Moreover, 
the operator $\partial_t$ hence $\bsig$ is not bounded on $\Fs$
\cite{Reed-1980,Stakgold:BVP:2000}, so \eqref{eq:Phi_Bounded} does not
hold. Consequently, the Lax-Milgram theorem cannot be directly
applied, and alternate techniques  
\cite{Friedman:1969:PDE,Friedman:1969:PDE:Parabolic} must be used to
prove the existence and uniqueness of a solution $\vec{\nabla}\chi_k\in\Fs$ 
satisfying the cell problem \eqref{eq:Cell_Problem}. This discussion
illustrates key differences in the analytic structure of the effective
parameter problem for $\bkappa^*$, between the cases of steady and
dynamic velocity fields $\vec{u}$, which stem from the unboundedness
of the operator $\partial_t$ on $\Hs_{\,\Tc}$, hence $\bsig$ on $\Fs$. In Section
\ref{sec:Integral_Rep}, we will discuss other consequences of this
boundedness/unboundedness property of the operator $\bsig$, and
demonstrate that it leads to significant differences in the spectral
measure underlying an  integral representation of $\bkappa^*$.     






\medskip

{\bf Acknowledgements.}
% We gratefully acknowledge support from the Division of Mathematical
% Sciences and the Office of Polar Programs at the U.S. 
% National Science Foundation (NSF) through Grants
% DMS-1009704, ARC-0934721, and DMS-0940249. We are also grateful for 
% support from the Office of Naval Research (ONR) through
% Grants N00014-13-10291 and N00014-12-10861. Finally, we would like to 
% thank the NSF Math Climate Research Network (MCRN) for their support
% of this work. 


\medskip

\bibliographystyle{plain}
\bibliography{murphy}
\end{document}

% LocalWords:  McMaster RM jk sig eps def Maxwells Milgram coercivity diag jX
% LocalWords:  mh Cond mu kk PtI Fs Es Hashin Shtrikman extremized Decomp chi
% LocalWords:  Acknowledgements DMS ONR MCRN murphy