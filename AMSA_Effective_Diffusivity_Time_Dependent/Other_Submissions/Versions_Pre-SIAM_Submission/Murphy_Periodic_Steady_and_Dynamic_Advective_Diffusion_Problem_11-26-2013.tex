\documentclass[11pt]{amsart}
%\usepackage{latexsym, amssymb, enumerate, amsmath}
\usepackage{graphicx,amssymb,amsmath,amsfonts,mathrsfs}

\setlength{\textwidth}{6.5in}
\setlength{\textheight}{9.0in}
\setlength{\oddsidemargin}{0in}
\setlength{\evensidemargin}{0in}
\setlength{\topmargin}{-0.5in}

\renewcommand{\topfraction}{0.85}
\renewcommand{\textfraction}{0.1}
\renewcommand{\floatpagefraction}{0.55}%0.75


\newcommand{\ph}{\hat{\phi}}
\newcommand{\pt}{\tilde{\phi}} 
\newcommand{\pc}{\check{\phi}}
\newcommand{\gh}{\hat{\gamma}}
\newcommand{\Dh}{\hat{\Delta}}
\newcommand{\dha}{\hat{\delta}}
\newcommand{\qh}{\hat{q}}
\newcommand{\xh}{\hat{x}}
\newcommand{\HM}{\mathcal{H}_{\text{max}}}
\newcommand{\Hm}{\mathcal{H}_{\text{min}}}
\newcommand{\sech}{\rm \hspace{0.7mm}sech}
\newcommand{\I}{\mathrm{i}}
\newcommand{\e}{\mathrm{e}}
\newcommand{\hh}{\hat{h}}
\newcommand{\mh}{m_r}
\newcommand{\mt}{m_i}
\newcommand{\Mb}{\mathbf{M}}
\newcommand{\Xb}{\mathbf{X}}
\newcommand{\Tb}{\mathbf{T}}
\newcommand{\Hb}{\mathbf{H}}
\newcommand{\Kb}{\mathbf{K}}
\newcommand{\Jb}{\mathbf{J}}
\newcommand{\Ib}{\mathbf{I}}

\newcommand\bsig{\mbox{\boldmath${\sigma}$}}
\newcommand\beps{\mbox{\boldmath${\epsilon}$}}
\newcommand\bxi{\mbox{\boldmath${\xi}$}}
\newcommand\bmu{\mbox{\boldmath${\mu}$}}
\newcommand\balpha{\mbox{\boldmath${\alpha}$}}
\newcommand\brho{\mbox{\boldmath${\rho}$}}
\newcommand\bDelta{\mbox{\boldmath${\Delta}$}}
\newcommand\bkappa{\mbox{\boldmath${\kappa}$}}


\newtheorem{thm}{Theorem}[section]
\newtheorem{prop}[thm]{Proposition}
\newtheorem{lem}[thm]{Lemma}
\newtheorem{cor}[thm]{Corollary}

    %\theoremstyle{definition}

\newtheorem{defn}[thm]{Definition}
\newtheorem{notation}[thm]{Notation}
\newtheorem{example}[thm]{Example}
\newtheorem{conj}[thm]{Conjecture}
\newtheorem{prob}[thm]{Problem}

    %\theoremstyle{remark}

\newtheorem{rem}[thm]{Remark}
    % Use the standard latex environments for theorems, etc. Here is one
          % possible method of declaring them: It numbers all results by the
          % section, and uses a common numbering system for the different
          % environmentts.

\begin{document}

\title{Spectral theory of advective diffusion \\
  by dynamic and steady periodic flows}


% AUTHORS 
%\author{N. B. Murphy, A. Gully, E. Cherkaev, and K. M. Golden}
\author{N. B. Murphy$^*$}
\address{$^*$Department of Mathematics, 340 Rowland Hall, University of
  California at Irvine, Irvine, CA 92697-3875, USA}
\email{nbmurphy@math.uci.edu}

\author{J. Xin$^{\dag}$}
\address{$^{\dag}$Department of Mathematics, 340 Rowland Hall, University of
  California at Irvine, Irvine, CA 92697-3875, USA} 
\email{jxin@math.uci.edu}



\maketitle
\vspace{-3ex}
\begin{center}
  Department of Mathematics, University of California at Irvine
\end{center}

%\vspace{3ex}


\begin{abstract}
%
The analytic continuation method for representing transport in
composites provides Stieltjes integral representations for the
effective coefficients of two-phase random media. Here we adapt this 
method to characterize the effective thermal transport properties of
advective diffusion, by steady and time-dependent, periodic flows. Our
novel approach yields a Stieltjes integral representation for the
effective diffusivity, which holds for dynamic and steady,
incompressible flows, involving the spectral measure of a self-adjoint
operator.
In the case of steady fluid velocity fields, the spectral
measure is associated with a Hermitian Hilbert-Schmidt integral
operator, and in the case of dynamic flows, it is associated with an
unbounded integro-differential operator.
We utalize the integral
representation to obtain asymptotic behavior of the effective
diffusivity, as the molecular diffusivity tends to zero, for model,
steady and dynamic flows. Our analytical results are supported by
numerical computations of the spectral measures and effective
diffusivities.    
%
\end{abstract}

\section{Introduction}\label{sec:Introduction}
%
The long time, large scale behavior of a diffusing particle   
or tracer being advected by an incompressible velocity field 
is equivalent to an enhanced diffusive process \cite{Taylor:PRSL:196} 
with an effective diffusivity tensor $\bkappa^*$.
Determining the effective transport properties of advection enhanced
diffusion is a challenging problem with theoretical and practical 
importance in many fields of science and engineering,
ranging from turbulent combustion to mass, heat, and salt transport in
geophysical flows \cite{Moffatt:RPP:621}. A broad range of
mathematical techniques have been developed that reduce the analysis
of complex fluid flows, with rapidly varying structures in space and
time, to solving averaged or \textit{homogenized} equations that do
not have rapidly varying data, and involve an effective parameter.




Homogenization of the advection-diffusion equation for thermal
transport by time-independent, random fluid velocity fields was
treated in \cite{McLaughlin:SIAM_JAM:780}. This 
reduced the analysis of turbulent diffusion to solving an anisotripic
diffusion equation involving a homogenized temperature and an
effective diffusivity tensor $\bkappa^*$. An important consequence of
this analysis is that $\bkappa^*$ is given in terms  
of a \emph{curl-free} stationary stochastic process which satisfies a
steady state diffusion equation involving a skew-symmetric random
matrix $\Hb$ \cite{Avellaneda:CMP-339,Avellaneda:PRL-753}. By adapting
the analytic continuation method (ACM) of homogenization theory for
composites \cite{Golden:CMP-473}, it was shown that the result in
\cite{McLaughlin:SIAM_JAM:780} leads to a Stieltjes integral
representation of $\bkappa^*$, involving a spectral measure of a
self-adjoint random operator
\cite{Avellaneda:CMP-339,Avellaneda:PRL-753}. This integral
representation of $\bkappa^*$ was generalized to the time-dependent
case in \cite{Avellaneda:PRE:3249,Biferale:PF:2725}. Remarkably, this 
method has also been extended to flows with incompressible
\emph{nonzero} effective drift 
\cite{Pavliotis:PHD_Thesis}, flows where particles diffuse according
to linear collisions \cite{Pavliotis:IMAJAM:951}, and solute transport
in porous media \cite{Bhattacharya:AAP:1999:951}. All these approaches
yield Stieltjes integral representations of the symmetric and, when
appropriate, the antisymmetric part of $\bkappa^*$.




Homogenization of the advection-diffusion equation for periodic or
cellular, incompressible flow fields was treated in
\cite{Fannjiang:SIAM_JAM:333,Fannjiang:1997:1033}. As in the case of
random flows, the effective diffusivity tensor
$\bkappa^*$ is given in terms of a \emph{curl-free} vector field which
satisfies a diffusion equation involving a skew-symmetric
matrix $\Hb$. Here, we demonstrate that the ACM can
be adapted to this periodic setting to provide a Stieltjes
integral representation for $\bkappa^*$, both for steady and
time-dependent flows, involving a self-adjoint operator and the
(non-dimensional) molecular diffusivity $\varepsilon$. In the case of steady
fluid velocity fields, the spectral measure is associated with a
Hermitian Hilbert-Schmidt integral operator involving the Green's
function of the Laplacian on a rectangle. While in the case of dynamic
flows, the spectral measure is associated with a Hermitian operator
which is the sum of that for steady flows and an unbounded
integro-differential operator which is a composition of the time
derivative and the inverse Laplacian on a rectangle.     
 

We utalize the analytic structure of the Stieltjes integral
representation for $\bkappa^*$ to obtain its asymptotic behavior for
model flows, as the molecular diffusivity $\varepsilon$ tends to zero. This is
the high P\'{e}clet number regime that is important for the
understanding of transport processes in real fluid flows, where the
molecular diffusivity is often quite small in comparison. In
particular, FINISH THIS PARAGRAPH WHEN WE HAVE CONCRETE RESULTS.
necessary and sufficient conditions for steady periodic flow
fields $\kappa^*\sim\epsilon^{1/2}$, generically, for steady flows and $\kappa^*\sim O(1)$ for
``chaotic'' time-dependent flows. 

%
\section{Mathematical Methods}\label{sec:Mathematical_Methods} 
%

Consider the advection enhanced difusion of a passive tracer
$\phi(t,\vec{x})$, $\vec{x}\in\mathbb{R}^d$, $t>0$, as described by the
advection-diffusion equation 
%
\begin{align}\label{eq:ADE}
  \partial_t\phi=\kappa_0\Delta \phi+\vec{v}\cdot\vec{\nabla}\phi, \quad
  \phi(0,\vec{x})=\phi_0(\vec{x}),
\end{align}
%
with $\phi_0(\vec{x})$ given. Here, $\partial_t$ denotes partial differentiation
with respect to time $t$, $\Delta=\vec{\nabla}\cdot\vec{\nabla}=\nabla^2$ is the Laplacian,
$\kappa_0>0$ is the molecular diffusivity, and $\vec{v}=\vec{v}(t,\vec{x})$
is the fluid velocity field, which is assumed to be incompressible,
i.e. $\vec{\nabla}\cdot\vec{v}=0$. We non-dimensionalize equation
\eqref{eq:ADE} as follows. Let $\ell$ and $\tau$ be typical length and time 
scales associated with the problem of interest. Mapping to
non-dimensional coordinates $t\mapsto t/\tau$ and $x_i\mapsto x_i/\ell$ in
\eqref{eq:ADE}, one sees that $\phi$ satisfies the advection-diffusion
equation in \eqref{eq:ADE} with a non-dimensional molecular
diffusivity $\varepsilon=\tau\kappa_0/\ell^2$ and velocity field $\vec{u}=\tau\vec{v}/\ell$,
where $x_i$ is the $i^{\text{th}}$ component of the vector
$\vec{x}$. Since $\vec{u}$ is incompressible, there is a
(non-dimensional) skew-symmetric matrix $\Hb(t,\vec{x})$,
$\Hb^{\,T}=-\Hb$, such that $\vec{u}=\vec{\nabla}\cdot\Hb$. 
% %
% \begin{align}
%  \vec{u}=\vec{\nabla}\cdot\Hb, \qquad  \Hb^{\,T}=-\Hb.
% \end{align}
% %
Using this representation of the velocity field $\vec{u}$, equation
\eqref{eq:ADE} can be written in divergence form, 
%
\begin{align}\label{eq:ADE_Divergence}
  \partial_t\phi%&=\varepsilon\Delta \phi+\vec{u}\cdot\vec{\nabla}\phi\\
    %&=\varepsilon\vec{\nabla}\cdot\vec{\nabla}\phi+(\vec{\nabla}\cdot\Hb)\cdot\vec{\nabla}\phi\\
    %&=\vec{\nabla}\cdot[\varepsilon I+\Hb]\vec{\nabla}\phi\\
    %&=\vec{\nabla}\cdot\bkappa\vec{\nabla}\phi
    =\vec{\nabla}\cdot\bkappa\vec{\nabla}\phi, \quad
    %\bkappa=\varepsilon I+\Hb,
    \phi(0,\vec{x})=\phi_0(\vec{x}),
    \qquad
    \bkappa=\varepsilon\Ib+\Hb,
\end{align}
%
where $\bkappa(t,\vec{x})=\varepsilon\Ib+\Hb(t,\vec{x})$ can be viewed as a local
diffusivity tensor with coefficients
%
\begin{align}\label{eq:kappa_coeff}
  \kappa_{jk}=\varepsilon\delta_{jk}+H_{jk},\quad j,k=1,\cdots,d,
\end{align}
%
$\Ib$ is the identity matrix on $\mathbb{R}^d$, and $\delta_{jk}$ is 
the Kronecker delta. 




We are interested in the dynamics of $\phi$ for \emph{large} length and
time scales. Anticipating that $\phi$ will have diffusive dynamics, we
rescale space and time by $\vec{x}\mapsto\vec{x}/\delta$ and $t\mapsto t/\delta^2$,
respectively, with $\delta\ll1$ while keeping the intital condition
$\phi(0,\vec{x})=\phi_0(\vec{x})$ independent 
of $\delta$. This is equivalent to assuming that the initial data is slowly
varying relative to the velocity field $\vec{u}$ in the unscaled
variables
\cite{McLaughlin:SIAM_JAM:780,Fannjiang:SIAM_JAM:333,Fannjiang:1997:1033}. 
For periodic diffusivity coefficients in \eqref{eq:kappa_coeff} which
are uniformly elliptic but not necessarily symmetric, it can be shown 
\cite{Fannjiang:SIAM_JAM:333} that the associated solution
$\phi^\delta(t,\vec{x})=\phi(t/\delta^2,\vec{x}/\delta)$  of \eqref{eq:ADE_Divergence},
involving the rescaled local diffusivity tensor
$\bkappa(t/\delta^2,\vec{x}/\delta)$, converges to $\bar{\phi}(t,\vec{x})$, which
satisfies an anisotropic diffusion equation involving the (constant)
diffusivity tensor $\bkappa^*$
%
\begin{align}
  \partial_t\bar{\phi}=\vec{\nabla}\cdot\bkappa^*\vec{\nabla}\bar{\phi}, \quad
  \bar{\phi}(0,\vec{x})=\phi_0(\vec{x}).
\end{align}
%
The convergence is in $L^2$ \cite{Fannjiang:SIAM_JAM:333},
%
\begin{align}
  \lim_{\delta\to0}\left[\,\sup_{0\leq t\leq t_0}
    \int\left|\phi^\delta(t,\vec{x})-\bar{\phi}(t,\vec{x})\right|^2d\vec{x}
    \;\right]=0,
\end{align}
%
for all $t_0<\infty$, where we used the notation $d\vec{x}=dx_1\cdots dx_d$ for
the product Lebesgue measure.





The effective diffusivity tensor $\bkappa^*$ is
obtained by solving the cell problem \cite{Fannjiang:SIAM_JAM:333}
%
\begin{align}\label{eq:Cell_Problem}
  \partial_t\chi=\vec{\nabla}\cdot\bkappa(\vec{\nabla}\chi+\vec{e}_j\,), \quad
    %=\vec{\nabla}\cdot[(\varepsilon\Ib+\Hb)(\vec{\nabla}\chi+\vec{e}_j\,)], \quad
  \langle\vec{\nabla}\chi\rangle=0,
\end{align}
%
for each standard basis vector $\vec{e}_j$, $j=1,\ldots,d$, where
$\chi=\chi(t,\vec{x}\,;\vec{e}_j)$. Equation \eqref{eq:Cell_Problem} also holds
\cite{Fannjiang:SIAM_JAM:333} when the velocity field is
time-independent, $\vec{u}=\vec{u}(\vec{x})$, however, in this case $\chi$ 
is time-independent and $\partial_t\chi=0$.  The 
components $\kappa^*_{jk}=\bkappa^*\vec{e}_j\cdot\vec{e}_k$ of the effective
diffusivity tensor are given in terms of the \emph{curl-free} vector
field $\vec{\nabla}\chi$ \cite{Fannjiang:SIAM_JAM:333} 
% 
\begin{align}\label{eq:Eff_Diffusivity}  
   \kappa^*_{jk}=\varepsilon\langle(\vec{\nabla}\chi+\vec{e}_i\,)\cdot(\vec{\nabla}\chi+\vec{e}_j\,)\rangle
       =\varepsilon(\delta_{jk}+\langle\vec{\nabla}\chi\cdot\vec{\nabla}\chi\rangle).
\end{align}
%.
Here, $\langle\cdot\rangle$ denotes spatial averaging over a period cell when the
velocity field is time-independent, $\vec{u}=\vec{u}(\vec{x})$, and
when the velocity field is time-dependent, $\vec{u}=\vec{u}(t,\vec{x})$,
$\langle\cdot\rangle$ denotes time averaging over a temporal period in addition to
spacial averaging. Equation \eqref{eq:Eff_Diffusivity} demonstrates
that the effective transport of the tracer $\phi$ is always inhanced by the
presence of a incompressible velocity field. 




We now recast \eqref{eq:Cell_Problem} 
in a more suggestive form by writing 
$\partial_t\chi=\Delta\Delta^{-1}\partial_t\chi=\vec{\nabla}\cdot(\Delta^{-1}\partial_t\Ib)\vec{\nabla}\chi$ and defining
$\vec{E}_k=\vec{\nabla}\chi+\vec{e}_k$ and 
$\bsig=\bkappa-\Delta^{-1}\partial_t\Ib=((\varepsilon-\Delta^{-1}\partial_t)\Ib+\Hb)$. With these
definitions, equation \eqref{eq:Cell_Problem} is equivalent to 
\cite{Fannjiang:SIAM_JAM:333}  
%
\begin{align}\label{eq:Maxwells_Equations}  
  \vec{\nabla}\times\vec{E}_k=0, \quad
  \vec{\nabla}\cdot\vec{J}_k=0, \quad  
  \vec{J}_k=\bsig\vec{E}_k,\quad
  \langle\vec{E}_k\rangle=\vec{e}_k,
\end{align}
%
where $\Delta^{-1}$ is based on convolution with the Green's function for
the Laplacian $\Delta$ and $\bsig=\bkappa$ in the steady flow
case when $\vec{u}=\vec{u}(\vec{x})$. The formulas in
\eqref{eq:Maxwells_Equations} are  
precisely the electrostatic version of Maxwell's equations for a
conductive medium \cite{Golden:CMP-473}, where $\vec{E}_k$ and
$\vec{J}_k$ are the local electric field and current density,
respectively, and $\bsig$ is the local conductivity tensor of the
medium. In the ACM for treating the effective transport properties of
composites, the effective conductivity tensor $\bsig^*$ is defined by 
%
\begin{align}\label{eq:sigma*}
  \langle\vec{J}_k\rangle=\bsig^*\langle\vec{E}_k\rangle.
\end{align}
%
The linear constituative relation $\vec{J}_k=\bsig\vec{E}_k$ in
\eqref{eq:Maxwells_Equations} relates the local intensity and flux,
while the linear relation in \eqref{eq:sigma*} relates the mean
intensity and mean flux. By the skew-symmetry of the operator
$\mathbf{S}=-\Delta^{-1}\partial_t\Ib+\Hb$, the intensity-flux relationship in
\eqref{eq:Maxwells_Equations} is similar to
that of a Hall medium \cite{Isichenko:JNS:1991:375}. Moreover, the
skew-symmetry of this operator implies that
$\langle\mathbf{S}\vec{E}_k\cdot\vec{E}_k\rangle=-\langle\mathbf{S}\vec{E}_k\cdot\vec{E}_k\rangle=0$. Consequently
by equation \eqref{eq:Maxwells_Equations} and the Helmholtz theorem
\cite{Golden:CMP-473,Denaro:2003:0271,Bhatia:IEE:1077}, equation \eqref{eq:sigma*} reduces to
\eqref{eq:Eff_Diffusivity}.    
%
\begin{align}
  \sigma^*_{kk}=\langle\bsig\vec{E}_k\cdot\vec{e}_k\rangle
       =\langle\bsig\vec{E}_k\cdot\vec{E}_k\rangle
       =\langle[(\varepsilon\Ib+\mathbf{S})]\vec{E}_k\cdot\vec{E}_k\rangle
       =\varepsilon\langle\vec{E}_k\cdot\vec{E}_k\rangle.
\end{align}
%
We will discuss this connection to the ACM in more detail in Section
\ref{sec:Integral_Rep}, where we provide a Stieltjes integral
representation for $\bkappa^*$, which holds for both for steady and
time-dependent flows. 




\subsection{Integral representation of the effective diffusivity for
  steady and dynamic flows}\label{sec:Integral_Rep}
%
In this section, we adapt the ACM for representing transport
incomposites \cite{Golden:CMP-473} to equations
\eqref{eq:Maxwells_Equations} and \eqref{eq:sigma*}. This method gives
an abstract Hilbert space formulation of the effective parameter
problem for $\bkappa$ and provides a Stieltjes integral representation
for the effective diffusivity $\bkappa^*$, which holds for both steady
and dynamic flows.


Let $\mathscr{H}$ be the Hilbert space of square integrable,
mean-zero, periodic, vector valued functions, with temporal 
period $T$ and spatial periodicity on the rectangular region
$\mathcal{R}$ with boundary $\partial\mathcal{R}$  
%
\begin{align}
  \mathscr{H}=\{\vec{Y}(t,\vec{x}):\langle|\vec{Y}|^2\rangle<\infty, \ \langle\vec{Y}\rangle=0
  %\  \vec{Y}(t+T,\vec{x}+\vec{X})=\vec{Y}(t,\vec{x}), \
  %\vec{X}\in\mathcal{R} 
  \},  
\end{align}
%
Also, consider the associated Hilbert spaces of curl-free
$\mathscr{H}_\times$, divergence-free $\mathscr{H}_\bullet$, and constant
$\mathscr{H}_0$ vector valued functions  
%
\begin{align}
  \mathscr{H}_\times=\{\vec{Y}\in\mathscr{H}:\vec{\nabla}\times\vec{Y}=0\}, \quad
  \mathscr{H}_\bullet=\{\vec{Y}\in\mathscr{H}:\vec{\nabla}\cdot\vec{Y}=0\}, \quad
  \mathscr{H}_0=\{\vec{Y}\in\mathscr{H}:\vec{Y} \text{ constant}\}.
\end{align}
%
By the Helmholtz theorem
\cite{Golden:CMP-473,Denaro:2003:0271,Bhatia:IEE:1077}, these 
subspaces provide an orthogonal decomposition of 
$\mathscr{H}$ \cite{Fannjiang:SIAM_JAM:333}, with
associated orthogonal projectors $\Gamma_\times=\vec{\nabla}(\Delta^{-1})\vec{\nabla}\cdot$,
$\Gamma_\bullet=\vec{\nabla}\times(\Delta^{-1})\vec{\nabla}\times$, and $\Gamma_0$,
\cite{Fannjiang:SIAM_JAM:333,Murphy:CMS:Submitted} 
%
\begin{align}
  \mathscr{H}=\mathscr{H}_\times\oplus\mathscr{H}_\bullet\oplus\mathscr{H}_0,\quad
  I=\Gamma_\times+\Gamma_\bullet+\Gamma_0,
\end{align}
%
where $I$ is the identity operator on $\mathbb{R}^d$.
Since we are focusing on periodic flows, it is convenient to consider
the Fourier representations of these opperators
%
\begin{align}
  \vec{Y}(t,\vec{x})=\sum_{\vec{k}\in\mathbb{Z}^d}\e^{\I\vec{k}\cdot\vec{x}}\hat{\mathbf{F}}(\vec{k}), 
\end{align}
%

In two dimensions, $d=2$, the matrix $\Hb$ is determined by a stream
function $H(t,\vec{x})$ 
%
\begin{align}\label{eq:u_H}  
  \Hb=\left[
  \begin{array}{cc}
    0  & H\\
    -H & 0
  \end{array}
  \right],
  \qquad
  \vec{u}=[\partial_{x_1}H, \ \partial_{x_2}H].
\end{align}
%

\newpage

Consider the following problems in electrostatics: find stationary
random vector fields $\vec{E}(\vec{x},\omega)$ and $\vec{J}(\vec{x},\omega)$
such that
%
\begin{align}   \label{eq:Maxwells_Equations_E}  
 &\vec{\nabla}\times\vec{E}=0, \quad
  \vec{\nabla}\cdot\vec{J}=0,\quad
  \vec{J}=\bsig\vec{E},\quad
  %\vec{E}=\vec{E}_0+\vec{E}_f, \quad
  \langle\vec{E}\,\rangle=\vec{E}_0, \\
%
  \label{eq:Maxwells_Equations_J}
   &\vec{\nabla}\times\vec{E}=0, \quad
   \vec{\nabla}\cdot\vec{J}=0, \quad
   \vec{E}=\brho\vec{J},\quad
   %\vec{J}=\vec{J}_0+\vec{J}_f,\quad
   \langle\vec{J}\,\rangle=\vec{J}_0,
   %\notag  
\end{align}
%
where $\langle\cdot\rangle$ denotes ensemble average over $\Omega$ with respect to the
measure $P$,
%e.g. $\langle\vec{E}(\vec{x},\omega)\,\rangle=\int_\Omega P(d\omega)\vec{E}(\vec{x},\omega)$,
we use the
simplified vector notation $\langle\vec{\xi}\,\rangle=(\langle\xi_1\rangle,\ldots,\langle\xi_d\rangle)^T$, and $\xi_i$ is
the $i^{\,\text{th}}$ component of the vector $\vec{\xi}$. Here,
$\vec{E}$ and $\vec{J}$ are the electric field and current density within
the polycrystalline medium, respectively. Their averages $\vec{E}_0$
and $\vec{J}_0$ are assumed to be \emph{given}, and by stationarity they are
independent of $\vec{x}\in\mathbb{R}^d$.  More specifically, we seek
stationary solutions $\vec{E}$ and $\vec{J}$ to each of equations
\eqref{eq:Maxwells_Equations_E} and \eqref{eq:Maxwells_Equations_J} of
the form    
%
\begin{align}\label{eq:Stationary_E_J}
  \vec{E}(\vec{x},\omega)=\vec{E}^{\,\prime}(\tau_{-x}\,\omega), \quad
  \vec{J}(\vec{x},\omega)=\vec{J}^{\,\prime}(\tau_{-x}\,\omega), \quad
  \forall \ \vec{x}\in\mathbb{R}^d, \ \omega\in\Omega,
\end{align}
%
where $\vec{E}^{\,\prime}(\omega)=\vec{E}(0,\omega)$, $\vec{J}^{\,\prime}(\omega)=\vec{J}(0,\omega)$,
$\vec{E}_0=\langle\vec{E}^{\,\prime}(\omega)\rangle$, and
$\vec{J}_0=\langle\vec{J}^{\,\prime}(\omega)\rangle$. Once suitable solutions are 
found, the effective conductivity and resistivity tensors $\bsig^*$
and $\brho^*$ are defined by \cite{Golden:CMP-473}    
%
\begin{align}\label{eq:eff_eps_def}
    \langle \vec{J} \,\rangle=  \bsig^* \vec{E}_0
    \quad \text{and} \quad    
    \langle \vec{E} \,\rangle=  \brho^*\vec{J}_0.
\end{align}
%
We will prove the existence and uniqueness of such solutions to equations
\eqref{eq:Maxwells_Equations_E} and \eqref{eq:Maxwells_Equations_J},
as the setup of the proof illustrates the mathematical properties of
$\bsig^*$ and $\brho^*$ which lead to the desired integral
representations of these effective parameters. From equations
\eqref{eq:Stationary_sig_rho}, \eqref{eq:Stationary_E_J} and
\eqref{eq:eff_eps_def}, we see that the effective parameters are
independent of $\vec{x}\in\mathbb{R}^d$.  





\subsubsection{Existence and uniqueness for the electrostatic problem}
\label{sec:Existence_and_Uniqueness}
%
If the components of the vector fields $\vec{E}$ and $\vec{J}$ were
continuously differentiable on all of $\mathbb{R}^d$, for every $\omega\in\Omega$,
then, since $\vec{\nabla}\times\vec{E}=0$ and $\vec{\nabla}\cdot\vec{J}=0$,
there would exist \cite{Jackson-1999} scalar and
vector potentials $\varphi$ and $\vec{A}$ such that
$\vec{E}=\vec{\nabla}\varphi$ and $\vec{J}=\vec{\nabla}\times\vec{A}$. In this case, equations
\eqref{eq:Maxwells_Equations_E} and \eqref{eq:Maxwells_Equations_J}
are equivalent to
%
\begin{align}\label{eq:Alternate_Maxwells}  
  \vec{\nabla}\cdot\bsig\vec{\nabla}\varphi=0
  \quad \text{ and } \quad
  \vec{\nabla}\times(\brho\,\vec{\nabla}\times\vec{A}\,)=0,
\end{align}
%
respectively, with $\langle\vec{\nabla}\varphi\rangle=\vec{E}_0$ and
$\langle\vec{\nabla}\times\vec{A}\,\rangle=\vec{J}_0$. However, $\sigma_{jk}$ and $\rho_{jk}$ may
be discontinuous, hence non-differentiable, across the crystallite
boundaries of the polycrystalline medium, so the formulas in equation
\eqref{eq:Alternate_Maxwells} may not exist in a classical sense. The
weak forms \cite{Folland:95} of these formulas are given by
%$\langle\bsig\vec{\nabla}\varphi\cdot\vec{\nabla}\psi\rangle_V=0$ and
%$\langle\brho\vec{\nabla}\times\vec{A}\cdot\vec{\nabla}\times\vec{\xi}\rangle_V=0$,
%
\begin{align}\label{eq:Weak_Alternate_Maxwells}
  \langle\bsig\vec{\nabla}\varphi\cdot\vec{\nabla}\psi\rangle_V=0
  \quad \text{ and } \quad
  \langle\brho\vec{\nabla}\times\vec{A}\cdot\vec{\nabla}\times\vec{\xi}\,\rangle_V=0,
\end{align}
%
where $\langle\cdot\rangle_V$ denotes
integration over all of $\mathbb{R}^d$, $\psi$ and $\xi_i$, $i=1,\ldots,d$,
are infinitely differentiable functions with compact support, and we
stress that $\vec{\nabla}\psi$ is a curl-free vector field and
$\vec{\nabla}\times\vec{\xi}$ is a divergence-free vector field. This 
would directly bypass the non-differentiability of $\sigma_{jk}$ and
$\rho_{jk}$. Although, indirectly, the discontinuous nature of these 
functions might still render $\vec{E}$ and $\vec{J}$
non-differentiable, so that the differential equations in
\eqref{eq:Maxwells_Equations_E} and \eqref{eq:Maxwells_Equations_J}
involving these vector fields again, may not exist in a classical
sense.  We address such issues by providing abstract Hilbert space
%variational
formulations \cite{Golden:CMP-473} of the equations 
in \eqref{eq:Weak_Alternate_Maxwells}.  




The group of transformations $\tau_x$ acting on $\Omega$ induces a group of
operators $T_x$ on the Hilbert space $L^2(\Omega,P)$ defined by
$(T_xf)(\omega)=f(\tau_{-x}\,\omega)$ for all $f\in L^2(\Omega,P)$. Since the group
generated by $\tau_x$ (through composition) is measure preserving, the
operators $T_x$ form a unitary group and therefore have closed densely
defined infinitesimal generators $L_i$ in each direction $i=1,\ldots,d$
with domain $\mathscr{D}_i\subset L^2(\Omega,P)$ 
\cite{Golden:CMP-473,Papanicolaou:RF-835}. Thus,  
%
\begin{align}\label{eq:Li}
  L_i=\left.\frac{\partial}{\partial x_i}T_x \right|_{x=0}, \quad i=1,\ldots,d,
\end{align}
%
where differentiation in \eqref{eq:Li} is defined in the sense of 
convergence in $L^2(\Omega,P)$ for elements of $\mathscr{D}_i$
\cite{Golden:CMP-473,Papanicolaou:RF-835}. The closed subset
$\mathscr{D}=\cap_{i=1}^d\mathscr{D}_i$ of $L^2(\Omega,P)$ is a Hilbert space
with inner product $\langle\cdot,\cdot\rangle_D$ given by $\langle f,g\rangle_D=\langle f,g\rangle_{L^2}+\sum_{i=1}^d\langle
L_if,L_ig\rangle_{L^2}$, where $\langle\cdot,\cdot\rangle_{L^2}$ is the $L^2(\Omega,P)$ inner product
\cite{Golden:CMP-473,Papanicolaou:RF-835}.




Now consider the Hilbert space $\mathscr{H}=\bigotimes_{i=1}^dL^2(\Omega,P)$ with
inner product $\langle\cdot,\cdot\rangle$ defined by
$\langle\vec{\xi},\vec{\zeta}\rangle=\langle\vec{\xi}\cdot\vec{\zeta}\rangle$, and define the Hilbert spaces
of ``curl free'' $\mathscr{H}_\times$ and ``divergence free''
$\mathscr{H}_{\bullet}$ random fields \cite{Golden:CMP-473}   
%
\begin{align}\label{eq:curlfreeHilbert}
  &\mathscr{H}_\times=
  \left\{\vec{\xi}^{\,\prime}(\omega)\in \mathscr{H} \ | \ L_i\xi_j^{\,\prime}-L_j\xi_i^{\,\prime}=0 \ \text{ weakly and }
    \langle\vec{\xi}^{\,\prime}\,\rangle=0
  \right\}, \\
&\mathscr{H}_{\bullet}=
\left\{\vec{\zeta}^{\,\prime}(\omega)\in \mathscr{H} \ \Big| \ \sum_{i=1}^dL_i\zeta_i^{\,\prime}=0 \ \text{ weakly and }
    \langle\vec{\zeta}^{\,\prime}\,\rangle=0\right\}.\notag 
\end{align}  
%
We write $\vec{E}(\vec{x},\omega)=\vec{E}_0+\vec{E}_f(\vec{x},\omega)$, where
$\vec{E}_f$ is the fluctuating field of mean zero about the (constant)
average $\vec{E}_0$, and similarly for
$\vec{J}(\vec{x},\omega)=\vec{J}_0+\vec{J}_f(\vec{x},\omega)$. 
In view of equations \eqref{eq:Stationary_sig_rho},
\eqref{eq:Stationary_E_J}, and \eqref{eq:Weak_Alternate_Maxwells}, we
consider the following variational problems
\cite{Golden:CMP-473}: given $\bsig^{\,\prime}$ and $\vec{E}_0$, and
$\brho^{\,\prime}$ and $\vec{J}_0$, find $\vec{E}_f^{\,\prime}(\omega)\in\mathscr{H}_\times$
and $\vec{J}_f^{\,\prime}(\omega)\in\mathscr{H}_\bullet$ such that         
%
\begin{align} 
  \label{eq:Weak_Curl_Free_Variational_Form_E}
  %&\Phi(\bsig^{\,\prime}\vec{E}^{\,\prime},\vec{\xi}^{\,\prime}):=
     \langle\bsig^{\,\prime}(\omega)(\vec{E}_0+\vec{E}_f^{\,\prime}(\omega))\cdot\vec{\xi}^{\;\prime}(\omega)\rangle=0
      \quad  \forall \ \vec{\xi}^{\,\prime}(\omega)\in\mathscr{H}_\times
  \\
  \label{eq:Weak_Curl_Free_Variational_Form_J}
  %&\Psi(\brho^{\,\prime}\vec{J}^{\,\prime},\vec{\zeta}^{\,\prime}):=
     \langle\brho^{\,\prime}(\omega)(\vec{J}_0+\vec{J}_f^{\,\prime}(\omega))\cdot\vec{\zeta}^{\,\prime}(\omega)\rangle=0
      \quad  \forall \ \vec{\zeta}^{\,\prime}(\omega)\in\mathscr{H}_{\bullet}\,,
  %\notag
\end{align}
%
respectively. In order to apply the Lax-Milgram Theorem, we
rewrite equations \eqref{eq:Weak_Curl_Free_Variational_Form_E} and
\eqref{eq:Weak_Curl_Free_Variational_Form_J} as
%
\begin{align}
  \label{eq:Bilinear_functional_E}
   &\Phi(\vec{E}_f^{\,\prime},\vec{\xi}^{\,\prime})
     =\langle\bsig^{\,\prime}(\omega)\vec{E}_f^{\,\prime}(\omega)\cdot\vec{\xi}^{\,\prime}(\omega)\rangle
     =-\langle\bsig^{\,\prime}(\omega)\vec{E}_0\cdot\vec{\xi}^{\,\prime}(\omega)\rangle
     =f_\sigma(\vec{\xi}^{\,\prime}),
  \\
  \label{eq:Bilinear_functional_J}
  &\Psi(\vec{J}_f^{\,\prime},\vec{\zeta}^{\,\prime})
     =\langle\brho^{\,\prime}(\omega)\vec{J}_f^{\,\prime}(\omega)\cdot\vec{\zeta}^{\,\prime}(\omega)\rangle
     =-\langle\brho^{\,\prime}(\omega)\vec{J}_0\cdot\vec{\zeta}^{\,\prime}(\omega)\rangle
     =f_\rho(\vec{\zeta}^{\,\prime}),
\end{align}
%
respectively.




By the Cauchy--Schwartz inequality and equation
\eqref{eq:Bounded_parameters}, each of the bilinear functionals $\Phi$
and $\Psi$ are bounded, and $f_\sigma$ and $f_\rho$ are both bounded linear
functionals on the Hilbert spaces $\mathscr{H}_\times$ and 
$\mathscr{H}_\bullet$, respectively. For now, we will assume that $\Phi$ and
$\Psi$ are coercive, e.g. that there exists a positive constant 
$\kappa>0$ such that $\Phi(\vec{\xi},\vec{\xi}\,)\geq\kappa\|\vec{\xi}\,\|^2$ for all 
$\vec{\xi}\in\mathscr{H}_\times$, where $\|\cdot\|$ denotes the norm induced by the
$\mathscr{H}$-inner-product. Later, we will demonstrate that the
coercivity condition determines analytic properties of the
effective parameters. By the Lax-Milgram theorem, there exist unique
$\vec{E}_f^{\,\prime}\in\mathscr{H}_\times$ and $\vec{J}_f^{\,\prime}\in\mathscr{H}_\bullet$
that satisfy equations \eqref{eq:Bilinear_functional_E} and
\eqref{eq:Bilinear_functional_J} for all $\vec{\xi}^{\,\prime}\in\mathscr{H}_\times$
and $\vec{\zeta}^{\,\prime}\in\mathscr{H}_\bullet$, respectively. By construction,
%
\begin{align}\label{eq:Unique_Solutions}
  \vec{E}^{\,\prime}(\omega)&=\vec{E}_0+\vec{E}_f^{\,\prime}(\omega), \qquad
  \vec{J}^{\,\prime}(\omega)=\bsig^{\,\prime}(\omega)\vec{E}^{\,\prime}(\omega),
  \\
  \vec{J}^{\,\prime}(\omega)&=\vec{J}_0+\vec{J}_f^{\,\prime}(\omega), \qquad
  \vec{E}^{\,\prime}(\omega)=\brho^{\,\prime}(\omega)\vec{J}^{\,\prime}(\omega),
\end{align}
%
are the unique solutions of equations \eqref{eq:Maxwells_Equations_E}
and \eqref{eq:Maxwells_Equations_J}, respectively, via equations
\eqref{eq:Stationary_sig_rho} and \eqref{eq:Stationary_E_J}. 
To simplify notation, we will henceforth drop
the distinction between the primed variables $\vec{E}_f^{\,\prime}(\omega)$ and 
$\vec{E}_f(\vec{x},\omega)$, for example, as the context of each notation
is now clear.





We conclude this section by noting that since
$\vec{E}_f\in\mathscr{H}_\times$ and $\vec{J}_f\in\mathscr{H}_\bullet$, equations
\eqref{eq:Weak_Curl_Free_Variational_Form_E} and
\eqref{eq:Weak_Curl_Free_Variational_Form_J} yield the energy (power)
\cite{Jackson-1999} constraints $\langle\vec{J}\cdot\vec{E}_f\rangle=0$ and
$\langle\vec{E}\cdot\vec{J}_f\rangle=0$, respectively, which lead to the following
reduced energy representations $\langle\vec{J}\cdot\vec{E}\,\rangle=\langle\vec{J\,}\rangle\cdot\vec{E}_0$
and $\langle\vec{E}\cdot\vec{J}\,\rangle=\langle\vec{E}\,\rangle\cdot\vec{J}_0$. By equation
\eqref{eq:eff_eps_def}, we have the following energy representations
involving the effective parameters  
%$\langle\vec{J}\cdot\vec{E}\rangle=\bsig^*\vec{E}_0\cdot\vec{E}_0=\brho^*\vec{J}_0\cdot\vec{J}_0$.
%
\begin{align}\label{eq:Energy_Reps}
  \langle\vec{J}\cdot\vec{E}\rangle=\bsig^*\vec{E}_0\cdot\vec{E}_0=\brho^*\vec{J}_0\cdot\vec{J}_0.
\end{align}
%



\subsubsection{Representation formulas for uniaxial polycrystalline
  media}\label{sec:Representation_formulas}
%
In Section \ref{sec:Existence_and_Uniqueness} we proved that equations
\eqref{eq:Maxwells_Equations_E} and \eqref{eq:Maxwells_Equations_J}
have unique solutions when the tensors $\bsig(\vec{x},\omega)$ and
$\brho(\vec{x},\omega)$ satisfy the boundedness conditions in
\eqref{eq:Bounded_parameters}, and when the bilinear functionals $\Phi$
and $\Psi$ of equations \eqref{eq:Bilinear_functional_E} and
\eqref{eq:Bilinear_functional_J} are coercive. In this section, we
derive Stieltjes integral representations for the effective
conductivity and resistivity tensors $\bsig^*$ and $\brho^*$, 
involving spectral measures of self-adjoint random operators, which
depend only on the geometry of the polycrystallin medium. We will also
show that the coercivity conditions on $\Phi$ and $\Psi$ determine analytic
properties these representations. 

 



In this section, we restrict our attention to polycrystalline media
with uniaxial microscopic asymmetry. We also interpret the
electrostatic version of Maxwell's equations in
\eqref{eq:Maxwells_Equations_E} and \eqref{eq:Maxwells_Equations_J} as
the quasi-static limit of the full set of Maxwell's equations. In this
case, these equations describe the transport properties of an
electromagnetic wave through a polycrystalline medium, when the
wavelength is very large compared to the typical size of the crystallites
within the random medium. Under these assumptions, the local
conductivity along one of the crystallite axes has the \emph{complex}
value $\sigma_1$, while the conductivity along all the other axes have the
complex value $\sigma_2$. The conductivity and resistivity tensors of such
polycrystalline media are given by   
%
\begin{align}\label{eq:polycrystal_parameters}
  &\bsig(\vec{x},\omega)=R^{\,T}(\vec{x},\omega)\text{diag}(\sigma_1,\sigma_2,\ldots,\sigma_2)R(\vec{x},\omega),
  \\
  &\brho(\vec{x},\omega)=R^{\,T}(\vec{x},\omega)\text{diag}(1/\sigma_1,1/\sigma_2,\ldots,1/\sigma_2)R(\vec{x},\omega),
  \notag
\end{align}
%
where $R(\vec{x},\omega)$ is a rotation matrix satisfying
$R^{\,T}=R^{\,-1}$. For example, when $d=2$ we have 
%
\begin{align}\label{eq:polycrystal_parameters_2D}
  \bsig=R^{\,T}
  \left[
  \begin{array}{cc}
    \sigma_1& 0\\
    0 & \sigma_2\\
    \end{array}
    \right]
    R,
    \quad
    \brho=R^{\,T}
  \left[
  \begin{array}{cc}
    1/\sigma_1& 0\\
    0 & 1/\sigma_2\\
    \end{array}
    \right]
    R,
    \qquad
    R=
  \left[
  \begin{array}{rr}
    \cos\theta& -\sin\theta\\
    \sin\theta & \cos\theta\\
    \end{array}
    \right],
\end{align}
%
where $\theta=\theta(\vec{x},\omega)$ is the orientation angle, measured from the
direction $\vec{e}_1$, of the polycrystallite which has an interior 
containing $\vec{x}\in\mathbb{R}^d$ for $\omega\in\Omega$. Here, $\vec{e}_j$,
$j=1,\ldots,d$, are standard basis vectors with components
$(\vec{e}_j)_k=\delta_{jk}$ and $\delta_{jk}$ is the Kronecker delta. In higher
dimensions, $d\geq3$, the rotation matrix $R$ is a composition of
``basic'' rotation matrices 
$R_i$, e.g. $R=\prod_{j=1}^dR_j$, where the matrix $R_j(\vec{x},\omega)$
rotates vectors in $\mathbb{R}^d$ by an angle
$\theta_j=\theta_j(\vec{x},\omega)$ about the $\vec{e}_j$ axis. For example, in three
dimensions 
%
\begin{align}\label{eq:polycrystal_parameters_3D}
  R_1=
  \left[
  \begin{array}{ccc}
    1     &0     &    0\\
    0     &\cos\theta_1 & -\sin\theta_1\\
    0     &\sin\theta_1 & \cos\theta_1\\
    \end{array}
    \right],
    \quad
   R_2=
  \left[
  \begin{array}{ccc}
    \cos\theta_2  & 0     &\sin\theta_2 \\
    0      & 1     & 0\\
    -\sin\theta_2 & 0     &\cos\theta_2\\
    \end{array}
    \right],
    \quad
    R_3=
  \left[
  \begin{array}{ccc}
    \cos\theta_3 & -\sin\theta_3 & 0\\
    \sin\theta_3 & \cos\theta_3  & 0\\
    0     & 0      & 1\\
    \end{array}
    \right].
\end{align}
%







Equation \eqref{eq:polycrystal_parameters} can be written in a more
suggestive form in terms of the matrix $C=\text{diag}(1,0,\ldots,0)$  
%
\begin{align}\label{eq:two-phase_eps}
  \bsig(\vec{x},\omega)=\sigma_1X_1(\vec{x},\omega)+\sigma_2X_2(\vec{x},\omega), \qquad
  \brho(\vec{x},\omega)=X_1(\vec{x},\omega)/\sigma_1+X_2(\vec{x},\omega)/\sigma_2.
\end{align}
%
Here $X_1=R^{\,T}CR$ and $X_2=R^{\,T}(I-C)R$, where $I$ is the
identity matrix on $\mathbb{R}^d$. Since $R^{\,T}=R^{\,-1}$ and $C$
is a diagonal projection matrix satisfying $C^{\,2}=C$, it is
clear that the $X_i$, $i=1,2$, are mutually orthogonal projection
matrices satisfying
%$X_jX_k=X_j\delta_{jk}$ and $X_1+X_2=I$.
%
\begin{align}\label{eq:Projection_Matrices}
  X_j^{\,T}=X_j, \quad X_jX_k=X_j\delta_{jk}, \quad X_1+X_2=I.
\end{align}
%
By equation
\eqref{eq:Stationary_sig_rho}, we have the existence of measurable
functions $[X_i^{\,\prime}(\omega)]_{jk}$, $j,k=1,\ldots,d$, such that
$[X_i(\vec{x},\omega)]_{jk}=[X_i^{\,\prime}(\tau_{-x}\,\omega)]_{jk}$ for all
$\vec{x}\in\mathbb{R}^d$ and $\omega\in\Omega$. From equations
\eqref{eq:eff_eps_def} and \eqref{eq:two-phase_eps}, we see that
$\sigma_{jk}^*(a\sigma_1,a\sigma_2)=a\sigma_{jk}^*(\sigma_1,\sigma_2)$, for any complex
number $a$, and similarly for the $\rho_{jk}^*$. Due to this homogeneity
of these functions, they depend only on the ratio $h=\sigma_1/\sigma_2$, and we
define the tensor-valued functions $\mathbf{m}(h)=\bsig^*/\sigma_2$,
$\mathbf{w}(z)=\bsig^*/\sigma_1$, $\tilde{\mathbf{m}}(h)=\sigma_1\brho^*$, and
$\tilde{\mathbf{w}}(z)=\sigma_2\brho^*$ with components  
%
\begin{align}\label{eq:m_h}
  m_{jk}(h)=\sigma_{jk}^*/\sigma_2, \quad
  w_{jk}(z)=\sigma_{jk}^*/\sigma_1, \quad
   \tilde{m}_{jk}(h)=\sigma_1\rho_{jk}^*, \quad
   \tilde{w}_{jk}(z)=\sigma_2\rho_{jk}^*.
\end{align}
%
where $z=1/h$. For example, from equations \eqref{eq:eff_eps_def} and
\eqref{eq:two-phase_eps} we have
%
\begin{align}\label{eq:eps*_rho*_h}
  \bsig^*(h)\vec{E}_0=\sigma_2\left\langle(hX_1+X_2)\vec{E}\right\rangle,
  \quad
  \brho^*(h)\vec{J}_0=(1/\sigma_1)\left\langle(X_1+hX_2)\vec{J}\,\right\rangle
\end{align}
%




We now show that the coercivity conditions on the bilinear functionals
$\Phi$ and $\Psi$ of equations \eqref{eq:Bilinear_functional_E} and
\eqref{eq:Bilinear_functional_J} imply that
%$\epsilon^*_{jk}(h)$ and $\rho^*_{jk}(h)$ are analytic functions of the complex
%variable $h$ everywhere except on the negative real axis.
%
\begin{align}\label{eq:Analytic_mh}
  &\sigma^*_{jk}(h) \text{ and } \rho^*_{jk}(h) \text{ are analytic functions
    of the complex variable $h$}
    \\
    &\text{ everywhere except on the negative real
    axis.} \notag
\end{align}
% 
We only prove the statement in \eqref{eq:Analytic_mh} for $\sigma^*_{jk}(h)$, as
the proof involving $\rho^*_{jk}(h)$ is analogous. The bounded bilinear
functional $\Phi$ is coercive if there exists a $\kappa>0$ such that
$|\Phi(\vec{\xi},\vec{\xi})|\geq\kappa\|\vec{\xi}\,\|^2$ for all
$\vec{\xi}\in\mathscr{H}_\times$ such that $\|\vec{\xi}\,\|\neq0$. From equations
\eqref{eq:Bilinear_functional_E} and \eqref{eq:two-phase_eps} this is
true only if  
%
\begin{align}\label{eq:Coercive_Phi_h}
  \left|
    \left\langle(hX_1+X_2)\vec{\xi}\cdot\overline{\vec{\xi}}\;\right\rangle
  \right|\geq (\kappa/|\sigma_2|)\|\vec{\xi}\,\|^2,
\end{align}
%
where $\vec{\xi}$ is complex-valued when $h$ is complex and
$\overline{\vec{\xi}}$ is the complex conjugate of the 
vector $\vec{\xi}$. Define $\nu$ by the ratio 
%
\begin{align} 
  \nu=\frac{\left\langle X_1\vec{\xi}\cdot\overline{\vec{\xi}}\;\right\rangle}          
         {\|\vec{\xi}\,\|^2}\,.
\end{align}
%
Since $X_1$ is a projection matrix, it is bounded by 1 in
operator norm. Moreover, we have that $\nu\geq0$ as
$\langle X_1\vec{\xi}\cdot\overline{\vec{\xi}}\;\rangle
=\langle X_1\vec{\xi}\cdot X_1\overline{\vec{\xi}}\;\rangle=\|X_1\vec{\xi}\,\|^2\geq0$.
Therefore, the Cauchy--Schwartz inequality and the definition of the
operator norm \cite{Folland:99} implies that $0\leq\nu\leq1$. Since
$X_2=I-X_1$, equation \eqref{eq:Coercive_Phi_h} can now be
written as
% 
\begin{align}\label{eq:Coercive_Phi_h_beta}
  |h\nu+1-\nu|\geq\kappa/|\sigma_2|>0.
\end{align}
%



Let $h=h_r+\I h_i$, where $h_r$ and $h_i$ are the real and imaginary
parts of the complex number $h$, respectively. Since 
$|h\nu+1-\nu|^2=|1+\nu(h_r-1)|^2+\nu^2|h_i|^2$, the formula in
\eqref{eq:Coercive_Phi_h_beta} always holds when $h_i\neq0$. If $h$ is
real, then $1+\nu(h-1)=0$ if and only if $h=1-1/\nu$  
for $0<\nu\leq1$, i.e. for $h$ on the negative real axis. In conclusion,
equation \eqref{eq:Coercive_Phi_h_beta} holds if and only if $h$ is
\emph{off} of the negative real axis including zero
\cite{Golden:CMP-473}. For any such complex value of $h$, $\Phi$ is
coercive and there exists a unique solution $\vec{E}_f$ to equation
\eqref{eq:Bilinear_functional_E}, hence to equation
\eqref{eq:Maxwells_Equations_E}. By differentiation with respect to
$h$, one deduces easily (WHY?) that $\vec{E}_f$ is analytic
in $h$ off the negative real axis with values in
$\mathscr{H}_\times$. Therefore by equation \eqref{eq:eps*_rho*_h},
$\sigma^*_{jk}(h)$ is analytic off the negative real axis in the complex
$h$-plane. This concludes our proof of the statement in equation
\eqref{eq:Analytic_mh}. 




The above analysis \cite{Golden:CMP-473} demonstrates that the
dimensionless functions $m_{jk}(h)$ and $\tilde{m}_{jk}(h)$ in
\eqref{eq:m_h} are analytic off the negative real axis in the
$h$-plane, while $w_{jk}(z)$ and $\tilde{w}_{jk}(z)$ are analytic off
the negative real axis in the $z$-plane. By equation
\eqref{eq:eps*_rho*_h} and its counterpart involving $z=1/h$, each
take the corresponding upper half plane to the upper half plane and
are therefore examples of Herglotz functions
\cite{Deift:2000:RMT,Golden:CMP-473}. 



A key step in the ACM is
obtaining Stieltjes integral representations for $\bsig^*$ and
$\brho^*$. These follow from resolvent representations for the
electric field $\vec{E}$ \cite{Golden:CMP-473} and current density
$\vec{J}$ \cite{Murphy:JMP:063506}         
%
\begin{align}\label{eq:Resolvent_representations_E_D}
  &\vec{E}=s(sI-\Gamma X_1)^{-1}\vec{E}_0=t(tI-\Gamma X_2)^{-1}\vec{E}_0 ,
  \quad
   s\in\mathbb{C}\backslash[0,1],\\
  &\vec{J}=t(tI-\Upsilon X_1)^{-1}\vec{J}_0=s(sI-\Upsilon X_2)^{-1}\vec{J}_0 ,
  \quad
   t\in\mathbb{C}\backslash[0,1],\notag 
\end{align}
%
where we have defined the complex variables $s=1/(1-h)$ and
$t=1/(1-z)=1-s$.  
The (non-random) operator $\Gamma=\vec{\nabla}(\Delta^{-1})\vec{\nabla}\cdot$ is based on
convolution with the free-space Green's function for the Laplacian
$\Delta=\vec{\nabla}\cdot\vec{\nabla}=\nabla^{\,2}$, and the (non-random) operator
$\Upsilon=-\vec{\nabla}\times(\vec{\nabla}\times\vec{\nabla}\times)^{-1}\vec{\nabla}\times$ involves the vector
Laplacian $\bDelta=- \vec{\nabla}\times\vec{\nabla}\times + \vec{\nabla}\vec{\nabla}\cdot $ for $d=2,3$
\cite{Golden:CMP-473,Murphy:JMP:063506}. These (projection) operators
are discussed in more detail below.      



If the current density $\vec{J}(\vec{x},\omega)$ and the electric
field $\vec{E}(\vec{x},\omega)$ are sufficiently smooth for all
$\vec{x}\in\mathbb{R}^d$ when $\omega\in\Omega$, equation 
\eqref{eq:Resolvent_representations_E_D} is obtained as
follows. The operator $\Delta^{-1}$ is
well defined in terms of convolution with respect to the free-space
Green's function of the Laplacian $\Delta$
\cite{Golden:CMP-473,Folland:95}. Similarly, on the Hilbert space 
$\mathscr{H}=\bigotimes_{i=1}^dL^2(\Omega,P)$, the inverse $\bDelta^{-1}$ of the
vector Laplacian $\bDelta$ is defined in terms of component-wise
convolution with respect to the free-space Green's function of the
Laplacian.

 

Applying the integro-differential operator $\vec{\nabla}(\Delta^{-1})$ to the
formula $\vec{\nabla}\cdot\vec{J}=0$ in equation
\eqref{eq:Maxwells_Equations_E} yields $\Gamma\vec{J}=0$, where
$\Gamma=\vec{\nabla}(\Delta^{-1})\vec{\nabla}\cdot$ is an orthogonal projection
\cite{Golden:CMP-473} from $\mathscr{H}$ onto the Hilbert space
$\mathscr{H}_\times$ of curl-free random fields,
$\Gamma:\mathscr{H}\mapsto\mathscr{H}_\times$. More specifically, for every
sufficiently smooth $\vec{\xi}\in\mathscr{H}_\times$ there exists
\cite{Jackson-1999} a scalar potential $\varphi$ which is unique up to a 
constant such that $\vec{\xi}=\vec{\nabla}\varphi$. Consequently, it is clear
that $\Gamma\vec{\xi}=\vec{\xi}$ for all such $\vec{\xi}\in\mathscr{H}_\times$. 



In order to discuss analogous properties of divergence free vector
fields, we restrict our attention to $d=2,3$, and avoid a more involved
discussion regarding differential forms. Applying the operator 
$\vec{\nabla}\times(\vec{\nabla}\times\vec{\nabla}\times)^{\,-1}$ to the formula
$\vec{\nabla}\times\vec{E}=0$, we have $\Upsilon\vec{E}=0$, where  
$\Upsilon=-\vec{\nabla}\times(\bDelta^{-1})\vec{\nabla}\times$ is an orthogonal projection from
$\mathscr{H}$ onto the Hilbert space $\mathscr{H}_{\bullet}$ of
divergence-free random fields (of transverse gauge)
\cite{Murphy:JMP:063506}. This can be seen as follows. For every
sufficiently smooth $\vec{\zeta}\in\mathscr{H}_\bullet$ we have
$\vec{\zeta}=\vec{\nabla}\times(\vec{A}+\vec{C})$, where $\vec{A}$ is the vector
potential associated with $\vec{\zeta}$ and the arbitrary vector field
$\vec{C}$ satisfies $\vec{\nabla}\times\vec{C}=0$ \cite{Jackson-1999}. Without
loss of generality, $\vec{C}$ can be chosen so that $\vec{A}$
satisfies $\vec{\nabla}\cdot\vec{A}=0$ \cite{Jackson-1999}. Hence,
$\vec{\nabla}\times\vec{\zeta}=\vec{\nabla}\times\vec{\nabla}\times\vec{A}
=\vec{\nabla}(\vec{\nabla}\cdot\vec{A})-\bDelta\vec{A}=-\bDelta\vec{A}$. The vector
$\vec{C}$ chosen in this manner gives the transverse \emph{gauge} of
$\vec{\zeta}$ \cite{Jackson-1999}. Choosing the members of
$\mathscr{H}_\bullet$ to have transverse gauge, the action of
$\vec{\nabla}\times\vec{\nabla}\times$ on $\mathscr{H}_\bullet$ is given by that of 
$-\bDelta$. Therefore, the action of $\Upsilon$ on $\mathscr{H}_\bullet$ is given
by that of 
%$\Upsilon=\vec{\nabla}\times(\vec{\nabla}\times\vec{\nabla}\times)^{-1}\vec{\nabla}\times=\vec{\nabla}\times(\bDelta^{-1})\vec{\nabla}\times$, 
%
\begin{align}\label{eq:GammaCurl}
  \Upsilon=\vec{\nabla}\times(\vec{\nabla}\times\vec{\nabla}\times)^{-1}\vec{\nabla}\times
  =-\vec{\nabla}\times(\bDelta^{-1})\vec{\nabla}\times, 
\end{align}
%
and it is clear from the above discussion that $\Upsilon\vec{\zeta}=\vec{\zeta}$ for
all such $\vec{\zeta}\in\mathscr{H}_\bullet$.







We now derive the formulas in equation
\eqref{eq:Resolvent_representations_E_D}.
Using $h=1-1/s$, $z=1-1/t$ and $X_1+X_2=I$, we write $\sigma$ and $\rho$ in  
equation \eqref{eq:two-phase_eps} as $\bsig=\sigma_2(I-X_1/s)=\sigma_1(I-X_2/t)$ and
$\brho=(I-X_2/s)/\sigma_1=(I-X_1/t)/\sigma_2$. Recall that
$\vec{E}=\vec{E}_0+\vec{E}_f$, where $\vec{E}_0$ is a \emph{constant}
field and $\vec{E}_f\in\mathscr{H}_\times$ so that
$\Gamma\vec{E}=\vec{E}_f$, and similarly
$\Upsilon\vec{J}=\vec{J}_f$. Consequently, from $\Gamma\vec{J}=0$ and $\Upsilon\vec{E}=0$
we have the following formulas which are equivalent to that in 
\eqref{eq:Resolvent_representations_E_D}  
% 
\begin{align}\label{eq:Proj_rep_Ef_Jf}
  \vec{E}_f=\frac{1}{s}\Gamma X_1\vec{E}=\frac{1}{t}\Gamma X_2\vec{E}, \qquad
  \vec{J}_f=\frac{1}{t}\Upsilon X_1\vec{J}=\frac{1}{s}\Upsilon X_2\vec{J}.
\end{align}
%
In general, the differential opperators $\vec{\nabla}$, $\vec{\nabla}\cdot$, and
$\vec{\nabla}\times$ are interpreted
in a weak sense in terms of the operators $L_i$ in \eqref{eq:Li}.




We now derive integral representations for the effective parameters
of uniaxial polycrystalline media, involving the resolvent formulas in
\eqref{eq:Resolvent_representations_E_D}. 
For the formulation of the effective parameter
problem involving $\mathscr{H}_\times$ and $\bsig^*$, define the
coordinate system so that in \eqref{eq:eff_eps_def} the constant
vector $\vec{E}_0=\langle\vec{E}\rangle$ is given by
$\vec{E}_0=E_0\,\vec{e}_j$. In the other formulation involving
$\mathscr{H}_\bullet$ and $\brho^*$, define
%the constant vector $\vec{J}_0=\langle\vec{J}\rangle$ by
$\vec{J}_0=J_0\,\vec{e}_j$. Recalling that $X_1+X_2=I$ yields
$\bsig=\sigma_2(I-X_1/s)$ and $\brho=(I-X_2/s)/\sigma_1$ in
\eqref{eq:two-phase_eps}, equations \eqref{eq:eps*_rho*_h} and
\eqref{eq:Resolvent_representations_E_D} 
imply that $\sigma^*_{jk}=\bsig^*\vec{e}_j\cdot\vec{e}_k$ and 
$\rho^*_{jk}=\brho^*\vec{e}_j\cdot\vec{e}_k$  satisfy
%
\begin{align}\label{eq:Eff_Cond_Tens_Def}
  \sigma^*_{jk}=\sigma_2(\delta_{jk}-\langle X_1(sI-\Gamma X_1)^{-1}\vec{e}_j\cdot\vec{e}_k\rangle) ,
                \quad
  \rho^*_{jk}=(1/\sigma_1)(\delta_{jk}-\langle X_2(sI-\Upsilon X_2)^{-1}\vec{e}_j\cdot\vec{e}_k\rangle), 
\end{align}
%
and similar formulas involving the contrast parameter $t$.
Therefore, it is more convenient to consider the functions 
$F_{jk}(s)=\delta_{jk}-m_{jk}(h)$ and $E_{jk}(s)=\delta_{jk}-\tilde{m}_{jk}(h)$
which are analytic off $[0,1]$ in the $s$-plane, and
$G_{jk}(t)=\delta_{jk}-w_{jk}(z)$ and $H_{jk}(t)=\delta_{jk}-\tilde{w}_{jk}(z)$
which are analytic off $[0,1]$ in the $t$-plane
\cite{Golden:CMP-473}.






On the Hilbert space $\mathscr{H}_\times$, the operators $\Gamma$ and $X_i$,
$i=1,2$, act as projectors. Therefore, $M_i=X_i\Gamma X_i$, $i=1,2$, are
compositions of projection operators on 
$\mathscr{H}_\times$, and are consequently positive definite and bounded by
1 in the underlying operator norm \cite{Rudin:87}. They are
self-adjoint with respect to the $\mathscr{H}$-inner-product
$\langle\cdot,\cdot\rangle$. Therefore, on the Hilbert space $\mathscr{H}_\times$ 
with weight $X_1$ in the inner-product, $\langle\cdot,\cdot\rangle_1=\langle X_1\,\cdot,\cdot\rangle$ for
example, $\Gamma X_1$ is a bounded linear self-adjoint operator with
spectrum contained in the interval $[0,1]$
\cite{Golden:CMP-473,Folland:95,Rudin:87}. Hence the resolvent
operator $(sI-\Gamma X_1)^{-1}$ in \eqref{eq:Resolvent_representations_E_D}
is also a linear self-adjoint operator with respect to the same
inner-product, and is bounded for  $s\in\mathbb{C}\backslash[0,1]$
\cite{Stone:64}. Similarly, $(tI-\Upsilon X_1)^{-1}$ in
\eqref{eq:Resolvent_representations_E_D} is a linear self-adjoint
operator on $\mathscr{H}_\bullet$ with respect to the inner-product
$\langle\cdot,\cdot\rangle_1$, and is bounded for $t\in\mathbb{C}\backslash[0,1]$.





By the spectral theorem for such
operators \cite{Reed-1980,Stone:64}, there exists an increasing family
of self-adjoint projection operators $\{Q(\lambda)\}$ - the resolution of the
identity - that satisfy $Q(0)=0$ and $Q(1)=I$ such that 
% 
\begin{align}\label{eq:Spectral_Theorem}
  f(M_1)=\int_0^1 f(\lambda)Q(d\lambda), \quad
  \langle f(M_1)\,\vec{e}_j\cdot\vec{e}_k\rangle_1= \int_0^1f(\lambda)\mu_{jk}(d\lambda), 
  %\mu_{jk}(\lambda)=\langle Q(\lambda)\vec{e}_j,\vec{e}_k\rangle_1
\end{align}
%
for example, for all bounded continuous functions
$f:\mathbb{C}\mapsto\mathbb{C}$. Here 
$0$ and $I$ are the null and identity operators on $\mathbb{R}^d$,
respectively, $Q(d\lambda)$ is the projection valued measure associated with 
the operator $Q(\lambda)$ \cite{Reed-1980}, and
$\mu_{jk}(d\lambda)=\langle Q(d\lambda)\,\vec{e}_j\cdot\vec{e}_k\rangle_1$, $j,k=1,\ldots,d$, are the
components of the matrix valued \emph{spectral measure} $\bmu(d\lambda)$ in
the $(\vec{e}_j,\vec{e}_k)$ state
\cite{Golden:CMP-473,Reed-1980,Stone:64}. As the spectrum of the
operator $M_1$ is contained in the interval $[0,1]$, the support
$\Sigma_{jk}$ of the measure $\mu_{jk}$ satisfies $\Sigma_{jk}\subseteq[0,1]$
\cite{Reed-1980}. Analogous results hold for the operators
$M_2$ and $K_i=X_i\Upsilon X_i$, $i=1,2$. In view of the formulas in equation
\eqref{eq:Eff_Cond_Tens_Def} and their counterparts involving the
contrast parameter $t$, setting $f(\lambda)=(s-\lambda)^{-1}$ in
\eqref{eq:Spectral_Theorem}, for example, yields the following
integral representations  
%\cite{Golden:CMP-473,Bergman:PRC-377,Bergman:AP-78,Murphy:JMP:063506} 
for the effective parameters $\sigma^*_{jk}$ and $\rho^*_{jk}$ 
%
\begin{align}\label{eq:Stieltjes_F}
  &m_{jk}(h)=\delta_{jk}-F_{jk}(s), \qquad
  F_{jk}(s)=\langle X_1(sI-\Gamma X_1)^{-1}\vec{e}_j\cdot\vec{e}_k\rangle=\int_0^1\frac{\mu_{jk}(d\lambda)}{s-\lambda}\,,
  \\
  &w_{jk}(z)=\delta_{jk}-G_{jk}(t), \qquad
  G_{jk}(t)=\langle X_2(tI-\Gamma X_2)^{-1}\vec{e}_j\cdot\vec{e}_k\rangle=\int_0^1\frac{\alpha_{jk}(d\lambda)}{t-\lambda}\,,
  \notag \\
  &\tilde{m}_{jk}(h)=\delta_{jk}-E_{jk}(s), \qquad
  E_{jk}(s)=\langle X_2(sI-\Upsilon X_2)^{-1}\vec{e}_j\cdot\vec{e}_k\rangle=\int_0^1\frac{\eta_{jk}(d\lambda)}{s-\lambda}\,,
  \notag \\
  &\tilde{w}_{jk}(z)=\delta_{jk}-H_{jk}(t), \qquad
  H_{jk}(t)=\langle X_1(tI-\Upsilon X_1)^{-1}\vec{e}_j\cdot\vec{e}_k\rangle=\int_0^1\frac{\kappa_{jk}(d\lambda)}{t-\lambda}\,.
  \notag
\end{align}
%
Here, $\mu_{jk}$ and $\alpha_{jk}$ are \emph{spectral measures} associated
with the random operators $X_1\Gamma X_1$ and $X_2\Gamma X_2$, respectively, while
$\eta_{jk}$ and $\kappa_{jk}$ are spectral measures associated
with the random operators $X_2\Upsilon X_2$ and $X_1\Upsilon X_1$, respectively. 


By the Stieltjes--Perron inversion theorem 
\cite{Henrici:1974:v2,MILTON:2002:TC}, the spectral measure $\bmu$, 
for example, is given by the weak limit 
$\bmu(d\lambda)=-\lim_{\epsilon\downarrow0}\text{Im}(\mathbf{F}(\lambda+\I\epsilon))(d\lambda/\pi)$, i.e.
%
\begin{align}\label{eq:weak_limit_mu}
  \int_0^1\xi(\lambda)\;\bmu(d\lambda)
  =-\frac{1}{\pi}\lim_{\epsilon\downarrow0}
        \int_0^1\xi(\lambda)\;\text{Im}(\mathbf{F}(\lambda+\I\epsilon))\, d\lambda,
\end{align}
%
for all smooth scalar test functions $\xi(\lambda)$, where
$(\mathbf{F}(s))_{jk}=F_{jk}(s)$. From equation
\eqref{eq:weak_limit_mu} and the identities 
$m_{jk}(h)=h\,w_{jk}(z)$ and $\tilde{m}_{jk}(h)=h\,\tilde{w}_{jk}(z)$,
which follow from equation \eqref{eq:m_h}, it can be
shown \cite{Murphy:JMP:063506} that the measures $\mu_{jk}$ and
$\alpha_{jk}$, and the measures $\eta_{jk}$ and $\kappa_{jk}$ are related by  
%
\begin{align}\label{eq:Measure_Relations}
  &\lambda\alpha_{jk}(\lambda)=(1-\lambda)\mu_{jk}(1-\lambda) +
       \lambda\,(\,m_{jk}(0)\delta_0(d\lambda)+w_{jk}(0)(\lambda-1)\delta_1(d\lambda)\,),
  \\
  &\lambda\kappa_{jk}(\lambda)=(1-\lambda)\eta_{jk}(1-\lambda) +
       \lambda\,(\,\tilde{m}_{jk}(0)\delta_0(d\lambda)+\tilde{w}_{jk}(0)(\lambda-1)\delta_1(d\lambda)\,).
  \notag     
\end{align}
%
Here, $m(0)=m(h)|_{h=0}$ and $w(0)=w(z)|_{z=0}$, for example, and
$\delta_a(d\lambda)$ is the delta measure concentrated at $\lambda=a$. Equations 
\eqref{eq:Stieltjes_F} and \eqref{eq:Measure_Relations} demonstrate
the many symmetries between the functions $m_{jk}(h)$, $w_{jk}(z)$,
$\tilde{m}_{jk}(h)$, and $\tilde{w}_{jk}(z)$, and the respective 
measures $\mu_{jk}$, $\alpha_{jk}$, $\eta_{ij}$, and $\kappa_{jk}$. Because of these
symmetries, for simplicity, we will focus on $m_{jk}(h)$ and $\mu_{jk}$,
and will refer to the other functions and measures where
appropriate.  



A key feature of equations \eqref{eq:eff_eps_def}, \eqref{eq:m_h}, and
\eqref{eq:Stieltjes_F} is that parameter information in $h$ and
$E_0$ is \emph{separated} from the geometry of the polycrystalline
medium, which is encoded in the spectral measure $\mu_{jk}$ via its
moments
%$\mu_{jk}^n=\int_0^1\lambda^n\mu_{jk}(d\lambda)$, $n=0,1,2,\ldots$.
%
\begin{align}\label{eq:Moments_mu}
  \mu_{jk}^n=\int_0^1\lambda^n\mu_{jk}(d\lambda)
       %=\int_0^1\lambda^n\langle X_1Q(d\lambda)\,\vec{e}_j\cdot\vec{e}_k\rangle
       =\langle X_1[\Gamma X_1]^n\,\vec{e}_j\cdot\vec{e}_k\rangle,
  \quad n=0,1,2,\ldots.,
\end{align}
%
%$n=0,1,2,\ldots.$,
which clearly contains statistical information regarding
the random projection matrix $X_1$. The mass $\mu_{jk}^0$ of the measure
$\mu_{jk}$ is given by  
% $\mu_{jk}^0=\langle X_1\vec{e}_j\cdot\vec{e}_k\rangle=\langle(X_1)_{jk}\rangle$,
%
\begin{align}\label{eq:Mass_mu}
  \mu_{jk}^0=\langle X_1\vec{e}_j\cdot\vec{e}_k\rangle, \quad
  \mu_{kk}^0=\langle |X_1\vec{e}_k|^2\rangle,
\end{align}
%
%the ensemble average of the matrix element $(X_1)_{jk}$.
where we used equation \eqref{eq:Projection_Matrices} in the second
formula. From this we see that the diagonal components $\mu_{kk}$,
$k=1\ldots,d$, of $\bmu$ are positive measures.
%The statistical average $\langle |X_1\vec{e}_k|^2\rangle$ in \eqref{eq:Mass_mu}
%can be thought of as the ``mean orientation,'' or as the percentage of
%crystallites oriented in the $k^{\text{th}}$ direction.
Since the projection 
matrix $X_1$ is bounded by one in the operator norm, the
Cauchy--Schwartz inequality and \eqref{eq:Mass_mu} imply that
$0\leq\mu_{kk}^0\leq1$. For example, in the case of  
two-dimensional polycrystalline media, $d=2$, equation
\eqref{eq:polycrystal_parameters_2D} implies that 
%$\mu_{11}^0=\langle\cos^2\theta\rangle$, $\mu_{22}^0=\langle\sin^2\theta\rangle$, and
%$\mu_{12}^0=\langle\sin\theta\cos\theta\rangle$.
%
\begin{align}
  \mu_{11}^0=\langle\cos^2\theta\rangle, \quad
  \mu_{22}^0=\langle\sin^2\theta\rangle, \quad
  \mu_{12}^0=\langle\sin\theta\cos\theta\rangle.
\end{align}
%
Equation \eqref{eq:Mass_mu} follows by  setting    
$f(M_1)=I$ ($f(\lambda)=1$) in equation  \eqref{eq:Spectral_Theorem}, which implies
that the projection valued measure satisfies $\int_0^1Q(d\lambda)=I$
\cite{Reed-1980,Stone:64}. Moreover, recall that the associated
operator $Q(\lambda)$ is a \emph{self-adjoint projector} on $\mathscr{H}_\times$
for $\lambda\in[0,1]$ \cite{Reed-1980,Stone:64}. Consequently,  we have 
% 
\begin{align}\label{eq:Mass_Sign_Measures}
   &\mu_{jk}^0%=\int_0^1\mu_{jk}(d\lambda)
         =\int_0^1\langle Q(d\lambda)\vec{e}_j\cdot\vec{e}_k\rangle_1
        %=\langle1\rangle_1\,\vec{e}_j\cdot\vec{e}_k
        =\langle\vec{e}_j\cdot\vec{e}_k\rangle_1
        =\langle X_1\vec{e}_j\cdot\vec{e}_k\rangle\\
        %=\langle X_1\rangle\,\delta_{jk},\\
   &\mu_{kk}(d\lambda)=\langle Q(d\lambda)\vec{e}_k\cdot\vec{e}_k\rangle_1
       =\langle Q(d\lambda)\vec{e}_k\cdot Q(d\lambda)\vec{e}_k\rangle_1
       =\|Q(d\lambda)\vec{e}_k\|_1^2,\notag
\end{align}
%
where we have used a Fubini theorem \cite{Folland:99}
and $\|\cdot\|_1$ denotes the norm induced by the inner-product
$\langle\cdot,\cdot\rangle_1$. From equation \eqref{eq:Mass_Sign_Measures} we see,
generically, that the diagonal components $\mu_{kk}$, $k=1,\ldots,d$, of
$\bmu$ are positive measures, while the off-diagonal components
$\mu_{jk}$, $j\neq k=1,\ldots,d$, may be signed measures \cite{Folland:99,Rudin:87}.
Generalizing equation \eqref{eq:polycrystal_parameters_3D}, with
$R=\prod_{j=1}^dR_j$, to dimensions $d\geq3$ shows that $\mu_{jk}^0$ is a
linear combination of averages of the 
form $\langle\prod_i\cos^{n_i}\theta_i\sin^{m_i}\theta_i\rangle$, where $n_i,m_i=0,1,2,\ldots$.



% The higher order moments $\mu_{jk}^n$, $n=1,2,3,\ldots$, in principle, may be
% found using a perturbation expansion of $F_{jk}(s)$ about a
% homogeneous medium $(\sigma_1=\sigma_2, \ s=\infty)$ \cite{Golden:CMP-473}. In
% particular $\mu_{jk}^0=p_1\delta_{jk}$, generically, and $\mu_{jk}^1=(p_1p_2/d)\,\delta_{jk}$
% for a statistically isotropic random medium 
% \cite{Golden:CMP-473,Golden:IMA-97,Bruno:JSP-365}, where
% $p_2=1-p_1=\langle X_2\rangle$ is the volume fraction of material component 2. In
% the case of a square bond lattice, which is an example of an
% infinitely interchangeable random medium \cite{Bruno:JSP-365},
% $\mu_{kk}^2=p_1p_2(1+(d-2)p_2)/d^{\,2}$ for any dimension $d$ and
% $\mu_{kk}^3=p_1p_2(p_2^2-p_2-1)/8$ for $d=2$. In general, the moments
% $\mu_{jk}^n$ depend on the $(n+1)$-point correlation functions of the
% random medium \cite{Golden:CMP-473,Bruno:JSP-365}.




A principal application of the ACM is to derive \emph{forward bounds}
on the diagonal components $\sigma_{kk}^*$ of the tensor $\bsig^*$,
$k=1,\ldots,d$, given partial information on the microgeometry
\cite{Bergman:PRL-1285,Milton:APL-300,Golden:CMP-473,Bergman:AP-78}. This
information may be given in terms of the moments $\mu_{kk}^n$,
$n=0,1,2,\ldots$, of the measure $\mu_{kk}$
\cite{Milton:JAP-5294,Golden:CMP-473}. Provided this information, the 
bounds on $\sigma_{kk}^*$ follow from the special structure of $F_{kk}(s)$
in \eqref{eq:Stieltjes_F}. More specifically, it is a \emph{linear}
functional of the \emph{positive} measure $\mu_{kk}$.  The bounds are
obtained by fixing the contrast parameter $s$, varying over an
admissible set of measures $\mu_{kk}$ (or geometries) which is
determined by the known information regarding the polycrystalline
medium.  Knowledge of the moments $\mu_{kk}^n$  for $n=1,\ldots,J$ confines 
$\sigma_{kk}^*$ to a region of the complex plane which is bounded by arcs
of circles, and the region becomes progressively smaller as more
moments are known \cite{Milton:JAP-5294,Golden:1986:BCP}. When 
all the moments are known the measure $\mu_{kk}$ is uniquely determined 
\cite{Akhiezer:Book:1965}, hence  $\sigma_{kk}^*$ is explicitly known. The
bounding procedure
\cite{Bergman:PRL-1285,Milton:APL-300,Golden:CMP-473,Bergman:AP-78} is
reviewed and extended to the polycrystalline setting in Section
\ref{sec:Bounding_Procedure}. 







\subsection{Lattice Setting}
\label{sec:Lattice_Setting}


\subsubsection{Infinite Lattice Setting}
\label{sec:Infinite_Lattice_Setting}



\subsubsection{Finite Lattice Setting}
\label{sec:Finite_Lattice_Setting}
%

\section{Bounding Procedure}\label{sec:Bounding_Procedure}
%
An important property of the integral representation for $F_{jk}(s)$,
$j,k=1,\ldots,d$, displayed in equation \eqref{eq:Stieltjes_F},
%and \eqref{eq:Stieltjes_F_Discrete},
is that parameter information in $s$
and $E_0$ is \emph{separated} from the geometry of the composite,
which is encoded in the spectral measure $\mu_{jk}$ via its moments  
$\mu^n_{jk}$, $n=0,1,2,\ldots$ \cite{Bruno:JSP-365,Golden:CMP-473}. Another
important property of the representation for $F_{jk}(s)$ is that it
is a \emph{linear} functional of the measure $\mu_{jk}$. Moreover, the
diagonal components $\mu_{kk}$ are \emph{positive} measures. These
properties are also shared by the function $E_{jk}(s)$ given in
equation \eqref{eq:Stieltjes_F}.
%and its discrete counterpart in \eqref{eq:Stieltjes_F_Discrete}.
These important properties may be
exploited to obtain rigorous bounds for the diagonal components of the
effective parameters
\cite{Bergman:PRC-377,Bergman:PRL-1285,Milton:APL-300,Golden:CMP-473,Bergman:AP-78}. 
In this section we review a bounding procedure which is presented in 
\cite{Golden:CMP-473,Golden:1986:BCP}. The bounds incorporate the
moments $\mu_{kk}^n$ and $\eta_{kk}^n$, $n=0,1,2,\ldots$, of the measures
$\mu_{kk}$ and $\eta_{kk}$ associated with the functions $F_{kk}(s)$ and
$E_{kk}(s)$, respectively. 



 
In this section, we will discuss the bounding procedure in terms of
the diagonal components $\sigma^*_{kk}$, $k=1,\ldots,d$, of the effective
complex conductivity tensor $\bsig^*$. For
simplicity, we will focus on one such component and set
$\sigma^*=\sigma_{kk}^*$, $F(s)=F_{kk}(s)$, $m(h)=m_{kk}(h)$, $\mu=\mu_{kk}$,
$E(s)=E_{kk}(s)$, $\tilde{m}(h)=\tilde{m}_{kk}(h)$, and
$\eta=\eta_{kk}$. Here, $\sigma^*=\sigma_2m(h)=\sigma_1/\tilde{m}(h)$,
$F(s)=1-m(h)$, and $E(s)=1-\tilde{m}(h)$. We will also exploit the
symmetries between $F(s)$ and $E(s)$ in equation
\eqref{eq:Stieltjes_F} and initially focus on the function $F(s)$ and
the measure $\mu$, introducing the function $E(s)$ and the measure $\eta$
when appropriate. 




Bounds on $\sigma^*$ are obtained as follows. The support of the measure
$\mu$ is contained in the interval $[0,1]$ and its mass is given by 
$\mu^0=\langle|X_1\vec{e}_k|^2\rangle$, where $0\leq\mu^0\leq1$. Consider the set $\mathscr{M}$ of
positive Borel measures on $[0,1]$ with mass $\leq1$. By equation
\eqref{eq:Stieltjes_F}, for fixed $s\in\mathbb{C}\backslash[0,1]$, $F(s)$ is a
linear functional of the measure $\mu$, $F:\mathscr{M}\mapsto\mathbb{C}$, and
we write $F(s)=F(s,\mu)$ and $m(h)=m(h,\mu)$. Suppose that we know the
moments $\mu^n$ of the measure $\mu$ for $n=0,\ldots,J$. Define the set
$\mathscr{M}_J^\mu\subset\mathscr{M}$
by 
% 
\begin{align}\label{eq:Measure_Set}
  \mathscr{M}_J^\mu
     =\left\{\nu\in\mathscr{M} \ \Big| \   \int_0^1\lambda^n\nu(d\lambda)=\mu^n, \  n=0,\ldots,J\right\}  . 
\end{align}
%
The set $A_J^\mu\subset\mathbb{C}$ that represents the possible
values of $m(h,\mu)=1-F(s,\mu)$ which is compatible with the
known information about the random medium is given by
%
\begin{align}\label{eq:Bounding_Set}
  A_J^\mu
     =\left\{\ m(h,\mu)\in\mathbb{C} \ | \
       \ h\not\in(-\infty,0], \ \mu\in \mathscr{M}_J^\mu\right\}. 
\end{align}
%



The set of measures $\mathscr{M}_J^\mu$ is a compact, convex
subset of $\mathscr{M}$ with the topology of weak convergence
\cite{Golden:CMP-473}. Since the mapping $F(s,\mu)$ in
\eqref{eq:Stieltjes_F} is linear in $\mu$, it follows that
$A_J^\mu$ is a compact convex subset of the complex plane
$\mathbb{C}$. The extreme points of $\mathscr{M}_0^\mu$ are the one 
point measures $a\delta_b$, $0\leq a,b\leq1$ \cite{Dunford_Schwartz:LinOp_PtI},
while the extreme points of $\mathscr{M}_J^\mu$ for $J>0$ are weak limits
of convex combinations of measures of the form
\cite{Karlin_Studden:Book:1966,Golden:CMP-473}  
%
\begin{align}\label{eq:Discrete_Measure}
  \mu_J(d\lambda)=\sum_{i=1}^{J+1}a_i\delta_{b_i}(d\lambda), \quad
  a_i\geq0, \quad 0\leq b_1<\cdots<b_{J+1}<1, \quad
  \sum_{i=1}^{J+1}a_ib_i^n=\mu^n,
%  \quad   n=0,1,2,\ldots.
\end{align}
%
for $n=0,1,\ldots,J$.
%Unresolved issue:
(WHAT CAN WE SAY ABOUT UNIAXIAL POLYCRYSTALLINE MEDIA WHICH IS
ANALOGOUS TO THE FOLLOWING PROPERTIES OF TWO-COMPONENT COMPOSITE MEDIA?
For the case of two-component random media in the
continuous setting, every measure $\mu\in\mathscr{M}_J^\mu$ gives rise
to a function $m(h,\mu)$ that is the effective (relative) conductivity
of a multi-rank laminate \cite{MILTON:2002:TC}.
%The analogous result for $d>2$ is not known.
However, in general \cite{Golden:CMP-473},  not every measure
$\mu\in\mathscr{M}_J^\mu$ gives rise to such a function $m(h,\mu)$. Therefore,
the set $A_J^\mu$ will only \emph{contain} the exact range of values of the
effective conductivity \cite{Golden:CMP-473}. This is sufficient for
the bounding procedure discussed in this section. 






By the symmetries between the formulas in equation
\eqref{eq:Stieltjes_F} and $X_1+X_2=I$ in
\eqref{eq:Projection_Matrices}, the support of the measure $\eta$ 
is contained in the interval $[0,1]$ and its mass is given by
$\eta^0=\langle|X_2\,\vec{e}_k|^2\rangle=\langle X_2\,\vec{e}_k\cdot\vec{e}_k\rangle=1-\mu^0$, which
implies that
$0\leq\eta^0\leq1$. We can therefore define compact, convex sets
$\mathscr{M}_J^\eta\subset\mathscr{M}$ and $A_J^\eta\subset\mathbb{C}$ which are
analogous to those defined in equations \eqref{eq:Measure_Set} and
\eqref{eq:Bounding_Set}, respectively, involving the function
$\tilde{m}(h,\eta)=1-E(s,\eta)$. Moreover, the extreme points of
$\mathscr{M}_0^\eta$ are the one point measures $c\delta_d$, $0\leq c,d\leq1$ 
while the extreme points of $\mathscr{M}_J^\eta$ are weak limits
of convex combinations of measures of the form given in equation
\eqref{eq:Discrete_Measure}. 



Consequently, in order to determine the extreme
points of the sets $A_J^\mu$ and $A_J^\eta$ it suffices to determine the
range of values in $\mathbb{C}$ of the functions $m(h,\mu_J)=1-F(s,\mu_J)$
and $\tilde{m}(h,\eta_J)=1-E(s,\eta_J)$, respectively, where  
%
\begin{align}\label{eq:Discrete_mh}
  F(s,\mu_J)=\sum_{i=1}^{J+1}\frac{a_i}{s-b_i}\,, \qquad
  E(s,\eta_J)=\sum_{i=1}^{J+1}\frac{c_i}{s-d_i}\,,
\end{align}
as the $a_i$, $b_i$, $c_i$, and $d_i$ vary under the
constraints given in equation  \eqref{eq:Discrete_Measure}. While
$F(s,\mu_J)$ and $E(s,\eta_J)$ in 
\eqref{eq:Discrete_mh} may not run over all points in $A_J^\mu$ and
$A_J^\eta$ as these parameters vary, they run over the
extreme points of these sets, which is sufficient due to their
convexity. It is important to note that, as the effective complex
conductivity $\sigma^*$ is given by $\sigma^*=\sigma_2m(h,\mu)=\sigma_1/\tilde{m}(h,\eta)$, the
regions $A_J^\mu$ and $A_J^\eta$ have to be mapped to the common
$\sigma^*$-plane to provide bounds for $\sigma^*$.    





In this section we discuss two different bounds for $\sigma^*$. The first
bound assumes that only the masses $\mu^0$ and $\eta^0$ of the measures $\mu$
and $\eta$ are known. While the second bound also assumes that the random
medium is statistically isotropic, so that the first moments of these
measures are also known, as described by the following lemma.
%
\begin{lem}\label{lem:first_moment}
%
If the orientations of the crystallites are statistically isotropic
then
%
\begin{align}\label{eq:First_Moments}
  \mu^1=\frac{d-1}{d^3}\,, \qquad
  \eta^1=\frac{p_1p_2(d-1)}{d}\,.
\end{align}
%
\end{lem}
%


Consider the first case, where $J=0$  in \eqref{eq:Discrete_mh} and the
volume fraction $p_1=1-p_2$ is fixed with $\mu^0=p_1$ and
$\eta^0=p_2=1-p_1$, so that $F(s,\mu_J)=p_1/(s-\lambda)$ and
$E(s,\eta_J)=p_2/(s-\tilde{\lambda})$. By the above discussion, the values of 
$F(s,\mu$) and $E(s,\eta)$ lie inside the circles $C_0(\lambda)$ and
$\tilde{C}_0(\tilde{\lambda})$, respectively, given by  
%
\begin{align}\label{eq:0th_order_Bounds}
    C_0(\lambda)=\frac{\mu^0}{s-\lambda}\,, \quad -\infty\leq\lambda\leq \infty, \qquad
    \tilde{C}_0(\tilde{\lambda})=\frac{\eta^0}{s-\tilde{\lambda}}\,, \quad
    -\infty\leq\tilde{\lambda}\leq \infty. 
\end{align}
%
In the $\sigma^*$-plane, the intersection of these two regions is bounded by
two circular arcs corresponding to $0\leq\lambda\leq p_2$ and $0\leq\tilde{\lambda}\leq p_1$
in \eqref{eq:0th_order_Bounds}, and the values of $\sigma^*$ lie inside
this region \cite{Golden:1986:BCP}. These bounds are optimal
\cite{Milton:JAP-5286,Bergman:AP-78}, and are obtained by a composite
of uniformly aligned spheroids of material 1 in all sizes coated with
confocal shells of material 2, and vice versa. The arcs are traced out
as the aspect ratio varies. When the value of the component
permittivities $\sigma_1$ and $\sigma_2$ are real and positive, the bounding
region collapses to the interval
%$1/(\mu^0/\sigma_1+\eta^0/\sigma_2)\leq\sigma^*\leq \mu^0\sigma_1+\eta^0\sigma_2$,
%
\begin{align}\label{eq:Wiener_bounds}
  \left(\frac{\mu^0}{\sigma_1}+\frac{\eta^0}{\sigma_2}\right)^{-1}\leq\sigma^*\leq \mu^0\sigma_1+\eta^0\sigma_2,
\end{align}
%
 which are the Wiener
bounds. The lower and upper bounds are obtained by parallel slabs of
the two materials aligned perpendicular and parallel to the field
$\vec{E}_0$, respectively \cite{Scaife-1989}.



Now consider the second case where $J=1$ in \eqref{eq:Discrete_mh},
the volume fraction $p_1=1-p_2$ is fixed, and the random medium is
statistically isotropic so that the first moments $\mu^1$ and $\eta^1$ of
the measures $\mu$ and $\eta$ are given, respectively, by that in equation  
\eqref{eq:First_Moments}.  A convenient way of including this
information is to use the transformations \cite{Bergman:AP-78}
%
\begin{align}\label{eq:Aux_Fs_Es}
  F_1(s)=\frac{1}{p_1}-\frac{1}{sF(s)}\,, \qquad
  E_1(s)=\frac{1}{p_2}-\frac{1}{sE(s)}\,.
\end{align}
%
Due to the symmetries between $F_1(s)$ and $E_1(s)$ in
\eqref{eq:Aux_Fs_Es} we will first focus on the function $F_1(s)$ and
introduce the function $E_1(s)$ when appropriate. The function
$F_1(s)$ is an upper half plane function analytic off $[0,1]$ and
therefore has an integral representation
\cite{Bergman:AP-78,Golden:1986:BCP} analogous to that in equation
\eqref{eq:Stieltjes_F}, involving a
measure $\mu_1$, say, which is supported in the interval $[0,1]$. Since
only the mass $\mu^0$ and the first moment $\mu^1=(d-1)/d^3$ of the
measure $\mu$ are known, the transformation \eqref{eq:Aux_Fs_Es}
determines only the mass $\mu_1^0=(d-1)/d$ of the measure $\mu_1$
\cite{Bergman:AP-78,Golden:1986:BCP}. This reveals the utility of the
transformation $F_1(s)$ in \eqref{eq:Aux_Fs_Es}, it reduces the second
case $(J=1)$ for $F(s)$ to the first case $(J=0)$ for $F_1(s)$.



By our previous analysis, the values of $F_1(s)$ lie inside a circle
$p_2/(p_1d(s-\lambda))$, $-\infty\leq\lambda\leq\infty$. Similarly, the values of $E_1(s)$ lie
inside a circle $p_1(d-1)/(p_2d(s-\tilde{\lambda}))$,
$-\infty\leq\tilde{\lambda}\leq\infty$. Since $F$ and $E$ are fractional linear in $F_1$ and
$E_1$, respectively, these circles are transformed to the circles
$C_1(\lambda)$ in the $F$-plane and $\tilde{C}_1(\tilde{\lambda})$ in the
$E$-plane given by \cite{Golden:1986:BCP}
%
\begin{align}\label{eq:Isotropic_Bounds}
  C_1(\lambda)=\frac{s-\lambda}{sd(s-\lambda-(d-1)/d^2)}\,, \quad  %&-\infty\leq\lambda\leq\infty, \quad
  \tilde{C}_1(\tilde{\lambda})=\frac{(d-1)(s-\tilde{\lambda})}{sd(s-\tilde{\lambda}-(d-1)/d^2)}\,,
   \qquad -\infty\leq\lambda,\tilde{\lambda}\leq\infty. %\notag
\end{align}
%
In the $\sigma^*$-plane the intersection of these two circular regions is
bounded by two circular arcs \cite{Golden:1986:BCP} corresponding to
$0\leq\lambda\leq(d-1)/d$ and $0\leq\tilde{\lambda}\leq1/d$ in \eqref{eq:Isotropic_Bounds}.




The vertices of the region,
$C_1(0)=p_1/(s-p_2/d)$ and $\tilde{C}(0)=p_2/(s-p_1(d-1)/d)$, are
attained by the Hashin--Shtrikman geometries (spheres of all
sizes of material 1 in the volume fraction $p_1$ coated with spherical
shells of material 2 in the volume fraction $p_2$ filling all of
$\mathbb{R}^d$, and vice versa), and lie on the arcs of the first
order bounds \cite{Golden:1986:BCP}. While there are at least five
points on the arc $C_1(\lambda)$ in \eqref{eq:Isotropic_Bounds} that are
attainable by composite microstructures \cite{Milton:JAP-5286}, the
arc $\tilde{C}_1(\tilde{\lambda})$ in \eqref{eq:Isotropic_Bounds} violates
\cite{Golden:1986:BCP} the interchange inequality $m(h)m(1/h)\geq1$
\cite{Keller:1964:TCC,Schulgasser:1976:CFR}, which becomes an equality
in two dimensions \cite{MILTON:2002:TC}.  Consequently the isotropic
bounds in \eqref{eq:Isotropic_Bounds} are not optimal, but have been
improved \cite{Milton:APL-300,Bergman:AP-78} by incorporating the
interchange inequality. When $\sigma_1$ and $\sigma_2$ are real and positive
with $\sigma_1\leq\sigma_2$, the region collapses to the interval
%
\begin{align}
  \sigma_1-\frac{\sigma_1}{sd-(d-1)/d}.
  \leq\sigma^*\leq 
  \frac{\sigma_1}{1-((d-1)/(sd-(d-1)/d))}
\end{align}
%
which are the Hashin--Shtrikman bounds.




The higher moments $\mu^n$, $n\geq2$ depend on $(n+1)$-point correlation
functions \cite{Golden:CMP-473} and cannot be calculated in general,
although the interchange inequality forces relations among them
\cite{Milton:JAP-5294}. If the moments $\mu^0,\ldots,\mu^J$ are known then the
transformation $F_1$ in \eqref{eq:Aux_Fs_Es} can be iterated to
produce an upper half plane function $F_J$ with a integral
representation, involving a positive measure $\mu_J$ which is supported
on the interval $[0,1]$. As in the case where $J=1$, the first $J$
moments of the measure $\mu$ determine only the mass $\mu_J^0$ of the
measure $\mu_J$ \cite{Golden:1986:BCP}, and the function $F_J(s)$ can
easily be extremized by the above procedure, and similarly for a
function $E_J(s)$ associated with the moments $\eta^0,\ldots,\eta^J$. The
resulting bounds form a nested sequence of lens-shaped regions
\cite{Golden:1986:BCP}.




\section{Numerical Results}\label{sec:Numerical_Results}
%
In Section \ref{sec:Finite_Lattice_Setting} we extended the ACM for
representing transport in composites to the finite lattice
setting. Here, we demonstrate how this mathematical 
framework can be utilized to compute spectral measures and the
associated effective parameters for such two-phase random media. In
particular, in the finite lattice setting, the operators $\Gamma$, $\Upsilon$, and 
$\chi_i$, $i=1,2$, are represented as real-symmetric matrices and the
spectral measures of the associated random matrices $M_i=\chi_i\Gamma\chi_i$ and 
$K_i=\chi_i\Upsilon\chi_i$ are explicitly determined by their eigenvalues and
eigenvectors, as displayed in equation
\eqref{eq:Stieltjes_F_Discrete}. In Section
\ref{sec:Finite_Lattice_Setting} we also introduced a projection method,
summarized by equations \eqref{eq:Spec_Decomp_chi_Gamma_chi} and
\eqref{eq:measure_equivalence}, which provides a numerically efficient
way to accomplish these computations. Furthermore, in the paragraph
following the statement of Theorem
\ref{thm:Discrete_Spectral_Theorem_ACM}, we introduced three 
classes of locally isotropic, statistically isotropic, and anisotropic
random media. In this section we employ the projection method to
directly calculate the spectral measures and effective parameters for
such composite media.




\medskip

{\bf Acknowledgements.}
We gratefully acknowledge support from the Division of Mathematical
Sciences and the Office of Polar Programs at the U.S. 
National Science Foundation (NSF) through Grants
DMS-1009704, ARC-0934721, and DMS-0940249. We are also grateful for 
support from the Office of Naval Research (ONR) through
Grants N00014-13-10291 and N00014-12-10861. Finally, we would like to 
thank the NSF Math Climate Research Network (MCRN) for their support
of this work. 


\medskip

\bibliographystyle{plain}
\bibliography{murphy}
\end{document}

% LocalWords:  McMaster RM jk sig eps def Maxwells Milgram coercivity diag jX
% LocalWords:  mh Cond mu kk PtI Fs Es Hashin Shtrikman extremized Decomp chi
% LocalWords:  Acknowledgements DMS ONR MCRN murphy
