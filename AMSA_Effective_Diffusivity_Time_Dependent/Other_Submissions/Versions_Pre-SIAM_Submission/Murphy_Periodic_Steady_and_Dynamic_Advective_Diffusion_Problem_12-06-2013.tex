\documentclass[11pt]{amsart}
%\usepackage{latexsym, amssymb, enumerate, amsmath}
\usepackage{graphicx,amssymb,amsmath,amsfonts,mathrsfs}

\setlength{\textwidth}{6.5in}
\setlength{\textheight}{9.0in}
\setlength{\oddsidemargin}{0in}
\setlength{\evensidemargin}{0in}
\setlength{\topmargin}{-0.5in}

\renewcommand{\topfraction}{0.85}
\renewcommand{\textfraction}{0.1}
\renewcommand{\floatpagefraction}{0.55}%0.75


\newcommand{\ph}{\hat{\phi}}
\newcommand{\pt}{\tilde{\phi}} 
\newcommand{\pc}{\check{\phi}}
\newcommand{\gh}{\hat{\gamma}}
\newcommand{\Dh}{\hat{\Delta}}
\newcommand{\dha}{\hat{\delta}}
\newcommand{\qh}{\hat{q}}
\newcommand{\xh}{\hat{x}}
\newcommand{\HM}{\mathcal{H}_{\text{max}}}
\newcommand{\Hm}{\mathcal{H}_{\text{min}}}
\newcommand{\sech}{\rm \hspace{0.7mm}sech}
\newcommand{\I}{\mathrm{i}}
\newcommand{\e}{\mathrm{e}}
\renewcommand{\d}{\mathrm{d}}
\newcommand{\hh}{\hat{h}}
\newcommand{\mh}{m_r}
\newcommand{\mt}{m_i}
\newcommand{\Mb}{\mathbf{M}}
\newcommand{\Xb}{\mathbf{X}}
\newcommand{\Tb}{\mathbf{T}}
\newcommand{\Hb}{\mathbf{H}}
\newcommand{\Kb}{\mathbf{K}}
\newcommand{\Jb}{\mathbf{J}}
\newcommand{\Ib}{\mathbf{I}}

\newcommand\bsig{\mbox{\boldmath${\sigma}$}}
\newcommand\beps{\mbox{\boldmath${\epsilon}$}}
\newcommand\bxi{\mbox{\boldmath${\xi}$}}
\newcommand\bmu{\mbox{\boldmath${\mu}$}}
\newcommand\balpha{\mbox{\boldmath${\alpha}$}}
\newcommand\brho{\mbox{\boldmath${\rho}$}}
\newcommand\bDelta{\mbox{\boldmath${\Delta}$}}
\newcommand\bkappa{\mbox{\boldmath${\kappa}$}}


\newtheorem{thm}{Theorem}[section]
\newtheorem{prop}[thm]{Proposition}
\newtheorem{lem}[thm]{Lemma}
\newtheorem{cor}[thm]{Corollary}

    %\theoremstyle{definition}

\newtheorem{defn}[thm]{Definition}
\newtheorem{notation}[thm]{Notation}
\newtheorem{example}[thm]{Example}
\newtheorem{conj}[thm]{Conjecture}
\newtheorem{prob}[thm]{Problem}

    %\theoremstyle{remark}

\newtheorem{rem}[thm]{Remark}
    % Use the standard latex environments for theorems, etc. Here is one
          % possible method of declaring them: It numbers all results by the
          % section, and uses a common numbering system for the different
          % environmentts.

\begin{document}

\title{Spectral theory of advective diffusion \\
  by dynamic and steady periodic flows}


% AUTHORS 
%\author{N. B. Murphy, A. Gully, E. Cherkaev, and K. M. Golden}
\author{N. B. Murphy$^\ast$}
\address{$^*$Department of Mathematics, 340 Rowland Hall, University of
  California at Irvine, Irvine, CA 92697-3875, USA}
\email{nbmurphy@math.uci.edu}

\author{J. Xin$^\dag$}
\address{$^{\dag}$Department of Mathematics, 340 Rowland Hall, University of
  California at Irvine, Irvine, CA 92697-3875, USA} 
\email{jxin@math.uci.edu}

\author{J. Zhu$^\star$}
\address{$^\star$University of Utah, Department of Mathematics, 155 S 1400 E
  RM 233, Salt Lake City, UT 84112-009, USA}
\email{zhu@math.utah.edu}

\author{E. Cherkaev$^\ddagger$}
\address{$^\ddagger$University of Utah, Department of Mathematics, 155 S 1400 E
  RM 233, Salt Lake City, UT 84112-009, USA} 
\email{elena@math.utah.edu}

\maketitle
\vspace{-3ex}
\begin{center}
  Department of Mathematics, University of California at Irvine
\end{center}

%\vspace{3ex}


\begin{abstract}
%
The analytic continuation method for representing transport in
composites provides Stieltjes integral representations for the
effective coefficients of two-phase random media. Here we adapt this 
method to characterize the effective thermal transport properties of
advective diffusion, by steady and time-dependent, periodic flows. Our
novel approach yields a Stieltjes integral representation for the
effective diffusivity, which holds for dynamic and steady,
incompressible flows, involving the spectral measure of a self-adjoint
linear operator. In the case of steady fluid velocity fields, the
spectral measure is associated with a Hermitian Hilbert-Schmidt
integral operator, and in the case of dynamic flows, it is associated
with an unbounded integro-differential operator. We utalize the
integral representation to obtain asymptotic behavior of the effective 
diffusivity, as the molecular diffusivity tends to zero, for model,
steady and dynamic flows. Our analytical results are supported by
numerical computations of the spectral measures and effective
diffusivities.    
%
\end{abstract}

\section{Introduction}\label{sec:Introduction}
%
The long time, large scale behavior of a diffusing particle   
or tracer being advected by an incompressible velocity field 
is equivalent to an enhanced diffusive process \cite{Taylor:PRSL:196} 
with an effective diffusivity tensor $\bkappa^*$.
Determining the effective transport properties of advection enhanced
diffusion is a challenging problem with theoretical and practical 
importance in many fields of science and engineering,
ranging from turbulent combustion to mass, heat, and salt transport in
geophysical flows \cite{Moffatt:RPP:621}. A broad range of
mathematical techniques have been developed that reduce the analysis
of complex fluid flows, with rapidly varying structures in space and
time, to solving averaged or \textit{homogenized} equations that do
not have rapidly varying data, and involve an effective parameter.




Homogenization of the advection-diffusion equation for thermal
transport by time-independent, random fluid velocity fields was
treated in \cite{McLaughlin:SIAM_JAM:780}. This 
reduced the analysis of turbulent diffusion to solving an anisotripic
diffusion equation involving a homogenized temperature and an
effective diffusivity tensor $\bkappa^*$. An important consequence of
this analysis is that $\bkappa^*$ is given in terms  
of a \emph{curl-free} stationary stochastic process which satisfies a
steady state diffusion equation involving a skew-symmetric random
matrix $\Hb$ \cite{Avellaneda:CMP-339,Avellaneda:PRL-753}. By adapting
the analytic continuation method (ACM) of homogenization theory for
composites \cite{Golden:CMP-473}, it was shown that the result in
\cite{McLaughlin:SIAM_JAM:780} leads to a Stieltjes integral
representation of $\bkappa^*$, involving a spectral measure of a
self-adjoint random operator
\cite{Avellaneda:CMP-339,Avellaneda:PRL-753}. This integral
representation of $\bkappa^*$ was generalized to the time-dependent
case in \cite{Avellaneda:PRE:3249,Biferale:PF:2725}. Remarkably, this 
method has also been extended to flows with incompressible
\emph{nonzero} effective drift 
\cite{Pavliotis:PHD_Thesis}, flows where particles diffuse according
to linear collisions \cite{Pavliotis:IMAJAM:951}, and solute transport
in porous media \cite{Bhattacharya:AAP:1999:951}. All these approaches
yield Stieltjes integral representations of the symmetric and, when
appropriate, the antisymmetric part of $\bkappa^*$.




Homogenization of the advection-diffusion equation for periodic or
cellular, incompressible flow fields was treated in
\cite{Fannjiang:SIAM_JAM:333,Fannjiang:1997:1033}. As in the case of
random flows, the effective diffusivity tensor
$\bkappa^*$ is given in terms of a \emph{curl-free} vector field which
satisfies a diffusion equation involving a skew-symmetric
matrix $\Hb$. Here, we demonstrate that the ACM can
be adapted to this periodic setting to provide a Stieltjes
integral representation for $\bkappa^*$, both for steady and
time-dependent flows, involving a self-adjoint linear operator and the 
(non-dimensional) molecular diffusivity $\varepsilon$. In the case of steady
fluid velocity fields, the spectral measure is associated with a
Hermitian Hilbert-Schmidt integral operator involving the Green's
function of the Laplacian on a square. While in the case of dynamic
flows, the spectral measure is associated with a Hermitian operator
which is the sum of that for steady flows and an unbounded
integro-differential operator which is a composition of the time
derivative and the inverse Laplacian on a square.     
 

We utalize the analytic structure of the Stieltjes integral
representation for $\bkappa^*$ to obtain its asymptotic behavior for
model flows, as the molecular diffusivity $\varepsilon$ tends to zero. This is
the high P\'{e}clet number regime that is important for the
understanding of transport processes in real fluid flows, where the
molecular diffusivity is often quite small in comparison. In
particular, FINISH THIS PARAGRAPH WHEN WE HAVE CONCRETE RESULTS.
necessary and sufficient conditions for steady periodic flow
fields $\kappa^*\sim\epsilon^{1/2}$, generically, for steady flows and $\kappa^*\sim O(1)$ for
``chaotic'' time-dependent flows. 

%
\section{Mathematical Methods}\label{sec:Mathematical_Methods} 
%

Consider the advection enhanced difusion of a passive tracer
$\phi(t,\vec{x})$, $\vec{x}\in\mathbb{R}^d$, $t>0$, as described by the
advection-diffusion equation 
%
\begin{align}\label{eq:ADE}
  \partial_t\phi=\kappa_0\Delta \phi+\vec{v}\cdot\vec{\nabla}\phi, \quad
  \phi(0,\vec{x})=\phi_0(\vec{x}),
\end{align}
%
with $\phi_0(\vec{x})$ given. Here, $\partial_t$ denotes partial differentiation
with respect to time $t$, $\Delta=\vec{\nabla}\cdot\vec{\nabla}=\nabla^2$ is the Laplacian,
$\kappa_0>0$ is the molecular diffusivity, and $\vec{v}=\vec{v}(t,\vec{x})$
is the fluid velocity field, which is assumed to be incompressible,
i.e. $\vec{\nabla}\cdot\vec{v}=0$. We non-dimensionalize equation
\eqref{eq:ADE} as follows. Let $\ell$ and $\tau$ be typical length and time 
scales associated with the problem of interest. Mapping to
non-dimensional coordinates $t\mapsto t/\tau$ and $x_i\mapsto x_i/\ell$ in
\eqref{eq:ADE}, one sees that $\phi$ satisfies the advection-diffusion
equation in \eqref{eq:ADE} with a non-dimensional molecular
diffusivity $\varepsilon=\tau\kappa_0/\ell^2$ and velocity field $\vec{u}=\tau\vec{v}/\ell$,
where $x_i$ is the $i^{\text{th}}$ component of the vector
$\vec{x}$. Since $\vec{u}$ is incompressible, there is a
(non-dimensional) skew-symmetric matrix $\Hb(t,\vec{x})$,
$\Hb^{\,T}=-\Hb$, such that $\vec{u}=\vec{\nabla}\cdot\Hb$. 
% %
% \begin{align}
%  \vec{u}=\vec{\nabla}\cdot\Hb, \qquad  \Hb^{\,T}=-\Hb.
% \end{align}
% %
Using this representation of the velocity field $\vec{u}$, equation
\eqref{eq:ADE} can be written as a diffusion equation, 
%
\begin{align}\label{eq:ADE_Divergence}
  \partial_t\phi%&=\varepsilon\Delta \phi+\vec{u}\cdot\vec{\nabla}\phi\\
    %&=\varepsilon\vec{\nabla}\cdot\vec{\nabla}\phi+(\vec{\nabla}\cdot\Hb)\cdot\vec{\nabla}\phi\\
    %&=\vec{\nabla}\cdot[\varepsilon I+\Hb]\vec{\nabla}\phi\\
    %&=\vec{\nabla}\cdot\bkappa\vec{\nabla}\phi
    =\vec{\nabla}\cdot\bkappa\vec{\nabla}\phi, \quad
    %\bkappa=\varepsilon I+\Hb,
    \phi(0,\vec{x})=\phi_0(\vec{x}),
    \qquad
    \bkappa=\varepsilon\Ib+\Hb,
\end{align}
%
where $\bkappa(t,\vec{x})=\varepsilon\Ib+\Hb(t,\vec{x})$ can be viewed as a local
diffusivity tensor with coefficients
%
\begin{align}\label{eq:kappa_coeff}
  \kappa_{jk}=\varepsilon\delta_{jk}+H_{jk},\quad j,k=1,\cdots,d,
\end{align}
%
$\Ib$ is the identity matrix on $\mathbb{R}^d$, and $\delta_{jk}$ is 
the Kronecker delta. 




We are interested in the dynamics of $\phi$ for \emph{large} length and
time scales. Anticipating that $\phi$ will have diffusive dynamics, we
rescale space and time by $\vec{x}\mapsto\vec{x}/\delta$ and $t\mapsto t/\delta^2$,
respectively, with $\delta\ll1$ while keeping the intital condition
$\phi(0,\vec{x})=\phi_0(\vec{x})$ independent 
of $\delta$. This is equivalent to assuming that the initial data is slowly
varying relative to the velocity field $\vec{u}$ in the unscaled
variables
\cite{McLaughlin:SIAM_JAM:780,Fannjiang:SIAM_JAM:333,Fannjiang:1997:1033}. 
For periodic diffusivity coefficients in \eqref{eq:kappa_coeff} which
are uniformly elliptic but not necessarily symmetric, it can be shown 
\cite{Fannjiang:SIAM_JAM:333} that the associated solution
$\phi^\delta(t,\vec{x})=\phi(t/\delta^2,\vec{x}/\delta)$  of \eqref{eq:ADE_Divergence},
involving the rescaled local diffusivity tensor
$\bkappa(t/\delta^2,\vec{x}/\delta)$, converges to $\bar{\phi}(t,\vec{x})$, which
satisfies a diffusion equation involving the (constant)
diffusivity tensor $\bkappa^*$
%
\begin{align}
  \partial_t\bar{\phi}=\vec{\nabla}\cdot\bkappa^*\vec{\nabla}\bar{\phi}, \quad
  \bar{\phi}(0,\vec{x})=\phi_0(\vec{x}).
\end{align}
%
The convergence is in $L^2$ \cite{Fannjiang:SIAM_JAM:333},
%
\begin{align}
  \lim_{\delta\to0}\left[\,\sup_{0\leq t\leq t_0}
    \int\left|\phi^\delta(t,\vec{x})-\bar{\phi}(t,\vec{x})\right|^2\d\vec{x}
    \;\right]=0,
\end{align}
%
for all $t_0<\infty$, where we used the notation $\d\vec{x}=\d x_1\cdots \d x_d$ for
the product Lebesgue measure.





The effective diffusivity tensor $\bkappa^*$ is
obtained by solving the cell problem \cite{Fannjiang:SIAM_JAM:333}
%
\begin{align}\label{eq:Cell_Problem}
  \partial_t\chi=\vec{\nabla}\cdot\bkappa(\vec{\nabla}\chi+\vec{e}_j\,), \quad
    %=\vec{\nabla}\cdot[(\varepsilon\Ib+\Hb)(\vec{\nabla}\chi+\vec{e}_j\,)], \quad
  \langle\vec{\nabla}\chi\rangle=0,
\end{align}
%
for each standard basis vector $\vec{e}_j$, $j=1,\ldots,d$, where
$\chi=\chi(t,\vec{x}\,;\vec{e}_j)$. Equation \eqref{eq:Cell_Problem} also holds
\cite{Fannjiang:SIAM_JAM:333} when the velocity field is
time-independent, $\vec{u}=\vec{u}(\vec{x})$, however, in this case $\chi$ 
is time-independent and $\partial_t\chi=0$.  The 
components $\kappa^*_{jk}=\bkappa^*\vec{e}_j\cdot\vec{e}_k$ of the effective
diffusivity tensor are given in terms of the \emph{curl-free} vector
field $\vec{\nabla}\chi$ \cite{Fannjiang:SIAM_JAM:333} 
% 
\begin{align}\label{eq:Eff_Diffusivity}  
   \kappa^*_{jk}=\varepsilon\langle(\vec{\nabla}\chi+\vec{e}_i\,)\cdot(\vec{\nabla}\chi+\vec{e}_j\,)\rangle
       =\varepsilon(\delta_{jk}+\langle\vec{\nabla}\chi\cdot\vec{\nabla}\chi\rangle).
\end{align}
%.
Here, $\langle\cdot\rangle$ denotes spatial averaging over a period cell when the
velocity field is time-independent, $\vec{u}=\vec{u}(\vec{x})$, and
when the velocity field is time-dependent, $\vec{u}=\vec{u}(t,\vec{x})$,
$\langle\cdot\rangle$ denotes time averaging over a temporal period in addition to
spacial averaging. Equation \eqref{eq:Eff_Diffusivity} demonstrates
that the effective transport of the tracer $\phi$ is always inhanced by the
presence of a incompressible velocity field. 




We now recast \eqref{eq:Cell_Problem} 
in a more suggestive form by writing 
$\partial_t\chi=\Delta\Delta^{-1}\partial_t\chi=\vec{\nabla}\cdot(\Delta^{-1}\partial_t\Ib)\vec{\nabla}\chi$ and defining
$\vec{E}_k=\vec{\nabla}\chi+\vec{e}_k$ and 
$\bsig=\bkappa-\Delta^{-1}\partial_t\Ib=((\varepsilon-\Delta^{-1}\partial_t)\Ib+\Hb)$. With these
definitions, the first formulat in equation \eqref{eq:Cell_Problem}
may be written as $\vec{\nabla}\cdot\bsig\vec{E}_k=0$. Consequently
\eqref{eq:Cell_Problem} is equivalent to \cite{Fannjiang:SIAM_JAM:333}    
%
\begin{align}\label{eq:Maxwells_Equations}    
  \vec{\nabla}\cdot\vec{J}_k=0, \quad
  \vec{\nabla}\times\vec{E}_k=0, \quad
  \vec{J}_k=\bsig\vec{E}_k,\quad
  \langle\vec{E}_k\rangle=\vec{e}_k,
\end{align}
%
where $\Delta^{-1}$ is based on convolution with the Green's function for
the Laplacian $\Delta$ and $\bsig=\bkappa$ in the steady flow
case when $\vec{u}=\vec{u}(\vec{x})$. The formulas in
\eqref{eq:Maxwells_Equations} are  
precisely the electrostatic version of Maxwell's equations for a
conductive medium \cite{Golden:CMP-473}, where $\vec{E}_k$ and
$\vec{J}_k$ are the local electric field and current density,
respectively, and $\bsig$ is the local conductivity tensor of the
medium. In the ACM for treating the effective transport properties of
composites, the effective conductivity tensor $\bsig^*$ is defined by 
%
\begin{align}\label{eq:sigma*}
  \langle\vec{J}_k\rangle=\bsig^*\langle\vec{E}_k\rangle.
\end{align}
%
The linear constituative relation $\vec{J}_k=\bsig\vec{E}_k$ in
\eqref{eq:Maxwells_Equations} relates the local intensity and flux,
while the linear relation in \eqref{eq:sigma*} relates the mean
intensity and mean flux. By the skew-symmetry of the operator
$\mathbf{S}=-\Delta^{-1}\partial_t\Ib+\Hb$, with respect to the inner-product
defined by $\langle\vec{\xi},\vec{\zeta}\rangle=\langle\vec{\xi}\cdot\vec{\zeta}\rangle$, the intensity-flux
relationship in \eqref{eq:Maxwells_Equations} is similar to
that of a Hall medium \cite{Isichenko:JNS:1991:375}. Moreover, the
skew-symmetry of this operator implies that
$\langle\mathbf{S}\vec{E}_k\cdot\vec{E}_k\rangle=-\langle\mathbf{S}\vec{E}_k\cdot\vec{E}_k\rangle=0$. Consequently,
by equation \eqref{eq:Maxwells_Equations} and the Helmholtz theorem
\cite{Golden:CMP-473,Denaro:2003:0271,Bhatia:IEE:1077}, equation
\eqref{eq:sigma*} reduces to \eqref{eq:Eff_Diffusivity}:    
%
\begin{align}
  \sigma^*_{kk}=\bsig^*\vec{e}_k\cdot\vec{e}_k 
       =\langle\bsig\vec{E}_k\cdot\vec{e}_k\rangle
       =\langle\bsig\vec{E}_k\cdot\vec{E}_k\rangle
       =\langle[(\varepsilon\Ib+\mathbf{S})]\vec{E}_k\cdot\vec{E}_k\rangle
       =\varepsilon\langle\vec{E}_k\cdot\vec{E}_k\rangle.
\end{align}
%
% In general, $\bkappa^*$ is nonsymmetric. However, when the matrix
% $\Hb$ satisfies $\Hb(t,\$
% We are primarilly concerned with the diagonal components $\kappa^*_{kk}$ of
% $\bkappa^*$, and for simplicity we set $\kappa^*=\kappa^*_{kk}$ for some $k=1,\ldots,d$.
We will discuss this connection to the ACM in more detail in Section
\ref{sec:Integral_Rep}, where we provide a Stieltjes integral
representation for $\bkappa^*$, which holds for both for steady and
time-dependent flows. 




\subsection{Integral representation of the effective diffusivity for
  steady and dynamic flows}\label{sec:Integral_Rep}
%
In this section, we adapt the ACM for representing transport
incomposites \cite{Golden:CMP-473} to equations
\eqref{eq:Maxwells_Equations} and \eqref{eq:sigma*} for advective
diffusion by periodic flows. This method gives an abstract Hilbert
space formulation of the effective parameter problem for $\bkappa$ and
provides a Stieltjes integral representation for the effective
diffusivity $\bkappa^*$, involving the molecular diffusivity $\varepsilon$ and a
\emph{spectral measure} of a self-adjoint linear operator. This
approach yields the same integral representation for $\bkappa^*$, both
for steady and time-dependent velocity fields $\vec{u}$. However, the
underlying operator is bounded for steady flows and is unbounded for 
time-dependent flows. This distinction leads to distinct behaviors of
$\bkappa^*$, as a function of $\varepsilon$, in each case. 


\subsubsection{Existence and
  uniqueness}\label{sec:Existence_Uniqueness}
%
Let $\mathscr{H}$ be the Hilbert space of square integrable,
mean-zero, periodic, vector valued functions, with temporal 
period $T$ and spatial periodicity on the $d$-dimensional region
$\mathcal{R}=[-\pi,\pi]^d$, with boundary $\partial\mathcal{R}$    
%
\begin{align}
  \mathscr{H}=\{\vec{Y}(t,\vec{x}):\langle|\vec{Y}|^2\rangle<\infty, \ \langle\vec{Y}\rangle=0
  %\  \vec{Y}(t+T,\vec{x}+\vec{X})=\vec{Y}(t,\vec{x}), \
  %\vec{X}\in\mathcal{R} 
  \}.  
\end{align}
%
Also, consider the associated Hilbert spaces of curl-free
$\mathscr{H}_\times$, divergence-free $\mathscr{H}_\bullet$, and
(spatially)
constant $\mathscr{H}_0$ vector valued functions   
%
\begin{align}
  \mathscr{H}_\times=&\{\vec{Y}\in\mathscr{H}:\vec{\nabla}\times\vec{Y}=0\}, \quad
  \mathscr{H}_\bullet=\{\vec{Y}\in\mathscr{H}:\vec{\nabla}\cdot\vec{Y}=0\},   \quad
  \mathscr{H}_0=\{\vec{Y}\in\mathscr{H}:\vec{Y} \text{ constant}\}. 
%\\
% &\mathscr{H}_0=
%    \{
%    \vec{Y}\in\mathscr{H}:\vec{Y}(t,\vec{x}+\vec{r})=\vec{Y}(t,\vec{x})
%    \ \forall \ \vec{r}\in\mathcal{R}
%    \}.
%    \notag
\end{align}
%
By the Helmholtz theorem
\cite{Denaro:2003:0271,Bhatia:IEE:1077}, these  
subspaces provide an orthogonal decomposition of 
$\mathscr{H}$ \cite{Fannjiang:SIAM_JAM:333}, with
associated orthogonal projectors
\cite{Fannjiang:SIAM_JAM:333,Murphy:CMS:Submitted}
$\Gamma_\times=\vec{\nabla}(\Delta^{-1})\vec{\nabla}\cdot$, $\Gamma_\bullet=-\vec{\nabla}\times(\Delta^{-1})\vec{\nabla}\times$, and
$\Gamma_0$, 
%
\begin{align}
  \mathscr{H}=\mathscr{H}_\times\oplus\mathscr{H}_\bullet\oplus\mathscr{H}_0,\quad
  I=\Gamma_\times+\Gamma_\bullet+\Gamma_0,
\end{align}
%
where $I$ is the identity operator on $\mathbb{R}^d$.
Since we are focusing on flows which are periodic on the spatial
region $\mathcal{R}$, it is convenient to consider the Fourier
representation of such a vector field $\vec{Y}(t,\vec{x})$, 
%
\begin{align}
  \vec{Y}(t,\vec{x})
    =\sum_{\vec{k}\in\mathbb{Z}^d}
       \hat{Y}(t,\vec{k})\e^{\I\vec{k}\cdot\vec{x}},
  \qquad
  \hat{Y}(t,\vec{k})
    =\frac{1}{(2\pi)^d}\int_{\mathcal{R}}
       \hat{Y}(t,\vec{x})\e^{-\I\vec{k}\cdot\vec{x}} \d\vec{x}.
\end{align}
%
%where $|\mathcal{R}|=(2\pi)^d$ is the Lebesgue measure of the region
%$\mathcal{R}$.
The associated action of the above projection opperators on
$\vec{Y}$ is given by \cite{Fannjiang:SIAM_JAM:333} 
%
\begin{align}\label{eq:Projections}
  \Gamma_0\vec{Y}(t,\vec{x})
    =\langle\vec{Y}(t,\vec{x})\rangle_x
    =\hat{Y}(t,0),
    \qquad
  \Gamma_\times\vec{Y}(t,\vec{x})
    =\sum_{\vec{k}\neq0}
       \frac{\vec{k}(\vec{k}\cdot\hat{Y}(t,\vec{k}))}{|\vec{k}|^2}
       \e^{\I\vec{k}\cdot\vec{x}},
    \\
  \Gamma_\bullet\vec{Y}(t,\vec{x})
    =\sum_{\vec{k}\neq0}
       \frac{\vec{k}\times(\vec{k}\times\hat{Y}(t,\vec{k}))}{|\vec{k}|^2}
       \e^{\I\vec{k}\cdot\vec{x}}
    =\sum_{\vec{k}\neq0}\left(I-
       \frac{\vec{k}\vec{k}\cdot}{|\vec{k}|^2}\right) 
       \hat{Y}(t,\vec{k})  \e^{\I\vec{k}\cdot\vec{x}},
       \notag
\end{align}
%
where $\langle\cdot\rangle_x$ denotes spatial averaging over $\mathcal{R}$. From
equation \eqref{eq:Projections} it is clear that $\Gamma_\times+\Gamma_\bullet+\Gamma_0=I$.


Now consider the space $\mathscr{A}$ of functions that are absolutely
continuous on the interval $\mathcal{T}=[0,T)$, that is functions
$f(t)$, $f\in L^2(\mathcal{T})$, such that $f^\prime\in L^2(\mathcal{T}$. 
$\vec{\nabla}\chi(\cdot,\vec{x})\in\mathscr{H}_\times$. Moreover,
$\vec{\nabla}\chi(t,\vec{x})$ is absolutely continuous in $t$, i.e. its time
derivative $[\partial_t\vec{\nabla}\chi](\cdot,\vec{x})\in L^2(\mathcal{R})$ for each $t\in[0,T)$.
WE NEED A DISCUSSION OF WHY THIS IS TRUE: Lax-Milgram shows that there
is a unique solution $\vec{\nabla}\chi(\vec{x})\in\mathscr{H}_\times$ for the
time-independent case since $\|\varepsilon\Ib+\Hb\|$ is uniformly bounded above
and below and
$\langle(\varepsilon\Ib+\Hb)\vec{\xi}\cdot\vec{\xi}\rangle=\varepsilon+\langle\Hb\vec{\xi}\cdot\vec{\xi}\rangle=\varepsilon>0$, hence
coercive. What is the analogous theorem for the time dependent case?



\subsubsection{Integral representation for effective
  diffusivity}\label{sec:Integral_Rep}
%
We now derive a Stieltjes integral representation for $\bkappa^*$
involving the spectral measure of a self-adjoint linear operator
$\Mb$, which follows from the following resolvent formula for
$\vec{\nabla}\chi$  
%
\begin{align}\label{eq:Resolvent}
  \vec{\nabla}\chi=(\varepsilon\Ib-\I\Mb)^{-1}\Gamma\Hb\vec{e}_k, \quad
  \Mb=\I\Gamma[\Hb-\Delta^{-1}\partial_t\Ib]\Gamma.
\end{align}
%
where we have defined $\Gamma=\Gamma_\times$ for notational simplicity. Equation
\eqref{eq:Resolvent} follows from applying the integro-differential
operator $\vec{\nabla}\Delta^{-1}$ to $\vec{\nabla}\cdot\bsig\vec{E}_k=0$ in equation 
\eqref{eq:Maxwells_Equations}, with $\vec{E}_k=\vec{\nabla}\chi+\vec{e}_k$ and
$\bsig=((\varepsilon-\Delta^{-1}\partial_t)\Ib+\Hb)$, yielding
$\Gamma(\varepsilon\Ib+[\Hb-\Delta^{-1}\partial_t\Ib])\vec{\nabla}\chi=-\Gamma\Hb\vec{e}_k$. Equation
\eqref{eq:Resolvent} now follows by writing $-\I^2=1$ and noting 
$\vec{\nabla}\chi\in\mathscr{H}_\times$ implies that $\Gamma\vec{\nabla}\chi=\vec{\nabla}\chi$. In this paper
we will consider only velocity fields $\vec{u}$ such that $\Hb$ is
uniformly bounded, $\|\Hb\|<\infty$, in the operator norm on
$\mathscr{H}_\times$. Therefore, since $\Hb$ is skew-symmetric and $\Gamma$ is a
projection operator which is bounded by one in the operator norm,
$\I\Gamma\Hb\Gamma$ is a self-adjoint linear operator on $\mathscr{H}_\times$ which
is bounded by $\|\Hb\|$. Assuming the commutability condition $\Delta^{-1}\partial_t=\partial_t\Delta^{-1}$ The symmetric operator $\I\Gamma\Delta^{-1}\partial_t\Gamma$






In two dimensions, $d=2$, the matrix $\Hb$ is determined by a stream
function $H(t,\vec{x})$ 
%
\begin{align}\label{eq:u_H}  
  \Hb=\left[
  \begin{array}{cc}
    0  & H\\
    -H & 0
  \end{array}
  \right],
  \qquad
  \vec{u}=[\partial_{x_1}H, \ \partial_{x_2}H].
\end{align}
%

\newpage





\medskip

{\bf Acknowledgements.}
We gratefully acknowledge support from the Division of Mathematical
Sciences and the Office of Polar Programs at the U.S. 
National Science Foundation (NSF) through Grants
DMS-1009704, ARC-0934721, and DMS-0940249. We are also grateful for 
support from the Office of Naval Research (ONR) through
Grants N00014-13-10291 and N00014-12-10861. Finally, we would like to 
thank the NSF Math Climate Research Network (MCRN) for their support
of this work. 


\medskip

\bibliographystyle{plain}
\bibliography{murphy}
\end{document}

% LocalWords:  McMaster RM jk sig eps def Maxwells Milgram coercivity diag jX
% LocalWords:  mh Cond mu kk PtI Fs Es Hashin Shtrikman extremized Decomp chi
% LocalWords:  Acknowledgements DMS ONR MCRN murphy
