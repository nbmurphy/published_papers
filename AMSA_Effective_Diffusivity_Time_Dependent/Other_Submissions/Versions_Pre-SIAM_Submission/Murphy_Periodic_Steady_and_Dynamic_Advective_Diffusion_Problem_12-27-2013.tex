\documentclass[11pt]{amsart}
%\usepackage{latexsym, amssymb, enumerate, amsmath}
\usepackage{graphicx,amssymb,amsmath,amsfonts,mathrsfs}

\setlength{\textwidth}{6.5in}
\setlength{\textheight}{9.0in}
\setlength{\oddsidemargin}{0in}
\setlength{\evensidemargin}{0in}
\setlength{\topmargin}{-0.5in}

\renewcommand{\topfraction}{0.85}
\renewcommand{\textfraction}{0.1}
\renewcommand{\floatpagefraction}{0.55}%0.75


\newcommand{\ph}{\hat{\phi}}
\newcommand{\pt}{\tilde{\phi}} 
\newcommand{\pc}{\check{\phi}}
\newcommand{\gh}{\hat{\gamma}}
\newcommand{\Dh}{\hat{\Delta}}
\newcommand{\dha}{\hat{\delta}}
\newcommand{\qh}{\hat{q}}
\newcommand{\xh}{\hat{x}} 
\newcommand{\HM}{\mathcal{H}_{\text{max}}}
\newcommand{\Hm}{\mathcal{H}_{\text{min}}}
\newcommand{\sech}{\rm \hspace{0.7mm}sech}
\newcommand{\I}{\mathrm{i}}
\newcommand{\e}{\mathrm{e}}
\renewcommand{\d}{\mathrm{d}}
\newcommand{\hh}{\hat{h}}
\newcommand{\mh}{m_r}
\newcommand{\mt}{m_i}

\newcommand{\Mb}{\mathbf{M}}
\newcommand{\Xb}{\mathbf{X}}
\newcommand{\Tb}{\mathbf{T}}
\newcommand{\Hb}{\mathbf{H}}
\newcommand{\Kb}{\mathbf{K}}
\newcommand{\Jb}{\mathbf{J}}
\newcommand{\Ib}{\mathbf{I}}
\newcommand{\Sb}{\mathbf{S}}
\newcommand{\Rb}{\mathbf{R}}
\newcommand{\Ab}{\mathbf{A}}
\newcommand{\Eb}{\mathbf{E}}
\newcommand{\Qb}{\mathbf{Q}}

\newcommand{\Kc}{\mathcal{K}}
\newcommand\Kbc{\mbox{\boldmath${\mathcal{K}}$}}

\newcommand{\Tc}{\mathcal{T}}
\newcommand{\Vc}{\mathcal{V}}

\newcommand{\Hs}{\mathscr{H}}
\newcommand{\As}{\mathscr{A}}
\newcommand{\Ds}{\mathscr{D}}


\newcommand\bsig{\mbox{\boldmath${\sigma}$}}
\newcommand\beps{\mbox{\boldmath${\epsilon}$}}
\newcommand\bxi{\mbox{\boldmath${\xi}$}}
\newcommand\bmu{\mbox{\boldmath${\mu}$}}
\newcommand\balpha{\mbox{\boldmath${\alpha}$}}
\newcommand\brho{\mbox{\boldmath${\rho}$}}
\newcommand\bDelta{\mbox{\boldmath${\Delta}$}}
\newcommand\bkappa{\mbox{\boldmath${\kappa}$}}
\newcommand\bGamma{\mbox{\boldmath${\Gamma}$}}


\newtheorem{thm}{Theorem}[section]
\newtheorem{prop}[thm]{Proposition}
\newtheorem{lem}[thm]{Lemma}
\newtheorem{cor}[thm]{Corollary}

    %\theoremstyle{definition}

\newtheorem{defn}[thm]{Definition}
\newtheorem{notation}[thm]{Notation}
\newtheorem{example}[thm]{Example}
\newtheorem{conj}[thm]{Conjecture}
\newtheorem{prob}[thm]{Problem}

    %\theoremstyle{remark}

\newtheorem{rem}[thm]{Remark}
    % Use the standard latex environments for theorems, etc. Here is one
          % possible method of declaring them: It numbers all results by the
          % section, and uses a common numbering system for the different
          % environmentts.

\begin{document}

\title{Spectral theory of advective diffusion \\
  by dynamic and steady periodic flows}


% AUTHORS 
%\author{N. B. Murphy, A. Gully, E. Cherkaev, and K. M. Golden}
\author{N. B. Murphy$^\ast$}
\address{$^*$Department of Mathematics, 340 Rowland Hall, University of
  California at Irvine, Irvine, CA 92697-3875, USA}
\email{nbmurphy@math.uci.edu}

\author{J. Xin$^\dag$}
\address{$^{\dag}$Department of Mathematics, 340 Rowland Hall, University of
  California at Irvine, Irvine, CA 92697-3875, USA} 
\email{jxin@math.uci.edu}

\author{J. Zhu$^\star$}
\address{$^\star$University of Utah, Department of Mathematics, 155 S 1400 E
  RM 233, Salt Lake City, UT 84112-009, USA}
\email{zhu@math.utah.edu}

\author{E. Cherkaev$^\ddagger$}
\address{$^\ddagger$University of Utah, Department of Mathematics, 155 S 1400 E
  RM 233, Salt Lake City, UT 84112-009, USA} 
\email{elena@math.utah.edu}

\maketitle
\vspace{-3ex}
\begin{center}
  Department of Mathematics, University of California at Irvine
\end{center}

%\vspace{3ex}


\begin{abstract}
%
The analytic continuation method for representing transport in
composites provides integral representations for the
effective coefficients of two-phase random media. Here we adapt this 
method to characterize the effective thermal transport properties of
advective diffusion, by steady and time-dependent, periodic flows. Our
novel approach yields an integral representation for the
effective diffusivity, which holds for dynamic and steady,
incompressible flows, involving the spectral measure of a self-adjoint
linear operator. In the case of steady fluid velocity fields, the
spectral measure is associated with a Hermitian Hilbert-Schmidt
integral operator, and in the case of dynamic flows, it is associated
with an unbounded integro-differential operator. We utilize the
integral representation to obtain asymptotic behavior of the effective 
diffusivity, as the molecular diffusivity tends to zero, for model,
steady and dynamic flows. Our analytical results are supported by
numerical computations of the spectral measures and effective
diffusivities.    
%
\end{abstract}

\section{Introduction}\label{sec:Introduction}
%
The long time, large scale behavior of a diffusing particle   
or tracer being advected by an incompressible velocity field 
is equivalent to an enhanced diffusive process \cite{Taylor:PRSL:196} 
with an effective diffusivity tensor $\bkappa^*$.
Determining the effective transport properties of advection enhanced
diffusion is a challenging problem with theoretical and practical 
importance in many fields of science and engineering,
ranging from turbulent combustion to mass, heat, and salt transport in
geophysical flows \cite{Moffatt:RPP:621}. A broad range of
mathematical techniques have been developed that reduce the analysis
of complex fluid flows, with rapidly varying structures in space and
time, to solving averaged or \textit{homogenized} equations that do
not have rapidly varying data, and involve an effective parameter.




Homogenization of the advection-diffusion equation for thermal
transport by time-independent, random fluid velocity fields was
treated in \cite{McLaughlin:SIAM_JAM:780}. This 
reduced the analysis of turbulent diffusion to solving an anisotropic
diffusion equation involving a homogenized temperature and an
effective diffusivity tensor $\bkappa^*$. An important consequence of
this analysis is that $\bkappa^*$ is given in terms  
of a \emph{curl-free} stationary stochastic process which satisfies a
steady state diffusion equation involving a skew-symmetric random
matrix $\Hb$ \cite{Avellaneda:CMP-339,Avellaneda:PRL-753}. By adapting
the analytic continuation method (ACM) of homogenization theory for
composites \cite{Golden:CMP-473}, it was shown that the result in
\cite{McLaughlin:SIAM_JAM:780} leads to an integral
representation of $\bkappa^*$, involving a spectral measure of a
self-adjoint random operator
\cite{Avellaneda:CMP-339,Avellaneda:PRL-753}. This integral
representation of $\bkappa^*$ was generalized to the time-dependent
case in \cite{Avellaneda:PRE:3249,Biferale:PF:2725}. Remarkably, this 
method has also been extended to flows with incompressible
\emph{nonzero} effective drift 
\cite{Pavliotis:PHD_Thesis}, flows where particles diffuse according
to linear collisions \cite{Pavliotis:IMAJAM:951}, and solute transport
in porous media \cite{Bhattacharya:AAP:1999:951}. All these approaches
yield  integral representations of the symmetric and, when
appropriate, the skew-symmetric part of $\bkappa^*$.




Homogenization of the advection-diffusion equation for periodic or
cellular, incompressible flow fields was treated in
\cite{Fannjiang:SIAM_JAM:333,Fannjiang:1997:1033}. As in the case of
random flows, the effective diffusivity tensor
$\bkappa^*$ is given in terms of a \emph{curl-free} vector field, which
satisfies a diffusion equation involving a skew-symmetric
matrix $\Hb$. Here, we demonstrate that the ACM can
be adapted to this periodic setting to provide an 
integral representation for $\bkappa^*$, both for steady and
time-dependent flows, involving a self-adjoint linear operator and the 
(non-dimensional) molecular diffusivity $\varepsilon$. In the case of steady
fluid velocity fields, the spectral measure is associated with a
Hermitian Hilbert-Schmidt integral operator involving the Green's
function of the Laplacian on a rectangle. While in the case of dynamic
flows, the spectral measure is associated with a Hermitian operator
which is the sum of that for steady flows and an unbounded
integro-differential operator.     
 

We utilize the analytic structure of the  integral
representation for $\bkappa^*$ to obtain its asymptotic behavior for
model flows, as the molecular diffusivity $\varepsilon$ tends to zero. This is
the high P\'{e}clet number regime that is important for the
understanding of transport processes in real fluid flows, where the
molecular diffusivity is often quite small in comparison. In
particular, FINISH THIS PARAGRAPH WHEN WE HAVE CONCRETE RESULTS.
necessary and sufficient conditions for steady periodic flow
fields $\kappa^*\sim\epsilon^{1/2}$, generically, for steady flows and $\kappa^*\sim O(1)$ for
``chaotic'' time-dependent flows. 

%
\section{Mathematical Methods}\label{sec:Mathematical_Methods} 
%
In this section, we formulate the effective parameter problem for
enhanced diffusive transport by advective, periodic flows, and provide
an integral representation for the effective diffusivity tensor
$\bkappa^*$, which holds for both steady and dynamic flows. The
effective parameter problem 
\cite{Fannjiang:SIAM_JAM:333} for such transport processes  is
reviewed in Section \ref{sec:Eff_Trans}. Parallels existing between this
problem of homogenization theory \cite{Bensoussan:Book:1978} and the
ACM for representing transport in composites are put into
correspondence in Section \ref{sec:ACM}. In particular, an abstract
Hilbert space framework is provided in Section \ref{sec:Hilbert_Space}
which places these different effective parameter problems on equal
mathematical footing. Within this Hilbert space setting, we derive in
Section \ref{sec:Integral_Rep} an integral representation for
$\bkappa^*$, involving the molecular diffusivity $\varepsilon$ and a
\emph{spectral measure} of a self-adjoint linear operator. This
integral representation is employed in Section \ref{sec:Assymptotics}
to obtain asymptotic behavior of $\bkappa^*$ in the scaling regime where
$\varepsilon\ll1$. 



\subsection{Effective transport by
  advective-diffusion} \label{sec:Eff_Trans}  
%
Consider the advection enhanced diffusive transport of a passive
tracer $\phi(t,\vec{x}\,)$, $t>0$, $\vec{x}\in\mathbb{R}^d$, as described by
the advection-diffusion equation 
%
\begin{align}\label{eq:ADE}
  \partial_t\phi+\vec{\nabla}\cdot(\vec{v}\phi)=\kappa_0\Delta \phi, \quad
  \phi(0,\vec{x})=\phi_0(\vec{x}),
  %\qquad
  %\vec{\nabla}\cdot\vec{v}=0
\end{align}
% 
with $\phi_0(\vec{x})$ given. Here, $\partial_t$ denotes partial differentiation
with respect to time $t$, $\Delta=\vec{\nabla}\cdot\vec{\nabla}=\nabla^2$ is the Laplacian,
$\kappa_0>0$ is the molecular diffusivity, and $\vec{v}=\vec{v}(t,\vec{x})$
is the fluid velocity field, which is assumed to be incompressible and
mean-zero, 
%
\begin{align}\label{eq:incompressible}
  \vec{\nabla}\cdot\vec{v}=0, \quad
  \langle\vec{v}\,\rangle=0.
\end{align}
%
Here, $t\in\Tc$, $\vec{x}\in\Vc$, and the bounded sets $\Tc\subset\mathbb{R}$ and
$\Vc\subset\mathbb{R}^d$ define the space-time period cell ($(d+1)$--torus)
$\Tc\otimes\Vc$. In equation \eqref{eq:incompressible}, $\langle\cdot\rangle$ denotes
spatial averaging over $\Vc$ in the case of a time-independent
velocity field $\vec{v}=\vec{v}(\vec{x})$, and when the velocity field
is time-dependent, $\vec{v}=\vec{v}(t,\vec{x})$, $\langle\cdot\rangle$ denotes
space-time averaging over $\Tc\otimes\Vc$.



We non-dimensionalize equation
\eqref{eq:ADE} as follows. Let $\ell$ and $\tilde{t}$ be typical length and time 
scales associated with the problem of interest. Mapping to the
non-dimensional variables $t\mapsto t/\tilde{t}$ and $x_i\mapsto x_i/\ell$, one finds that
$\phi$ satisfies the advection-diffusion 
equation in \eqref{eq:ADE} with a non-dimensional molecular
diffusivity $\varepsilon=\tilde{t}\kappa_0/\ell^{\,2}$ and velocity field
$\vec{u}=\tilde{t}\,\vec{v}/\ell$, where $x_i$ is the $i^{\,\text{th}}$
component of the vector $\vec{x}$.




We are interested in the dynamics of $\phi$ in \eqref{eq:ADE} for
\emph{large} length and time scales, and when the initial density
$\phi_0$ is slowly varying relative to the velocity field $\vec{u}$, so
that 
%
\begin{align}\label{eq:IC}
  \phi(0,\vec{x})=\phi_0(\delta\vec{x}),
\end{align}
%
with $\delta\ll1$. Anticipating that
$\phi$ will have diffusive dynamics, we re-scale space and time by
$\vec{x}\mapsto\vec{y}=\vec{x}/\delta$ and $t\mapsto\tau=t/\delta^{\,2}$, respectively. This
variable change, along with equations 
\eqref{eq:incompressible} and \eqref{eq:IC}, transforms equation
\eqref{eq:ADE} into \cite{McLaughlin:SIAM_JAM:780}  
%
\begin{align}\label{eq:ADE_delta}
  \partial_t\phi^\delta(t,\vec{x})+\delta^{\,-1}\,\vec{u}(\tau,\vec{y})\cdot\vec{\nabla}\phi^\delta(t,\vec{x})
      =\varepsilon\Delta\phi^\delta(t,\vec{x}), \quad
  \phi^\delta(0,\vec{x})=\phi_0(\vec{x}).  
\end{align}
%
The effective behavior of $\phi^\delta$ as $\delta\to0$ follows by a multiple scale
method
\cite{McLaughlin:SIAM_JAM:780,Papanicolaou:1981:36:8,Papanicolaou:RF-835,Bensoussan:Book:1978},
which involves expanding $\phi^\delta$ in powers of $\delta$
\cite{McLaughlin:SIAM_JAM:780}. This leads to a sequence of problems
and demonstrates, for periodic diffusivity coefficients in
\eqref{eq:kappa_coeff} which are uniformly elliptic but not
necessarily symmetric \cite{Fannjiang:SIAM_JAM:333}, that
$\phi^\delta(t,\vec{x})$ converges to $\bar{\phi}(t,\vec{x})$, which satisfies a
diffusion equation involving a (constant) effective diffusivity tensor
$\Kbc^*$ 
%
\begin{align}\label{eq:phi_bar}
  \partial_t\bar{\phi}=\vec{\nabla}\cdot\Kbc^*\vec{\nabla}\bar{\phi}, \quad
  \bar{\phi}(0,\vec{x})=\phi_0(\vec{x}).
\end{align}
%
The effective diffusivity tensor $\bkappa^*$ is obtained by solving
the cell problem \cite{Fannjiang:SIAM_JAM:333}
% 
\begin{align}\label{eq:Cell_Problem}
  \partial_t\chi_j=\vec{\nabla}\cdot\bkappa(\vec{\nabla}\chi_j+\vec{e}_j\,), \quad
    %=\vec{\nabla}\cdot[(\varepsilon\Ib+\Hb)(\vec{\nabla}\chi_j+\vec{e}_j\,)], \quad
  \langle\vec{\nabla}\chi_j\rangle=0,
%  \quad   j=1,\ldots,d,
\end{align}
%
for each standard basis vector $\vec{e}_j$, $j=1,\ldots,d$, where
$\chi_j=\chi_j(t,\vec{x}\,;\vec{e}_j)$ Equation
\eqref{eq:Cell_Problem} also holds \cite{Fannjiang:SIAM_JAM:333} when
the velocity field is time-independent $\vec{u}=\vec{u}(\vec{x})$,
however, in this case $\partial_t\chi_j=0$.  The components $\kappa^*_{jk}=\bkappa^*\vec{e}_j\cdot\vec{e}_k$,
$j,k=1,\ldots,d$, of the effective diffusivity tensor are given in terms of
the \emph{curl-free} vector field $\vec{\nabla}\chi_j$
\cite{Fannjiang:SIAM_JAM:333}   
% 
\begin{align}\label{eq:Eff_Diffusivity}  
   \kappa^*_{jk}=\varepsilon\langle(\vec{\nabla}\chi_j+\vec{e}_j\,)\cdot(\vec{\nabla}\chi_k+\vec{e}_k\,)\rangle
       =\varepsilon(\delta_{jk}+\langle\vec{\nabla}\chi_j\cdot\vec{\nabla}\chi_k\rangle).
\end{align}
%.
Equation \eqref{eq:Eff_Diffusivity} demonstrates that the effective
transport of the tracer $\phi$ in the principle directions $\vec{e}_k$,
$k=1,\ldots,d$, is always enhanced by the presence of an incompressible
velocity field, $\kappa^*_{kk}\geq\varepsilon$. The details of equations
\eqref{eq:phi_bar}--\eqref{eq:Eff_Diffusivity} are similar to that
given in \cite{McLaughlin:SIAM_JAM:780,Fannjiang:SIAM_JAM:333} and are
provided in Section \ref{sec:Multiscal_Method}. 



Since by \eqref{eq:incompressible} $\vec{u}(t,\vec{x})$ is
incompressible, $\vec{\nabla}\cdot\vec{u}=0$, there is a (non-dimensional)
skew-symmetric matrix $\Hb(t,\vec{x})$ such that
%$\Hb^{\,T}=-\Hb$, such that $\vec{u}=\vec{\nabla}\cdot\Hb$. 
% %
\begin{align}
 \vec{u}=\vec{\nabla}\cdot\Hb, \qquad  \Hb^{\,T}=-\Hb.
\end{align}
% %
Using this representation of the velocity field $\vec{u}$, equation
\eqref{eq:ADE} can be written as a diffusion equation, 
%
\begin{align}\label{eq:ADE_Divergence}
  \partial_t\phi%&=\varepsilon\Delta \phi+\vec{u}\cdot\vec{\nabla}\phi\\
    %&=\varepsilon\vec{\nabla}\cdot\vec{\nabla}\phi+(\vec{\nabla}\cdot\Hb)\cdot\vec{\nabla}\phi\\
    %&=\vec{\nabla}\cdot[\varepsilon I+\Hb]\vec{\nabla}\phi\\
    %&=\vec{\nabla}\cdot\bkappa\vec{\nabla}\phi
    =\vec{\nabla}\cdot\bkappa\vec{\nabla}\phi, \quad
    %\bkappa=\varepsilon I+\Hb,
    \phi(0,\vec{x})=\phi_0(\vec{x}),
    \qquad
    \bkappa=\varepsilon\Ib+\Hb,
\end{align}
%
where $\bkappa(t,\vec{x})=\varepsilon\Ib+\Hb(t,\vec{x})$ can be viewed as a local
diffusivity tensor with coefficients
%
\begin{align}\label{eq:kappa_coeff}
  \kappa_{jk}=\varepsilon\delta_{jk}+H_{jk},\quad j,k=1,\cdots,d.
\end{align}
%
Here, $\delta_{jk}$ is the Kronecker delta and, for notational simplicity,
we denote $\Ib$ the identity operator on all linear spaces in
question.   








% For notational simplicity, we will focus
% on one diagonal component $\kappa^*_{kk}$ of the tensor $\bkappa^*$, for
% some $k=1,\ldots,d$ and set $\vec{\nabla}\chi=\vec{\nabla}\chi_k$, reintroducing the
% notation $\vec{\nabla}\chi_k$ where necessary.
In Section \ref{sec:ACM},  we
recast equations \eqref{eq:Cell_Problem} and
\eqref{eq:Eff_Diffusivity} into a form which parallels the problem of
characterizing effective transport in composite media
\cite{Golden:CMP-473}. 




%Integral representation of the effective diffusivity for steady and
%dynamic flows 
\subsection{The ACM for advection enhanced diffusion by periodic
  flows} \label{sec:ACM}   
%
The ACM for representing transport in composites gives a Hilbert space
formulation of the effective parameter problem and provides an
integral representation for the effective transport coefficients of
composite media, involving a \emph{spectral measure} of a self-adjoint
linear operator which depends only on the composite geometry
\cite{Golden:CMP-473,Murphy:JMP:063506,MILTON:2002:TC}. Here 
we establish a correspondence between this effective parameter problem
and that for enhanced diffusive transport by advective velocity
fields. In Section \ref{sec:Hilbert_Space}, we formulate the Hilbert
space framework associated with advective diffusion. In
Section \ref{sec:Integral_Rep} we utalize this mathematical framework
to obtain an integral representation for the effective diffusivity
tensor $\bkappa^*$, involving a spectral measure which depends only on
the fluid velocity field $\vec{u}$.  




Toward this goal, we write the first formula in equation
\eqref{eq:Cell_Problem} in a more suggestive, divergence
form. Under the commutability condition
$\vec{\nabla}\Delta^{-1}\partial_t=\partial_t\Delta^{-1}\vec{\nabla}$ (see Section
\ref{sec:Appendix}), we write
\cite{Fannjiang:SIAM_JAM:333}  
$\partial_t\chi_k=\Delta\Delta^{-1}\partial_t\chi_k=\vec{\nabla}\cdot(\Delta^{-1}\partial_t\Ib)\vec{\nabla}\chi_k$ and define
$\vec{E}_k=\vec{\nabla}\chi_k+\vec{e}_k$ and $\bsig=\bkappa-\Delta^{-1}\partial_t\Ib
% =((\varepsilon-\Delta^{-1}\partial_t)\Ib+\Hb)
$. Here, the inverse operation $\Delta^{-1}$ is based on convolution with
the Green's function for the Laplacian $\Delta$ and $\bsig=\bkappa$ in the
case of steady fluid velocity fields, $\vec{u}=\vec{u}(\vec{x})$. 
With these definitions, equation \eqref{eq:Cell_Problem} may be
written as $\vec{\nabla}\cdot\bsig\vec{E}_k=0$, $\langle\vec{E}_k\rangle=\vec{e}_k$, which
is equivalent to  
%
\begin{align}\label{eq:Maxwells_Equations}    
  \vec{\nabla}\cdot\vec{J}_k=0, \quad
  \vec{\nabla}\times\vec{E}_k=0, \quad
  \vec{J}_k=\bsig\vec{E}_k,\quad
  \langle\vec{E}_k\rangle=\vec{e}_k,\qquad
  \bsig%=\bkappa-\Delta^{-1}\partial_t\Ib
       =((\varepsilon-\Delta^{-1}\partial_t)\Ib+\Hb).
\end{align}
%
The formulas in
\eqref{eq:Maxwells_Equations} are  
precisely the electrostatic version of Maxwell's equations for a
conductive medium \cite{Golden:CMP-473}, where $\vec{E}_k$ and
$\vec{J}_k$ are the local electric field and current density,
respectively, and $\bsig$ is the local conductivity tensor of the
medium. In the ACM for composites, the effective conductivity tensor
$\bsig^*$ is defined as
% 
\begin{align}\label{eq:sigma*}
  \langle\vec{J}_k\rangle=\bsig^*\langle\vec{E}_k\rangle.
\end{align}
%
The linear constitutive relation $\vec{J}_k=\bsig\vec{E}_k$ in
\eqref{eq:Maxwells_Equations} relates the local intensity and flux,
while the linear relation in \eqref{eq:sigma*} relates the mean
intensity and mean flux. We demonstrate in Section
\ref{sec:Hilbert_Space} that, for 
\emph{time-independent} velocity fields $\vec{u}$, the definition of
the effective parameter given in equation \eqref{eq:sigma*} reduces to
\eqref{eq:Eff_Diffusivity} for diagonal components. This follows by
adapting the Hilbert space formulation of the ACM for composites, to
treat the effective transport properties of advective
diffusion by incompressible velocity fields. This Hilbert
space framework is developed in Section \ref{sec:Hilbert_Space}.      




\subsubsection{Hilbert space formulation of the effective parameter
  problem} \label{sec:Hilbert_Space} 
%
In this section we utalize equation \eqref{eq:Eff_Diffusivity}, which
expresses the effective diffusivity tensor $\bkappa^*$ in terms of a
\emph{curl-free} vector field, to provide an abstract Hilbert
space formulation of the effective parameter problem for advective
diffusion. Consider the Hilbert spaces $\Hs_{\,\Tc}$ and $\Hs_{\,\Vc}$
(over the complex field 
$\mathbb{C}$) of periodic, square integrable, vector valued functions
with temporal periodicity $T$ on the interval $\Tc=(0,T)$ and spatial
periodicities $V_i$, $i=1,\ldots,d$, on the $d$-dimensional region 
$\Vc=(0,V_1)\times\cdots\times(0,V_d)$, respectively,  
%
\begin{align}
  \Hs_{\,\Tc}=\{ 
     \vec{\xi}\in\otimes_{i=1}^dL^2(\Tc):
     \vec{\xi}(0)=\vec{\xi}(T) 
                        \}, \qquad
  \Hs_{\,\Vc}=\{ 
     \vec{\xi}\in\otimes_{i=1}^dL^2(\Vc):
     \vec{\xi}(0)=\vec{\xi}(\vec{V}) 
                        \}. \notag
\end{align}
%
Here, we have defined $\vec{V}=(V_1,\ldots,V_d)$ and for notational
convenience we denote by $0$ the null element of all linear spaces in
question. By the Helmholtz theorem \cite{Denaro:2003:0271,Bhatia:IEE:1077},
the Hilbert space $\Hs_{\,\Vc}$ can be decomposed into orthogonal
subspaces of curl-free $\Hs_\times$, divergence-free $\Hs_\bullet$, and constant
$\Hs_{\,0}$ vector fields, with associated orthogonal projectors
$\bGamma_\times=\vec{\nabla}(\Delta^{-1})\vec{\nabla}\cdot$,
$\bGamma_\bullet=-\vec{\nabla}\times(\Delta^{-1})\vec{\nabla}\times$, and 
$\bGamma_0$
%, respectively,
\cite{Fannjiang:SIAM_JAM:333,MILTON:2002:TC}    
%
\begin{align}\label{eq:Helmholtz}
  &\Hs_{\,\Vc}=\Hs_\times\oplus\Hs_\bullet\oplus\Hs_{\,0},\qquad
  \Ib=\bGamma_\times+\bGamma_\bullet+\bGamma_0, \\  
  \Hs_\times=\{\vec{\xi}:\vec{\nabla}\times\vec{\xi}&=0 \text{ weakly}\}, \quad
  \Hs_\bullet
      =\{\vec{\xi}:\vec{\nabla}\cdot\vec{\xi}=0 \text{ weakly}\},   \quad
  \Hs_{\,0}
      =\{\vec{\xi}:\vec{\xi}=\langle\vec{\xi}\,\rangle\}.
     \notag  
\end{align}
%



%It is important to note that, from equation \eqref{eq:Cell_Problem},
%we have $|\partial_t\chi_k|<\infty$ for all $(t,\vec{x})\in\Tc\otimes\Vc$.
From equations \eqref{eq:Cell_Problem} and \eqref{eq:Eff_Diffusivity},
and the condition $0\leq\kappa^*_{jk}<\infty$, we have that $\vec{\nabla}\chi_k$ is (weakly)
curl-free, mean-zero, and
$\vec{\nabla}\chi_k\in\otimes_{i=1}^dL^2(\Tc\otimes\Vc)$. Hence,
it is natural for our analysis of the effective parameter problem for
$\bkappa^*$ in \eqref{eq:Eff_Diffusivity}, and that of the associated
integral representation discussed in Section \ref{sec:Integral_Rep},
to be conducted on the Hilbert space
%$\Hs$ of mean-zero vector valued functions   
%
\begin{align}\label{eq:Hilbert_Space}
  \Hs=\{\vec{\xi}\in\Hs_{\,\Tc}\otimes\Hs_\times: \langle\vec{\xi}\,\rangle=0\},
\end{align}
%
with \emph{sesquilinear} inner-product $\langle\cdot,\cdot\rangle$ defined by
$\langle\vec{\xi},\vec{\zeta}\,\rangle=\langle\vec{\xi}\cdot\vec{\zeta}\,\rangle$. 
Here, we have used the simplified notation $\langle\vec{\xi}\,\rangle=0\iff\langle\xi_i\rangle=0$ for
all $i=1,\ldots,d$. We henceforth assume that $\vec{\nabla}\chi_k\in\mathscr{H}$ for
time-dependent velocity fields $\vec{u}$, and for time-independent
$\vec{u}$ we set $\mathscr{H}_{\Tc}=\emptyset$ so that $\vec{\xi}\in\mathscr{H}$
implies that $\vec{\xi}\in\mathscr{H}_\times$ with $\langle\vec{\xi}\,\rangle=0$. We also
assume that the flow matrix $\Hb(t,\vec{x})$ and its (component-wise)
time derivative $\partial_t\Hb(t,\vec{x})$ are bounded in the operator norm
$\|\cdot\|$ induced by the inner-product $\langle\cdot,\cdot\rangle$
\cite{Reed-1980,Stone:64,Stakgold:BVP:2000}   
%
\begin{align}\label{eq:Bounded_H}
  \|\Hb\|<\infty, \quad \|\partial_t\Hb\|<\infty, \text{ on all of } \Hs.
\end{align}
%









Before we discuss how the Hilbert space framework presented above
leads to an  integral representation for $\bkappa^*$, we first
discuss some key differences in the theory between the cases of steady
and dynamic velocity fields $\vec{u}$. These differences are reflected
in the measure underlying this integral representation for $\bkappa^*$
and stem from the \emph{unboundedness} of the operator $\partial_t$ on the
Hilbert space $\Hs_{\,\Tc}$ \cite{Reed-1980,Stone:64}. For steady
$\vec{u}$, in general, equation \eqref{eq:sigma*} reduces to
\eqref{eq:Eff_Diffusivity} for  diagonal components of the effective
parameter.  However, for dynamic $\vec{u}$, this is not true in
general. The details are as follows. For dynamic $\vec{u}$, the
operator $\bsig$ in \eqref{eq:Maxwells_Equations} can be written as
$\bsig=\varepsilon\Ib+\Sb$, where  $\Sb=\Hb-\Delta^{-1}\partial_t\Ib$ is skew-symmetric 
$\langle\Sb\vec{\xi},\vec{\zeta}\,\rangle=-\langle\vec{\xi},\Sb\vec{\zeta}\,\rangle$ for all
$\vec{\xi},\vec{\zeta}\in\Hs$ such that
$|\langle\partial_t\vec{\xi},\vec{\zeta}\,\rangle|,|\langle\vec{\xi},\partial_t\vec{\zeta}\,\rangle|<\infty$ (see Section
\ref{sec:Appendix} for details).  
This property of the operator $\Sb$ implies that
%$\langle\Sb\vec{E}_k\cdot\vec{E}_k\rangle=-\langle\Sb\vec{E}_k\cdot\vec{E}_k\rangle=0$.
%
\begin{align}\label{eq:Sb_Skew}
  \langle\Sb\vec{\xi}\cdot\vec{\xi}\,\rangle=-\langle\Sb\vec{\xi}\cdot\vec{\xi}\,\rangle=0,
  \qquad
  \Sb=\Hb-\Delta^{-1}\partial_t\Ib,
\end{align}
%
for all such $\vec{\xi}\in\Hs$. In this dynamic setting, equation
\eqref{eq:Sb_Skew} does not hold for every $\vec{\xi}\in\Hs$, as the
unbounded operator $\partial_t$ is defined only on a proper (dense) subset of
the Hilbert space $\Hs_{\,\Tc}$ \cite{Reed-1980}, and it may be that
$|\langle\partial_t\vec{\xi},\vec{\xi}\,\rangle|=\infty$. In the case of a steady velocity field
we have $\Sb\equiv\Hb$ and, by equation \eqref{eq:Bounded_H} and the Cauchy
Schwartz inequality, $|\langle\Sb\vec{\xi},\vec{\xi}\,\rangle|\leq\|\Hb\|\|\vec{\xi}\,\|^2<\infty$ for
all $\vec{\xi}\in\Hs$, so equation \eqref{eq:Sb_Skew} holds for all
$\vec{\xi}\in\Hs$.   





We now use \eqref{eq:Sb_Skew} to show that, in the case of steady
flows, equation \eqref{eq:sigma*} reduces to
\eqref{eq:Eff_Diffusivity} for diagonal components of the effective
parameter. In this steady case $\bsig\equiv\bkappa=\varepsilon\Ib+\Hb$ which, by
equation \eqref{eq:Bounded_H}, is bounded in operator norm for all
$\varepsilon<\infty$. Consequently, $\vec{J}_k=\bsig\vec{E_k}$ in 
\eqref{eq:Maxwells_Equations} satisfies $\vec{J}_k\in\mathscr{H}_\bullet$. By
the Helmholtz theorem displayed in \eqref{eq:Helmholtz} and
$\vec{\nabla}\chi_k\in\mathscr{H}_\times$, we have $\langle\vec{J}_k\cdot\vec{\nabla}\chi_k\rangle=0$, where 
$\langle\cdot\rangle$ denotes spatial averaging over $\Vc$ in this time-independent
setting. Therefore, by this and equations \eqref{eq:sigma*} and 
\eqref{eq:Sb_Skew}, we have
%that \eqref{eq:sigma*} reduces to \eqref{eq:Eff_Diffusivity} for the
%diagonal components of the effective parameter
\cite{Fannjiang:SIAM_JAM:333} 
%
\begin{align}\label{eq:Reduction}
  \sigma^*_{kk}=\bsig^*\vec{e}_k\cdot\vec{e}_k 
       =\langle\bsig\vec{E}_k\cdot\vec{e}_k\rangle
       =\langle\bsig\vec{E}_k\cdot\vec{E}_k\rangle
       =\langle[(\varepsilon\Ib+\Sb)]\vec{E}_k\cdot\vec{E}_k\rangle
       =\varepsilon\langle\vec{E}_k\cdot\vec{E}_k\rangle.
       %=\kappa^*_{kk}.
\end{align}
%
This demonstrates that, for steady flows $\vec{u}$, the effective
parameter problem for $\bkappa^*$ in \eqref{eq:Cell_Problem} and
\eqref{eq:Eff_Diffusivity}, and that of the ACM \cite{Golden:CMP-473}
in \eqref{eq:Maxwells_Equations} and \eqref{eq:sigma*}  are on equal
mathematical footing. It is worth  
mentioning that, due to the skew-symmetry of $\Sb=\Hb$ for the case of
a time-independent velocity field $\vec{u}$, the intensity-flux
relationship $\vec{J}_k=\bsig\vec{E}_k=\bkappa\vec{E}_k$ in
\eqref{eq:Maxwells_Equations} is similar to that of a Hall medium
\cite{Isichenko:JNS:1991:375}.    




Another immediate consequence of equation \eqref{eq:Sb_Skew}, for
steady $\vec{u}$, is the coercivity of the bilinear functional
$\Phi(\vec{\xi},\vec{\zeta}\,)=\langle\bsig\vec{\xi}\cdot\vec{\zeta}\,\rangle$ for all $\varepsilon>0$. By equation
\eqref{eq:Bounded_H}, this functional is also bounded in the case of
steady $\vec{u}$ for all $\varepsilon<\infty$. Therefore, the Lax-Milgram theorem
\cite{McOwen:2003:PDE} provides the existence and uniqueness of a
solution $\vec{\nabla}\chi_k\in\Hs$ satisfying the cell problem
\eqref{eq:Cell_Problem}, or equivalently equation
\eqref{eq:Maxwells_Equations}, in this time-independent case. The
details are as follows. 




The distributional form of equation \eqref{eq:Cell_Problem}, written
as $\vec{\nabla}\cdot\bsig\vec{E}_k=0$, is given by
$\langle\bsig(\vec{\nabla}\chi_k+\vec{e}_k)\cdot\vec{\nabla}\zeta\rangle=0$, where $\zeta$ is a compactly
supported, infinitely differentiable function on $\Tc\otimes\Vc$, and we
stress that $\vec{\nabla}\zeta$ is \emph{curl-free}. Motivated by this, we
consider the following variational problem: find $\vec{\nabla}\chi_k\in\Hs$ such
that   
%
\begin{align}\label{eq:Variational}
  \langle\bsig(\vec{\nabla}\chi_k+\vec{e}_k)\cdot\vec{\xi}\,\rangle=0, \text{ for all }
  \vec{\xi}\in\Hs.
\end{align}
%
In order to directly apply the Lax-Milgram Theorem, we rewrite
equation \eqref{eq:Variational} as 
%
\begin{align}  \label{eq:Bilinear_functional_E} 
   &\Phi(\vec{\nabla}\chi_k,\vec{\xi}\,)
     =\langle\bsig\vec{\nabla}\chi_k\cdot\vec{\xi}\,\rangle
     =-\langle\bsig\vec{e}_k\cdot\vec{\xi}\,\rangle
     =f(\vec{\xi}\,). 
\end{align}
%
By equation \eqref{eq:Sb_Skew} $\Phi$ is coercive, i.e.
%$\Phi(\vec{\xi},\vec{\xi})=\langle[(\varepsilon\Ib+\Sb)]\vec{\xi}\cdot\vec{\xi}\rangle=\varepsilon\|\vec{\xi}\|^2>0$ 
%
\begin{align}\label{eq:Phi_Coercive}
  \Phi(\vec{\xi},\vec{\xi}\,)=\langle[(\varepsilon\Ib+\Sb)]\vec{\xi}\cdot\vec{\xi}\,\rangle=\varepsilon\|\vec{\xi}\,\|^2>0,
   \text{ for all } \vec{\xi}\in\Hs
\end{align}
%
such that $\|\vec{\xi}\,\|\neq0$ and $\varepsilon>0$, where $\|\cdot\|$
is the norm induced by the inner-product $\langle\cdot,\cdot\rangle$. Recall that
$\Sb=\Hb$ in this time-independent case. This, equation 
\eqref{eq:Bounded_H}, the triangle inequality,
and the Cauchy-Schwartz inequality imply that $\Phi$ is also bounded for
all $\varepsilon<\infty$
%for all $\vec{\xi},\vec{\zeta}\in\Hs$
% $\Phi(\vec{\xi},\vec{\zeta})\leq(\varepsilon+\|H\|)\|\vec{\xi}\|\|\vec{\zeta}\|$
%
\begin{align}\label{eq:Phi_Bounded}
  \Phi(\vec{\xi},\vec{\zeta})\leq(\varepsilon+\|H\|)\|\vec{\xi}\,\|\|\vec{\zeta}\,\|<\infty,
  \text{ for all } \vec{\xi}\in\Hs.
\end{align}
%
For the same reasons, the linear functional $f(\vec{\xi}\,)$ in
\eqref{eq:Bilinear_functional_E} is bounded for all
$\vec{\xi}\in\Hs$. Therefore, the Lax-Milgram theorem
\cite{McOwen:2003:PDE} provides the existence of a unique
$\vec{\nabla}\chi_k\in\Hs$ satisfying \eqref{eq:Cell_Problem} in this
time-independent case. 


In the time-dependent case, equation \eqref{eq:Sb_Skew} hence
\eqref{eq:Phi_Coercive} does not hold for all $\vec{\xi}\in\Hs$. Moreover, 
the operator $\partial_t$ hence $\bsig$ is not bounded on $\Hs$
\cite{Reed-1980,Stakgold:BVP:2000}, so \eqref{eq:Phi_Bounded} does not
hold. Consequently, the Lax-Milgram theorem cannot be directly
applied, and alternate techniques  
\cite{Friedman:1969:PDE,Friedman:1969:PDE:Parabolic} must be used to
prove the existence and uniqueness of a solution $\vec{\nabla}\chi_k\in\Hs$ 
satisfying the cell problem \eqref{eq:Cell_Problem}. This discussion
illustrates key differences in the analytic structure of the effective
parameter problem for $\bkappa^*$, between the cases of steady and
dynamic velocity fields $\vec{u}$, which stem from the unboundedness
of the operator $\partial_t$ on $\Hs_{\,\Tc}$, hence $\bsig$ on $\Hs$. In Section
\ref{sec:Integral_Rep}, we will discuss other consequences of this
boundedness/unboundedness property of the operator $\bsig$, and
demonstrate that it leads to significant differences in the spectral
measure underlying an  integral representation of $\bkappa^*$.     









\subsubsection{Integral representation for the effective
  diffusivity}\label{sec:Integral_Rep}
%
In this section, we employ the Hilbert space formulation of the effective
parameter problem for $\bkappa^*$, discussed in Section
\ref{sec:Hilbert_Space}, and the \emph{spectral theorem}
\cite{Reed-1980,Stone:64,Stakgold:BVP:2000}, to provide an 
integral representation for $\bkappa^*$ involving a \emph{spectral
  measure} of a self-adjoint linear operator $\Mb$. This integral
representation follows from the resolvent formula for $\vec{\nabla}\chi_k$
involving $\Mb$        
% 
\begin{align}\label{eq:Resolvent_Rep}
  \vec{\nabla}\chi_k=\I(-\I\varepsilon\Ib-\Mb)^{-1}[\bGamma\Hb\vec{e}_k], \quad
  \Mb=\I\bGamma\Sb\bGamma,
\end{align}
%
where $\I=\sqrt{-1}\,$, $\Sb=\Hb-\Delta^{-1}\partial_t\Ib$ was defined in
\eqref{eq:Sb_Skew}, and we have defined in \eqref{eq:Resolvent_Rep}
$\bGamma=\bGamma_\times$ for notational simplicity. Equation
\eqref{eq:Resolvent_Rep} follows from 
applying the integro-differential operator $\vec{\nabla}\Delta^{-1}$ to
$\vec{\nabla}\cdot\bsig\vec{E}_k=0$ in equation \eqref{eq:Maxwells_Equations},
with $\vec{E}_k=\vec{\nabla}\chi_k+\vec{e}_k$ and  
$\bsig=\varepsilon\Ib+\Sb$, yielding 
%
\begin{align}\label{eq:Pre_Resolvent}
  \bGamma(\varepsilon\Ib+\Sb)\vec{\nabla}\chi_k=-\bGamma\Hb\vec{e}_k.
\end{align}
%
The equivalence of equations \eqref{eq:Resolvent_Rep} and
\eqref{eq:Pre_Resolvent} can be seen by noting that $\vec{\nabla}\chi_k\in\Hs$
implies $\bGamma\vec{\nabla}\chi_k=\vec{\nabla}\chi_k$, and writing $1=-\I^{\,2}$. We 
stress that, even though the imaginary unit $\I$ was introduced in
\eqref{eq:Resolvent_Rep}, the representation of $\vec{\nabla}\chi_k$ in this
equation is \emph{real-valued}. 





In the case of a time-independent velocity field $\vec{u}$, $\Sb=\Hb$
and the operator $\Mb$ in \eqref{eq:Resolvent_Rep} is given by
$\Mb=\I\bGamma\Hb\bGamma$. Recall that $\bGamma$ is an orthogonal
projector from $\Hs_{\,\Vc}$ to $\Hs_\times$, and therefore has unit operator
norm $\|\bGamma\|=1$ on $\Hs$ \cite{Reed-1980,Stone:64}. By equation
\eqref{eq:Bounded_H}, $\Hb$ is also bounded in operator
norm on $\Hs$. Therefore, in this time-independent case, $\Mb$ is a
\emph{bounded} linear operator on $\Hs$ with operator norm 
$\|\Mb\|\leq\|\bGamma\|\|\Hb\|\|\bGamma\|=\|\Hb\|<\infty$. Since $\Mb$ is bounded, its
(Hilbert space) adjoint $\Mb^*$ is also bounded with 
$\|\Mb\|=\|\Mb^*\|$ \cite{Reed-1980} and they consequently have common
domains, 
%
\begin{align}\label{eq:Domain_M}
  D(\Mb)=D(\Mb^*),
\end{align}
%
which are the entire space, $D(\Mb)=D(\Mb^*)=\Hs$. In
Section \ref{sec:Appendix} we show that $\Mb$ is symmetric  
%
\begin{align}\label{eq:Symmetric_M}
  \langle\Mb\vec{\xi}\cdot\vec{\zeta}\,\rangle=\langle\vec{\xi}\cdot\Mb\vec{\zeta}\,\rangle,
  \, \text{ for all } \; \vec{\xi},\vec{\zeta}\in D(\Mb).
\end{align}
%
By definition \cite{Reed-1980}, the two properties \eqref{eq:Domain_M}
and \eqref{eq:Symmetric_M} together imply that the operator $\Mb$ is
\emph{self-adjoint}, i.e. $\Mb=\Mb^*$. Conversely, the
Hellinger--Toeplitz theorem \cite{Reed-1980} states that if the
operator $\Mb$ satisfies equation \eqref{eq:Symmetric_M} 
for \emph{every} $\vec{\xi},\vec{\zeta}\in\Hs$, then $\Mb$ is bounded. This
suggests that when $\Mb$ is unbounded on $\Hs$, it is defined as a 
self-adjoint operator only on a proper (dense) subset of $\Hs$. 




In the case of a time-dependent velocity field $\vec{u}$, the operator
$\Sb$ is given by $\Sb=\Hb-\Delta^{-1}\partial_t\Ib$. Since $\Vc$ is a bounded
domain, $\Delta^{-1}$ is a compact \cite{Stakgold:BVP:2000},
hence bounded operator on $\Hs$ (more precisely $\Hs_{\,\Vc}$). The
operator $\partial_t\Ib$, on the other hand, is \emph{unbounded}
\cite{Reed-1980,Stone:64} on $\Hs$ 
(more precisely $\Hs_{\,\Tc}$). Hence $\Mb$ is unbounded on $\Hs$. The
unboundedness of $\partial_t\Ib$ on $\Hs$ can be seen by considering the
orthonormal set of functions $\{\vec{\psi}_n\}\subset\Hs$ with components
$(\vec{\psi}_n)_j$, $j=1,\ldots,d$, defined by
%$\langle\vec{\psi}_n\cdot\vec{\psi}_m\rangle=\delta_{nm}$, $n,m\in\mathbb{N}$,
%
\begin{align}\label{eq:Orthonormal}
  (\vec{\psi}_n)_j(t,\vec{x})=\alpha\sin((n+j)\pi t/T), \quad
  \alpha=\sqrt{2/(Td)},
  \qquad
  \langle\vec{\psi}_n\cdot\vec{\psi}_m\rangle=\delta_{nm}, \quad
  n,m\in\mathbb{N}.
\end{align}
%
The action of the operator $\partial_t\Ib$ on the members of the set
$\{\vec{\psi}_n\}$ is denoted $\{\partial_t\vec{\psi}_n\}$, and has components
$(\partial_t\vec{\psi}_n)_j$ and $\Hs$-norm $\|\partial_t\vec{\psi}_n\|$ given by
%$(\partial_t\vec{\psi}_n)_j(t,\vec{x})=b_{nj}\cos((n+j)\pi t/T)$, $b_{nj}=(n+j)a\pi/T$.
%
\begin{align}\label{eq:Orthonormal_Diff}
  (\partial_t\vec{\psi}_n)_j(t,\vec{x})=[(n+j)\alpha\pi/T]\cos((n+j)\pi t/T),\quad
  \|\partial_t\vec{\psi}_n\|^2%=(T/2)\sum_jb_{nj}^2
               =\sum_j[(n+j)\pi/T]^2,
\end{align}
%
%The $\Hs$-norm of $\partial_t\vec{\psi}_n$ is given by
%$\|\partial_t\vec{\psi}_n\|^2=(T/2)\sum_jb_{nj}^2=\sum_j[(n+j)\pi/T]^2$. 
which clearly demonstrates the unboundedness of the operator $\partial_t\Ib$
on $\Hs$.




The above analysis demonstrates that the unbounded operator
$\Tb=\I\partial_t\Ib$ is defined only on a proper subset of $\Hs$,
i.e. $D(\Tb)\subset\Hs$. However, the domain $D(\Tb)$ of $\Tb$ can be
defined as a (dense) subset of $\Hs$ such that $\Tb$ is bounded
\cite{Reed-1980,Stone:64}. Moreover, on this domain, $\Tb$ can be
extended to a \emph{closed} linear operator \cite{Reed-1980,Stone:64}.
For such a domain, simple integration by parts and the sesquilinearity 
of the inner-product $\langle\cdot,\cdot\rangle$ shows that, as an operator on $D(\Tb)$,
$\Tb$ is symmetric, i.e. it satisfies \eqref{eq:Symmetric_M} with
$\Tb$ in place of $\Mb$. Although, in general \cite{Reed-1980}, the
domain $D(\Tb^*)$ of its adjoint $\Tb^*$ does not satisfy the property
displayed in equation \eqref{eq:Domain_M} and in such circumstances,
$\Tb$ is \emph{not} self-adjoint on $D(\Tb)$. Only for self-adjoint
linear operators does the spectral theorem hold \cite{Reed-1980},
which provides the existence of the promised integral representation
for $\bkappa^*$, involving a spectral measure of $\Mb$. It is therefore
necessary that we find a domain $D(\Mb)$ for which $\Mb$ is a
self-adjoint operator.   




Toward this goal, and to illustrate these ideas, we consider the
operator $\I\partial_t$ with three different domains, which are everywhere
dense in $L^2(\Tc)$ \cite{Stone:64}. First, consider the set
$\hat{\Ds}_{\Tc}$ of all functions $\xi\in L^2(\Tc)$ such that $\xi(t)$ is
absolutely continuous \cite{Royden:1988:RA} on the interval $\Tc$ and
has a derivative $\xi^{\,\prime}(t)$ belonging to $L^2(\Tc)$,
i.e. \cite{Stone:64,Royden:1988:RA}  
%
\begin{align}\label{eq:AC_L2}
  \hat{\Ds}_{\Tc}=\left\{
       \xi\in L^2(\Tc) \ : \ \xi(t)=c+\int_0^tg(\tau)d\tau, \quad  g\in L^2(\Tc)
       \right\},
\end{align}
%
where the constant $c$ and function $g\in L^2(\Tc)$ are
arbitrary. Second, consider the set $\tilde{\Ds}_{\Tc}$ of all
functions $\xi\in\hat{\Ds}_{\Tc}$ that satisfy the periodic boundary
condition $\xi(0)=\xi(T)$, i.e. functions $\xi$ satisfying the properties of
equation \eqref{eq:AC_L2} with $\int_0^Tg(\tau)d\tau=0$. Finally, consider the
set $\Ds_{\Tc}$ of all functions $\xi\in\hat{\Ds}_{\Tc}$ that satisfy the
Dirichlet boundary condition $\xi(0)=\xi(T)=0$, i.e. functions $\xi$
satisfying the properties of equation \eqref{eq:AC_L2} with $c=0$ and
$\int_0^Tg(\tau)d\tau=0$. These sets satisfy
$\Ds_{\Tc}\subset\tilde{\Ds}_{\Tc}\subset\hat{\Ds}_{\Tc}$ and are each everywhere
dense in $L^2(\Tc)$ \cite{Stone:64}. Let the operators $\hat{\Ab}$,
$\tilde{\Ab}$, and $\Ab$ be identified as $\I\partial_t$ with domains
$\hat{\Ds}_{\Tc}$, $\tilde{\Ds}_{\Tc}$, and $\Ds_{\Tc}$,
respectively. Then \cite{Stone:64}, $\Ab$ is a closed, linear,
symmetric, operator with adjoint $\Ab^*\equiv\hat{\Ab}$, and the operator
$\tilde{\Ab}$ is a \emph{self-adjoint} extension of $\Ab$. 



Since the set $\tilde{\Ds}_{\Tc}$ is everywhere dense in $L^2(\Tc)$
and consists of periodic functions $\xi(t)$ satisfying $\xi(0)=\xi(T)$,
the set $\As_{\Tc}=\otimes_{i=1}^d\tilde{\Ds}_{\Tc}$ is everywhere dense in
$\Hs_{\,\Tc}$. Moreover, since $\tilde{\Ab}=\I\partial_t$ with domain
$\tilde{\Ds}_{\Tc}$ is self-adjoint, the operator $\Tb=\I\partial_t\Ib$ with
domain $D(\Tb)=\As_{\Tc}$ is self-adjoint. This is seen by noting
that, for all $\vec{\xi},\vec{\zeta}\in\As_{\Tc}$,
$\Tb\vec{\xi}=(\tilde{\Ab}\xi_1,\ldots,\tilde{\Ab}\xi_d)$, for example, and  
%
\begin{align}
  \langle\Tb\vec{\xi}\cdot\vec{\zeta}\rangle=\sum_j\langle\tilde{\Ab}\xi_j,\zeta_j\rangle_2
                    =\sum_j\langle\xi_j,\tilde{\Ab}\zeta_j\rangle_2
                    =\langle\vec{\xi}\cdot\Tb\vec{\zeta}\rangle,
\end{align}
%
where $\langle\cdot,\cdot\rangle_2$ denotes the $L^2(\Tc)$ inner-product.
It follows that the operator $\Mb$ defined in equation
\eqref{eq:Resolvent_Rep} is self-adjoint on the space $\Hs_{\,t}\subset\Hs$,
which is everywhere dense in $\Hs$, given by
%
\begin{align}\label{eq:Ht}
  \Hs_{\,t}=\{\vec{\xi}\in\As_{\Tc}\otimes\Hs_\times: \langle\vec{\xi}\,\rangle=0\}.
\end{align}
%



We are now ready to provide an integral representation
for the effective diffusivity tensor $\bkappa^*$. This follows from the
spectral theorem of operational calculus in Hilbert space
\cite{Reed-1980,Stone:64}, which states that there is a
one-to-one correspondence between the self-adjoint operator $\Mb$ and
a family $\{\Qb(\lambda)\}$, $-\infty<\lambda<\infty$, of \emph{projection} operators - the
resolution of the identity - which satisfies $\lim_{\lambda\to-\infty}\Qb(\lambda)=0$ and
$\lim_{\lambda\to\infty}\Qb(\lambda)=\Ib$. Moreover, let $\vec{\xi},\vec{\zeta}\in\Hs_{\,t}$ and
consider the following functions $\mu_{\xi\zeta}(\lambda)=\langle\Qb(\lambda)\vec{\xi}\cdot\vec{\zeta}\,\rangle$
and $\mu_{\xi\xi}(\lambda)=\langle\Qb(\lambda)\vec{\xi}\cdot\vec{\xi}\,\rangle=\|\Qb(\lambda)\vec{\xi}\,\|^2$ of 
\emph{bounded variation} with associated Radon
measures $\mu_{\xi\zeta}(d\lambda)$ and $\mu_{\xi\xi}(d\lambda)$ \cite{Stone:64}
%
\begin{align}\label{eq:Bounded_Variation}
  \mu_{\xi\zeta}(d\lambda)=\langle\Qb(d\lambda)\vec{\xi}\cdot\vec{\zeta}\,\rangle, \quad
  \mu_{\xi\xi}(d\lambda)=\|\Qb(d\lambda)\vec{\xi}\,\|^2.
\end{align}
%
Let $F(\lambda)$ be an arbitrary complex-valued function and denote by
$\mathscr{D}(F)$ the set of all $\vec{\xi}\in\Hs_{\,t}$ such that
$F\in L^2(\mu_{\xi\xi})$, the class of square $\mu_{\xi\xi}$-integrable
functions. Then $\mathscr{D}(F)$ is a linear manifold and there exists
a linear transformation $\Mb(F)$ with domain $\mathscr{D}(F)$ defined
in terms of the Radon-Stieltjes integrals \cite{Stone:64}
%
\begin{align}\label{eq:Spectral_Theorem}
  \langle\Mb(F)\vec{\xi}\cdot\vec{\zeta}\,\rangle=\int_{-\infty}^\infty F(\lambda)\,\mu_{\xi\zeta}(d\lambda), \qquad
  &\forall \, \vec{\xi}\in\mathscr{D}(F), \ \vec{\zeta}\in\Hs_{\,t}
  \\
  \langle\Mb(F)\vec{\xi}\cdot\Mb(G)\vec{\zeta}\,\rangle=\int_{-\infty}^\infty F(\lambda)\bar{G}(\lambda)\,\mu_{\xi\zeta }(d\lambda), \quad
  &\forall \, \vec{\xi}\in\mathscr{D}(F), \ \vec{\zeta}\in\mathscr{D}(G),
  \notag
\end{align}
%
where $\bar{G}$ denotes the complex conjugate of the function $G$, and
the operator $\Mb(G)$ and set $\mathscr{D}(G)$ involving $G$ and
$\vec{\zeta}$ are defined analogously to that for $F$ and $\vec{\xi}$. A
Radon-Stieltjes integral representation of the functional
$\|\Mb(F)\vec{\xi}\,\|^2$ follows from the second equation in
\eqref{eq:Spectral_Theorem} with $G=F$ and $\vec{\xi}=\vec{\zeta}$, and
involves measure $\mu_{\xi\xi}(d\lambda)$ in \eqref{eq:Bounded_Variation}
\cite{Stone:64}.    


An integral representation for the components $\kappa^*_{jk}$, $j,k=1,\ldots,d$,
of $\bkappa^*$ follows from equations \eqref{eq:Eff_Diffusivity} and
\eqref{eq:Resolvent_Rep}, and the second formula in
\eqref{eq:Spectral_Theorem} with 
%
\begin{align}\label{eq:F_G}
  F(\lambda)=G(\lambda)=\I(-\I\varepsilon-\lambda)^{-1}, \quad
  \vec{\xi}=\vec{g}_j, \quad
  \vec{\zeta}=\vec{g}_k, \quad
  \vec{g}_j=\bGamma\Hb\vec{e}_j,
  \quad  j,k=1,\ldots,d.
\end{align}
%
More specifically, the orthogonality of the projection operators
$\bGamma_\times=\bGamma$ and $\bGamma_0$ in \eqref{eq:Helmholtz} implies
that the vector field
$\vec{g}_j(t,\vec{x})=\bGamma\Hb(t,\vec{x})\vec{e}_j$ is curl-free and
mean-zero. Moreover, by equation \eqref{eq:Bounded_H}, $\vec{g}_j$
and $\partial_t\vec{g}_j$ are bounded in $\Hs$-norm, with $\|\vec{g}_j\|\leq\|\Hb\|$
and $\|\partial_t\vec{g}_j\|\leq\|\partial_t\Hb\|$, so that
$\vec{g}_j\in\Hs_{\,t}$ for all $j=1,\ldots,d$. Also, since
$\I\varepsilon\not\in\mathbb{R}$ for all $\varepsilon>0$ and the measure $\mu_{\xi\xi}(d\lambda)$
is of bounded mass \cite{Stone:64} 
%$\mu^0_{\xi\xi}=\int\mu_{\xi\xi}(d\lambda)=\|\vec{\xi}\,\|^2\leq\|\Hb\|^2<\infty$,
%
\begin{align}\label{eq:Mass}
  \mu^0_{\xi\xi}=\int_{-\infty}^\infty\mu_{\xi\xi}(d\lambda)
        =\int_{-\infty}^\infty\langle\Qb(d\lambda)\vec{\xi}\cdot\vec{\xi}\,\rangle   
       =\|\vec{\xi}\,\|^2
       =\langle\Hb^T\bGamma\Hb\vec{e}_k\cdot\vec{e}_k\rangle
       \leq\|\Hb\|^2<\infty,
\end{align}
%
we have that $F(\lambda)$ in \eqref{eq:F_G} satisfies $F\in L^2(\mu_{\xi\xi})$, hence 
$\vec{g}_j\in\mathscr{D}(F)$, for all $\varepsilon>0$ and $j=1,\ldots,d$. Therefore, the 
following Radon-Stieltjes integral representation for the components
$\kappa^*_{jk}$ of $\bkappa^*$, involving the components $\mu_{jk}(d\lambda)$ of
the matrix-valued measure $\bmu(d\lambda)$, holds for all $\varepsilon>0$
%
\begin{align}\label{eq:Integral_Rep_kappa*}
  \kappa^*_{jk}=\varepsilon\left(\delta_{jk}+\int_{-\infty}^\infty\frac{\mu_{jk}(d\lambda)}{\varepsilon^2+\lambda^2}\right), \quad
         \mu_{jk}(d\lambda)=\langle\Qb(d\lambda)\vec{g}_j\cdot\vec{g}_k\rangle,
  \quad  j,k=1,\ldots,d,
\end{align}
%
where we have used the notation $\mu_{jk}(d\lambda)=\mu_{\xi\zeta}(d\lambda)$ for
$\vec{\xi}=\vec{g}_j$ and $\vec{\zeta}=\vec{g}_k$.


We conclude this section with a few remarks regarding the integral
representation in \eqref{eq:Integral_Rep_kappa*}. The Radon measure
$\mu_{jk}(d\lambda)$ is a \emph{spectral measure} associated with the
self-adjoint linear operator $\Mb$ in the $(\vec{g}_j,\vec{g}_k)$
state \cite{Reed-1980}. Since $\Qb(\lambda)$ is a projection operator, the
diagonal components of $\bmu(d\lambda)$ are \emph{positive} measures, 
$\mu_{kk}(d\lambda)=\|\Qb(d\lambda)\vec{g}_k\|^2$. In the case of a time-independent
velocity field $\vec{u}$, where $\Mb=\bGamma\Hb\bGamma$, the range of
integration in equation \eqref{eq:Integral_Rep_kappa*} is given by
$-\|\Mb\|\leq\lambda\leq\|\Mb\|$, with $\|\Mb\|\leq\|\Hb\|<\infty$, by \eqref{eq:Bounded_H}.  A
key feature of the integral representation for $\bkappa^*$ in
\eqref{eq:Integral_Rep_kappa*} is that parameter information in $\varepsilon$ is
\emph{separated} from the geometry and dynamics of the velocity field,
which is encapsulated in the measure $\bmu$. In Section 
\ref{sec:Assymptotics} we employ the integral representation for
$\bkappa^*$ in \eqref{eq:Integral_Rep_kappa*} to obtain asymptotic
behavior of $\kappa^*_{jk}$ as $\varepsilon\to0$. 









\section{Asymptotic analysis of effective
  diffusivity} \label{sec:Assymptotics} 
In two dimensions, $d=2$, the matrix $\Hb$ is determined by a stream
function $H(t,\vec{x})$ 
%
\begin{align}\label{eq:u_H}  
  \Hb=\left[
  \begin{array}{cc}
    0  & H\\
    -H & 0
  \end{array}
  \right],
  \qquad
  \vec{u}=[\partial_{x_1}H, \ \partial_{x_2}H].
\end{align}
%


\section{Numerical Results}\label{sec:Num_Results}
%
Since we are focusing on flows which are periodic on the spatial
region $\Vc$, it is convenient to consider the Fourier
representation of such a vector field $\vec{\xi}(t,\vec{x})$, 
%
\begin{align}
  \vec{\xi}(t,\vec{x})
    =\sum_{\vec{k}\in\mathbb{Z}^d}
       \hat{Y}(t,\vec{k})\e^{\I\vec{k}\cdot\vec{x}},
  \qquad
  \hat{Y}(t,\vec{k})
    =\frac{1}{(2\pi)^d}\int_{\Vc}
       \hat{Y}(t,\vec{x})\e^{-\I\vec{k}\cdot\vec{x}} \d\vec{x}.
\end{align}
%
%where $|\Vc|=(2\pi)^d$ is the Lebesgue measure of the region
%$\Vc$.
The associated action of the above projection operators on a function
$\vec{\xi}\in\Hs$ is given by \cite{Fannjiang:SIAM_JAM:333} 
%
\begin{align}\label{eq:Projections}
  \bGamma_0\vec{\xi}(t,\vec{x})
    =\langle\vec{\xi}(t,\vec{x})\rangle_x
    =\hat{Y}(t,0),
    \qquad
  \bGamma_\times\vec{\xi}(t,\vec{x})
    =\sum_{\vec{k}\neq0}
       \frac{\vec{k}(\vec{k}\cdot\hat{Y}(t,\vec{k}))}{|\vec{k}|^2}
       \e^{\I\vec{k}\cdot\vec{x}},
    \\
  \bGamma_\bullet\vec{\xi}(t,\vec{x})
    =\sum_{\vec{k}\neq0}
       \frac{\vec{k}\times(\vec{k}\times\hat{Y}(t,\vec{k}))}{|\vec{k}|^2}
       \e^{\I\vec{k}\cdot\vec{x}}
    =\sum_{\vec{k}\neq0}\left(I-
       \frac{\vec{k}\vec{k}\cdot}{|\vec{k}|^2}\right) 
       \hat{Y}(t,\vec{k})  \e^{\I\vec{k}\cdot\vec{x}},
       \notag
\end{align}
%
where $\langle\cdot\rangle_x$ denotes spatial averaging over $\Vc$. From
equation \eqref{eq:Projections} it is clear that
$\bGamma_\times+\bGamma_\bullet+\bGamma_0=\Ib$. 

%\newpage

% redefine the command that creates the equation no.
  \setcounter{equation}{1}  % reset equation counter
  \setcounter{section}{0}  % reset section counter
  \renewcommand{\theequation}{A-\arabic{equation}} 
\renewcommand{\thesection}{A-\arabic{section}}
%
\section{Appendix} 
\label{sec:Appendix}
%
\subsection{Multiple scale Method}\label{sec:Multiscal_Method}
%
In this section we provide the details of the multiple scale method
which leads to \eqref{}


which involves the following expansion of $\phi^\delta$ in powers of $\delta$
%
\begin{align}\label{eq:Expand}
  \phi^\delta(t,\vec{x})=\bar{\phi}(t,\vec{x})
                 +\delta\phi^{(1)}(t,\vec{x},\tau\,\vec{y})
                 +\delta^{\,2}\phi^{(2)}(t,\vec{x},\tau\,\vec{y})+\cdots.
\end{align}
%
Inserting this in \eqref{eq:ADE_delta}, writing
% $\partial_t\phi^{(j)}=\partial_t\phi^{(j)}+\delta^{-2}\partial_\tau\phi^{(j)}$,
% $\vec{\nabla}\phi^{(j)}=\vec{\nabla}_x\phi^{(j)}+\delta^{-1}\vec{\nabla}_y\phi^{(j)}$, and
% $\Delta\phi^{(j)}=[\Delta_x+2\delta^{-1}]\phi^{(j)}\vec{\nabla}_x\cdot\vec{\nabla}_y+\delta^{-2}\Delta_y\phi^{(j)}$
%
\begin{align}\label{eq:Trans_Op_delta}
  \partial_t\phi^{(j)}=[\partial_t+\delta^{-2}\partial_\tau]\phi^{(j)}, \quad
  \vec{\nabla}\phi^{(j)}=[\vec{\nabla}_x+\delta^{-1}\vec{\nabla}_y]\phi^{(j)}, \quad
  \Delta\phi^{(j)}=[\Delta_x+2\delta^{-1}\vec{\nabla}_x\cdot\vec{\nabla}_y+\delta^{-2}\Delta_y]\phi^{(j)},
\end{align}
%
and collecting coefficients of like powers of $\delta$ leads to a sequence
of problems. Due to the independence of $\bar{\phi}$ on the fast
variables $\tau$ and $\vec{y}$, the coefficients of $\delta^{-2}$ vanish.





The convergence is in $L^2$ \cite{Fannjiang:SIAM_JAM:333},
%
\begin{align}
  \lim_{\delta\to0}\left[\,\sup_{0\leq t\leq t_0}
    \int\left|\phi^\delta(t,\vec{x})-\bar{\phi}(t,\vec{x})\right|^2\d\vec{x}
    \;\right]=0,
\end{align}
%
for all $t_0<\infty$, where we have used the notation $\d\vec{x}=\d x_1\cdots \d
x_d$ for the product Lebesgue measure.


\medskip

{\bf Acknowledgements.}
We gratefully acknowledge support from the Division of Mathematical
Sciences and the Office of Polar Programs at the U.S. 
National Science Foundation (NSF) through Grants
DMS-1009704, ARC-0934721, and DMS-0940249. We are also grateful for 
support from the Office of Naval Research (ONR) through
Grants N00014-13-10291 and N00014-12-10861. Finally, we would like to 
thank the NSF Math Climate Research Network (MCRN) for their support
of this work. 


\medskip

\bibliographystyle{plain}
\bibliography{murphy}
\end{document}

% LocalWords:  McMaster RM jk sig eps def Maxwells Milgram coercivity diag jX
% LocalWords:  mh Cond mu kk PtI Fs Es Hashin Shtrikman extremized Decomp chi
% LocalWords:  Acknowledgements DMS ONR MCRN murphy
