\documentclass{cmslatex}   
%\usepackage{latexsym, amssymb, enumerate, amsmath}
\usepackage{graphicx,amssymb,amsmath,amsfonts,mathrsfs}

 %
\sloppy

\thinmuskip = 0.5\thinmuskip \medmuskip = 0.5\medmuskip
\thickmuskip = 0.5\thickmuskip \arraycolsep = 0.3\arraycolsep

\renewcommand{\theequation}{\arabic{section}.\arabic{equation}}

\def\open#1{\setbox0=\hbox{$#1$}
\baselineskip = 0pt \vbox{\hbox{\hspace*{0.4 \wd0}\tiny 
$\circ$}\hbox{$#1$}} \baselineskip = 10pt\!}

\newcommand{\ph}{\hat{\phi}}
\newcommand{\pt}{\tilde{\phi}} 
\newcommand{\pc}{\check{\phi}}
\newcommand{\gh}{\hat{\gamma}}
\newcommand{\Dh}{\hat{\Delta}}
\newcommand{\dha}{\hat{\delta}}
\newcommand{\qh}{\hat{q}}
\newcommand{\xh}{\hat{x}}
\newcommand{\HM}{\mathcal{H}_{\text{max}}}
\newcommand{\Hm}{\mathcal{H}_{\text{min}}}
\newcommand{\sech}{\rm \hspace{0.7mm}sech}
\newcommand{\I}{\mathrm{i}}
\newcommand{\hh}{\hat{h}}
\newcommand{\mh}{m_r}
\newcommand{\mt}{m_i}
\newcommand{\M}{\mathbf{M}}
\newcommand{\X}{\mathbf{X}}
\newcommand{\T}{\mathbf{T}}
\newcommand{\Hb}{\mathbf{H}}
\newcommand{\K}{\mathbf{K}}
\newcommand{\J}{\mathbf{J}}
\renewcommand{\d}{\text{d}}

\newcommand\bsig{\mbox{\boldmath${\sigma}$}}
\newcommand\beps{\mbox{\boldmath${\epsilon}$}}
\newcommand\bxi{\mbox{\boldmath${\xi}$}}
\newcommand\bmu{\mbox{\boldmath${\mu}$}}
\newcommand\bEta{\mbox{\boldmath${\eta}$}}
\newcommand\balpha{\mbox{\boldmath${\alpha}$}}
\newcommand\brho{\mbox{\boldmath${\rho}$}}
\newcommand\bDelta{\mbox{\boldmath${\Delta}$}}
\newcommand\bXi{\mbox{\boldmath${\Xi}$}}


\newtheorem{thm}{Theorem}[section]
\newtheorem{prop}[thm]{Proposition}
\newtheorem{lem}[thm]{Lemma} 
\newtheorem{cor}[thm]{Corollary}

    %\theoremstyle{definition}

\newtheorem{defn}[thm]{Definition}
\newtheorem{notation}[thm]{Notation}
\newtheorem{example}[thm]{Example}
\newtheorem{conj}[thm]{Conjecture}
\newtheorem{prob}[thm]{Problem}

    %\theoremstyle{remark}

\newtheorem{rem}[thm]{Remark}
    % Use the standard latex environments for theorems, etc. Here is one
          % possible method of declaring them: It numbers all results by the
          % section, and uses a common numbering system for the different
          % environmentts.

\begin{document}

\title{Spectral measure computations \\ for composite materials}
%\thanks{%{Received date / Revised version date}
          % The correct dates will be entered by the CMS editor}}
%}}
          %For each author, make a block with the following four
          %macros:
% \author{full name
% \thanks {address, (email).}
% \and full name \thanks {address, (email).}}
%
\author{N. B. Murphy\thanks {murphy@math.utah.edu}
%
\and E. Cherkaev\thanks {elena@math.utah.edu}
%
\and C. Hohenegger\thanks {choheneg@math.utah.edu}
%
\and K. M. Golden\thanks {golden@math.utah.edu}
}
          %{Put the URL for your home page here if you have one}

          %Use \thanks statements for acknowledgments of grants and
          %support. They will appear below all the authors' addresses, so be
          %specific about which author is thanking whom:

          %\thanks{}

\pagestyle{myheadings} \markboth{Spectral measure computations}{Murphy et. al.}\maketitle


\begin{center}
University of Utah, Department of Mathematics \\ 155 S 1400 E
  RM 233, Salt Lake City, UT 84112-0090, USA
\end{center}

\vspace{3ex}


\begin{abstract}
The analytic continuation method of homogenization theory provides
Stieltjes integral representations for the effective parameters of
composite media. These representations involve the spectral measures
of self-adjoint random operators which depend only on the composite
geometry. On finite bond lattices, these random operators are
represented by random matrices and the spectral measures are given
explicitly in terms of their eigenvalues and eigenvectors. Here we
provide the mathematical foundation for rigorous computation of
spectral measures for such composite media, and develop a numerically
efficient projection method to enable such computations. This is
accomplished by providing a novel formulation of the analytic
continuation method which is equivalent to the original formulation
and holds for both the finite lattice setting and the infinite
settings. We also introduce a family of random bond lattices and
directly compute the associated spectral measures and effective
parameters. The computed spectral measures are in excellent agreement
with known theoretical results. The behavior of the associated
effective parameters is consistent with the symmetries and theoretical
predictions of models, and the computed values fall within rigorous
bounds. Some previous calculations of spectral measures have relied on
finding the boundary values of the imaginary part of the effective
parameter in the complex plane. Our method instead relies on direct
computation of the eigenvalues and eigenvectors, which enables, for
example, statistical analysis of the spectral data. 
\end{abstract}

\begin{keywords}
composite materials, random resistor network, percolation,
homogenization, spectral measure, random matrix 
\smallskip

{\bf subject classifications.}
% GENERAL: Conference proceedings and collections of papers:
% Conference proceedings and collections of papers 
00B15,
% Measures, integration, derivative, holomorphy (all involving infinite-
% dimensional spaces): Measures and integration on abstract linear spaces
%46G12,
% OPERATOR THEORY: Special classes of linear operators: Hermitian and
% normal operators (spectral measures, functional calculus, etc.)
47B15,
% NUMERICAL ANALYSIS: Numerical linear algebra: Eigenvalues, eigenvectors
65C60,
% MECHANICS OF DEFORMABLE SOLIDS: Generalities, axiomatics,
% foundations of continuum mechanics of solids:
% Random materials and composite materials 
%74A40,
% MECHANICS OF DEFORMABLE SOLIDS: Generalities, axiomatics,
% foundations of continuum mechanics of solids: Nonsimple materials
%74E30,
% FUNCTIONS OF A COMPLEX VARIABLE: Analytic continuation 
30B40,
% OPTICS, ELECTROMAGNETIC THEORY: General: Composite media; random media
78A48,
% OPTICS, ELECTROMAGNETIC THEORY: Basic methods: Spectral methods
%78M22,
% CLASSICAL THERMODYNAMICS, HEAT TRANSFER: Basic methods: Spectral
% methods
%80M22,
% CLASSICAL THERMODYNAMICS, HEAT TRANSFER: Basic methods: Homogenization
80M40,
% PROBABILITY THEORY AND STOCHASTIC PROCESSES: Special processes:
% Interacting random processes; statistical mechanics type models;
% percolation theory
60K35
\end{keywords}


\section{Introduction}\label{Introduction}
%
Over the years a broad range of mathematical techniques have been
developed that reduce the analysis of complex composite materials,
with rapidly varying structures in space, to solving averaged, or
\emph{homogenized} equations involving an effective parameter.
Homogenization for composite media with rapidly varying coefficients
of thermal conductivity,  
electrical conductivity, electrical permittivity, or magnetic permeability,
for example,
was established by Papanicolaou and Varadhan \cite{Papanicolaou:RF-835}
for the steady state, static case with real parameters \cite{MILTON:2002:TC}. 
This work was extended
by Golden and Papanicolaou \cite{Golden:CMP-473,Golden:JSP-655} to the
quasi-static frequency dependent case with complex
parameters. Analysis of the effective dielectric problem for the fully
frequency dependent case described by the Helmholtz equation is given
in \cite{Simeonova:MMS:1113}.




The analytic continuation method (ACM) of homogenization theory
for \emph{two-component} media in the
quasi-static limit was developed by Bergman \cite{Bergman:PRL-1285},
Milton \cite{Milton:APL-300}, and Golden and Papanicolaou
\cite{Golden:CMP-473},
leading to Stieltjes integral representations for the effective
parameters.  The Golden-Papanicolaou formulation of this
method is based on the spectral theorem
and resolvent formulas involving random self-adjoint operators. This
formulation demonstrated that the measures underlying 
these integral representations are \emph{spectral measures} associated
with the random operators, which depend only on the composite
geometry. These measures contain all the information about the mixture
geometry, and provide a link between microgeometry and
transport. Local geometry is encoded in ``geometric'' resonances in
the measures \cite{Jonckheere_Luck_JPA_1998}, while global
connectivity is encoded by spectral gaps in the measures at the
spectral endpoints
\cite{Murphy:JMP:063506,Jonckheere_Luck_JPA_1998}. A remarkable
feature of the method is that once the spectral measures are found for
a given composite geometry, by the spectral coupling of the governing
equations
\cite{Cherkaev:IP-1203,MILTON:2002:TC,Cherkaev:2004:331,Cherkaev:PBCM:0921},
the effective electrical, magnetic, and thermal transport properties
are \emph{all} completely determined by these measures.


 
The integral representations yield rigorous \emph{forward bounds} on
the effective parameters of composites, given partial information on
the microgeometry
\cite{Bergman:PRL-1285,Milton:APL-300,Golden:CMP-473,Bergman:AP-78,Bruno:PRSLA-353}. One  
can also use the integral representations to obtain inverse bounds, 
where data on the electromagnetic response of a sample, for example,
is used to bound its structural parameters, such as the volume
fractions of the components
\cite{McPhedran:APA:19,McPhedran:MRSP:1990:195,Cherkaev:WRM-437,Cherkaev:IP-1203,Cherkaev:IP-065008,Zhang:JCP-5390,Bonifasi-Lista:PMB-3063,Cherkaev:JBiomech-345,Day:JPCM-99,Golden:JBM:337},
and even the separation of the inclusions in 
matrix particle composites \cite{Orum:PRSLA:2012}. Furthermore, the
spectral measure can be \emph{uniquely} reconstructed
\cite{Cherkaev:IP-1203} when the data is given for a continuous
interval of electromagnetic frequency. This, in turn, can be used to
calculate other effective parameters, such as the viscoelastic modulus
\cite{Cherkaev:JBiomech-345}, effective thermal conductivity
\cite{Cherkaev:2004:331,Cherkaev:PBCM:0921}, and recover the
associated structural parameters  
\cite{Cherkaev:IP-1203,Cherkaev:IP-065008,Zhang:JCP-5390,Bonifasi-Lista:PMB-3063,Cherkaev:JBiomech-345,Day:JPCM-99,Golden:JBM:337}. 
For classes of composites which undergo a percolation transition
\cite{Stauffer-92,Torquato:RHM-02}, the integral representations have
been used to obtain detailed information regarding the critical
behavior of the effective parameters in the scaling regime
\cite{Golden:PRL-3935,Murphy:JMP:063506}. The relationship between the
effective parameters and the system energy \cite{Murphy:JMP:063506}
has also led to a physically consistent statistical mechanics model for
two-component dielectric media which is also mathematically tractable
\cite{Murphy:PHD_Thesis}. 




Despite the many applications which have stemmed from the ACM,
explicit analytical calculations of the effective parameters and
spectral measures have been obtained for only a handful of composite
microstructures. There are various numerical methods which have been
used to compute the effective parameters of two-component
composites. These computations may, in principle, be used to compute
the corresponding spectral measures through the Stieltjes--Perron
inversion theorem. This theorem states that the measure is recovered
as a weak limit of the imaginary part of the effective parameter in
the complex plane.




Highly accurate numerical computations of the effective permittivity
for a class of continuum composites which have sharp corners are
described in \cite{Helsing:NJP:115005}. The computations are based on
a multigrid recursive compressed inverse preconditioning method
\cite{Helsing:2008:8820,Helsing:JCP:1171,Helsing:JCP:7533} developed 
for calculation of the effective conductivity of random
checkerboards. In \cite{Day:JPCM-96} the effective 
conductivity of the 2D random resistor network (RRN) 
was computed using an efficient algorithm that implements $Y$-$\Delta$
transformations of the network. In
\cite{Greengard:1994:379,Cheng:1997:629,Greengard:2006:64} the Fast
Multipole Method was exploited to compute the electrostatic fields and
the effective conductivity for two-component matrix particle
composites.      


In \cite{Helsing:NJP:115005,Day:JPCM-96}, the spectral measures
associated with the composite microstructures of interest were
computed using the Stieltjes--Perron inversion theorem. However, the
presence of delta components or essential singularities in the
measures, for example, makes it difficult to resolve details of the
spectrum using this approach. To help overcome this limitation, here
we develop a mathematical framework which provides a rigorous way to
directly compute the spectral measures and effective parameters for
finite lattice composite microstructures, or discretizations of
continuum composites. In particular, we provide a novel
formulation of the ACM which is equivalent to
the original formulation \cite{Golden:CMP-473,Bruno:PRSLA-353} and
holds for both the finite lattice setting and the infinite
settings. This analysis demonstrates that, in the finite lattice
setting, the random operators underlying the integral representations
of the effective parameters are represented by random matrices, and
the spectral measures are determined explicitly by their eigenvalues
and eigenvectors.   



As a consequence,
our approach provides a direct connection between the statistical
behavior of spectral data of random matrices and the behavior of the
effective transport processes of composites. This, in turn, has
provided a direct connection between the ACM and random matrix theory,
and has shown that transitions in the connectedness or percolation
properties of composites are reflected in the short and long range
eigenvalue correlations of the underlying random matrices
\cite{Murphy:PHD_Thesis}. Moreover, this transitional behavior
is captured by a one parameter universality class of random matrix
ensembles and provides a mechanism for the collapse of gaps in the
spectral measures \cite{Murphy:PHD_Thesis}, which leads to
critical behavior in the effective transport coefficients of
composites \cite{Murphy:JMP:063506}. This characterization of critical
behavior of transport in composites by the statistical properties of
eigenvalues and eigenvectors of random matrices is a key feature of
the ACM and our computational approach. 

    

\section{Mathematical Methods}\label{sec:Mathematical_Methods} 
% {\it The analytic continuation method (general operator
%   case). Spectral measure decomposition. Geometric resonances and gaps
%   in the spectral measure. Lattice systems (random matrix case) and
%   the direct calculation of the spectral measure. Forward bounds. What
%   new structure we found regarding the spectral measure
%   decomposition.} 
%
We now formulate the effective parameter problem for random
two-phase conductive media in the continuum and lattice settings,
yielding Stieltjes integral representations for the effective
conductivity and resistivity tensors. In Section
\ref{sec:Continuum_Setting}, we review and extend the ACM for the
continuum setting \cite{Golden:CMP-473}, while the lattice setting is
discussed in Section \ref{sec:Lattice_Setting}. The mathematical
framework underlying the \emph{infinite} lattice setting
\cite{Bruno:JSP-365,Golden:CMP-467}, reviewed in Section 
\ref{sec:Infinite_Lattice_Setting}, is analogous to that of
the continuum case \cite{Bruno:JSP-365}, and the integral
representations for the effective parameters follow with minor
modifications in the theory. In Section
\ref{sec:Finite_Lattice_Setting}, we develop a mathematical
framework for the \emph{finite} lattice setting, leading to integral 
representations for the effective parameters, summarized in Theorem 
\ref{thm:Discrete_Spectral_Theorem_ACM}, which are analogous to that
of the infinite, continuum and lattice cases. In order to derive the
integral representations for the finite lattice setting, significant
modifications must be made to the underlying mathematical
framework. Toward this goal, in Section
\ref{sec:Unification_finite_infinite} we provide a novel formulation
of the ACM which unifies the infinite settings and the finite lattice
setting. The proof of Theorem \ref{thm:Discrete_Spectral_Theorem_ACM}
is given in Section \ref{sec:Theorem_Proof}.    



\subsection{Continuum Setting}\label{sec:Continuum_Setting}
%
Consider a random two-phase conductive medium filling all of
$\mathbb{R}^d$, which is determined by the probability space
$(\Omega,P)$. Here, $\Omega$ is the set of all  geometric realizations of our
random medium, which is indexed by the parameter $\omega\in\Omega$ representing
one particular geometric realization, and $P$ is the associated
probability measure. Details regarding the underlying sigma-algebra
are discussed in \cite{Papanicolaou:RF-835}. Let $\bsig(\vec{x},\omega)$
and $\brho(\vec{x},\omega)$, $\vec{x}\in\mathbb{R}^d$, be the local complex
conductivity and resistivity tensors associated with the conductive
medium, which are related by $\bsig=\brho^{-1}$ and have components
$\sigma_{jk}(\vec{x},\omega)$ and $\rho_{jk}(\vec{x},\omega)$, $j,k=1,\ldots,d$, that are
(spatially) stationary random fields. 



A \emph{stationary} random field, $f:\mathbb{R}^d\times\Omega\to\mathbb{C}$, is a
field such that the  
joint distribution of $f(\vec{x}_1,\omega),\ldots,f(\vec{x}_n,\omega)$ and that of
$f(\vec{x}_1+\vec{\xi},\omega),\ldots,f(\vec{x}_n+\vec{\xi},\omega)$ is the same for all
$\vec{\xi}\in\mathbb{R}^d$ and $n\in\mathbb{N}$
\cite{Golden:CMP-473,Papanicolaou:RF-835}.
More specifically, we assume that there is a group of transformations
$\tau_x:\Omega\to\Omega$ and measurable functions $f^{\prime}(\omega)=f(0,\omega)$ on $\Omega$ such that
$f(\vec{x},\omega)=f^\prime(\tau_{-x}\,\omega)$ for all $\vec{x}\in\mathbb{R}^d$ and
$\omega\in\Omega$, with $\tau_x\tau_y=\tau_{x+y}$.  Moreover, we shall assume that the
group is one-to-one and preserves the measure $P$, i.e., $P(\tau_xA)=P(A)$
for all $P$-measurable sets $A$
\cite{Golden:CMP-473,Papanicolaou:RF-835}. For notational simplicity,
we will not distinguish between the functions $f^\prime:\Omega\to\mathbb{C}$ and
$f:\mathbb{R}^d\times\Omega\to\mathbb{C}$, as the context of each is now
clear. 






The group of transformations $\tau_x$ acting on $\Omega$ induces a group of
operators $T_x$ on the Hilbert space $L^2(\Omega,P)$ defined by
$(T_xf)(\omega)=f(\tau_{-x}\,\omega)$ for all $f\in L^2(\Omega,P)$. Since $\tau_x$ is measure
preserving, the operators $T_x$ form a unitary group and therefore
have closed densely defined infinitesimal generators $L_i$ in each
direction $i=1,\ldots,d$ with domain $\mathscr{D}_i\subset L^2(\Omega,P)$
\cite{Golden:CMP-473,Papanicolaou:RF-835}. Thus,  
%
\begin{align}\label{eq:Li}
  L_i=\left.\frac{\partial}{\partial x_i}T_x \right|_{x=0}, \quad i=1,\ldots,d,
\end{align}
%
where $x_i$
%$x_i=(\vec{x}_i)$
is the $i^{\,\text{th}}$ component of the vector
$\vec{x}$ and differentiation is defined in the sense of convergence
in $L^2(\Omega,P)$ for elements of $\mathscr{D}_i$
\cite{Golden:CMP-473}. The closed subset
$\mathscr{D}=\cap_{i=1}^d\mathscr{D}_i$ of $L^2(\Omega,P)$ is a Hilbert space
\cite{Golden:CMP-473} with inner product $\langle \cdot,\cdot\rangle_D$ given by
$\langle f,g\rangle_D=\langle f,g\rangle_{L^2}+\sum_{i=1}^d\langle L_if,L_ig\rangle_{L^2}$, where
$\langle\cdot,\cdot\rangle_{L^2}$ is the $L^2(\Omega,P)$ inner product. 



Consider the Hilbert space $\mathscr{H}=\bigotimes_{i=1}^dL^2(\Omega,P)$ with inner
product $\langle\cdot,\cdot\rangle$ defined by $\langle\vec{\xi},\vec{\zeta}\,\rangle=\langle\vec{\xi}\cdot\vec{\zeta}\,\rangle$, where
$\vec{\xi}\cdot\vec{\zeta}=\vec{\xi}^{\;\,\dagger}\vec{\zeta}$ denotes the dot-product 
on $\mathbb{C}^d$ and $\langle\cdot\rangle$ means ensemble average over $\Omega$ or, by an
ergodic theorem \cite{Golden:CMP-473}, spatial average over all of
${\mathbb{R}}^d$. Define the Hilbert spaces \cite{Golden:CMP-473} of
``curl free'' $\mathscr{H}_\times$ and ``divergence free''
$\mathscr{H}_{\bullet}$ random fields  
%
\begin{align}\label{eq:curlfreeHilbert}
  &\mathscr{H}_\times=
  \left\{\vec{Y}\in \mathscr{H} \ | \ \vec{\nabla} \times\vec{Y}=0 \text{ weakly and }
    \langle\vec{Y}\rangle=0
  \right\}, \\
&\mathscr{H}_{\bullet}=
\left\{\vec{Y}\in \mathscr{H} \ | \ \vec{\nabla}\cdot\vec{Y}=0 \text{ weakly and }
    \langle\vec{Y}\rangle=0\right\},\notag 
\end{align}  
%
% \begin{align}\label{eq:curlfreeHilbert_Precise}
%   &\mathscr{H}_\times=
%   \left\{Y_i(\omega)\in L^2(\Omega,P), \ i=1,\ldots,d \ | \ L_iY_j-L_jY_i=0 \text{ weakly and }
%     \langle Y_i\rangle=0\right\},
%     \\
%   &\mathscr{H}_\bullet=
%   \left\{Y_i(\omega)\in L^2(\Omega,P), \ i=1,\ldots,d \ \Big| \ \sum_{i=1}^dL_iY_i=0 \text{ weakly and }
%     \langle Y_i\rangle=0\right\}.
%     \notag
% \end{align}
% %
% where $j=1,\ldots,d$.
%
where we have used the simplified notation $\langle\vec{Y}\rangle=0 \iff \langle Y_i\rangle=0$ for
all $i=1,\ldots,d$, $\vec{\nabla}\cdot\vec{Y}=\sum_{i=1}^dL_iY_i$, and $\vec{\nabla} \times\vec{Y}=0$
means $L_iY_j-L_jY_i=0$ for all $i,j=1,\ldots,d$. Consider the following
variational problems \cite{Golden:CMP-473}. Find
$\vec{E}_f\in\mathscr{H}_\times$ and $\vec{J}_f\in\mathscr{H}_\bullet$ such that     
%
\begin{align}
 \label{eq:Weak_Curl_Free_Variational_Form}
 &\langle\bsig(\vec{E}_0+\vec{E}_f)\cdot\vec{Y}\rangle=0 \quad
 \forall \  \vec{Y}\in\mathscr{H}_\times 
\\
 &\langle\brho(\vec{J}_0+\vec{J}_f)\cdot\vec{Y}\rangle=0 \quad
 \hspace{0.5em}\forall \ \vec{Y}\in\mathscr{H}_{\bullet}\,,
  \notag
\end{align}
%
respectively. When the bilinear forms
$\Psi(\vec{\xi},\vec{\zeta})=\bsig\,\vec{\xi}\cdot\vec{\zeta}\;$ and
$\Phi(\vec{\xi},\vec{\zeta})=\brho\,\vec{\xi}\cdot\vec{\zeta}\;$ are bounded and
coercive, these problems have unique solutions
\cite{Golden:CMP-473,Papanicolaou:RF-835} satisfying the quasi-static
limit of Maxwell's equations  \cite{Jackson-1999} 
%
\begin{align}   \label{eq:Maxwells_Equations_E}  
 &\vec{\nabla}\times\vec{E}=0, \quad
  \vec{\nabla}\cdot\vec{J}=0,\quad
  \vec{J}=\bsig\vec{E},\quad
  %\vec{E}=\vec{E}_0+\vec{E}_f, \quad
  \langle\vec{E}\,\rangle=\vec{E}_0, \\
%
  %\label{eq:Maxwells_Equations_D}
   &\vec{\nabla}\times\vec{E}=0, \quad
   \vec{\nabla}\cdot\vec{J}=0, \quad
   \vec{E}=\brho\vec{J},\quad
   %\vec{J}=\vec{J}_0+\vec{J}_f,\quad
   \langle\vec{J}\,\rangle=\vec{J}_0.
   \notag  
\end{align}
%
Here, $\vec{E}(\vec{x},\omega)=\vec{E}_0+\vec{E}_f(\vec{x},\omega)$ is the
random electric field, where $\vec{E}_f$ is the fluctuating field of
mean zero about the (constant) average $\vec{E}_0$. Similarly,
$\vec{J}(\vec{x},\omega)=\vec{J}_0+\vec{J}_f(\vec{x},\omega)$ is the random  
current density. Moreover, the components of $\vec{E}_f$ and
$\vec{J}_f$ are stationary random fields \cite{Golden:CMP-473}. 
% In component form equation \eqref{eq:Maxwells_Equations_E} is given by
% %
% \begin{align}   \label{eq:Maxwells_Equations_Component}  
%   \partial_{x_j}E_i-\partial_{x_i}E_j=0, \quad
%   \sum_{j=1}^d\partial_{x_j}D_j=0, \quad
%   D_j=\sum_{k=1}^d\sigma_{jk}E_k, \quad
% %  \vec{E}=\vec{E}_0+\vec{E}_f, \quad
%   \langle\vec{E}\,\rangle=\vec{E}_0, 
% \end{align}




As $\vec{E}_f\in\mathscr{H}_\times$ and $\vec{J}_f\in\mathscr{H}_\bullet$, equation
\eqref{eq:Weak_Curl_Free_Variational_Form} yields the energy (power)
\cite{Jackson-1999} constraints $\langle\vec{J}\cdot\vec{E}_f\rangle=0$ and
$\langle\vec{E}\cdot\vec{J}_f\rangle=0$, respectively, which leads to the following
reduced energy representations $\langle\vec{J}\cdot\vec{E}\rangle=\langle\vec{J}\,\rangle\cdot\vec{E}_0$
and $\langle\vec{E}\cdot\vec{J}\,\rangle=\langle\vec{E}\rangle\cdot\vec{J}_0$. The effective complex
conductivity and resistivity tensors, $\bsig^*$ and $\brho^*$, are
defined by   
%
\begin{align}\label{eq:eff_eps_def}
    \langle \vec{J} \,\rangle=  \bsig^* \vec{E}_0
    \\
    \langle \vec{E} \,\rangle=  \brho^*\vec{J}_0.
    \notag
\end{align}
%
Consequently, we have the following energy representations involving
the effective parameters 
%$\langle\vec{J}\cdot\vec{E}\rangle=\bsig^*\vec{E}_0\cdot\vec{E}_0=\brho^*\vec{J}_0\cdot\vec{J}_0$.
%
\begin{align}\label{eq:Energy_Reps}
  &\langle\vec{J}\cdot\vec{E}\rangle=\bsig^*\vec{E}_0\cdot\vec{E}_0,
  \\
  &\langle\vec{E}\cdot\vec{J}\;\rangle=\brho^*\vec{J}_0\cdot\vec{J}_0.
  \notag
\end{align}
%


We assume that the composite is a locally isotropic random medium so
that $\sigma_{jk}(\vec{x},\omega)=\sigma(\vec{x},\omega)\delta_{jk}$ and
$\rho_{jk}(\vec{x},\omega)=\rho(\vec{x},\omega)\delta_{jk}$, where $\delta_{jk}$ is the
Kronecker delta and $j,k=1,\ldots,d$. We further assume that the composite
is a two-component medium, so that $\sigma(\vec{x},\omega)$ takes the
\emph{complex} values $\sigma_1$ and $\sigma_2$, and $\rho(\vec{x},\omega)$ takes the
complex values $1/\sigma_1$ and $1/\sigma_2$, and satisfy \cite{Golden:CMP-473}
% 
\begin{align}\label{eq:two-phase_eps}
  \sigma(\vec{x},\omega)&=\sigma_1\chi_1(\vec{x},\omega)+\sigma_2\chi_2(\vec{x},\omega),
  \\
  \rho(\vec{x},\omega)&=\sigma_1^{-1}\chi_1(\vec{x},\omega)+\sigma_2^{-1}\chi_2(\vec{x},\omega).
  \notag
\end{align}
%
%$\sigma(\vec{x},\omega)=\sigma_1\chi_1(\vec{x},\omega)+\sigma_2\chi_2(\vec{x},\omega)$,
Here, $\chi_i(\vec{x},\omega)$ is the characteristic function of medium
$i=1,2$, which equals one for all $\omega\in\Omega$ having medium $i$ at $\vec{x}$
and zero otherwise, with $\chi_1=1-\chi_2$. For simplicity, we focus on one
component, $\sigma^*_{jk}=[\bsig^*]_{jk}$ and $\rho^*_{jk}=[\brho^*]_{jk}$, of
these symmetric tensors, for some $j,k=1,\ldots,d$.  



Due to the homogeneity of these functions,
e.g., $\sigma_{jk}^*(a\sigma_1,a\sigma_2)=a\sigma_{jk}^*(\sigma_1,\sigma_2)$ for any complex number
$a$, they depend only on the ratio $h=\sigma_1/\sigma_2$, and we define the
tensor-valued functions $\mathbf{m}(h)=\bsig^*/\sigma_2$,
$\mathbf{w}(z)=\bsig^*/\sigma_1$, $\tilde{\mathbf{m}}(h)=\sigma_1\brho^*$, and
$\tilde{\mathbf{w}}(z)=\sigma_2\brho^*$ with components  
%
\begin{align}\label{eq:m_h}
  &m_{jk}(h)=\sigma_{jk}^*/\sigma_2, \quad
  w_{jk}(z)=\sigma_{jk}^*/\sigma_1, \\
   &\tilde{m}_{jk}(h)=\sigma_1\rho_{jk}^*, \quad
   \hspace{0.5em}\tilde{w}_{jk}(z)=\sigma_2\rho_{jk}^*,
   \notag
\end{align}
%
where $z=1/h$. The dimensionless functions $m_{jk}(h)$ and
$\tilde{m}_{jk}(h)$ are analytic off the negative real axis in the
$h$-plane, while $w_{jk}(z)$ and $\tilde{w}_{jk}(z)$ are analytic off
the negative real axis in the $z$-plane \cite{Golden:CMP-473}. Each
take the corresponding upper half plane to the upper half plane and
are therefore examples of Herglotz functions
\cite{Deift:2000:RMT,Golden:CMP-473}.



A key step in the ACM is
obtaining Stieltjes integral representations for $\bsig^*$ and
$\brho^*$. These follow from resolvent representations for the
electric field $\vec{E}$ \cite{Golden:CMP-473} and current density
$\vec{J}$ \cite{Murphy:JMP:063506}
%\cite{Murphy:JMP:063506,Fannjiang:SIAM_JAM:333}         
%
\begin{align}\label{eq:Resolvent_representations_E_D}
  &\vec{E}=s(sI-\Gamma\chi_1)^{-1}\vec{E}_0=t(tI-\Gamma\chi_2)^{-1}\vec{E}_0 ,
  \quad
   s\in\mathbb{C}\backslash[0,1],\\
  &\vec{J}=t(tI-\Upsilon\chi_1)^{-1}\vec{J}_0=s(sI-\Upsilon\chi_2)^{-1}\vec{J}_0 ,
  \quad
   t\in\mathbb{C}\backslash[0,1],\notag 
\end{align}
%
where $I$ is the identity operator on $\mathbb{R}^d$ and we have
defined the complex variables $s=1/(1-h)$ and $t=1/(1-z)=1-s$. 
The operator $\Gamma=\vec{\nabla}(\Delta^{-1})\vec{\nabla}\cdot$ is based on
convolution with the free-space Green's function for the Laplacian
$\Delta=\vec{\nabla}\cdot\vec{\nabla}=\nabla^{\,2}$, and the operator
$\Upsilon=\vec{\nabla}\times(\vec{\nabla}\times\vec{\nabla}\times)^{-1}\,\vec{\nabla}\times$ involves the vector
Laplacian $\bDelta=- \vec{\nabla}\times\vec{\nabla}\times + \vec{\nabla}\vec{\nabla}\cdot $ when $d=3$
\cite{Golden:CMP-473,Murphy:JMP:063506}. These (non-random)
integro-differential operators and the origin of the resolvent
equations in \eqref{eq:Resolvent_representations_E_D} are discussed in
more detail below.       



If the current density $\vec{J}(\vec{x},\omega)$ and the electric
field $\vec{E}(\vec{x},\omega)$ are sufficiently smooth for all
$\vec{x}\in\mathbb{R}^d$ when $\omega\in\Omega$, equation 
\eqref{eq:Resolvent_representations_E_D} is obtained as
follows. The operator $\Delta^{-1}$ is
well defined in terms of convolution with respect to the free-space
Green's function of the Laplacian $\Delta$
\cite{Golden:CMP-473,Folland:95}. Similarly, the inverse
$\bDelta^{-1}$ of the vector Laplacian $\bDelta$ is defined in terms
of component-wise convolution with respect to the free-space Green's
function of the Laplacian.


Applying the integro-differential
operator $\vec{\nabla}(\Delta^{-1})$ to the
formula $\vec{\nabla}\cdot\vec{J}=0$ in equation
\eqref{eq:Maxwells_Equations_E} yields $\Gamma\vec{J}=0$, where
$\Gamma=\vec{\nabla}(\Delta^{-1})\vec{\nabla}\cdot$ is an orthogonal projection
\cite{Golden:CMP-473} from $\mathscr{H}$ onto the Hilbert space
$\mathscr{H}_\times$ of curl-free random fields,
$\Gamma:\mathscr{H}\mapsto\mathscr{H}_\times$. More specifically, for every
sufficiently smooth $\vec{\zeta}\in\mathscr{H}_\times$ there exists
\cite{Jackson-1999} a scalar potential $\varphi$ which is unique up to a 
constant such that $\vec{\zeta}=\vec{\nabla}\varphi$. Consequently, since
$\Delta=\vec{\nabla}\cdot\vec{\nabla}$, it is clear that $\Gamma\vec{\zeta}=\vec{\zeta}$ for all such
$\vec{\zeta}\in\mathscr{H}_\times$.  


For simplicity, we discuss only the analogous properties of divergence
free vector fields and the projection operator
$\Upsilon=\vec{\nabla}\times(\vec{\nabla}\times\vec{\nabla}\times)^{-1}\,\vec{\nabla}\times$, restricting our
attention to $d=3$ to avoid a more involved discussion regarding
differential forms \cite{Darling:Differential_Forms:1994}.
Applying the integro-differential
operator $-\vec{\nabla}\times(\bDelta^{-1})$ to the formula
$\vec{\nabla}\times\vec{E}=0$ in equation \eqref{eq:Maxwells_Equations_E} yields  
$\Upsilon\vec{E}=0$. Here, $\Upsilon=-\vec{\nabla}\times(\bDelta^{-1})\vec{\nabla}\times$ is an
orthogonal projection from $\mathscr{H}$ onto the Hilbert space 
$\mathscr{H}_{\bullet}$ of divergence-free random fields,
$\Upsilon:\mathscr{H}\mapsto\mathscr{H}_\bullet$, of transverse gauge
\cite{Murphy:JMP:063506}. This can be seen as follows. For every
sufficiently smooth $\vec{\zeta}\in\mathscr{H}_\bullet$ we have the representation
$\vec{\zeta}=\vec{\nabla}\times(\vec{A}+\vec{C})$, where $\vec{A}$ is a vector
potential associated with $\vec{\zeta}$ and the arbitrary vector field
$\vec{C}$ satisfies $\vec{\nabla}\times\vec{C}=0$ \cite{Jackson-1999}. Without
loss of generality, the vector field $\vec{C}$ can be chosen so that
$\vec{A}$ satisfies $\vec{\nabla}\cdot\vec{A}=0$ \cite{Jackson-1999}. Hence,
$\vec{\nabla}\times\vec{\zeta}=\vec{\nabla}\times\vec{\nabla}\times\vec{A}
=\vec{\nabla}(\vec{\nabla}\cdot\vec{A})-\bDelta\vec{A}=-\bDelta\vec{A}$. The vector field
$\vec{C}$ chosen in this manner gives the transverse \emph{gauge} of
$\vec{\zeta}$ \cite{Jackson-1999}. Choosing the members of 
$\mathscr{H}_\bullet$ to have transverse gauge, the action of
$\vec{\nabla}\times\vec{\nabla}\times$ on $\mathscr{H}_\bullet$ is given by that of
$-\bDelta$. Therefore, the action of $\Upsilon$ on $\mathscr{H}_\bullet$ is given
by that of 
%$\Upsilon=\vec{\nabla}\times(\vec{\nabla}\times\vec{\nabla}\times)^{-1}\vec{\nabla}\times=\vec{\nabla}\times(\bDelta^{-1})\vec{\nabla}\times$, 
%
\begin{align}\label{eq:GammaCurl}
  \Upsilon=\vec{\nabla}\times(\vec{\nabla}\times\vec{\nabla}\times)^{-1}\vec{\nabla}\times
  =-\vec{\nabla}\times(\bDelta^{-1})\vec{\nabla}\times, 
\end{align}
%
and it is clear from the above discussion that $\Upsilon\vec{\zeta}=\vec{\zeta}$ for
all such $\vec{\zeta}\in\mathscr{H}_\bullet$. In general, the differential
operators $\vec{\nabla}$, $\vec{\nabla}\cdot$, and $\vec{\nabla}\times$ are interpreted
in a weak sense in terms of the operators $L_i$ in \eqref{eq:Li}
\cite{Golden:CMP-473}. 






We now derive the resolvent formulas in equation
\eqref{eq:Resolvent_representations_E_D}.
Write $\sigma$ and $\rho$ in 
equation \eqref{eq:two-phase_eps} as $\sigma=\sigma_2(1-\chi_1/s)=\sigma_1(1-\chi_2/t)$ and
$\rho=(1-\chi_2/s)/\sigma_1=(1-\chi_1/t)/\sigma_2$. Recall that
$\vec{E}=\vec{E}_0+\vec{E}_f$, where $\vec{E}_0$ is a \emph{constant}
field and $\vec{E}_f\in\mathscr{H}_\times$ so that
$\Gamma\vec{E}=\vec{E}_f$, and similarly
$\Upsilon\vec{J}=\vec{J}_f$. Consequently, from $\Gamma\vec{J}=0$ and $\Upsilon\vec{E}=0$
we have the following formulas which are equivalent to that in 
\eqref{eq:Resolvent_representations_E_D}  
% 
\begin{align}\label{eq:Proj_rep_Ef_Jf}
  \vec{E}_f&=\frac{1}{s}\Gamma\chi_1\vec{E}=\frac{1}{t}\Gamma\chi_2\vec{E}, \\
  \vec{J}_f&=\frac{1}{s}\Upsilon\chi_2\vec{J}=\frac{1}{t}\Upsilon\chi_1\vec{J}.\notag
\end{align}
%




On the Hilbert space $\mathscr{H}_\times$, the operators $\Gamma$ and $\chi_i$,
$i=1,2$, act as projectors \cite{Golden:CMP-473}. Therefore 
$M_i=\chi_i\Gamma\chi_i$, $i=1,2$, are compositions of projection operators on
$\mathscr{H}_\times$, and are consequently positive definite and bounded by
1 in the underlying operator norm \cite{Rudin:87}. They are
self-adjoint with respect to the $\mathscr{H}$-inner-product $\langle\cdot,\cdot\rangle$
\cite{Golden:CMP-473}. Therefore, on the Hilbert space $\mathscr{H}_\times$
with weight $\chi_1$ in the inner-product, $\langle\cdot,\cdot\rangle_1=\langle\chi_1\,\cdot,\cdot\rangle$ for
example, $\Gamma\chi_1$ is a bounded linear self-adjoint operator with
spectrum contained in the interval $[0,1]$
\cite{Golden:CMP-473,Folland:95,Rudin:87}. Hence the resolvent
operator $(sI-\Gamma\chi_1)^{-1}$ in \eqref{eq:Resolvent_representations_E_D}
is also a linear self-adjoint operator with respect to the same
inner-product, and is bounded for  $s\in\mathbb{C}\backslash[0,1]$
\cite{Stone:64}. Similarly, $(tI-\Upsilon\chi_1)^{-1}$ in
\eqref{eq:Resolvent_representations_E_D} is a linear self-adjoint
operator on $\mathscr{H}_\bullet$ with respect to the inner-product
$\langle\cdot,\cdot\rangle_1$, and is bounded for $t\in\mathbb{C}\backslash[0,1]$.





To obtain integral representations for $\bsig^*$ and
$\brho^*$, it is more convenient to consider the functions 
$F_{jk}(s)=\delta_{jk}-m_{jk}(h)$ and $E_{jk}(s)=\delta_{jk}-\tilde{m}_{jk}(h)$
which are analytic off $[0,1]$ in the $s$-plane, and
$G_{jk}(t)=\delta_{jk}-w_{jk}(z)$ and $H_{jk}(t)=\delta_{jk}-\tilde{w}_{jk}(z)$
which are analytic off $[0,1]$ in the $t$-plane
\cite{Golden:CMP-473}. For the formulation of the effective parameter
problem involving $\mathscr{H}_\times$ and $\bsig^*$, define the
coordinate system so that in \eqref{eq:eff_eps_def} the constant
vector $\vec{E}_0$ is given by $\vec{E}_0=E_0\,\vec{e}_j$, where
$\vec{e}_j$ is the standard basis vector on $\mathbb{R}^d$ in the
$j^{\,\text{th}}$ direction for some $j=1,\ldots,d$. In the other
formulation involving $\mathscr{H}_\bullet$ and $\brho^*$, define
$\vec{J}_0=J_0\,\vec{e}_j$. Equations \eqref{eq:eff_eps_def} and 
\eqref{eq:Resolvent_representations_E_D}, and the spectral theorem for
bounded linear self-adjoint operators \cite{Reed-1980,Stone:64} then
yield the following Stieltjes integral representations
\cite{Golden:CMP-473,Bergman:PRC-377,Bergman:AP-78,Murphy:JMP:063506}  
for the effective parameters $\sigma^*_{jk}$ and $\rho^*_{jk}$ (see Sections
\ref{sec:Unification_finite_infinite} and
\ref{sec:The_Spectral_Theorem_Continuum} for more details) 
%
\begin{align}\label{eq:Stieltjes_F}
  &m_{jk}(h)=\delta_{jk}-F_{jk}(s), \qquad
  F_{jk}(s)=\langle\chi_1(sI-\Gamma\chi_1)^{-1}\vec{e}_j\cdot\vec{e}_k\rangle=\int_0^1\frac{\d\mu_{jk}(\lambda)}{s-\lambda}\,,
  \\
  &w_{jk}(z)=\delta_{jk}-G_{jk}(t), \qquad
  G_{jk}(t)=\langle\chi_2(tI-\Gamma\chi_2)^{-1}\vec{e}_j\cdot\vec{e}_k\rangle=\int_0^1\frac{\d\alpha_{jk}(\lambda)}{t-\lambda}\,,
  \notag \\
  &\tilde{m}_{jk}(h)=\delta_{jk}-E_{jk}(s), \qquad
  E_{jk}(s)=\langle\chi_2(sI-\Upsilon\chi_2)^{-1}\vec{e}_j\cdot\vec{e}_k\rangle=\int_0^1\frac{\d\eta_{jk}(\lambda)}{s-\lambda}\,,
  \notag \\
  &\tilde{w}_{jk}(z)=\delta_{jk}-H_{jk}(t), \qquad
  H_{jk}(t)=\langle\chi_1(tI-\Upsilon\chi_1)^{-1}\vec{e}_j\cdot\vec{e}_k\rangle=\int_0^1\frac{\d\kappa_{jk}(\lambda)}{t-\lambda}\,.
  \notag
\end{align}
%



Equation \eqref{eq:Stieltjes_F} displays Stieltjes integrals involving
\emph{spectral measures of random operators}. More specifically,
$\d\mu_{jk}(\lambda)$ and $\d\alpha_{jk}(\lambda)$ are spectral measures associated 
with the random operators $\chi_1\Gamma\chi_1$ and $\chi_2\Gamma\chi_2$, while
$\d\eta_{jk}(\lambda)$ and $\d\kappa_{jk}(\lambda)$ are spectral measures associated
with the random operators $\chi_2\Upsilon\chi_2$ and $\chi_1\Upsilon\chi_1$, respectively. In
particular, there is a one-to-one correspondence between the bounded 
linear operator $\chi_1\Gamma\chi_1$ on $\mathscr{H}_\times$, for example, and a
family of projection operators $Q(\lambda)$, parameterized by $\lambda\in[0,1]$, which
satisfies $\lim_{\lambda\to0}Q(\lambda)=0$ and $\lim_{\lambda\to1}Q(\lambda)=I$, where $0$ and $I$
are the null and identity operators on $\mathscr{H}_\times$, respectively
\cite{Stone:64}. The strictly increasing function
$\mu_{jk}(\lambda)=\langle Q(\lambda)\vec{e}_j,\vec{e}_k\rangle_1$ of the spectral variable $\lambda$ is
of bounded variation \cite{Stone:64}. The spectral measure
$\d\mu_{jk}(\lambda)$ is a \emph{Stieltjes measure} \cite{Folland:99}
associated with the function $\mu_{jk}(\lambda)$ \cite{Stone:64} (see Section
\ref{sec:The_Spectral_Theorem_Continuum} for more details). For
notational simplicity, we will often refer to the measure $\mu_{jk}$,
not to be confused with the function $\mu_{jk}(\lambda)$.  


By the Stieltjes--Perron inversion theorem
\cite{Henrici:1974:v2,MILTON:2002:TC}, the matrix valued function
$\bmu(\lambda)$ with components $\mu_{jk}(\lambda)$, $j,k=1,\ldots,d$, for example, is
given by the weak limit  
$\bmu(\lambda)=-(1/\pi)\lim_{\varepsilon\downarrow0}\text{Im}(\mathbf{F}(\lambda+\imath\varepsilon))$, i.e., 
%
\begin{align}\label{eq:weak_limit_mu}
  \int_0^1\xi(\lambda)\;\d\bmu(\lambda)
  =-\frac{1}{\pi}\lim_{\varepsilon\downarrow0}
        \int_0^1\xi(\lambda)\;\text{Im}(\mathbf{F}(\lambda+\imath\varepsilon))\, \d\lambda,
\end{align}
%
for all smooth test functions $\xi(\lambda)$, where
$[\mathbf{F}(s)]_{jk}=F_{jk}(s)$ and
$[\d\bmu(\lambda)]_{jk}=\d\mu_{jk}(\lambda)$. From equation 
\eqref{eq:weak_limit_mu} and the identities 
$m_{jk}(h)=h\,w_{jk}(z)$ and $\tilde{m}_{jk}(h)=h\,\tilde{w}_{jk}(z)$,
which follow from equation \eqref{eq:m_h}, it has been
shown \cite{Murphy:JMP:063506} that the functions $\mu_{jk}(\lambda)$ and
$\alpha_{jk}(\lambda)$, and the functions $\eta_{jk}(\lambda)$ and $\kappa_{jk}(\lambda)$ are
related by
%
\begin{align}\label{eq:Measure_Relations}
  &\lambda\alpha_{jk}(\lambda)=(1-\lambda)\mu_{jk}(1-\lambda)+\lambda\varrho(\lambda), \qquad
  \d\varrho(\lambda)=m_{jk}(0)\delta_0(\d\lambda)+w_{jk}(0)(\lambda-1)\delta_1(\d\lambda),
  \notag     \\
  &\lambda\kappa_{jk}(\lambda)=(1-\lambda)\eta_{jk}(1-\lambda)+\lambda\tilde{\varrho}(\lambda), \qquad
  \d\tilde{\varrho}(\lambda)=\tilde{m}_{jk}(0)\delta_0(\d\lambda)+\tilde{w}_{jk}(0)(\lambda-1)\delta_1(\d\lambda).  
\end{align}
%
% %
% \begin{align}%\label{eq:Measure_Relations}
%   &\lambda\alpha_{jk}(\lambda)=(1-\lambda)\mu_{jk}(1-\lambda)+ 
%        \lambda\,(\,m_{jk}(0)\delta_0(\lambda)+w_{jk}(0)(\lambda-1)\delta_1(\lambda)\,),
%   \\
%   &\lambda\kappa_{jk}(\lambda)=(1-\lambda)\eta_{jk}(1-\lambda) +
%        \lambda\,(\,\tilde{m}_{jk}(0)\delta_0(\lambda)+\tilde{w}_{jk}(0)(\lambda-1)\delta_1(\lambda)\,).
%   \notag     
% \end{align}
% %
Here, $m(0)=m(h)|_{h=0}$ and $w(0)=w(z)|_{z=0}$, for example, and
$\delta_a(\d\lambda)$ is the delta measure concentrated at $\lambda=a$.



Equations
\eqref{eq:Stieltjes_F} and \eqref{eq:Measure_Relations} demonstrate
the many symmetries between 
the functions $m_{jk}(h)$, $w_{jk}(z)$, $\tilde{m}_{jk}(h)$, and
$\tilde{w}_{jk}(z)$, and the respective measures $\mu_{jk}$, $\alpha_{jk}$,
$\eta_{ij}$, and $\kappa_{jk}$. Because of these symmetries, for simplicity,
we will focus on $m_{jk}(h)$ and $\mu_{jk}$, and will reintroduce the
other functions and measures where appropriate.  



A key feature of equations \eqref{eq:eff_eps_def}, \eqref{eq:m_h}, and
\eqref{eq:Stieltjes_F} is that the parameter information in $h$ and
$E_0$ is \emph{separated} from the geometry of the composite, which is
encoded in the spectral measure $\mu_{jk}$ via its moments $\mu_{jk}^n$
\cite{Golden:CMP-473,Bruno:JSP-365}
%$\mu_{jk}^n=\int_0^1\lambda^n\d\mu_{jk}(\lambda)$, $n=0,1,2,\ldots$.
%
\begin{align}\label{eq:Moments_mu}
  \mu_{jk}^n=\int_0^1\lambda^n\d\mu_{jk}(\lambda)       
       %=\langle[\Gamma \chi_1]^n\,\vec{e}_j\cdot\vec{e}_k\rangle_1,
       =\langle\chi_1[\Gamma \chi_1]^n\,\vec{e}_j\cdot\vec{e}_k\rangle,
  \quad n=0,1,2,\ldots.,
\end{align}
%
where the second equality follows from the spectral theorem displayed
in equation \eqref{eq:Spectral_Theorem} with $f(\lambda)=\lambda^n$. Since $\chi_1$
operates pointwise on $\mathbb{R}^d$ and the constant vectors
$\vec{e}_j$, $j=1,\ldots,d$, are non-random, we see from equation
\eqref{eq:Moments_mu} that the mass $\mu_{jk}^0$ of the measure
$\mu_{jk}$ is given by
%$\mu_{jk}^0=p_1\delta_{jk}$,
%
\begin{align}\label{eq:Measure_Mass_ConT}
  \mu_{jk}^0=p_1\delta_{jk},
\end{align}
%
where $p_1=\langle\chi_1\rangle$ is the volume fraction of material component
one. This demonstrates that the diagonal components $\mu_{kk}$,
$k=1,\ldots,d$, of $\bmu$ are \emph{positive measures}, 
while the off-diagonal components $\mu_{jk}$, $j\neq k=1,\ldots,d$, have 
zero mass and are consequently \emph{signed measures}
\cite{Folland:99,Rudin:87}. The positivity of the measure $\mu_{kk}$ also
follows from the fact that $Q(\lambda)$ is a \emph{self-adjoint projector}
on $\mathscr{H}_\times$ so that  $\langle Q(\lambda)\vec{e}_k\cdot\vec{e}_k\rangle_1=\langle Q(\lambda)\vec{e}_k\cdot
Q(\lambda)\vec{e}_k\rangle_1=\|Q(\lambda)\vec{e}_k\|_1^2$ is a strictly increasing 
function of $\lambda$ \cite{Reed-1980,Stone:64}. Therefore, the measure of
an arbitrary set $A\subseteq[0,1]$ is positive:
% 
\begin{align}\label{eq:Mass_Sign_Measures}   
   &\mu_{kk}(A)=\int_Ad\mu_{kk}(\lambda)
           %=\int_Ad\langle Q(\lambda)\vec{e}_k\cdot\vec{e}_k\rangle_1
           %=\int_Ad\langle Q(\lambda)\vec{e}_k\cdot Q(\lambda)\vec{e}_k\rangle_1
           =\int_Ad\|Q(\lambda)\vec{e}_k\|_1^2\geq0,
\end{align}
%
where $\|\cdot\|_1$ denotes the norm induced by the inner-product
$\langle\cdot,\cdot\rangle_1$. 



The higher order moments $\mu_{jk}^n$, $n=1,2,3,\ldots$, in principle, may be
found using a perturbation expansion of $F_{jk}(s)$ about a
homogeneous medium $(\sigma_1=\sigma_2, \ s=\infty)$ \cite{Golden:CMP-473}. In
particular $\mu_{jk}^0=p_1\delta_{jk}$, generically, and $\mu_{jk}^1=(p_1p_2/d)\,\delta_{jk}$
for a statistically isotropic random medium 
\cite{Golden:CMP-473,Golden:IMA-97,Bruno:JSP-365}, where
$p_2=1-p_1=\langle\chi_2\rangle$ is the volume fraction of material component 2. In
the case of a square bond lattice, which is an example of an
infinitely interchangeable random medium \cite{Bruno:JSP-365},
$\mu_{kk}^2=p_1p_2(1+(d-2)p_2)/d^{\,2}$ for any dimension $d$ and
$\mu_{kk}^3=p_1p_2(p_2^2-p_2-1)/8$ for $d=2$. In general, the moments
$\mu_{jk}^n$ depend on the $(n+1)$-point correlation functions of the
random medium \cite{Golden:CMP-473,Bruno:JSP-365}.




A principal application of the ACM is to derive \emph{forward bounds}
on the diagonal components $\sigma_{kk}^*$ of the tensor $\bsig^*$,
$k=1,\ldots,d$, given partial information on the microgeometry
\cite{Bergman:PRL-1285,Milton:APL-300,Golden:CMP-473,Bergman:AP-78}. This
information may be given in terms of the moments $\mu_{kk}^n$,
$n=0,1,2,\ldots$, of the measure $\mu_{kk}$
\cite{Milton:JAP-5294,Golden:CMP-473}. Given this information, the 
bounds on $\sigma_{kk}^*$ follow from the special structure of $F_{kk}(s)$
in \eqref{eq:Stieltjes_F}. More specifically, it is a \emph{linear}
functional of the \emph{positive} measure $\mu_{kk}$.  The bounds are
obtained by fixing the contrast parameter $s$, varying over an
admissible set of measures $\mu_{kk}$ (or geometries) which is
determined by the known information regarding the two-component
composite.  Knowledge of the moments $\mu_{kk}^n$  for $n=1,\ldots,J$ confines
$\sigma_{kk}^*$ to a region of the complex plane which is bounded by arcs
of circles, and the region becomes progressively smaller as more
moments are known \cite{Milton:JAP-5294,Golden:JMPS-333}. When 
all the moments are known, the measure $\mu_{kk}$ is uniquely determined 
\cite{Akhiezer:Book:1965}, hence  $\sigma_{kk}^*$ is explicitly known. The
bounding procedure is reviewed in Section
\ref{sec:Bounding_Procedure}.



We conclude this section with a discussion regarding some 
consequences of the energy constraints
$\langle\vec{J}\cdot\vec{E}_f\rangle=0=\langle\vec{E}\cdot\vec{J}_f\rangle$, which follow from
equation \eqref{eq:Weak_Curl_Free_Variational_Form}, and are at the
heart of the existence and uniqueness of solutions to the formulas of equation
\eqref{eq:Maxwells_Equations_E}. We first note that the formulas
$\Gamma\vec{E}=\vec{E}_f$ and $\Upsilon\vec{J}=\vec{J}_f$ are sufficient
conditions for these constraints. The sufficiency of these conditions
can be seen by writing $\sigma=\sigma_2(1-\chi_1/s)$ and $\rho=(1-\chi_1/t)/\sigma_2$ in
$\vec{J}=\sigma\vec{E}$ and $\vec{E}=\rho\vec{J}$, respectively, to obtain           
% 
\begin{align}\label{eq:Field_Rep_s}
  \langle\vec{J}\cdot\vec{E}_f\rangle=\sigma_2(\langle\vec{E}\cdot\vec{E}_f\rangle-\langle\chi_1\vec{E}\cdot\vec{E}_f\rangle/s),
 \quad
  \langle\vec{E}\cdot\vec{J}_f\rangle=(\langle\vec{J}\cdot\vec{J}_f\rangle-\langle\chi_1\vec{J}\cdot\vec{J}_f\rangle/t)/\sigma_2, 
\end{align}
%
for $s\neq0$ $(h\neq+\infty)$ and $t\neq0$ $(h\neq0)$. Now, if we have $\Gamma\vec{E}=\vec{E}_f$ then 
$\vec{\nabla}\cdot\vec{J}=0$ yields the formula $\vec{E}_f=\Gamma\chi_1\vec{E}/s$ of equation
\eqref{eq:Proj_rep_Ef_Jf}. Therefore, as $\Gamma$ is a self-adjoint 
operator on $\mathscr{H}$ \cite{Stakgold:BVP:2000,Stone:64,Folland:95}, we have     
%
\begin{align}\label{eq:Suff_Cond}
  \langle\chi_1\vec{E}\cdot\vec{E}_f\rangle=\langle\chi_1\vec{E}\cdot\Gamma\vec{E}\rangle
                    =\langle\Gamma\chi_1\vec{E}\cdot\vec{E}\rangle
                    =s\langle\vec{E}_f\cdot\vec{E}\rangle.
 %  \langle\chi_1\vec{J}\cdot\vec{J}_f\rangle=\langle\chi_1\vec{J}\cdot\Upsilon\vec{J}\rangle
%                     =\langle\Upsilon\chi_1\vec{J}\cdot\vec{J}\rangle
%                     =t\langle\vec{J}_f\cdot\vec{J}\rangle.                        
\end{align}
%
Consequently, from equation \eqref{eq:Field_Rep_s}
we have $\langle\vec{J}\cdot\vec{E}_f\rangle=0$ for $s\neq0$. The argument involving the
operator $\Upsilon$ and the vector field $\vec{J}_f$ is analogous.


 We see from equation \eqref{eq:Field_Rep_s} that the energy
 constraints are equivalent to the following ``field representations''
 for the contrast parameters $s$ and $t$
%
\begin{align}\label{eq:Field_Rep_s_t}
  \langle\chi_1\vec{E}\cdot\vec{E}_f\rangle/\langle\vec{E}\cdot\vec{E}_f\rangle
  =s=1-t
  =1- \langle\chi_1\vec{J}\cdot\vec{J}_f\rangle/\langle\vec{J}\cdot\vec{J}_f\rangle,
\end{align}
%
%$s=\langle\chi_1\vec{E}\cdot\vec{E}_f\rangle/\langle\vec{E}\cdot\vec{E}_f\rangle$,
when $\langle\vec{E}\cdot\vec{E}_f\rangle\neq0$ (if and only if
$\langle\chi_1\vec{E}\cdot\vec{E}_f\rangle\neq0$ when $s\neq0$ from \eqref{eq:Suff_Cond}), for
example. Moreover, the energy constraints provide the limiting
behavior of the ratio
$\mathcal{R}(h)=\langle\vec{E}\cdot\vec{E}_f\rangle/\langle\chi_1\vec{E}\cdot\vec{E}_f\rangle=1/s$, for example,    
%
\begin{align}
  \lim_{h\to0}\mathcal{R}(h)=1, \quad
  \lim_{h\to1}\mathcal{R}(h)=0, \quad
  \lim_{h\to+\infty}\mathcal{R}(h)=-\infty,
\end{align}
%
which is otherwise a very complicated object in the absence of these
energy constraints.
We also note that equation \eqref{eq:Field_Rep_s_t} provides
a relationship between the members $\vec{E}_f$ and $\vec{J}_f$ of the
Hilbert spaces $\mathscr{H}_\times$ and $\mathscr{H}_{\bullet}$,
respectively. 

The energy constraints also lead to detailed decompositions of the
system energy $\langle\vec{J}\cdot\vec{E}\rangle$ in terms of Herglotz functions
involving the measures $\mu_{jj}$, $\alpha_{jj}$, $\eta_{jj}$, and $\kappa_{jj}$
\cite{Murphy:JMP:063506,Murphy:PHD_Thesis}. For 
example, $\langle\vec{J}\cdot\vec{E}_f\rangle=0$, $\vec{E}=\vec{E}_0+\vec{E}_f$, 
$\vec{E}_0=E_0\vec{e}_j$, $\langle\vec{E}_f\,\rangle=0$, and $\sigma=\sigma_2(1-\chi_1/s)$ together
imply that $0=\langle\sigma\vec{E}\cdot\vec{E}_f\rangle=\langle\sigma_2(1-\chi_1/s)(\vec{E}_f\cdot\vec{E}_0+E_f^2)\rangle 
=\sigma_2\left[\langle E_f^2\rangle- (\langle\chi_1\vec{E}_f\cdot\vec{E}_0\rangle+\langle\chi_1E_f^2\rangle)/s\right].$
% %
% \begin{align}
%   0=\langle\sigma\vec{E}\cdot\vec{E}_f\rangle=\langle\sigma_2(1-\chi_1/s)(\vec{E}_0\cdot\vec{E}_f+E_f^2)\rangle
%  =\sigma_2\left(\langle E_f^2\rangle- \frac{1}{s}\left(\langle\chi_1\vec{E}_0\cdot\vec{E}_f\rangle
%      + \langle\chi_1E_f^2\rangle\right)\right).
% \end{align}
% %
Equations \eqref{eq:Resolvent_representations_E_D} and
\eqref{eq:Stieltjes_F}, and the spectral theorem \cite{Reed-1980} then
yield \cite{Murphy:JMP:063506}  
%
\begin{align}\label{eq:Herglotz_energy_Reps}
 \frac{\langle E_f^2\rangle}{E_0^2}=\int_0^1 \frac{\lambda\,\d\mu_{jj}(\lambda)}{(s-\lambda)^2}
           =\int_0^1 \frac{\lambda\,\d\alpha_{jj}(\lambda)}{(t-\lambda)^2}\,. 
\end{align}
%
Equation \eqref{eq:Herglotz_energy_Reps}, in turn, leads to Herglotz
representations of all such energy components involving
these measures \cite{Murphy:PHD_Thesis}. Analogous energy
decompositions involving $\vec{J}_f$ and the measures $\eta_{jj}$ and
$\kappa_{jj}$ similarly follow. In \cite{Murphy:PHD_Thesis} this
energy decomposition has lead to a physically transparent statistical
mechanics model of two-phase dielectric media.


\subsection{Lattice Setting}
\label{sec:Lattice_Setting}
%
In this section, we formulate the effective parameter problem for the
infinite and finite, two-component bond lattice on $\mathbb{Z}^d$
(formulations for other lattice topologies are analogous). The
infinite bond lattice, reviewed in Section
\ref{sec:Infinite_Lattice_Setting}, is a special case of the
stationary random medium considered in Section
\ref{sec:Continuum_Setting}. In Section
\ref{sec:Finite_Lattice_Setting}, we develop a mathematical framework
for the ACM in the finite lattice setting, a key theoretical
contribution of this work. 



\subsubsection{Infinite Lattice Setting}
\label{sec:Infinite_Lattice_Setting}
%
Consider a two-component bond lattice on all of $\mathbb{Z}^d$
determined by the probability space $(\Omega,P)$, and let
$\bsig(\vec{x},\omega)$ be the local complex conductivity tensor with
components $\sigma_{jk}(\vec{x},\omega)=\sigma^j(\vec{x},\omega)\delta_{jk}$, $j,k=1,\ldots,d$. Here,
$\sigma^j(\vec{x},\omega)$ is the conductivity of the bond emanating from 
$\vec{x}\in\mathbb{Z}^d$ in the positive $j^{\,\text{th}}$ direction for
$\omega\in\Omega$, which is a stationary random field that takes the \emph{complex} values
$\sigma_1$ and $\sigma_2$ with probabilities $p_1$ and $p_2=1-p_1$,
respectively \cite{Golden:CMP-467,Bruno:JSP-365}. The configuration
space $\Omega=\{\sigma_1,\sigma_2\}^{d\mathbb{Z}^d}$ represents the set of all
realizations of the random medium and the 
probability measure $P$ is compatible with stationarity. Analogous to
equation \eqref{eq:two-phase_eps}, the local conductivity
$\sigma^j(\vec{x},\omega)$ of the two-phase random medium takes the form
\cite{Golden:CMP-467} 
%
\begin{align}\label{eq:two-phase_sigma}
  \sigma^j(\vec{x},\omega)=\sigma_1\chi_1^j(\vec{x},\omega)+\sigma_2\chi_2^j(\vec{x},\omega), \quad j=1,\ldots,d.
\end{align}
%
Here, $\chi_i^j(\vec{x},\omega)$ is the characteristic function of medium
$i=1,2$, which equals one for all realizations $\omega\in\Omega$ having medium $i$
in the $j^{\,\text{th}}$ positive bond at $\vec{x}$, and equals zero
otherwise.




In this lattice setting, the differential operators $\vec{\nabla}\times$ and
$\vec{\nabla}\cdot$ in equation \eqref{eq:Maxwells_Equations_E} are given 
\cite{Golden:CMP-467,Bruno:JSP-365} in terms of forward and backward
difference operators $D_j^+$ and $D_j^-$, respectively, where
%
\begin{align}\label{eq:Difference_Operators}
  D_j^+=T_j^+-I, \quad D_j^-=I-T_j^-, \quad j=1,\ldots,d.
\end{align}
%
Here, $I$ is the identity operator on $\mathbb{Z}^d$, and 
$T_j^+=T_{+e_j}$ and $T_j^-=T_{-e_j}$ are the generators (through 
composition) of the unitary group $T_x$ acting on $L^2(\Omega,P)$ defined
by $(T_xf)(0,\omega)=f(\vec{x},\omega)$, for any $f\in L^2(\Omega,P)$
%$f:\mathbb{Z}^d\times\Omega\to\mathbb{R}$,
which is a stationary random field
\cite{Golden:CMP-467}. Define $\mathscr{H}=\bigotimes_{i=1}^dL^2(\Omega,P)$ and let
$\vec{E},\vec{J}\in \mathscr{H}$ be  
the random electric field and current density,
respectively, where $\vec{E}(\vec{x},\omega)=(E^1(\vec{x},\omega),\ldots
E^d(\vec{x},\omega))$ and $E^j(\vec{x},\omega)$ is the electric field in the
bond emanating from $\vec{x}$ in the positive $j^{\,\text{th}}$
direction, and similarly for $\vec{J}(\vec{x},\omega)$. 


As in Section \ref{sec:Continuum_Setting} we write
$\vec{E}=\vec{E}_0+\vec{E}_f$, where $\vec{E}_f$ is the fluctuating
field of mean zero about the (constant) average $\vec{E}_0$. 
The variational problem in \eqref{eq:Weak_Curl_Free_Variational_Form}
for this lattice setting has a unique solution satisfying Kirchhoff's
circuit laws \cite{Golden:CMP-473,Bruno:JSP-365}      
%
\begin{align}\label{eq:Kirchhiff's__Equations}
  D_i^+E^j-D_j^+E^i=0, \quad
  \sum_{k=1}^dD_k^-J^k=0, \quad
  J^i=\sigma^iE^i, \quad
%  \vec{E}=\vec{E}_0+\vec{E}_f, \quad
  \langle\vec{E}\rangle=\vec{E}_0,
\end{align}
%
where $i,j=1,\ldots,d$ and the components $E^i(\vec{x},\omega)$ and
$J^i(\vec{x},\omega)$ of $\vec{E}(\vec{x},\omega)$ and $\vec{J}(\vec{x},\omega)$ are
stationary random fields.  
Equation \eqref{eq:Kirchhiff's__Equations} is a direct analogue of
equation \eqref{eq:Maxwells_Equations_E} when written in component
form \cite{Golden:CMP-473}. The effective complex 
conductivity tensor $\bsig^*$ is defined by
$\langle\vec{J}\rangle=\bsig^*\vec{E}_0$, and has components
$\sigma^*_{jk}=\sigma_2\,m_{jk}(h)$, $j,k=1,\ldots,d$, where $h=\sigma_1/\sigma_2$. The
representation formula for $m_{jk}(h)$ in \eqref{eq:Stieltjes_F}
still holds in this infinite lattice setting,  
with $\Gamma$ in \eqref{eq:Resolvent_representations_E_D} now given by 
%
\begin{align}\label{eq:Discrete_Gamma}
  \Gamma=\nabla^+(\Delta^{-1})\nabla^-, \quad \nabla^\pm = (D_1^\pm,\ldots,D_d^\pm),
\end{align}
%
where $\Delta^{-1}$ is based on discrete convolution with the lattice
Green's function for the Laplacian $\Delta=\nabla^-\nabla^+$ \cite{Bruno:JSP-365}. The
formulation of the ACM for the effective resistivity tensor $\brho^*$
in the infinite lattice setting is analogous to that for $\bsig^*$
given here. In Section \ref{sec:Finite_Lattice_Setting} we discuss in
detail the operator $\Upsilon$ underlying the integral representations for
$\brho^*$ in the lattice setting. 






\subsubsection{Finite Lattice Setting}
\label{sec:Finite_Lattice_Setting}
%
Consider a finite, two-component bond lattice on
$\mathbb{Z}_L^d\subset\mathbb{Z}^d$ determined by the probability space
$(\Omega,P)$, where
%
\begin{align}\label{eq:ZLd}
  \mathbb{Z}_L^d=\{\vec{x}\in\mathbb{Z}^d \ | \ 1\leq x_i\leq L, \ i=1,\ldots,d\},   
\end{align}
%
$L\in\mathbb{N}$, $L\geq2$, and $x_i=(\vec{x}\,)_i$ is the $i^{\,\text{th}}$
component of the vector $\vec{x}$. Let $\bsig(\vec{x},\omega)$ be the local
complex conductivity tensor with components
$\sigma_{jk}(\vec{x},\omega)=\sigma^j(\vec{x},\omega)\delta_{jk}$, $j,k=1,\ldots,d$, where
$\sigma^j(\vec{x},\omega)$ is defined in equation \eqref{eq:two-phase_sigma} for
$\vec{x}\in \mathbb{Z}_L^d$ and $\omega\in\Omega$. The configuration space
$\Omega=\{\sigma_1,\sigma_2\}^{d\mathbb{Z}_L^d}$ represents the set of all $2^N$
realizations of the finite random bond lattice, where $N=d\,L^d$ and
$P$ is the associated (discrete) probability measure. Define
$\mathscr{H}=\bigotimes_{i=1}^dL^2(\Omega,P)$ and let
$\vec{E},\vec{J}\in \mathscr{H}$ be the random
electric field and current density, respectively, which satisfy
Kirchhoff's circuit laws \eqref{eq:Kirchhiff's__Equations} with
appropriate boundary conditions. Analogous to equation 
\eqref{eq:eff_eps_def}, the effective complex conductivity tensor
$\bsig^*$ is defined by $\langle\vec{J}\rangle=\bsig^*\vec{E}_0$, and
has components $\sigma_{jk}^*=\sigma_2m_{jk}(h)$, where $\vec{E}_0=\langle\vec{E}\rangle$ and
$\langle\cdot\rangle$ denotes ensemble average over $\Omega$. In a similar way we define
the functions $\sigma_{jk}^*=\sigma_1w_{jk}(z)$ and
$\rho^*_{jk}=\sigma_1\tilde{m}_{jk}(h)=\sigma_2\tilde{w}_{jk}(z)$ introduced in
Section \ref{sec:Continuum_Setting}.  




In this section, we obtain discrete versions of the integral
representations for $m_{jk}(h)$ and $\tilde{w}_{jk}(z)$ in equation 
\eqref{eq:Stieltjes_F} for this finite bond lattice setting, involving
spectral measures $\mu_{jk}$ and $\kappa_{jk}$ associated with real-symmetric
random matrices. The formulation involving the functions
$\tilde{m}_{jk}(h)$ and $w_{jk}(z)$ are analogous. Toward this goal,
we define a bijective mapping $\Theta$ from the $d$-dimensional set
$\mathbb{Z}_L^d$ onto the one dimensional set
$\mathbb{N}_L\subset\mathbb{N}$, $\Theta:\mathbb{Z}_L^d\to\mathbb{N}_L$, given by
% 
\begin{align}\label{eq:Bijection_Z_N}
  \mathbb{N}_L=\{i\in\mathbb{N} \ | \ i\leq dL^d\}, \qquad
  \Theta(\vec{x}\,)=x_1+\sum_{k=2}^d(x_k-1)L^{k-1}.
  %\quad x_k=1,\ldots,L, \ \forall \ k.
\end{align}
%


Under the bijection $\Theta$ the components
$E^j(\vec{x},\omega)$, $j=1,\ldots,d$, of the random electric field
$\vec{E}(\vec{x},\omega)=(E^1(\vec{x},\omega),\ldots ,E^d(\vec{x},\omega))$ are mapped to
vector valued functions $E^j(\vec{x},\omega)\mapsto\vec{E}^j(\omega)=(E^j_1(\omega),\ldots,
E^j_{L^d}(\omega))$ so that
% 
\begin{align}\label{eq:bijection_vector_mapping}
  \Theta(\vec{E}(\vec{x},\omega))=(\vec{E}^1(\omega),\ldots ,\vec{E}^d(\omega))\in\mathbb{C}^N,
\end{align}
%
%$\vec{E}(\vec{x},\omega)\mapsto(\vec{E}^1(\omega),\ldots \vec{E}^d(\omega))\in\mathbb{R}^N$,
for each $\omega\in\Omega$, and similarly for $\vec{J}(\vec{x},\omega)$. Moreover, the
bijection $\Theta$ maps the standard basis vector
$\vec{e}_1=(1,0,\ldots,0)\in\mathbb{Z}^d$, for example, to the
vector $(\vec{1},\vec{0},\ldots,\vec{0})\in\mathbb{Z}^N$, where
$\vec{1}$ and $\vec{0}$ are vectors of ones and zeros of length
$L^d$, respectively, and similarly for the $\vec{e}_j$ for
$j=2,\ldots,d$. Therefore, the vectors $\hat{e}_i$, $i=1,\ldots,d$, satisfying
%
\begin{align}\label{eq:Lattice_Basis_e}
  \hat{e}_i=\Theta(\vec{e}_i)/L^{d/2},
  \qquad
  \hat{e}_i\cdot\hat{e}_j=\delta_{ij},
\end{align}
%
serve as the standard basis vectors on $\mathbb{N}_L$.
% In view of equation
% \eqref{eq:bijection_vector_mapping} it is natural to decompose the
% set $\mathbb{N}_L$ as follows
% %
% \begin{align}\label{eq:decompose}
%   \mathbb{N}_L=\cup_{j=1}^d\mathbb{N}_L^j, \quad
%   \mathbb{N}_L^j=\{i\in\mathbb{N}_L \ | \ (j-1)L+1\leq i\leq jL\}, \quad
%   j=1,\ldots,d. 
% \end{align}
% %


On $\mathbb{N}_L$ the difference operators $D_j^\pm$, $j=1,\ldots,d$, in
equation \eqref{eq:Difference_Operators} are given in terms of finite 
difference matrices $D_j$ \cite{Demmel:1997}, where the rows of $D_j$
correspond to the bonds of the lattice, the columns
correspond to the nodes, and the numbering of the nodes on
$\mathbb{N}_L$ is determined by the bijection $\Theta$ in
\eqref{eq:Bijection_Z_N}. In this finite lattice 
setting, the Laplacian $\Delta$ and the projection operator $\Gamma$ in
\eqref{eq:Discrete_Gamma} are replaced by the real-symmetric matrices
$\Delta=\nabla^{\;T}\nabla$ and $\Gamma=\nabla(\Delta^{-1})\nabla^{\;T}$, respectively, where
$\nabla^{\;T}=(D_1^T,\ldots,D_d^T)$. The matrices $\Delta$ and $\Gamma$ depend only on the
topology and the boundary conditions of the underlying finite bond
lattice $\mathbb{Z}_L^d$, and $\Gamma$ is a projection matrix
satisfying $\Gamma^{\,2}=\Gamma$. 




The matrix $\Gamma$ is invariant under arbitrary permutations in the
numbering of the nodes on $\mathbb{N}_L$, and is therefore independent
of the specific form of the bijective mapping
$\Theta:\mathbb{Z}_L^d\mapsto\mathbb{N}_L$ in equation 
\eqref{eq:Bijection_Z_N}. More specifically, let $\Xi$ be a permutation
matrix satisfying $\Xi^{\,-1}=\Xi^T$ such that $\vec{\xi}^{\; T}\Xi$ is the vector
$\vec{\xi}^{\; T}$ with the entries permuted in an arbitrary manner. Such
a permutation in the numbering of the nodes is equivalent to the
mapping $D_j\mapsto D_j\Xi$, $j=1,\ldots,d$. By the properties of transposition and
inversion for products of matrices \cite{Horn_Johnson-1990}, it is
easily verified that the matrix $\Gamma=\nabla(\Delta^{-1})\nabla^{\;T}$ is invariant under this mapping. Similarly,
permuting the numbering of the bonds is equivalent to the mapping
$D_j\mapsto \Xi\,D_j$, and under this mapping $\Gamma\mapsto \Xi\,\Gamma\, \Xi^T$.




The projection matrix representation of the operator $\Upsilon$ for the
lattice setting is obtained as follows. For simplicity, we restrict
our attention to $d=2,3$. On $\mathbb{R}^3$ the curl 
operation $\vec{\nabla}\times$ is given by 
%
\begin{align}\label{eq:Curl_3D}
  \vec{\nabla}\times\vec{\zeta}=
     \text{Det}\left[
  \begin{array}{ccc}
    \vec{e}_1 &\vec{e}_2 &\vec{e}_3\\
       \partial_1    &   \partial_2    &   \partial_3   \\
       \zeta_1    &   \zeta_2    &   \zeta_3   
    \end{array}
    \right]
    =C\vec{\zeta}, \quad
    C=
    \left[
    \begin{array}{ccc}
       0  & -\partial_3  &   \partial_2 \\
       \partial_3 &  0   &  -\partial_1 \\
      -\partial_2 &  \partial_1  &   0
    \end{array}
    \right],    
\end{align}
%
where $\vec{\zeta}=\vec{\zeta}(\vec{x})$ for $\vec{x}\in\mathbb{R}^3$, we
have denoted $\partial_i$, $i=1,2,3$, to be partial differentiation in the
$i^{\;\text{th}}$ direction $\vec{e}_i$, and $C$ is the curl
operator $\vec{\nabla}\times$ in matrix form. One can check directly
that $C^{\,2}=-C^TC=-\bDelta+\vec{\nabla}\vec{\nabla}\cdot$, where $\bDelta$ is the vector
Laplacian.
% For dimensions $d>3$, the condition $\vec{\nabla}\times\vec{\zeta}=0$ is
% given by $\partial_j\zeta_j-\partial_j\zeta_i=0$, $i,j=1,\ldots,d$
% \cite{Golden:CMP-473}. Consequently, the matrix curl operator $C$
% analogous to that in \eqref{eq:Curl_3D} for $d>3$ becomes increasingly 
% rectangular with increasing dimension.
The two-dimensional case 
follows from \eqref{eq:Curl_3D} by setting
$\vec{\zeta}(\vec{x})=[\zeta_1(\vec{x}),\zeta_2(\vec{x}),0]^T$ with
$\vec{x}=[x_1,x_2,0]^T$, yielding   
%
\begin{align}\label{eq:Curl_2D}
  \vec{\nabla}\times\vec{\zeta}=(\partial_1\zeta_2-\partial_2\zeta_1)\vec{e}_3
%  =
%   \left[
%     \begin{array}{ccc}
%        -\partial_2  &   \partial_1 \\      
%     \end{array}
%   \right]
%  
  %  \left[
%     \begin{array}{ccc}
%        \zeta_1  \\
%        \zeta_2   
%     \end{array}
%   \right]\vec{e}_3
%  
  =(\vec{\nabla}\cdot R\vec{\zeta}_2)\vec{e}_3,
%
  \quad
%  
  \vec{\nabla}\cdot
    =\left[
    \begin{array}{ccc}
       \partial_1  &   \partial_2 \\      
    \end{array}
  \right],
  \quad
  R=
  \left[
    \begin{array}{rr}
        0  &  1  \\
       -1  &  0  
    \end{array}
  \right],
 %  \quad
%   \vec{\zeta}_2=
%   \left[
%     \begin{array}{ccc}
%        \zeta_1  \\
%        \zeta_2   
%     \end{array}
%   \right].
\end{align}
%
where $R$ is a $90^\circ$ rotation matrix, we have defined $\vec{\zeta}_2=[\zeta_1
\ \zeta_2]^T$, and the action of $\vec{\nabla}\cdot R$ on $\vec{\zeta}_2$ is given by that
of the operator $[ - \partial_2 \ \partial_1]$.



In view of equations \eqref{eq:Kirchhiff's__Equations} and
\eqref{eq:Curl_3D}, the matrix representation of the curl operator
$\vec{\nabla}\times$ for the \emph{infinite} lattice setting on $\mathbb{Z}^3$
is given by $C$ in \eqref{eq:Curl_3D} under the mapping $\partial_i\mapsto D^+_i$,
$i=1,2,3$, while on $\mathbb{N}_L$ the curl operator is given by
$C$ in \eqref{eq:Curl_3D} under the mapping $\partial_i\mapsto D_i$. In two
dimensions, pointwise rotations of fields by $90^\circ$ convert curl free
fields to divergence free fields, and vice versa
\cite{MILTON:2002:TC}. With this in mind and in view of equation
\eqref{eq:Curl_2D}, in \emph{two-dimensions} it is natural to define
the curl operator by $\vec{\nabla}\times=\vec{\nabla}\cdot R=[ - \partial_2 \
\partial_1]$. Consequently, for the infinite lattice setting on
$\mathbb{Z}^2$ we have $\vec{\nabla}\times=[ - D^+_2 \ D^+_1]$, while on
$\mathbb{N}_L$ we have  
%
\begin{align}
  \vec{\nabla}\times\vec{\zeta}=C^T\vec{\zeta}, \quad
  C^T=
  \left[
    \begin{array}{ccc}
       -D_2^T  &   D_1^T
    \end{array}
  \right],
\end{align}
%
where $C^TC=\nabla^T\nabla=\Delta$, the matrix representation of the Laplacian. From
the above discussion and in view of equation \eqref{eq:GammaCurl}, in
the lattice setting, it is natural to define the operator $\Upsilon$ as
%
\begin{align}\label{eq:GammaCurl_NL}
  \Upsilon=\vec{\nabla}\times(\vec{\nabla}\times\vec{\nabla}\times)^{-1}\vec{\nabla}\times
   =C(C^TC)^{-1}C^T,
\end{align}
%
which is clearly a projection operator satisfying $\Upsilon^2=\Upsilon$. With this
definition of curl and $\Upsilon$ for \emph{two-dimensions}, we have $\Upsilon=R^T\Gamma R$. 




Analogous
to the properties of the matrix $\Gamma$, in the finite lattice setting the 
matrix $\Upsilon$ is invariant under arbitrary permutations in the numbering
of the nodes. More specifically, let $\Xi$ be defined as above and
define $\bXi=\text{diag}(\Xi,\ldots,\Xi)$, so that $\bXi^{\,-1}=\bXi^T$. Such a
permutation in the 
numbering of the nodes is equivalent to the mapping
$C\mapsto C\,\bXi$. It is straight forward to verify that $\Upsilon$ is invariant
under this mapping. Similarly, permuting the numbering of
the bonds is equivalent to the mapping $C\mapsto \bXi C$, and under
this mapping $\Upsilon\mapsto \bXi \Upsilon {\bXi}^T$.  




We now discuss the matrix representation of the characteristic
function $\chi_1^j(\vec{x},\omega)$ on $\mathbb{N}_L$. By writing the
constitutive relation $J^j(\vec{x},\omega)=\sigma^j(\vec{x},\omega)E^j(\vec{x},\omega)$
displayed in equation \eqref{eq:Kirchhiff's__Equations} as
$J^j(\vec{x},\omega)=\sigma_2(1-\chi_1^j(\vec{x},\omega)/s)E^j(\vec{x},\omega)$, 
we see that the characteristic
function $\chi_1^j(\vec{x},\omega)$ in \eqref{eq:two-phase_sigma} operates
on the electric field
$E^j(\vec{x},\omega)$ in each individual bond $j=1,\ldots,d$ emanating from
$\vec{x}\in\mathbb{Z}_L^d$. In view of this and equation
\eqref{eq:bijection_vector_mapping}, on
$\mathbb{N}_L$ the characteristic function $\chi_1^j(\vec{x},\omega)$ is
represented by a block diagonal matrix and
%
\begin{align}\label{eq:block_diag_chi}  
  \chi_1(\omega)=\text{diag}(\chi_1^1(\omega),\ldots,\chi_1^d(\omega)), \qquad
  \chi_2(\omega)=I-\chi_1(\omega),
\end{align}
%
where $\chi_1^j(\omega)$, $j=1,\ldots,d$, is a \emph{diagonal} matrix of size $L^d\times
L^d$ with zeros and ones distributed according to $P$ along the main
diagonal and $I$ is the identity matrix on $\mathbb{R}^N$. Moreover,
the matrix $\chi_1^j(\omega)$ acts on the vector 
$\vec{E}^j(\omega)=\Theta(E^j(\vec{x},\omega))$ in
\eqref{eq:bijection_vector_mapping} for each $j=1,\ldots,d$. Consequently, 
$\chi_1(\omega)$ is also a real-symmetric projection matrix of size $N\times N$,
which determines the geometry and component connectivity of the
two-phase random medium. In summary, on $\mathbb{N}_L$ the operators
$M_1=\chi_1\Gamma\chi_1$ and $K_1=\chi_1\Upsilon\chi_1$ are represented by real-symmetric
random matrices of size $N\times N$
\cite{Golden:JBM:337,Murphy:JMP:063506}. The matrix
representations of the operators $M_2=\chi_2\Gamma\chi_2$ and $K_2=\chi_2\Upsilon\chi_2$ are
then determined by the relation $\chi_2(\omega)=I-\chi_1(\omega)$, where $I$ is the
identity matrix on $\mathbb{R}^N$.





The following theorem provides a rigorous mathematical formulation of
integral representations for the effective parameters of two-phase
random media with finite lattice composite microstructure. The theorem
and proof are formulated in terms of the random matrix
$M_1=\chi_1\Gamma\chi_1$. The formulations involving the matrices $M_2=\chi_2\Gamma\chi_2$
and $K_i=\chi_i\Upsilon\chi_i$, $i=1,2$, are analogous.  

\vspace{0.15in}
% 
\begin{theorem}\label{thm:Discrete_Spectral_Theorem_ACM}
  For each $\omega\in\Omega$, let $M_1(\omega)=U(\omega)\Lambda(\omega)\,U(\omega)$ be the eigenvalue
  decomposition of the real-symmetric matrix
  $M_1(\omega)=\chi_1(\omega)\,\Gamma\;\chi_1(\omega)$. Here, the columns of the matrix $U(\omega)$
  consist of the orthonormal eigenvectors $\vec{u}_i(\omega)$, $i=1,\ldots,N$,
  of $M_1(\omega)$ and the diagonal matrix $\Lambda(\omega)={\rm diag}(\lambda_1(\omega),\ldots,\lambda_N(\omega))$
  involves its eigenvalues $\lambda_i(\omega)$. If the electric field
  $\vec{E}(\omega)$ satisfies $\vec{E}(\omega)=\vec{E}_0+\vec{E}_f(\omega)$, with
  $\vec{E}_0=\langle\vec{E}(\omega)\rangle$ and $\Gamma\vec{E}(\omega)=\vec{E}_f(\omega)$, then the
  effective complex conductivity tensor $\bsig^*$ has components
  $\sigma_{jk}^*=\sigma_2\,m_{jk}(h)$, $j,k=1,\ldots,d$,  which satisfy       
%
\begin{align}\label{eq:Stieltjes_F_Discrete}
  &m_{jk}(h)=\delta_{jk}-F_{jk}(s), 
  &&F_{jk}(s)=\int_0^1\frac{\d\mu_{jk}(\lambda)}{s-\lambda}\,, 
  &&\d\mu_{jk}(\lambda)=\sum_{i=1}^N\langle \delta_{\lambda_i}(\d\lambda)\;\chi_1\,Q_i\hat{e}_j\cdot\hat{e}_k\rangle,  
\end{align}
%
where $Q_i=\vec{u}_i\,\vec{u}_i^{\;T}$. Furthermore, the mass $\mu_{jk}^0$ of the
measure $\mu_{jk}$ satisfies 
%
\begin{align}\label{eq:Measure_Mass_theorem}
  \mu_{jk}^0=\langle\chi_1\hat{e}_k\cdot\hat{e}_k\rangle\,\delta_{jk}
       %=d\,\frac{\langle N_1^k(\omega)\rangle}{N}\,\delta_{jk}
       =d\;p_1^k\,\delta_{jk}.
\end{align}
%
Here, we have defined $p_1^k=\langle N_1^k(\omega)\rangle/N$ to be the average number
fraction of type-one bonds in the positive $k^{\text{th}}$ direction,
$N_1^k(\omega)={\rm Trace}(\chi_1^k(\omega))$ is the total number such bonds for
$\omega\in\Omega$, and the matrix $\chi_1^k(\omega)$ is defined in equation
\eqref{eq:block_diag_chi}.  
% 
\end{theorem}


Taking $\vec{E}=\vec{E}_0+\vec{E}_f$ with the 
condition $\Gamma\vec{E}=\vec{E}_f$ as a definition greatly simplifies the
proof of Theorem \ref{thm:Discrete_Spectral_Theorem_ACM}, by avoiding
the formulation and proof of some technical lemmas regarding the
commutativity of the matrices $D_i$, $D_i^{\;T}$, $i=1,\ldots,d$, and
$(\Delta^{-1})$. To assume the condition $\Gamma\vec{E}=\vec{E}_f$ is natural, 
as we showed in equation \eqref{eq:Suff_Cond} that it is a 
sufficient condition for the energy constraint
$\langle\vec{J}\cdot\vec{E}_f\rangle=0$, which is at the heart of the existence of
solutions to equations \eqref{eq:Maxwells_Equations_E} and
\eqref{eq:Kirchhiff's__Equations} in the infinite, continuum and
lattice settings, respectively. In the finite lattice setting, where
$\Gamma$ and $\chi_1$ are matrices, this condition leads to equation
\eqref{eq:Proj_rep_Ef_Jf} exactly as in Section
\ref{sec:Continuum_Setting}.  




The proof of Theorem \ref{thm:Discrete_Spectral_Theorem_ACM} is given
in Section \ref{sec:Theorem_Proof}, after we present a novel
formulation of the ACM in Section
\ref{sec:Unification_finite_infinite}, which unifies the infinite
settings and the finite lattice setting and makes the proof of
Theorem \ref{thm:Discrete_Spectral_Theorem_ACM} more
transparent. Before we do so, we first introduce an important class of
composite microstructures. Namely, the 
class of finite bond lattices such that $N_1^k(\omega)$ is a non-random
constant $N_1^k$ for all $k=1,\ldots,d$, i.e., $N_1^k(\omega)=N_1^k$ for all
$\omega\in\Omega$. Consequently, the number $N_1(\omega)={\rm Trace}(\chi_1(\omega))$ of ones
along the main diagonal of $\chi_1(\omega)$ satisfies $N_1(\omega)=N_1$ for all
$\omega\in\Omega$, with $N_1=\sum_kN_1^k$. Moreover, the number fraction of type-one
bonds in the $k^{\text{th}}$ positive direction is given by
$p_1^k=N_1^k/N$ and the total number fraction of type-one bonds is
given by $p_1=N_1/N$, with $p_1=\sum_kp_1^k$.





Given a fixed number
fraction $p_1$ of type-one  
bonds, one can define a class of highly \emph{anisotropic} composites
by fixing $p_1^k$ close to $p_1$ for some $k=1,\ldots,d$,
i.e., $p_1-p_1^k\ll1$. A class of \emph{locally isotropic} random media is 
obtained by requiring that every bond emanating from
$\vec{x}\in\mathbb{Z}^d_L$ in the positive direction is of the same
type, i.e., $\chi_1^j(\omega)=\chi_1^k(\omega)$ for all $j,k=1,\ldots,d$ and $\omega\in\Omega$. Hence
$N_1^j=N_1^k$ for all $j,k=1,\ldots,d$, so that $N_1^k=N_1/d$ and
$p_1^k=p_1/d$ for all $k=1,\ldots,d$. In this case, equation
\eqref{eq:Measure_Mass_theorem} reduces to 
% 
\begin{align}\label{eq:Meas_mass_Isotropic_iid}
  \mu_{jk}^0=p_1\,\delta_{jk},
\end{align}
%
which is a direct analogue of equation \eqref{eq:Measure_Mass_ConT}.
Equation \eqref{eq:Meas_mass_Isotropic_iid} also holds for
\emph{statistically isotropic} random media, where the total number
$N_1$ of type-one bonds is fixed and randomly distributed in a uniform
fashion among the the total number $N$ of bonds. In other words, the 
main diagonals of the matrices $\chi_1(\omega)$, $\omega\in\Omega$, are random
permutations of one another.  In this case, the
$N_1^k(\omega)$, $k=1,\ldots,d$, are independent, identically distributed random
variables with mean $\langle N_1^k(\omega)\rangle=p_1N/d$.
%Unresolved Issue: IS THIS PRECISELY TRUE IN THIS CONTEXT, OR ONLY IN
%THE INFINITE VOLUME LIMIT?  


We note that, by the law of large numbers
\cite{Durrett:Book:Probability}, the formula
$\mu_{jk}^0=d\,p_1^k\,\delta_{jk}$ in equation
\eqref{eq:Measure_Mass_theorem} also holds in the infinite lattice
setting, where $p_1^k=\lim_{N\to\infty}\langle N_1^k(\omega)\rangle/N$ is the volume fraction 
of type-one bonds in the $k^{\text{th}}$ direction. Here, the infinite
lattice is obtained as the infinite volume limit $L\to\infty$ $(N\to\infty)$ of the
finite lattice -- with $\mathbb{Z}_L^d$ in \eqref{eq:ZLd} redefined in a
suitable way so that $\lim_{L\to\infty}\mathbb{Z}_L^d=\mathbb{Z}^d$. Consequently, equation
\eqref{eq:Meas_mass_Isotropic_iid} also holds in the infinite lattice
setting for locally and statistically isotropic random media.
%Unresolved Issue: IS THIS PRECISELY TRUE?





\subsubsection{Unifying formulation of the ACM for the finite lattice
  setting and the infinite settings}\label{sec:Unification_finite_infinite}
%
When considering the formulation of Stieltjes integral representations
for the effective parameters of two-phase random media with finite
lattice composite microstructure, there is a fundamental issue with
the original formulation of the ACM given in Sections
\ref{sec:Continuum_Setting} and
\ref{sec:Infinite_Lattice_Setting}. Namely, the original formulation
\cite{Golden:CMP-473,Bruno:PRSLA-353} holds for the \emph{infinite} 
continuum and lattice settings, but it is incompatible with the
\emph{finite} lattice setting of Section
\ref{sec:Finite_Lattice_Setting}. In this section, we address this
issue by providing a novel formulation of the ACM, which is equivalent to
the original formulation and holds for both the finite lattice setting
and the infinite settings.    




In the infinite settings, the (infinite-dimensional) operator $\Gamma\chi_1$
appears in the bilinear functional underlying the Stieltjes integral
representation for the effective conductivity tensor
$\bsig^*=\sigma_2\mathbf{m}(h)$, displayed in equation
\eqref{eq:Stieltjes_F}. The underlying Hilbert space is
$\mathscr{H}_\times$, defined in \eqref{eq:curlfreeHilbert}, equipped with
the $\mathscr{H}$-inner-product weighted by the characteristic function $\chi_1$, and $\Gamma\chi_1$
is a self-adjoint operator on $\mathscr{H}_\times$ with respect to this
inner-product. In this abstract (infinite-dimensional) Hilbert space
formulation of the effective parameter problem, the resolvent
$(sI-\Gamma\chi_1)^{-1}$ is also self-adjoint with respect to this
inner-product \cite{Stone:64}. 



In contrast, the finite lattice formulation of the effective parameter
problem involves a finite dimensional Hilbert space, and the operators
$\Gamma$ and $\chi_1$ are real-symmetric, non-commutable matrices. In this
case, the matrix $\Gamma\chi_1$ is \emph{not} symmetric, it typically has
complex spectrum, and it may not even have a full set of
eigenvectors. Consequently, the resolvent $(sI-\Gamma\chi_1)^{-1}$ of this
matrix is not symmetric and, in general, is 
not defined for all $s\in\mathbb{C}\backslash[0,1]$ as required. Therefore, the
integral formula of Theorem \ref{thm:Discrete_Spectral_Theorem_ACM}
displayed in equation \eqref{eq:Stieltjes_F_Discrete}, which follows
from the spectral theorem displayed in equation
\eqref{eq:Discrete_Spectral_Theorem} for the
\emph{real-symmetric} matrix $\chi_1\Gamma\chi_1$, fails to hold for the matrix
$\Gamma\chi_1$, in general. Due to this fundamental difference of the finite
lattice setting, the mathematical framework must be 
modified from that of the infinite settings, discussed in Sections
\ref{sec:Continuum_Setting} and \ref{sec:Infinite_Lattice_Setting}.



We now develop a novel formulation of the ACM which holds for both the
finite lattice setting and the infinite settings, and yields the
integral representations for the effective parameters displayed in
equations \eqref{eq:Stieltjes_F} and
\eqref{eq:Stieltjes_F_Discrete}. To make the formulation independent
of the setting, whether finite or infinite, we make use of generic
terms such as symmetric operator, for example, which means
real-symmetric matrix in the finite lattice setting and self-adjoint
operator in the infinite settings. Essential differences in notation
will be explicitly stated.



Recall the definition of the effective conductivity tensor
$\langle\vec{J}\,\rangle=\langle\sigma\vec{E}\rangle=\bsig^*\langle\vec{E}\rangle$ and that $\sigma=\sigma_2(1-\chi_1/s)$ and
$\langle\vec{E}\rangle=\vec{E}_0$, together yielding  
% 
\begin{align}\label{eq:Eff_Cond_Tens_Def}
  \bsig^*\vec{E}_0=\sigma_2(\vec{E}_0 -\langle\chi_1\vec{E}\rangle/s\,).
\end{align}
%
Define the coordinate system so that $\vec{E}_0=E_0\vec{e}_j$ for
some $j=1,\ldots,d$ (in the matrix formulation
$\vec{e}_j\mapsto\hat{e}_j$, where $\hat{e}_j$ is defined in equation
\eqref{eq:Lattice_Basis_e}). Therefore, taking the dot product of
equation  \eqref{eq:Eff_Cond_Tens_Def} with the (non-random) basis vector
$\vec{e}_k$ yields 
%
\begin{align}\label{eq:Component_sigma_def}
  \sigma^*_{jk}=\bsig^*\vec{e}_j\cdot\vec{e}_k=\sigma_2\left(\delta_{jk}-\langle\chi_1\vec{E}\cdot\vec{e}_k\rangle/(sE_0)\right). 
\end{align}
%
This demonstrates that the key functional underlying the Stieltjes
integral representation for the effective complex conductivity tensor
is $\langle\chi_1\vec{E}\cdot\vec{e}_k\rangle$. In fact, in view of equations
\eqref{eq:Stieltjes_F} and \eqref{eq:Component_sigma_def}, we have
that $F_{jk}(s)=\langle\chi_1\vec{E}\cdot\vec{e}_k\rangle/(sE_0)$. 


We now derive a resolvent formula for the vector field
$\chi_1\vec{E}$ involving the symmetric operator $\chi_1\Gamma\chi_1$. With use of the
identity $\vec{E}=\vec{E}_0+\vec{E}_f$ we rewrite the first formula of
equation \eqref{eq:Proj_rep_Ef_Jf} as  
%
\begin{align}\label{eq:Pre_Resolvent}
  (sI-\Gamma\chi_1)\vec{E}=s\vec{E}_0,
\end{align}
%
where $I$ is the identity operator on the underlying vector space
($\mathbb{R}^d$ for the infinite settings and $\mathbb{R}^N$ for the
finite lattice setting). It is now clear that the formula for
$m_{jk}(h)=\sigma^*_{jk}/\sigma_2$ displayed in \eqref{eq:Stieltjes_F} follows by writing
the formula in equation \eqref{eq:Pre_Resolvent} as
$\vec{E}=s(sI-\Gamma\chi_1)^{-1}\vec{E}_0$ with $\vec{E}_0=E_0\vec{e}_j$, and
substituting this into \eqref{eq:Component_sigma_def}. We wish to derive an
analogous formula for $m_{jk}(h)$ involving the symmetric operator
$\chi_1\Gamma\chi_1$. In order to introduce this operator and to isolate
$\chi_1\vec{E}$ in equation \eqref{eq:Pre_Resolvent}, we premultiply
this formula by the \emph{projection} operator $\chi_1$, yielding 
%
\begin{align}\label{eq:Discrete_Resolvent}
  (sI-\chi_1\Gamma\chi_1)[\chi_1\vec{E}]=s\chi_1\vec{E}_0.
\end{align}
%
Equation \eqref{eq:Discrete_Resolvent} is equivalent to the following
resolvent formula for $\chi_1\vec{E}$
%
\begin{align}\label{eq:Resolvent_chiE}
  \chi_1\vec{E}=s(sI-\chi_1\Gamma\chi_1)^{-1}\chi_1\vec{E}_0,
\end{align}
%
which is analogous to that of equation
\eqref{eq:Resolvent_representations_E_D} for the electric field
$\vec{E}$. Inserting the resolvent formula for $\chi_1\vec{E}$ in
\eqref{eq:Resolvent_chiE}, with $\vec{E}_0=E_0\vec{e}_j$, into
equation \eqref{eq:Component_sigma_def} yields
$F_{jk}(s)=\langle(sI-\chi_1\Gamma\chi_1)^{-1}\chi_1\vec{e}_j\cdot\vec{e}_k\rangle$, which is a
bilinear functional representation of the function
$F_{jk}(s)$. Now, applying the spectral theorem for the symmetric
operator $\chi_1\Gamma\chi_1$, displayed in equations \eqref{eq:Spectral_Theorem} and
\eqref{eq:Discrete_Spectral_Theorem} with $f(\lambda)=(s-\lambda)^{-1}$, to this
functional representation of $F_{jk}(s)$ yields the following
Stieltjes integral representation for $m_{jk}(h)=\sigma^*_{jk}/\sigma_2$ 
%
\begin{align}\label{eq:Stieltjes_m}
  m_{jk}(h)=\delta_{jk}-F_{jk}(s), \qquad
  F_{jk}(s)=\langle(sI-\chi_1\Gamma\chi_1)^{-1}\chi_1\vec{e}_j\cdot\vec{e}_k\rangle=\int_0^1\frac{\d\mu_{jk}(\lambda)}{s-\lambda}\,.
\end{align}
%


Equation \eqref{eq:Stieltjes_m} demonstrates that, as in the formulation
of the ACM given in Section \ref{sec:Continuum_Setting}, the natural
Hilbert space underlying the integral representation in equation
\eqref{eq:Stieltjes_m} is $\mathscr{H}_\times$ equipped with the
$\mathscr{H}$-inner-product weighted by $\chi_1$. However, in Section
\ref{sec:Continuum_Setting} the weighting of the inner-product is
defined by premultiplication of $\chi_1$, so that
$\langle f(\Gamma\chi_1)\vec{e}_j,\vec{e}_k\rangle_1=\langle\chi_1f(\Gamma\chi_1)\vec{e}_j\cdot\vec{e}_k\rangle$, for
all complex valued functions $f\in L^2(\mu_{jk})$. Here, the weighting of
the inner-product is defined by post multiplication of $\chi_1$, so that
the inner-product $\langle\cdot,\cdot\rangle_1$ is instead defined by
$\langle f(\chi_1\Gamma\chi_1)\vec{e}_j,\vec{e}_k\rangle_1=\langle f(\chi_1\Gamma\chi_1)\chi_1\vec{e}_j\cdot\vec{e}_k\rangle$,
for all $f\in L^2(\nu_{jk})$. In the infinite, continuum and lattice
settings, the two inner-product definitions are equivalent, as $\chi_1$
acts \emph{pointwise} on the underlying vector space ($\mathbb{R}^d$
in the continuous setting and $\mathbb{Z}^d$ in the lattice
setting). However, in the finite lattice setting where $\chi_1$ is
represented as a matrix, the two inner-product definitions are no
longer equivalent, for all such functions $f$. 



We now argue that the formula for $m_{jk}(h)$ in equation
\eqref{eq:Stieltjes_m} is equivalent to that of equation
\eqref{eq:Stieltjes_F} for the infinite, continuum and lattice
settings. From equation \eqref{eq:Stieltjes_F} write 
$F_{jk}(s;\mu_{jk})=\langle\chi_1(sI-\Gamma\chi_1)^{-1}\vec{e}_j\cdot\vec{e}_k\rangle$ and from equation
\eqref{eq:Stieltjes_m} write
$\tilde{F}_{jk}(s,\nu_{jk})=\langle(sI-\chi_1\Gamma\chi_1)^{-1}\chi_1\vec{e}_j\cdot\vec{e}_k\rangle$. We will
argue that $\mu_{jk}\equiv\nu_{jk}$ so that
$F_{jk}(s;\mu_{jk})\equiv\tilde{F}_{jk}(s,\nu_{jk})$. From the spectral
theorem, we have that the moments $\mu^n_{jk}$ and $\nu^n_{jk}$,
$n=0,1,2,\ldots$, of the measures $\mu_{jk}$ and $\nu_{jk}$ satisfy 
%
\begin{align}\label{eq:Moments_mu_nu}
   \mu_{jk}^n=\int_0^1\lambda^n\d\mu_{jk}(\lambda)=\langle\chi_1[\Gamma\chi_1]^n\,\vec{e}_j\cdot\vec{e}_k\rangle, \quad
   \nu_{jk}^n=\int_0^1\lambda^n\d\nu_{jk}(\lambda)=\langle[\chi_1\Gamma\chi_1]^n\chi_1\vec{e}_j\cdot\vec{e}_k\rangle.
\end{align}
%
However, since $\chi_1$ is a projection operator, we have that
$\chi_1=\chi_1^m$ on $\mathscr{H}_\times$ for all $m\in\mathbb{N}$, hence
$\chi_1[\Gamma\chi_1]^n=[\chi_1\Gamma\chi_1]^n\chi_1$ on $\mathscr{H}_\times$ for all
$n=0,1,2,\ldots$. This and equation \eqref{eq:Moments_mu_nu} imply that
$\mu^n_{jk}\equiv\nu^n_{jk}$  for all $n=0,1,2,\ldots$. Since the Hausdorff moment
problem is determined \cite{Shohat:1963}, i.e., knowledge of all the
moments uniquely determines the measure, we have that
$\mu_{jk}\equiv\nu_{jk}$. This, in turn, implies that
$F_{jk}(s;\mu_{jk})\equiv\tilde{F}_{jk}(s,\nu_{jk})$, which is what we set out
to establish.






\subsubsection{Proof of Theorem
  \ref{thm:Discrete_Spectral_Theorem_ACM}}
\label{sec:Theorem_Proof}
%
In this section, we prove the various assertions of Theorem
\ref{thm:Discrete_Spectral_Theorem_ACM}, which was stated in Section
\ref{sec:Finite_Lattice_Setting}. In particular, we prove that the functional
$F_{jk}(s)=\langle(sI-\chi_1\Gamma\chi_1)^{-1}\chi_1\hat{e}_j\cdot\hat{e}_k\rangle$ in
\eqref{eq:Stieltjes_m} (with
$\vec{e}_j\mapsto\hat{e}_j$) has the integral representation displayed in
equation \eqref{eq:Stieltjes_F_Discrete}, involving the spectral
measure $\mu_{jk}$ of the real-symmetric matrix $\chi_1\Gamma\chi_1$, with mass
$\mu^0_{jk}$ given by that in \eqref{eq:Measure_Mass_theorem}. We also provide a
projection method for the numerically efficient, rigorous computation of
$\mu_{jk}$. This projection method is summarized by equations
\eqref{Pi_coordinates_E}--\eqref{eq:Fs_U1} below.   





Toward this goal,
for each $\omega\in\Omega$, define the sets $\mathbb{N}_L^1(\omega)$ and
$\mathbb{N}_L^{\,0}(\omega)$ by
% 
\begin{align}\label{eq:Zero_One_indices}
  \mathbb{N}_L^1(\omega)=\{i\in\mathbb{N}_L \ | \ [\chi_1(\omega)]_{ii}=1\}, \qquad
  \mathbb{N}_L^{\,0}(\omega)=\mathbb{N}_L\backslash \mathbb{N}_L^1(\omega).
\end{align}
%
Also, define elementary permutation matrices \cite{Demmel:1997}
$\Pi_{\ell,m}(\omega)$, $\ell,m=1,\ldots,N$, $N=dL^d$, which satisfy
$\Pi_{\ell,m}=\Pi_{\ell,m}^{\,-1}=\Pi_{\ell,m}^{\;T}$ and $\Pi_{\ell,m}\vec{\xi}$ is the
vector $\vec{\xi}$ with the $\ell^{\,\text{th}}$ and $m^{\text{th}}$
entries interchanged. 




Since $\chi_1(\omega)$ is a diagonal matrix with $N_1(\omega)$ ones and
$N_0(\omega)=N-N_1(\omega)$ zeros along its main diagonal, it is clear that
there exists a permutation matrix $\Pi(\omega)$ which is a composition of
elementary permutation matrices such that 
%
\begin{align}\label{eq:chi_Perm} 
  \Pi\chi_1\Pi^{\;T}= 
  \left[
  \begin{array}{ccc}
    0_{00}&0_{01}\\
    0_{10}&I_1   
    \end{array}
\right],
\qquad
\Pi=\prod_{\ell,m\in\mathbb{N}_L}\Pi_{\ell m},
\end{align}
%
where $\ell\in\mathbb{N}_L^1$, $m\in\mathbb{N}_L^{\,0}$, $I_1$ is the
identity matrix of size $N_1\times N_1$, and $0_{ab}$ is a matrix of zeros
of size $N_a\times N_b$, for $a,b=0,1$. Therefore, since  $\Pi^{\;T}=\Pi^{\,-1}$
we have 
%
\begin{align}\label{eq:Spec_Decomp_chi_Gamma_chi_Proof}
  \chi_1\Gamma\chi_1&=
  \Pi^{\;T}
  \left[
  \begin{array}{ccc}
    0_{00}&0_{01}\\
    0_{10}&I_1   
    \end{array}
\right]
\Gamma_\Pi
\left[
  \begin{array}{ccc}
    0_{00}&0_{01}\\
    0_{10}&I_1   
    \end{array}
\right]
\Pi
=
\Pi^{\;T}
\left[
  \begin{array}{ccc}
    0_{00}&0_{01}\\
    0_{10}&\Gamma_1   
    \end{array}
\right]
\Pi
%\\
%&=
=
\Pi^{\;T}
\left[
  \begin{array}{ccc}
    0_{00}&0_{01}\\
    0_{10}&U_1\Lambda_1U_1^{\;T} 
    \end{array}
\right]
\Pi
\notag\\
&=
%=
\Pi^{\;T}
\left[
  \begin{array}{ccc}
    I_0&0_{01}\\
    0_{10}&U_1 
    \end{array}
\right]    
\left[
  \begin{array}{ccc}
    0_{00}&0_{01}\\
    0_{10}&\Lambda_1
    \end{array}
\right]    
\left[
  \begin{array}{ccc}
    I_0&0_{01}\\
    0_{10}&U_1^{\;T} 
    \end{array}
\right]    
\Pi,
%\notag
\end{align}
%
where $I_0$ is the identity matrix of size $N_0\times N_0$. Here, we have
defined the real-symmetric matrix $\Gamma_\Pi=\Pi\,\Gamma\,\Pi^{\;T}$, $\Gamma_1$ is its
lower right principal sub-matrix of size $N_1\times N_1$, and
$\Gamma_1=U_1\Lambda_1U_1^{\;T}$ is the eigenvalue decomposition of $\Gamma_1$. As
$\Gamma_1$ is a real-symmetric matrix, $U_1$ is an orthogonal matrix
\cite{Horn_Johnson-1990}. Also, since $\Gamma_\Pi=\Pi\,\Gamma\,\Pi^{\;T}$ is a
similarity transformation of a projection matrix and $\Pi\chi_1\Pi^{\;T}$ is
a projection matrix, $\Lambda_1$ is a diagonal matrix with entries
$\lambda_i^1\in[0,1]$, $i=1,\ldots,N_1$, along its diagonal 
\cite{Horn_Johnson-1990,Demmel:1997}. 




Consequently, equation  
\eqref{eq:Spec_Decomp_chi_Gamma_chi_Proof} implies that the
eigenvalue decomposition of the matrix $\chi_1\Gamma\chi_1$ is given by 
%$M_1=U\Lambda U^{\;T}$, where
%
\begin{align}\label{eq:Spec_Decomp_chi_Gamma_chi}
\chi_1\Gamma\chi_1=U\Lambda U^{\;T},
\qquad
U=\Pi^{\;T}\left[
  \begin{array}{ccc}
    I_0&0_{01}\\
    0_{10}&U_1   
    \end{array}
\right],
\quad
\Lambda=\left[
  \begin{array}{ccc}
    0_{00}&0_{01}\\
    0_{10}&\Lambda_1   
    \end{array}
\right].
\end{align}
%
Here, $U$ is an orthogonal matrix satisfying $U^TU=UU^T=I$, $I$ is the
identity matrix on $\mathbb{R}^N$, and $\Lambda$ is a diagonal matrix with
entries $\lambda_i\in[0,1]$, $i=1,\ldots,N$, along its diagonal.
% Equation \eqref{eq:Spec_Decomp_chi_Gamma_chi} shows that the
% eigenvalues $\lambda_i$ of the matrix $\chi_1\Gamma\chi_1$ are zero, $\lambda_i=0$, for
% $i=1,\ldots,N_0$, and its eigenvectors $\vec{u}_i$ are given by
% $\Pi^T\vec{e}_i$ for the same index range, where $\vec{e}_i$ is a
% standard basis vector on $\mathbb{R}^N$.  





The eigenvalue decomposition of the matrix $\chi_1\Gamma\chi_1$ displayed in
equation \eqref{eq:Spec_Decomp_chi_Gamma_chi} demonstrates that its
resolvent $(sI-\chi_1\Gamma\chi_1)^{-1}$ is well defined for all
$s\in\mathbb{C}\backslash[0,1]$. In particular, by the orthogonality of the
matrix $U$, it has the following useful representation
$(sI-\chi_1\Gamma\chi_1)^{-1}=U(sI-\Lambda)^{-1}U^T$, where $(sI-\Lambda)^{-1}$ is a diagonal
matrix with entries $1/(s-\lambda_i)$ along its diagonal. This, in
turn, implies that the functional
$F_{jk}(s)=\langle(sI-\chi_1\Gamma\chi_1)^{-1}\chi_1\hat{e}_j\cdot\hat{e}_k\rangle$ 
displayed in equation \eqref{eq:Stieltjes_m} (with
$\vec{e}_j\mapsto\hat{e}_j$) can be written as     
%
\begin{align}\label{eq:Matrix_Functional_proof}
  F_{jk}(s)%=\langle(sI-\chi_1\Gamma\chi_1)^{-1}\chi_1\hat{e}_j\cdot\hat{e}_k\rangle
          =\langle(sI-\Lambda)^{-1}\;[\chi_1U]^T\hat{e}_j\,\cdot\,U^T\hat{e}_k\rangle. 
\end{align}
%
Since
$\Pi^{\;T}=\Pi^{-1}$, equations \eqref{eq:chi_Perm} and
\eqref{eq:Spec_Decomp_chi_Gamma_chi} imply that
%
\begin{align}\label{eq:Projection_Eigenspace}
  \chi_1U=\Pi^{\;T}\left[
  \begin{array}{ccc}
    0_{00}&0_{01}\\
    0_{10}&U_1  
    \end{array}
\right],
\quad
\Longrightarrow
\quad
\chi_1\vec{u}_i=
  \begin{cases}
  0, &\text{ for } i=1,\ldots,N_0  \\
  \vec{u}_i  &\text{ otherwise}
  \end{cases}.
\end{align}
%
This, in turn, implies that
%$[\chi_1U]^T\hat{e}_j\,\cdot\,U^T\hat{e}_k=[\chi_1U]^T\hat{e}_j\,\cdot\,[\chi_1U]^T\hat{e}_k$.
%
\begin{align}\label{eq:Weights_chi}
  [\chi_1U]^T\hat{e}_j\,\cdot\,U^T\hat{e}_k=[\chi_1U]^T\hat{e}_j\,\cdot\,[\chi_1U]^T\hat{e}_k.
\end{align}
%


We are now ready to provide the integral representation displayed in
\eqref{eq:Stieltjes_F_Discrete} for the functional $F_{jk}(s)$ in
equation \eqref{eq:Matrix_Functional_proof}. Denote by
$Q_i=\vec{u}_i\,\vec{u}_i^{\;T}$, $i=1,\ldots,N$, the mutually orthogonal
projection matrices, $Q_\ell\,Q_m=Q_\ell\,\delta_{\ell m}$, onto the eigen-spaces
spanned by the orthonormal eigenvectors $\vec{u}_i$. Equation
\eqref{eq:Projection_Eigenspace} implies that
$\chi_1Q_i=Q_i\chi_1=\chi_1Q_i\chi_1$, as $\chi_1Q_i=0$ for $i=1,\ldots,N_0$ and
$\chi_1Q_i=Q_i$ otherwise. This allows us to write the quadratic form
$[\chi_1U]^T\hat{e}_j\,\cdot\,[\chi_1U]^T\hat{e}_k$ as  
%
\begin{align}\label{eq:Quadratic_form}
  [\chi_1U]^T\hat{e}_j\,\cdot\,[\chi_1U]^T\hat{e}_k=\sum_{i=1}^N(\chi_1\vec{u}_i\cdot\hat{e}_j)
                                     (\chi_1\vec{u}_i\cdot\hat{e}_k)
                              =\sum_{i=1}^N \chi_1Q_i\chi_1\hat{e}_j\cdot\hat{e}_k
                              =\sum_{i=1}^N \chi_1Q_i\hat{e}_j\cdot\hat{e}_k.
\end{align}
%
This and equations \eqref{eq:Matrix_Functional_proof} and
\eqref{eq:Weights_chi} yield
%
\begin{align}\label{eq:Stieltjes_F_DiscretE}
  F_{jk}(s)=\int_0^1\frac{\d\mu_{jk}(\lambda)}{s-\lambda}\,, \quad
  \d\mu_{jk}(\lambda)=\sum_{i=1}^N\langle \delta_{\lambda_i}(\d\lambda)\;\chi_1\,Q_i\hat{e}_j\cdot\hat{e}_k\rangle.
\end{align}
%



From equation \eqref{eq:Matrix_Rep_Spec_Theorem} we have that
$\sum_iQ_i=I$, which implies that the mass $\mu^0_{jk}$ of the measure
$\mu_{jk}$ is given by
%
\begin{align}\label{eq:Measure_Mass_Lattice}
  \mu^0_{jk}=\int_0^1\d\mu_{jk}(\lambda)
       =\int_0^1\sum_{i=1}^N \langle\delta_{\lambda_i}(\d\lambda)\chi_1Q_i\,\hat{e}_j\cdot\hat{e}_k\rangle
       =\langle\chi_1\hat{e}_j\cdot\hat{e}_k\rangle
       =\langle\chi_1\hat{e}_k\cdot\hat{e}_k\rangle\,\delta_{jk},          
\end{align}
%
as $\chi_1$ is a diagonal matrix and the underlying probability space is
finite. Therefore, as in the continuum setting, the diagonal
components $\mu_{kk}$ of the matrix valued measure $\bmu$ are positive
measures with mass
$\langle\chi_1\hat{e}_k\cdot\hat{e}_k\rangle=\langle\chi_1\hat{e}_k\cdot\chi_1\hat{e}_k\rangle=\langle|\chi_1\hat{e}_k|^2\rangle\geq0$,
as $\chi_1$ is a symmetric projection matrix. The off-diagonal
components $\mu_{jk}$, for $j\neq k$, have zero mass and are consequently
signed measures.




Using equation \eqref{eq:block_diag_chi} we may write
$\mu^0_{jk}$ in equation \eqref{eq:Measure_Mass_Lattice} in a more suggestive
form. Recall that $\hat{e}_1=(\vec{1},\vec{0},\ldots,\vec{0})/L^{d/2}$,
where $\vec{1}$ and $\vec{0}$ are vectors of ones and zeros of length
$L^d$, respectively, and similarly for the $\vec{e}_j$ for
$j=2,\ldots,d$. Since $\chi_1$ is a symmetric projection matrix, equations 
\eqref{eq:block_diag_chi} and \eqref{eq:Measure_Mass_Lattice} imply
that 
%
\begin{align}\label{eq:Measure_Mass_Lattice_Trace}
  \mu^0_{jk}%=\langle\chi_1\hat{e}_k\cdot\hat{e}_k\rangle\,\delta_{jk}
       =\langle\chi_1\hat{e}_k\cdot\chi_1\hat{e}_k\rangle\,\delta_{jk}
       =\frac{1}{L^d}\langle\chi_1^k\vec{1}\cdot \chi_1^k\vec{1}\rangle\,\delta_{jk}
       =\frac{1}{L^d}\langle\text{Trace}(\chi_1^k)\rangle\,\delta_{jk}
       =d\,\frac{\langle N_1^k(\omega)\rangle}{N}\,\delta_{jk},       
\end{align}
%
where $N_1^k(\omega)={\rm Trace}(\chi_1^k(\omega))$ is the total number of type-one
bonds in the positive $k^{\text{th}}$ direction for $\omega\in\Omega$ and
$N=d\,L^d$. This proves equation \eqref{eq:Measure_Mass_theorem} and
concludes our proof of Theorem \ref{thm:Discrete_Spectral_Theorem_ACM}
$\Box\,$.       






We conclude this section with the formulation of a projection method
for numerically efficient, rigorous computation of spectral
measures and effective parameters for composite media with finite
lattice microstructure. Note that the sum in equation 
\eqref{eq:Stieltjes_F_DiscretE} runs only over the index set
$i=N_0+1,\ldots,N$, as equation \eqref{eq:Projection_Eigenspace} implies
that the masses $\chi_1\,Q_i\hat{e}_j\cdot\hat{e}_k$ of the measure $\mu_{jk}$
are zero for $i=1,\ldots,N_0$.  Denote by $\lambda_i^1$ and $\vec{u}_i^{\,1}$,
$i=1,\ldots,N_1$, the eigenvalues and eigenvectors of the $N_1\times N_1$
matrix $\Gamma_1=U_1\Lambda_1U_1^{\;T} $, defined in equation
\eqref{eq:Spec_Decomp_chi_Gamma_chi_Proof}. Now, write       
%
\begin{align}\label{Pi_coordinates_E}
\Pi\hat{e}_j=
  \left[
  \begin{array}{ccc}
    \hat{e}_j^{\,\pi_0}\\
    \hat{e}_j^{\,\pi_1}
    \end{array}
\right],
\end{align}
%
where $\hat{e}_j^{\,\pi_0}\in\mathbb{R}^{N_0}$ and
$\hat{e}_j^{\,\pi_1}\in\mathbb{R}^{N_1}$. Therefore, writing the
matrix $\chi_1U$ in equation \eqref{eq:Projection_Eigenspace} in block
diagonal form, $\chi_1U=\Pi^T\text{diag}(0_{00},U_1)$, we have that
%
\begin{align}\label{eq:Reduced_Weights}
  [\chi_1U]^T\hat{e}_j\,\cdot\,[\chi_1U]^T\hat{e}_k=[\text{diag}(0_{00},U_1^T)\Pi\hat{e}_j]
                                    \cdot[\text{diag}(0_{0,0},U_1^T)\Pi\hat{e}_k]
                                   =[U_1^T\hat{e}_j^{\,\pi_1}]
                                   \cdot[U_1^T\hat{e}_k^{\,\pi_1}]. 
\end{align}
%
Denote by $Q^1_i=\vec{u}^{\,1}_i[\vec{u}^{\,1}_i]^{\;T}$, $i=1,\ldots,N_1$,
the mutually orthogonal projection matrices,
$Q^1_\ell\,Q^1_m=Q^1_\ell\,\delta_{\ell m}$, onto the eigen-spaces spanned by the
orthonormal eigenvectors $\vec{u}^{\,1}_i$. Equations
\eqref{eq:Matrix_Functional_proof}, \eqref{eq:Weights_chi}, and
\eqref{eq:Reduced_Weights} then yield    
%
\begin{align}\label{eq:Fs_U1}
  F_{jk}(s)=\langle(sI_1-\Lambda_1)^{-1}[U_1^T\hat{e}_j^{\,\pi_1}]
                       \cdot[U_1^T\hat{e}_k^{\,\pi_1}]\rangle
          =\left\langle\sum_{i=1}^{N_1} 
          \frac{Q^1_i\hat{e}_j^{\,\pi_1}\cdot\hat{e}_k^{\,\pi_1}                
              }{s-\lambda_i^1}
              \right\rangle.        
\end{align}
%



Equation \eqref{eq:Fs_U1} demonstrates that only the spectral
information of the matrices $U_1$ and $\Lambda_1$ appear in the functional
representation for $F_{jk}(s)$ in
\eqref{eq:Matrix_Functional_proof} and its integral representation in
\eqref{eq:Stieltjes_F_Discrete}. From a computational standpoint, 
this means that only the eigenvalues and eigenvectors of the $N_1\times N_1$
matrix $\Gamma_1$ need to be computed in order to compute the spectral
measures underlying the integral representations of the effective
parameters for finite lattice systems. This is extremely cost
effective for large dilute systems, where $N\gg1$ and $N_1\ll N$, as the
numerical cost of finding all the eigenvalues and eigenvectors of a
real-symmetric $N\times N$ matrix is $O(N^3)$ \cite{Demmel:1997}.






\subsection{Bounding Procedure}\label{sec:Bounding_Procedure}
%
In this section, we review a procedure which yields rigorous bounds for
the effective transport coefficients of composite media
\cite{Golden:CMP-473,Golden:JMPS-333}. The bounding procedure
associated with the functions $F_{kk}(s)$ and $E_{kk}(s)$, 
defined in equation \eqref{eq:Stieltjes_F}, for example, fixes the
contrast parameter $s$ and varies over admissible sets of measures
$\mu_{kk}$ and $\eta_{kk}$, subject to
known information regarding the composite. This information is given
in terms of the moments $\mu_{kk}^n$ and $\eta_{kk}^n$, $n=0,1,2,\ldots$, of
these measures. Knowledge of these moments for $n=1,\ldots,J$ confines the
value of the effective complex conductivity $\sigma_{kk}^*$ to a region of
the complex plane which is bounded by arcs of circles, and the region
becomes progressively smaller as more moments are known
\cite{Milton:JAP-5294,Golden:JMPS-333}. Since the bounding procedure
associated with the functions $G_{kk}(t)$ and $H_{kk}(t)$ in
\eqref{eq:Stieltjes_F} is analogous, we will focus on that involving
$F_{kk}(s)$ and $E_{kk}(s)$.    


The bounds for $\sigma_{kk}^*$ and $\rho^*_{kk}$ follow from three important
properties of the functions $F_{kk}(s)$ and $E_{kk}(s)$. First, their
integral representations, displayed in equations
\eqref{eq:Stieltjes_F} and \eqref{eq:Stieltjes_F_Discrete},
\emph{separate} parameter information in $s$ and $E_0$ from the
geometry of the composite, which is encoded in the underlying spectral
measures $\mu_{kk}$ and $\eta_{kk}$ via their moments $\mu_{kk}^n$ and
$\eta_{kk}^n$, $n=0,1,2,\ldots$ \cite{Bruno:JSP-365,Golden:CMP-473}. Second,
these integral representations are \emph{linear} functionals of the
spectral measures. Finally, $\mu_{kk}$ and $\eta_{kk}$ are \emph{positive}
measures, in contrast to $\mu_{jk}$ and $\eta_{jk}$ for $j\neq k$. In this
section, we review how these three properties yield rigorous bounds
for the diagonal components of the effective parameters $\sigma^*_{kk}$ and
$\rho^*_{kk}$ \cite{Golden:CMP-473,Golden:JMPS-333}.




We start our discussion with the masses $\mu_{kk}^0$ and $\eta_{kk}^0$ of
the measures $\mu_{kk}$ and $\eta_{kk}$ for the continuum and 
lattice settings. By equation \eqref{eq:Measure_Mass_ConT} and  
the symmetries between the functions $F_{kk}(s)$ and $E_{kk}(s)$
displayed in equation \eqref{eq:Stieltjes_F}, in the continuum setting,
the masses $\mu_{kk}^0$ and $\eta_{kk}^0$ of the measures $\mu_{kk}$ and
$\eta_{kk}$ are generically given by $\mu_{kk}^0=p_1$ and $\eta_{kk}^0=p_2$, so that
%
\begin{align}\label{eq:Measure_Masses_Lattice}
  \mu_{kk}^0+\eta_{kk}^0=1, \quad  k=1,\ldots,d.
\end{align}
%
By equation \eqref{eq:Measure_Mass_theorem}, in
the finite lattice setting, we have $\mu_{kk}^0=d\,p_1^k$ generically. The masses
$\mu_{kk}^0$ and $\eta_{kk}^0$ of the measures $\mu_{kk}$ and $\eta_{kk}$ are
related in this finite lattice setting as follows. From equation
\eqref{eq:block_diag_chi} we have that $\chi_1^k(\omega)+\chi_2^k(\omega)=I_{L^d}$ for
all $k=1,\ldots,d$ and $\omega\in\Omega$, where $I_{L^d}$ is the identity matrix of
size $L^d\times L^d$. Consequently, by the linearity of the trace
operation, we have that
$\text{Trace}(\chi_1^k(\omega))+\text{Trace}(\chi_2^k(\omega))=\text{Trace}(I_{L^d})$,     
thus $N_1^k(\omega)+N_2^k(\omega)=L^d=N/d$. Averaging this formula
over $\Omega$ and rearranging yields equation \eqref{eq:Measure_Masses_Lattice},
where $\eta_{kk}^0=d\,p_2^k$ and $p_2^k=\langle N_2^k(\omega)\rangle/N$ is the average
number fraction of type-two bonds in the positive $k^{\text{th}}$
direction. For isotropic random media with finite lattice composite
microstructure, we have from 
\eqref{eq:Meas_mass_Isotropic_iid} that $\mu_{kk}^0=p_1$ and
$\eta_{kk}^0=p_2$. By the discussion in the paragraph following equation 
\eqref{eq:Meas_mass_Isotropic_iid}, the formulas $\mu_{kk}^0=d\,p_1^k$
and $\eta_{kk}^0=d\,p_2^k$ also hold for the infinite lattice setting
with $p_i^k=\lim_{N\to\infty}\langle N_i(\omega)\rangle/N$, $i=1,2$, and are given by
$\mu_{kk}^0=p_1$ and $\eta_{kk}^0=p_2$ for isotropic random media.  


 
For simplicity, we will focus on one diagonal component $\sigma^*_{kk}$ and
$\rho^*_{kk}$ of the effective conductivity and resistivity tensors
$\bsig^*$ and $\brho^*$, for some $k=1,\ldots,d$, and set $\sigma^*=\sigma_{kk}^*$,
$F(s)=F_{kk}(s)$, $m(h)=m_{kk}(h)$, $\mu=\mu_{kk}$, $E(s)=E_{kk}(s)$,
$\tilde{m}(h)=\tilde{m}_{kk}(h)$, and $\eta=\eta_{kk}$. 
Here,
%$\sigma^*=\sigma_2m(h)=\sigma_1/\tilde{m}(h)$,
$F(s)=1-m(h)$ and $E(s)=1-\tilde{m}(h)$.
We will also exploit the symmetries between
$F(s)$ and $E(s)$ in equation \eqref{eq:Stieltjes_F} and initially 
focus on the function $F(s)$ and the measure $\mu$, referring to the
function $E(s)$ and the measure $\eta$ where appropriate. 




Bounds for $\sigma^*$ are obtained as follows, while those for $\rho^*$ are
obtained analogously. The support of the measure 
$\mu$ is contained in the interval $[0,1]$ and its mass is given by
$\mu^0=p_1$, where $0\leq p_1\leq1$. Consider the set $\mathscr{M}$ of
positive Borel measures on $[0,1]$ with mass $\leq1$. By equation
\eqref{eq:Stieltjes_F}, for fixed $s\in\mathbb{C}\backslash[0,1]$, $F(s)$ is a
linear functional of the measure $\mu$, $F:\mathscr{M}\mapsto\mathbb{C}$, and
we write $F(s)=F(s,\mu)$ and $m(h)=m(h,\mu)$. Suppose that we know the
moments $\mu^n$ of the measure $\mu$ for $n=0,\ldots,J$. Define the set
$\mathscr{M}_J^\mu\subset\mathscr{M}$ of measures by
% 
\begin{align}\label{eq:Measure_Set}
  \mathscr{M}_J^\mu
     =\left\{\nu\in\mathscr{M} \ \Big| \   \int_0^1\lambda^n\d\nu(\lambda)=\mu^n, \  n=0,\ldots,J\right\}  . 
\end{align}
%
The set $A_J^\mu\subset\mathbb{C}$ that represents the possible
values of $m(h,\mu)=1-F(s,\mu)$ which is compatible with the
known information about the random medium is given by
%
\begin{align}\label{eq:Bounding_Set}
  A_J^\mu
     =\left\{\ m(h,\mu)\in\mathbb{C} \ | \
       \ h\not\in(-\infty,0], \ \mu\in \mathscr{M}_J^\mu\right\}. 
\end{align}
%



The set of measures $\mathscr{M}_J^\mu$ is a compact, convex
subset of $\mathscr{M}$ with the topology of weak convergence
\cite{Golden:CMP-473}. Since the mapping $F(s,\mu)$ in
\eqref{eq:Stieltjes_F} is linear in $\mu$ it follows that
$A_J^\mu$ is a compact convex subset of the complex plane
$\mathbb{C}$. The extreme points of $\mathscr{M}_0^\mu$ are the one 
point measures $a\delta_b$, $0\leq a,b\leq1$ \cite{Dunford_Schwartz:LinOp_PtI},
while the extreme points of $\mathscr{M}_J^\mu$ for $J>0$ are weak limits
of convex combinations of measures of the form
\cite{Karlin_Studden:Book:1966,Golden:CMP-473}  
%
\begin{align}\label{eq:Discrete_Measure}
  \d\mu_J(\lambda)=\sum_{i=1}^{J+1}a_i\delta_{b_i}(\d\lambda), \quad
  a_i\geq0, \quad 0\leq b_1<\cdots<b_{J+1}<1, \quad
  \sum_{i=1}^{J+1}a_ib_i^n=\mu^n,
%  \quad   n=0,1,2,\ldots.
\end{align}
%
for $n=0,1,\ldots,J$.


For the case of two-dimensional random media in the
continuous setting, every measure $\mu\in\mathscr{M}_J^\mu$ gives rise
to a function $m(h,\mu)$ that is the effective (relative) conductivity
of a multi-rank laminate \cite{MILTON:2002:TC}.
%The analogous result for $d>2$ is not known.
However, in general \cite{Golden:CMP-473},  not every measure
$\mu\in\mathscr{M}_J^\mu$ gives rise to such a function $m(h,\mu)$. Therefore,
the set $A_J^\mu$ will \emph{contain} the exact range of values of the
effective conductivity \cite{Golden:CMP-473}. This is sufficient for
the bounding procedure discussed in this section. 






By the symmetries between the formulas in equation
\eqref{eq:Stieltjes_F}, the support of the measure $\eta$ is contained in
the interval $[0,1]$ and its mass is given by $\eta^0=p_2=1-p_1$, where
$0\leq p_2\leq1$. We can therefore define compact, convex sets
$\mathscr{M}_J^\eta\subset\mathscr{M}$ and $A_J^\eta\subset\mathbb{C}$ which are
analogous to those defined in equations \eqref{eq:Measure_Set} and
\eqref{eq:Bounding_Set}, respectively, involving the function
$\tilde{m}(h,\eta)=1-E(s,\eta)$. Moreover, the extreme points of
$\mathscr{M}_0^\eta$ are the one point measures $c\delta_d$, $0\leq c,d\leq1$, 
while the extreme points of $\mathscr{M}_J^\eta$ are weak limits
of convex combinations of measures of the form given in equation
\eqref{eq:Discrete_Measure}. 



Consequently, in order to determine the extreme
points of the sets $A_J^\mu$ and $A_J^\eta$, it suffices to determine the
range of values in $\mathbb{C}$ of the functions $m(h,\mu_J)=1-F(s,\mu_J)$
and $\tilde{m}(h,\eta_J)=1-E(s,\eta_J)$, respectively, where  
%
\begin{align}\label{eq:Discrete_mh}
  F(s,\mu_J)=\sum_{i=1}^{J+1}\frac{a_i}{s-b_i}\,, \qquad
  E(s,\eta_J)=\sum_{i=1}^{J+1}\frac{c_i}{s-d_i}\,,
\end{align}
as the $a_i$, $b_i$, $c_i$, and $d_i$ vary under the
constraints given in equation  \eqref{eq:Discrete_Measure}. While
$F(s,\mu_J)$ and $E(s,\eta_J)$ in 
\eqref{eq:Discrete_mh} may not run over all points in $A_J^\mu$ and
$A_J^\eta$ as these parameters vary, they run over the
extreme points of these sets, which is sufficient due to their
convexity. It is important to note that, as the effective complex
conductivity $\sigma^*$ is given by $\sigma^*=\sigma_2m(h,\mu)=\sigma_1/\tilde{m}(h,\eta)$, the
regions $A_J^\mu$ and $A_J^\eta$ have to be mapped to the common
$\sigma^*$-plane to provide bounds for $\sigma^*$.    





We will discuss the bounds for $\sigma^*$ in detail for the cases where
$J=0,1$, and briefly explain how the procedure is generalized to
obtain a sequence of nested bounds for $J=2,3,\ldots$
\cite{Golden:JMPS-333}. The bounds corresponding to the case where $J=0$
follows from the knowledge of only the masses $\mu^0$ and $\eta^0$ of the
measures $\mu$ and $\eta$. For simplicity, we assume that $\mu^0=p_1$ and
$\eta^0=p_2$. If the random medium is also known to be statistically
isotropic, so that the effective tensors $\bsig^*$ and $\brho^*$ are
diagonal \cite{MILTON:2002:TC}, the first moments $\mu^1$ and $\eta^1$ are
also known to be given by \cite{Golden:JMPS-333}    
%
\begin{align}\label{eq:First_Moments}
  \mu^1=\frac{p_1p_2}{d}\,, \qquad
  \eta^1=\frac{p_1p_2(d-1)}{d}\,,
\end{align}
%
which leads to bounds for the case where $J=1$.




Consider the case where $J=0$  in \eqref{eq:Discrete_mh} and the
volume fraction $p_1=1-p_2$ is fixed with $\mu^0=p_1$ and
$\eta^0=p_2$, so that $F(s,\mu_J)=p_1/(s-\lambda)$ and
$E(s,\eta_J)=p_2/(s-\tilde{\lambda})$. By the above discussion, the values of 
$F(s,\mu$) and $E(s,\eta)$ lie inside the circles $C_0(\lambda)$ and
$\tilde{C}_0(\tilde{\lambda})$, respectively, given by  
%
\begin{align}\label{eq:0th_order_Bounds}
    C_0(\lambda)=\frac{\mu^0}{s-\lambda}\,, \quad -\infty\leq\lambda\leq \infty, \qquad
    \tilde{C}_0(\tilde{\lambda})=\frac{\eta^0}{s-\tilde{\lambda}}\,, \quad
    -\infty\leq\tilde{\lambda}\leq \infty. 
\end{align}
%
In the $\sigma^*$-plane, the intersection of these two regions is bounded by
two circular arcs corresponding to $0\leq\lambda\leq p_2$ and $0\leq\tilde{\lambda}\leq p_1$
in \eqref{eq:0th_order_Bounds}, and the values of $\sigma^*$ lie inside
this region \cite{Golden:JMPS-333}. These bounds are optimal
\cite{Milton:JAP-5286,Bergman:AP-78}, and are obtained by a composite
of uniformly aligned spheroids of material 1 in all sizes coated with
confocal shells of material 2, and vice versa. The arcs are traced out
as the aspect ratio varies. When the value of the component
conductivities $\sigma_1$ and $\sigma_2$ are real and positive, the bounding
region collapses to the interval
$1/(p_1/\sigma_1+p_2/\sigma_2)\leq\sigma^*\leq p_1\sigma_1+p_2\sigma_2$, which are the Wiener
bounds. The lower and upper bounds are obtained by parallel slabs of
the two materials aligned perpendicular and parallel to the field
$\vec{E}_0$, respectively \cite{Scaife-1989}.



Now consider the case where $J=1$ in
\eqref{eq:Discrete_mh}. Here, the volume fraction $p_1=1-p_2$ is
fixed so that $\mu^0=p_1$ and $\eta^0=p_2$, and the random medium is
statistically isotropic so that the first moments $\mu^1$ and $\eta^1$ are
given by that in equation \eqref{eq:First_Moments}.  A convenient way
of including this information is to use the transformations
\cite{Bergman:AP-78}  
%
\begin{align}\label{eq:Aux_Fs_Es}
  F_1(s)=\frac{1}{p_1}-\frac{1}{sF(s)}\,, \qquad
  E_1(s)=\frac{1}{p_2}-\frac{1}{sE(s)}\,.
\end{align}
%
Due to the symmetries between $F_1(s)$ and $E_1(s)$ in
\eqref{eq:Aux_Fs_Es}, we will first focus on the function $F_1(s)$ and
introduce the function $E_1(s)$ when appropriate. The function
$F_1(s)$ is an upper half plane function analytic off $[0,1]$ and
therefore has an integral representation
\cite{Bergman:AP-78,Golden:JMPS-333} analogous to that in equation
\eqref{eq:Stieltjes_F}, involving a
measure $\mu_1$, say, which is supported in the interval $[0,1]$. Since
only the mass $\mu^0=p_1$ and the first moment $\mu^1=p_1p_2/d$ of the
measure $\mu$ are known, the transformation \eqref{eq:Aux_Fs_Es}
determines only the mass $\mu_1^0=p_2/(p_1d)$ of the measure $\mu_1$
\cite{Bergman:AP-78,Golden:JMPS-333}. This reveals the utility of the
transformation $F_1(s)$ in equation \eqref{eq:Aux_Fs_Es}, it reduces the 
$J=1$ case for $F(s)$ to the $J=0$ case for $F_1(s)$.



By our previous analysis, the values of $F_1(s)$ lie inside a circle
$p_2/(p_1d(s-\lambda))$, $-\infty\leq\lambda\leq\infty$. Similarly, the values of $E_1(s)$ lie
inside a circle $p_1(d-1)/(p_2d(s-\tilde{\lambda}))$,
$-\infty\leq\tilde{\lambda}\leq\infty$. Since $F$ and $E$ are fractional linear in $F_1$ and
$E_1$, respectively, these circles are transformed to the circles
$C_1(\lambda)$ in the $F$-plane and $\tilde{C}_1(\tilde{\lambda})$ in the
$E$-plane given by \cite{Golden:JMPS-333}
%
\begin{align}\label{eq:Isotropic_Bounds}
  C_1(\lambda)=\frac{p_1(s-\lambda)}{s(s-\lambda-p_2/d)}\,, \quad  %&-\infty\leq\lambda\leq\infty, \quad
  \tilde{C}_1(\tilde{\lambda})=\frac{p_2(s-\tilde{\lambda})}{s(s-\tilde{\lambda}-p_1(d-1)/d)}\,,
   \qquad -\infty\leq\lambda,\tilde{\lambda}\leq\infty. %\notag
\end{align}
%
In the $\sigma^*$-plane the intersection of these two circular regions is
bounded by two circular arcs \cite{Golden:JMPS-333} corresponding to
$0\leq\lambda\leq(d-1)/d$ and $0\leq\tilde{\lambda}\leq1/d$ in \eqref{eq:Isotropic_Bounds}.




The vertices of the region,
$C_1(0)=p_1/(s-p_2/d)$ and $\tilde{C}(0)=p_2/(s-p_1(d-1)/d)$, are
attained by the Hashin--Shtrikman geometries (spheres of all
sizes of material 1 in the volume fraction $p_1$ coated with spherical
shells of material 2 in the volume fraction $p_2$ filling all of
$\mathbb{R}^d$, and vice versa), and lie on the arcs of the first
order bounds \cite{Golden:JMPS-333}. While there are at least five
points on the arc $C_1(\lambda)$ in \eqref{eq:Isotropic_Bounds} that are
attainable by composite microstructures \cite{Milton:JAP-5286}, the
arc $\tilde{C}_1(\tilde{\lambda})$ in \eqref{eq:Isotropic_Bounds} violates
\cite{Golden:JMPS-333} the interchange inequality $m(h)m(1/h)\geq1$
\cite{Keller:1964:TCC,Schulgasser:1976:CFR}, which becomes an equality
in two dimensions \cite{MILTON:2002:TC}.  Consequently, the isotropic
bounds in \eqref{eq:Isotropic_Bounds} are not optimal, but have been
improved \cite{Milton:APL-300,Bergman:AP-78} by incorporating the
interchange inequality. When $\sigma_1$ and $\sigma_2$ are real and positive
with $\sigma_1\leq\sigma_2$, the region collapses to the interval
%
\begin{align}
  \sigma_1+p_2\left/  \left(\frac{1}{\sigma_2-\sigma_1}+\frac{p_1}{d\sigma_1}\right)\right.
  \leq\sigma^*\leq 
  \sigma_2+p_1\left/  \left(\frac{1}{\sigma_1-\sigma_2}+\frac{p_2}{d\sigma_2}\right)\right.,
\end{align}
%
which are the Hashin--Shtrikman bounds.




The higher moments $\mu^n$, for $n\geq2$, depend on the $(n+1)$-point
correlation functions of the medium \cite{Golden:CMP-473} and have not
be calculated in general. Although, the interchange inequality forces
relations among them \cite{Milton:JAP-5294}. If the moments
$\mu^0,\ldots,\mu^J$ are known, then the transformation $F_1$ in
\eqref{eq:Aux_Fs_Es} can be iterated to produce an upper half plane
function $F_J$ with a integral representation, involving a positive
measure $\mu_J$ which is supported on the interval $[0,1]$. As in the
case where $J=1$, the first $J$ moments of the measure $\mu$ determine
only the mass $\mu_J^0$ of the measure $\mu_J$ \cite{Golden:JMPS-333}, and
the function $F_J(s)$ can easily be extremized by the above procedure,
and similarly for a function $E_J(s)$ associated with the moments
$\eta^0,\ldots,\eta^J$. The resulting bounds form a nested sequence of
lens-shaped regions \cite{Golden:JMPS-333}.




\section{Numerical Results}\label{sec:Numerical_Results}
%
In Sections \ref{sec:Finite_Lattice_Setting}--\ref{sec:Theorem_Proof}
we extended the ACM for representing transport in composites
to the case of two-phase random media with finite lattice composite 
microstructure. This led to discrete, Stieltjes integral
representations for the effective transport coefficients of such
media, involving spectral measures associated with the random
operators $M_i=\chi_i\Gamma\chi_i$ and $K_i=\chi_i\Upsilon\chi_i$, $i=1,2$. More specifically,
we demonstrated in Section \ref{sec:Finite_Lattice_Setting} that, in
this finite lattice setting, these random operators are represented by
random matrices. In Section \ref{sec:Unification_finite_infinite}, we
provided a novel formulation of the ACM, which holds for both the
matrix setting and the abstract linear operator setting discussed in
Sections \ref{sec:Continuum_Setting} and
\ref{sec:Infinite_Lattice_Setting}. In Section \ref{sec:Theorem_Proof}
we utilized this novel formulation of the ACM to prove Theorem 
\ref{thm:Discrete_Spectral_Theorem_ACM}, which was stated in Section  
\ref{sec:Finite_Lattice_Setting}. The proof of this theorem
establishes the existence of the integral representations for the
effective transport coefficients in the matrix setting, and
demonstrates that the underlying spectral measures are given
explicitly in terms of the eigenvalues and eigenvectors of the random
matrices. 


In this section, we utilize the mathematical framework described above
to compute spectral measures and effective transport coefficients
associated with the family of random bond lattices introduced in
Section \ref{sec:Finite_Lattice_Setting}. In particular, 
we developed in Section \ref{sec:Theorem_Proof} a numerically
efficient projection method, summarized by equations
\eqref{Pi_coordinates_E}--\eqref{eq:Fs_U1}, to facilitate such 
computations. Here, we employ this projection method to directly
compute spectral measures and effective transport coefficients
associated with this family of composites, which has various isotropic
and anisotropic finite lattice composite microstructures.





In order to explore the relationship between the values of the
effective transport coefficients and the associated bounds discussed
in Section \ref{sec:Bounding_Procedure}, we will focus on the diagonal
components of the effective tensors and the underlying spectral
measures, e.g., $\sigma^*_{kk}=\sigma_2m_{kk}(h)$ and $\mu_{kk}$ for $k=1,\ldots,d$. In
this section, the values of the component conductivities $\sigma_1$ and
$\sigma_2$ are taken to be that of the brine and pure ice phase,
respectively, for a sample of sea ice measured at a frequency of
4.75GHz \cite{Backstrom:2007:Book}, with $\sigma_1=51.0741+\imath\,45.1602$ and
$\sigma_2=3.07+\imath\,0.0019$, so that $s\approx-0.034+\imath\,0.032$. We stress that both
the values of the effective complex conductivity $\sigma^*_{kk}$ and
resistivity $\rho^*_{kk}$, as well as the associated bounds, depend on the
value of the contrast parameter $s=1/(1-\sigma_1/\sigma_2)$. 



We now discuss our numerical method for computing spectral
measures and effective transport coefficients in this matrix setting. 
Consider a two-phase random medium with finite lattice composite
microstructure, as described by the matrix $\chi_1(\omega)$ defined in
equation \eqref{eq:block_diag_chi}, for $\omega\in\Omega$. From
equation \eqref{eq:Stieltjes_F_Discrete}, we see that the spectral
measure $\mu_{kk}$, $k=1,\ldots,d$, for example, is an ensemble average of
spectral measures $\mu_{kk}(\omega)$ associated with the matrices
$M_1(\omega)=\chi_1(\omega)\,\Gamma\,\chi_1(\omega)$, for $\omega\in\Omega$. In particular, for fixed $\omega\in\Omega$,
the measure  $\mu_{kk}(\omega)$ is a weighted sum of $\delta$-measures centered at
the eigenvalues $\lambda_i(\omega)$ of $M_1(\omega)$, $i=1,\ldots,N$, with weights
$[\chi_1(\omega)\,Q_i(\omega)\hat{e}_k]\cdot\hat{e}_k$ involving the eigenvectors 
$\vec{u}_i(\omega)$ of $M_1(\omega)$ via $Q_i=\vec{u}_i\vec{u}_i^{\;T}$.



However, equation \eqref{eq:Projection_Eigenspace} implies that the
measure weights $[\chi_1(\omega)\,Q_k(\omega)\hat{e}_k]\cdot\hat{e}_k$ are 
identically zero for $i=1,\ldots,N_0(\omega)$. This was used in equation
\eqref{eq:Fs_U1} to show that the measure $\mu_{kk}(\omega)$ depends only on
the eigenvalues $\lambda^1_i(\omega)$, $i=1,\ldots,N_1(\omega)$, and eigenvectors
$\vec{u}^{\,1}_i(\omega)$ of the principle sub-matrix $\Gamma_1(\omega)$ of
$\Pi(\omega)M_1(\omega)\Pi^T(\omega)$, introduced in 
equation \eqref{eq:Spec_Decomp_chi_Gamma_chi_Proof}, and that the
measure weights may be expressed more explicitly as
$Q^1_i(\omega)\hat{e}_k^{\,\pi_1}\cdot\hat{e}_k^{\,\pi_1}$ with 
$Q^1_i=\vec{u}^{\,1}_i[\vec{u}^{\,1}_i]^T$.  Consequently,
for fixed $s\in\mathbb{C}\backslash[0,1]$, the value of
the effective complex conductivity $\sigma^*_{kk}=\sigma_2(1-F_{kk}(s))$ of
the medium can be obtained by computing $\lambda^1_i(\omega)$ and
$\vec{u}^{\,1}_i(\omega)$ for all $i=1,\ldots,N_1(\omega)$ and each $\omega\in\Omega$. Since the
computational cost of finding all the eigenvalues and eigenvectors of
a $N\times N$ real-symmetric matrix is $O(N^3)$ \cite{Demmel:1997}, this ``projection
method'' makes the numerical computation of $\mu_{kk}$ and $\sigma^*_{kk}$
much more efficient, especially for dilute systems where the size
$N_1(\omega)$ of the matrix $\Gamma_1(\omega)$ satisfies $N_1(\omega)\ll N$ for all $\omega\in\Omega$.   


For a random two-component bond lattice on $\mathbb{Z}^d_L$, with
dimension $d$ and size $L$, the cardinality $|\Omega|$ of the sample space
$\Omega$ of geometric configurations is 
given by $|\Omega|=2^N$, where $N=dL^d$. For large $N$, it
becomes numerically expensive to compute the eigenvalues and
eigenvectors of the matrix $\Gamma_1(\omega)$ for \emph{every} $\omega\in\Omega$. In our
numerical computations of the spectral measure $\mu_{kk}$, for example,
we instead used a reduced sample space $\Omega_0\subset\Omega$ of randomly generated
configurations of $\Omega$. For each $\omega\in\Omega_0$, \emph{all} of the eigenvalues
and eigenvectors of the matrix $\Gamma_1(\omega)$ were computed using the MATLAB
function \emph{eig()}. We used lattice sizes $L=60$ for $d=2$ and
$L=10$ -- $15$ for  
$d=3$ and typically averaged over $|\Omega_0|\sim10^4$ -- $10^5$ geometric
configurations. 


In order to visually determine the behavior of the function
$\mu_{kk}(\lambda)=\langle Q(\lambda)\hat{e}_k,\hat{e}_k\rangle_1$ underlying the spectral
measure $\mu_{kk}$, for a given random lattice, we plot a histogram
representation of $\mu_{kk}(\lambda)$ called the \emph{spectral function},
which we will also denote by $\mu_{kk}(\lambda)$. We now describe how 
we computed this graphical representation of the measure
$\mu_{kk}$. First, the spectral interval $[0,1]$ was divided into $R$
sub-intervals $I_r$, $r=1,\ldots,R$, of equal length $1/R$. Second, for
fixed $r$, we identified all of the eigenvalues that satisfy
$\lambda^{\,1}_i(\omega)\in I_r$, for $i=1,\ldots,N_1(\omega)$ and $\omega\in\Omega_0$. The assigned
value of $\mu_{kk}(\lambda)$ at the midpoint $\lambda$ of the interval $I_r$, is the
sum of the spectral weights
$Q^1_i(\omega)\hat{e}_k^{\,\pi_1}\cdot\hat{e}_k^{\,\pi_1}$ associated with all such
$\lambda^{\,1}_i(\omega)\in I_r$. In our computations of the spectral functions, we 
typically used $R\sim10^2$. As the system size increases, the eigenvalues
become increasingly dense in the spectral interval $[0,1]$. For a
large enough fixed system or for a random system averaged over many
statistical realizations, the spectral functions $\mu_{kk}(\lambda)$,
$k=1,\ldots,d$, begin to resemble smooth curves, as shown in Figure 
\ref{fig:Anisotropic_Spectral_Measures}.     



%
\begin{figure}[t]
  \centerline{\includegraphics[scale=0.74]{Anisotropic_RRN_p=0_5.eps}} 
\caption{Spectral measures and effective complex conductivities for
  anisotropic random media. Statistical realizations of the 2D square
  bond lattice for $p_1=0.5$ and various values of $p_1^k$, $k=1,2$,
  are displayed in (a). The type-one bonds are colored black, while
  the largest connected cluster of type-one bonds is colored grey. The
  corresponding spectral functions $\mu_{11}(\lambda)$ and $\mu_{22}(\lambda)$ are
  displayed in (b) and (c), respectively. The values of the
  effective complex conductivities $\sigma^*_{11}$ and $\sigma^*_{22}$ are
  displayed in (d) for $p_1^1=p_1^2$ along with the first-order
  bounds for $s\approx-0.034+\imath\,0.032$. The computed spectral functions have
  been rescaled so that   the area under the graph is the measure mass
  $\mu^0_{kk}=d\,p_1^k$.     
        }
\label{fig:Anisotropic_Spectral_Measures}
\end{figure}
%




%
\begin{figure}[t]
  \centerline{\includegraphics[scale=0.73]{A_Locally_Isotropic_RRN_11.eps}}
\caption{Spectral measures and effective complex conductivities and
  resistivities for locally isotropic random media. Realizations of
  the two-dimensional lattice model are displayed in (a). The type-one
  bonds are colored black, while the largest connected cluster of
  type-one bonds is colored grey. The corresponding spectral function
  $\mu_{11}(\lambda)$ is displayed in (b). The values of the effective complex
  conductivity $\sigma^*_{kk}$ and resistivity $\rho^*_{kk}$, $k=1,2$, are
  displayed in (c) and (d), respectively, along with the corresponding
  isotropic bounds for   $s\approx-0.034+\imath\,0.032$. The computed spectral
  functions have been rescaled so that the area under the graph is the
  measure mass $\mu^0_{11}=p_1$.    
        } 
\label{fig:LocIsotropic_RRN_11}
\end{figure}
%


   


In Figure \ref{fig:Anisotropic_Spectral_Measures}(a), statistical  
realizations of the anisotropic 2D bond lattice are displayed for
$L=60$ and a volume (number) fraction $p_1=0.5$ of type-one bonds,
with various values of $p_1^k$, $k=1,2$, the volume fraction of
type-one bonds in the positive $k^{\text{th}}$ direction. The type-one
bonds are colored black, while the largest connected cluster of
type-one bonds is colored grey. In Figure
\ref{fig:Anisotropic_Spectral_Measures}(b) and (c), we display the
behavior of the spectral functions $\mu_{11}(\lambda)$ and $\mu_{22}(\lambda)$,
respectively, as $p_1^k$ varies. In Figure
\ref{fig:Anisotropic_Spectral_Measures}(d), the computed values of the
effective complex conductivities $\sigma^*_{11}$ and $\sigma^*_{22}$
are displayed along with the first order bounds of equation
\eqref{eq:0th_order_Bounds} with $s\approx-0.034+\imath\,0.032$. These
bounds depend only on the mass $\mu^0_{kk}=d\,p_1^k$ of the measure
$\mu_{kk}$ and the value of the contrast parameter
$s=1/(1-\sigma_1/\sigma_2)$. Consistent with the symmetries of the
model, these spectral functions and effective complex conductivities 
satisfy $\mu_{11}(\lambda)=\mu_{22}(\lambda)$ and
$\sigma^*_{11}=\sigma^*_{22}$ for $p_1^1=p_1^2$ (to numerical accuracy
and statistical truncation). 



%
%
\begin{figure}[t]
  \centerline{\includegraphics[scale=0.69]{A_Duality_RRN_11.eps}}
\caption{Statistically self-dual random media. Realizations of various 
  2D lattice models are displayed in (a). The type-one
  bonds are colored black, while the largest connected cluster of
  type-one bonds is colored grey. The corresponding spectral function
  $\mu_{11}(\lambda)$ or $\kappa_{11}(\lambda)$ is displayed in (b). The values of the
  effective complex conductivity $\sigma^*_{11}$ or resistivity $\rho^*_{11}$,
  for $s\approx-0.034+\imath\,0.032$, are displayed in (c). Also displayed in
  (b) is the theoretical prediction for infinite, self-dual composite
  microstructures. The theoretical prediction for 
  the value of the effective complex conductivity or resistivity, as
  well as the first-order and isotropic bounds, are also displayed in
  (c). The computed spectral functions have been rescaled so that the
  area under the graph is the measure mass $\mu^0_{11}=p_1$.                
        }
\label{fig:Duality_RRN_11}
\end{figure}
%




We now consider the locally isotropic and statistically isotropic
composite classes introduced in Section
\ref{sec:Finite_Lattice_Setting}.
%after the statement of Theorem
%\ref{thm:Discrete_Spectral_Theorem_ACM}.
In Figure
\ref{fig:LocIsotropic_RRN_11}, we display the behavior of the 
spectral functions and the effective complex conductivity and
resistivity, as a function of $p_1$, for locally isotropic random
media with $L=60$. Statistical realizations of the composite microstructure
are displayed in Figure \ref{fig:LocIsotropic_RRN_11}(a), with the
same bond color scheme as that for Figure
\ref{fig:Anisotropic_Spectral_Measures}(a). The
associated spectral functions $\mu_{11}(\lambda)$ displayed in Figure
\ref{fig:LocIsotropic_RRN_11}(b) exhibit a rich 
resonance structure for small values of $p_1$. These so called
``geometric'' resonances have been attributed
\cite{Jonckheere_Luck_JPA_1998} to the recurrence of 
local geometric structures called ``fractal animals.''  Consistent
with isotropy, the behavior of the spectral function $\mu_{22}(\lambda)$ is
very similar to that of $\mu_{11}(\lambda)$ shown in Figure
\ref{fig:LocIsotropic_RRN_11}(b). The spectral functions $\mu_{kk}(\lambda)$,
$k=1,2$, were computed in \cite{Murphy:JMP:063506} for the case of
statistically isotropic random media. They look very similar to
$\mu_{11}(\lambda)$ in Figure \ref{fig:LocIsotropic_RRN_11}(b).  In Section
\ref{sec:Finite_Lattice_Setting} we noted that, in
\emph{two-dimensions}, the projection matrices $\Gamma$ and $\Upsilon$ 
are related by $\Upsilon=R^T\Gamma R$, where $R$ is $90^\circ$ rotation
matrix. As a consequence, the spectral functions $\kappa_{kk}(\lambda)$, $k=1,2$, for
2D locally and statistically \emph{isotropic} random media, look very
similar to $\mu_{11}(\lambda)$ displayed in Figure
\ref{fig:LocIsotropic_RRN_11}(b). In Figure
\ref{fig:LocIsotropic_RRN_11}(c) and (d), the values of the 
effective complex conductivities $\sigma^*_{kk}$ and resistivities
$\rho^*_{kk}$, $k=1,2$ are displayed, respectively,
along with the isotropic bounds from equation
\eqref{eq:Isotropic_Bounds} for $s\approx-0.034+\imath\,0.032$. Consistent with
isotropy, we have that $\sigma^*_{11}=\sigma^*_{22}$ and $\rho^*_{11}=\rho^*_{22}$ (to
numerical accuracy and statistical truncation) and
$\sigma^*_{kk}=1/\rho^*_{kk}$ to a relative error
$|\sigma^*_{kk}-1/\rho^*_{kk}|/|\sigma^*_{kk}|\lesssim10^{-2}$.







In the infinite lattice setting, the statistically and locally
isotropic composite microstructures are statistically self-dual
\cite{MILTON:2002:TC} for $d=2$ and $p_1=0.5$. Note that the class of
anisotropic random media for the special case of $p_1^k=p_1/d$, for all
$k=1,\ldots,d$, is also statistically isotropic and self-dual for  $d=2$ and
$p_1=0.5$. For such systems, the spectral measures and 
effective transport coefficients may be explicitly calculated
\cite{MILTON:2002:TC},
e.g. $\d\mu_{kk}(\lambda)=(\sqrt{(1-\lambda)/\lambda}\;)(\d\lambda/\pi)$ and  
$\sigma^*_{kk}=\sqrt{\sigma_1\sigma_2}\;$, $k=1,\ldots,d$. In particular, the spectral
measure $\mu_{kk}$ is absolutely continuous with respect to the Lebesgue
measure \cite{Folland:95}, with density $\mu_{kk}(\lambda)=(\sqrt{(1-\lambda)/\lambda}\;)/\pi$. 




These theoretical predictions, holding for
infinite systems, are displayed in Figure \ref{fig:Duality_RRN_11}
along with our computations of spectral functions and effective
transport coefficients for a finite system size $L=60$. Statistical 
realizations of the finite lattice microstructures 
are displayed in Figure \ref{fig:Duality_RRN_11}(a), with the same
bond color scheme as that for Figure
\ref{fig:Anisotropic_Spectral_Measures}(a). It is remarkable that even
for the finite system size $L=60$, the computed spectral functions
displayed in Figure \ref{fig:Duality_RRN_11}(b) agree quite well with
the theoretical duality prediction, which holds for infinite lattices. The  
anomalous difference between the theory and the numerical computation
seen in Figure \ref{fig:Duality_RRN_11}(b) for locally isotropic
random media, becomes less prominent as $L$ increases and is virtually
absent for $L=100$. In Figure \ref{fig:Duality_RRN_11}(c), 
the computed values of the effective 
transport coefficients are displayed along with the
duality prediction and the first-order
and isotropic bounds from equations \eqref{eq:0th_order_Bounds} and
\eqref{eq:Isotropic_Bounds}, respectively, with
$s\approx-0.034+\imath\,0.032$. The computed values of the effective
transport coefficients are in excellent agreement with that of the
duality prediction, which holds for infinite systems. The
deviation in the computed values of the effective parameters, relative
to the duality prediction, is typically $\lesssim10^{-2}$ for $L=60$
and decreases with increasing $L$.




The integral representation displayed in equation
\eqref{eq:Stieltjes_F_Discrete} is also valid for the effective
transport coefficients of two-phase random media with
\emph{three-dimensional}, finite lattice composite microstructure. We
now discuss our computations of spectral measures and effective
transport coefficients for such random media. Typical of numerical
simulations associated with three-dimensional systems, there are
fundamental numerical challenges that arise when extending our
spectral measure computations to 3D composite microstructures. These
challenges are consequences of the size $N=d\,L^d$ of the matrices
$M_1=\chi_1\Gamma\chi_1$ and $K_1=\chi_1\Upsilon\chi_1$, for example, which rapidly increases
with system size $L$ when $d=3$.




One challenge is the numerical cost of computing \emph{all} of the
eigenvalues and eigenvectors of a $N\times N$ real-symmetric matrix, which
is $O(N^3)$ \cite{Demmel:1997}. However, for the statistically
self-dual random lattices discussed above, the deviation in the
computed 
values of the effective parameters for $L=15$, relative to the
theoretical duality prediction for the \emph{infinite lattice}, is
typically $\lesssim10^{-1}$. This indicates that the computations of the
effective transport coefficients are reasonably accurate even 
for small system sizes $L$. Moreover, for random media with geometric
configurations that are statistically independent of each other, the
numerical computations of the associated eigenvalues and eigenvectors
can be performed in parallel.






Another challenge associated with a large matrix size $N$, is the
numerical accuracy of the computations. We computed the matrices
$\Gamma=\nabla(\Delta^{-1})\nabla^T$ and $\Upsilon=C(C^TC)^{-1}C^T$ using the MATLAB
\emph{mldivide} function $A\backslash B$, i.e., $\Gamma=\nabla(\Delta\backslash \nabla^T)$ and
$\Upsilon=C[(C^TC)\backslash C^T]$. Since $\nabla$ and $C$ are \emph{sparse} matrices with
\emph{integer elements}, the matrices $\Gamma$ and $\Upsilon$ were efficiently
computed using MATLAB's sparse architecture, which also reduces roundoff
error in the computations. The numerical accuracy of these ``matrix
inversions'' depends on the matrix condition number $\mathcal{K}(A)$,
for $A=\Delta,\;C^TC$. The matrix $A$ is said to be
\emph{well-conditioned} when $\mathcal{K}(A)$ is small and
\emph{ill-conditioned} when $\mathcal{K}(A)$ is large. One must always
expect to ``lose $\,\log_{10}\mathcal{K}(A)$ digits'' of accuracy in
computing the solution, except under very special circumstances
\cite{Trefethen:1997:NLA}. The numerical accuracy of the eigenvalue
problem for the matrices $M_1=\chi_1\Gamma\chi_1$ and $K_1=\chi_1\Upsilon\chi_1$ is also determined
by the associated eigenvalue condition numbers, which are the
reciprocals of the cosines of the angles between the left and right
eigenvectors. Large eigenvalue condition numbers of a symmetric matrix
$A$ implies that it is near a matrix with multiple eigenvalues, while
eigenvalue condition numbers $\approx1$ imply that the eigenvalue problem
is well-conditioned. 



We now discuss the condition numbers of the matrices $\Delta$ and $C^TC$ for
the system sizes considered in our computations. Recall for $d=2$ that
$C^TC=\Delta$. In this 2D case, $\mathcal{K}(\Delta)\sim10^3$ for 
$L=60$ and $L=100$. 
%$\mathcal{K}(\Delta)\approx1.6\times10^3$ for $L=60$ and $\mathcal{K}(\Delta)\approx4.4\times10^3$ for
%$L=100$.
In the 3D case $C^TC\neq\Delta$, and  
%$\mathcal{K}(\Delta)\approx51$ for $L=10$ and $\mathcal{K}(\Delta)\approx109$ for $L=15$.
$\mathcal{K}(\Delta)\sim10^1$ for $L=10$ and $\sim10^2$ for $L=15$,
%$\mathcal{K}(C^TC)\approx6.4\times10^5$ for $L=10$ and
%$\mathcal{K}(C^TC)\approx7.4\times10^6$ for $L=15$.
while $\mathcal{K}(C^TC)\sim10^6$ for $L=10$ and $\sim10^7$ for $L=15$.
These condition numbers were    
estimated using the MATLAB function \emph{condest()}. The eigenvalue
condition numbers for the matrices $M_1=\chi_1\Gamma\chi_1$ and $K_1=\chi_1\Upsilon\chi_1$
were computed using the MATLAB function \emph{condeig()}. They are all
$\approx1$ for the system sizes considered, indicating that the associated
eigenvalue problems are well-conditioned. In
summary, within the double precision architecture of MATLAB with a
\emph{machine epsilon} $\epsilon\sim10^{-16}$, for the system sizes $L$
considered, the spectral measure computations associated with the
matrices $M_1$ and $K_1$ are well-conditioned for
$d=2$. The spectral measure computations associated with the matrix
$M_1$ are also well-conditioned for $d=3$, while the spectral
measure computations associated with the matrix $K_1$ are
relatively ill-conditioned for $d=3$. The problem of finding an
appropriate \emph{preconditioner} for the matrix $C^TC$ in the 3D case
is a topic of current work.  



 






%
\begin{figure}[t]
  \centerline{\includegraphics[scale=0.73]{3D_Spectral_Measures_Gamma.eps}} 
\caption{Spectral measures and effective conductivities for 3D locally
  isotropic random media. The spectral function $\mu_{11}(\lambda)$
  is displayed in (a) for various volume fractions $p_1$ of type-one
  bonds. Computed values of the effective complex conductivity
  $\sigma^*_{kk}$, $k=1,\ldots,d$, are displayed in (b) along with the isotropic
  bounds, for $s\approx-0.034+\imath\,0.032$. The spectral functions have been
  rescaled so that the area under the graph is the measure mass
  $\mu^0_{11}=p_1$.                  
        }
\label{fig:3D_Spectral_Measures}
\end{figure}
%

%
\begin{figure}[t]
  \centerline{\includegraphics[scale=0.73]{3D_Spectral_Measure_GammaCurl.eps}} 
\caption{Spectral measures for 3D locally isotropic random media. The
  spectral function $\kappa_{11}(\lambda)$ is displayed for various volume
  fractions $p_1$ of type-one bonds. The spectral functions have been
  rescaled so that the area under the graph is the measure mass
  $\kappa^0_{11}=p_1$.                  
        }
\label{fig:3D_Spectral_Measure_Curl}
\end{figure}
%


Displayed in Figure \ref{fig:3D_Spectral_Measures} are 
computations of spectral functions and effective complex
conductivities, for three-dimensional locally isotropic random media
with $L=15$. Like its 2D counterpart, the spectral 
function $\mu_{11}(\lambda)$ displayed in Figure
\ref{fig:3D_Spectral_Measures}(a) has a rich resonant structure for
small values of $p_1$. Consistent with isotropy, the behavior of the
spectral functions $\mu_{kk}(\lambda)$ for $k=2,3$ are very similar to that of
$\mu_{11}(\lambda)$ shown in Figure \ref{fig:3D_Spectral_Measures}(a). The
spectral functions $\mu_{kk}(\lambda)$, $k=1,2,3$, were computed in
\cite{Murphy:JMP:063506} for the case of statistically isotropic
random media. They look very similar to $\mu_{11}(\lambda)$ in Figure
\ref{fig:3D_Spectral_Measures}(a). In Figure
\ref{fig:3D_Spectral_Measures}(b) the values of the effective complex
conductivities $\sigma^*_{kk}$, $k=1,2,3$, are displayed along with the
isotropic bounds from equation \eqref{eq:Isotropic_Bounds} for
$s\approx-0.034+\imath\,0.032$. Consistent with isotropy, $\sigma^*_{jj}=\sigma^*_{kk}$ for
all $j,k=1,2,3$ (to numerical accuracy and statistical truncation).




Displayed in Figure \ref{fig:3D_Spectral_Measure_Curl} are 
computations of the spectral function $\kappa_{11}(\lambda)$ associated with the
effective complex resistivity $\rho^*_{11}$, for 3D
locally isotropic random media with various values of $p_1$. In order
to increase the numerical stability of the computation, we reduced the 
system size from $L=15$ to $L=10$. The limited numerical accuracy
in the computation of $\Upsilon=C(C^TC)^{-1}C^T$, which is then propagated to
the eigenvalue problem for $K_1=\chi_1\Upsilon\chi_1$, has a smoothing effect, and
there are no prominent resonances in the spectral functions for small
values of $p_1$. This smoothing effect is also typical for
regularization of ill-posed inverse problems for the reconstruction of
spectral measures \cite{Cherkaev:IP-1203}. Consistent with isotropy,
$\kappa_{11}(\lambda)\approx\kappa_{33}(\lambda)$ and $\rho^*_{11}\approx\rho^*_{33}$. Although, due to the
limited accuracy of the computations, the behavior of $\kappa_{22}(\lambda)$ and
the value of $\rho^*_{22}$ is significantly different from that of the
other two components.  
 



We now discuss the gap behavior of the spectral measures
\cite{Murphy:JMP:063506,Jonckheere_Luck_JPA_1998} and the governing
role that it plays in critical transitions exhibited by the integral
representations for the effective transport coefficients
\cite{Murphy:JMP:063506,Golden:PRL-3935}. In the    
infinite lattice setting, the isotropic composite microstructures
discussed in this section are examples of lattice percolation models
\cite{Stauffer-92,Torquato:RHM-02}, which are parameterized by the
volume fraction $p_1=1-p_2$ of the constituents. In these 
lattice percolation models, the bonds are open with probability $p_1$,
say, and closed with probability $p_2$. Connected sets of open bonds
are called open clusters. The average cluster size grows as $p_1$
increases, and there is a critical probability $p_c\,$, $0<p_c<1$,
called the \emph{percolation threshold}, where an infinite cluster of
open bonds first appears. For the two-dimensional lattice percolation
model, $p_c=0.5$ and in three-dimensions $p_c\approx0.2488$
\cite{Stauffer-92,Torquato:RHM-02}.



Now consider transport through the
associated RRN, where the bonds are assigned electrical conductivities 
$\sigma_1$ with probability $p_1$ and $\sigma_2$ with probability $p_2$. The
effective conductivity $\sigma^*(p_1,h)$, for example, exhibits two types
of critical behavior as $h=\sigma_1/\sigma_2\to0$. First, when $\sigma_1\to0$ and
$0<\sigma_2<\infty$, $\sigma^*(p_1,0)=0$ for $p_1>p_c$ while $\sigma^*>0$ for
$p_1<p_c\,$. Second, when $\sigma_2\to\infty$ and $0<\sigma_1<\infty$, $\sigma^*(p_2,0) \to \infty$ as
$p_2\to p_c^-$. Since $s=1/(1-h)$ and $t=1-s$, we see
from equation \eqref{eq:Stieltjes_F} that the associated critical
behavior of the integral representations for
$m_{kk}(p_1,h)=\sigma^*(p_1,h)/\sigma_2$ and $w_{kk}(p_2,z)=\sigma^*(p_2,z)/\sigma_2$ as
$h\to0$ depends, in turn, on the behavior of the spectral measures
$\mu_{kk}(p_1)$ and $\alpha_{kk}(p_2)$ at the spectral endpoints $\lambda=0,1$. 


%
\begin{figure}[t]
  \centerline{\includegraphics[scale=0.68]{Effective_Parameter_Behavior_4_75GHz.eps}} 
\caption{Behavior of effective complex conductivities $\sigma^*_{kk}$,
  $k=1,\ldots,d$, with $s\approx-0.034+\imath\,0.032$, and the masses of the spectral
  measures $\mu_{kk}$ at $\lambda=0$, as a function of volume fraction $p_1$
  for 2D (a) and 3D (b) locally isotropic random resistor network.   
        }
\label{fig:Effective_Parameter_Behavior}
\end{figure}
%



%
\begin{figure}[t]
  \centerline{\includegraphics[scale=0.97]{Spectral_Function_Symmetries.eps}} 
\caption{Spectral measure symmetries. Transformations of the computed
  spectral functions for 2D (a) and 3D (b) random resistor network,
  for various values of the volume fraction $p_1$. The computed
  spectral functions have been rescaled to make the area under the
  graph the measure mass.   
        }
\label{fig:Spectral_Function_Symmetries}
\end{figure}
%



Consider the behavior of the spectral measure $\mu_{11}$ at the spectral
endpoints $\lambda=0,1$ for the 2D lattice percolation model. In Figures
\ref{fig:LocIsotropic_RRN_11}(b) and \ref{fig:Duality_RRN_11}(b) we
see that, as $p_1$ increases from zero  
and the system becomes increasingly connected, gaps in the spectral
function $\mu_{11}(\lambda)$ at the spectral endpoints $\lambda=0,1$ shrink and then
vanish  symmetrically at a value of $p_1=p_c=0.5$. The graphs of 
these spectral functions indicate that the vanishing of the spectral
gaps leads to a buildup in the mass of the measure at $\lambda=0$, while the
mass of the measure is approximately zero for $\lambda=1$, i.e.,
$\mu_{11}(1)\approx0$. Moreover, as $p_1$ increases beyond the percolation
threshold $p_c$, the mass of the measure at $\lambda=1$ remains
approximately zero, while the buildup of the measure mass at
$\lambda=0$ \emph{persists and grows}. 



Now consider the behavior of the spectral measures $\mu_{11}$ and
$\kappa_{11}$ at the spectral
endpoints $\lambda=0,1$ for the 3D lattice percolation model. In Figure
\ref{fig:3D_Spectral_Measures}(a), we see as $p_1$ increases from
zero and approaches the percolation threshold $p_c\approx0.2488$, a spectral
gap about $\lambda=0$ shrinks and then vanishes, leading to a
buildup in the mass of the measure $\mu_{11}$ at $\lambda=0$ for $p_1=p_c$. As
$p_1$ increases beyond $p_c$, the mass of $\mu_{11}$ at $\lambda=0$ continues
to grow, while a spectral gap at $\lambda=1$ shrinks and
then vanishes for $p_1=1-p_c\approx0.7512$, with $\mu_{11}(1)\approx0$. The
spectral function $\kappa_{11}(\lambda)$ displayed in Figure
\ref{fig:3D_Spectral_Measure_Curl} has an analogous transitional
behavior. In particular, as $p_1$ increases from zero and approaches
the percolation threshold $p_c\approx0.2488$, a spectral gap about $\lambda=1$
shrinks and then vanishes, with $\mu_{11}(1)\approx0$. As $p_1$ increases
beyond $p_c$ and approaches $1-p_c\approx0.7512$, a spectral gap about $\lambda=0$
shrinks and vanishes, leading to a buildup in the mass of the measure
$\kappa_{11}$ at $\lambda=0$. Since $t=1-s$, it is physically consistent that the
roles of the spectral endpoints for $\kappa_{11}$ have swithced from that
of $\mu_{11}$.     


For finite lattice systems, the existence of gaps in the spectrum of
$\mu_{11}$ about $\lambda=0,1$ for $p_1\ll1$, as well as their
collapse as $p_1\to1$, is a direct consequence \cite{Murphy:JMP:063506} of the
projective nature of the matrices $\chi_1$ and $\Gamma$. However, it has been argued
that, for infinite lattice percolation models, the spectrum of $\mu_{11}$
extends all the way to the spectral endpoints $\lambda=0,1$, with
exponentially decaying \emph{Lifshitz tails} for all $0<p_1\ll1$. The
detailed nature of the Lifshitz tails was numerically verified in
\cite{Jonckheere_Luck_JPA_1998} for the finite, 2D lattice percolation
model for $p_1=$0.05, 0.1, 0.15, and 0.2, demonstrating that this
behavior of $\mu_{11}$ is present even in the finite lattice
setting. The presence of Lifshitz tails in $\mu_{11}$ for $p_1<p_c$
explains the presence of measure masses near $\lambda=0$, shown as vertical
lines in the spectral function $\mu_{11}(\lambda)$ displayed in Figures
\ref{fig:LocIsotropic_RRN_11} and \ref{fig:3D_Spectral_Measures}.   

 





Displayed in Figure \ref{fig:Effective_Parameter_Behavior} is the
behavior of the effective complex conductivity $\sigma^*_{kk}$ and the
mass of the measure $\mu_{kk}$ concentrated at $\lambda=0$, for $k=1,\ldots,d$, as
a function of volume fraction $p_1$, for the 2D (a) and 3D (b) locally
isotropic lattice percolation models in Figures
\ref{fig:LocIsotropic_RRN_11}--\ref{fig:3D_Spectral_Measures}. 
It can be seen in Figure \ref{fig:Effective_Parameter_Behavior} that a
very small fraction of the 
measure mass is concentrated at $\lambda=0$ for $p_1<p_c$, where $p_c=0.5$
for 2D and $p_c\approx0.2488$ for 3D. However, as $p_1$ surpasses $p_c$, a
significant amount of the measure mass becomes concentrated at the
spectral endpoint $\lambda=0$. This $\delta$-function behavior in the measure at
$\lambda=0$ leads to large changes in the value of effective complex
conductivity $\sigma^*_{kk}$ as the volume fraction $p_1$ surpasses
$p_c$. The associated mass of $\mu_{kk}$ concentrated at $\lambda=1$ is
$\lesssim10^{-30}$ for both the 2D and 3D lattices.  










The gap behavior of the spectral measures discussed above is
consistent with equation \eqref{eq:Measure_Relations}, which holds for
general stationary random media in the infinite setting
\cite{Murphy:JMP:063506}, and consequently holds for percolation
models of such media. This equation characterizes the percolation
transition with the formation of delta components in the spectral
measures at the spectral endpoints $\lambda=0,1$, \emph{precisely} at
$p_1=p_c$ and $p_1=1-p_c$. More specifically, recall that the weights
$m_{kk}(0)$ and $w_{kk}(0)$ of the delta components at $\lambda=0$ and $\lambda=1$ in
\eqref{eq:Measure_Relations}, for example, have the following
behavior. When $\sigma_1=0$ $(h=0$), the function
$m_{kk}(0)=m_{kk}(p_1,0)$, $k=1,\ldots,d$, increases from zero as 
$p_1$ surpasses $p_c$ $(p_1\to p_c^+)$. Similarly, when $\sigma_2=0$
$(z=0)$, the function $w_{kk}(0)=w_{kk}(p_2,0)$ increases from zero as
$p_1$ surpasses $1-p_c$ $(p_1\to1-p_c^-)$. For conductor/insulator or
conductor/superconductor systems, this behavior in the spectral
endpoints of the measures leads to critical behavior in the effective
conductivity \cite{Murphy:JMP:063506,Golden:JMP-5627}. 




Equation \eqref{eq:Measure_Relations}, which holds for infinite
systems, also provides a relationship 
between the measures $\mu_{kk}(p_1)$ and $\alpha_{kk}(p_2)$, and the measures
$\kappa_{kk}(p_1)$ and $\eta_{kk}(p_2)$. In Figure \ref{fig:Spectral_Function_Symmetries} we
demonstrate that this relationship between the spectral measures
persists in the finite lattice setting. Displayed in Figure
\ref{fig:Spectral_Function_Symmetries}(a) are graphs of 
transformations of the spectral function $\kappa_{22}(\lambda)$ for the 2D
lattice percolation model. In particular,
the graph of the function $(1-\lambda)\kappa_{22}(1-\lambda)$ is displayed for
volume fractions $p_1=0.1$, 0.3, and 0.5, along with $\lambda\kappa_{22}(\lambda)$ for
volume fractions $1-p_1=0.9$, 0.7, and 0.5. Similarly, in Figure
\ref{fig:Spectral_Function_Symmetries}(b) the graphs of
$(1-\lambda)\mu_{33}(1-\lambda)$ and $\lambda\mu_{33}(\lambda)$ are displayed for the 3D lattice
percolation model with various values
of $p_1$ and $1-p_1$, respectively. The graphs of the 
transformed spectral functions are virtually identical except for a
``$\delta$-function'' at $\lambda=0$, in excellent agreement with
\eqref{eq:Measure_Relations}. We conclude this section by noting that,
despite the lack of numerical accuracy in our computations of the
spectral function $\kappa_{22}(\lambda)$ for 3D finite lattice composite
microstructures, the functions $(1-\lambda)\kappa_{22}(p_1,1-\lambda)$ and
$\lambda\kappa_{22}(p_2,\lambda)$ are also virtually identical, other than a singularity
at $\lambda=0$. 








\section{Conclusion}
%
In Sections \ref{sec:Continuum_Setting} and
\ref{sec:Infinite_Lattice_Setting} we reviewed and extended the ACM for
representing transport in two-phase random media, for the
\emph{infinite} continuum and lattice settings, respectively. This
method provides the Stieltjes integral representations displayed in
equation \eqref{eq:Stieltjes_F} for the effective transport
coefficients of such composite media, which involve spectral measures
associated with the self-adjoint random operators $M_i=\chi_i\Gamma\chi_i$ and
$K_i=\chi_i\Upsilon\chi_i$. Here, $\chi_i$ is the characteristic function for material
phase $i=1,2$ and the operators $\Gamma=\vec{\nabla}(\Delta^{-1})\vec{\nabla}\cdot$ and
$\Upsilon=-\vec{\nabla}\times(\bDelta^{-1})\vec{\nabla}\times$ act as projectors onto curl-free
and divergence-free fields, respectively. 






In Section \ref{sec:Finite_Lattice_Setting} we developed the ACM for
representing transport in two-phase random media, with \emph{finite}
lattice composite microstructure. This novel formulation yields
discrete Stieltjes integral representations for the effective
transport coefficients of such media, displayed in equation
\eqref{eq:Stieltjes_F_Discrete} of Theorem
\ref{thm:Discrete_Spectral_Theorem_ACM}, which is a key theoretical
contribution of this work. We accomplished this by developing a novel
formulation of the ACM in Section
\ref{sec:Unification_finite_infinite}, that is equivalent to the
original formulation \cite{Golden:CMP-473}, which holds for both the
finite lattice setting and the infinite, continuum and lattice
settings. We also provided a projection method for numerically
efficient, rigorous computation of spectral measures and effective
parameters for composite media with finite lattice
microstructure. This projection method is summarized by equations 
\eqref{Pi_coordinates_E}--\eqref{eq:Fs_U1}. In this finite lattice
case, the operators $\chi_i$, $\Gamma$, and $\Upsilon$ are represented by
real-symmetric projection matrices, and the spectral measures of the
associated real-symmetric random matrices $M_i$ and $K_i$ are given
explicitly in terms of their eigenvalues and eigenvectors, as
displayed in equation \eqref{eq:Stieltjes_F_Discrete}.  



In Section \ref{sec:Finite_Lattice_Setting}, following the statement
of Theorem \ref{thm:Discrete_Spectral_Theorem_ACM}, we introduced three 
families of locally isotropic, statistically isotropic, and
anisotropic random media with finite lattice composite microstructure. In Section
\ref{sec:Numerical_Results} we employed the projection method to
compute the spectral measures and effective parameters associated with
these families of random media. To our knowledge, this is the first
time that the spectral measures $\eta_{kk}$ and $\kappa_{kk}$ underlying the
effective complex resistivity $\rho^*_{kk}$ have been computed for such
composite microstructures. These computations not only demonstrate
several important properties of the spectral measures and effective
parameters, but they also serve as a consistency check to the theory
developed here. 



The computed spectral functions and effective complex parameters for
anisotropic random media, displayed in Figure
\ref{fig:Anisotropic_Spectral_Measures},  are consistent with the
symmetries of the model. Consistent with general theory
\cite{MILTON:2002:TC}, our computations of the effective parameters
for isotropic random media 
satisfy $\sigma^*_{kk}=1/\rho^*_{kk}$, $k=1,\ldots,d$ (to numerical accuracy and
statistical truncation). Moreover, the computed spectral functions and
effective parameters are consistent with isotropy and satisfy
$\mu_{jj}(\lambda)=\mu_{kk}(\lambda)$ and $\sigma^*_{jj}=\sigma^*_{kk}$, for example, for all
$j,k=1,\ldots,d$ (to numerical accuracy, finite size effects, and
statistical truncation). Figure
\ref{fig:Duality_RRN_11} demonstrates that the projection method
accurately calculates the spectral measures and 
effective parameters for statistically self-dual composite
microstructures. Furthermore, Figure
\ref{fig:Spectral_Function_Symmetries} shows that the computed 
spectral measures are in excellent agreement with equation
\eqref{eq:Measure_Relations}, which holds for general stationary
two-phase random media \cite{Murphy:JMP:063506}. 



The self-consistent mathematical framework developed here helps lay
the groundwork for studies in the effective transport properties of a
broad range of important composites, such as electrorheological fluids
\cite{Murphy:PHD_Thesis}, multiscale sea ice structures
\cite{Murphy_Multiscale_Sea_Ice}, and bone
\cite{Golden:JBM:337}. Remarkably, the ACM  has also been adapted to
provide Stieltjes integral representations for  effective transport
coefficients underlying a wide variety of transport processes, such
as: the effective diffusivity for steady
\cite{McLaughlin:SIAM_JAM:780,Avellaneda:CMP-339,Murphy_Advective_Diffusion}
and time-dependent
\cite{Avellaneda:PRE:3249,Murphy_Advective_Diffusion_Dynamic}
fluid velocity fields, the effective complex permittivity for uniaxial
polycrystalline media
\cite{Barabash:JPCM:10323,Gully_Golden_Polycrystalline,Murphy_Polycrystalline_Media_ACM},
and the effective elastic moduli of two-phase elastic composites
\cite{Ou:MMAS:655,Ou:2012:411}. The Golden-Papanicolaou formulation of
the ACM has been pivotal in the development of these mathematical
frameworks, and in the understanding of these important transport
processes.  

% redefine the command that creates the equation no.
  \setcounter{equation}{1}  % reset equation counter
  \setcounter{section}{0}  % reset section counter
  \renewcommand{\theequation}{A-\arabic{equation}} 
\renewcommand{\thesection}{A-\arabic{section}}
\section{Appendix: The Spectral Theorem} 
\label{sec:The_Spectral_Theorem}
%
In equations \eqref{eq:Stieltjes_F} and
\eqref{eq:Stieltjes_F_Discrete} of Sections 
\ref{sec:Continuum_Setting} and \ref{sec:Finite_Lattice_Setting},
we display integral 
representations for the functions $F_{jk}(s)$ and
$G_{jk}(t)$, $j,k=1,\ldots,d$, involving spectral measures $\mu_{jk}$ and
$\alpha_{jk}$ associated with the operators $M_i=\chi_i\Gamma\chi_i$, $i=1,2$, as
well as that for the functions 
$E_{jk}(s)$ and $H_{jk}(t)$ involving spectral measures $\eta_{jk}$ and
$\kappa_{jk}$ associated with the operators $K_i=\chi_i\Upsilon\chi_i$, $i=1,2$. In this section, we
discuss the spectral theorem as it pertains to the ACM, which
provides the existence of these Stieltjes integral representations. The abstract,
bounded linear self-adjoint operator case \cite{Reed-1980,Stone:64},
associated with the infinite, continuum and lattice settings, is
discussed in Section \ref{sec:The_Spectral_Theorem_Continuum}. While
the real-symmetric matrix case
\cite{Halmos-1958,Kreyszig:JWS-1989,Stakgold:BVP:2000}, associated with
the finite lattice setting, is discussed in Section
\ref{sec:The_Spectral_Theorem_Finite_Lattice}. Since the formulations
associated with each of the operators $M_i=\chi_i\Gamma\chi_i$ and $K_i=\chi_i\Upsilon\chi_i$,
$i=1,2$, are analogous, for simplicity, we will focus on that for the
operator $M_1=\chi_1\Gamma\chi_1$. Also, in Section
\ref{sec:Unification_finite_infinite} we provided a novel formulation
of the ACM involving the operator $M_1$, which is 
equivalent to the original formulation 
\cite{Golden:CMP-473,Bruno:PRSLA-353} involving the operator $\Gamma\chi_1$, and holds for
both the finite lattice setting and the infinite, continuum and lattice
settings. Due to this unification, we will focus on the formulation
of the ACM associated with the operator $M_1$. 
%
\subsection{Infinite Continuum and Lattice Settings}
\label{sec:The_Spectral_Theorem_Continuum} 
%
In this section, we review the spectral theorem as it pertains to the
ACM for the infinite, continuous and lattice settings. Consider the
Hilbert space $\mathscr{H}_\times$ defined in equation
\eqref{eq:curlfreeHilbert}. Now, define the Hilbert space
$\mathscr{H}_0=\mathscr{H}_\times\cup\mathbb{C}^d$ by 
%
\begin{align}\label{eq:Hilbert0}
  \mathscr{H}_0=
  \left\{\vec{Y}\in \mathscr{H} \ | \ \vec{\nabla} \times\vec{Y}=0 \text{ weakly}
  \right\},
\end{align}
%
where $\vec{\nabla} \times\vec{Y}=0$ means that $L_iY_j-L_jY_i=0$ for all $i,j=1,\ldots,d$.
In other words, $\mathscr{H}_0$ is the Hilbert space $\mathscr{H}_\times$
with the constant fields $\mathbb{C}^d$ included. Equip $\mathscr{H}_0$ with the
$\mathscr{H}$-inner-product weighted by the characteristic function 
$\chi_1$, which we denote by $\langle\cdot,\cdot\rangle_1$. In the infinite, continuum and 
lattice settings, the characteristic function acts \emph{pointwise} on
the underlying vector space, $\mathbb{R}^d$ or $\mathbb{Z}^d$, and it is
therefore a self-adjoint operator on $\mathscr{H}_0$. Clearly, it is
also a linear projection operator satisfying
$\langle\chi_1\vec{\xi},\vec{\zeta}\,\rangle=\langle\chi_1^2\vec{\xi},\vec{\zeta}\,\rangle$ for all
$\vec{\xi},\vec{\zeta}\in\mathscr{H}_0$, and is therefore bounded on
$\mathscr{H}_0$ with operator norm $\|\chi_1\|\leq1$.

On $L^2(\Omega,P)$, the linear operator $\Delta^{-1}$ is bounded and self-adjoint 
\cite{Stakgold:BVP:2000}. For all $\vec{\xi}\in\mathscr{H}_0$ we
have $\vec{\nabla}\cdot\vec{\xi}\in L^2(\Omega,P)$, and for all $\zeta\in L^2(\Omega,P)$ we have
$\|\vec{\nabla}\Delta^{-1}\zeta\|<\infty$, where $\|\cdot\|$ denotes the norm induced by the
$\mathscr{H}$-inner-product. It follows that  the linear operator
$\Gamma=\vec{\nabla}(\Delta^{-1})\vec{\nabla}\cdot$ is bounded on
$\mathscr{H}_0$. Integration by parts then
establishes that $\Gamma$ is self-adjoint on $\mathscr{H}_0$
\cite{Golden:CMP-473}. It is also clear that $\Gamma$ is a 
projection operator satisfying
$\langle\Gamma\vec{\xi},\vec{\zeta}\,\rangle=\langle\Gamma^2\vec{\xi},\vec{\zeta}\,\rangle$ for all
$\vec{\xi},\vec{\zeta}\in\mathscr{H}_0$, with operator norm $\|\Gamma\|\leq1$.  



It follows that $M_1=\chi_1\Gamma\chi_1$ is a bounded linear self-adjoint
operator on the Hilbert space $\mathscr{H}_0$, with operator norm
$\|M_1\|\leq1$ \cite{Reed-1980,Stone:64}. The spectrum $\Sigma$ of the
self-adjoint operator $M_1$ is real-valued and the spectral radius of $M_1$ is equal to
its operator norm \cite{Reed-1980}, which implies that
$\Sigma\subseteq[-1,1]$. However, since $\chi_1$ and $\Gamma$ are self-adjoint projection
operators on $\mathscr{H}_0$, we have 
$\langle\chi_1\Gamma\chi_1\vec{\xi},\vec{\xi}\,\rangle=\langle\Gamma\chi_1\vec{\xi},\Gamma\chi_1\vec{\xi}\,\rangle=\|\Gamma\chi_1\vec{\xi}\,\|^2\geq0$ 
for all $\vec{\xi}\in\mathscr{H}_0$. This implies that $M_1$ is also a
positive operator, which implies that its spectrum satisfies $\Sigma\subseteq[0,\infty)$
\cite{Stone:64}.  Consequently, the spectrum $\Sigma$ of $M_1$ satisfies
$\Sigma\subseteq[0,1]$.       



Since $\Sigma\subseteq[0,1]$, the spectral theorem for bounded linear self-adjoint
operators in Hilbert space \cite{Stone:64} states that there is a
one-to-one correspondence between the operator $M_1$ and a family of
self-adjoint projection operators $\{Q(\lambda)\}_{\lambda\in[0,1]}$ --- the
resolution of the identity --- that satisfies $\lim_{\lambda\to0}Q(\lambda)=0$ and
$\lim_{\lambda\to1}Q(\lambda)=I$, where $0$ and $I$ are the null and identity
operators on $\mathbb{R}^d$. Furthermore, for all
$\vec{\xi},\vec{\zeta}\in\mathscr{H}_0$, the function of $\lambda$ defined by 
$\mu_{\xi\zeta}(\lambda)=\langle Q(\lambda)\vec{\xi},\vec{\zeta}\,\rangle_1$ is strictly increasing and of
bounded variation, and therefore has a Stieltjes measure $\mu_{\xi\zeta}$
associated with it \cite{Stieltjes:1995,Stone:64,Folland:95}. The
spectral theorem also states that, for all complex
valued functions $f\in L^2(\mu_{\xi\zeta})$, there exists a linear operator
denoted by $f(M_1)$ which is defined in terms of the functional
$\langle f(M_1)\vec{\xi}\cdot\vec{\zeta}\,\rangle_1=\langle f(M_1)\chi_1\vec{\xi}\cdot\vec{\zeta}\,\rangle$. Moreover,
this functional has the following integral representation involving the
Stieltjes measure $\mu_{\xi\zeta}$  
% 
\begin{align}\label{eq:Spectral_Theorem}  
  \langle f(M_1)\,\vec{\xi}\cdot\vec{\zeta}\,\rangle_1= \int_0^1f(\lambda)\d\mu_{\xi\zeta}(\lambda), \quad
  \mu_{\xi\zeta}(\lambda)=\langle Q(\lambda)\vec{\xi},\vec{\zeta}\,\rangle_1,
\end{align}
%
where the integration is over the spectrum $\Sigma$ of $M_1$
\cite{Reed-1980,Stone:64}. Setting $f(\lambda)=(s-\lambda)^{-1}$ for 
$s\in\mathbb{C}\backslash[0,1]$, $\vec{\xi}=\vec{e}_j$, and $\vec{\zeta}=\vec{e}_k$ in
equation \eqref{eq:Spectral_Theorem}, yields the integral formula for
$F_{jk}(s)$ displayed in equation \eqref{eq:Stieltjes_m}, which is
equivalent to that displayed in \eqref{eq:Stieltjes_F}. It is now
clear why the Hilbert space $\mathscr{H}_\times$ was extended to
$\mathscr{H}_0$ in our formulation of the spectral theorem for the
ACM: the appearance of the \emph{constant} fields $\vec{e}_j$,
$j=1,\ldots,d$, in equation \eqref{eq:Stieltjes_m}.




\subsection{Finite Lattice Setting}
\label{sec:The_Spectral_Theorem_Finite_Lattice}
%
In this section, we review the spectral theorem as it pertains to the
ACM for the finite lattice setting discussed in Section
\ref{sec:Finite_Lattice_Setting}. In this case, the operator
$M_1=\chi_1\Gamma\chi_1$ is represented by a real-symmetric random matrix and the
spectral theorem for such matrices provides a discrete version of the
integral representation displayed in equation
\eqref{eq:Spectral_Theorem}. This formulation leads to the discrete
integral representation of the function $F_{jk}(s)$ displayed in
equations \eqref{eq:Stieltjes_F_Discrete} and \eqref{eq:Stieltjes_m}.   



Recall that we defined in Section \ref{sec:Finite_Lattice_Setting} a
bijective mapping $\Theta:\mathbb{Z}_L^d\to\mathbb{N}_L$ from the finite
$d$-dimensional bond lattice $\mathbb{Z}_L^d$ of size $L$ onto the
one dimensional set $\mathbb{N}_L$ of size $N=d\,L^d$. Moreover, we showed that, under the
mapping $\Theta$, the random operator $M_1=\chi_1\Gamma\chi_1$ can be represented by a
random matrix of size $N\times N$
\cite{Golden:JBM:337,Murphy:JMP:063506}. More specifically, $\Gamma$ 
is a \emph{non-random}, real-symmetric projection matrix  satisfying
$\Gamma^2=\Gamma$. Consequently, $\|\Gamma\|\leq1$, where $\|\cdot\|$ denotes the matrix norm
induced by the dot-product on $\mathbb{C}^N$
\cite{Demmel:1997}. In this finite lattice setting, the characteristic
function $\chi_1$ is represented by a \emph{random}, diagonal projection
matrix satisfying $\chi_1^2=\chi_1$, with zeros and ones along its
diagonal. Consequently, the matrix $\chi_1$ is real-symmetric and satisfies
$\|\chi_1\|\leq1$. 



It follows that $M_1$ is a real-symmetric matrix with $\|M_1\|\leq1$
\cite{Demmel:1997}. It is also a composition of projection matrices,
and is consequently positive definite, i.e., for every
$\vec{\xi}\in\mathbb{C}^N$ we have that
$\chi_1\Gamma\chi_1\vec{\xi}\cdot\vec{\xi}=(\Gamma\chi_1\vec{\xi}\;)\cdot(\Gamma\chi_1\vec{\xi}\;)\geq0$. This,
implies that the spectrum $\Sigma$ of $M_1$ is comprised of real
eigenvalues $\lambda_i$, $i=1,\ldots,N$, and that $\Sigma\subseteq[0,\infty)$
\cite{Horn_Johnson-1990}. Furthermore, the largest eigenvalue of the
matrix $M_1$ is equal to $\|M_1\|$ \cite{Demmel:1997}. It follows that $\Sigma\subseteq[0,1]$.   






It is well known \cite{Horn_Johnson-1990,Keener-2000} that the
eigenvectors $\vec{u}_i$, 
$i=1,\ldots,N$, of the real-symmetric matrix $M_1$ form an orthonormal basis for
$\mathbb{R}^N$, i.e., $\vec{u}_\ell^{\;T}\vec{u}_m=\delta_{\ell m}$  and for every
$\vec{\xi}\in\mathbb{R}^N$ we have
$\vec{\xi}=\sum_{i=1}^N(\vec{u}_i^{\;T}\vec{\xi}\;)\vec{u}_i
=\left(\sum_{i=1}^N\vec{u}_i\vec{u}_i^{\;T}\right)\vec{\xi}\,$. Consequently,         
%
\begin{align}\label{eq:Matrix_Rep_Spec_Theorem}
  \sum_{i=1}^NQ_i=I, \qquad
  Q_i=\vec{u}_i\vec{u}_i^{\;T},  \qquad
  Q_\ell Q_m=Q_\ell\,\delta_{\ell m},
\end{align}
%
where $I$ is the identity matrix on $\mathbb{R}^N$. Here, we have
defined $Q_i$, $i=1,\ldots,N$, to be the mutually orthogonal projection
matrices onto the eigenspaces spanned by the $\vec{u}_i$.




Since $M_1\vec{u}_i=\lambda_i\vec{u}_i$, the identity $Q_i=\vec{u}_i\vec{u}_i^{\;T}$
implies that we also have $M_1Q_i=\lambda_iQ_i$. This and equation
\eqref{eq:Matrix_Rep_Spec_Theorem} then imply that the matrix  $M_1$
has the spectral decomposition $M_1=\sum_{i=1}^N\lambda_iQ_i$. By the mutual
orthogonality of the projection matrices $Q_i$ and by induction, we
have that  
$M_1^n=\sum_{i=1}^N\lambda_i^nQ_i$ for all $n\in\mathbb{N}$. This, in turn,
implies that $f(M_1)=\sum_{i=1}^Nf(\lambda_i)Q_i$ for any polynomial
$f:\mathbb{R}\mapsto\mathbb{C}$.  Consequently, for all
$\vec{\xi},\vec{\zeta}\in\mathbb{C}^N$, the functional
$\langle f(M_1)\vec{\xi}\cdot\vec{\zeta}\,\rangle_1=\langle f(M_1)\chi_1\vec{\xi}\cdot\vec{\zeta}\,\rangle$ has the
following integral representation
%
\begin{align}\label{eq:Discrete_Spectral_Theorem}
  \langle f(M_1)\vec{\xi}\cdot\vec{\zeta}\,\rangle_1= \int_0^1f(\lambda)\d\mu_{\xi\zeta}(\lambda), \quad
  \d\mu_{\xi\zeta}(\lambda)=\sum_{i=1}^N\langle\delta_{\lambda_i}(\d\lambda)Q_i\,\vec{\xi}\cdot\vec{\zeta}\,\rangle_1.
\end{align}
%
The proof of Theorem \ref{thm:Discrete_Spectral_Theorem_ACM} given in
Section \ref{sec:Theorem_Proof} demonstrates that equation
\eqref{eq:Discrete_Spectral_Theorem} also holds for the function
$f(\lambda)=(s-\lambda)^{-1}$ when $s\in\mathbb{C}\backslash[0,1]$. In this matrix setting,
the projection valued operator $Q(\lambda)$ associated with the strictly
increasing function  $\mu_{\xi\zeta}(\lambda)=\langle Q(\lambda)\vec{\xi}\cdot\vec{\zeta}\,\rangle_1$, discussed
in Section \ref{sec:The_Spectral_Theorem_Continuum}, can be written
explicitly as 
% 
\begin{align}
  Q(\lambda)=\sum_{i:\lambda_i<\lambda}\theta(\lambda-\lambda_i)\,Q_i.
\end{align}
%
Here, $\theta(\lambda)$ is the Heaviside function which takes the value $\theta(\lambda)=0$
for $\lambda<0$ and $\theta(\lambda)=1$ for $\lambda>0$ \cite{Keener-2000}. 






\medskip

{\bf Acknowledgments.}
We gratefully acknowledge support from the Division of Mathematical
Sciences and the Division of Polar Programs at the U.S. National
Science Foundation (NSF) through Grants DMS-1009704, ARC-0934721,
DMS-0940249, and DMS-1413454. We are also grateful for support from
the Office of Naval Research (ONR) through Grants N00014-13-10291 and
N00014-12-10861. Finally, we would like to thank the NSF Math Climate
Research Network (MCRN) for their support of this work. 



\medskip

\bibliographystyle{siam}
\bibliography{murphy}
\end{document}

% LocalWords:  CMS RM al holomorphy nite DEFORMABLE Nonsimple Varadhan JPA Stat
% LocalWords:  Mech iY jY dL jk eps def rep Ef Jf jL Det ccc diag chi iid Cond
% LocalWords:  Decomp Perm Eigenspace ui Ri Es PtI ELENA mh Fs Hashin Shtrikman
% LocalWords:  extremized kk eigenspace Nf ij GRAEME'S Acknowledgement murphy
