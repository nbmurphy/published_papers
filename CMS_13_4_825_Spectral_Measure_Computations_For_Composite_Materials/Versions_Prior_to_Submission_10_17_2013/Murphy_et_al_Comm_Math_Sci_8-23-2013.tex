\documentclass{cmslatex}
%\usepackage{latexsym, amssymb, enumerate, amsmath}
\usepackage{graphicx,amssymb,amsmath,amsfonts,mathrsfs}
 
\sloppy

\thinmuskip = 0.5\thinmuskip \medmuskip = 0.5\medmuskip
\thickmuskip = 0.5\thickmuskip \arraycolsep = 0.3\arraycolsep

\renewcommand{\theequation}{\arabic{section}.\arabic{equation}}

\def\open#1{\setbox0=\hbox{$#1$}
\baselineskip = 0pt \vbox{\hbox{\hspace*{0.4 \wd0}\tiny
$\circ$}\hbox{$#1$}} \baselineskip = 10pt\!}

\newcommand{\ph}{\hat{\phi}}
\newcommand{\pt}{\tilde{\phi}} 
\newcommand{\pc}{\check{\phi}}
\newcommand{\gh}{\hat{\gamma}}
\newcommand{\Dh}{\hat{\Delta}}
\newcommand{\dha}{\hat{\delta}}
\newcommand{\qh}{\hat{q}}
\newcommand{\xh}{\hat{x}}
\newcommand{\HM}{\mathcal{H}_{\text{max}}}
\newcommand{\Hm}{\mathcal{H}_{\text{min}}}
\newcommand{\sech}{\rm \hspace{0.7mm}sech}
\newcommand{\I}{\mathrm{i}}
\newcommand{\hh}{\hat{h}}
\newcommand{\mh}{m_r}
\newcommand{\mt}{m_i}
\newcommand{\M}{\mathbf{M}}
\newcommand{\X}{\mathbf{X}}
\newcommand{\T}{\mathbf{T}}
\newcommand{\Hb}{\mathbf{H}}
\newcommand{\K}{\mathbf{K}}
\newcommand{\J}{\mathbf{J}}

\newcommand\bsig{\mbox{\boldmath${\sigma}$}}
\newcommand\beps{\mbox{\boldmath${\epsilon}$}}
\newcommand\bxi{\mbox{\boldmath${\xi}$}}
\newcommand\bmu{\mbox{\boldmath${\mu}$}}
\newcommand\balpha{\mbox{\boldmath${\alpha}$}}
\newcommand\brho{\mbox{\boldmath${\rho}$}}


\newtheorem{thm}{Theorem}[section]
\newtheorem{prop}[thm]{Proposition}
\newtheorem{lem}[thm]{Lemma}
\newtheorem{cor}[thm]{Corollary}

    %\theoremstyle{definition}

\newtheorem{defn}[thm]{Definition}
\newtheorem{notation}[thm]{Notation}
\newtheorem{example}[thm]{Example}
\newtheorem{conj}[thm]{Conjecture}
\newtheorem{prob}[thm]{Problem}

    %\theoremstyle{remark}

\newtheorem{rem}[thm]{Remark}
    % Use the standard latex environments for theorems, etc. Here is one
          % possible method of declaring them: It numbers all results by the
          % section, and uses a common numbering system for the different
          % environmentts.

\begin{document}

\title{Spectral measure computations \\ for composite materials
\thanks{%{Received date / Revised version date}
          % The correct dates will be entered by the CMS editor}}
}}
          %For each author, make a block with the following four
          %macros:
% \author{full name
% \thanks {address, (email).}
% \and full name \thanks {address, (email).}}
%
\author{N. B. Murphy
\thanks {University of Utah, Department of Mathematics, 155 S 1400 E
  RM 233, Salt Lake City, UT 84112-009, USA, (murphy@math.utah.edu)}
%
\and E. Cherkaev
\thanks {University of Utah, Department of Mathematics, 155 S 1400 E
  RM 233, Salt Lake City, UT 84112-009, USA, (elena@math.utah.edu).}
%
\and C. Hohenegger
\thanks {University of Utah, Department of Mathematics, 155 S 1400 E
  RM 233, Salt Lake City, UT 84112-009, USA, (choheneg@math.utah.edu).}
%
\and K. M. Golden
\thanks {University of Utah, Department of Mathematics, 155 S 1400 E
  RM 233, Salt Lake City, UT 84112-009, USA, (golden@math.utah.edu).}
}
          %{Put the URL for your home page here if you have one}

          %Use \thanks statements for acknowledgements of grants and
          %support. They will appear below all the authors' addresses, so be
          %specific about which author is thanking whom:

          %\thanks{}

\pagestyle{myheadings} \markboth{Spectral measure computations}{Murphy et. al.}\maketitle

\begin{abstract}
The analytic continuation method of homogenization theory provies
Stieltjes integral representations for the effective parameters of
composite media, involving spectral measures of self-adjoint random
operators which depend only on the composite geometry. On finite bond
lattices, these random operators are represented by random matrices
and the spectral measures are given explicitely in terms of their
eigenvalues and eigenvectors. Here we develop a projection method
which provides the mathematical foundation for numerically efficient,
rigorous computation of spectral measures for such composite media. We
also introduce a large family of random bond lattices and directly
compute the associated spectral measures and effective parameters. The
behavior of the effective parameters is shown to be consistent with
rigorous bounds, and the computed spectral measures are consistent
with known theoretical results. 
\end{abstract}

\begin{keywords}
composite materials, random resistor network, percolation,
homogenization, spectral measure 
\smallskip

{\bf subject classifications.}
% GENERAL: Conference proceedings and collections of papers:
% Conference proceedings and collections of papers 
00B15,
% Measures, integration, derivative, holomorphy (all involving infinite-
% dimensional spaces): Measures and integration on abstract linear spaces
46G12,
% OPERATOR THEORY: Special classes of linear operators: Hermitian and
% normal operators (spectral measures, functional calculus, etc.)
47B15,
% NUMERICAL ANALYSIS: Numerical linear algebra: Eigenvalues, eigenvectors
65C60,
% MECHANICS OF DEFORMABLE SOLIDS: Generalities, axiomatics,
% foundations of continuum mechanics of solids:
% Random materials and composite materials 
%74A40,
% MECHANICS OF DEFORMABLE SOLIDS: Generalities, axiomatics,
% foundations of continuum mechanics of solids: Nonsimple materials
%74E30,
% FUNCTIONS OF A COMPLEX VARIABLE: Analytic continuation 
30B40,
% OPTICS, ELECTROMAGNETIC THEORY: General: Composite media; random media
78A48,
% OPTICS, ELECTROMAGNETIC THEORY: Basic methods: Spectral methods
78M22,
% CLASSICAL THERMODYNAMICS, HEAT TRANSFER: Basic methods: Spectral
% methods
80M22,
% CLASSICAL THERMODYNAMICS, HEAT TRANSFER: Basic methods: Homogenization
80M40,
% PROBABILITY THEORY AND STOCHASTIC PROCESSES: Special processes:
% Interacting random processes; statistical mechanics type models;
% percolation theory
60K35
\end{keywords}


\section{Introduction}\label{Introduction}
%
Over the years a broad range of mathematical techniques have been
developed that reduce the analysis of complex composite materials,
with rapidly varying structures in space, to solving averaged, or
\emph{homogenized} equations that do not have rapidly varying data,
and involve an effective parameter. Homogenization for the effective
parameter problem in composite media with rapidly varying coefficients
of thermal conductivity, or equivalently \cite{MILTON:2002:TC}
electrical conductivity and permittivity, and magnetic permeability,
was developed by Papanicolaou and Varadhan \cite{Papanicolaou:RF-835}
for the steady state, static case. This work was extended
\cite{Golden:CMP-473,Golden:JSP-655} by Golden and Papanicolaou to the
quasi-static frequency dependent case with complex
parameters. Analysis of the effective dielectric problem for the fully
frequency dependent case described by the Helmholtz equation is given
in \cite{Simeonova:MMS:1113}. 

The effective parameter problem for \emph{two component} media in the
quasi-static limit was developed by Bergman \cite{Bergman:PRL-1285},
Milton \cite{Milton:APL-300}, and Golden and Papanicolaou
\cite{Golden:CMP-473}, leading to Stieltjes integral representations 
for the effective parameters of composite media. The
Golden-Papanicolaou formulation of this analytic continuation method
(ACM) is based on the spectral theorem and resolvent formulas
involving random self-adjoint operators. This formulation demonstrated
that the measures underlying these integral representations are
\emph{spectral measures} associated with these random operators, which
depend only on the composite geometry. The spectral measures contain
all the information about the mixture geometry, and provide a link
between microgeometry and transport. Local geometry is encoded
in ``geometric'' resonances in the measure \cite{Jonckheere_Luck_JPA_1998},
while global connectivity is encoded by spectral gaps in the measure
at the spectral endpoints
\cite{Murphy:JMP:063506,Golden:PRL-3935,Jonckheere_Luck_JPA_1998}. A
remarkable feature of the method is that once the spectral measures
are found for a given composite geometry, by the symmetries in the
governing equations \cite{MILTON:2002:TC}, the effective electrical,
magnetic, and thermal transport properties are \emph{all} completely
determined. 

The integral representations yield rigorous \emph{forward bounds} on
the effective parameters of composites, given partial information on
the microgeometry
\cite{Bergman:PRL-1285,Milton:APL-300,Golden:CMP-473,Bergman:AP-78}. One  
can also use the integral representations to obtain inverse bounds,
where data on the electromagnetic response of a sample, for example,
is used to bound its structural parameters, such as the volume
fractions of the components
\cite{Cherkaeva:WRM-437,Cherkaeva:IP-1203,Cherkaeva:IP-065008,Zhang:JCP-5390,Bonifasi-Lista:PMB-3063,Cherkaev:JBiomech-345,Day:JPCM-99,Golden:J_Biomech:337},
and even the separation of the inclusions in matrix particle
composites \cite{Orum:PRSLA:2012}. The integral representations have
also been used to obtain detailed information regarding critical
transitions in the behavior of the effective parameters as a composite
medium undergoes a percolation transition
\cite{Golden:PRL-3935,Murphy:JMP:063506}.

Dispite the many applications which have stemmed from the ACM,
explicit results are available for only a handfull of composite
microstructures. To help overcome this limitation, here we develop a
mathematical framework which provides a rigorous way to directly
compute the spectral measures, hence the effective properties, for
finite lattices. 



\section{Put title of the next section here}\label{an apprpriate
label}

          %If you have subsections use:
\subsection{Subsection title}\label{another label}

Don't forget to give each section, subsection, equation, theorem,
corollary, etc. a unique label, and when you refer to the results
later in the text use \ref{<labelname>} instead of explicitly
writing the number of the environment in question.

This use of \ label and \ ref is REQUIRED for  papers.

Similarly, always use \cite{biblabelname} to refer to
bibliographic references, which would then be entered in the
bibliography via
          %\bibitem{biblabelname}.

          %
          % For figures use

          %\begin{figure}

          %The use of .eps files is encouraged, in which case you should
          %un-comment the \uspackage{graphics} command above, and use the
          %command
          %\include{figure.eps}
          % to insert the figure file.

          %\end{figure}


          % BibTeX users please use

          % \bibliographystyle{}

          % \bibliography{}

          %

          % Non-BibTeX users please use
\medskip

{\bf Acknowledgement.}
\medskip

\bibliographystyle{siam}
\bibliography{murphy}
% \begin{thebibliography}{10}

%           %

%           % and use \bibitem to create references.

%           %

% \bibitem{A}Author, {\em title of paper}, Journal Name
% Volume,  page numbers, year.

%           % Format for Journal Reference. For example

% \bibitem{T1}C. Taubes, {\em The Seiberg-Witten invariants
%  and symplectic forms}, Math. Res. Letters, 1, 809--822, 1994.
% \end{thebibliography}

\end{document}
