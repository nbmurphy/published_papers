\documentclass{cmslatex}
%\usepackage{latexsym, amssymb, enumerate, amsmath}
\usepackage{graphicx,amssymb,amsmath,amsfonts,mathrsfs}
 
\sloppy

\thinmuskip = 0.5\thinmuskip \medmuskip = 0.5\medmuskip
\thickmuskip = 0.5\thickmuskip \arraycolsep = 0.3\arraycolsep

\renewcommand{\theequation}{\arabic{section}.\arabic{equation}}

\def\open#1{\setbox0=\hbox{$#1$}
\baselineskip = 0pt \vbox{\hbox{\hspace*{0.4 \wd0}\tiny
$\circ$}\hbox{$#1$}} \baselineskip = 10pt\!}

\newcommand{\ph}{\hat{\phi}}
\newcommand{\pt}{\tilde{\phi}} 
\newcommand{\pc}{\check{\phi}}
\newcommand{\gh}{\hat{\gamma}}
\newcommand{\Dh}{\hat{\Delta}}
\newcommand{\dha}{\hat{\delta}}
\newcommand{\qh}{\hat{q}}
\newcommand{\xh}{\hat{x}}
\newcommand{\HM}{\mathcal{H}_{\text{max}}}
\newcommand{\Hm}{\mathcal{H}_{\text{min}}}
\newcommand{\sech}{\rm \hspace{0.7mm}sech}
\newcommand{\I}{\mathrm{i}}
\newcommand{\hh}{\hat{h}}
\newcommand{\mh}{m_r}
\newcommand{\mt}{m_i}
\newcommand{\M}{\mathbf{M}}
\newcommand{\X}{\mathbf{X}}
\newcommand{\T}{\mathbf{T}}
\newcommand{\Hb}{\mathbf{H}}
\newcommand{\K}{\mathbf{K}}
\newcommand{\J}{\mathbf{J}}

\newcommand\bsig{\mbox{\boldmath${\sigma}$}}
\newcommand\beps{\mbox{\boldmath${\epsilon}$}}
\newcommand\bxi{\mbox{\boldmath${\xi}$}}
\newcommand\bmu{\mbox{\boldmath${\mu}$}}
\newcommand\balpha{\mbox{\boldmath${\alpha}$}}
\newcommand\brho{\mbox{\boldmath${\rho}$}}
\newcommand\bDelta{\mbox{\boldmath${\Delta}$}}


\newtheorem{thm}{Theorem}[section]
\newtheorem{prop}[thm]{Proposition}
\newtheorem{lem}[thm]{Lemma}
\newtheorem{cor}[thm]{Corollary}

    %\theoremstyle{definition}

\newtheorem{defn}[thm]{Definition}
\newtheorem{notation}[thm]{Notation}
\newtheorem{example}[thm]{Example}
\newtheorem{conj}[thm]{Conjecture}
\newtheorem{prob}[thm]{Problem}

    %\theoremstyle{remark}

\newtheorem{rem}[thm]{Remark}
    % Use the standard latex environments for theorems, etc. Here is one
          % possible method of declaring them: It numbers all results by the
          % section, and uses a common numbering system for the different
          % environmentts.

\begin{document}

\title{Spectral measure computations \\ for composite materials
\thanks{%{Received date / Revised version date}
          % The correct dates will be entered by the CMS editor}}
}}
          %For each author, make a block with the following four
          %macros:
% \author{full name
% \thanks {address, (email).}
% \and full name \thanks {address, (email).}}
%
\author{N. B. Murphy
\thanks {University of Utah, Department of Mathematics, 155 S 1400 E
  RM 233, Salt Lake City, UT 84112-009, USA, (murphy@math.utah.edu)}
%
\and E. Cherkaev
\thanks {University of Utah, Department of Mathematics, 155 S 1400 E
  RM 233, Salt Lake City, UT 84112-009, USA, (elena@math.utah.edu).}
%
\and C. Hohenegger
\thanks {University of Utah, Department of Mathematics, 155 S 1400 E
  RM 233, Salt Lake City, UT 84112-009, USA, (choheneg@math.utah.edu).}
%
\and K. M. Golden
\thanks {University of Utah, Department of Mathematics, 155 S 1400 E
  RM 233, Salt Lake City, UT 84112-009, USA, (golden@math.utah.edu).}
}
          %{Put the URL for your home page here if you have one}

          %Use \thanks statements for acknowledgments of grants and
          %support. They will appear below all the authors' addresses, so be
          %specific about which author is thanking whom:

          %\thanks{}

\pagestyle{myheadings} \markboth{Spectral measure computations}{Murphy et. al.}\maketitle

\begin{abstract}
The analytic continuation method of homogenization theory provides
Stieltjes integral representations for the effective parameters of
composite media, involving spectral measures of self-adjoint random
operators which depend only on the composite geometry. On finite bond
lattices, these random operators are represented by random matrices
and the spectral measures are given explicitly in terms of their
eigenvalues and eigenvectors. Here we provide the mathematical
foundation for rigorous computation of spectral measures for such
composite media. We also introduce a large family of random bond
lattices and directly compute the associated spectral measures and
effective parameters. The computed spectral measures agree with known
theoretical results, and the behavior of the effective parameters is
shown to be consistent with rigorous bounds. 
\end{abstract}

\begin{keywords}
composite materials, random resistor network, percolation,
homogenization, spectral measure 
\smallskip

{\bf subject classifications.}
% GENERAL: Conference proceedings and collections of papers:
% Conference proceedings and collections of papers 
00B15,
% Measures, integration, derivative, holomorphy (all involving infinite-
% dimensional spaces): Measures and integration on abstract linear spaces
%46G12,
% OPERATOR THEORY: Special classes of linear operators: Hermitian and
% normal operators (spectral measures, functional calculus, etc.)
47B15,
% NUMERICAL ANALYSIS: Numerical linear algebra: Eigenvalues, eigenvectors
65C60,
% MECHANICS OF DEFORMABLE SOLIDS: Generalities, axiomatics,
% foundations of continuum mechanics of solids:
% Random materials and composite materials 
%74A40,
% MECHANICS OF DEFORMABLE SOLIDS: Generalities, axiomatics,
% foundations of continuum mechanics of solids: Nonsimple materials
%74E30,
% FUNCTIONS OF A COMPLEX VARIABLE: Analytic continuation 
30B40,
% OPTICS, ELECTROMAGNETIC THEORY: General: Composite media; random media
78A48,
% OPTICS, ELECTROMAGNETIC THEORY: Basic methods: Spectral methods
%78M22,
% CLASSICAL THERMODYNAMICS, HEAT TRANSFER: Basic methods: Spectral
% methods
%80M22,
% CLASSICAL THERMODYNAMICS, HEAT TRANSFER: Basic methods: Homogenization
80M40,
% PROBABILITY THEORY AND STOCHASTIC PROCESSES: Special processes:
% Interacting random processes; statistical mechanics type models;
% percolation theory
60K35
\end{keywords}


\section{Introduction}\label{Introduction}
%
Over the years a broad range of mathematical techniques have been
developed that reduce the analysis of complex composite materials,
with rapidly varying structures in space, to solving averaged, or
\emph{homogenized} equations that do not have rapidly varying data,
and involve an effective parameter. Homogenization for the effective
parameter problem in composite media with rapidly varying coefficients
of thermal conductivity, or equivalently \cite{MILTON:2002:TC}
electrical conductivity and permittivity, and magnetic permeability,
was developed by Papanicolaou and Varadhan \cite{Papanicolaou:RF-835}
for the steady state, static case. This work was extended
\cite{Golden:CMP-473,Golden:JSP-655} by Golden and Papanicolaou to the
quasi-static frequency dependent case with complex
parameters. Analysis of the effective dielectric problem for the fully
frequency dependent case described by the Helmholtz equation is given
in \cite{Simeonova:MMS:1113}. 

The effective parameter problem for \emph{two-component} media in the
quasi-static limit was developed by Bergman \cite{Bergman:PRL-1285},
Milton \cite{Milton:APL-300}, and Golden and Papanicolaou
\cite{Golden:CMP-473}, leading to Stieltjes integral representations 
for the effective parameters of composite media. The
Golden-Papanicolaou formulation of this analytic continuation method
(ACM) is based on the spectral theorem and resolvent formulas
involving random self-adjoint operators. This formulation demonstrated
that the measures underlying these integral representations are
\emph{spectral measures} associated with these random operators, which
depend only on the composite geometry. These measures contain all the
information about the mixture geometry, and provide a link between
microgeometry and transport. Local geometry is encoded in
``geometric'' resonances in the measures
\cite{Jonckheere_Luck_JPA_1998}, while global connectivity is encoded
by spectral gaps in the measures at the spectral endpoints
\cite{Murphy:JMP:063506,Jonckheere_Luck_JPA_1998}. A remarkable
feature of the method is that once the spectral measures are found for 
a given composite geometry, by the symmetries in the governing
equations \cite{MILTON:2002:TC}, the effective electrical, magnetic,
and thermal transport properties are \emph{all} completely determined
by these measures.


The integral representations yield rigorous \emph{forward bounds} on
the effective parameters of composites, given partial information on
the microgeometry
\cite{Bergman:PRL-1285,Milton:APL-300,Golden:CMP-473,Bergman:AP-78}. One  
can also use the integral representations to obtain inverse bounds,
where data on the electromagnetic response of a sample, for example,
is used to bound its structural parameters, such as the volume
fractions of the components
\cite{Cherkaeva:WRM-437,Cherkaeva:IP-1203,Cherkaeva:IP-065008,Zhang:JCP-5390,Bonifasi-Lista:PMB-3063,Cherkaev:JBiomech-345,Day:JPCM-99,Golden:J_Biomech:337},
and even the separation of the inclusions in matrix particle
composites \cite{Orum:PRSLA:2012}. For classes of composites which
undergo a percolation transition, the integral representations have
been used to obtain detailed information regarding the critical
behavior of the effective parameters in the scaling regime
\cite{Golden:PRL-3935,Murphy:JMP:063506}. Furthermore, the
relationship between the effective parameters and the system energy
has also led to a physically consistent statistical mechanics model
for two-component dielectric media which is also mathematically
tractable \cite{Murphy_Thermo_Stat_Mech}.  



Despite the many applications which have stemmed from the ACM,
explicit computations of the effective parameters have been obtained
for only a handful of composite microstructures. To help overcome
this limitation, here we develop a mathematical framework which
provides a rigorous way to directly compute the spectral measures and
effective parameters for finite lattice composite microstructures. A
projection method is introduced which provides a numerically efficient
way to accomplish these computations. We also introduce a large class
of locally isotropic, statistically isotropic, and anisotropic random
bond networks and directly compute the associated spectral measures
and effective parameters. Our numerical calculations of the spectral
measures are in excellent agreement with known theoretical results.
Moreover, the behavior of the associated effective parameters are
consistent with rigorous bounds. (FINISH THIS PARAGRAPH WHEN THE 
MATHEMATICAL METHODS AND RESULTS SECTION IS FINISHED).       

\section{Mathematical Methods}\label{sec:Mathematical_Methods} 
% {\it The analytic continuation method (general operator
%   case). Spectral measure decomposition. Geometric resonances and gaps
%   in the spectral measure. Lattice systems (random matrix case) and
%   the direct calculation of the spectral measure. Forward bounds. What
%   new structure we found regarding the spectral measure
%   decomposition.} 
%
We now formulate the effective parameter problem for random
two-phase conductive media in the lattice and continuum
settings. In Section \ref{sec:Continuum_Setting} we review the ACM for  
the continuum setting \cite{Golden:CMP-473}, while the \emph{infinite}
lattice setting \cite{Bruno:JSP-365,Golden:CMP-467} is reviewed in
Section \ref{sec:Infinite_Lattice_Setting}. The mathematical framework
underlying the infinite lattice case is analogous to that of
the continuum case \cite{Bruno:JSP-365}, and the integral
representation for the effective parameters follow with minor
modifications in the theory. In the \emph{finite} lattice setting, the
integral representations for the effective parameters are analogous to
that of the continuum and infinite lattice cases. However,
significant modifications must be made to the underlying mathematical
framework. A key theoretical contribution of this work is the
formulation of the ACM for the finite lattice case, which is discussed
in Section \ref{sec:Finite_Lattice_Setting}. In order to streamline
the presentation of the method, we have placed many of the technical
details in an appendix. The statement of Theorem
\ref{thm:Discrete_Spectral_Theorem_ACM} of Section
\ref{sec:The_Spectral_Theorem_Finite_Lattice} summarizes the
formulation of the ACM for the finite lattice setting.   


\subsection{Continuum Setting}\label{sec:Continuum_Setting}
%
Consider a two-phase random conductive medium filling all of
$\mathbb{R}^d$, which is determined by the probability space
$(\Omega,P)$. Here $\Omega$ is the set of all  geometric realizations of our
random medium, which is indexed by the parameter $\omega\in\Omega$ representing
one particular geometric realization, and $P$ is the associated
probability measure. Let $\bsig(\vec{x},\omega)$ be the local complex
conductivity tensor associated with the conductive medium, which has 
components $\sigma_{jk}(\vec{x},\omega)$, $j,k=1,\ldots,d$, which are (spatially)
stationary random fields (see Section \ref{sec:Stationarity} for
details).  




Consider the Hilbert space $\mathscr{H}=\bigotimes_{i=1}^dL^2(\Omega,P)$ with inner
product $\langle\cdot,\cdot\rangle$ defined by $\langle\vec{\xi},\vec{\zeta}\rangle=\langle\vec{\xi}\cdot\vec{\zeta}\rangle$,
where $\vec{\xi}\cdot\vec{\zeta}=\vec{\xi}^{\;\,T}\vec{\zeta}$ denotes the dot-product
on $\mathbb{R}^d$ and $\langle\cdot\rangle$ means ensemble average over $\Omega$ or, by an
ergodic theorem \cite{Golden:CMP-473}, spatial average over all of
${\mathbb{R}}^d$. Define the Hilbert spaces \cite{Golden:CMP-473} of
``curl free'' $\mathscr{H}_\times$ and ``divergence free''
$\mathscr{H}_{\bullet}$ random fields (see equation
\eqref{eq:curlfreeHilbert_Precise} of Section \ref{sec:Stationarity}
for details)       
%
\begin{align}\label{eq:curlfreeHilbert}
  &\mathscr{H}_\times=
  \left\{\vec{Y}\in \mathscr{H} \ | \ \vec{\nabla} \times\vec{Y}=0 \text{ weakly and }
    \langle\vec{Y}\rangle=0
  \right\}, \\
&\mathscr{H}_{\bullet}=
\left\{\vec{Y}\in \mathscr{H} \ | \ \vec{\nabla}\cdot\vec{Y}=0 \text{ weakly and }
    \langle\vec{Y}\rangle=0\right\},\notag 
\end{align}
%
and consider the following variational problems 
\cite{Golden:CMP-473}. Find $\vec{E}_f\in\mathscr{H}_\times$ and
$\vec{J}_f\in\mathscr{H}_\bullet$ such that     
%
\begin{align}
 \label{eq:Weak_Curl_Free_Variational_Form}
 &\langle\bsig(\vec{E}_0+\vec{E}_f)\cdot\vec{Y}\rangle=0 \quad  \forall \
  \vec{Y}\in\mathscr{H}_\times &&\text{and}
%
 &&\langle\brho(\vec{J}_0+\vec{J}_f)\cdot\vec{Y}\rangle=0 \quad  \forall \
  \vec{Y}\in\mathscr{H}_{\bullet}\,,  
\end{align}
%
respectively. When the bilinear forms
$\Psi(\vec{\xi},\vec{\zeta})=\bsig\,\vec{\xi}\cdot\vec{\zeta}\;$ and
$\Phi(\vec{\xi},\vec{\zeta})=\brho\,\vec{\xi}\cdot\vec{\zeta}\;$ are bounded and
coercive, these problems have unique solutions
\cite{Golden:CMP-473,Papanicolaou:RF-835} satisfying the quasi-static
limit of Maxwell's equations  \cite{Jackson-1999} 
%
\begin{align}   \label{eq:Maxwells_Equations_E}  
 &\vec{\nabla}\times\vec{E}=0, \quad
  \vec{\nabla}\cdot\vec{J}=0,\quad
  \vec{J}=\bsig\vec{E},\quad
  %\vec{E}=\vec{E}_0+\vec{E}_f, \quad
  \langle\vec{E}\,\rangle=\vec{E}_0, \\
%
  %\label{eq:Maxwells_Equations_D}
   &\vec{\nabla}\times\vec{E}=0, \quad
   \vec{\nabla}\cdot\vec{J}=0, \quad
   \vec{E}=\brho\vec{J},\quad
   %\vec{J}=\vec{J}_0+\vec{J}_f,\quad
   \langle\vec{J}\,\rangle=\vec{J}_0.
   \notag  
\end{align}
%
Here, $\vec{E}(\vec{x},\omega)=\vec{E}_0+\vec{E}_f(\vec{x},\omega)$ is the
random electric field, where $\vec{E}_0=\langle\vec{E}\rangle$ and $\vec{E}_f$ is
the fluctuating field of mean zero about the (constant) average
$\vec{E}_0$. Similarly,
$\vec{J}(\vec{x},\omega)=\vec{J}_0+\vec{J}_f(\vec{x},\omega)$ is the random 
current density. Moreover, $\vec{E}_f$ and $\vec{J}_f$ are stationary
random fields \cite{Golden:CMP-473}. 
% In component form equation \eqref{eq:Maxwells_Equations_E} is given by
% %
% \begin{align}   \label{eq:Maxwells_Equations_Component}  
%   \partial_{x_j}E_i-\partial_{x_i}E_j=0, \quad
%   \sum_{j=1}^d\partial_{x_j}D_j=0, \quad
%   D_j=\sum_{k=1}^d\sigma_{jk}E_k, \quad
% %  \vec{E}=\vec{E}_0+\vec{E}_f, \quad
%   \langle\vec{E}\,\rangle=\vec{E}_0, 
% \end{align}




As $\vec{E}_f\in\mathscr{H}_\times$ and $\vec{J}_f\in\mathscr{H}_\bullet$, equation
\eqref{eq:Weak_Curl_Free_Variational_Form} yields the energy (power)
\cite{Jackson-1999} constraints $\langle\vec{J}\cdot\vec{E}_f\rangle=0$ and
$\langle\vec{E}\cdot\vec{J}_f\rangle=0$, respectively, which leads to the following
reduced energy representations $\langle\vec{J}\cdot\vec{E}\rangle=\langle\vec{J}\rangle\cdot\vec{E}_0$
and $\langle\vec{E}\cdot\vec{J}\rangle=\langle\vec{E}\rangle\cdot\vec{J}_0$. The effective complex
conductivity and resistivity tensors, $\bsig^*$ and $\brho^*$, are
defined by   
%
\begin{align}\label{eq:eff_eps_def}
    \langle \vec{J} \,\rangle=  \bsig^* \vec{E}_0
    \quad \text{and} \quad    
    \langle \vec{E} \,\rangle=  \brho^*\vec{J}_0.
\end{align}
%
Consequently, we have the following energy representations involving
the effective parameters
$\langle\vec{J}\cdot\vec{E}\rangle=\bsig^*\vec{E}_0\cdot\vec{E}_0=\brho^*\vec{J}_0\cdot\vec{J}_0$. We
assume that the composite is a locally isotropic medium so that 
$\sigma_{jk}(\vec{x},\omega)=\sigma(\vec{x},\omega)\delta_{jk}$ and
$\rho_{jk}(\vec{x},\omega)=\rho(\vec{x},\omega)\delta_{jk}$, where $\delta_{jk}$ is the
Kronecker delta and $j,k=1,\ldots,d$. We further assume that the composite
is a two-component medium, so that $\sigma(\vec{x},\omega)$ takes the
\emph{complex} values $\sigma_1$ and $\sigma_2$, and $\rho(\vec{x},\omega)$ takes the
values $1/\sigma_1$ and $1/\sigma_2$, and satisfy \cite{Golden:CMP-473}
% 
\begin{align}\label{eq:two-phase_eps}
  \sigma(\vec{x},\omega)=\sigma_1\chi_1(\vec{x},\omega)+\sigma_2\chi_2(\vec{x},\omega), \qquad
  \rho(\vec{x},\omega)=\chi_1(\vec{x},\omega)/\sigma_1+\chi_2(\vec{x},\omega)/\sigma_2.
\end{align}
%
%$\sigma(\vec{x},\omega)=\sigma_1\chi_1(\vec{x},\omega)+\sigma_2\chi_2(\vec{x},\omega)$,
Here $\chi_i(\vec{x},\omega)$ is the characteristic function of medium
$i=1,2$, which equals one for all $\omega\in\Omega$ having medium $i$ at $\vec{x}$
and zero otherwise, with $\chi_1=1-\chi_2$. For simplicity, we focus on one
component of these tensors, $\sigma^*_{jk}$ and $\rho^*_{jk}$, for some
$j,k=1,\ldots,d$.  



Due to the homogeneity of these functions,
e.g. $\sigma_{jk}^*(a\sigma_1,a\sigma_2)=a\sigma_{jk}^*(\sigma_1,\sigma_2)$ for any complex number
$a$, they depend only on the ratio $h=\sigma_1/\sigma_2$, and we define the
tensor-valued functions $\mathbf{m}(h)=\bsig^*/\sigma_2$,
$\mathbf{w}(z)=\bsig^*/\sigma_1$, $\tilde{\mathbf{m}}(h)=\sigma_1\brho^*$, and
$\tilde{\mathbf{w}}(z)=\sigma_2\brho^*$ with components  
%
\begin{align}\label{eq:m_h}
  m_{jk}(h)=\sigma_{jk}^*/\sigma_2, \quad
  w_{jk}(z)=\sigma_{jk}^*/\sigma_1, \quad
   \tilde{m}_{jk}(h)=\sigma_1\rho_{jk}^*, \quad
   \tilde{w}_{jk}(z)=\sigma_2\rho_{jk}^*.
\end{align}
%
where $z=1/h$. The dimensionless functions $m_{jk}(h)$ and
$\tilde{m}_{jk}(h)$ are analytic off the negative real axis in the
$h$-plane, while $w_{jk}(z)$ and $\tilde{w}_{jk}(z)$ are analytic off
the negative real axis in the $z$-plane \cite{Golden:CMP-473}. Each
take the corresponding upper half plane to the upper half plane and
are therefore examples of Herglotz functions
\cite{Deift:2000:RMT,Golden:CMP-473}. A key step in the ACM is
obtaining Stieltjes integral representations for $\bsig^*$ and
$\brho^*$. These follow from resolvent representations for the
electric field \cite{Golden:CMP-473} and current density
\cite{Murphy:JMP:063506}  (see Section 
\ref{sec:Resolvent_Representations} for details)       
%
\begin{align}\label{eq:Resolvent_representations_E_D}
  &\vec{E}=s(sI-\Gamma\chi_1)^{-1}\vec{E}_0=t(tI-\Gamma\chi_2)^{-1}\vec{E}_0 ,
  \quad
   s\in\mathbb{C}\backslash[0,1],\\
  &\vec{J}=t(tI-\Upsilon\chi_1)^{-1}\vec{J}_0=s(sI-\Upsilon\chi_2)^{-1}\vec{J}_0 ,
  \quad
   t\in\mathbb{C}\backslash[0,1],\notag 
\end{align}
%
where $I$ is the identity operator on $\mathbb{R}^d$ and we have
defined the complex variables $s=1/(1-h)$ and $t=1/(1-z)=1-s$. The
operator $\Gamma=\vec{\nabla}(\Delta^{-1})\vec{\nabla}\cdot$ is an orthogonal projection  
from $\mathscr{H}$ onto the Hilbert space $\mathscr{H}_\times$ of curl-free 
random fields, and is based on convolution with the free-space Green's
function for the Laplacian $\Delta=\nabla\cdot\nabla=\nabla^{\,2}$ 
\cite{Golden:CMP-473,Murphy:JMP:063506}. Similarly, the operator
$\Upsilon=-\vec{\nabla}\times(\bDelta^{-1})\vec{\nabla}\times$ is an orthogonal projection from
$\mathscr{H}$ onto the Hilbert space $\mathscr{H}_\bullet$ of
divergence-free random fields of transverse gauge, involving the
vector Laplacian $\bDelta=- \vec{\nabla}\times\vec{\nabla}\times + \vec{\nabla}\vec{\nabla}\cdot $ (see
Section \ref{sec:Resolvent_Representations} for details).    




It is more convenient to consider the functions
$F_{jk}(s)=\delta_{jk}-m_{jk}(h)$ and $E_{jk}(s)=\delta_{jk}-\tilde{m}_{jk}(h)$
which are analytic off $[0,1]$ in the $s$-plane, and
$G_{jk}(t)=\delta_{jk}-w_{jk}(z)$ and $H_{jk}(t)=\delta_{jk}-\tilde{w}_{jk}(z)$
which are analytic off $[0,1]$ in the $t$-plane
\cite{Golden:CMP-473}. For the formulation of the effective parameter
problem involving $\mathscr{H}_\times$ and $\bsig^*$, define the
coordinate system so that in \eqref{eq:eff_eps_def} the constant
vector $\vec{E}_0$ is given by $\vec{E}_0=E_0\,\vec{e}_j$, where
$\vec{e}_j$ is a standard basis vector on $\mathbb{R}^d$ in the
$j^{\,\text{th}}$ direction for some $j=1,\ldots,d$. In the other
formulation involving $\mathscr{H}_\bullet$ and $\brho^*$ define
$\vec{J}_0=J_0\,\vec{e}_j$.  Also, in \eqref{eq:two-phase_eps} write 
$\sigma=\sigma_2(1-\chi_1/s)=\sigma_1(1-\chi_2/t)$ and
$\rho=(1-\chi_2/s)/\sigma_1=(1-\chi_1/t)/\sigma_2$. Equations \eqref{eq:eff_eps_def} and 
\eqref{eq:Resolvent_representations_E_D}, and the spectral theorem
\cite{Reed-1980,Stone:64} then yield the following integral
representations
\cite{Golden:CMP-473,Bergman:PRC-377,Bergman:AP-78,Murphy:JMP:063506}  
for the effective parameters $\sigma^*_{jk}$ and $\rho^*_{jk}$
%
\begin{align}\label{eq:Stieltjes_F}
  &m_{jk}(h)=\delta_{jk}-F_{jk}(s), \qquad
  F_{jk}(s)=\langle\chi_1(sI-\Gamma\chi_1)^{-1}\vec{e}_j\cdot\vec{e}_k\rangle=\int_0^1\frac{\mu_{jk}(d\lambda)}{s-\lambda}\,,
  \\
  &w_{jk}(z)=\delta_{jk}-G_{jk}(t), \qquad
  G_{jk}(t)=\langle\chi_2(tI-\Gamma\chi_2)^{-1}\vec{e}_j\cdot\vec{e}_k\rangle=\int_0^1\frac{\alpha_{jk}(d\lambda)}{t-\lambda}\,,
  \notag \\
  &\tilde{m}_{jk}(h)=\delta_{jk}-E_{jk}(s), \qquad
  E_{jk}(s)=\langle\chi_2(sI-\Upsilon\chi_2)^{-1}\vec{e}_j\cdot\vec{e}_k\rangle=\int_0^1\frac{\eta_{jk}(d\lambda)}{s-\lambda}\,,
  \notag \\
  &\tilde{w}_{jk}(z)=\delta_{jk}-H_{jk}(t), \qquad
  H_{jk}(t)=\langle\chi_1(tI-\Upsilon\chi_1)^{-1}\vec{e}_j\cdot\vec{e}_k\rangle=\int_0^1\frac{\kappa_{jk}(d\lambda)}{t-\lambda}\,.
  \notag
\end{align}
%
Here $\mu_{jk}$ and $\alpha_{jk}$ are \emph{spectral measures} associated
with the random operators $\chi_1\Gamma\chi_1$ and $\chi_2\Gamma\chi_2$, respectively, while
$\eta_{jk}$ and $\kappa_{jk}$ are spectral measures associated
with the random operators $\chi_2\Upsilon\chi_2$ and $\chi_1\Upsilon\chi_1$, respectively (see
Section \ref{sec:The_Spectral_Theorem} for details).

Through the Stieltjes--Perron inversion theorem 
\cite{Henrici:1974:v2,MILTON:2002:TC}, it can be shown
\cite{Murphy:JMP:063506} that the measures $\mu_{jk}$ and $\alpha_{jk}$, and
the measures $\eta_{jk}$ and $\kappa_{jk}$ are related by (see Section
\ref{sec:Stieltjes-Perron} for details)   
%
\begin{align}\label{eq:Measure_Relations}
  &\lambda\alpha_{jk}(\lambda)=(1-\lambda)\mu_{jk}(1-\lambda) +
       \lambda\,(\,m_{jk}(0)\delta_0(d\lambda)+w_{jk}(0)(\lambda-1)\delta_1(d\lambda)\,),
  \\
  &\lambda\kappa_{jk}(\lambda)=(1-\lambda)\eta_{jk}(1-\lambda) +
       \lambda\,(\,\tilde{m}_{jk}(0)\delta_0(d\lambda)+\tilde{w}_{jk}(0)(\lambda-1)\delta_1(d\lambda)\,),
  \notag     
\end{align}
%
where $\delta_a(d\lambda)$ is the delta measure concentrated at $\lambda=a$. Equations
\eqref{eq:Stieltjes_F} and \eqref{eq:Measure_Relations} demonstrate
the many symmetries between the functions $m_{jk}(h)$, $w_{jk}(z)$,
$\tilde{m}_{jk}(h)$, and $\tilde{w}_{jk}(z)$, and the respective 
measures $\mu_{jk}$, $\alpha_{jk}$, $\eta_{ij}$, and $\kappa_{jk}$. Because of these
symmetries, for simplicity, we will focus on $m_{jk}(h)$ and $\mu_{jk}$,
and will reintroduce the other functions and measures where
appropriate.  



A key feature of equations \eqref{eq:eff_eps_def}, \eqref{eq:m_h}, and
\eqref{eq:Stieltjes_F} is that the parameter information in $h$ and
$E_0$ is \emph{separated} from the geometry of the composite, which is
encoded in the spectral measure $\mu_{jk}$ via its moments 
$\mu_{jk}^n=\int_0^1\lambda^n\mu_{jk}(d\lambda)$, $n=0,1,2,\ldots$. For example, the mass
$\mu_{jk}^0$ of the measure $\mu_{jk}$ is given by $\mu_{jk}^0=p_1\delta_{jk}$
(see equation \eqref{eq:Mass_Sign_Measures}), where $p_1=\langle\chi_1\rangle$ is the
volume fraction of material component 1. Moreover, if the medium is 
statistically isotropic then $\mu_{jk}^1=(p_1p_2/d)\,\delta_{jk}$
\cite{Bruno:JSP-365,Golden:IMA-97}, where $p_2=1-p_1=\langle\chi_2\rangle$ is the
volume fraction of material component 2.   




A principal application of the ACM is to derive \emph{forward bounds}
on the diagonal components $\sigma_{kk}^*$ of the tensor $\bsig^*$,
$k=1,\ldots,d$, given partial information on the microgeometry
\cite{Bergman:PRL-1285,Milton:APL-300,Golden:CMP-473,Bergman:AP-78}. This
information may be given in terms of the $(n+1)$-point correlation
functions of the medium, or equivalently the moments $\mu_{kk}^n$,
$n=0,1,2,\ldots$, of the measure $\mu_{kk}$
\cite{Milton:JAP-5294,Golden:CMP-473}. Given this information, the 
bounds on $\sigma_{kk}^*$ follow from the special structure of $F_{kk}(s)$
in \eqref{eq:Stieltjes_F}. More specifically, it is a \emph{linear}
functional of the \emph{positive} measure $\mu_{kk}$.  The bounds are
obtained by fixing the contrast parameter $s$, varying over an
admissible set of measures $\mu_{kk}$ (or geometries) which is
determined by the known information regarding the two-component
composite.  Knowledge of the moments $\mu_{kk}^n$  for $n=1,\ldots,J$ confines
$\sigma_{kk}^*$ to a region of the complex plane which is bounded by arcs
of circles, and the region becomes progressively smaller as more
moments are known \cite{Milton:JAP-5294,Golden:1986:BCP}. When 
all the moments are known the measure $\mu_{kk}$ is uniquely determined 
\cite{Akhiezer:Book:1965}, hence  $\sigma_{kk}^*$ is explicitly known. For
your convenience, the bounding procedure is reviewed in Section
\ref{sec:Bounding_Procedure}.  









\subsection{Lattice Setting}
\label{sec:Lattice_Setting}
%
In this section we formulate the effective parameter problem for the
infinite and finite, two-component bond lattice on $\mathbb{Z}^d$. The
infinite bond lattice, reviewed in Section
\ref{sec:Infinite_Lattice_Setting}, is a special case of the
stationary random medium considered in Section
\ref{sec:Continuum_Setting}. In Section
\ref{sec:Finite_Lattice_Setting} we develop the mathematical framework
for the ACM in the finite lattice setting. 



\subsubsection{Infinite Lattice Setting}
\label{sec:Infinite_Lattice_Setting}
%
Consider a two-component bond lattice on all of $\mathbb{Z}^d$
determined by the probability space $(\Omega,P)$, and let
$\bsig(\vec{x},\omega)$ be the local complex conductivity tensor with
components $\sigma_{jk}(\vec{x},\omega)=\sigma^j(\vec{x},\omega)\delta_{jk}$, $j,k=1,\ldots,d$. Here
$\sigma^j(\vec{x},\omega)$ is the conductivity of the bond emanating from 
$\vec{x}\in\mathbb{Z}^d$ in the positive $j^{\,\text{th}}$ direction, 
which is a stationary random field that takes the \emph{complex} values
$\sigma_1$ and $\sigma_2$ with probabilities $p_1$ and $p_2=1-p_1$,
respectively \cite{Golden:CMP-467,Bruno:JSP-365}. The configuration
space $\Omega=\{\sigma_1,\sigma_2\}^{d\mathbb{Z}^d}$ represents the set of all
realizations of the random medium and the 
probability measure $P$ is compatible with stationarity, as defined in
Section \ref{sec:Resolvent_Representations}. Analogous to equation
\eqref{eq:two-phase_eps}, the local conductivity $\sigma^j(\vec{x},\omega)$ of
the two-phase random medium takes the form \cite{Golden:CMP-467}
%
\begin{align}\label{eq:two-phase_sigma}
  \sigma^j(\vec{x},\omega)=\sigma_1\chi_1^j(\vec{x},\omega)+\sigma_2\chi_2^j(\vec{x},\omega), \quad j=1,\ldots,d.
\end{align}
%
Here $\chi_i^j(\vec{x},\omega)$ is the characteristic function of medium
$i=1,2$, which equals one for all realizations $\omega\in\Omega$ having medium $i$
in the $j^{\,\text{th}}$ positive bond at $\vec{x}$, and equals zero
otherwise.




In this lattice setting, the differential operators $\vec{\nabla}\times$ and
$\vec{\nabla}\cdot$ in equation \eqref{eq:Maxwells_Equations_E} are given 
\cite{Golden:CMP-467,Bruno:JSP-365} in terms of forward and backward
difference operators $D_j^+$ and $D_j^-$, respectively, where
%
\begin{align}\label{eq:Difference_Operators}
  D_j^+=T_j^+-I, \quad D_j^-=I-T_j^-, \quad j=1,\ldots,d.
\end{align}
%
Here $I$ is the identity operator on $\mathbb{Z}^d$, and 
$T_j^+=T_{+\vec{e}_j}$ and $T_j^-=T_{-\vec{e}_j}$ are the generators (through 
composition) of the unitary group $T_x$ acting on $L^2(\Omega,P)$ defined
by $(T_xf)(0,\omega)=f(\vec{x},\omega)$, for any $f\in L^2(\Omega,P)$
%$f:\mathbb{Z}^d\times\Omega\to\mathbb{R}$,
which is a stationary random field
\cite{Golden:CMP-467} (see Section \ref{sec:Stationarity} for
details). Define $\mathscr{H}=\bigotimes_{i=1}^dL^2(\Omega,P)$ and let
$\vec{E},\vec{J}\in \mathscr{H}$ be  
the random electric field and current density,
respectively, where $\vec{E}(\vec{x},\omega)=(E^1(\vec{x},\omega),\ldots
E^d(\vec{x},\omega))$ and $E^j(\vec{x},\omega)$ is the electric field in the
bond emanating from $\vec{x}$ in the positive $j^{\,\text{th}}$
direction, and similarly for $\vec{J}(\vec{x},\omega)$. 


As in Section \ref{sec:Continuum_Setting} we write
$\vec{E}=\vec{E}_0+\vec{E}_f$, where $\vec{E}_f$ is the fluctuating
field of mean zero about the (constant) average $\vec{E}_0$. 
The variational problem in \eqref{eq:Weak_Curl_Free_Variational_Form}
for this lattice setting has a unique solution satisfying Kirchhoff's
circuit laws \cite{Golden:CMP-473,Bruno:JSP-365}      
%
\begin{align}\label{eq:Kirchhiff's__Equations}
  D_i^+E^j-D_j^+E^i=0, \quad
  \sum_{k=1}^dD_k^-J^k=0, \quad
  J^i=\sigma^iE^i, \quad
%  \vec{E}=\vec{E}_0+\vec{E}_f, \quad
  \langle\vec{E}\rangle=\vec{E}_0,
\end{align}
%
where $i,j=1,\ldots,d$ and the components $E^i(\vec{x},\omega)$ and
$J^i(\vec{x},\omega)$ of $\vec{E}(\vec{x},\omega)$ and $\vec{J}(\vec{x},\omega)$ are
stationary random fields.  
Equation \eqref{eq:Kirchhiff's__Equations} is a direct analogue of
equation \eqref{eq:Maxwells_Equations_E} when written in component
form \cite{Golden:CMP-473}. Analogous to equation
\eqref{eq:eff_eps_def}, the effective complex 
conductivity tensor $\bsig^*$ is defined by
$\langle\vec{J}\rangle=\bsig^*\vec{E}_0$, which has components
$\sigma^*_{jk}=\sigma_2\,m_{jk}(h)$, $j,k=1,\ldots,d$, where $h=\sigma_1/\sigma_2$. All of the
results stated in Section \ref{sec:Continuum_Setting}, including the
representation formula for $m_{jk}(h)$ in
\eqref{eq:Stieltjes_F}, still hold in this infinite lattice setting 
with $\Gamma$ in \eqref{eq:Resolvent_representations_E_D} now given by 
%
\begin{align}\label{eq:Discrete_Gamma}
  \Gamma=\nabla^+(\Delta^{-1})\nabla^-, \quad \nabla^\pm = (D_1^\pm,\ldots,D_d^\pm),
\end{align}
%
where $\Delta^{-1}$ is based on discrete convolution with the lattice
Green's function for the Laplacian $\Delta=\nabla^2$ \cite{Bruno:JSP-365}. The
formulation of the ACM for the effective resistivity tensor $\brho^*$
in the infinite lattice setting is a direct analogue of that for
$\bsig^*$ given here (see Sections \ref{sec:Finite_Lattice_Setting}
and \ref{sec:Resolvent_Representations} for more details regarding
this formulation).


\subsubsection{Finite Lattice Setting}
\label{sec:Finite_Lattice_Setting}
%
Consider a finite, two-component bond lattice on
$\mathbb{Z}_L^d\subset\mathbb{Z}^d$ determined by the probability space
$(\Omega,P)$, where
%
\begin{align}\label{eq:ZLd}
  \mathbb{Z}_L^d=\{\vec{x}\in\mathbb{Z}^d \ | \ 1\leq x_i\leq L, \ i=1,\ldots,d\},   
\end{align}
%
$L\in\mathbb{N}$, $L\geq2$, and $x_i=(\vec{x}\,)_i$ is the $i^{\,\text{th}}$
component of the vector $\vec{x}$. Let $\bsig(\vec{x},\omega)$ be the local
complex conductivity tensor with components
$\sigma_{jk}(\vec{x},\omega)=\sigma^j(\vec{x},\omega)\delta_{jk}$, $j,k=1,\ldots,d$, where
$\sigma^j(\vec{x},\omega)$ is defined in equation \eqref{eq:two-phase_sigma} for
$\vec{x}\in \mathbb{Z}_L^d$ and $\omega\in\Omega$. The configuration space
$\Omega=\{\sigma_1,\sigma_2\}^{d\mathbb{Z}_L^d}$ represents the set of all $2^N$
realizations of the finite random lattice, where $N=d\,L^d$, and $P$ is
the associated probability measure. Define
$\mathscr{H}=\bigotimes_{i=1}^dL^2(\Omega,P)$ and let
$\vec{E},\vec{J}\in \mathscr{H}$ be the random
electric field and current density, respectively, which satisfy
Kirchhoff's circuit laws \eqref{eq:Kirchhiff's__Equations} with
appropriate boundary conditions. Analogous to equation 
\eqref{eq:eff_eps_def}, the effective complex conductivity tensor
$\bsig^*$ is defined by $\langle\vec{J}\rangle=\bsig^*\vec{E}_0$, which
has components $\sigma_{jk}^*=\sigma_2m_{jk}(h)$, $\vec{E}_0=\langle\vec{E}\rangle$, and
$\langle\cdot\rangle$ denotes ensemble average over $\Omega$.




In this section we obtain discrete versions of the integral
representations for $m_{jk}(h)$ and $\tilde{w}_{jk}(z)$ in
\eqref{eq:Stieltjes_F} for this finite bond lattice setting, involving
spectral measures $\mu_{jk}$ and $\kappa_{jk}$ associated with real-symmetric
random matrices. The integral representations for $m_{jk}(h)$ and
$w_{jk}(z)$ are analogous. Toward this goal, we define a bijective
mapping $\Theta$ from the two-dimensional set $\mathbb{Z}_L^d$ onto the one
dimensional set $\mathbb{N}_L\subset\mathbb{N}$ given by 
%$\Theta:\mathbb{Z}_L^d\to\mathbb{N}_L$, by
% 
\begin{align}\label{eq:Bijection_Z_N}
  \mathbb{N}_L=\{i\in\mathbb{N} \ | \ i\leq dL^d\}, \qquad
  \Theta(\vec{x}\,)=x_1+\sum_{k=2}^d(x_k-1)L^{k-1}.
  %\quad x_k=1,\ldots,L, \ \forall \ k.
\end{align}
%
Under the bijection $\Theta$ the components $E^j(\vec{x},\omega)$, $j=1,\ldots,d$, of
the random electric field
$\vec{E}(\vec{x},\omega)=(E^1(\vec{x},\omega),\ldots ,E^d(\vec{x},\omega))$ are mapped to
vector valued functions $E^j(\vec{x},\omega)\mapsto\vec{E}^j(\omega)=(E^j_1(\omega),\ldots,
E^j_{L^d}(\omega))$ so that
% 
\begin{align}\label{eq:bijection_vector_mapping}
  \Theta(\vec{E}(\vec{x},\omega))=(\vec{E}^1(\omega),\ldots ,\vec{E}^d(\omega))\in\mathbb{R}^N
\end{align}
%
%$\vec{E}(\vec{x},\omega)\mapsto(\vec{E}^1(\omega),\ldots \vec{E}^d(\omega))\in\mathbb{R}^N$,
for each $\omega\in\Omega$, and similarly for $\vec{J}(\vec{x},\omega)$. Moreover, the
bijection $\Theta$ maps the standard basis vector
$\vec{e}_1=(1,0,\ldots,0)\in\mathbb{Z}^d$, for example, to the
vector $(\vec{1},\vec{0},\ldots,\vec{0})\in\mathbb{Z}^N$, where
$\vec{1}$ and $\vec{0}$ are vectors of ones and zeros of length
$L^d$, respectively, and similarly for the $\vec{e}_j$ for
$j=2,\ldots,d$. Therefore, the vectors $\hat{e}_i=\Theta(\vec{e}_i)/L^{d/2}$,
$i=1,\ldots,d$, serve as the standard basis vectors on $\mathbb{N}_L$, with
$\hat{e}_i\cdot\hat{e}_j=\delta_{ij}$.
% In view of equation
% \eqref{eq:bijection_vector_mapping} it is natural to decompose the
% set $\mathbb{N}_L$ as follows
% %
% \begin{align}\label{eq:decompose}
%   \mathbb{N}_L=\cup_{j=1}^d\mathbb{N}_L^j, \quad
%   \mathbb{N}_L^j=\{i\in\mathbb{N}_L \ | \ (j-1)L+1\leq i\leq jL\}, \quad
%   j=1,\ldots,d. 
% \end{align}
% %


On $\mathbb{N}_L$ the difference operators $D_j^\pm$, $j=1,\ldots,d$, in
equation \eqref{eq:Difference_Operators} are given in terms of finite 
difference matrices $D_j$
%of size $L^d\times L^d$
\cite{Demmel:1997}. Moreover, the Laplacian $\Delta$ and
the projection operator $\Gamma$ in \eqref{eq:Discrete_Gamma} are replaced 
by the real-symmetric matrices $\Delta=\nabla^{\,T}\nabla$ and $\Gamma=\nabla(\Delta^{-1})\nabla^{\,T}$,
respectively, where $\nabla=(D_1,\ldots,D_d)^T$. The matrices $\Delta$ and $\Gamma$
depend only on the topology and the boundary conditions 
of the underlying finite bond lattice $\mathbb{Z}_L^d$, and $\Gamma$ is a
projection matrix satisfying $\Gamma^{\,2}=\Gamma$.


The projection matrix representation the operator $\Upsilon$ on
$\mathbb{N}_L$ is obtained as follows. In three dimensions the curl
operation is given by 
%
\begin{align}\label{eq:Curl_3D}
  \vec{\nabla}\times\vec{\zeta}=
     \text{Det}\left[
  \begin{array}{ccc}
    \vec{e}_1 &\vec{e}_2 &\vec{e}_3\\
       \partial_1    &   \partial_2    &   \partial_3   \\
       \zeta_1    &   \zeta_2    &   \zeta_3   
    \end{array}
    \right]
    =C\vec{\zeta}, \quad
    C=
    \left[
    \begin{array}{ccc}
       0  & -\partial_3  &   \partial_2 \\
       \partial_3 &  0   &  -\partial_1 \\
      -\partial_2 &  \partial_1  &   0
    \end{array}
    \right],    
\end{align}
%
where $\vec{\zeta}=\vec{\zeta}(\vec{x})$ for $\vec{x}\in\mathbb{R}^3$, we
have denoted $\partial_i$, $i=1,2,3$, to be partial differentiation in the
$i^{\;\text{th}}$ direction $\vec{e}_i$, and $C$ is the curl operator in
matrix form. One can check directly
that $C^{\,2}=-C^TC=-\bDelta+\vec{\nabla}\vec{\nabla}\cdot$, where $\bDelta$ is the vector
Laplacian. For dimensions $d>3$, the condition $\vec{\nabla}\times\vec{\zeta}=0$ is
given by $\partial_j\zeta_j-\partial_j\zeta_i=0$, $i,j=1,\ldots,d$,
\cite{Golden:CMP-473}. Consequently, the corresponding matrix
representation $C$ of the operator $\vec{\nabla}\times$ becomes increasingly
rectangular with increasing dimension. The two-dimensional case
follows from \eqref{eq:Curl_3D} by setting
$\vec{\zeta}(\vec{x})=[\zeta_1(\vec{x}),\zeta_2(\vec{x}),0]^T$ with
$\vec{x}=[x_1,x_2,0]^T$, yielding   
%
\begin{align}\label{eq:Curl_2D}
  \vec{\nabla}\times\vec{\zeta}=(\partial_1\zeta_2-\partial_2\zeta_1)\vec{e}_3=
  \left[
    \begin{array}{ccc}
       -\partial_2  &   \partial_1 \\      
    \end{array}
  \right]
   \left[
    \begin{array}{ccc}
       \zeta_1  \\
       \zeta_2   
    \end{array}
  \right]\vec{e}_3
  =
  \left[
    \begin{array}{ccc}
       \partial_1  &   \partial_2 \\      
    \end{array}
  \right]
  \left[
    \begin{array}{ccc}
        0  &  1  \\
       -1  &  0  
    \end{array}
  \right]
  \left[
    \begin{array}{ccc}
       \zeta_1  \\
       \zeta_2   
    \end{array}
  \right]\vec{e}_3.  
\end{align}
%



In view of equation \eqref{eq:Curl_3D}, the matrix
representation of the curl operator $\vec{\nabla}\times$ on $\mathbb{N}_L$ in
\emph{three dimensions} is given by $C$ in \eqref{eq:Curl_3D} under the
mapping $\partial_i\mapsto D_i$, $i=1,2,3$. In two dimensions, pointwise rotations
of fields by $90^\circ$ convert curl free fields to divergence free
fields, and vice versa \cite{MILTON:2002:TC}. In view of
the rotation matrix in \eqref{eq:Curl_2D}, for \emph{two dimensions},
it is natural to define the curl operator $\vec{\nabla}\times$ on 
$\mathbb{N}_L$ by 
%
\begin{align}
  \vec{\nabla}\times\vec{\zeta}=C^T\vec{\zeta}, \quad
  C^T=\left[
    \begin{array}{ccc}
       -D_2  &   D_1
    \end{array}
  \right],
\end{align}
%
which satisfies $C^TC=\Delta$. In view of \eqref{eq:GammaCurl} we define the
matrix representation of $\Upsilon$ on $\mathbb{N}_L$ to be
%
\begin{align}\label{eq:GammaCurl_NL}
  \Upsilon=C(C^TC)^{-1}C^T,
\end{align}
%
which is clearly a projection matrix $\Upsilon^2=\Upsilon$.


We now discuss the matrix representation of
the characteristic 
function $\chi_1^j(\vec{x},\omega)$. By writing the constitutive
relation $J^j(\vec{x},\omega)=\sigma^j(\vec{x},\omega)E^j(\vec{x},\omega)$ displayed in
equation \eqref{eq:Kirchhiff's__Equations} as
$J^j(\vec{x},\omega)=\sigma_2(1-\chi_1^j(\vec{x},\omega)/s)E^j(\vec{x},\omega)$, 
we see that the characteristic
function $\chi_1^j(\vec{x},\omega)$ in \eqref{eq:two-phase_sigma} operates,
according to the probability measure $P$, on the electric field
$E^j(\vec{x},\omega)$ in each individual bond $j=1,\ldots,d$ emanating from
$\vec{x}\in\mathbb{Z}_L^d$. In view of this and equation
\eqref{eq:bijection_vector_mapping}, on
$\mathbb{N}_L$ the characteristic function $\chi_1^j(\vec{x},\omega)$ is
represented by a block diagonal matrix 
%
\begin{align}\label{eq:block_diag_chi}  
  \chi_1(\omega)=\text{diag}(\chi_1^1(\omega),\ldots,\chi_1^d(\omega)),
\end{align}
%
where $\chi_1^j(\omega)$, $j=1,\ldots,d$, is a diagonal matrix of size $L^d\times L^d$ with
zeros and ones distributed according to $P$ along the main
diagonal. Moreover, the matrix $\chi_1^j(\omega)$ acts on the vector
$\vec{E}^j(\omega)=\Theta(E^j(\vec{x},\omega))$ in
\eqref{eq:bijection_vector_mapping} for each $j=1,\ldots,d$. Consequently, 
$\chi_1(\omega)$ is also a real-symmetric projection matrix of size $N\times N$,
which determines the geometry and component connectivity of the
two-phase random medium. In summary, on $\mathbb{N}_L$ the operators
$M_1=\chi_1\Gamma\chi_1$ and $K_1=\chi_1\Upsilon\chi_1$ are represented by real-symmetric
random matrices of size $N\times N$
\cite{Golden:J_Biomech:337,Murphy:JMP:063506}. 




In Section \ref{sec:The_Spectral_Theorem_Continuum} we discuss how the
spectral theorem under the ACM leads to the Stieltjes integral
representations for the effective parameters displayed in equation
\eqref{eq:Stieltjes_F}, for the continuum and infinite lattice
settings. In Section \ref{sec:The_Spectral_Theorem_Finite_Lattice} we
discuss how these spectral methods must be modified for the finite
lattice setting presented here. These modifications yield
the following Stieltjes integral representation for the effective
conductivity $\bsig^*$, which has components $\sigma^*_{jk}=\sigma_2\,m_{jk}(h)$,
$j,k=1,\ldots,d$, that satisfy
%
\begin{align}\label{eq:Stieltjes_F_Discrete}
  &m_{jk}(h)=\delta_{jk}-F_{jk}(s), %\quad
  &&F_{jk}(s)=\int_0^1\frac{\mu_{jk}(d\lambda)}{s-\lambda}\,, %\quad
  &&\mu_{jk}(d\lambda)=\sum_{i=1}^N\langle \delta_{\lambda_i}(d\lambda)\chi_1R_i\hat{e}_j\cdot\hat{e}_k\rangle.
\end{align}
%
Here we have defined $h=\sigma_1/\sigma_2$,
$s=1/(1-h)$, $R_i=\vec{u}_i\vec{u}_i^{\,T}$ for $i=1,\ldots,N$, $\vec{u}_i$
is the $i^{\,\text{th}}$ eigenvector of the matrix $M_1$ 
with associated eigenvalue $\lambda_i$, $\delta_{\lambda_i}(d\lambda)$ is the delta measure 
concentrated at $\lambda_i$, and $\langle\cdot\rangle$ denotes ensemble average over
$\Omega$. The integral representations of the functions $w_{jk}(z)$,
$\tilde{m}_{jk}(h)$, and $\tilde{w}_{jk}(z)$ are analogous. The mass
$\mu_{jk}^0$ of the spectral measure $\mu_{jk}$ is given by  
%
\begin{align*}
  \mu_{jk}^0=\langle\chi_1\hat{e}_k\cdot\hat{e}_k\rangle\,\delta_{jk}
       =d\,\frac{\langle N_1^k(\omega)\rangle}{N}\,\delta_{jk}       
\end{align*}
%
where $N_1^k(\omega)={\rm Trace}(\chi_1^k(\omega))$ is the total number of type-one
bonds in the positive $k^{\text{th}}$ direction for
$\omega\in\Omega$. Furthermore, in the case of locally isotropic random media
$\mu_{jk}^0=p_1\delta_{jk}$, where  $p_1$ is the number fraction of type-one
bonds (see Section \ref{sec:The_Spectral_Theorem_Finite_Lattice} for
details).  







\section{Numerical Results}\label{sec:Numerical_Results}
{\it Eigenvalue densities, Spectral measures, forward bounds on
  effective parameters. What the new data and bounds tell us about the
  transitional behavior of the multiscale sea ice structures.}
%
In this section we discuss how the theoretical framework developed in
Sections \ref{sec:Finite_Lattice_Setting} and
\ref{sec:The_Spectral_Theorem_Finite_Lattice} can be used to directly
calculate the spectral measure $\bmu$, hence the effective parameters
for various composite microstructures. In Section \ref{} we calculate
$\bmu$ and  
%
\section{Conclusion}
{\it What's new in this electromagnetic/thermal case and what can it tell us about sea ice? How could this potentially allow small scale processes to be up-scaled into coarsened climate models? How can the electromagnetic response of sea ice provide information regarding the fluid flow processes? Inverse problems. Spectral measures for the effective elasticity tensor of the ice pack....}

\newpage

% redefine the command that creates the equation no.
  \setcounter{equation}{1}  % reset equation counter
  \setcounter{section}{0}  % reset section counter
  \renewcommand{\theequation}{A-\arabic{equation}}
\renewcommand{\thesection}{A-\arabic{section}}
\section{Appendix} \label{sec:Appendix}


%
 
%
In this section we provide many of the details to, and derivations of
the formulas given in Section \ref{sec:Mathematical_Methods}. Much of this
appendix is devoted to reviewing well established results, in order to
clarify the technical details underlying the mathematical
framework. However, a key theoretical contribution of this work is
given in Theorem \ref{thm:Discrete_Spectral_Theorem_ACM} of Section
\ref{sec:The_Spectral_Theorem_Finite_Lattice} below.   

\subsection{Stationary Random Fields}\label{sec:Stationarity}
%
A \emph{stationary} random field $f\in L^2(\Omega,P)$,
$f:\mathbb{R}^d\times\Omega\to\mathbb{R}$, is a field such that the 
joint distribution of $f(\vec{x}_1,\omega),\ldots,f(\vec{x}_n,\omega)$ and that of
$f(\vec{x}_1+\vec{\xi},\omega),\ldots,f(\vec{x}_n+\vec{\xi},\omega)$ is the same for all
$\vec{\xi}\in\mathbb{R}^d$ and $n\in\mathbb{N}$ \cite{Golden:CMP-473}. In
particular, the ensemble average $\langle\cdot\rangle$ of $f(\vec{x},\omega)$ over $\Omega$ is
invariant under the translation group $\tau_y:\Omega\to\Omega$ defined by
$f(\tau_{-y}\vec{x},\omega)=f(\vec{x}+\vec{y},\omega)$ for all
$\vec{x},\vec{y}\in\mathbb{R}^d$, with $\tau_x\tau_y=\tau_{x+y}$. Consequently, 
$\langle f(\vec{x},\omega)\rangle=\langle f(0,\omega)\rangle$ and we can focus on the origin and drop the
$\vec{x}$ notation by writing $f(0,\omega)=f(\omega)$, with
$f(\tau_{-x}\,\omega)=f(\vec{x},\omega)$. We shall assume that there is such a group
of transformations that is one-to-one and preserves the measure $P$,
i.e. $P(\tau_xA)=P(A)$ for all $P$-measurable sets $A$
\cite{Golden:CMP-473,Papanicolaou:RF-835}.

The group of transformations $\tau_x$ acting on $\Omega$ induces a group of
operators $T_x$ on the Hilbert space $L^2(\Omega,P)$ defined by
$(T_xf)(\omega)=f(\tau_{-x}\,\omega)$ for all $f\in L^2(\Omega,P)$. Since $\tau_x$ is measure
preserving, the operators $T_x$ form a unitary group and therefore
have closed densely defined infinitesimal generators $L_i$ in each
direction $i=1,\ldots,d$ with domain $\mathscr{J}_i\subset L^2(\Omega,P)$
\cite{Golden:CMP-467}. Thus,  
%
\begin{align*}
  L_i=\left.\frac{\partial}{\partial x_i}T_x \right|_{x=0}, \quad i=1,\ldots,d,
\end{align*}
%
where $x_i$
%$x_i=(\vec{x}_i)$
is the $i^{\,\text{th}}$ component of the vector
$\vec{x}$ and differentiation is defined in the sense of convergence in $L^2(\Omega,P)$
for elements of $\mathscr{J}_i$ \cite{Golden:CMP-467}. The closed subset
$\mathscr{J}=\cap_{i=1}^d\mathscr{J}_i$ of $L^2(\Omega,P)$ is a Hilbert space
\cite{Golden:CMP-467} with inner product $\langle \cdot,\cdot\rangle_D$ given by 
$\langle f,g\rangle_D=\langle f,g\rangle+\sum_{i=1}^d\langle L_if,L_ig\rangle$, where $\langle\cdot,\cdot\rangle$ is the $L^2(\Omega,P)$
inner product.
(IS THERE A LIMIT THEOREM FROM LATTICE TO CONTINUUM THAT WE COULD CITE
HERE?)
In the case of a two-phase locally isotropic random medium, where the
local fields can be non-differentiable at the phase boundaries, the
Hilbert spaces $\mathscr{H}_\times$ and $\mathscr{H}_\bullet$ of ``curl free''
and ``divergence free'' random fields given in equation
\eqref{eq:curlfreeHilbert} are more precisely defined by
\cite{Golden:CMP-467}   
%
\begin{align}\label{eq:curlfreeHilbert_Precise}
  &\mathscr{H}_\times=
  \left\{Y_i(\omega)\in L^2(\Omega,P), \ i=1,\ldots,d \ | \ L_iY_j-L_jY_i=0 \text{ weakly and }
    \langle Y_i\rangle=0\right\},
    \\
  &\mathscr{H}_\bullet=
  \left\{Y_i(\omega)\in L^2(\Omega,P), \ i=1,\ldots,d \ \Big| \ \sum_{i=1}^dL_iY_i=0 \text{ weakly and }
    \langle Y_i\rangle=0\right\}.
    \notag
\end{align}
%
where $j=1,\ldots,d$.




\subsection{Resolvent Representations of Physical Fields}
\label{sec:Resolvent_Representations}
%
When the current density $\vec{J}(\vec{x},\omega)$ and the electric
field $\vec{E}(\vec{x},\omega)$ are sufficiently smooth in a neighborhood
of $\vec{x}\in\mathbb{R}^d$ for $\omega\in\Omega$, equation
\eqref{eq:Resolvent_representations_E_D} is obtained as
follows. On the Hilbert space $L^2(\Omega,P)$, the operator $\Delta^{-1}$ is
well defined in terms of convolution with respect to the free space
Green's function of the Laplacian $\Delta=\vec{\nabla}\cdot\vec{\nabla}$
\cite{Golden:CMP-473,Folland:95}. Similarly, on the Hilbert space
$\mathscr{H}=\bigotimes_{i=1}^dL^2(\Omega,P)$, the inverse $\bDelta^{-1}$ of the
vector Laplacian $\bDelta$ is defined in terms of component-wise
convolution with respect to the free space Green's function of the
Laplacian.


Applying the integro-differential operator $\vec{\nabla}(\Delta^{-1})$ to the
formula $\vec{\nabla}\cdot\vec{J}=0$ in equation
\eqref{eq:Maxwells_Equations_E} yields $\Gamma_\bullet\vec{J}=0$, where
$\Gamma_\bullet=\Gamma=\vec{\nabla}(\Delta^{-1})\vec{\nabla}\cdot$ is an orthogonal projection
\cite{Golden:CMP-473} from $\mathscr{H}$ onto the Hilbert space
$\mathscr{H}_\times$ of curl-free random fields,
$\Gamma:\mathscr{H}\mapsto\mathscr{H}_\times$. More specifically, for every
sufficiently smooth $\vec{\zeta}\in\mathscr{H}_\times$ there exists
\cite{Jackson-1999} a scalar potential $\varphi$ which is unique up to a
constant such that $\vec{\zeta}=\vec{\nabla}\varphi$, so that $\Gamma\vec{\zeta}=\vec{\zeta}$.



Similarly we have $\Upsilon\vec{E}=0$, where
$\Upsilon=\vec{\nabla}\times(\bDelta^{-1})\vec{\nabla}\times$ is an orthogonal projection from 
$\mathscr{H}$ onto the Hilbert space \cite{Golden:CMP-473}
$\mathscr{H}_{\bullet}$ of divergence-free random fields (of Coulomb or 
transverse gauge) \cite{Murphy:JMP:063506}. This can be seen as
follows. For every sufficiently smooth $\vec{\zeta}\in\mathscr{H}_\bullet$ we have
$\vec{\zeta}=\vec{\nabla}\times(\vec{A}+\vec{C})$, where $\vec{A}$ is a vector
potential associated with $\vec{\zeta}$ and the vector $\vec{C}$
satisfies $\vec{\nabla}\times\vec{C}=0$ \cite{Jackson-1999}. The arbitrary vector
$\vec{C}$ can be chosen so that $\vec{A}$ satisfies $\vec{\nabla}\cdot\vec{A}=0$
\cite{Jackson-1999}. Hence, $\vec{\nabla}\times\vec{\zeta}=\vec{\nabla}\times\vec{\nabla}\times\vec{A}
=\vec{\nabla}(\vec{\nabla}\cdot\vec{A})-\bDelta\vec{A}=-\bDelta\vec{A}$. The vector
$\vec{C}$ chosen in this manner gives the Coulomb (or transverse)
\emph{gauge} of $\vec{\zeta}$ \cite{Jackson-1999}. Choosing the members of
$\mathscr{H}_\bullet$ to have transverse gauge, the action of $\Upsilon$ on
$\mathscr{H}_\bullet$ is given by
%$\Upsilon=\vec{\nabla}\times(\vec{\nabla}\times\vec{\nabla}\times)^{-1}\vec{\nabla}\times=\vec{\nabla}\times(\bDelta^{-1})\vec{\nabla}\times$, 
%
\begin{align}\label{eq:GammaCurl}
  \Upsilon=\vec{\nabla}\times(\vec{\nabla}\times\vec{\nabla}\times)^{-1}\vec{\nabla}\times
  =\vec{\nabla}\times(\bDelta^{-1})\vec{\nabla}\times, 
\end{align}
%
and it is clear from the above discussion that $\Upsilon\vec{\zeta}=\vec{\zeta}$ for
all such $\vec{\zeta}\in\mathscr{H}_\bullet$.   



When considering a two-phase locally isotropic random media, the
local fields can be non-differentiable at the phase boundaries. In
this case, the differential operators $\vec{\nabla}\times$, $\vec{\nabla}\cdot$, and
$\vec{\nabla}$ are given \cite{Golden:CMP-473} in terms of the
infinitesimal generators $L_i$ defined in Section
\ref{sec:Stationarity}, and the Hilbert space $\mathscr{H}_\times$, for
example, is given by equation \eqref{eq:curlfreeHilbert_Precise}. The
Hilbert space $\mathscr{H}_\bullet$ is analogously defined.


We now derive the formulas in equation
\eqref{eq:Resolvent_representations_E_D}. Write $\sigma$ and $\rho$ in
equation \eqref{eq:two-phase_eps} as $\sigma=\sigma_2(1-\chi_1/s)=\sigma_1(1-\chi_2/t)$ and
$\rho=(1-\chi_2/s)/\sigma_1=(1-\chi_1/t)/\sigma_2$. Note that
$\vec{E}=\vec{E}_0+\vec{E}_f$, where $\vec{E}_0$ is a \emph{constant}
field and $\vec{E}_f\in\mathscr{H}_\times$ so that
$\Gamma\vec{E}=\vec{E}_f$, and similarly
$\Upsilon\vec{J}=\vec{J}_f$. Consequently, from $\Gamma\vec{J}=0$ and $\Upsilon\vec{E}=0$
we have the following formulas which are equivalent to that in 
\eqref{eq:Resolvent_representations_E_D}  
% 
\begin{align}\label{eq:Proj_rep_Ef_Jf}
  \vec{E}_f=\frac{1}{s}\Gamma\chi_1\vec{E}=\frac{1}{t}\Gamma\chi_2\vec{E}, \qquad
  \vec{J}_f=\frac{1}{t}\Upsilon\chi_1\vec{J}=\frac{1}{s}\Upsilon\chi_2\vec{J}.
\end{align}
%
%Here we have defined  $s=1/(1-h)$ and $t=1/(1-z)=1-s$, where
%$h=\sigma_1/\sigma_2$ and $z=1/h$.



It is important to note that the formulas $\Gamma\vec{E}=\vec{E}_f$ and
$\Upsilon\vec{J}=\vec{J}_f$ are sufficient conditions for the energy
constraints $\langle\vec{J}\cdot\vec{E}_f\rangle=0$ and $\langle\vec{E}\cdot\vec{J}_f\rangle=0$,
respectively, which follow from equation
\eqref{eq:Weak_Curl_Free_Variational_Form}. These energy constraints
are at the heart of the existence and uniqueness of solutions to
equations \eqref{eq:Maxwells_Equations_E} and
\eqref{eq:Kirchhiff's__Equations} for the continuum and lattice
settings, respectively. The sufficiency of these conditions can be seen
by writing $\sigma=\sigma_2(1-\chi_1/s)$ in $\vec{J}=\sigma\vec{E}$, for example, to
obtain          
% 
\begin{align}\label{eq:Field_Rep_s}
  \langle\vec{J}\cdot\vec{E}_f\rangle=\sigma_2(\langle\vec{E}\cdot\vec{E}_f\rangle-\langle\chi_1\vec{E}\cdot\vec{E}_f\rangle/s),
  %s=\frac{\langle\chi_1\vec{E}\cdot\vec{E}_f\rangle}{\langle\vec{E}\cdot\vec{E}_f\rangle}
  % =1-\frac{\langle\chi_2\vec{E}\cdot\vec{E}_f\rangle}{\langle\vec{E}\cdot\vec{E}_f\rangle},
\end{align}
%
for $s\neq0$ $(h\neq\pm\infty)$. Now, if we have $\Gamma\vec{E}=\vec{E}_f$ then
$\vec{\nabla}\cdot\vec{J}=0$ yields equation \eqref{eq:Proj_rep_Ef_Jf}
($\vec{E}_f=\Gamma\chi_1\vec{E}/s$). Therefore, as $\Gamma$ is a self-adjoint
operator on $\mathscr{H}$ \cite{Folland:95}, we have     
%
\begin{align}\label{eq:Suff_Cond}
  \langle\chi_1\vec{E}\cdot\vec{E}_f\rangle=\langle\chi_1\vec{E}\cdot\Gamma\vec{E}\rangle
                    =\langle\Gamma\chi_1\vec{E}\cdot\vec{E}\rangle
                    =s\langle\vec{E}_f\cdot\vec{E}\rangle.
\end{align}
%
Consequently, from equation \eqref{eq:Field_Rep_s}
we have $\langle\vec{J}\cdot\vec{E}_f\rangle=0$ for $s\neq0$. The argument involving the
operator $\Upsilon$ and the vector field $\vec{J}_f$ is analogous.


We conclude this section with some final remarks regarding the energy
constraint $\langle\vec{J}\cdot\vec{E}_f\rangle=0$, for example. We see from equation
\eqref{eq:Field_Rep_s} that the energy constraint is equivalent to the
following ``field representation'' for the contrast parameter
$s=\langle\chi_1\vec{E}\cdot\vec{E}_f\rangle/\langle\vec{E}\cdot\vec{E}_f\rangle$, when
$\langle\vec{E}\cdot\vec{E}_f\rangle\neq0$ (if and only if
$\langle\chi_1\vec{E}\cdot\vec{E}_f\rangle\neq0$). Since $h=1-1/s$ and $h\geq0$ for
$h\in\mathbb{R}$, this implies that
$|\langle\vec{E}\cdot\vec{E}_f\rangle|\leq|\langle\chi_1\vec{E}\cdot\vec{E}_f\rangle|$ for
$h\in\mathbb{R}$. Moreover, the energy constraint also provides the
limiting behavior of the ratio
$\mathcal{R}(h)=\langle\vec{E}\cdot\vec{E}_f\rangle/\langle\chi_1\vec{E}\cdot\vec{E}_f\rangle=1/s$
%
\begin{align*}
  \lim_{h\to0}\mathcal{R}(h)=1, \quad
  \lim_{h\to1}\mathcal{R}(h)=0, \quad
  \lim_{h\to+\infty}\mathcal{R}(h)=-\infty.
\end{align*}
%
Analogous formulas involving the vector field $\vec{J}_f$ also hold.



THE FORMULA $s=\langle\chi_1\vec{E}\cdot\vec{E}_f\rangle/\langle\vec{E}\cdot\vec{E}_f\rangle$ AND THE
ANALOGOUS FORMULA INVOLVING THE CURRENT DENSITY PROVIDES A
NON-LINEAR RELATIONSHIP BETWEEN $\sigma^*$ and $\rho^*$.

\subsection{The Stieltjes--Perron Inversion Theorem}
\label{sec:Stieltjes-Perron}
TAKE THIS FROM GRAEME'S BOOK
$m_{jk}(h)=h\,w_{jk}(z)$, and
$\tilde{m}_{jk}(h)=h\,\tilde{w}_{jk}(z)$

\subsection{The Spectral Theorem Under the ACM}
\label{sec:The_Spectral_Theorem}
%
In this section we discuss the spectral theorem
\cite{Reed-1980,Stone:64} as it pertains to the ACM. In Section
\ref{sec:The_Spectral_Theorem_Continuum} we review the spectral
theorem for the operators  $M_i=\chi_i\Gamma\chi_i$, $i=1,2$, in the continuum
and infinite lattice settings discussed in Sections
\ref{sec:Continuum_Setting} and \ref{sec:Infinite_Lattice_Setting},
respectively.  In Section
\ref{sec:The_Spectral_Theorem_Finite_Lattice} we discuss the spectral
theorem in the finite lattice setting, where the operators $M_i$,
$i=1,2$, are real-symmetric random matrices. In each case we obtain
the Stieltjes integral representation for $\bsig^*$ and $\bsig^*$
given in equations \eqref{eq:Stieltjes_F} and
\eqref{eq:Stieltjes_F_Discrete}, respectively, involving a matrix
valued spectral measure $\bmu$.     
%
\subsubsection{Continuum  and Infinite Lattice Settings}
\label{sec:The_Spectral_Theorem_Continuum} 
%
Due to the inherent symmetries in the ACM between the continuum and
infinite lattice settings discussed in Sections
\ref{sec:Continuum_Setting} and \ref{sec:Infinite_Lattice_Setting},
respectively, here we focus on the continuum setting. The
associated discussion regarding the infinite lattice setting is a
direct parallel to that given here. Let the random operators $\chi_i$,
$i=1,2$, and the non-random integro-differential operator $\Gamma$ be
defined as in equations \eqref{eq:two-phase_eps} and
\eqref{eq:Resolvent_representations_E_D}, respectively. On the Hilbert
space $\mathscr{H}_\times\subset\mathscr{H}$, $\Gamma$ and $\chi_i$, $i=1,2$, are
projection operators \cite{Golden:CMP-473}. Therefore $M_i=\chi_i\Gamma\chi_i$, 
$i=1,2$, are compositions of projection operators on $\mathscr{H}_\times$,
and are consequently positive definite and bounded by 1 in the
underlying operator norm \cite{Rudin:87}. They 
are self-adjoint with respect to the $\mathscr{H}$-inner-product
$\langle\cdot,\cdot\rangle$ \cite{Golden:CMP-473}. Therefore, on the Hilbert space
$\mathscr{H}_\times$ with weight $\chi_1$ in the inner-product,
$\langle\cdot,\cdot\rangle_1=\langle\chi_1\,\cdot,\cdot\rangle$ for example, $\Gamma\chi_1$ is a bounded linear
self-adjoint operator with spectrum in the interval $[0,1]$
\cite{Golden:CMP-473}. Hence the resolvent operator $(sI-\Gamma\chi_1)^{-1}$
is also self-adjoint with respect to the same inner-product, and is
bounded for $s\in\mathbb{C}\backslash[0,1]$ \cite{Stone:64}.   



By the spectral theorem \cite{Reed-1980,Stone:64} for the operator
$\Gamma\chi_1$ on the Hilbert space $\mathscr{H}_\times$ with inner-product
$\langle\cdot,\cdot\rangle_1$, there exists an increasing family of self-adjoint
projection operators $\{R(\lambda)\}$ - the resolution of the identity - that
satisfy $R(0)=0$ and $R(1)=I$ such that    
%
\begin{align}\label{eq:Spectral_Theorem}
  f(M_1)=\int f(\lambda)R(d\lambda), \quad
  \langle f(M_1)\,\vec{e}_j\cdot\vec{e}_k\rangle_1= \int_0^1f(\lambda)\mu_{jk}(d\lambda), 
  %\mu_{jk}(\lambda)=\langle R(\lambda)\vec{e}_j,\vec{e}_k\rangle_1
\end{align}
%
for all bounded continuous functions $f:\mathbb{C}\mapsto\mathbb{C}$. Here
$0$ and $I$ are the null and identity operators on $\mathbb{R}^d$,
respectively, $R(d\lambda)$ is the projection valued measure associated with
the operator $R(\lambda)$ \cite{Reed-1980}, and
$\mu_{jk}(d\lambda)=\langle R(d\lambda)\,\vec{e}_j\cdot\vec{e}_k\rangle_1$, $j,k=1,\ldots,d$, are the
components of the matrix valued \emph{spectral measure} $\bmu(d\lambda)$ in
the $(\vec{e}_j,\vec{e}_k)$ state
\cite{Golden:CMP-473,Reed-1980,Stone:64}.
Setting $f(\lambda)=(s-\lambda)^{-1}$ in \eqref{eq:Spectral_Theorem} yields the
the integral formula for $F_{jk}(s)$ and $\sigma_{jk}^*=\sigma_2m_{jk}(h)$ in
equation \eqref{eq:Stieltjes_F}. An analogous discussion involving the
operator $\chi_2\Gamma_\times\chi_2$ on the Hilbert space
$\mathscr{H}_{\bullet}$ and the function $E_{jk}(s)$ introduced in Section 
\ref{sec:Resolvent_Representations} leads to the integral
representation for $E_{jk}(s)$ displayed in equation
\eqref{eq:Stieltjes_Es} below.  




As the spectrum of the operator $M_1$ is contained in the interval
$[0,1]$, the support $\Sigma_{jk}$ of the measure $\mu_{jk}$ satisfies $\Sigma_{jk}\subseteq[0,1]$
\cite{Reed-1980}. The mass $\mu_{jk}^0$ of the measure $\mu_{jk}$ satisfies 
$\mu_{jk}^0=p_1\delta_{jk}$ for all $p_1\in[0,1]$, where $p_1=\langle\chi_1\rangle$ is the
volume fraction of material component one. To see this, note that 
setting $f(M_1)=I$ ($f(\lambda)=1$) in equation \eqref{eq:Spectral_Theorem}
implies that $\int_0^1R(d\lambda)=I$. Moreover, recall that the projection 
operator $R(\lambda)$ is self-adjoint on $\mathscr{H}_\times$ for all $\lambda\in\Sigma_{jk}$ 
\cite{Reed-1980,Stone:64}. Consequently, we have    
%
\begin{align}\label{eq:Mass_Sign_Measures}
   &\mu_{jk}^0=\int_0^1\mu_{jk}(d\lambda)=\int_0^1\langle R(d\lambda)\vec{e}_j\cdot\vec{e}_k\rangle_1
        =\langle1\rangle_1\,\vec{e}_j\cdot\vec{e}_k
        =\langle\chi_1\rangle\,\delta_{jk},\\
   &\mu_{kk}(d\lambda)=\langle R(d\lambda)\vec{e}_k\cdot\vec{e}_k\rangle_1
       =\langle R(d\lambda)\vec{e}_k\cdot R(d\lambda)\vec{e}_k\rangle_1
       =\|R(d\lambda)\vec{e}_k\|_1^2>0,\notag
\end{align}
%
where we have used a Fubini theorem \cite{Folland:99}
and $\|\cdot\|_1$ denotes the norm induced by the inner-product
$\langle\cdot,\cdot\rangle_1$. From equation \eqref{eq:Mass_Sign_Measures} we see,
generically, that the diagonal components $\mu_{kk}$ of $\bmu$ are 
positive measures of mass $p_1$, while the off-diagonal components
$\mu_{jk}$, $j\neq k=1,\ldots,d$, have zero mass and are consequently signed
measures \cite{Folland:99,Rudin:87}.



The higher order moments $\mu_{jk}^n$, $n=1,2,3,\ldots$, in principle, may be
found using a perturbation expansion of $F_{jk}(s)$ about a
homogeneous medium $(\sigma_1=\sigma_2, \ s=\infty)$ \cite{Golden:CMP-473}. In
particular $\mu_{jk}^0=p_1\delta_{jk}$, generically, and $\mu_{jk}^1=(p_1p_2/d)\,\delta_{jk}$
for a statistically isotropic random medium 
\cite{Golden:CMP-473,Bruno:JSP-365}. In the case of a  
square bond lattice, which is an example of an infinitely
interchangeable random medium, $\mu_{kk}^2=p_1p_2(1+(d-2)p_2)/d^2$
for any dimension $d$ and $\mu_{kk}^3=p_1p_2(p_2^2-p_2-1)/8$ for $d=2$.  
In general, the moments $\mu_{jk}^n$ depend on the $(n+1)$-point
correlation functions of the random medium
\cite{Golden:CMP-473,Bruno:JSP-365}.


% It is worth mentioning a few direct consequences of what we have
% established in this section. We first note that
% and the spectral theorem leads to a Herglotz integral representation
% of $\langle E_f^2\rangle=\langle\vec{E}\cdot\vec{E}_f\rangle=\langle\chi_1\vec{E}\cdot\vec{E}_f\rangle/s$ involving
% the measure $\bmu$ \cite{Murphy:JMP:063506}  
% %
% \begin{align}
%   \frac{\langle E_f^2\rangle}{E_0^2}=\frac{1}{s}\left(\langle\chi_1\vec{E}_f\cdot\vec{E}_0\rangle 
%                                +\langle\chi_1E_f^2\rangle
%                               \right)
%                   =\int_0^1\frac{\lambda\,\mu_{jj}(d\lambda)}{(s-\lambda)^2}\,,
% \end{align}
% %
% which, in turn, leads to a detailed decomposition of the system energy
% in terms of Hergloz functions involving the measure $\mu_{jj}$,
% $j=1,\ldots,d$ (see Section \ref{sec:The_Spectral_Theorem} for details).
% (FINISH THIS PARAGRAPH LATER AND VERIFY THAT THE POSITIVITY OF THE
% CONTRAST PARAMETER $h\in\mathbb{R}$ GIVES THE CORRECT SIGN FOR THE
% HERGLOTZ REPRESENTATIONS FOR THE QUANTITIES $\langle\vec{E}\cdot\vec{E}_f\rangle$ and 
% $\langle\chi_1\vec{E}\cdot\vec{E}_f\rangle$ in equation \eqref{eq:Field_Rep_s})





\subsubsection{Finite Lattice Setting}
\label{sec:The_Spectral_Theorem_Finite_Lattice}
%
In this section we derive a discrete version of the spectral theorem
given in equation  \eqref{eq:Spectral_Theorem}. This leads to
the discrete integral representation for the effective
conductivity tensor $\bsig^*$ displayed in equation
\eqref{eq:Stieltjes_F_Discrete}. Toward this goal, we defined in
Section \ref{sec:Finite_Lattice_Setting} a bijective mapping
$\Theta:\mathbb{Z}_L^d\to\mathbb{N}_L$ from the finite $d$-dimensional bond
lattice $\mathbb{Z}_L^d$ defined in \eqref{eq:ZLd} onto the one
dimensional set $\mathbb{N}_L$ defined in equation
\eqref{eq:Bijection_Z_N}. Moreover we showed that, under the mapping
$\Theta$, the random operator $M_1=\chi_1\Gamma\chi_1$ can be represented as a
\emph{real-symmetric} random matrix of size $N\times N$, where $N=d\,L^d$
\cite{Golden:J_Biomech:337,Murphy:JMP:063506}. More specifically, $\Gamma$
is a \emph{non-random} real-symmetric projection matrix ($\Gamma^2=\Gamma$) and
$\chi_1$ is a \emph{random} diagonal projection matrix with zeros and
ones along the main diagonal, and has the block diagonal form
displayed in equation \eqref{eq:block_diag_chi}. Since $M_1(\omega)$ is a
composition of projection matrices, it is positive definite,
$M_1(\omega)\vec{\xi}\cdot\vec{\xi}=(\Gamma\chi_1(\omega)\vec{\xi})\cdot(\Gamma\chi_1(\omega)\vec{\xi})\geq0$ for every
$\omega\in\Omega$ and $\vec{\xi}\in\mathbb{R}^N$, and consequently has spectra
$\Sigma^\lambda(\omega)\subseteq[0,1]$ \cite{Halmos-1958}.   






It is well known \cite{Halmos-1958,Keener-2000} that the eigenvectors
$\vec{u}_i(\omega)$, $i=1,\ldots,N$, of the symmetric matrix $M_1(\omega)$ form an orthonormal
basis for $\mathbb{R}^N$, for each $\omega\in\Omega$, i.e., 
$\vec{u}_j^{\,T}\vec{u}_k=\delta_{jk}$  and for every
$\vec{\xi}\in\mathbb{R}^N$ we have
$\vec{\xi}=\sum_{i=1}^N(\vec{u}_i^{\,T}\vec{\xi}\,)\vec{u}_i  
=\left(\sum_{i=1}^N\vec{u}_i\vec{u}_i^{\,T}\right)\vec{\xi}\,$. Consequently,       
%
\begin{align}\label{eq:Matrix_Rep_Spec_Theorem}
  \sum_{i=1}^NR_i(\omega)=I, \quad
  R_i(\omega)=\vec{u}_i(\omega)\vec{u}_i^{\,T}(\omega),  \quad
  \forall \     \omega\in\Omega,
\end{align}
%
where $I$ is the identity operator on $\mathbb{R}^N$ and the matrix
$R_i$ is the orthogonal projector ($R_iR_j=R_i\delta_{ij}$) onto the
eigenspace spanned by $\vec{u}_i$, which is associated with the
\emph{real} eigenvalue $\lambda_i(\omega)\in\Sigma^\lambda(\omega)$. 


Since $M_1\vec{u}_i=\lambda_i\vec{u}_i$, for each $i=1,\ldots,N$, equation
\eqref{eq:Matrix_Rep_Spec_Theorem} implies that we also have
$M_1R_i=\lambda_iR_i$ which, in turn, implies that the matrix  $M_1$ has the
spectral decomposition $M_1=\sum_{i=1}^N\lambda_iR_i$. By the orthogonality of
the projection matrices $R_i$ and by induction we have
$M_1^n=\sum_{i=1}^N\lambda_i^nR_i$ for all $n\in\mathbb{N}$, which implies that
$f(M_1)=\sum_{i=1}^Nf(\lambda_i)R_i$ for any polynomial
$f:\mathbb{C}\mapsto\mathbb{C}$.  This formula is a discrete version of the
first formula in equation \eqref{eq:Spectral_Theorem} for polynomial
$f(\lambda)$, and leads to a discrete version of the functional
representation of  $f(M_1)$ in \eqref{eq:Spectral_Theorem} involving a
matrix valued spectral measure $\bmu(d\lambda)$ with components $\mu_{jk}(d\lambda)$
%
\begin{align}\label{eq:Discrete_Spectral_Theorem}
%f(M_1)=\sum_{i=1}^Nf(\lambda_i)R_i, \quad
  \langle f(M_1)\hat{e}_j\cdot\hat{e}_k\rangle= \int_0^1f(\lambda)\mu_{jk}(d\lambda), \quad
  \mu_{jk}(d\lambda)=\sum_{i=1}^N\langle\delta_{\lambda_i}(d\lambda)R_i\,\hat{e}_j\cdot\hat{e}_k\rangle.
\end{align}
%
Here $R(d\lambda)=\sum_i\delta_{\lambda_i}(d\lambda)R_i$ is a discrete version of the
projection valued measure introduced in equation
\eqref{eq:Spectral_Theorem}, $\delta_{\lambda_i}(d\lambda)$ is the Dirac measure
concentrated at $\lambda_i$, the orthonormal vectors
$\hat{e}_i=\Theta(\vec{e}_i)/L^{d/2}$, for $i=1,\ldots,d$, represent the standard
basis vectors on $\mathbb{N}_L$, and $\langle\cdot\rangle$ denotes ensemble
average over $\Omega$. The spectral end points $\lambda_{jk}^0$ and $\lambda_{jk}^1$ of the
support $\Sigma_{jk}\subseteq[\lambda_{jk}^0,\lambda_{jk}^1]\subseteq[0,1]$ of the measure $\mu_{jk}$ in
\eqref{eq:Discrete_Spectral_Theorem} are given by
$\lambda_{jk}^0=\inf A_{jk}$ and $\lambda_{jk}^1=\sup A_{jk}$, where
%
\begin{align*}
 % \lambda_{jk}^0=\inf A_{jk}, \quad \lambda_{jk}^1=\sup A_{jk}, \quad
  A_{jk}= \cup_{\omega\in\Omega}\{\lambda_i(\omega)\in\Sigma^\lambda(\omega), \   i=1,\ldots,N \ | \ R_i(\omega)\hat{e}_j\cdot\hat{e}_k\neq0\}
  % \lambda_{jk}^0&=\inf \cup_{\omega\in\Omega}\{\lambda_i(\omega)\in\Sigma^\lambda(\omega), \
%   i=1,\ldots,N \ | \ R_i(\omega)\hat{e}_j\cdot\hat{e}_k\neq0\},\\
%   \lambda_{jk}^1&=\sup \cup_{\omega\in\Omega}\{\lambda_i(\omega)\in\Sigma^\lambda(\omega), \
%   i=1,\ldots,N \ | \ R_i(\omega)\hat{e}_j\cdot\hat{e}_k\neq0\},\notag
\end{align*}
%
and $\Sigma^\lambda_1(\omega)\subseteq[0,1]$ is the support of the eigenvalues of the matrix
$M_1(\omega)$ for $\omega\in\Omega$. 



We now show that equation \eqref{eq:Discrete_Spectral_Theorem} also
holds for the function $f(\lambda)=(s-\lambda)^{-1}$ when $s\not\in[0,1]$. For each
$\omega\in\Omega$, let $U(\omega)$ denote the matrix with columns consisting of the
eigenvectors $\vec{u}_i(\omega)$ of $M_1(\omega)$ and let
$\Lambda(\omega)=\text{diag}(\lambda_1(\omega),\ldots,\lambda_N(\omega))$ denote the diagonal matrix of the
corresponding eigenvalues $\lambda_i(\omega)$, $i=1,\ldots,N$, so that $M_1=U\Lambda U^T$
\cite{Halmos-1958}. By the orthogonality, $U^TU=UU^T=I$, of the matrix
$U$ we have   
%
\begin{align}\label{eq:Discrete_Stieltjes_F_Derivation}
     \langle f(M_1)\hat{e}_j\cdot\hat{e}_k\rangle
        &=\langle(sI-U\Lambda U^T)^{-1}\hat{e}_j\cdot\hat{e}_k\rangle
        = \langle U(sI-\Lambda)^{-1}U^T\hat{e}_j\cdot\hat{e}_k\rangle\\
        &=\langle(sI-\Lambda)^{-1}U^T\hat{e}_j\cdot U^T\hat{e}_k\rangle
       % =\sum_{i=1}^N\left\langle
       %      \frac{(\vec{u}_i^{\,T}\hat{e}_j)(\vec{u}_i^{\,T}\hat{e}_k)}{s-\lambda_i}
       %        \right\rangle
        =\sum_{i=1}^N\left\langle
          \frac{R_i\hat{e}_j}{s-\lambda_i}\cdot\hat{e}_k
          \right\rangle.               
%        m_{ij}(i)=\langle R_i\hat{e}_i,\hat{e}_j\rangle
%                =\langle(\vec{u}_i^{\,T}\hat{e}_i)(\vec{u}_i^{\,T}\hat{e}_j)\rangle,
        \notag
\end{align}
%
Equation \eqref{eq:Discrete_Stieltjes_F_Derivation} is equivalent to equation
\eqref{eq:Discrete_Spectral_Theorem} when the function
$f(\lambda)=(s-\lambda)^{-1}$, $s\in\mathbb{C}\backslash[0,1]$.



We now discuss the fundamental difference in the mathematical
framework between the \emph{infinite}
settings formulated in Sections \ref{sec:Continuum_Setting},
\ref{sec:Infinite_Lattice_Setting}, and
\ref{sec:The_Spectral_Theorem_Continuum}, and the \emph{finite}
lattice setting formulated here. In the infinite settings, the
(infinite-dimensional) operator $\Gamma\chi_1$ appears in the integral
representation \eqref{eq:Stieltjes_F} for the effective parameter,
involving the $\mathscr{H}$-inner-product 
weighted by $\chi_1(\vec{x},\omega)$. In this abstract (infinite-dimensional)
Hilbert space formulation of the effective parameter problem, the
resolvent $(sI-\Gamma\chi_1)^{-1}$ is a linear self-adjoint operator which is
bounded for $s\in\mathbb{C}\backslash[0,1]$ \cite{Stone:64}. In contrast, the
finite lattice formulation of the effective parameter problem involves
a finite dimensional Hilbert space, and the operators $\Gamma$ and $\chi_1$
are matrices. In this case, the matrix $\Gamma\chi_1$ is \emph{not} symmetric,
it typically has complex spectrum, and it may not even have a full set
of eigenvectors which span $\mathbb{R}^N$. Consequently, the integral
formulas in equations \eqref{eq:Discrete_Spectral_Theorem} and
\eqref{eq:Discrete_Stieltjes_F_Derivation}, which were derived for the
\emph{symmetric} matrix $M_1=\chi_1\Gamma\chi_1$, fail to hold for the matrix
$\Gamma\chi_1$ in general. Due to this fundamental difference in the theory
for the finite lattice setting, the mathematical framework must be 
significantly modified from that for the infinite settings. The ACM
for the finite lattice case is summarized by the following theorem.
% 
\begin{theorem}\label{thm:Discrete_Spectral_Theorem_ACM}
  For each $\omega\in\Omega$, let the real-symmetric projection matrices $\Gamma$
  and $\chi_1(\omega)$ of size $N\times N$ be defined as in Section
  \ref{sec:Finite_Lattice_Setting},  and let $N_1(\omega)={\rm
    Trace}(\chi_1(\omega))$ be the number of ones along the main diagonal of
  $\chi_1(\omega)$, with $N_0(\omega)=N-N_1(\omega)$. Moreover, let the matrix
  $M_1(\omega)=\chi_1(\omega)\Gamma\chi_1(\omega)$ and let $M_1(\omega)=U(\omega)\Lambda(\omega)U(\omega)$ be its spectral
  decomposition. Here, the columns of the matrix $U(\omega)$ consist of the
  orthonormal eigenvectors $\vec{u}_i(\omega)$, $i=1,\ldots,N$, of $M_1(\omega)$ and the
  diagonal matrix $\Lambda(\omega)={\rm diag}(\lambda_1(\omega),\ldots,\lambda_N(\omega))$ involves the
  eigenvalues $\lambda_i(\omega)\in[0,1]$ of $M_1(\omega)$. Then, there exists a
  permutation matrix $\Pi(\omega)$ of size $N\times N$, an orthogonal matrix
  $U_1(\omega)$ of size $N_1(\omega)\times N_1(\omega)$, and a diagonal matrix $\Lambda_1(\omega)$
  of size $N_1(\omega)\times N_1(\omega)$ such that     
  %
  \begin{align}\label{eq:Spec_Decomp_chi_Gamma_chi}
    %
U=\Pi^{\,T}\left[
  \begin{array}{ccc}
    I_0&0_{01}\\
    0_{10}&U_1   
    \end{array}
\right],
\qquad
\Lambda=\left[
  \begin{array}{ccc}
    O_{00}&O_{01}\\
    O_{10}&\Lambda_1   
    \end{array}
\right],
  \end{align}
  %
  where $I_0$ is the identity matrix of size $N_0(\omega)\times N_0(\omega)$ and $O_{ab}$
  is a matrix of zeros of size $N_a(\omega)\times N_b(\omega)$, for $a,b=0,1$. 
  Furthermore, let $R_i(\omega)$ be the projection matrices defined in
  equation \eqref{eq:Matrix_Rep_Spec_Theorem}. If the electric field
  $\vec{E}(\omega)$ satisfies $\vec{E}(\omega)=\vec{E}_0+\vec{E}_f(\omega)$, with
  $\vec{E}_0=\langle\vec{E}(\omega)\rangle$ and $\Gamma\vec{E}(\omega)=\vec{E}_f(\omega)$, then the
  effective complex conductivity tensor $\bsig^*$ has components 
  $\sigma_{jk}^*=\sigma_2\,m_{jk}(h)$, $j,k=1,\ldots,d$,  which satisfy     
%
\begin{align}\label{eq:Stieltjes_F_Discrete_appendix}
  &m_{jk}(h)=\delta_{jk}-F_{jk}(s), 
  &&F_{jk}(s)=\int_0^1\frac{\mu_{jk}(d\lambda)}{s-\lambda}\,, 
  &&\mu_{jk}(d\lambda)=\sum_{i=1}^N\langle \delta_{\lambda_i}(d\lambda)\chi_1R_i\hat{e}_j\cdot\hat{e}_k\rangle.  
\end{align}
%
Moreover, the mass $\mu_{jk}^0$ of the measure $\mu_{jk}$ satisfies
%
\begin{align}\label{eq:Measure_Mass_theorem}
  \mu_{jk}^0=\langle\chi_1\hat{e}_k\cdot\hat{e}_k\rangle\,\delta_{jk}=d\,\frac{\langle N_1^k(\omega)\rangle}{N}\,\delta_{jk},
\end{align}
%
where $N_1^k(\omega)={\rm Trace}(\chi_1^k(\omega))$ is the total number of type-one
bonds in the positive $k^{\text{th}}$ direction for $\omega\in\Omega$ and the
matrix $\chi_1^k(\omega)$ is defined in equation \eqref{eq:block_diag_chi}. 
% 
\end{theorem}

Before we prove Theorem \ref{thm:Discrete_Spectral_Theorem_ACM}, we
first discuss some consequences of equation
\eqref{eq:Measure_Mass_theorem}. In particular, we use this
formula to explicitly determine $\mu^0_{kk}$ for some important
examples of a large class of composite micro-geometries. Namely, the
class of geometries such that $N_1^k(\omega)$ is a non-random constant
$N_1^k$ for all $k=1,\ldots,d$, i.e. $N_1^k(\omega)=N_1^k$ for all
$\omega\in\Omega$. Consequently, $N_1(\omega)=N_1$ for all $\omega\in\Omega$ and
$N_1=\sum_kN_1^k$. Let $p_1^k=N_1^k/N$ be the number fraction of type-one
bonds in the positive $k^{\text{th}}$ direction, so that
$p_1=\sum_kp_1^k$. By equation \eqref{eq:Measure_Mass_theorem}, for this
class of composites we have
% 
\begin{align}\label{eq:Meas_mass_Anisotropic}
  \mu_{jk}^0=d\,p_1^k\,\delta_{jk}.
\end{align}
%
Given a fixed number fraction $p_1=N_1/N$ of type-one bonds, one can
define a class of highly anisotropic composite geometries by fixing
$p_1^k$ close to $p_1$ for some $k=1,\ldots,d$,
i.e. $p_1-p_1^k\ll1$. A class of locally isotropic random media is
obtained by requiring that every bond emanating from
$\vec{x}\in\mathbb{Z}^d_L$ in the positive direction is of the same
type, i.e. $\chi_1^j(\omega)=\chi_1^k(\omega)$ hence $N_1^j(\omega)=N_1^k(\omega)$ for all
$j,k=1,\ldots,d$ and $\omega\in\Omega$. In this case $N_1^k(\omega)=N_1/d$, thus
$p_1^k=p_1/d$ for all $k=1,\ldots,d$ and $\omega\in\Omega$. Consequently, equations
\eqref{eq:Measure_Mass_theorem} and \eqref{eq:Meas_mass_Anisotropic}
yield 
%
\begin{align}\label{eq:Meas_mass_Isotropic_iid}
  \mu_{jk}^0=p_1\,\delta_{jk}.
\end{align}
%
In this case, equations \eqref{eq:Measure_Mass_theorem} and
\eqref{eq:Meas_mass_Isotropic_iid} provide a direct
analogue of equation \eqref{eq:Mass_Sign_Measures} for the continuum
and infinite lattice settings (WE MUST RESOLVE WHETHER EQUATION
\eqref{eq:Meas_mass_Anisotropic} HOLDS IN SOME SENSE FOR THE INFINITE
LATTICE SETTING. IF SO, WE NEED A SEPARATE DISCUSSION FOR THE INFINTIE
LATTICE SETTING WHICH LEADS TO THE ANALOGUE OF EQUATION
\eqref{eq:Mass_Sign_Measures}). Equation
\eqref{eq:Meas_mass_Isotropic_iid} also holds when each of the $N$ 
bonds are chosen (independently) to be type-one with probability
$p_1=N_1/N$ and type-two with probability $1-p_1$. In this case the
$N_1^k(\omega)$, $k=1,\ldots,d$, are independent, identically distributed random 
variables with mean $\langle N_1^k(\omega)\rangle=p_1$ (IS THIS STATEMENT CORRECT?).

\indent
\textbf{Proof of Theorem \ref{thm:Discrete_Spectral_Theorem_ACM}.}
Taking $\vec{E}=\vec{E}_0+\vec{E}_f$ with the 
condition $\Gamma\vec{E}=\vec{E}_f$ as a definition greatly simplifies the
proof of Theorem \ref{thm:Discrete_Spectral_Theorem_ACM}, by avoiding
the formulation and proof of some technical lemmas regarding the
commutativity of the matrices $D_j$, $D_j^{\,T}$, and $(\Delta^{-1})$ for
$j=1,\ldots,d$. This is a natural assumption to make, since in equation 
\eqref{eq:Suff_Cond} we showed that the condition $\Gamma\vec{E}=\vec{E}_f$
is sufficient for the energy constraint $\langle\vec{J}\cdot\vec{E}_f\rangle=0$, which
is at the heart of the existence of solutions to equations
\eqref{eq:Maxwells_Equations_E} and \eqref{eq:Kirchhiff's__Equations}
in the (infinite) continuum and lattice settings,
respectively. In the finite lattice setting where $\Gamma$ and $\chi_1$ are
matrices, this condition leads to equation \eqref{eq:Proj_rep_Ef_Jf}
exactly as in Section \eqref{sec:Resolvent_Representations}, which is
equivalent to the formula $(sI-\Gamma\chi_1)\vec{E}=s\vec{E}_0$  and the  
resolvent representation of the electric field in
\eqref{eq:Resolvent_representations_E_D}. As discussed above in this
section, the matrix $\Gamma\chi_1$ is not symmetric and we therefore multiply
this formula by the matrix $\chi_1$, yielding the following equation
involving the real-symmetric random matrix $M_1=\chi_1\Gamma\chi_1$ 
%
\begin{align}\label{eq:Discrete_Resolvent}
  (s\chi_1-\chi_1\Gamma\chi_1)\vec{E}=s\chi_1\vec{E}_0.
\end{align}
%

Define the sets $\mathbb{N}_L^1(\omega)$ and $\mathbb{N}_L^0(\omega)$ by
%
\begin{align}\label{eq:Zero_One_indices}
  \mathbb{N}_L^1(\omega)=\{i\in\mathbb{N}_L \ | \ (\chi_1(\omega))_{ii}=1\}, \qquad
  \mathbb{N}_L^0(\omega)=\mathbb{N}_L\backslash \mathbb{N}_L^1(\omega).
\end{align}
%
Also, define elementary permutation matrices \cite{Demmel:1997} $\Pi_{i,j}$,
$i,j=1,\ldots,N$, such that $\Pi_{i,j}=\Pi_{i,j}^{\,-1}=\Pi_{i,j}^{\,T}$ and
$\Pi_{i,j}\vec{\xi}$ is the vector $\vec{\xi}$ with the $i^{\,\text{th}}$
and $j^{\,\text{th}}$ entries interchanged. Since $\chi_1(\omega)$ is a
diagonal matrix with $N_1(\omega)$ ones and $N_0(\omega)$ zeros along it's main
diagonal, it is clear that there exists a permutation matrix $\Pi(\omega)$
such that 
%
\begin{align}\label{eq:chi_Perm}
  \Pi(\omega)\chi_1(\omega)\Pi^{\,T}(\omega)=
  \left[
  \begin{array}{ccc}
    0_{00}&0_{01}\\
    0_{10}&I_1   
    \end{array}
\right],
\quad
\Pi(\omega)=\prod_{i,j\in\mathbb{N}_L}\Pi_{i,j}(\omega),
\end{align}
%
for each $\omega\in\Omega$, where $i\in\mathbb{N}_L^1(\omega)$, $j\in\mathbb{N}_L^0(\omega)$,
and $I_1$ is the identity matrix of size $N_1(\omega)\times N_1(\omega)$. Therefore,
as $\Pi^{\,T}=\Pi^{\,-1}$ we have
%
\begin{align}\label{eq:Spec_Decomp_chi_Gamma_chi_Proof}
  \chi_1\Gamma\chi_1&=
  \Pi^{\,T}
  \left[
  \begin{array}{ccc}
    0_{00}&0_{01}\\
    0_{10}&I_1   
    \end{array}
\right]
\Gamma_\Pi
\left[
  \begin{array}{ccc}
    0_{00}&0_{01}\\
    0_{10}&I_1   
    \end{array}
\right]
\Pi
=
\Pi^{\,T}
\left[
  \begin{array}{ccc}
    0_{00}&0_{01}\\
    0_{10}&\Gamma_1   
    \end{array}
\right]
\Pi
%\\
%&=
=
\Pi^{\,T}
\left[
  \begin{array}{ccc}
    0_{00}&0_{01}\\
    0_{10}&U_1\Lambda_1U_1^{\,T} 
    \end{array}
\right]
\Pi
\notag\\
&=
%=
\Pi^{\,T}
\left[
  \begin{array}{ccc}
    I_0&0_{01}\\
    0_{10}&U_1 
    \end{array}
\right]    
\left[
  \begin{array}{ccc}
    0_{00}&0_{01}\\
    0_{10}&\Lambda_1
    \end{array}
\right]    
\left[
  \begin{array}{ccc}
    I_0&0_{01}\\
    0_{10}&U_1^{\,T} 
    \end{array}
\right]    
\Pi,
%\notag
\end{align}
%
where $\Gamma_\Pi=\Pi\,\Gamma\,\Pi^{\,T}$, $\Gamma_1$ is the (real-symmetric) lower right
principal sub-matrix of $\Gamma_\Pi$ of size $N_1(\omega)\times N_1(\omega)$, and 
$\Gamma_1=U_1\Lambda_1U_1^{\,T}$ is its eigenvalue decomposition. As $\Gamma_1$ is a
real-symmetric matrix, $U_1$ is an orthogonal matrix. Moreover, since
$\Gamma_\Pi=\Pi\,\Gamma\,\Pi^{\,T}$ is a similarity transformation of a projection matrix
and $\Pi\chi_1\Pi^{\,T}$ is a projection matrix, $\Lambda_1$ is a diagonal matrix
with entries $\lambda_j^{\Pi_1}\in[0,1]$, $j=1,\ldots,N_1 (\omega)$, along the main diagonal
\cite{Demmel:1997}. Equation
\eqref{eq:Spec_Decomp_chi_Gamma_chi_Proof} and $\chi_1\Gamma\chi_1=U\Lambda U^{\,T}$ 
imply equation \eqref{eq:Spec_Decomp_chi_Gamma_chi} in the statement
of the theorem, where $U$ is an orthogonal matrix and $\Lambda$ is a
diagonal matrix with entries $\lambda_j\in[0,1]$, $j=1,\ldots,N$, along the main
diagonal.    


From $\Pi^{\T}=\Pi^{\,-1}$ and equations \eqref{eq:Discrete_Resolvent},
\eqref{eq:chi_Perm}, and \eqref{eq:Spec_Decomp_chi_Gamma_chi_Proof} we
have 
%
\begin{align}\label{eq:Discrete_Resolvent_Pi}
  \left[
  \begin{array}{ccc}
    0_{00}&0_{01}\\
    0_{10}&(sI_1-U_1\Lambda_1U_1^T)
    \end{array}
\right]
\Pi\vec{E}
=
\left[
  \begin{array}{ccc}
    0_{00}&0_{01}\\
    0_{10}&sI_1
    \end{array}
\right]
\Pi\vec{E}_0.
\end{align}
%
Define the coordinate system such that $\vec{E}_0=E_0\hat{e}_j$, for
some $j=1,\ldots,d$, and write 
%
\begin{align}\label{Pi_coordinates_E}
  \Pi\vec{E}=
  \left[
  \begin{array}{ccc}
    \vec{E}^{\,\Pi_0}\\
    \vec{E}^{\,\Pi_1}
    \end{array}
\right],
\quad
 \Pi\vec{E}_0=
  \left[
  \begin{array}{ccc}
    \vec{E}_0^{\,\Pi_0}\\
    \vec{E}_0^{\,\Pi_1}
    \end{array}
\right]
=
E_0
\left[
  \begin{array}{ccc}
    \hat{e}_j^{\,\Pi_0}\\
    \hat{e}_j^{\,\Pi_1}
    \end{array}
\right],
\quad
 \Pi\vec{u}_i=
  \left[
  \begin{array}{ccc}
    \vec{u}_i^{\,\Pi_0}\\
    \vec{u}_i^{\,\Pi_1}
    \end{array}
\right],
\end{align}
%
where $\vec{E}^{\,\Pi_0}\in\mathbb{R}^{N_0}$,
$\vec{E}^{\,\Pi_1}\in\mathbb{R}^{N_1}$, and similarly for the vectors
$\Pi\vec{E}_0$ and $\Pi\vec{u}_i$. From equation
\eqref{eq:Discrete_Resolvent_Pi} we have
$(sI_1-U_1\Lambda_1U_1^{\,T})\vec{E}^{\,\Pi_1}=s\vec{E}_0^{\,\Pi_1}$, which
yields the following resolvent representation for $\vec{E}^{\,\Pi_1}$
that is a direct analogue of equation
\eqref{eq:Resolvent_representations_E_D} 
%
\begin{align}\label{eq:Resolvent_representations_E_Pi}
  \vec{E}^{\,\Pi_1}=s(sI_1-U_1\Lambda_1U_1^{\,T})^{-1}\vec{E}_0^{\,\Pi_1}, \quad
  s\in\mathbb{C}\backslash[0,1].
\end{align}
%

Equation \eqref{eq:Resolvent_representations_E_Pi} leads to a
Stieltjes integral representation for $\bsig^*$ with components
$\sigma^*_{jk}=\bsig^*\hat{e}_j\cdot\hat{e}_k$ as follows. Recall the
definition of the effective complex conductivity tensor $\bsig^*$:
%$ \bsig^*\vec{E}_0=\langle\vec{J}\rangle=\sigma_2\langle(1-\chi_1/s)\vec{E}\rangle=\sigma_2(\vec{E}_0-\langle\chi_1\vec{E}\rangle/s)$.
%
\begin{align}\label{eq:Eff_Cond_Tens_Def}
  \bsig^*\vec{E}_0=\langle\vec{J}\rangle 
                =\sigma_2\langle(1-\chi_1/s)\vec{E}\rangle
                =\sigma_2(\vec{E}_0 -\langle\chi_1\vec{E}\rangle/s).
\end{align}
%
Since the symmetric matrix $\chi_1$ satisfies $\chi_1^2=\chi_1$ and
$\Pi^{\,T}=\Pi^{-1}$, equations \eqref{eq:chi_Perm} and
\eqref{Pi_coordinates_E} yield the following projection identity  
%
\begin{align}\label{eq:Projection_Identity}
 \chi_1\vec{E}\cdot\hat{e}_k=\vec{E}^{\,\Pi_1}\cdot\hat{e}_k^{\,\Pi_1}.
\end{align}
%
%$\langle\chi_1\vec{E}\cdot\hat{e}_k\rangle=\langle\vec{E}^{\,\Pi_1}\cdot\hat{e}_k^{\,\Pi_1}\rangle$.
Therefore, equations \eqref{eq:Discrete_Stieltjes_F_Derivation},
\eqref{eq:Zero_One_indices}, and
\eqref{Pi_coordinates_E}--\eqref{eq:Projection_Identity} 
% \eqref{Pi_coordinates_E},
% \eqref{eq:Resolvent_representations_E_Pi},
% \eqref{eq:Eff_Cond_Tens_Def}, and \eqref{eq:Projection_Identity}
imply that  
%
\begin{align}\label{eq:Disc_Stieltjes_Rep_Sigma}
  \delta_{ij}-\sigma^*_{jk}/\sigma_2&=\langle\chi_1\vec{E}\cdot\hat{e}_k\rangle/(sE_0)
          =\langle\vec{E}^{\,\Pi_1}\cdot\hat{e}_k^{\,\Pi_1}\rangle/(sE_0)
          \notag\\
          &=\langle(sI_1-U_1\Lambda_1U_1^{\,T})^{-1}\hat{e}_j^{\,\Pi_1}\cdot\hat{e}_k^{\,\Pi_1}\rangle          
          =\sum_{i\in\mathbb{N}_L^1}\left\langle
          \frac{R_i^{\,\Pi_1}\hat{e}_j^{\,\Pi_1}}{s-\lambda_i^{\Pi_1}}\cdot\hat{e}_k^{\,\Pi_1}
          \right\rangle,  
\end{align}
%
where $R_i^{\,\Pi_1}$, $i\in N_L^1(\omega)$, is the projection matrix of size
$N_1(\omega)\times N_1(\omega)$ associated with the columns $\vec{u}_i^{\,\Pi_1}$ of
the orthogonal matrix $U_1$. 



We now show that equation \eqref{eq:Disc_Stieltjes_Rep_Sigma} is
equivalent to equation \eqref{eq:Stieltjes_F_Discrete_appendix}. Since
$\Pi^{\,T}=\Pi^{-1}$, equations \eqref{eq:Spec_Decomp_chi_Gamma_chi} and 
\eqref{eq:chi_Perm} imply that
%
\begin{align}\label{eq:Projection_Eigenspace}
  \chi_1U=\Pi^{\,T}\left[
  \begin{array}{ccc}
    0_{00}&0_{01}\\
    0_{10}&U_1  
    \end{array}
\right].
\end{align}
%
Recalling the definitions of $\vec{u}_i$ and $R_i$ from the
statement of the theorem, equation \eqref{eq:Projection_Eigenspace}
implies that 
%
\begin{align}\label{eq:chi_ui_chi_Ri}
  \chi_1\vec{u}_i=
  \begin{cases}
  \vec{u}_i, &\text{ for } i\in\mathbb{N}_L^1  \\
  0,        &\text{ for } i\in\mathbb{N}_L^0
  \end{cases}
  \quad \Rightarrow \quad
   \chi_1R_i=
  \begin{cases}
  R_i, &\text{ for } i\in\mathbb{N}_L^1  \\
  0,  &\text{ for } i\in\mathbb{N}_L^0
  \end{cases}.
\end{align}
%
Similar to equation \eqref{eq:Projection_Identity} we have the
projection identity
$\vec{u}_i^{\,\Pi_1}\cdot\hat{e}_j^{\,\Pi_1}=\chi_1\vec{u}_i\cdot\hat{e}_j$. This and
equation \eqref{eq:chi_ui_chi_Ri} yield another important projection
identity 
%
\begin{align*}
  R_i^{\,\Pi_1}\hat{e}_j^{\,\Pi_1}\cdot\hat{e}_k^{\,\Pi_1}
  =
   (\vec{u}_i^{\,\Pi_1}\cdot\hat{e}_j^{\,\Pi_1})(\vec{u}_i^{\,\Pi_1}\cdot\hat{e}_k^{\,\Pi_1}) 
  =(\chi_1\vec{u}_i\cdot\hat{e}_j)(\chi_1\vec{u}_i\cdot\hat{e}_k)
  =\chi_1R_i\hat{e}_j\cdot\hat{e}_k.
\end{align*}
%
Therefore, equation \eqref{eq:chi_ui_chi_Ri} implies that
%
\begin{align*}
 \sum_{i\in\mathbb{N}_L^1}\left\langle
          \frac{R_i^{\,\Pi_1}\hat{e}_j^{\,\Pi_1}}{s-\lambda_i^{\Pi_1}}\cdot\hat{e}_k^{\,\Pi_1}
          \right\rangle 
 =
 \sum_{i\in\mathbb{N}_L^1}\left\langle
          \frac{\chi_1R_i\hat{e}_j}{s-\lambda_i^{\Pi_1}}\cdot\hat{e}_k
          \right\rangle 
          =
 \sum_{i=1}^N\left\langle
          \frac{\chi_1R_i\hat{e}_j}{s-\lambda_i}\cdot\hat{e}_k
          \right\rangle ,
\end{align*}
%
which proves our claim that equation \eqref{eq:Disc_Stieltjes_Rep_Sigma} is
equivalent to equation \eqref{eq:Stieltjes_F_Discrete_appendix}. Note
that the equivalence of equations \eqref{eq:Disc_Stieltjes_Rep_Sigma}
and \eqref{eq:Stieltjes_F_Discrete_appendix} implies that the spectral 
measure $\mu_{jk}$ in \eqref{eq:Stieltjes_F_Discrete_appendix} satisfies 
%
\begin{align}\label{eq:measure_equivalence}
  \mu_{jk}(d\lambda)=\sum_{i=1}^N \langle\delta_{\lambda_i}(d\lambda)\chi_1R_i\hat{e}_j\cdot\hat{e}_k\rangle
         \equiv\sum_{i\in\mathbb{N}_L}
          \langle\delta_{\lambda_i^{\Pi_1}}(d\lambda)R_i^{\,\Pi_1}\hat{e}_j^{\,\Pi_1}\cdot\hat{e}_k^{\,\Pi_1}\rangle.
\end{align}
%

We now discuss the analogue of equation \eqref{eq:Mass_Sign_Measures}
for the finite lattice setting and prove equation
\eqref{eq:Measure_Mass_theorem} in the statement of the
theorem. Recall that the projection matrices $R_i^{\,\Pi_1}$,
$i\in\mathbb{N}_L$, are symmetric. Therefore, by equations
\eqref{eq:Matrix_Rep_Spec_Theorem} and \eqref{eq:measure_equivalence}
we have   
%
\begin{align}\label{eq:Measure_Mass_Lattice}
  &\mu^0_{jk}=\int_0^1\mu_{jk}(d\lambda)
       =\int_0^1\sum_{i=1}^N \langle\delta_{\lambda_i}(d\lambda)\chi_1R_i\hat{e}_j\cdot\hat{e}_k\rangle
       =\langle\chi_1\hat{e}_j\cdot\hat{e}_k\rangle
       =\langle\chi_1\hat{e}_k\cdot\hat{e}_k\rangle\,\delta_{jk}, \notag\\
 &\mu_{kk}(d\lambda)=\sum_{i\in\mathbb{N}_L}
             \langle\delta_{\lambda_i^{\Pi_1}}(d\lambda)R_i^{\,\Pi_1}\hat{e}_k^{\,\Pi_1}\cdot\hat{e}_k^{\,\Pi_1}\rangle
          %=\sum_{i\in\mathbb{N}_L}
          %   \langle\delta_{\lambda_i^{\Pi_1}}(d\lambda)R_i^{\,\Pi_1}\hat{e}_k^{\,\Pi_1}\cdot R_i^{\,\Pi_1}\hat{e}_k^{\,\Pi_1}\rangle
         =\sum_{i\in\mathbb{N}_L}
             \langle\delta_{\lambda_i^{\Pi_1}}(d\lambda)|R_i^{\,\Pi_1}\hat{e}_k^{\,\Pi_1}|^2\rangle\geq0,            
\end{align}
%
where $|\cdot|$ denotes the $l^2$ norm on $\mathbb{R}^{N_1(\omega)}$.
Therefore, as in the continuum and infinite lattice settings, the
diagonal components $\mu_{kk}$ of the matrix valued measure $\bmu$ are
positive measures of mass $\langle\chi_1\hat{e}_k\cdot\hat{e}_k\rangle$ while the
off-diagonal components $\mu_{jk}$, for $j\neq k$, have zero mass and are
consequently signed measures. Using equation
\eqref{eq:block_diag_chi} we may write $\mu^0_{jk}$ in
\eqref{eq:Measure_Mass_Lattice} in a more suggestive form. Recall that  
$\hat{e}_1=(\vec{1},\vec{0},\ldots,\vec{0})/L^{d/2}$, where $\vec{1}$ and
$\vec{0}$ are vectors of ones and zeros of length $L^d$, respectively,
and similarly for the $\vec{e}_j$ for $j=2,\ldots,d$ (see Section
\ref{sec:Finite_Lattice_Setting} for details). Since $\chi_1$ is a symmetric
projection matrix, equations \eqref{eq:block_diag_chi} and
\eqref{eq:Measure_Mass_Lattice} imply that
%
\begin{align}\label{eq:Measure_Mass_Lattice_Trace}
  \mu^0_{jk}=\langle\chi_1\hat{e}_k\cdot\hat{e}_k\rangle\,\delta_{jk}
       =\langle\chi_1\hat{e}_k\cdot\chi_1\hat{e}_k\rangle\,\delta_{jk}
       =\frac{1}{L^d}\langle\chi_1^k\vec{1}\cdot \chi_1^k\vec{1}\rangle\,\delta_{jk}
       =\frac{1}{L^d}\langle\text{Trace}(\chi_1^k)\rangle\,\delta_{jk}
       =d\,\frac{\langle N_1^k(\omega)\rangle}{N}\,\delta_{jk},       
\end{align}
%
where $N_1^k(\omega)={\rm Trace}(\chi_1^k(\omega))$ is the total number of type-one
bonds in the positive $k^{\text{th}}$ direction for $\omega\in\Omega$ and
$N=d\,L^d$. This proves equation \eqref{eq:Measure_Mass_theorem} and
concludes our proof of Theorem \ref{thm:Discrete_Spectral_Theorem_ACM}
$\Box\,$.       



\subsection{Bounding Procedure}\label{sec:Bounding_Procedure}
%
A key feature of the integral representation for $F_{jk}(s)$,
$j,k=1,\ldots,d$, displayed in equations \eqref{eq:Stieltjes_F} and
\eqref{eq:Stieltjes_F_Discrete}, is that parameter information in $s$
and $E_0$ is \emph{separated} from the geometry of the composite,
which is encoded in the spectral measure $\mu_{jk}$ via its moments  
$\mu^n_{jk}$, $n=0,1,2,\ldots$ \cite{Bruno:JSP-365,Golden:CMP-473}. Another
important feature of the representation for $F_{jk}(s)$ is that it
is a \emph{linear} functional of the measure $\mu_{jk}$. Moreover, the
diagonal components $\mu_{kk}$ are \emph{positive} measures. These
important properties are also shared by the function $E_{jk}(s)$ which
was introduced in Section \ref{sec:Resolvent_Representations}. It also
has a Stieltjes integral representation which involves a spectral measure
$\eta_{jk}$ associated with the self-adjoint random operator $\chi_2\Gamma_\times\chi_2$
on the Hilbert space $\mathscr{H}_{\bullet}$ (see Sections
\ref{sec:Resolvent_Representations} and \ref{sec:The_Spectral_Theorem}
for details)   
% 
\begin{align}\label{eq:Stieltjes_Es}
  E_{jk}(s)=\langle\chi_2(s-\Gamma_\times\chi_2)^{-1}\vec{e}_j\cdot\vec{e}_k\rangle=\int_0^1\frac{\eta_{jk}(d\lambda)}{s-\lambda}\,,\quad
  s\in\mathbb{C}\backslash[0,1].
\end{align}
%
These important properties of the functions $F_{jk}(s)$ and
$E_{jk}(s)$ may be exploited to obtain rigorous bounds for the
diagonal components of the effective parameters. 



In this section we review a bounding procedure which was 
introduced in \cite{Golden:CMP-473}. The bounds incorporate
the moments $\mu_{kk}^n$ and $\eta_{kk}^n$ of the measures $\mu_{kk}$ and
$\eta_{kk}$.  In the infinite continuum and lattice settings, by equation
\eqref{eq:Mass_Sign_Measures} and the symmetries between $F_{kk}(s)$
and $E_{kk}(s)$, the masses of these measures are generically given by
$\mu_{kk}^0=p_1$ and $\eta_{kk}^0=p_2$. However, in the finite lattice
setting, these symmetries and equation
\eqref{eq:Measure_Mass_Lattice_Trace} yield $\mu_{kk}^0=d\,\langle N_1^k(\omega)\rangle/N$
and $\eta_{kk}^0=d\,\langle N_2^k(\omega)\rangle/N$. Here $N_1^k(\omega)={\rm Trace}(\chi_1^k(\omega))$
is the total number of type-one bonds in the positive $k^{\text{th}}$ 
direction for $\omega\in\Omega$, and similarly for $N_2^k(\omega)$. This is a
fundamental difference between the infinite and finite lattice
settings. We discuss the bounding procedure for the infinite continuum
and lattice settings in Section \ref{sec:Bounds_infinite_setting} and
that for the finite lattice setting in Section
\ref{sec:Bounds_finite_lattice_setting}.   


\subsubsection{Continuum and infinite lattice settings}
\label{sec:Bounds_infinite_setting}
%
In this section, we will discuss the bounding procedure in terms of
the diagonal components $\sigma^*_{kk}$, $k=1,\ldots,d$, of the effective
complex conductivity tensor $\bsig^*$. For
simplicity, we will focus on one such component and set
$\sigma^*=\sigma_{kk}^*$, $F(s)=F_{kk}(s)$, $m(h)=m_{kk}(h)$, $\mu=\mu_{kk}$,
$E(s)=E_{kk}(s)$, $\tilde{m}(h)=\tilde{m}_{kk}(h)$, and
$\eta=\eta_{kk}$. Here $\sigma^*=\sigma_2m(h)=\sigma_1/\tilde{m}(h)$, $F(s)=1-m(h)$, and
$E(s)=1-\tilde{m}(h)$. We will also exploit the symmetries between
equations \eqref{eq:Stieltjes_F} and \eqref{eq:Stieltjes_Es}, and
initially focus on the function $F(s)$ and the measure $\mu$,
introducing the function $E(s)$ and the measure $\eta$ when
appropriate.




Bounds on $\sigma^*$ are obtained as follows. By equation
\eqref{eq:Mass_Sign_Measures} the support of the measure $\mu$ is
contained in the interval $[0,1]$ and its mass is given by $\mu^0=p_1$,
where $0\leq p_1\leq1$. Consider the set $\mathscr{M}$ of positive Borel
measures on $[0,1]$ with mass $\leq1$. By equation
\eqref{eq:Stieltjes_F}, for fixed $s\in\mathbb{C}\backslash[0,1]$, $F(s)$ is a
linear functional of the measure $\mu$, $F:\mathscr{M}\mapsto\mathbb{C}$, and
we write $F(s)=F(s,\mu)$ and $m(h)=m(h,\mu)$. Suppose that we know the
moments $\mu^n$ of the measure $\mu$ for $n=1,\ldots,J$ and let
% 
\begin{align}\label{eq:Measure_Set}
  \mathscr{M}_J^\mu
     =\left\{\nu\in\mathscr{M} \ \Big| \   \int_0^1\lambda^n\nu(d\lambda)=\mu^n, \  n=0,\ldots,J\right\}  . 
\end{align}
%
The set $A_J^\mu\subset\mathbb{C}$ which represents the possible
values of the function $m(h,\mu)=1-F(s,\mu)$ that is compatible with the
known information about the random medium is given by
%
\begin{align}\label{eq:Bounding_Set}
  A_J^\mu
     =\left\{\ m(h,\mu)\in\mathbb{C} \ | \
       \ h\not\in(-\infty,0], \ \mu\in \mathscr{M}_J^\mu\right\}. 
\end{align}
%



The set of measures $\mathscr{M}_J^\mu$ is a compact, convex
subset of $\mathscr{M}$ with the topology of weak convergence
\cite{Golden:CMP-473}. Since the mapping $F(s,\mu)$ in
\eqref{eq:Stieltjes_F} is linear in $\mu$ it follows that
$A_J^\mu$ is a compact convex subset of the complex plane
$\mathbb{C}$. The extreme points of $\mathscr{M}_0^\mu$ are the one 
point measures $\alpha\delta_a$, $0\leq\alpha,a\leq1$ \cite{Dunford_Schwartz:LinOp_PtI},
while the extreme points of $\mathscr{M}_J^\mu$ for $J>0$ are weak limits
of convex combinations of measures of the form \cite{Golden:CMP-473} 
%
\begin{align}\label{eq:Discrete_Measure}
  \mu_J(d\lambda)=\sum_{i=1}^{J+1}\alpha_i\delta_{\lambda_i}(d\lambda), \quad
  \alpha_i\geq0, \quad 0\leq\lambda_1<\cdots<\lambda_{J+1}<1, \quad
  \sum_{i=1}^{J+1}\alpha_i\lambda_i^n=\mu^n,
%  \quad   n=0,1,2,\ldots.
\end{align}
%
for $n=0,1,\ldots,J$. It is important to note that, except for the case of 
multi-rank laminates \cite{MILTON:2002:TC}, (ELENA, IS THIS WORDING AND
REFERENCE CORRECT?) not every measure $\mu\in\mathscr{M}_J^\mu$ gives rise
to a function $m(h,\mu)$ that is the effective (relative) conductivity a
random medium \cite{Golden:CMP-473}. Therefore, in 
general, the set $A_J^\mu$ will \emph{contain} the exact range of values
of the effective conductivity \cite{Golden:CMP-473}. This is
sufficient for the bounding procedure discussed in this section.






By the symmetries between equations \eqref{eq:Stieltjes_F} and
\eqref{eq:Stieltjes_Es}, and equation \eqref{eq:Mass_Sign_Measures},
the support of the measure $\eta$ is contained in the interval $[0,1]$
and its mass is given by $\eta^0=p_2=1-p_1$, where $0\leq p_2\leq1$. We can
therefore define compact, convex sets $\mathscr{M}_J^\eta\subset\mathscr{M}$
and $A_J^\eta\subset\mathbb{C}$ which are analogous to those defined in
equations \eqref{eq:Measure_Set} and \eqref{eq:Bounding_Set},
respectively, involving the functions
$\tilde{m}(h,\eta)=1-E(s,\eta)=\sigma_1/\sigma^*$. Moreover, the extreme points of
$\mathscr{M}_0^\eta$ are the one point measures $\alpha\delta_a$, $0\leq\alpha,a\leq1$ 
while the extreme points of $\mathscr{M}_J^\eta$ are weak limits
of convex combinations of measures of the form given in equation
\eqref{eq:Discrete_Measure}. 



Consequently, in order to determine the extreme
points of the sets $A_J^\mu$ and $A_J^\eta$ it suffices to determine the
range of values in $\mathbb{C}$ of the functions $m(h,\mu_J)=1-F(s,\mu_J)$
and $\tilde{m}(h,\eta_J)=1-E(s,\eta_J)$, respectively, where  
%
\begin{align}\label{eq:Discrete_mh}
  F(s,\mu_J)=\sum_{i=1}^{J+1}\frac{\alpha_i}{s-\lambda_i}\,, \qquad
  E(s,\eta_J)=\sum_{i=1}^{J+1}\frac{\alpha_i}{s-\lambda_i}\,,
\end{align}
as the $\alpha_i$ and $\lambda_i$ vary under the constraints given in equation
\eqref{eq:Discrete_Measure}. While $F(s,\mu_J)$ and $E(s,\eta_J)$ in
\eqref{eq:Discrete_mh} may not run over all points in $A_J^\mu$ and
$A_J^\eta$ as the $\alpha_i$ and $\lambda_i$ vary, they run over the
extreme points of these sets, which is sufficient due to their
convexity. It is important to note that, as the effective complex
conductivity $\sigma^*$ is given by $\sigma^*=\sigma_2m(h,\mu)=\sigma_1/\tilde{m}(h,\eta)$, the
regions $A_J^\mu$ and $A_J^\eta$ have to be mapped to the common
$\sigma^*$-plane to provide bounds for $\sigma^*$.    





In this section we discuss two different bounds for $\sigma^*$. The first
bound assumes that only the masses $\mu^0=p_1$ and $\eta^0=p_2$ of the
measures $\mu$ and $\eta$ are known. While the second bound also assumes
that the random medium is statistically isotropic, so that the first
moments of these measures are also known, and are given by
\cite{Golden:1986:BCP}   
%
\begin{align}\label{eq:First_Moments}
  \mu^1=\frac{p_1p_2}{d}\,, \qquad
  \eta^1=\frac{p_1p_2(d-1)}{d}\,.
\end{align}
%


Consider the first case, where $J=0$  in \eqref{eq:Discrete_mh} and the
volume fraction $p_1=1-p_2$ is fixed with $\mu^0=p_1$ and
$\eta^0=p_2=1-p_1$, so that $F(s,\mu_J)=p_1/(s-\lambda)$ and
$E(s,\eta_J)=p_2/(s-\tilde{\lambda})$. By the above discussion, the values of 
$F(s,\mu$) and $E(s,\eta)$ lie inside the circles $C_0(\lambda)$ and
$\tilde{C}_0(\tilde{\lambda})$, respectively, given by  
%
\begin{align}\label{eq:0th_order_Bounds}
    C_0(\lambda)=\frac{\mu^0}{s-\lambda}\,, \quad -\infty\leq\lambda\leq \infty, \qquad
    \tilde{C}_0(\tilde{\lambda})=\frac{\eta^0}{s-\tilde{\lambda}}\,, \quad
    -\infty\leq\tilde{\lambda}\leq \infty. 
\end{align}
%
In the $\sigma^*$-plane, the intersection of these two regions is bounded by
two circular arcs corresponding to $0\leq\lambda\leq p_2$ and $0\leq\tilde{\lambda}\leq p_1$
in \eqref{eq:0th_order_Bounds}, and the values of $\sigma^*$ lie inside
this region \cite{Golden:1986:BCP}. These bounds are optimal
\cite{Milton:JAP-5286,Bergman:AP-78}, and are obtained by a composite
of uniformly aligned spheroids of material 1 in all sizes coated with
confocal shells of material 2, and vice versa. The arcs are traced out
as the aspect ratio varies. When the value of the component
permittivities $\sigma_1$ and $\sigma_2$ are real and positive, the bounding
region collapses to the interval
$1/(p_1/\sigma_1+p_2/\sigma_2)\leq\sigma^*\leq p_1\sigma_1+p_2\sigma_2$, which are the Wiener
bounds. The lower and upper bounds are obtained by parallel slabs of
the two materials aligned perpendicular and parallel to the field
$\vec{E}_0$, respectively \cite{Scaife-1989}.



Now consider the second case, where $J=1$ in \eqref{eq:Discrete_mh},
the volume fraction $p_1=1-p_2$ is fixed, and the random medium is
statistically isotropic so that the first moments $\mu^1$ and $\eta^1$ of
the measures $\mu$ and $\eta$ are given, respectively, by that in equation  
\eqref{eq:First_Moments}.  A convenient way of including this
information is to use the transformations \cite{Bergman:AP-78}
%
\begin{align}\label{eq:Aux_Fs_Es}
  F_1(s)=\frac{1}{p_1}-\frac{1}{sF(s)}\,, \qquad
  E_1(s)=\frac{1}{p_2}-\frac{1}{sE(s)}\,.
\end{align}
%
Due to the symmetries between $F_1(s)$ and $E_1(s)$ in
\eqref{eq:Aux_Fs_Es} we will first focus on the function $F_1(s)$ and
introduce the function $E_1(s)$ when appropriate. The function
$F_1(s)$ is an upper half plane function analytic off $[0,1]$ and
therefore has an integral representation
\cite{Bergman:AP-78,Golden:1986:BCP} analogous to that in equations
\eqref{eq:Stieltjes_F} and \eqref{eq:Stieltjes_Es}, involving a
measure $\mu_1$, say, which is supported in the interval $[0,1]$. Since
only the mass $\mu^0=p_1$ and the first moment $\mu^1=p_1p_2/d$ of the
measure $\mu$ are known, the transformation \eqref{eq:Aux_Fs_Es}
determines only the mass $\mu_1^0=p_2/(p_1d)$ of the measure $\mu_1$
\cite{Bergman:AP-78,Golden:1986:BCP}. This reveals the utility of the
transformation $F_1(s)$ in \eqref{eq:Aux_Fs_Es}, it reduces the second
case $(J=1)$ for $F(s)$ to the first case $(J=0)$ for $F_1(s)$.



By our previous analysis, the values of $F_1(s)$ lie inside a circle
$p_2/(p_1d(s-\lambda))$, $-\infty\leq\lambda\leq\infty$. Similarly, the values of $E_1(s)$ lie
inside a circle $p_1(d-1)/(p_2d(s-\tilde{\lambda}))$,
$-\infty\leq\tilde{\lambda}\leq\infty$. Since $F$ and $E$ are fractional linear in $F_1$ and
$E_1$, respectively, these circles are transformed to the circles
$C_1(\lambda)$ in the $F$-plane and $\tilde{C}_1(\tilde{\lambda})$ in the
$E$-plane given by \cite{Golden:1986:BCP}
%
\begin{align}\label{eq:Isotropic_Bounds}
  C_1(\lambda)=\frac{p_1(s-\lambda)}{s(s-\lambda-p_2/d)}\,, \quad  %&-\infty\leq\lambda\leq\infty, \quad
  \tilde{C}_1(\tilde{\lambda})=\frac{p_2(s-\tilde{\lambda})}{s(s-\tilde{\lambda}-p_1(d-1)/d)}\,,
   \qquad -\infty\leq\lambda,\tilde{\lambda}\leq\infty. %\notag
\end{align}
%
In the $\sigma^*$-plane the intersection of these two circular regions is
bounded by two circular arcs \cite{Golden:1986:BCP} corresponding to
$0\leq\lambda\leq(d-1)/d$ and $0\leq\tilde{\lambda}\leq1/d$ in \eqref{eq:Isotropic_Bounds}.




The vertices of the region,
$C_1(0)=p_1/(s-p_2/d)$ and $\tilde{C}(0)=p_2/(s-p_1(d-1)/d)$, are
attained by the Hashin--Shtrikman geometries (spheres of all
sizes of material 1 in the volume fraction $p_1$ coated with spherical
shells of material 2 in the volume fraction $p_2$ filling all of
$\mathbb{R}^d$, and vice versa), and lie on the arcs of the first
order bounds \cite{Golden:1986:BCP}. While there are at least five
points on the arc $C_1(\lambda)$ in \eqref{eq:Isotropic_Bounds} that are
attainable by composite microstructures \cite{Milton:JAP-5286}, the
arc $\tilde{C}_1(\tilde{\lambda})$ in \eqref{eq:Isotropic_Bounds} violates
\cite{Golden:1986:BCP} the interchange inequality $m(h)m(1/h)\geq1$
\cite{Keller:1964:TCC,Schulgasser:1976:CFR}, which becomes an equality
in two dimensions \cite{MILTON:2002:TC}.  Consequently the isotropic
bounds in \eqref{eq:Isotropic_Bounds} are not optimal, but have been
improved \cite{Milton:APL-300,Bergman:AP-78} by incorporating the
interchange inequality. When $\sigma_1$ and $\sigma_2$ are real and positive
with $\sigma_1\leq\sigma_2$, the region collapses to the interval
%
\begin{align*}
  \sigma_1+p_2\left/  \left(\frac{1}{\sigma_2-\sigma_1}+\frac{p_1}{d\sigma_1}\right)\right.
  \leq\sigma^*\leq 
  \sigma_1+p_1\left/  \left(\frac{1}{\sigma_1-\sigma_2}+\frac{p_2}{d\sigma_2}\right)\right.,
\end{align*}
%
which are the Hashin--Shtrikman bounds.


The higher moments $\mu^n$, $n\geq2$ depend on $(n+1)$-point correlation
functions \cite{Golden:CMP-473} and cannot be calculated in general,
although the interchange inequality forces relations among them
\cite{Milton:JAP-5294}. If the moments $\mu^0,\ldots,\mu^J$ are known then the
transformation $F_1$ in \eqref{eq:Aux_Fs_Es} can be iterated to
produce an upper half plane function $F_J$ with a integral
representation, involving a positive measure $\mu_J$ which is supported
on the interval $[0,1]$. As in the case where $J=1$, the first $J$
moments of the measure $\mu$ determine only the mass $\mu_J^0$ of the
measure $\mu_J$ \cite{Golden:1986:BCP}, and the function $F_J(s)$ can
easily be extremized by the above procedure, and similarly for a
function $E_J(s)$ associated with the moments $\eta^0,\ldots,\eta^J$. The
resulting bounds form a nested sequence of lens-shaped regions
\cite{Golden:1986:BCP}.




\subsubsection{Finite lattice setting}
\label{sec:Bounds_finite_lattice_setting}
%
In this section we discuss the bounding procedure of the ACM for the
finite lattice setting. For explicitness, we again discuss the
procedure in terms of the components $\sigma^*_{kk}$, $k=1,\ldots,d$, of the
effective complex conductivity tensor $\bsig^*$. Recall that
$N_1^k(\omega)={\rm Trace}(\chi_1^k(\omega))$ denotes the total number of type-one
bonds in the positive $k^{\text{th}}$ direction for $\omega\in\Omega$, and
similarly for $N_2^k(\omega)$. A key difference in the ACM between the
continuum and finite lattice settings is that the masses $\mu_{kk}^0$
and $\eta_{kk}^0$ of the measures $\mu_{kk}$ and $\eta_{kk}$ are given by
$\mu_{kk}^0=p_1$ and $\eta_{kk}^0=p_2$ in the continuum case, while
$\mu_{kk}^0=d\,\langle N_1^k(\omega)\rangle/N$ and $\eta_{kk}^0=d\,\langle N_2^k(\omega)\rangle/N$ in the
finite lattice case (WE NEED TO RESOLVE THIS FOR THE INFINITE LATTICE
SETTING). The measure masses of both cases are the same for locally
isotropic media, where $N_1^k(\omega)=N_1/d$ for all $k=1,\ldots,d$ and $\omega\in\Omega$,
and for statistically isotropic media, where $\langle N_1^k(\omega)\rangle=N_1/d$, with
$p_1=N_1/N$ denoting the number fraction of type-one bonds (see the 
paragraph following the statement of Theorem
\ref{thm:Discrete_Spectral_Theorem_ACM} for more details). 





While the bounding \emph{procedure} is the same for both cases, this
fundamental difference between the continuum and finite lattice
settings effects the bounds given in Section
\eqref{sec:Bounds_infinite_setting}. The bounds given in equation
\eqref{eq:0th_order_Bounds} depend only on the masses $\mu_{kk}^0$ and
$\eta_{kk}^0$ of the measures $\mu_{kk}$ and $\eta_{kk}$. They are related as
follows. Note that $\chi_1^k(\omega)+\chi_2^k(\omega)=I_{L^d}$, where $I_{L^d}$ is the
identity matrix of size $L^d\times L^d$. By the linearity of the trace
operation, we therefore have
$\text{Trace}(\chi_1^k(\omega))+\text{Trace}(\chi_2^k(\omega))=\text{Trace}(I_{L^d})$   
or equivalently $N_1^k(\omega)+N_2^k(\omega)=L^d=N/d$. Averaging this formula
over $\Omega$ and rearranging yields
%$\mu_{kk}^0+\eta_{kk}^0=1$, for each $k=1,\ldots,d$.
%
\begin{align*}
  \mu_{kk}^0+\eta_{kk}^0=1, \quad  k=1,\ldots,d.
\end{align*}
%
By the discussion in Section \ref{sec:Bounds_infinite_setting}, the
values of $F_{kk}(s)$ and $E_{kk}(s)$ lie inside the circles
$C^k_0(\lambda)$ and $\tilde{C}^k_0(\tilde{\lambda})$, respectively, given by
equation \eqref{eq:0th_order_Bounds}  
%
\begin{align}\label{eq:0th_order_Bounds_finite_lattice}
    C_0^k(\lambda)=\frac{\mu_{kk}^0}{s-\lambda}\,, \quad -\infty\leq\lambda\leq \infty, \qquad
    \tilde{C}^k_0(\tilde{\lambda})=\frac{\eta_{kk}^0}{s-\tilde{\lambda}}\,, \quad
    -\infty\leq\tilde{\lambda}\leq \infty. 
\end{align}
%  
In the $\sigma^*_{kk}$-plane, the intersection of these two regions is bounded by
two circular arcs corresponding to $0\leq\lambda\leq\eta_{kk}^0$ and $0\leq\tilde{\lambda}\leq\mu_{kk}^0$
in \eqref{eq:0th_order_Bounds_finite_lattice}, and the values of
$\sigma^*_{kk}$ lie inside this region \cite{Golden:1986:BCP}. The
isotropic bounds given in equation \eqref{eq:Isotropic_Bounds} do not
hold for anisotropic media, where $\langle N_1^j(\omega)\rangle\neq\langle N_1^k(\omega)\rangle$ for some
$j\neq k$. However, in Section \ref{sec:Numerical_Results} we show that
the isotropic bounds in \eqref{eq:Isotropic_Bounds} capture the data
for locally isotropic and statistically isotropic random media in the
finite lattice setting. 




\medskip

{\bf Acknowledgement.}
\medskip

\bibliographystyle{siam}
\bibliography{murphy}
\end{document}
