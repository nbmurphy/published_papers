\documentclass[english,12pt,jmp,graphicx]{revtex4-1}
%\documentclass[aip,reprint]{revtex4-1}

\usepackage{graphicx}% Include figure files
\usepackage{dcolumn}% Align table columns on decimal point
\usepackage{bm}% bold math
\usepackage{amsmath,amssymb,latexsym,mathrsfs}

\draft % marks overfull lines with a black rule on the right

\newtheorem{lemma}{Lemma}[section]
\newtheorem{theorem}{Theorem}[section]
\newtheorem{definition}{Definition}[section]
\newtheorem{remark}{Remark}[section]


\newcommand{\ph}{\hat{\phi}}
\newcommand{\gh}{\hat{\gamma}}
\newcommand{\Dh}{\hat{\Delta}}
\newcommand{\dha}{\hat{\delta}}
\newcommand{\qh}{\hat{q}}
\newcommand{\xh}{\hat{x}}
\newcommand{\HM}{\mathcal{H}_{\text{max}}}
\newcommand{\Hm}{\mathcal{H}_{\text{min}}}
\newcommand{\sech}{\rm \hspace{0.7mm}sech}
\newcommand{\I}{\mathrm{i}}
\newcommand{\hh}{\hat{h}}
\newcommand{\mh}{m_r}
\newcommand{\mt}{m_i}
% \newcommand\beps{\mbox{\boldmath${\epsilon}$}}
% \newcommand\bmu{\mbox{\boldmath${\mu}$}}
% \newcommand\bsig{\mbox{\boldmath${\sigma}$}}

\begin{document}

% Use the \preprint command to place your local institutional report number 
% on the title page in preprint mode.
% Multiple \preprint commands are allowed.
%\preprint{}

\title{A Unification of the Critical Theory of Transport in Binary
  Composite Media} %Title of paper

% repeat the \author .. \affiliation  etc. as needed
% \email, \thanks, \homepage, \altaffiliation all apply to the current author.
% Explanatory text should go in the []'s, 
% actual e-mail address or url should go in the {}'s for \email and \homepage.
% Please use the appropriate macro for the type of information

% \affiliation command applies to all authors since the last \affiliation command. 
% The \affiliation command should follow the other information.

\author{N. B. Murphy}
%\email[]{benmurphy.math@gmail.com}
%\homepage[]{Your web page}
%\thanks{}
%\altaffiliation{}
\affiliation{University of Utah, Department of Mathematics, 155 S 1400
  E RM 233, Salt Lake City, UT 84112-009, USA}

% Collaboration name, if desired (requires use of superscriptaddress option in \documentclass). 
% \noaffiliation is required (may also be used with the \author command).
%\collaboration{}
%\noaffiliation

\date{\today}

\begin{abstract}
We demonstrate that Lee--Yang--Ruelle--Baker critical theory of the
Ising model may be adapted to characterize percolation driven critical
transitions in transport. In a novel unified approach, we reproduce
Golden's pioneering, static, results for insulator/conductor binary
composite media, and produce the analogus static results for
conductor/superconductor binary composite media, finding the
(two--parameter) scaling relations of each system. Moreover, we extend
the results pertaining to each system to the quasi--static limit,
where the material contrast parameter becomes complex. Under a
physically consistent symmetry assumption, we link these two sets of
scaling relations so that, under this assumption, the scaling
relations of both systems are determined by only (two) parameters. We
also provide a general proof of the fundamental assumption underlying
Golden's origional work: the existence of a spectral gap which
collapses as $p\to p_c$. The proof thereof identifies the phase
transition with the appearance of a delta function in the spectrum at
the right spectral end point, and the divergence of the weight of the
essential delta function in the spectrum at left spectral end point
\emph{precisely} at the percolation threshold.           
\end{abstract}

%\pacs{}% insert suggested PACS numbers in braces on next line

\maketitle %\maketitle must follow title, authors, abstract and \pacs

% Body of paper goes here. Use proper sectioning commands. 
% References should be done using the \cite, \ref, and \label commands
 \section{Introduction}\label{sec:Introduction}
%
In 1997 K. M. Golden proved that, in the static limit,
Lee--Yang--Ruelle--Baker critical theory may be adapted to
characterize percolation driven critical transitions in transport  
\cite{Golden:PRL-3935}. This deep and far reaching result puts these
two classes of seemingly unrelated problems on an equal mathematical
footing. He did so by considering percolation models, where the
connectedness of the system is determined by the volume fraction $p$
of defect inclusions in an otherwise homogeneous medium. He
demonstrated that the function $m(h):=m(p,h)$ plays the role of the
magnetization per spin $M(T,H)$ in the Ising 
model. Here, the volume fraction $p$ mimics the temperature $T$
while the contrast ratio $h$ mimics the applied magnetic field
$H$. More specifically, the critical insulator/conductor behavior in
transport arises when $h=0$, as $p\to p_c$ \cite{Golden:PRL-3935}, and
in the Ising model the analogous critical behavior arises when $H=0$,
as $T\to T_c$ \cite{Christensen-2005}. Using these mathematical
parallels, K. M. Golden showed that the critical exponents of
transport satisfy Baker's inequalities, Baker's (two--parameter) scaling
relations, etc.
------------------------------------------------------------------------------
%
\section{The Analytic Continuation Method}\label{eq:TACM}
Define the Hilbert space of stationary random fields by $\mathscr{H}_s\subset L^2(\Omega,P)$
where the probability measure $P(d\omega)$ is compatible with
stationarity \cite{Golden:CMP-473}. Furthermore, define the associated
Hilbert space of stationary curl free random fields  
%
\begin{align}\label{eq:curlfreeHilbert}
  \mathscr{H}_\times:=\{\vec{Y}(\omega)\in \mathscr{H}_s \ | \ \vec{\nabla} \times\vec{Y}=0 \text{ weakly and }
\langle\vec{Y}\rangle=0\}, 
\end{align}
%
where $\langle\cdot\rangle$ means ensemble average over $\Omega$, $\omega\in\Omega$, and
$\vec{Y}:\Omega\mapsto\mathbb{R}^d$. Consider the following variational problem: 
find $\vec{E}_f\in\mathscr{H}_\times$ such that   
%
\begin{align}\label{eq:Weak_Curl_Free_Variational_Form}
  \langle\bm{\epsilon}(\vec{E}_0+\vec{E}_f)\cdot\vec{Y}\rangle=0 \quad  \forall \
  \vec{Y}\in\mathscr{H}_\times
\end{align}
%
where $\vec{E}_0$ is the external uniform electric field applied to
the system \cite{Golden:CMP-473}. Under the assumption that the
bilinear form
$a(\vec{u},\vec{v})=\vec{u}^{\;T}\bm{\epsilon}(\vec{x},\omega)\vec{v}$, where
$\vec{x},\vec{u},\vec{v}\in\mathbb{R}^d$, is bounded and coercive, this
problem has a unique solution \cite{Golden:CMP-473} satisfying
\eqref{eq:Maxwells_Equations_ED}.    


Let $(\Omega,P)$ be a probability space and let $\bm{\epsilon}(\vec{x},\omega)$  be
the local permittivity, a (spatially) stationary random field in
 $\vec{x}\in\mathbb{R}^d$ and $\omega \in \Omega$, where $\Omega$ is the set of all
realizations of our random medium, and $P(d\omega)$ is the underlying
probability measure, compatible with stationarity
\cite{Golden:CMP-473}. We assume $\bm{\epsilon}(\vec{x},\omega)$ takes the
values $\epsilon_1$ and $\epsilon_2$ and write $\bm{\epsilon}(\vec{x},\omega) = \epsilon_1 \chi_1
(\vec{x},\omega) + \epsilon_2 \chi_2(\vec{x},\omega)$, where $\chi_j$ is the
characteristic function of medium $j=1,2$, which equals one for all
$\omega\in\Omega$ having medium $j$ at $\vec{x}$, and zero otherwise
\cite{Golden:CMP-473}.   

Let $\vec{E}(\vec{x})$ and $\vec{D}(\vec{x})$ be the stationary random
electric and displacement fields, related by $\vec{D}= \bm{\epsilon}
\vec{E}$, satisfying \cite{Golden:CMP-473} 
%
\begin{align}\label{eq:Maxwells_Equations_ED}  
	\vec{\nabla}\times\vec{E}=0, \quad
	\vec{\nabla}\cdot\vec{D}=0,\quad \quad	
	\vec{E}=\vec{E}_0+\vec{E}_f, \quad
	\langle\vec{E}\rangle=\vec{E}_0,
\end{align}
%
where $\langle \cdot \rangle$ denotes ensemble average over $\Omega$ or, by an ergodic
theorem \cite{Golden:CMP-473}, spatial average
over all of ${\mathbb{R}}^d$ and $\vec{E}_f$ is the fluctuating field
with mean zero about $\vec{E}_0$. We write $\vec{E}_0=E_0\vec{e}_k$,
where $\vec{e}_k$ is a unit vector, for some $k = 1, \ldots, d$. The
effective complex permittivity tensor $\bm{\epsilon}^*$ is defined as 
%
\begin{equation}                                    \label{eq:eff_eps_def}
    \langle \vec{D} \rangle=  \bm{\epsilon} ^* \langle \vec{E} \rangle.
    %\quad \text{and}  \quad 
    %\langle \vec{E} \rangle=  (\bm{\epsilon}^*)^{-1} \langle \vec{D} \rangle,
\end{equation}
%
For simplicity we focus on one diagonal
coefficient $\epsilon^*:=\bm{\epsilon}^*_{kk}$. Central to our studies is the,
system energy given by $\frac{1}{2}\;\langle\vec{D}\cdot\vec{E}\rangle$
\cite{Jackson-1999}. A key variational calculation \cite{Golden:CMP-473}
yields the energy constraint $\langle\vec{D}\cdot\vec{E}_f\rangle=0$. Therefore
the energy may be expressed as  
%
\begin{equation}                                    \label{eq:Reduced_Energy}
    \frac{1}{2}\;\langle\vec{D}\cdot\vec{E}\rangle
      =\frac{1}{2}\;\langle\vec{D}\rangle\cdot\vec{E}_0
        =  \frac{1}{2}\;\epsilon^* E_0^2.
\end{equation}
%
Due to the homogeneity of $\epsilon^*$, $\epsilon^*(a\sigma_1,a\sigma_2)=a\epsilon^*(\sigma_1,\sigma_2)$, it
depends only on the ratio $h = \epsilon_1/\epsilon_2$ and we define $m(h)
=\epsilon^*/\epsilon_2$. The function $m(h)$ is analytic off the negative real axis
in the $h$--plane, taking the upper half plane to the upper half plane
\cite{Golden:CMP-473}. For definiteness we assume that $|\epsilon_1|<|\epsilon_2|$
so that $0<|h|<1$. Metal inclusions are modeled by letting $|\epsilon_2|\to\infty$
or $|h|\to0$ \cite{Wen:SoftMatter-200,Jackson-1999}.    

The key step in the method is obtaining an integral representation for
$\epsilon^*$. It is more convenient to consider $F(s):=1-m(h)$, $s=1/(1-h)$,
which is analytic off $[0,1]$ in the $s$--plane
\cite{Bergman:PRC-377,Golden:CMP-473}.  Then \cite{Golden:CMP-473}
%
\begin{align} \label{eq:Fs_Integral}
	F(s)=1-m(h)
	=\int_0^1 \frac{d\mu(\lambda)}{s-\lambda}\;,
   \qquad m(h)=\frac{\epsilon^*}{\epsilon_2},
   \quad s = \frac{1}{1-\epsilon_1/\epsilon_2}\;,   
 \end{align}
%
where $\mu$ is a positive measure on $[0,1]$, and by definition
$0\leq|F(s)|\leq1$. Equation \eqref{eq:Fs_Integral} is a
general formula holding for two component stationary random media in
the lattice and continuum settings \cite{Golden:PRL-3935}. For later
physical considerations we also define the function $w(z)=\epsilon^*/\epsilon_1$,
$z=1/h$ which is analytic of the negative real axis in the $z$--plane, 
taking the upper half plane to the upper half plane
\cite{Golden:CMP-473}. Again, it is more convenient to consider
$G(t):=1-w(z)$, $t=1/(1-z)=1-s$, which is analytic off $[0,1]$ in the 
$t$--plane \cite{Bergman:PRC-377,Golden:CMP-473,Bergman:AP-78}. Then  
%
\begin{equation}\label{eq:Gt_Integral}
	G(t)=1-w(z) 
	=\int_0^1\frac{d\alpha(\lambda)}{t-\lambda}\;,
        \qquad w(z)=\frac{\epsilon^*}{\epsilon_1},
   \quad t=1-s\;,
\end{equation}
%
where $\alpha$ is a positive measure on $[0,1]$, and by definition
$-\infty<|G(t)|\leq0$.

These Stieltjes transforms of $\mu$ etc. are examples of
Herglotz functions. As will be shown in section
\ref{subsec:Spec_Decomp_Energy} below, applying The Spectral Theorem
\cite{Reed-1980} to the energy constraint  $\langle\vec{D}\cdot\vec{E}_f \rangle=0$
yields 
% 
\begin{equation} \label{eq:Energy_Constraint}
	\langle E_f^2\rangle=E_0^2\int_0^1\frac{\lambda \; d\mu(\lambda)}{(s-\lambda)^2}\;, 
\end{equation}
% 
which in turn leads to an exact, detailed description of all other
energy contributions within the system in terms of Herglotz functions
involving $\mu$. A key feature of formulas
\eqref{eq:Reduced_Energy}--\eqref{eq:Energy_Constraint}
%\eqref{eq:Fs_Integral},\eqref{eq:Gt_Integral}, and
%\eqref{eq:Energy_Constraint}
is that the parameter information in $s$ 
and $E_0$ is {\it separated} from the geometry of the composite
encapsulated in $\mu$, through its moments $\mu_i=\int_0^1 \lambda^i d\mu(\lambda)$, which
depend on the correlation functions of the medium
\cite{Golden:CMP-473}. For example, $\mu_0=p_1$, the volume fraction of
$\epsilon_1$. The Stieltjes integral representation for $F(s)$ may be used to
obtain rigorous bounds on $\epsilon^*$ given microstructural information 
\cite{Bergman:PRL-1285,Milton:APL-300,Golden:CMP-473,Bergman:AP-78}. 

As will be shown in section \ref{sec:Resolv_Rep_E_D} below, these
Stieltjes formulas arise from the resolvent representation of the
electric field $\vec{E}$ and displacement field $\vec{D}$. For
example, $\vec{E} = s(s + \Gamma \chi_1)^{-1}\vec{E}_0$, is obtained via
manipulation of \eqref{eq:Maxwells_Equations_ED}, where $\Gamma=\vec{\nabla}(-\Delta)^{-1}\nabla\cdot$,
yielding $F(s)=\langle\chi_1[(s+\Gamma\chi_1)^{-1}\vec{e}_k]\cdot\vec{e}_k \rangle$. The operator
$\Gamma$ is a projection onto curl-free fields, based on convolution with
the free-space Green's function for $-\Delta$ \cite{Golden:CMP-473}. In the
Hilbert space $L^2(\Omega,P)$ with weight $\chi_1$ in the inner product,
$\Gamma\chi_1$ a bounded linear self--adjoint integro--differential operator
with simple spectrum \cite{Golden:CMP-473}. Moreover, the
underlying self--adjoint operator, $\mathbf{M}:=\chi_1\Gamma\chi_1$ is
composition of projection operators so that $\|\mathbf{M}\|\leq1$, where
$\|\cdot\|$ denotes the operator norm on $L^2(\Omega,P)$. Formula
\eqref{eq:Fs_Integral} involves a spectral representation of the
resolvent $(s+\Gamma\chi_1)^{-1}$, where $\mu$ is a spectral measure of the
family of projections of $-\Gamma\chi_1$, in the $\langle\vec{e}_k,\vec{e}_k\rangle$ state
\cite{Golden:CMP-473,Reed-1980}.

The equivalence of these two representations allows one to derive
relationships between the measures $\mu$ and $\alpha$ via The Stieltjes-Perron
Inversion Theorem \cite{Day:JPCM-96,Henrici:1974:v3,MILTON:2002:TC}. 
%
\begin{align}\label{eq:Stieltjes-Perron}
  \mu(y)=-\frac{1}{\pi}\lim_{\delta\downarrow0}\text{Im}F(y+\I\delta)\;, \quad
  y\in\text{supp}(\mu):=\Sigma_\mu 
\end{align}
%
To evoke this theorem directly we note that $t=1-s$ and write
%
\begin{align}\label{eq:Gts}
  G(t(s))=-\int_0^1\frac{d\alpha(\lambda)}{s-(1-\lambda)}
         :=-\int_0^1\frac{d\tilde{\alpha}(\lambda)}{s-\lambda}
         :=-\tilde{G}(s),
\end{align}
%
where $d\tilde{\alpha}(\lambda)=-d\alpha(1-\lambda)$. By equations
\eqref{eq:Herglotz_Energy_Rep_E}, \eqref{eq:Herglotz_Energy_Rep_G},
and $h=1-1/s$ we have  
%
\begin{align}\label{eq:Fs_relationships_G}
  1-F(s)=(1-1/s)(1+\tilde{G}(s))
\end{align}
%
Therefore, by setting $s=y+\I\delta$ for $y\in\Sigma_\mu\cap\Sigma_\alpha$ and $y\neq0$, where
$\Sigma_\mu:=\text{supp}(\mu)$, equations 
\eqref{eq:Fs_Integral} and
\eqref{eq:Stieltjes-Perron}--\eqref{eq:Fs_relationships_G} imply that
%
\begin{align}
  \mu(y)+\tilde{\alpha}(y)&=\frac{1}{\pi}\lim_{\delta\downarrow 0}\text{Im}
        \left[ \frac{-y+\I\delta}{y^2+\delta^2}
          \left(\text{Re}\,\tilde{G}(y+\I\delta)+\I\,\text{Im}\,\tilde{G}(y+\I\delta)
          \right)\right]\\
       &=\frac{1}{\pi}\lim_{\delta\downarrow 0}
        \left[\frac{\delta}{y^2+\delta^2}\text{Re}\,\tilde{G}(y+\I\delta) 
          -\frac{y}{y^2+\delta^2}\text{Im}\,\tilde{G}(y+\I\delta)
          \right]\notag \\
          &:= \frac{1}{y}\tilde{\alpha}(y)-\varrho(y),\notag
  \quad \varrho(y):=-\frac{1}{\pi}\lim_{\delta\downarrow 0}
  \frac{\delta}{y^2+\delta^2}\text{Re}\tilde{G}(y+\I\delta).\notag 
\end{align}
%
Thus,
%
\begin{align}\label{eq:BM_measure_relationship_E}
  y\mu(y)&=(1-y)[-\alpha(1-y)] - y\varrho(y), \\
  \varrho(y)&=-\frac{1}{\pi}\lim_{\delta\downarrow 0}\frac{\delta}{y^2+\delta^2}
  \int_0^1\frac{(y+\lambda-1)d\alpha(\lambda)}{(y+\lambda-1)^2+\delta^2}\,,\notag
\end{align}
%
where the minus sign in $[-\alpha(1-y)]$ is different from
\cite{Day:JPCM-96}, and accounts for reversal in the integration
order, $1-y:[0,1]\mapsto[1,0]$. Equation
\eqref{eq:BM_measure_relationship_E} is a general formula holding for  
two component stationary random media in the lattice and continuum
settings \cite{Golden:PRL-3935}. In Section \ref{sec:Measure_Equiv}
below we prove, for percolation models where the connectivity of
the system is determined by the volume fraction $p$ of defect
inclusions within an otherwise homogeneous medium, that the measure
$\varrho$ is given by $\varrho(dy)=W_0(p)\delta_0(dy)+m(p,0)(1-y)\delta_1(dy)$, where
$\delta_{y_0}(dy)$ is the Dirac measure centered at $y_0$, $W(p)\not\equiv0$,
and we have set $m(h):=m(p,h)$. 

%
%--------------------------------------------------------------------------------
%
%
\subsection{A General Spectral Decomposition of the System Energy}
\label{subsec:Spec_Decomp_Energy}
%
In this section we extend the utility of the mathematical and physical
properties of the effective permittivity by constructing a detailed
decomposition of the system energy, holding for general two component
stationary random media in the lattice and continuum settings.

In
section \ref{sec:Resolv_Rep_E_D} we showed that the variational 
problems \eqref{eq:Weak_Curl_Free_Variational_Form} and
\eqref{eq:Weak_Div_Free_Variational_Form} provide reduced
representations of the system energy in terms of $\epsilon^*$ and
$(\epsilon^*)^{-1}$. In particular, we showed that
%
\begin{align}\label{eq:Energy_Constraint_E_D}
  \langle\vec{D}\cdot\vec{E}_f\rangle=\langle\vec{E}\cdot\vec{D}_f\rangle=0,
\end{align}
%
where $\vec{E}_f:=\vec{E}-\vec{E}_0$ and
$\vec{D}_f:=\vec{D}-\vec{D}_0$ are the fluctuations about the mean
electric field, $\langle\vec{E}\rangle:=\vec{E}_0$, and mean displacement field, 
$\langle\vec{D}\rangle:=\vec{D}_0$, respectively, so that $\langle\vec{E}_f\rangle=\langle\vec{D}_f\rangle=0$, where
$\langle\cdot\rangle$ denotes ensemble average over $\Omega$ or volume average over all
$\mathbb{R}^d$, and $d$ is the dimension of the system. A direct
consequence of these variational problems is that the system energy
per-unit-volume (puv) has the following reduced representations
\cite{Jackson-1999}  
%
\begin{align}\label{eq:Reduced_Energy_E_D}
  \frac{1}{2}\langle\vec{D}\cdot\vec{E}\rangle=\frac{1}{2}\langle\vec{D}\cdot\vec{E}_0\rangle
  =\frac{1}{2}\langle\vec{D}_0\cdot\vec{E}\rangle.
\end{align}
%

By the symmetries discussed in sections \ref{sec:Background_TACM} and
\ref{sec:Resolv_Rep_E_D} we, with out loss of generality, focus on one
diagonal coefficient of the effective permittivity
$\epsilon^*:=\bm{\epsilon}^*_{kk}$ and the representation
$\langle\vec{D}\cdot\vec{E}_0\rangle:=\bm{\epsilon}^*\vec{E}_0\cdot\vec{E}_0:=\epsilon^*E_0^2$, with
$\vec{E}_0=E_0\vec{e}_k$, $\vec{e}_k$ the $k$th standard basis vector
in $\mathbb{R}^d$. Denote by $\mathbf{R}_s=(s+\Gamma\chi_1)^{-1}$ the
resolvent of the operator $-\Gamma\chi_1$, 
which is self adjoint in the $L^2(\Omega,P)$ inner product weighted by
$\chi_1$ \cite{Golden:CMP-473}. Hence $\mathbf{R}_s$ is also self adjoint
with respect to this inner--product for $s\in\mathbb{C}\backslash[0,1]$
\cite{Stone:64}. The energy constraint
\eqref{eq:Energy_Constraint_E_D} yields 
% 
\begin{align}
 0=\langle\vec{D}\cdot\vec{E}_f\rangle&=\langle\epsilon_2(1-\chi_1/s)\vec{E}\cdot\vec{E}_f\rangle
 =\langle\epsilon_2(1-\chi_1/s)(\vec{E}_0\cdot\vec{E}_f+E_f^2)\rangle\\
 &=\epsilon_2\left(\langle E_f^2\rangle- \left(\langle\chi_1\vec{E}_0\cdot\vec{E}_f\rangle
     + \langle\chi_1E_f^2\rangle\right)/s\right).\notag
\end{align}
%
Therefore, by The Spectral Theorem \ref{thm:Spectral_Theorem}  
%
\begin{align}\label{eq:seed}
  \langle E_f^2\rangle&=\frac{E_0^2}{s}\left( \langle\chi_1(s\mathbf{R}_s\vec{e}_k\cdot\vec{e}_k-1)\rangle
    +\langle\chi_1\|(s\mathbf{R}_s-1)\vec{e}_k\|^2\rangle   \right) \\
    &= \frac{E_0^2}{s} \left(
      \int_0^1\left[ \frac{s}{s-\lambda}-1 \right] d\mu(\lambda)
      + \int_0^1\left[ \frac{s}{s-\lambda}-1 \right]^2 d\mu(\lambda)
      \right)\notag\\
    &=\frac{E_0^2}{s}\int_0^1 \frac{\lambda(s-\lambda)+\lambda^2}{(s-\lambda)^2} d\mu(\lambda)\notag\\
    &=E_0^2\int_0^1 \frac{\lambda d\mu(\lambda)}{(s-\lambda)^2}:=-E_0^2\frac{\partial F_1(s)}{\partial s}, \notag
\end{align}
%
where the Stieltjes transformation, $F_i(s)$, of the measure $\lambda^i\mu$ is
defined in equation \eqref{Eq:Energy_Basis_F}. Equation
\eqref{eq:seed} is a general formula holding for two component
stationary random media in the lattice and continuum settings
\cite{Golden:PRL-3935}.

The main theorem of this section follows immediately
from formula \eqref{eq:seed} and The Spectral Theorem
\ref{thm:Spectral_Theorem}, which shows that the $F_j(s)$, $j=0,1,2$ and
their first derivatives serve as basis functions for a detailed
decomposition of the system energy in terms of Herglotz functions
involving $\mu$, where $F_0(s):=F(s)$. 

% 
\begin{theorem}\label{thm:Herglotz_Decomp_Energy}
   Denote by $F_i(s)$, the Stieltjes transformation of the measure
   $\lambda^i\mu(d\lambda)$ (see equation \eqref{Eq:Energy_Basis_F}). Then the
   following are general formula holding for two component stationary  
   random media in the lattice and continuum settings: 
%
  \begin{align}\label{eq:Ed2_Energy_Partitions_Ef2}
  \frac{\langle\chi_1E_f^2\rangle}{E_0^2}=-\frac{\partial F_2(s)}{\partial s},   \quad
  \frac{\langle E_f^2\rangle}{E_0^2}=-\frac{\partial F_1(s)}{\partial s}, \quad
  % \frac{\langle\chi_2E_f^2\rangle}{E_0^2}=\frac{\partial F_2(s)}{\partial s}-\frac{\partial F_1(s)}{\partial s}
\end{align}  
%
  \begin{align}\label{eq:Ed2_Energy_Partitions_Ef*E0}
  \frac{ \langle\chi_1\vec{E}_f\cdot\vec{E}_0\rangle}{E_0^2}=F_1(s), \quad
  \frac{ \langle\vec{E}_f\cdot\vec{E}_0\rangle}{E_0^2}=0, \quad
 % \frac{ \langle\chi_2\vec{E}_f\cdot\vec{E}_0\rangle}{E_0^2}=-F_1(s),
  \end{align}
%
  \begin{align}\label{eq:Ed2_Energy_Partitions_E2}
   \frac{\langle\chi_1E^2\rangle}{E_0^2}=-s^2\frac{\partial F(s)}{\partial s}, \quad
   \frac{ \langle E^2\rangle}{E_0^2}=1-\frac{\partial F_1(s)}{\partial s}, \quad
  % \frac{ \langle\chi_2E^2\rangle}{E_0^2}=1-\frac{\partial F_1(s)}{\partial s}+s^2\frac{\partial F(s)}{\partial s},
  \end{align}
%
  \begin{align}\label{eq:Ed2_Energy_Partitions_E*E0}
    \frac{ \langle\chi_1\vec{E}\cdot\vec{E}_0\rangle}{E_0^2}=sF(s),\quad
    \frac{\langle\vec{E}\cdot\vec{E}_0\rangle}{E_0^2}=1,\quad
   % \frac{ \langle\chi_2\vec{E}\cdot\vec{E}_0\rangle}{E_0^2}=1-sF(s),
  \end{align}
%
  \begin{align}\label{eq:Ed2_Energy_Partitions_E*Ef}
    \frac{ \langle\chi_1\vec{E}\cdot\vec{E}_f\rangle}{E_0^2}=-s^2\frac{\partial F(s)}{\partial s}-sF(s),\quad
    \frac{\langle\vec{E}\cdot\vec{E}_f\rangle}{E_0^2}=-\frac{\partial F_1(s)}{\partial s},\quad
   % \frac{ \langle\chi_2\vec{E}\cdot\vec{E}_f\rangle}{E_0^2}=(s-1)\frac{\partial F_1(s)}{\partial s}.
  \end{align}
%
  where the formulas involving $\chi_2$ are given by the relation $\chi_2=1-\chi_1$.
\end{theorem}
%
\noindent \textbf{Proof}:
  Using equation \eqref{eq:seed} and The
  Spectral Theorem \ref{thm:Spectral_Theorem} we have
%  
  \begin{align*}
  \frac{\langle\chi_1E_f^2\rangle}{E_0^2}&= \langle\chi_1(s\mathbf{R}_s-1)^2\vec{e}_k\cdot\vec{e}_k\rangle
             =\int_0^1 \left(\frac{s}{s-\lambda} -1\right)^2 d\mu(\lambda)
             =\int_0^1 \frac{\lambda^2d\mu(\lambda)}{(s-\lambda)^2}=-\frac{\partial F_2(s)}{\partial s}. \\
  \frac{\langle E_f^2\rangle}{E_0^2}&=  \langle\|(s\mathbf{R}_s-1)\vec{e}_k\|^2 \rangle
                  =\int_0^1\frac{\lambda d\mu(\lambda)}{(s-\lambda)^2}=-\frac{\partial F_1(s)}{\partial s}.\notag 
 % \langle\chi_2E_f^2\rangle/E_0^2&=  \langle(1-\chi_1)E_f^2 /E_0^2\rangle =\int_0^1\frac{\lambda(1-\lambda)d\mu(\lambda)}{(s-\lambda)^2}
\end{align*}
%
\begin{align*}
  \frac{\langle\chi_1\vec{E}_f\cdot\vec{E}_0\rangle}{E_0^2}&=
     \langle\chi_1(s\mathbf{R}_s-1)\vec{e}_k\cdot\vec{e}_k \rangle
     =\int_0^1\frac{\lambda d\mu(\lambda)}{s-\lambda}=F_1(s).\\
  \frac{\langle\vec{E}_f\cdot\vec{E}_0\rangle}{E_0^2}&= 0. \notag
 % \langle\chi_2\vec{E}_f\cdot\vec{E}_0\rangle/E_0^2&=- \langle\chi_1\vec{E}_f\cdot\vec{E}_0/E_0^2\rangle\notag 
\end{align*}
%
\begin{align*}
  \frac{\langle\chi_1E^2\rangle}{E_0^2}&= \langle\chi_1s^2R^2_s\vec{e}_k\cdot\vec{e}_k \rangle
           =s^2\int_0^1\frac{d\mu(\lambda)}{(s-\lambda)^2}=-s^2\frac{\partial F(s)}{\partial s}.\\
 \frac{\langle E^2\rangle}{E_0^2}&=  \frac{\langle(E_0^2+2\vec{E}_f\cdot\vec{E}_0+E_f^2)\rangle}{E_0^2} 
  =1+\int_0^1\frac{\lambda d\mu(\lambda)}{(s-\lambda)^2}=1-\frac{\partial F_1(s)}{\partial s}.\notag
 % \langle\chi_2E^2\rangle/E_0^2&=  \langle(1-\chi_1)E^2/E_0^2 \rangle
 % =1+\int_0^1\frac{\lambda d\mu(\lambda)}{(s-\lambda)^2}-s^2\int_0^1\frac{d\mu(\lambda)}{(s-\lambda)^2}\notag 
\end{align*}
%
\begin{align*}
  \frac{\langle\chi_1\vec{E}\cdot\vec{E}_0\rangle}{E_0^2}&=
      \langle\langle\chi_1s\mathbf{R}_s\vec{e}_k\cdot\vec{e}_k\rangle
       =s\int_0^1\frac{d\mu(\lambda)}{s-\lambda}=sF(s).\\
  \frac{\langle\vec{E}\cdot\vec{E}_0\rangle}{E_0^2}&= 1.\notag 
 % \langle\chi_2\vec{E}\cdot\vec{E}_0 /E_0^2\rangle&=
 %     \langle(1-\chi_1)\vec{E}\cdot\vec{E}_0\rangle/E_0^2
 % =1-s\int_0^1\frac{d\mu(\lambda)}{s-\lambda}\notag 
\end{align*}
%
\begin{align*}
  \frac{\langle\chi_1\vec{E}\cdot\vec{E}_f\rangle}{E_0^2}&=\frac{\langle\chi_1E^2\rangle-\langle\chi_1\vec{E}\cdot\vec{E}_0\rangle}{E_0^2}        
        =-s^2\frac{\partial F(s)}{\partial s}-sF(s).\\
  \frac{\langle\vec{E}\cdot\vec{E}_f\rangle}{E_0^2}&=\frac{\langle E^2\rangle-\langle\vec{E}\cdot\vec{E}_0\rangle}{E_0^2}
     =\int_0^1\frac{\lambda d\mu(\lambda)}{(s-\lambda)^2}=-\frac{\partial F_1(s)}{\partial s}.
 % \langle\chi_2\vec{E}\cdot\vec{E}_f\rangle/E_0^2&=\langle(1-\chi_2)\vec{E}\cdot\vec{E}_f\rangle/E_0^2
 % =(1-s)\int_0^1\frac{\lambda d\mu(\lambda)}{(s-\lambda)^2}.\notag 
\end{align*}
%
These formula hold for general two component stationary random media in
the lattice and continuum settings \cite{Golden:PRL-3935}.
%
All the possible variants of energy per--unit--volume (puv) may be
easily found using Theorem \ref{thm:Herglotz_Decomp_Energy}; for
example $\langle\epsilon_1\chi_1\vec{E}\cdot\vec{E}_f\rangle=-\epsilon_1E_0^2(s^2F^\prime(s)+sF(s))$. If we
denote Theorem \ref{thm:Herglotz_Decomp_Energy} ``The 
$(\vec{E},F(s),\mu)$ Energy Theorem'', then, by the symmetries discussed in
sections \ref{sec:Background_TACM} and \ref{sec:Resolv_Rep_E_D}, the analogous
$(\vec{E},G(t),\alpha)$, $(\vec{D},E(s),\eta)$, and $(\vec{D},H(t),\tau)$ energy
theorems also hold.
%

\section{Critical Behavior of Transport in Lattice and Continuum
  Percolation Models}
\label{sec:Crit_Behav_of_Transport}
%
In 1997 K. M. Golden proved that, in the static limit,
Lee--Yang--Ruelle--Baker critical theory may be adapted to
characterize percolation driven critical transitions in transport  
\cite{Golden:PRL-3935}. This deep and far reaching result puts these
two classes of seemingly unrelated problems on an equal mathematical
footing. He did so by considering percolation models, where the
connectedness of the system is determined by the volume fraction $p$
of defect inclusions in an otherwise homogeneous medium. He
demonstrated that the function $m(h):=m(p,h)$ \eqref{eq:Fs_Integral}
plays the role of the magnetization per spin $M(T,H)$ in the Ising
model. Here, the volume fraction $p$ mimics the temperature $T$
while the contrast ratio $h$ mimics the applied magnetic field
$H$. More specifically, the critical insulator/conductor behavior in
transport arises when $h=0$, as $p\to p_c$ \cite{Golden:PRL-3935}, and
in the Ising model the analogous critical behavior arises when $H=0$,
as $T\to T_c$ \cite{Christensen-2005}. Using these mathematical
parallels, K. M. Golden showed that the critical exponents of
transport satisfy Baker's inequalities, Baker's (two--parameter) scaling
relations, etc.

The most natural formulation of this problem is in terms of the
conduction problem in the continuum $\mathbb{R}^d$, which includes the
lattice $\mathbb{Z}^d$ as a special case
\cite{Golden:JMP-5627,Golden:CMP-473}. Although, the underlying
symmetries in the effective parameter problem of electrical
conductivity and permittivity, magnetic permeability, and thermal 
conductivity, immediately generalize the results of this section to
all of these systems \cite{MILTON:2002:TC}. Let $\sigma_j$ denote the
complex conductivity of material component $j=1,2$ of the binary
composite \cite{Efros:PSSB-303}, and $\sigma^*(p,h)$ denote the effective
complex conductivity. In the limit $h:=\sigma_1/\sigma_2\to0$, the composite may  
be interpreted as a conductor/superconductor system ($\sigma_2\to\infty$ while
$0<|\sigma_1|<\infty$), or a conductor/insulator system ($\sigma_1\to0$ while
$0<|\sigma_2|<\infty$). 

The relationship \eqref{eq:Fs_relationships_G} between the different 
representations of the effective complex conductivity,
$\sigma^*(p,h):=\sigma_2\,m(p,h)=\sigma_1w(p,z(h))$
\eqref{eq:Fs_Integral}--\eqref{eq:Gt_Integral}, is exploited to
illuminate many of the symmetries in this framework. We show how 
symmetries between the integral representations of $\sigma^*$ and
$[\sigma^{-1}]^*$ may be used to immediately generalize the results of
this section in terms of $[\sigma^{-1}]^*$. Some of the more subtle measure
theoretic details regarding the underlying symmmetries between
$m(p,h)$ and $w(p,z(h))$ are discussed in section
\ref{sec:Measure_Equiv}. This leads to a generalization of a result
\cite{Day:JPCM-96} which determines the measure $\varrho(d\lambda)$ introduced in
equation \eqref{eq:BM_measure_relationship_E}, and characterizes the
phase transition by the appearance of a delta function component in
$\varrho(d\lambda)$ \emph{precisely} at the percolation threshold. In section
\ref{sec:StatMech_of_Composites} we will demonstrate that this
critical theory of transport may be extended further, characterizing 
the GBCE statistical mechanics description of (electrically/thermally)
driven phase transitions in binary composite media.           
%
\subsection{Formulation} \label{sec:Crit_Theory_Formulation} 
%
As we have already discussed the effective permittivity problem in
section \ref{sec:Background_TACM} and the effective permeability
problem in section \ref{sec:Magnetic_Systems}, we now give an extremely
brief formulation of the effective conductivity problem
\cite{Golden:JMP-5627,Golden:CMP-473,Golden:PRL-3935}. Let the local
conductivity be defined
as $\bm{\sigma}(\vec{x},\omega):=\sigma_1\chi_1(\vec{x},\omega)+\sigma_2\chi_2(\vec{x},\omega)$ 
and its inverse be defined as
$[\bm{\sigma}^{-1}](\vec{x},\omega):=\chi_1(\vec{x},\omega)/\sigma_1+\chi_2(\vec{x},\omega)/\sigma_2$,
which are two--valued stationary random fields in
$\vec{x}\in\mathbb{R}^d$ and $\omega\in\Omega$. Let $\vec{E}(\vec{x},\omega)$ and
$\vec{J}(\vec{x},\omega)$ be stationary random electric and current fields,
which are related by $\vec{J}=\bm{\sigma}\vec{E}$ and satisfy 
%
\begin{align}\label{eq:Maxwells_Equations_EJ}  
  	\vec{\nabla}\times\vec{E}=0, \quad
	\vec{\nabla}\cdot\vec{J}=0,\qquad 	
	\vec{E}=\vec{E}_0+\vec{E}_f, \quad
	\langle\vec{E}\rangle=\vec{E}_0:=E_0\vec{e}_k.
\end{align}
%
For the random resistor network (RRN), the differential equations
become difference equations (Kirchoff's laws)
\cite{Golden:CMP-467,Golden:JMP-5627}.

The effective complex conductivity tensor, $\bm{\sigma}^*$, and 
$[\bm{\sigma}^{-1}]^*$ are defined via the averages
$\langle\vec{J}\rangle=\bm{\sigma}^*\langle\vec{E}\rangle$ and $\langle\vec{E}\rangle=[\bm{\sigma}^{-1}]^*\langle\vec{J}\rangle$,
respectively. Without loss of  
generality, we focus on one diagonal component of these
symmetric tensors: $\sigma^*:=\bm{\sigma}^*_{kk}$ and
$[\sigma^{-1}]^*:=[\bm{\sigma}^{-1}]^*_{kk}$. Due to the homogeneity of the
functions $\sigma^*$ and $[\sigma^{-1}]^*$, we consider the dimensionless functions
$m(h)=\sigma^*/\sigma_2$, $w(z)=\sigma^*/\sigma_1$, $\tilde{m}(h)=\sigma_1[\sigma^{-1}]^*$, and
$\tilde{w}(z)=\sigma_2[\sigma^{-1}]^*$, defined in
\eqref{eq:Fs_Integral}--\eqref{eq:Ht_Integral}, where $h=\sigma_1/\sigma_2$ and
$z=1/h$. To simplify the presentation of this framework we focus on
the variable $h$, and make the following definitions: $w(h):=w(z(h))$
and $\tilde{w}(h):=\tilde{w}(z(h))$. We assume that $|h|<1$,
i.e. $0<|\sigma_1|<|\sigma_2|<\infty$, and we further restrict $h$ in the complex
plane to the set  
%
\begin{align}\label{eq:h_Domain}
  \mathcal{U}:=\{h:=h_r+\I h_i\in\mathbb{C}: |h|<1 \text{ and } h\not\in(-1,0]\},
\end{align}
%
where $m(h)$, $w(h)$, $\tilde{m}(h)$, and $\tilde{w}(h)$ are analytic
functions of $h$ \cite{Golden:CMP-473}.  

The associated Stieltjes (or Herglotz) functions are given by $F(s)=1-m(h)$,
$G(t)=1-w(z)$, $E(s)=1-\tilde{m}(h)$, and $H(t)=1-\tilde{w}(z)$, where
$s=1/(1-h)$ and $t=1-s$. In order to illuminate the symmetries between
these functions we focus on the variable $s$, and define
$G(s):=G(t(s))$ and $H(s):=H(t(s))$. Using $h$ and $s$ in lieu of $z$
and $t$, respectively, these functions have the following integral
representations 
% 
\begin{align}\label{eq:Herglotz_Funs_sed_LYRB}
  F(s)&=\langle\chi_1(s+\Gamma\chi_1)^{-1}\vec{e}_k\cdot\vec{e}_k\rangle=\int_{\lambda_0}^{\lambda_1}\frac{d\mu(\lambda)}{s-\lambda}\,,\\
  E(s)&=\langle\chi_2(s-\Upsilon\chi_2)^{-1}\vec{d}_k\cdot\vec{d}_k\rangle=\int_{\tilde{\lambda}_0}^{\tilde{\lambda}_1}\frac{d\eta(\lambda)}{s-\lambda}\,,
   \notag \\
  G(s)&=\langle\chi_2(1-s+\Gamma\chi_2)^{-1}\vec{e}_k\cdot\vec{e}_k\rangle=-\int_{1-\hat{\lambda}_1}^{1-\hat{\lambda}_0}\frac{[-d\alpha(1-\lambda)]}{s-\lambda}\,,
   \notag \\
  H(s)&=\langle\chi_1(1-s-\Upsilon\chi_1)^{-1}\vec{d}_k\cdot\vec{d}_k\rangle=-\int_{1-\check{\lambda}_1}^{1-\check{\lambda}_0}\frac{[-d\tau(1-\lambda)]}{s-\lambda}\,.
  \notag
\end{align}
%
Here $\mu$, $\eta$, $\alpha$, and $\tau$ are bounded positive measures which
depend only on the geometry of the medium
\cite{Golden:CMP-473,Bergman:AP-78}, and are supported on
$\Sigma_\mu,\Sigma_\eta,\Sigma_\alpha,\Sigma_\tau\subseteq[0,1]$, respectively, where $\lambda_0:=\inf(\Sigma_\mu)\in[0,1)$, 
$\lambda_1:=\sup(\Sigma_\mu)\in(0,1]$, $\tilde{\lambda}_0:=\inf(\Sigma_\eta)\in[0,1)$, $\tilde{\lambda}_1:=\sup(\Sigma_\eta)\in(0,1]$,
$\hat{\lambda}_0:=\inf(\Sigma_\alpha)\in[0,1)$, $\hat{\lambda}_1:=\sup(\Sigma_\alpha)\in(0,1]$,
$\check{\lambda}_0:=\inf(\Sigma_\tau)\in[0,1)$, and  $\check{\lambda}_1:=\sup(\Sigma_\tau)\in(0,1]$ (see section
\ref{sec:Resolv_Rep_E_D} for details).

By the positivity of these measures, the following inequalities hold
for all $p\in[0,1]$: 
%
\begin{align}\label{eq:Herglotz_Inneq}
  \frac{\partial^{2n}F}{\partial s^{2n}}>0, \qquad
  \frac{\partial^{2n-1}F}{\partial s^{2n-1}}<0,
    \qquad \qquad
  \frac{\partial^{2n}E}{\partial s^{2n}}>0,\qquad
  \frac{\partial^{2n-1}E}{\partial s^{2n-1}}<0,
    \\ 
  \frac{\partial^{2n}G}{\partial s^{2n}}<0, \qquad
  \frac{\partial^{2n-1}G}{\partial s^{2n-1}}>0,
    \qquad \qquad
  \frac{\partial^{2n}H}{\partial s^{2n}}<0, \qquad
  \frac{\partial^{2n-1}H}{\partial s^{2n-1}}>0,
  \notag
\end{align}
%
where $n\geq0$ and $h\in\mathcal{U}$. Equation
\eqref{eq:Herglotz_Inneq} is the analogue of equation
\eqref{eq:Gtau_inneq} in the Ising model. The formula $\partial^2F/\partial s^2>0$
in \eqref{eq:Herglotz_Inneq}, for example, is a macroscopic version of
the fact that the effective resistance of a finite network is a
concave downward function of the resistances of the individual network
elements \cite{Golden:JMP-5627}. When $h\in\mathcal{U}$ such that $h_i\neq0$,
equations \eqref{eq:Herglotz_Inneq} become
%
\begin{align}\label{eq:Herglotz_NonZero}
  \left|\frac{\partial^nF}{\partial s^n}\right|>0, \quad
  \left|\frac{\partial^nE}{\partial s^n}\right|>0, \quad
  \left|\frac{\partial^nG}{\partial s^n}\right|>0, \quad
  \left|\frac{\partial^nH}{\partial s^n}\right|>0, \quad p\in[0,1].
\end{align}
%

The formulas in equation \eqref{eq:Herglotz_Funs_sed_LYRB} are, up to
sign and reflection of the spectrum about $\lambda=1/2$, Stieltjes
transforms of the measures $\mu$, $\eta$, $\alpha$, and $\tau$. These can be
converted into Stieltjes functions \cite{Baker-1990} of $h$ via the
change of variables $s=1/(1-h)$ and $\lambda(y)=y/(y+1)\iff y(\lambda)=\lambda/(1-\lambda)$ so
that, for example,  
%
\begin{align}\label{eq:var_subs_Fs}
  F(s)&=\int_{S_0}^{S}\frac{d\mu(\frac{y}{y+1})}
                {\frac{1}{1-h}-\frac{y}{y+1}}
                :=(1-h)\int_{S_0}^{S}\frac{(y+1)d\mu(\frac{y}{y+1})}{1+hy}
                %:=(1-h)\int_{S_0}^{S}\frac{d\phi(y)}{1+hy}
                \,,  \notag\\
  G(s)&=-\int_{\hat{S}_0}^{\hat{S}}\frac{[-d\alpha(\frac{1}{y+1})]}
                {\frac{1}{1-h}-\frac{y}{y+1}}
                :=(h-1)\int_{\hat{S}_0}^{\hat{S}}\frac{(y+1)[-d\alpha(\frac{1}{y+1})]}{1+hy}
                %:=(h-1)\int_{\hat{S}_0}^{\hat{S}}\frac{d\ph(y)}{1+hy}
                \,.               
\end{align}    
%
Here $S_0:=\lambda_0/(1-\lambda_0)$, $S:=\lambda_1/(1-\lambda_1)$,
$\hat{S}_0:=(1-\hat{\lambda}_1)/\hat{\lambda}_1$, and $\hat{S}:=(1-\hat{\lambda}_0)/\hat{\lambda}_0$,
so that $\lim_{\lambda_0\to0}S_0=0$, $\lim_{\lambda_1\to1}S=\infty$,
$\lim_{\hat{\lambda}_1\to1}\hat{S}_0=0$, $\lim_{\hat{\lambda}_0\to0}\hat{S}=\infty$, and
$S\in\mathbb{R}^+$ is not to be confused with entropy. Therefore, by
equation \eqref{eq:var_subs_Fs} and the underlying symmetries in
equations \eqref{eq:Herglotz_Funs_sed_LYRB}, the Stieltjes function
representations of the formulas in equation
\eqref{eq:Herglotz_Funs_sed_LYRB} are given by         
% 
\begin{align}\label{eq:mh_Stieltjes_rep} 
    m(h)&=1+(h-1)g(h),\quad
    g(h):=\int_0^\infty\frac{d\phi(y)}{1+hy}\,, \quad
    d\phi(y):=(y+1)d\mu\left(\frac{y}{y+1}\right),\notag \\
%     
    \tilde{m}(h)&=1+(h-1)\tilde{g}(h), \quad
    \tilde{g}(h):=\int_0^\infty\frac{d\tilde{\phi}(y)}{1+hy}\,,\quad
    d\tilde{\phi}(y):=(y+1)d\eta\left(\frac{y}{y+1}\right),\notag \\
%    
     w(h)&=1-(h-1)\hat{g}(h),\quad
     \hat{g}(h):=\int_0^\infty\frac{d\ph(y)}{1+hy}\,, \quad
     d\ph(y):=(y+1)\left[-d\alpha\left(\frac{1}{y+1}\right)\right],\notag \\
%     
    \tilde{w}(h)&=1-(h-1)\check{g}(h),
      \quad \check{g}(h):=\int_0^\infty\frac{d\check{\phi}(y)}{1+hy}\,,\quad
      d\check{\phi}(y):=(y+1)\left[-d\tau\left(\frac{1}{y+1}\right)\right].
\end{align}
%
As $\mu$, $\eta$, $\alpha$, and $\tau$ are bounded positive measures on
$[0,1]$, $\phi$, $\tilde{\phi}$, $\ph$, and $\check{\phi}$ are positive
measures on $[0,\infty]$, and are also bounded if the supremum of the
support of these measures is finite. Equations
\eqref{eq:mh_Stieltjes_rep} are general formula holding for two
component stationary random media in lattice and continuum settings
\cite{Golden:PRL-3935}, and should be compared to equation
\eqref{Ising_Stieltjes_Fun} regarding the Ising model.       

By equation \eqref{eq:mh_Stieltjes_rep}, the moments $\phi_n$ of $\phi$
satisfy  
%
\begin{align}\label{eq:phi_moments}
  \phi_n=\int_0^\infty y^nd\phi(y)
    =\int_0^\infty y^n(y+1)d\mu\left(\frac{y}{y+1}\right)
    =\int_0^1\frac{\lambda^nd\mu(\lambda)}{(1-\lambda)^{n+1}}\,.
\end{align}
%
A partial fraction expansion of $\lambda^n/(1-\lambda)^{n+1}$ then shows that
%
\begin{align}\label{eq:phi_moments_F(s)}
  \frac{(-1)^n}{n!}\lim_{s\to1}\frac{\partial^nF(s)}{\partial s^n}=\int_0^1\frac{d\mu(\lambda)}{(1-\lambda)^{n+1}}
                                =\sum_{j=0}^n{n \choose j} \phi_j\,,
\end{align}
%
demonstrating that $\phi_n$ depends on $\int_0^1d\mu(\lambda)/(1-\lambda)^{n+1}$
(and) all the lower moments of $\phi$: $\phi_j$, $j=0,1,\ldots,n-1$. Moreover,
equation \eqref{eq:phi_moments} suggests that the moments $\phi_n$ become
singular as $\sup\{\Sigma_\mu\}\to1$. However
we will show that this is only true for the moments of order $j\geq1$,
and that $\lambda=1$ is a removable simple singularity under $\mu$. Theorem
\ref{thm:Herglotz_Decomp_Energy} of section
\ref{subsec:Spec_Decomp_Energy} further identifies the first two
moments of $\phi$ with energy components: 
%
\begin{align}\label{eq:phi_energy_relations}
  \phi_0=\lim_{s\to1}\frac{\langle\chi_1\vec{E}\cdot\vec{E}_0\rangle}{E_0^2},   \quad
  \phi_1=\lim_{s\to1}\frac{\langle E_f^2\rangle}{E_0^2}.
\end{align}
%

Similarly, the moments $\ph_n$ of $\ph$ satisfy
%
\begin{align}\label{eq:phi_hat_moments}
  \ph_n%&=\int_0^\infty y^nd\ph(y)
      %=\int_0^\infty y^n(y+1)d\alpha\left(\frac{1}{y+1}\right)
      &=\int_0^1\frac{\lambda^n[-d\alpha(1-\lambda)]}{(1-\lambda)^{n+1}}
      =\int_0^1\frac{(1-\lambda)^nd\alpha(\lambda)}{\lambda^{n+1}}
      =\sum_{j=0}^n(-1)^j {n \choose j} \int_0^1\frac{d\alpha(\lambda)}{\lambda^{n+1-j}}\notag\\
      &=\sum_{j=0}^n\frac{(-1)^{n+1}}{(n-j)!}{n \choose j}
             \lim_{s\to1}\frac{\partial^{n-j}G(s)}{\partial s^{n-j}}.
\end{align}
%
Equation \eqref{eq:phi_hat_moments} suggests, and we will show, that
the moments $\ph_n$ become singular as $\inf\{\Sigma_\alpha\}\to0$ for all
$n\geq0$. Theorem \ref{thm:Herglotz_Decomp_Energy} of section
\ref{subsec:Spec_Decomp_Energy} similarly identifies the first two
moments of $\ph$ with energy components. By the symmetries in
equations \eqref{eq:Herglotz_Funs_sed_LYRB} and
\eqref{eq:mh_Stieltjes_rep}, equations 
\eqref{eq:phi_moments}--\eqref{eq:phi_moments_F(s)} hold for
$\tilde{\phi}$ with $E(s)$ and $\eta$ in lieu of $F(s)$ and $\mu$, and equation
\eqref{eq:phi_hat_moments} holds for $\check{\phi}$ with $H(s)$ and $\tau$ in lieu
of $G(s)$ and $\alpha$. Furthermore, Theorem \ref{thm:Herglotz_Decomp_Energy}
identifies the first two moments of $\tilde{\phi}$ and $\check{\phi}$ with
energy components. In order to make connections to $F(s)$ and $G(s)$ in the
representation of equations \eqref{eq:phi_moments_F(s)} and
\eqref{eq:phi_hat_moments}, we have asumed that $F(s)$ and $G(s)$ may
be differentiated under the intergral sign with respect to $s$. This
is warrented by Lemma \ref{lem:h_diff_commutation} below. 

For percolation models such as the RRN
\cite{Stauffer-92,Torquato:RHM-02}, the connectedness of the system is
determined by the volume fraction $p$ of type two inclusions in an
otherwise homogeneous type one medium. The average cluster size of
these  inclusions grows as $p$ increases, and there is a critical
volume fraction $p_c$, $\;0<p_c<1$, called the \emph{percolation
  threshold}, where an infinite cluster of the inclusions first 
appears. Consider transport through a RRN \cite{Golden:PRL-3935} where 
bonds are assigned electrical conductivities $\sigma_2$ with probability
$p$, and $\sigma_1$ with probability $1-p$. As $h\to0$ ($\sigma_1\to0$ and
$0<|\sigma_2|<\infty$), the effective conductivity $\sigma^*(p,h):=\sigma_2\,m(p,h)$ and the
effective inverse conductivity
$[\sigma^{-1}]^*(p,h):=\sigma_2^{-1}\tilde{w}(p,h)$ undergo a
conductor/insulator critical transition. While, as $h\to0$ ($\sigma_2\to\infty$ and
$0<|\sigma_1|<\infty$), the effective conductivity $\sigma^*(p,h):=\sigma_1w(p,h)$ and
inverse effective conductivity
$[\sigma^{-1}]^*(p,h):=\sigma_1^{-1}\tilde{m}(p,h)$ undergo a 
conductor/superconductor critical transition:      
%
\begin{align}\label{eq:Cond-Insul_Crit_Beh_pc}
  &|\sigma^*(p,0)|:=|\sigma_2\,m(p,0)|=\left\{
    \begin{array}{ll}
      0, &       \text{for } p<p_c\\
      0<|\sigma_1|<|\sigma^*(p)|<|\sigma_2|, & \text{for } p>p_c
    \end{array}
    \right. ,
\\
  &|[\sigma^{-1}]^*(p,0)|:=|\sigma_2^{-1}\tilde{w}(p,0)|=\left\{
    \begin{array}{ll}
      \infty, &       \text{for } p<p_c\\
     |\sigma_2|^{-1}<|[\sigma^{-1}]^*(p)|<|\sigma_1|^{-1}, & \text{for } p>p_c
    \end{array}
    \right. ,\notag\\
\label{eq:Cond-SuperCond_Crit_Beh_pc}
  &|\sigma^*(p,0)|:=|\sigma_1w(p,0)|=\left\{
    \begin{array}{ll}
      0<|\sigma^*(p)|<\infty, &       \text{for } p<p_c\\
      \infty, & \text{for } p>p_c
    \end{array}
    \right. ,
\\
  &|[\sigma^{-1}]^*(p,0)|:=|\sigma_1^{-1}\tilde{m}(p,0)|=\left\{
    \begin{array}{ll}
      0<|[\sigma^{-1}]^*(p)|<\infty, &       \text{for } p<p_c\\
      0, & \text{for } p>p_c
    \end{array}
    \right. .\notag
  \end{align}
%

We will focus on the conductor/superconductor critical transition of
the effective conductivity $\sigma^*(p,h)=\sigma_1w(p,h)$ and the
conductor/insulator critical transition of the the effective
conductivity $\sigma^*(p,h)=\sigma_2\,m(p,h)$. It is clear from equations
\eqref{eq:mh_Stieltjes_rep} and 
\eqref{eq:Cond-Insul_Crit_Beh_pc}--\eqref{eq:Cond-SuperCond_Crit_Beh_pc},
that the corresponding results immediately generalize to
$[\sigma^{-1}]^*(p,h)=\sigma_2^{-1}\tilde{w}(p,h)$ and
$[\sigma^{-1}]^*(p,h)=\sigma_1^{-1}\tilde{m}(p,h)$, respectively, with $p\mapsto1-p$. 

The critical behavior of binary conductors is made more precise
through the definition of the following critical exponents. For
$h\in\mathbb{R}\cap\mathcal{U}$, as $h\to0$ the effective conductivity
$\sigma^*(p,h)=\sigma_2\,m(p,h)$ exhibits critical conductor/insulator behavior
near the percolation threshold $p_c$, $\sigma^*(p,0)\sim(p-p_c)^t$ as
$p\to p_c^+$, moreover at $p=p_c$,
$\sigma^*(p_c,h)\sim h^{1/\delta}$ as $h\to0$. We assume the existence of the
critical exponents $t$ and $\delta$, as well as $\gamma$, defined via a
conductive susceptibility $\chi(p,0):=\partial m(p,0)/\partial h\sim(p-p_c)^{-\gamma}$ as
$p\to p_c^+$. Furthermore, for $p>p_c$, we assume that there is a gap 
$\theta_\mu\sim(p-p_c)^\Delta$ in the support of $\mu$ around $h=0$ or $s=1$ which
collapses as $p\to p_c^+$, or that any spectrum in this region does not
affect power law behavior \cite{Golden:PRL-3935}. Therefore, for our
percolation models with $p>p_c$, the support of $\phi$ is contained in
the compact interval $[0,S(p)]\subset\subset\mathbb{R}^+$, where $S(p)\sim(p-p_c)^{-\Delta}$ as
$p\to p_c^+$. As the moments of $\phi$ become singular as $\theta_\mu\to0$ 
\eqref{eq:phi_moments}, we also assume that there exist critical
exponents $\gamma_n$ such that $\phi_n(p)\sim(p-p_c)^{-\gamma_n}$ as $p\to p_c^+$,
$n\geq0$. When $h\in\mathcal{U}$ such that $h_i\neq0$, we
assume the existence of critical exponents $t_r$, $\delta_r$, $t_i$ and
$\delta_i$ corresponding to $m_r(p,h):=\text{Re}(m(p,h))$ and
$m_i(p,h):=\text{Im}(m(p,h))$. The critical exponents, $\gamma_n$ and $\Delta$,
associated with the measure $\phi$ are independent of $h$ and are thus
unaffected. 

In summary:  
%
\begin{eqnarray}\label{eq:Crit_Exponents_mh}
  &m(p,0)\sim(p-p_c)^t,  &\text{as  } p \to p_c^+,\\
  &m_r(p,0)\sim(p-p_c)^{t_r},  &\text{as  } p \to p_c^+,\notag\\
  &m_i(p,0)\sim(p-p_c)^{t_i},  &\text{as  } p \to p_c^+,\notag\\
  &m(p_c,h)\sim h^{1/ \delta },  &\text{as } h \to 0, \notag\\
  &m_r(p_c,h)\sim h^{1/ \delta_r },  &\text{as } |h| \to 0, \notag\\
  &m_i(p_c,h)\sim h^{1/ \delta_i },  &\text{as } |h| \to 0, \notag\\
  &\chi(p,0)\sim(p-p_c)^{-\gamma},  &\text{as }  p\to p_c^+,\notag\\
  &\phi_n\sim(p-p_c)^{-\gamma_n},  &\text{as }  p\to p_c^+. \notag\\
  &\theta_\mu(p)\sim(p-p_c)^\Delta,  &\text{as }  p\to p_c^+,\notag\\
  &S(p)\sim(p-p_c)^{-\Delta},  &\text{as } p \to p_c^+.\notag
\end{eqnarray} 
%
We also assume the existence of critical exponents associated with the
left hand limit $p\to p_c^-$: $\gamma^\prime$, $\gamma^\prime_n$, and $\Delta^\prime$. The
conductivity critical exponent, $t$, is believed to be
\emph{universal} for lattices, depending only on dimension
\cite{Golden:PRL-3935}. The critical exponents $\gamma$, $\delta$, $\Delta$, and
$\gamma_n$ for transport are different from those defined in section
\ref{sec:Magnetic_Systems} for the Ising model.

For $h\in\mathbb{R}\cap\mathcal{U}$, as $h\to0$ the effective conductivity
$\sigma^*(p,h)=\sigma_1w(p,h)$ exhibits critical conductor/superconductor
behavior near $p_c$, $\sigma^*(p,0)\sim(p-p_c)^{-s}$ as $p\to p_c^-$, and at
$p=p_c$, $\sigma^*(p_c,h)\sim h^{-1/\dha}$ as $h\to0$, where the superconductor
critical exponent $s$ is not to be confused with the contrast
parameter. We assume the existence of the critical exponents $s$ and
$\dha$, as well as $\gh$, defined via a conductive susceptibility
$\hat{\chi}(p):=\partial w(p,0)/\partial h\sim(p-p_c)^{-\gh}$ as $p\to p_c^-$. Furthermore,
for $p<p_c$, we assume that there is a gap $\theta_\alpha\sim(p-p_c)^{\Dh^\prime}$ in the
support of $[-d\alpha(1-\lambda)]$ around $h=0$ or $s=1$ which collapses as
$p\to p_c^-$, so that the support of $\ph$ is contained in the compact
interval $[0,\hat{S}(p)]\subset\subset\mathbb{R}^+$, where
$\hat{S}(p)\sim(p-p_c)^{-\Delta}$ as $p\to p_c^+$. As the moments of $\ph$ become
singular as $\theta_\alpha\to0$ \eqref{eq:phi_hat_moments}, we also assume that
there exist critical exponents $\gh_n^\prime$ such that
$\ph_n(p)\sim(p-p_c)^{-\gh_n^\prime}$ as $p\to p_c^-$, $n\geq0$. When
$h\in\mathcal{U}$ such that $h_i\neq0$, we assume the existence of critical
exponents $s_r$, $s_i$, $\dha_r$, and $\dha_i$ corresponding to
$w_r(p,h):=\text{Re}(w(p,h))$ and $w_i(p,h):=\text{Im}(w(p,h))$. The
critical exponents, $\gamma_n^\prime$ and $\Delta^\prime$, associated with the measure
$\ph$ are independent of $h$ and are thus unaffected.

In summary:
%
\begin{eqnarray}\label{eq:Crit_Exponents_wh}
  &\hat{w}(p,0)\sim(p-p_c)^s,  &\text{as } p \to p_c^-,\\
  &\hat{w}_r(p,0)\sim(p-p_c)^{s_r}, &\text{as  } p \to p_c^-,\notag\\
  &\hat{w}_i(p,0)\sim(p-p_c)^{s_i}, &\text{as  } p \to p_c^-,\notag\\                     
  &\hat{w}(p_c,h)\sim h^{1/ \dha }, &\text{as } h \to 0, \notag\\
 &\hat{w}_r(p_c,h)\sim h^{1/ \dha_r },&\text{as } |h| \to 0, \notag\\
 &\hat{w}_i(p_c,h)\sim h^{1/ \dha_i }, &\text{as } |h| \to 0, \notag\\            
 &\hat{\chi}(p,0)\sim(p-p_c)^{-\gh^\prime}, &\text{as }  p\to p_c^-, \notag\\             
  &\ph_n\sim(p-p_c)^{-\gh_n^\prime}, &\text{as }  p\to p_c^-, \notag\\
  &\theta_\alpha(p)\sim(p-p_c)^{\Dh^\prime},  &\text{as }  p\to p_c^-,\notag\\
  &\hat{S}(p)\sim(p-p_c)^{-\Dh^\prime}, &\text{as } p \to p_c^-.\notag
\end{eqnarray} 
%
We also assume the existence of critical exponents associated with the
right hand limit $p\to p_c^+$: $\gh$, $\gh_n$, and $\Dh$. To be more precise,
when we assume the existence of a critical exponent, we assume the
existence, for example, of the following limit \cite{Baker-1990}:
% 
\begin{align}
  \gh^\prime:=\limsup_{p\to p_c^-}\left( \frac{-\ln \hat{\chi}(p,0)}{\ln(p-p_c)}  \right).
\end{align}
%

We now briefly discuss the gaps $\theta_\mu$ (for $p>p_c$)  and $\theta_\alpha$ (for
$p<p_c$). While, in the infinite volume limit, the spectra
actually extends all the way to $h=0$, the part close to $h=0$
corresponds to very large, but very rare connected regions of the
insulating (superconducting) phase, Lifshitz phenomenon, and is
believed to give exponentially small contributions to $\sigma^*$, and not
affect power law behavior \cite{Golden:PRL-3935}. Bruno
\cite{Bruno:PRSLA-353} has proven the existence of a spectral gap in
matrix/particle systems with polygonal inclusions, and studied how the
gap vanishes as the inclusions touch (like $p\to p_c$). In section
\ref{sec:Spectral_Gap} we proved the existence of spectral gaps for
finite lattice systems. Furthermore, in section
\ref{sec:Calc_Spec_Meas_Comp_Micro} we numerically demonstrated the
existence of spectral gaps for random resistor networks and correlated
their collapse with the percolation threshold. For the actual model we
expect behavior similar to the lattice case.
%
\subsection{Baker's Critical Theory for Transport in Binary Composite
  Media}
%
Baker's critical theory characterizes phase transitions via the
asymptotic behaviors of underlying Stieltjes functions, near a critical 
point. This powerfull method has been very successfull in the Ising
model, precisely characterizing spontaneous magnetization
\cite{Baker-1990}. We will show that this method has far reaching
utility in the characterization of phase transitions in transport,
exhibited by a wide variety of binary composite media.   
%
\begin{definition}  \label{def:stieltjes}
  A function f(z) is said to be a \emph{Stieltjes function} (or
  \emph{series of Stieltjes}), if 
  %
  \begin{align} \label{eq:stieltjes}
    f(z)=\int_0^\infty\frac{d\phi(u)}{1+uz}
    =\sum_{j =0}^\infty(-z)^j\int_0^\infty u^jd\phi(u)
    :=\sum_{j =0}^\infty(-z)^j\phi_j,
  \end{align}
  %
  where $\phi(u)$ is a bounded, non-decreasing function, taking on
  infinitely many values, and all the moments $\phi_j$ of $\phi$ are
  finite.  
\end{definition}
%
By hypothesis, for $p<p_c$ the measure $\ph$ is compactly supported,
hence bounded with bounded moments of all orders. Therefore the
function $\hat{g}(h):=\hat{g}(p,h)$ is a Stieltjes function for
$p<p_c$. For all $h\in\mathcal{U}$ the Stieltjes function $\hat{g}(p,h)$
is analytic and has a convergent series representation
\eqref{eq:stieltjes} for all $h\in\mathcal{U}$ such that
$|h|\hat{S}(p)<1$ \cite{Golden:PRL-3935,Golden:CMP-473}. Similarly for
$p>p_c$, $g(h):=g(p,h)$ is a Stieltjes function and is analytic
with a convergent series representation \eqref{eq:stieltjes} for all
$h\in\mathcal{U}$ such that $|h|S(p)<1$. The following theorem
characterizes Stieltjes functions \cite{Baker-1990}.  
% 
\begin{theorem} \label{thm:stieltjes_Characterization}
   Let $D(i,j)$ denote the following determinant
    \begin{align} \label{eq:Detf} 
     D(i,j) = \left|
                 \begin{matrix}
                   \phi_i&\phi_{i+1}&\cdots&\phi_{i+j}\\ 
                   \vdots&\vdots&\ddots&\vdots\\
                   \phi_{i+j}&\phi_{i+j+1}&\cdots&\phi_{i+2j}                            
                   \end{matrix}
              \right| ,    
   \end{align}
   where $D(i,j)=D(j,i)$. The $\phi_n$ form a series of Stieltjes if and
   only if $\mathcal{D}(i,j) \geq 0$ for all $i,j =0,1,2,\ldots$

 \end{theorem}
%
Baker's inequalities for the sequence $\gamma_n$ immediately follows from
Theorem \ref{thm:stieltjes_Characterization} and equations
\eqref{eq:Crit_Exponents_mh}--\eqref{eq:Crit_Exponents_wh}. Indeed,
$\phi_n\sim(p-p_c)^{-\gamma_n}$ and Theorem \ref{thm:stieltjes_Characterization}
with $i=n$ and $j=1$ imply, for $|p-p_c|\ll1$, that    
%
\begin{align}
  (p-p_c)^{-\gamma_n - \gamma_{n+2}}-(p-p_c)^{-2\gamma_{n+1}} &\geq  0
  \notag \\
%  
  \iff (p-p_c)^{-\gamma_n - \gamma_{n+2} + 2\gamma_{n+1} }&\geq1
  \notag \\
%  
  \iff-\gamma_n - \gamma_{n+2} + 2\gamma_{n+1} &\leq 0
  \notag\\
%
  \label{eq:CondBakerIneq_m}
  \iff   \gamma_{n+1}-2\gamma_n+\gamma_{n-1}&\geq  0.
\end{align}
% 
The sequence of inequalities \eqref{eq:CondBakerIneq_m} are
\emph{Baker's inequalities} for transport, corresponding to $m(p,h)$,
and they imply that the sequence $\gamma_n$ increases at least linearly
with $n$.  The symmetries in equations \eqref{eq:mh_Stieltjes_rep} and
\eqref{eq:Crit_Exponents_mh}--\eqref{eq:Crit_Exponents_wh} imply that
Baker's inequalities also hold for the sequences $\gamma_n^\prime$, $\gh_n$, and
$\gh_n^\prime$. 

The key results of this section are the two--parameter scaling
relations between the critical exponents in the conductor/insulator
system,
defined in equations \eqref{eq:Crit_Exponents_mh},
and that of the conductor/superconductor system.
defined in equations \eqref{eq:Crit_Exponents_wh}.
By equations  \eqref{eq:BM_measure_relationship_E} we know that the
measures $\mu$ and $\alpha$ are related, therefore the measures $\phi$ and $\ph$
are related. Moreover, by equations \eqref{eq:Herglotz_Energy_Rep_E}
and \eqref{eq:Herglotz_Energy_Rep_G} we know that $m(p,h)$ and
$w(p,h)$ are related, therefore the Stieltjes functions $g(p,h)$ and
$\hat{g}(p,h)$ are related. We therefore anticipate that these sets of
critical exponents 
%defined in equations \eqref{eq:Crit_Exponents_mh} and  
% \eqref{eq:Crit_Exponents_wh}
are also related. This is indeed the case, and the resultant
relationship between the insulation critical exponent $t$ and the
superconduction critical exponent $s$ is in agreement with the seminal
paper by A. L. Efros and B. I. Shklovskii \cite{Efros:PSSB-303}.
These results are summarized in Theorem \ref{thm:Crit_Theory_m_w}
below. 
%
% The key results of this section are the two--parameter scaling
% relations between the critical exponents defined in equations
% \eqref{eq:Crit_Exponents_mh}--\eqref{eq:Crit_Exponents_wh}. These
% results are summarized in Theorem \ref{thm:Crit_Theory_m_w} below.
%
%
\begin{theorem} \label{thm:Crit_Theory_m_w}
  Let $t$, $t_r$, $t_i$, $\delta$, $\delta_r$, $\delta_i$, $\gamma$, $\gamma_n$, $\Delta$, $\gamma_n^\prime$,
  and $\Delta^\prime$ be   defined as in equations \eqref{eq:Crit_Exponents_mh},
  and $s$, $s_r$, $s_i$, $\dha$, $\dha_r$, $\dha_i$, $\gh$, $\gh_n^\prime$,
  $\Dh^\prime$, $\gh_n$, and $\Dh$ be defined as in equations
  \eqref{eq:Crit_Exponents_wh}. Then the following scaling relations
  hold:
%  
  \begin{align*}   
   &1) \gamma_1=\gamma, \ \gamma_1^\prime=\gamma^\prime, \ \gh_1=\gh, \text{ and } \gh_1^\prime=\gh^\prime\\
   &2) \ \gamma_0^\prime=0, \ \gamma_0<0, \ \gamma_n^\prime>0 \text{ and } \gamma_n>0 \text{ for } n\geq1\\
   &3) \ \gh_n^\prime>0 \text{ for } n\geq0\\
   &4) \ \gamma_1=\gh_0 \text{ and } \Delta=\Dh\\
   &5) \ \gamma_1^\prime=\gh_0^\prime \text{ and } \Delta^\prime=\Dh^\prime \\
   &6) \ \gamma_n=\gamma+\Delta(n-1) \text{ for } n\geq1 \\
   &7) \ \gh_n^\prime=\gh_0^\prime+\Dh^\prime n=\gh+\Dh^\prime(n-1) \text{ for } n\geq0 \\
   &8) \ t=\Delta-\gamma \\
   &9) \ s=\gh_0^\prime=\gh-\Dh^\prime \\
   &10) \ \delta=\frac{\Delta}{\Delta-\gamma} \\
   &11) \ \dha\;^\prime=\frac{\Dh^\prime}{\gh_0^\prime}=\frac{\Dh^\prime}{\gh-\Dh^\prime} \\
   &12) \ t_r=t_i=t \\
   &13) \ s_r=s_i=s \\
   &14) \ \delta_r=\delta_i=\delta \text{ and } \dha_r=\dha_i=\dha \\
   &15) \text{ If } \Delta=\Delta^\prime \text{ and } \gamma=\gamma^\prime, \text{ then } t+s=\Delta \text{
     and }  \delta=1/(1-1/\dha\,^\prime)
  \end{align*}
%  
\end{theorem}
%

Theorem \ref{thm:Crit_Theory_m_w} will be proven via a sequence of
lemmas as we collect some important properties of $m(p,h)$, $g(p,h)$,
$w(p,h)$, and $\hat{g}(p,h)$, and how they are related. From equations
\eqref{eq:Herglotz_Energy_Rep_E} and \eqref{eq:Herglotz_Energy_Rep_G}
we have   
%
\begin{align}\label{eq:m_w_relation}
  m(p,h)=hw(p,h), \quad \forall \ p\in[0,1], \ h\in\mathcal{U}.
\end{align}
%
Using equations \eqref{eq:mh_Stieltjes_rep}, minor algebraic
manipulation in equation \eqref{eq:m_w_relation} implies that 
%
\begin{align}\label{eq:g_ghat_relation}
  g(p,h)+h\hat{g}(p,h)=1, \quad \forall \ p\in[0,1], \ h\in\mathcal{U}.
\end{align}
%
For all $h\in\mathcal{U}$ and $p\in[0,1]$, the functions $g(p,h)$ and
$\hat{g}(p,h)$ are analytic in $h$ \cite{Golden:CMP-473}, and
therefore have bounded $h$ derivatives of all orders
\cite{Rudin:87}. An inductive argument applied to equation
\eqref{eq:g_ghat_relation} yields  
%
\begin{align}\label{eq:Diff_g_ghat_relation}
  \frac{\partial^ng}{\partial h^n}+n\frac{\partial^{n-1}\hat{g}}{\partial h^{n-1}}+h\frac{\partial^n\hat{g}}{\partial h^n}=0,
  \quad \forall \ p\in[0,1], \ h\in\mathcal{U}, \ n\geq1.
\end{align}
%
In the complex quasi--static case, where $h\in\mathcal{U}$ such that
$h_i\neq0$, the complex representation of equation
\eqref{eq:Diff_g_ghat_relation} is        
%
\begin{align}\label{eq:Complex_Diff_g_ghat_relation}
  &\frac{\partial^ng_r}{\partial h^n}+n\frac{\partial^{n-1}\hat{g}_r}{\partial h^{n-1}}
  +h_r\frac{\partial^n\hat{g}_r}{\partial h^n}-h_i\frac{\partial^n\hat{g}_i}{\partial h^n}=0,
  \quad n\geq1, \\
%  
  &\frac{\partial^ng_i}{\partial h^n}+n\frac{\partial^{n-1}\hat{g}_i}{\partial h^{n-1}}
  +h_r\frac{\partial^n\hat{g}_i}{\partial h^n}+h_i\frac{\partial^n\hat{g}_r}{\partial h^n}=0,
  \quad n\geq1, \notag
\end{align}
%
which holds for all $p\in[0,1]$ and $h\in\mathcal{U}$, where we have set, for
$n\geq0$, 
%
\begin{align*}
  \frac{\partial^ng_r}{\partial h^n}:=\text{Re}\frac{\partial^ng}{\partial h^n}, \qquad
  \frac{\partial^ng_i}{\partial h^n}:=\text{Im}\frac{\partial^ng}{\partial h^n},
  \\
  \frac{\partial^n\hat{g}_r}{\partial h^n}:=\text{Re}\frac{\partial^n\hat{g}}{\partial h^n}, \qquad
  \frac{\partial^n\hat{g}_i}{\partial h^n}:=\text{Im}\frac{\partial^n\hat{g}}{\partial h^n}.
\end{align*}
Equations
\eqref{eq:m_w_relation}--\eqref{eq:Complex_Diff_g_ghat_relation} are
general formulas holding for two component stationary random media in
the lattice and continuum settings \cite{Golden:PRL-3935}.

The integral representation \eqref{eq:mh_Stieltjes_rep} of equations
\eqref{eq:g_ghat_relation}--\eqref{eq:Complex_Diff_g_ghat_relation}
follows from the formulas in the following lemma.
%---------------------------------------------------------------------------------
\begin{lemma}\label{lem:h_diff_commutation}  
  %
  For all $h\in\mathcal{U}$ and $p\in[0,1]$, the Stieltjes functions
  $g(p,h)$ and $\hat{g}(p,h)$ may be differentiated under the integral
  sign: 
  %
  \begin{align}\label{eq:Diff_g}
    %\frac{\partial^nF(s)}{\partial s^n}&=\frac{\partial^n}{\partial s^n}\int_0^1\frac{d\mu(\lambda)}{s-\lambda}
    %                 =(-1)^nn!\int_0^1\frac{d\mu(\lambda)}{(s-\lambda)^{n+1}}
    %    \iff\\
    \frac{\partial^ng(p,h)}{\partial h^n}&=\frac{\partial^n}{\partial h^n}\int_0^\infty\frac{d\phi(y)}{1+hy}
                     =(-1)^nn!\int_0^\infty\frac{y^nd\phi(y)}{(1+hy)^{n+1}}\sim\phi_n\,.
         \\
    %\frac{\partial^nG(s)}{\partial s^n}&=-\frac{\partial^n}{\partial s^n}\int_0^1\frac{d\alpha(1-\lambda)}{s-\lambda}
    %                 =(-1)^{n+1}n!\int_0^1\frac{d\alpha(1-\lambda)}{(s-\lambda)^{n+1}}
    %    \iff\notag\\
    \frac{\partial^n\hat{g}(p,h)}{\partial h^n}&=\frac{\partial^n}{\partial h^n}\int_0^\infty\frac{d\ph(y)}{1+hy}
                     =(-1)^nn!\int_0^\infty\frac{y^nd\ph(y)}{(1+hy)^{n+1}}\sim\ph_n,
           \notag           
  \end{align}
  %
  where $n\geq1$ and the asymptotics in equation \eqref{eq:Diff_g} hold when
  $0<|h|\ll1$ and $|p-p_c|\ll1$. Moreover, for all $h\in\mathcal{U}$ and
  $p\in[0,1]$, and all $j\leq n+1$,
  %
  \begin{align}\label{eq:Complex_Diff_g_bounds}
   \int_0^\infty\frac{y^{n+j}d\phi(y)}{|1+hy|^{2(n+1)}},
   \int_0^\infty\frac{y^{n+j}d\ph(y)}{|1+hy|^{2(n+1)}}<\infty.
  \end{align}
  %
\end{lemma}
%
\noindent \textbf{Proof}:
%
For general $p\in[0,1]$, the support of $\phi$ and $\ph$ are given by
$\Sigma_\phi:=[S_0(p),S(p)]$ and $\Sigma_{\ph}:=[\hat{S}_0(p),\hat{S}(p)]$,
respectively, and are defined in terms of the support of $\mu$ and $\alpha$,
respectively, below equation \eqref{eq:var_subs_Fs}. By hypothesis, the
extremum of these sets satisfy
$\lim_{p\to p_c^+}S_0(p)=\lim_{p\to p_c^-}\hat{S}_0(p)=0$ and
$\lim_{p\to p_c^+}S(p)=\lim_{p\to p_c^-}\hat{S}(p)=\infty$. For every
$h\in\mathcal{U}$, it is clear that there exist real, strictly positive
numbers $S_h$ and $\hat{S}_h$ such that 
%
\begin{align}\label{eq:S1_asymp}
  1\ll|h|S_h<\infty, \quad 1\ll|h|\hat{S}_h<\infty.
\end{align}
%
We show that the formulas in equation \eqref{eq:Diff_g} hold for all
$h\in\mathcal{U}$ and $n\in\mathbb{N}$ by showing that the function
$y^n/(1+hy)^{n+1}$ is a member of the sets $L^1(\phi)$ and $L^1(\ph)$ for
all $h\in\mathcal{U}$ and $n\in\mathbb{N}$ (see Theorem 2.27 in
\cite{Folland:95}).   

We first show that $y^n/(1+hy)^{n+1}\in L^1(\phi)$ for all $h\in\mathcal{U}$
and all $n\in\mathbb{N}$. Set $h\in\mathcal{U}$, $n\in\mathbb{N}$, and
$0\ll S_h<\infty$ satisfying \eqref{eq:S1_asymp}, and write
$\Sigma_\phi:=[S_0(p),S_h]\cup(S_h,S(P)]$. For all $p\in[0,1]$, equations
\eqref{eq:phi_moments_F(s)} and \eqref{eq:Cond-Insul_Crit_Beh_pc}
imply that $0<\lim_{h\to0}|m(p,h)|=1-\phi_0<1$, which implies that the mass
$\phi_0$ of $\phi$ is uniformly bounded for all $p\in[0,1]$. Therefore,
%
\begin{align}\label{eq:L1(phi)_bound_finite_set}
  \int_{S_0(p)}^{S_h}\left|\frac{y^n}{(1+hy)^{n+1}}\right|d\phi(y)\leq
  \frac{S_h^n\,\phi([S_0(p),S_h])}{|1+hS_0(p)|^{n+1}}<\infty,
\end{align}
%
where $\phi([S_0(p),S_h])$ is the \emph{bounded} $\phi$ measure of the set
$[S_0(p),S_h]$. If $S(p)<\infty$ we simply set $S_h\equiv S(p)$ and we are
done. Otherwise set $S(p)\equiv\infty$. In terms of the support of $\mu$, we have
$\lambda_1(p):=S(p)/(1+S(p))\equiv1$ and $\lambda_h:=S_h/(1+S_h)\gg0$. Equations
\eqref{eq:mh_Stieltjes_rep} and \eqref{eq:S1_asymp} imply
%
\begin{align}\label{eq:L1(phi)_bound_infinite_set}
   \int_{S_h}^{S(p)}\left|\frac{y^n}{(1+hy)^{n+1}}\right|d\phi(y)
      &\sim\frac{1}{|h|^{n+1}}\int_{S_h}^{S(p)}\frac{d\phi(y)}{y}
      =\frac{1}{|h|^{n+1}}\int_{S_h}^{S(p)}
                 \frac{1+y}{y}d\mu\left(\frac{y}{1+y}\right)\notag\\
      &=\frac{1}{|h|^{n+1}}\int_{\lambda_h}^{\lambda_1(p)}\frac{d\mu(\lambda)}{\lambda}
      \leq \frac{1}{|h|^{n+1}}\,\frac{\mu_0}{\lambda_h}<\infty
\end{align}
%
for all $h\in\mathcal{U}$ and all $n\in\mathbb{N}$, where
$\mu_0=1-p$. Therefore, $y^n/(1+hy)^{n+1}\in L^1(\phi)$ for all
$h\in\mathcal{U}$ and all $n\geq0$. 

We now show that $y^n/(1+hy)^{n+1}\in L^1(\ph)$ for all $h\in\mathcal{U}$
and all $n\geq0$. Set $h\in\mathcal{U}$ and $0\ll\hat{S}_h<\infty$ satisfying
\eqref{eq:S1_asymp}, and write
$\Sigma_{\ph}:=[\hat{S}_0(p),\hat{S}_h]\cup(\hat{S}_h,\hat{S}(P)]$. We have
%
\begin{align}\label{eq:L1(ph)_bound_finite_set}
  \int_{\hat{S}_0(p)}^{\hat{S}_h}\left|\frac{y^n}{(1+hy)^{n+1}}\right|d\phi(y)\leq
  \frac{\hat{S}_h^n\,\ph([\hat{S}_0(p),\hat{S}_h])}{|1+h\hat{S}_0(p)|^{n+1}}<\infty.
\end{align}
%
The boundedness of equation \eqref{eq:L1(ph)_bound_finite_set} follows
from equations \eqref{eq:var_subs_Fs}--\eqref{eq:mh_Stieltjes_rep},
showing that the $\ph$ measure of the interval
$[\hat{S}_0(p),\hat{S}_h]$ is bounded. More specifically, in terms of
the support of $\alpha$, we have
$\hat{S}_0(p)=(1-\hat{\lambda}_1(p))/\hat{\lambda}_1(p) 
\iff\hat{\lambda}_1(p)=1-\hat{S}_0(p)/(1+\hat{S}_0(p))$ and
$\hat{S}_h:=(1-\hat{\lambda}_h)/\hat{\lambda}_h\gg0
\iff\hat{\lambda}_h=1-\hat{S}_h/(1+\hat{S}_h)\ll1$. Then equation
\eqref{eq:var_subs_Fs}--\eqref{eq:mh_Stieltjes_rep} imply that   
%
\begin{align}\label{eq:ph_measure_of_finite_set_bound}
  \ph([\hat{S}_0(p),\hat{S}_h])&=\int_{\hat{S}_0(p)}^{\hat{S}_h}d\ph(y)
         =\int_{\hat{S}_0(p)}^{\hat{S}_h}(1+y)\left[-d\alpha\left(\frac{1}{1+y}\right)\right]
         =\int_{1-\hat{\lambda}_1(p)}^{1-\hat{\lambda}_h}\frac{[-d\alpha(1-\lambda)]}{1-\lambda}\notag\\
         &=\int_{\hat{\lambda}_h}^{\hat{\lambda}_1(p)}\frac{d\alpha(\lambda)}{1-\lambda}
         \leq\frac{\alpha_0}{1-\hat{\lambda}_h}<\infty,
\end{align}
%
where $\alpha_0=p$. If $\hat{S}(p)<\infty$ we simply set $\hat{S}_h\equiv S(p)$ and we
are done. Otherwise set $\hat{S}(p)\equiv\infty$, where
$\hat{S}(p)=(1-\hat{\lambda}_0(p))/\hat{\lambda}_0(p)
\iff\hat{\lambda}_0(p)=1-\hat{S}(p)/(1+\hat{S}(p))\equiv0$. Therefore, similar to
equation \eqref{eq:L1(phi)_bound_infinite_set}, we have   
%
\begin{align}\label{eq:L1(ph)_bound_infinite_set}
   \int_{\hat{S}_h}^{\hat{S}(p)}\left|\frac{y^n}{(1+hy)^{n+1}}\right|d\ph(y)
      %&\sim\frac{1}{|h|^{n+1}}\int_{\hat{S}_h}^{\hat{S}(p)}
      %           \frac{1+y}{y}d\alpha\left(\frac{1}{1+y}\right)
      %            \notag\\
      &\sim\frac{1}{|h|^{n+1}}\int_{1-\hat{\lambda}_h}^{1-\hat{\lambda}_0(p)}\frac{[-d\alpha(1-\lambda)]}{\lambda}        
      =\frac{1}{|h|^{n+1}}\int_{\hat{\lambda}_0(p)}^{\hat{\lambda}_h}\frac{d\alpha(\lambda)}{1-\lambda}
      \notag\\
      &\leq \frac{1}{|h|^{n+1}}\frac{\alpha_0}{1-\hat{\lambda}_h}<\infty.
\end{align}
%
Therefore, $y^n/(1+hy)^{n+1}\in L^1(\ph)$ for all $n\geq0$ and all
$h\in\mathcal{U}$.

The asymptotic behaviors in equation \eqref{eq:Diff_g} follow from
equations \eqref{eq:phi_moments}--\eqref{eq:phi_moments_F(s)}, Baker's
inequalities \eqref{eq:CondBakerIneq_m}, and equation
\eqref{eq:Diff_mh_Fs} ($g(p,h)=sF(p,s)$ and $\hat{g}(p,h)=-s\,G(p,s)$)
which implies that 
%
\begin{align}\label{eq:Diff_mh_Fs}
  \lim_{h\to0}\frac{\partial^ng(p,h)}{\partial h^n}
         =\sum_{j=0}^nc_j\lim_{s\to1}\frac{\partial^jF(p,s)}{\partial s^j}\sim\phi_n\,,\\
  \lim_{h\to0}\frac{\partial^n\hat{g}(p,h)}{\partial h^n}
         =\sum_{j=0}^nb_j\lim_{s\to1}\frac{\partial^jG(p,s)}{\partial s^j}\sim\ph_n\,,  \notag     
\end{align}
where $c_j,b_j\in\mathbb{Z}$.

The proof that the function $y^{n+j}/|1+hy|^{2(n+1)}$ is in the sets
$L^1(\phi)$ and $L^1(\ph)$. Is very similar to the proof of equation
\eqref{eq:Diff_g}. As before, splitting up the supports $\Sigma_\phi$ and
$\Sigma_{\ph}$ into unions bounded and unbounded intervals, the bounds
analogous to equations \eqref{eq:L1(phi)_bound_finite_set} and
\eqref{eq:L1(ph)_bound_finite_set} follow as before. Noting that
$(1+y)/y^i=(1-\lambda)^{i-1}/\lambda^i$ one may easily find the following bounds
using the same procedure as in equations
\eqref{eq:L1(phi)_bound_infinite_set} and
\eqref{eq:L1(ph)_bound_infinite_set}:
%
\begin{align}
  \int_0^\infty\frac{y^{n+j}d\phi(y)}{|1+hy|^{2(n+1)}}
     &\leq\frac{\mu_0}{|h|^{2(n+1)}}\,\frac{(1-\lambda_h)^{n+1-j}}{\lambda_h^{n+2-j}}\\
  \int_0^\infty\frac{y^{n+j}d\ph(y)}{|1+hy|^{2(n+1)}}
     &\leq\frac{\alpha_0}{|h|^{2(n+1)}}\,\frac{(\lambda_h)^{n+1-j}}{1-\lambda_h^{n+2-j}},  
\end{align}
%
for all $h\in\mathcal{U}$ and $j\leq n+1$ $\Box$.
%-----------------------------------------------------------------------

Equations \eqref{eq:Diff_g_ghat_relation} and \eqref{eq:Diff_g} imply
that   
%
\begin{align}\label{eq:Diff_g_ghat_relation_Integral}
  \int_0^{S(p)}\frac{y^nd\phi(y)}{(1+hy)^{n+1}}=\int_0^{\hat{S}(p)}\frac{y^{n-1}d\ph(y)}{(1+hy)^n} 
                                -h \int_0^{\hat{S}(p)}\frac{y^nd\ph(y)}{(1+hy)^{n+1}}
  \,,                              
\end{align}
%
which holds for all $n\geq1$, $p\in[0,1]$, and $h\in\mathcal{U}$. The
integral representations of equations
\eqref{eq:Complex_Diff_g_ghat_relation} may be obtained by equation
\eqref{eq:Diff_g} as follows:  
%
\begin{align}\label{eq:Complex_Diff_g}
  \frac{\partial^ng}{\partial h^n}
   &=(-1)^nn!\int_0^{S(p)}\frac{y^nd\phi(y)}{|1+hy|^{2(n+1)}}(1+\bar{h}y)^{n+1}\\
   &=(-1)^nn!\sum_{j=0}^{n+1}{n+1 \choose j}\bar{h}^j
                 \int_0^{S(p)}\frac{y^{n+j}d\phi(y)}{|1+hy|^{2(n+1)}}\,.
 \notag
\end{align}
%

In Lemma \ref{lem:h_diff_commutation} we showed that the function
$y^n/(1+hy)^{n+1}\in L^1(\phi),L^1(\ph)$ for all $n\geq0$, $h\in\mathcal{U}$, and
$p\in[0,1]$. Therefore, both sides of equation
\eqref{eq:Diff_g_ghat_relation_Integral} are bounded for all
$n\geq0$, $h\in\mathcal{U}$, and $p\in[0,1]$. However, on can easily see that the
bounds given in Lemma \ref{lem:h_diff_commutation} are violated as
$h\to0$. We now show that equation
\eqref{eq:Diff_g_ghat_relation_Integral} is defined as $h\to0$.

Let $h$ By the proof of Lemma \ref{lem:h_diff_commutation} we have
$y^i/(1+hy)^j\in L^1(\ph)$ for all $j>i$ and $h\in\mathcal{U}$. Indeed, the
boundedness of the $\ph$ integral of the modulus of this function over
the compact interval $[0,S_1]$, where $S_1<\infty$ is arbitrary and fixed,
directly follows from equation
\eqref{eq:L1(ph)_bound_finite_set}  
%
%----------------------------------------------------------------
\begin{lemma}\label{lem:nonzero_gamma1_etc}
  $\gamma_1=\gamma$, $\gamma_1^\prime=\gamma^\prime$, $\gh_1=\gh$, and $\gh_1^\prime=\gh^\prime$
\end{lemma}
%
\noindent \textbf{Proof}:
%
Set $0<p-p_c\ll1$, by equation \eqref{eq:mh_Stieltjes_rep}
$(g(h)=sF(s))$, \eqref{eq:phi_moments_F(s)}, and
\eqref{eq:Crit_Exponents_mh} we have    
%
\begin{align}\label{eq:gamma1_gamma}
  (p-p_c)^{-\gamma}\sim\chi(p,0)
          :=\frac{\partial m(p,0)}{\partial h}
          =\lim_{s\to1}\left(F(p,s)+\frac{\partial F(p,s)}{\partial s}\right)
          =-\phi_1\sim(p-p_c)^{-\gamma_1},
\end{align}
%
hence $\gamma_1=\gamma$. Similarly for $0<p_c-p\ll1$, we have $\gamma_1^\prime=\gamma^\prime$. By
equation \eqref{eq:gamma1_gamma}, and the symmetries between $m$ and
$w$ \eqref{eq:mh_Stieltjes_rep} and the critical exponent definitions 
\eqref{eq:Crit_Exponents_mh}--\eqref{eq:Crit_Exponents_wh}, we also
have that $\gh_1=\gh$ and $\gh_1^\prime=\gh^\prime$ $\Box$.   
%----------------------------------------------------------------

Equation \eqref{eq:m_w_relation} is consistent with, and provides a
link between equations \eqref{eq:Cond-Insul_Crit_Beh_pc} and
\eqref{eq:Cond-SuperCond_Crit_Beh_pc}. We will see that the
fundamental asymmetry  between $m(p,h)$ and $w(p,h)$ ($\gamma_0^\prime=0$ and
$\gh_0^\prime>0$), given by Theorem \ref{thm:Crit_Theory_m_w}.2-3, is a
direct and essential consequence of equation \eqref{eq:m_w_relation},
and has deep and far reaching implications.      
%
%----------------------------------------------------------------
\begin{lemma}\label{lem:zero_gamma0}
  %
  Let the sequences $\gamma_n$ and $\gamma_n^\prime$, $n\geq0$, be defined as in
  equation \eqref{eq:Crit_Exponents_mh}. Then
  %
  \begin{align*}
    &1) \quad \gamma_0^\prime=0, \ \gamma_0<0, \ \gamma_n^\prime>0,   \text{ and } \ \gamma_n>0 \
        \text{ for } \ n\geq1 \\
    &2) \quad 0<\lim_{h\to0}\langle\chi_1\vec{E}\cdot\vec{E}_0\rangle/E_0^2<1 \
         \text{ for all } \ p\in[0,1]
  \end{align*}
  %
\end{lemma}
%
\noindent \textbf{Proof}:
%
By equation \eqref{eq:Cond-SuperCond_Crit_Beh_pc} $|w(p,0)|$ is  
bounded for all $p<p_c$. Thus for all $p<p_c$, equations
\eqref{eq:phi_moments_F(s)}, \eqref{eq:Crit_Exponents_mh}, 
and \eqref{eq:m_w_relation} imply that
%
\begin{align*}
  0=\lim_{h\to0}hw(p,h)=\lim_{h\to0}m(p,h)=\lim_{s\to1}(1-F(p,s))=1-\phi_0(p)\sim1-(p_c-p)^{-\gamma_0^\prime},
\end{align*}
%
for $0<p_c-p\ll1$, which is consistent with equation
\eqref{eq:Cond-Insul_Crit_Beh_pc}. Therefore, $\gamma_0^\prime=0$ and $\phi$ is a
probability measure for all $p<p_c$. The strict positivity of the
$\gamma_n^\prime$, for $n\geq1$, follows from Baker's inequalities
\eqref{eq:CondBakerIneq_m}. Therefore, from equation
\eqref{eq:gamma1_gamma} we have that
%
\begin{align}\label{eq:div_phi1}
  \infty=\lim_{p\to p_c^-}\phi_1(p)=-\lim_{p\to p_c^-}\frac{\partial m(p,0)}{\partial h}.
\end{align}
%
For $p>p_c$, equations \eqref{eq:phi_moments_F(s)} and
\eqref{eq:Cond-Insul_Crit_Beh_pc} imply that
$0<\lim_{h\to0}|m(p,h)|=1-\phi_0<1$. Therefore, $(p-p_c)^{-\gamma_0}\sim\phi_0<1$ for
all $0<p-p_c\ll1$, hence $\gamma_0<0$. The strict positivity of $\gamma_1$ follows
from equation \eqref{eq:div_phi1}, and the strict positivity of the
$\gamma_n$ for $n\geq2$ follows from Baker's inequalities
\eqref{eq:CondBakerIneq_m}. Equation \eqref{eq:phi_energy_relations}
and the inequality $0<\lim_{h\to0}|m(p,h)|=1-\phi_0<1$ implies that
$0<\lim_{h\to0}\langle\chi_1\vec{E}\cdot\vec{E}_0\rangle/E_0^2<1$ for all $p\in[0,1]$ $\Box$.    
%    
%------------------------------------------------------
%
%-------------------------------------------------------
\begin{lemma}\label{lem:nonzero_gh_n}
  %
  Let the sequence $\gh_n^\prime$, $n\geq0$, be defined as in equation
  \eqref{eq:Crit_Exponents_wh}. Then
  %
  \begin{align*}
  &1) \quad \gh_n^\prime>0 \ \text{ for all } \ n\geq0\\
  &2) \quad \lim_{h\to0}\langle E_f^2\rangle=\infty \ \text{ for all } \ p>p_c.
  \end{align*}
  %
\end{lemma}
%
\noindent \textbf{Proof}:
%
By equation \eqref{eq:Cond-Insul_Crit_Beh_pc} we have
$0<\lim_{h\to0}|m(p,h)|<1$, for all $p>p_c$. Therefore equation
\eqref{eq:m_w_relation} implies that
$\lim_{h\to0}w(p,h)=\lim_{h\to0}m(p,h)/h=\infty$, for all $p>p_c$, which is
consistent with equation
\eqref{eq:Cond-SuperCond_Crit_Beh_pc}. More specifically, equations
\eqref{eq:Cond-Insul_Crit_Beh_pc} and \eqref{eq:m_w_relation} imply
that $0<\lim_{h\to0}|m(p,h)|=\lim_{h\to0}|hw(p,h)|:=L(p)<1$, where
$\lim_{p\to p_c^+}L(p)=0$. Therefore, by 
equation \eqref{eq:mh_Stieltjes_rep}, we have
%
\begin{align}\label{eq:Divergence_Rate_w(p,h)}
  &\lim_{h\to0}|hw(p,h)|=\lim_{h\to0}|h\hat{g}(p,h)|\in(0,1), 
                        \text{ for all } p>p_c, 
 \\
  &\lim_{h\to0}|hw(p,h)|=\lim_{h\to0}|h\hat{g}(p_c,h)|=0
         \text{ for all } p<p_c . \notag                                       
\end{align}
%
As will be shown below, equation \eqref{eq:Divergence_Rate_w(p,h)} has
very important consequences. By equations \eqref{eq:phi_hat_moments},
\eqref{eq:Cond-SuperCond_Crit_Beh_pc}, and
\eqref{eq:Crit_Exponents_wh} we have,
%
\begin{align*}
  \infty=\lim_{p\to p_c^-}\lim_{h\to0}w(p,h)
   =\lim_{p\to p_c^-}\lim_{s\to1}(1-G(p,s))
   =1+\lim_{p\to p_c^-}\ph_0(p)
   \sim1+\lim_{p\to p_c^-}(p_c-p)^{-\gh_0^\prime},
\end{align*}
%
hence $\gh_0^\prime>0$. Baker's inequalities \eqref{eq:CondBakerIneq_m}
then imply that $\gh_n^\prime>0$ for all $n\geq0$. Equation
\eqref{eq:phi_energy_relations} and  $\gh_0^\prime>0$ implies that 
%
\begin{align*}
  \lim_{h\to0}\langle E_f^2\rangle=\infty \quad \forall \ p>p_c, \quad \Box.
\end{align*}  
%   
%----------------------------------------------------------------------

The asymptotic behavior of $\hat{g}(p,h)$ in equation 
\eqref{eq:Diff_g} and Lemma \ref{lem:nonzero_gh_n} motivates the
following fundamental homogenization assumption of this section
\cite{Baker-1990}:   
%
\begin{remark}\label{rem:homogenization_w}
Near the critical point $(p,h)=(p_c,0)$, the asymptotic behavior of
the Stieltjes function $\hat{g}(p,h)$ is determined primarily by the
mass $\ph_0(p)$ of the measure $\ph$ and the rate of collapse of the
spectral gap $\theta_\alpha$.  
\end{remark}
%
\noindent By remark \ref{rem:homogenization_w}, and in light of Lemmas
\ref{lem:nonzero_gamma1_etc}--\ref{lem:nonzero_gh_n}, we make the
following changes in variables 
%
\begin{align}\label{eq:variable_change_w}
  &\qh:=y(p_c-p)^{\Dh^\prime}, \quad \hat{Q}(p):=\hat{S}(p)(p_c-p)^{\Dh^\prime},
      %\quad x:=h(p-p_c)^{\Dh^\prime},\\
      \quad d\hat{\pi}(\qh):=(p_c-p)^{\gh_0^\prime} \;d\ph(y),
  \\
  \label{eq:variable_change_m}
   &q:=y(p-p_c)^\Delta, \quad Q(p):=S(p)(p-p_c)^\Delta,
      %\quad x:=h(p-p_c)^{\Dh},\\
      \quad d\pi(q):=(p-p_c)^\gamma \,y\,d\ph(y), 
\end{align}
%
so that $\hat{Q}(p),Q(p)\sim1$ and the masses $\hat{\pi}_0$ and $\pi_0$ of
the measures $\hat{\pi}$ and $\pi$, respectively, satisfy $\hat{\pi}_0,\pi_0\sim1$
as $p\to p_c$. Equations
\eqref{eq:variable_change_w}--\eqref{eq:variable_change_m} define the   
following scaling functions $G_{n-1}(x)$, $\hat{G}_n(\xh)$, $\mathcal{G}_{n-1,j}(x)$,
and $\hat{\mathcal{G}}_{n,j}(\xh)$ as follows.

For $h\in\mathcal{U}\cap\mathbb{R}$, equation \eqref{eq:Diff_g} of Lemma
\ref{lem:h_diff_commutation} and equations
\eqref{eq:variable_change_w}--\eqref{eq:variable_change_m} imply, for 
$n\geq0$, that       
%
\begin{align}\label{eq:Scaling_fun_Def}
  &\frac{\partial^ng}{\partial h^n}\propto(p-p_c)^{-(\gamma+\Delta(n-1))} G_{n-1}(x),
      %&=\int_0^{S(p)}\frac{y^n d\phi(y)}{(1+hy)^{n+1}}
      %=(p_c-p)^{-(\gamma+\Delta n)}\int_0^{Q(p)}
      %   \frac{q^n d\pi(q)}{(1+xq)^{(n+1)}}\\
%     
&&
  \frac{\partial^n\hat{g}}{\partial h^n}\propto(p_c-p)^{-(\gh_0^\prime+\Dh^\prime n)} \hat{G}_n(\xh), 
      %&=\int_0^{\hat{S}(p)}\frac{y^nd\ph(y)}{(1+hy)^{n+1}}
      %=(p_c-p)^{-(\gh_0^\prime+\Dh^\prime n)}\int_0^{\hat{Q}(p)}
      %   \frac{\qh^nd\hat{\pi}(\qh)}{(1+xq)^{(n+1)}}\\
\\ 
  &G_{n-1}(x):=\int_0^{Q(p)}\frac{q^{n-1}d\pi(q)}{(1+xq)^{n+1}},
&&
  \hat{G}_n(\xh):=\int_0^{\hat{Q}(p)}\frac{\qh^nd\hat{\pi}(\qh)}{(1+\xh \qh)^{n+1}},
\notag\\  
  &x:=h(p-p_c)^{-\Delta}, \quad 0<p-p_c\ll1,
  &&
  \xh:=h(p_c-p)^{-\Dh^\prime}, \quad 0<p_c-p\ll1. \notag
\end{align}
%
Analogous formulas are defined for the opposite limits, involving
$\Dh$, $\gh_0$, $\Delta^\prime$, and $\gamma^\prime$. 

For $h\in\mathcal{U}$ such that $h_i\neq0$, we analogously define scaling
functions $\mathcal{R}_{n-1}(x)$, $\mathcal{I}_{n-1}(x)$,
$\hat{\mathcal{R}}_{n}(\xh)$, and $\hat{\mathcal{I}}_{n}(\xh)$ as
follows. Using equations \eqref{eq:Complex_Diff_g} (which follows from
Remark \ref{lem:h_diff_commutation}) and
\eqref{eq:variable_change_w}--\eqref{eq:variable_change_m} we have,
for $0<p-p_c\ll1$,  
%
\begin{align}\label{eq:Complex_Scaling_fun_Def}
\frac{\partial^ng}{\partial h^n}   
   &=(-1)^nn!\sum_{j=0}^{n+1}{n+1 \choose j}\bar{h}^j
                 \int_0^{S(p)}\frac{y^{n+j}d\phi(y)}{|1+hy|^{2(n+1)}}\\
   &:=(-1)^nn!\sum_{j=0}^{n+1}{n+1 \choose j}[\bar{x}(p-p_c)^\Delta]^j
                 (p-p_c)^{-(\gamma+\Delta(n-1+j))}\mathcal{G}_{n-1,j}(x)\notag\\
   &:=(-1)^nn!(p-p_c)^{-(\gamma+\Delta(n-1))}\mathcal{K}_{n-1}(x)\notag\\
   &:=(-1)^nn!(p-p_c)^{-(\gamma+\Delta(n-1))}
      \left[\mathcal{R}_{n-1}(x)+\I\,\mathcal{I}_{n-1}(x)\right],
   \text{ and similarly}, \notag\\
\frac{\partial^n\hat{g}}{\partial h^n}
     &:=(-1)^nn!(p-p_c)^{-(\gh_0+\Dh n)}
       \left[\hat{\mathcal{R}}_{n}(\xh)+\I\,\hat{\mathcal{I}}_{n}(\xh)\right],\notag
\end{align}
%
where $x$ and $\xh$ are defined in equation \eqref{eq:Scaling_fun_Def}
and  
%
\begin{align}\label{eq:Complex_Scaling_fun_Def_Integrals}
 &\mathcal{G}_{n-1,j}(x):=\int_0^{Q(p)}\frac{q^{n-1+j}d\pi(q)}{|1+xq|^{2(n+1)}}
 &&
 &\hat{\mathcal{G}}_{n,j}(\xh):=\int_0^{\hat{Q}(p)}\frac{\qh^{n+j}d\hat{\pi}(\qh)}{|1+\xh\qh|^{2(n+1)}}
 \\
 &\mathcal{K}_{n-1}(x):=\sum_{j=0}^{n+1}{n+1 \choose j}\bar{x}^j
                       \mathcal{G}_{n-1,j}(x)
 &&
 &\hat{\mathcal{K}}_n(\xh):=\sum_{j=0}^{n+1}{n+1 \choose j}\bar{\xh}^j
                       \hat{\mathcal{G}}_{n,j}(\xh)
 \notag\\
 &\mathcal{R}_{n-1}(x):=\text{Re}(\mathcal{K}_{n-1}(x)),
 &&
 &\hat{\mathcal{R}}_n(\xh):=\text{Re}(\hat{\mathcal{K}}_n(\xh)),
   \notag\\   
 &\mathcal{I}_{n-1}(\xh):=\text{Im}(\mathcal{K}_{n-1}(x)),
 &&
 &\hat{\mathcal{I}}_n(\xh):=\text{Im}(\hat{\mathcal{K}}_n(\xh)).
 \notag
\end{align}
%
Analogous formulas are defined for the opposite limit, $0<p_c-p\ll1$
involving $\Dh^\prime$, $\gh^\prime_0$, $\Delta^\prime$, and $\gamma^\prime$. 

From equations \eqref{eq:Herglotz_Inneq}--\eqref{eq:Herglotz_NonZero}
we have, for $h\in\mathcal{U}$, $p\in[0,1]$, and $n\geq0$,
%
\begin{align}\label{eq:Non-negative_Gn_Ghn}
   G_{n-1}(x)&>0, \qquad
  \mathcal{G}_{n-1,j}(x)>0,\\
%
  \hat{G}_n(\xh)&>0, \qquad
  \hat{\mathcal{G}}_{n,j}(\xh)>0. \notag
\end{align}
%
We asume that, for all $p<p_c$, $w(p,0)$ is analytic, and for all
$p>p_c$, $m(p,0)$ is analytic (or that $h$ derivatives of $g(p,h)$ and
$\hat{g}(p,h)$ of all orders are bounded at $h=0$ for $p>p_c$ and
$p<p_c$ respectively
(\cite{Golden:CMP-473,Golden:CMP-467,Golden:SIAM89} and refrences 
therein)). Therefore we have, for $n\geq0$,   
% 
\begin{align}\label{eq:Bounded_Gn_h}
  &\lim_{h\to0}G_{n-1}(x)<\infty, \qquad
  \lim_{h\to0}\mathcal{G}_{n-1,j}(x)<\infty,  \quad
  p>p_c,\\
%
  &\lim_{h\to0}\hat{G}_n(\xh)<\infty, \qquad
  \lim_{h\to0}\hat{\mathcal{G}}_{n,j}(\xh)<\infty,  \quad
  p<p_c. \notag
\end{align}
%
%-----------------------------------------------------------------------------------
 \begin{lemma}\label{lem:asymp_Scaling_funs_x_to_0_p>pc}
   Let $\hat{G}_n(\xh)$, $G_{n-1}(x)$, and the associated critical
   exponents be defined as in equation \eqref{eq:Scaling_fun_Def}, for
   $p>p_c$. Then  
   %
     \begin{itemize}
    \item[1)] $G_{n-1}(x)\sim1$ as $x\to0$ $(h\to0$ and $0<p-p_c\ll1)$ for all $n\geq1$
    \item[2)] $[\hat{G}_{n-1}(\xh)-\xh\hat{G}_n(\xh)]\sim1
      %\iff \xh^n\hat{G}_n(\xh)\sim\hat{G}_0(\xh)
      $ as $\xh\to0$ $(h\to0$ 
      and $0<p-p_c\ll1)$ for all $n\geq1$  
    \item[3)] $\gamma=\gh_0$  
    \item[4)] $\Delta=\Dh$    
     \end{itemize}
   %
 \end{lemma}
%
\noindent \textbf{Proof}:
%
Let $h\in\mathcal{U}\cap\mathbb{R}$, so that $g(p,h)$ and $\hat{g}(p,h)$ are
real analytic \cite{Golden:CMP-473}, and $p>p_c$ so that, by equation
\eqref{eq:Bounded_Gn_h}, all $h$ derivatives of $g(p,h)$ are bounded
for $h=0$. Therefore, equations
\eqref{eq:Diff_g_ghat_relation_Integral},    
\eqref{eq:Scaling_fun_Def}, and
\eqref{eq:Non-negative_Gn_Ghn}--\eqref{eq:Bounded_Gn_h} imply that,
for all $n\geq1$, $0<p-p_c\ll1$, and $0<h\ll1$,   
%
\begin{align}\label{eq:Matching_Condition_Gn_Gnhat_p>pc}
  (0,\infty)\ni(p-p_c)^{-(\gamma+\Delta(n-1))}G_{n-1}(x)
       =(p-p_c)^{-(\gh+\Dh(n-1))}[\hat{G}_{n-1}(\xh)-\xh\hat{G}_n(\xh)].
\end{align}
%
Equations \eqref{eq:Non-negative_Gn_Ghn}--\eqref{eq:Bounded_Gn_h}
imply that $G_{n-1}(x)\sim1$ as $x\to0$, for all $n\geq1$. Equation
\eqref{eq:Matching_Condition_Gn_Gnhat_p>pc} then implies that 
$[\hat{G}_{n-1}(\xh)-\xh\hat{G}_n(\xh)]\sim1$ as $\xh\to0$,
for all $n\geq1$, a competition in sign between two diverging
terms. %Or equivalently,  $\xh^n\hat{G}_n(\xh)\sim\hat{G}_0(\xh)$
%for all $n\geq1$.
Therefore, 
%
\begin{align}
  \gamma+\Delta(n-1)=\gh_0+\Dh(n-1), \quad n\geq1.
\end{align}
%
Which in turn, implies that $\gamma=\gh_0$ and $\Delta=\Dh$ $\Box$.
%
%-------------------------------------------------------
%
%-------------------------------------------------------
 \begin{lemma}\label{lem:asymp_Scaling_funs_x_to_0_p<pc}
   Let $\hat{G}_n(\xh)$, $G_{n-1}(x)$, and the associated critical
   exponents be defined as in equation \eqref{eq:Scaling_fun_Def}, for
   $p<p_c$. Then
   %
     \begin{itemize}
    \item[1)] $\hat{G}_{n-1}(\xh)\sim1$ as $\xh\to0$ $(h\to0$ and $0<p_c-p\ll1)$
      for all $n\geq1$ 
    \item[2)] $G_{n-1}(x)\sim1$ as $x\to0$ $(h\to0$ and $0<p_c-p\ll1$ for all $n\geq1$
    \item[3)] $\gamma^\prime=\gh_0^\prime$  
    \item[4)] $\Delta^\prime=\Dh^\prime$    
     \end{itemize}
   %
 \end{lemma}
%
\noindent \textbf{Proof}:
%
Let $h\in\mathcal{U}\cap\mathbb{R}$, so that $g(p,h)$ and $\hat{g}(p,h)$ are
real analytic \cite{Golden:CMP-473}. Moreover let $p<p_c$ so that, by
equation \eqref{eq:Bounded_Gn_h}, all $h$ derivatives of
$\hat{g}(p,h)$ are bounded for $h=0$. Thus by equation
\eqref{eq:Diff_g} we have   
%
\begin{align*}
  \lim_{h\to0}h \int_0^{S(p)}\frac{y^nd\ph(y)}{(1+hy)^{n+1}}=0.
\end{align*}
%
Therefore, equations \eqref{eq:Diff_g_ghat_relation_Integral}, 
\eqref{eq:Scaling_fun_Def}, and
\eqref{eq:Non-negative_Gn_Ghn}--\eqref{eq:Bounded_Gn_h} imply that,
for all $n\geq1$, $0<p_c-p\ll1$, and $0<h\ll1$,  
%
\begin{align}\label{eq:Matching_Condition_Gn_Gnhat_p<pc}
  (0,\infty)\ni(p_c-p)^{-(\gh^\prime+\Dh^\prime(n-1))}\hat{G}_{n-1}(\xh)
       \sim(p_c-p)^{-(\gamma^\prime+\Delta^\prime(n-1))}G_{n-1}(x).
\end{align}
%
Equations \eqref{eq:Non-negative_Gn_Ghn}--\eqref{eq:Bounded_Gn_h}
imply that $\hat{G}_{n-1}(\xh)\sim1$ as $\xh\to0$ for all $n\geq1$. Equation 
\eqref{eq:Matching_Condition_Gn_Gnhat_p<pc} then implies that
$G_{n-1}(x)\sim1$ as $x\to0$ for all $n\geq1$. Therefore, 
%
\begin{align*}
  \gamma^\prime+\Delta^\prime(n-1)=\gh_0^\prime+\Dh^\prime(n-1), \quad n\geq1.
\end{align*}
%
Which in turn, implies that $\gamma^\prime=\gh_0^\prime$ and $\Delta^\prime=\Dh^\prime$ $\Box$.
%
%-------------------------------------------------------
%
%-------------------------------------------------------
 \begin{lemma}\label{lem:Scaling_rel_t_s_gamman}
   Let $\hat{G}_n(\xh)$, $G_{n-1}(x)$, and the associated critical
   exponents be defined as in equation
   \eqref{eq:Scaling_fun_Def}. Then   
   %
     \begin{itemize}
    \item[1)] $\gamma_n= \gamma+\Delta(n-1)$ for all $n\geq1$ 
    \item[2)] $\gh_n^\prime=\gh_0^\prime+\Dh^\prime n=\gh^\prime+\Dh^\prime(n-1)$ for all $n\geq0$ 
    \item[3)] $t=\Delta-\gamma$
    \item[4)] $s=\gh_0^\prime=\gh^\prime-\Dh^\prime$  
     \end{itemize}
   %
 \end{lemma}
%
\noindent \textbf{Proof}:
%
Let $0<p-p_c\ll1$. By equations  \eqref{eq:Diff_g} of
Remark \ref{lem:h_diff_commutation}, \eqref{eq:Crit_Exponents_mh}, 
\eqref{eq:Scaling_fun_Def}, and Lemma
\ref{lem:asymp_Scaling_funs_x_to_0_p>pc} we have, for all $n\geq1$,
%
\begin{align*}
  (p-p_c)^{-\gamma_n}&\sim\phi_n
             \sim\lim_{h\to0}\frac{\partial^ng(p,h)}{\partial h^n}
             \sim(p-p_c)^{-(\gamma+\Delta(n-1))}\lim_{x\to0}G_{n-1}(x)\\
             &\sim(p-p_c)^{-(\gamma+\Delta(n-1))}.\notag 
\end{align*}
%
Therefore $\gamma_n=\gamma+\Delta(n-1)$ for all $n\geq1$, with constant gap
$\gamma_i-\gamma_{i-1}=\Delta$, which is consistent with the absence of multifractal
behavior for the bulk conductivity \cite{Stauffer-92}.

Let $0<p_c-p\ll1$. Similarly, by equations \eqref{eq:Diff_g} of
Remark \ref{lem:h_diff_commutation}, \eqref{eq:Crit_Exponents_wh}, 
\eqref{eq:Scaling_fun_Def}, and Lemma
\ref{lem:asymp_Scaling_funs_x_to_0_p<pc} and we have, for all $n\geq1$, 
%
\begin{align*}
  (p_c-p)^{-\gh_n}\sim\ph_n
             \sim\lim_{h\to0}\frac{\partial^n\hat{g}(p,h)}{\partial h^n}
             \propto(p_c-p)^{-(\gh_0^\prime+\Dh^\prime n)}\lim_{\xh\to0}\hat{G}_n(\xh)
             \sim(p_c-p)^{-(\gh_0^\prime+\Dh^\prime n)}. 
\end{align*}
%
Therefore, by Lemma \ref{lem:nonzero_gamma1_etc}, we have
$\gh_n=\gh_0^\prime+\Dh^\prime n=\gh^\prime+\Dh^\prime(n-1)$ for all $n\geq0$, with constant
gap $\gh^\prime_i-\gh^\prime_{i-1}=\Dh$, which is consistent with the absence of
multifractal behavior for the bulk conductivity \cite{Stauffer-92}.

Let $0<p-p_c\ll1$. Equations \eqref{eq:mh_Stieltjes_rep},
\eqref{eq:Crit_Exponents_mh}, \eqref{eq:g_ghat_relation},
\eqref{eq:Divergence_Rate_w(p,h)}, and \eqref{eq:Scaling_fun_Def} yield
%
\begin{align}\label{eq:t_calculation}
  (p-p_c)^t&\sim\lim_{h\to0}m(p,h)
        =1-\lim_{h\to0}g(p,h)
        =\lim_{h\to0}h\hat{g}(p,h)
        =(p-p_c)^{\Dh-\gh_0 }\lim_{\xh\to0}\xh\hat{G}_0(\xh)\notag\\
        &\sim(p-p_c)^{\Dh-\gh_0}.
\end{align}
%
Therefore, by Lemma \ref{lem:asymp_Scaling_funs_x_to_0_p>pc} we have
$t=\Dh-\gh_0=\Delta-\gamma$.

Let $0<p_c-p\ll1$. By equations \eqref{eq:mh_Stieltjes_rep},
\eqref{eq:Crit_Exponents_wh}, \eqref{eq:Scaling_fun_Def}, and Lemmas
\ref{lem:nonzero_gh_n} and \ref{lem:asymp_Scaling_funs_x_to_0_p<pc},
%
\begin{align*}
  (p_c-p)^{-s}\sim\lim_{h\to0}w(p,h)
           \sim\lim_{h\to0}\hat{g}(p,h)
           =(p_c-p)^{-\gh_0^\prime}\lim_{\xh\to0}\hat{G}_0(\xh)
           \sim(p_c-p)^{-\gh_0^\prime}. 
\end{align*}
%
Therefore, by Lemma \ref{lem:Scaling_rel_t_s_gamman}.2, we have
$s=\gh_0^\prime=\gh^\prime-\Dh^\prime$ $\Box$. 
%---------------------------------------------------------------------------
%
%-------------------------------------------------------
 \begin{lemma}\label{lem:G_ghat_asymp_x_to_infty}
   Let $\hat{G}_n(\xh)$, $G_{n-1}(x)$, and the associated critical
   exponents be defined as in equation \eqref{eq:Scaling_fun_Def}, for
   $p>p_c$ and $p<p_c$. Then for all $n\geq1$ 
   %
     \begin{itemize}
    \item[1)] $G_{n-1}(x)\sim[\hat{G}_{n-1}(\xh)-\xh\hat{G}_n(\xh)]\sim
      x^{-(\gamma+\Delta(n-1))/\Delta}$ as $x\to\infty$ $(p\to p_c^+$ and 
      $0<h\ll1)$
    \item[2)] $G_{n-1}(x)\sim[\hat{G}_{n-1}(\xh)-\xh\hat{G}_n(\xh)]\sim
      x^{-(\gamma^\prime+\Delta^\prime(n-1))/\Delta^\prime}$ as $x\to\infty$ $(p\to p_c^-$ and $0<h\ll1)$           
    \item[3)] $\delta=\Delta/(\Delta-\gamma)$
    \item[4)] $\dha\,^\prime=\Dh^\prime/\gh_0^\prime=\Dh^\prime/(\gh^\prime-\Dh^\prime)$  
     \end{itemize}
   %
 \end{lemma}
%
\noindent \textbf{Proof}:
%
Let $0<h\ll1$, so that $g(p,h)$ and $\hat{g}(p,h)$ are analytic for
all $p\in[0,1]$ \cite{Golden:CMP-473}. The analyticity of $g(p,h)$ and
$\hat{g}(p,h)$ implies that all orders of $h$ derivatives of these
functions are bounded as $p\to p_c$, from the left or the
right. Therefore, equation \eqref{eq:Matching_Condition_Gn_Gnhat_p>pc}
holds for $0<p-p_c\ll1$, and 
%
\begin{align}\label{eq:x_infty_p<pc}
  (0,\infty)\ni(p_c-p)^{-(\gamma^\prime+\Delta^\prime(n-1))}G_{n-1}(x)
       =(p_c-p)^{-(\gh^\prime+\Dh^\prime(n-1))}[\hat{G}_{n-1}(\xh)-\xh\hat{G}_n(\xh)]
\end{align}
%
holds for $0<p_c-p\ll1$. Moreover, in order to cancel the diverging $p$
dependent 
prefactors in equations \eqref{eq:Matching_Condition_Gn_Gnhat_p>pc}
and \eqref{eq:x_infty_p<pc} we must have, for all $n\geq1$,  
%
\begin{align}\label{eq:Asymp_Gn_Ghn_x_to_infty}
  &G_{n-1}(x)\sim x^{-(\gamma+\Delta(n-1))/\Delta}, %\quad
  &&
  &[\hat{G}_{n-1}(\xh)-\xh\hat{G}_n(\xh)]\sim\xh^{-(\gh+\Dh(n-1))/\Dh}, \quad
      \text{as } p\to p_c^+,
\\
  &G_{n-1}(x)\sim x^{-(\gamma^\prime+\Delta^\prime(n-1))/\Delta^\prime}, %\quad
  &&
  &[\hat{G}_{n-1}(\xh)-\xh\hat{G}_n(\xh)]\sim\xh^{-(\gh^\prime+\Dh^\prime(n-1))/\Dh^\prime}, \quad
      \text{as }   p\to p_c^-.    \notag
\end{align}
%
Lemma \ref{lem:G_ghat_asymp_x_to_infty}.1-2 follows from equation
\eqref{eq:Asymp_Gn_Ghn_x_to_infty} and Lemmas
\ref{lem:asymp_Scaling_funs_x_to_0_p>pc}--\ref{lem:asymp_Scaling_funs_x_to_0_p<pc}.

Now by equations \eqref{eq:mh_Stieltjes_rep}, \eqref{eq:Crit_Exponents_mh},
\eqref{eq:m_w_relation}, \eqref{eq:Scaling_fun_Def}, and
\eqref{eq:Asymp_Gn_Ghn_x_to_infty} for $n=1$, we have
%
\begin{align}
  h^{1/\delta}&\sim\lim_{p\to p_c^+}m(p,h)
      \sim\lim_{p\to p_c^+}h\hat{g}(p,h)
      =h\lim_{p\to p_c^+}(p-p_c)^{-\gh_0 }\hat{G}_0(\xh)\\
      &\sim h(p-p_c)^{-\gh_0 }h^{-\gh_0/\Dh}(p-p_c)^{-\Dh(-\gh_0/\Dh) }
      =h^{(\Dh-\gh_0)/\Dh}, \text{ as } h\to0.\notag
\end{align}
%
for $0<h\ll1$. Therefore by Lemma
\ref{lem:asymp_Scaling_funs_x_to_0_p<pc}, we have 
$\delta=\Dh/(\Dh-\gh_0)=\Delta/(\Delta-\gamma)$. Similarly by equations
\eqref{eq:mh_Stieltjes_rep}, \eqref{eq:Crit_Exponents_wh},
\eqref{eq:Scaling_fun_Def}, and \eqref{eq:Asymp_Gn_Ghn_x_to_infty}
for $n=1$, and Lemma \ref{lem:nonzero_gh_n} 
%
\begin{align}
   h^{-1/{\dha^\prime}}\sim\lim_{p\to p_c^-}w(p,h)
      \sim\lim_{p\to p_c^-}\hat{g}(p,h)
      =\lim_{p\to p_c^-}(p-p_c)^{-\gh_0^\prime}\hat{G}_0(\xh)      
      =h^{-\gh_0^\prime/\Dh\,^\prime},
\end{align}
%
for $0<h\ll1$. Therefore, by Lemma \ref{lem:Scaling_rel_t_s_gamman} we have 
$\dha\,^\prime=\Dh^\prime/\gh_0^\prime=\Dh^\prime/(\gh^\prime-\Dh^\prime)$ $\Box$. 
%-------------------------------------------------------
%
%-----------------------------------------------------------------------------------
 \begin{lemma}\label{lem:Complex_s_t}
   Let $h\in\mathcal{U}$ such that $h_i\neq0$, and $\hat{\mathcal{G}}_{n,j}(\xh)$,
   $\hat{\mathcal{R}}_n(\xh)$, $\hat{\mathcal{I}}_n(\xh)$, and the
   associated critical exponents be defined as in equations
   \eqref{eq:Complex_Scaling_fun_Def}--\eqref{eq:Complex_Scaling_fun_Def_Integrals} 
   for $p>p_c$ and $p<p_c$. Furthermore, let $s_r$, $s_i$, $t_r$, and
   $t_i$ be defined as in equations
   \eqref{eq:Crit_Exponents_mh}--\eqref{eq:Crit_Exponents_wh}. Then,       
   %
     \begin{itemize}
    \item[1)] $\hat{\mathcal{R}}_0(\xh)\sim\hat{\mathcal{I}}_0(\xh)\sim1$ as
      $\xh\to0$ $(h\to0$ and $0<p_c-p\ll1)$
    \item[2)]
      $\lim_{\xh\to0}[\xh_r\hat{\mathcal{R}}_0(\xh)-\xh_i\hat{\mathcal{I}}_0(\xh)]
      \sim\lim_{\xh\to0}[\xh_r\hat{\mathcal{I}}_0(\xh)+\xh_i\hat{\mathcal{R}}_0(\xh)]\sim1$
      for $0<p-p_c\ll1$  
    \item[3)] $s_r=s_i=\gh_0^\prime=s$ 
    \item[4)] $t_r=t_i=\Delta-\gamma=t$ 
     \end{itemize}
   %
 \end{lemma}
%
\noindent \textbf{Proof}:
%
Let $0<p_c-p\ll1$, $h\in\mathcal{U}$ such that $h_i\neq0$, and $0<|h|\ll1$. By
equation
\eqref{eq:Complex_Scaling_fun_Def}--\eqref{eq:Complex_Scaling_fun_Def_Integrals}, 
for $n=0$, we have  
%
\begin{align}
  \hat{g}(p,h)=\int_0^{\hat{S}(p)}\frac{d\ph(y)}{|1+hy|^2}
                +\bar{h}\int_0^{\hat{S}(p)}\frac{y\,d\ph(y)}{|1+hy|^2}
              =(p-p_c)^{-\gh_0^\prime}[\hat{\mathcal{G}}_{0,0}(\xh)
                +\bar{\xh}\hat{\mathcal{G}}_{0,1}(\xh)],
\end{align}
%
so that
%
\begin{align}\label{eq:Complex_ghat}
  \hat{g}_r&=(p_c-p)^{-\gh_0^\prime}\hat{\mathcal{R}}_0(\xh)
          =(p_c-p)^{-\gh_0^\prime}[\hat{\mathcal{G}}_{0,0}(\xh)
                +\xh_r\hat{\mathcal{G}}_{0,1}(\xh)]\\
  \hat{g}_i&=(p_c-p)^{-\gh_0^\prime}\hat{\mathcal{I}}_0(\xh)
          =-(p_c-p)^{-\gh_0^\prime}\xh_i\hat{\mathcal{G}}_{0,1}(\xh).
          \notag
\end{align}
%
Equations \eqref{eq:Non-negative_Gn_Ghn}--\eqref{eq:Bounded_Gn_h}
imply that $\hat{\mathcal{R}}_0(\xh)\sim\hat{\mathcal{I}}_0(\xh)\sim1$ as
$\xh\to0$ $(h\to0$ and $0<p_c-p\ll1)$. Therefore, equations
\eqref{eq:mh_Stieltjes_rep}, \eqref{eq:Crit_Exponents_wh},
\eqref{eq:Complex_ghat} and Lemma \ref{lem:nonzero_gh_n} imply that 
%
\begin{align}\label{eq:Complex_w_asymp}
  (p_c-p)^{-s_r}\sim w_r(p,0)
              \sim\hat{g}_r(p,0)
              \sim(p_c-p)^{-\gh_0^\prime}\lim_{\xh\to0}\hat{\mathcal{R}}_0(\xh)
              \sim(p_c-p)^{-\gh_0^\prime},\\
   (p_c-p)^{-s_i}\sim w_i(p,0)
              \sim\hat{g}_i(p,0)
              \sim(p_c-p)^{-\gh_0^\prime}\lim_{\xh\to0}\hat{\mathcal{I}}_0(\xh)
              \sim(p_c-p)^{-\gh_0^\prime}. \notag            
\end{align}
%
Equation \eqref{eq:Complex_w_asymp} and Lemma
\ref{lem:Scaling_rel_t_s_gamman} imply that $s_r=s_i=\gh_0^\prime=s$, which
generalizes the result involving $s$ in Lemma
\ref{lem:Scaling_rel_t_s_gamman}. It's worth noting that these scaling
relations are independent of the path of the limit $h\to0$. The above
proof does not preclude that either $s_r=0$ or $s_i=0$, but not
both. Although, this is not physically consistent
\cite{Efros:PSSB-303}.  

Let $0<p-p_c\ll1$ and $0<|h|\ll1$. Equation \eqref{eq:t_calculation} shows
that we have $m(p,0)=\lim_{h\to0}h\hat{g}(p,h)$. Therefore equation
\eqref{eq:Complex_ghat}, for $p>p_c$, implies that 
%
\begin{align}\label{eq:Complex_ghat_m}
  m_r(p,0)&\sim\lim_{h\to0}[h_r\hat{g}_r(p,h)-h_i\hat{g}_i(p,h)]
         =(p-p_c)^{\Dh-\gh_0}
           \lim_{\xh\to0}[\xh_r\hat{\mathcal{R}}_0(\xh)-\xh_i\hat{\mathcal{I}}_0(\xh)],
           \notag\\
  m_i(p,0)&\sim\lim_{h\to0}[h_r\hat{g}_i(p,h)+h_i\hat{g}_r(p,h)]
         =(p-p_c)^{\Dh-\gh_0}
            \lim_{\xh\to0}[\xh_r\hat{\mathcal{I}}_0(\xh)+\xh_i\hat{\mathcal{R}}_0(\xh)].
\end{align}
%
By equation \eqref{eq:Divergence_Rate_w(p,h)}, 
$\lim_{\xh\to0}[\xh_r\hat{\mathcal{R}}_0(\xh)-\xh_i\hat{\mathcal{I}}_0(\xh)]\sim
\lim_{\xh\to0}[\xh_r\hat{\mathcal{I}}_0(\xh)-\xh_i\hat{\mathcal{R}}_0(\xh)]\sim1$ 
for $0<p-p_c\ll1$. Therefore, equations \eqref{eq:Crit_Exponents_mh} and
\eqref{eq:Complex_ghat_m} imply that
%
\begin{align}\label{eq:Complex_m_asymp}
  (p-p_c)^{t_r}\sim m_r(p,0)\sim(p-p_c)^{\Dh-\gh_0}, \quad (p-p_c)^{t_i}\sim m_i(p,0)\sim(p-p_c)^{\Dh-\gh_0}
\end{align}
%
Equation \eqref{eq:Complex_m_asymp} and Lemmas
\ref{lem:asymp_Scaling_funs_x_to_0_p>pc} and
\ref{lem:Scaling_rel_t_s_gamman} imply that
$t_r=t_i=\Dh-\gh_0=\Delta-\gamma=t$, which generalizes the result involving $t$
in Lemma \ref{lem:Scaling_rel_t_s_gamman}. It's worth noting that
these scaling relations are independent of the path of the limit
$h\to0$. The above proof does not preclude that either $t_r=0$ or
$t_i=0$, but not both. Although, this is not physically consistent
\cite{Efros:PSSB-303} $\Box$.   
%-----------------------------------------------------------------------------------
%
%-----------------------------------------------------------------------------------
 \begin{lemma} \label{lem:Complex_delta}
   Let $h\in\mathcal{U}$ such that $h_i\neq0$, and $\hat{\mathcal{G}}_{n,j}(\xh)$,
   $\hat{\mathcal{R}}_n(\xh)$, $\hat{\mathcal{I}}_n(\xh)$, and the
   associated critical exponents be defined as in equations
   \eqref{eq:Complex_Scaling_fun_Def}--\eqref{eq:Complex_Scaling_fun_Def_Integrals} 
   for $p>p_c$ and $p<p_c$. Furthermore, let $\dha_r$, $\dha_i$, $\delta_r$, and
   $\delta_i$ be defined as in equations
   \eqref{eq:Crit_Exponents_mh}--\eqref{eq:Crit_Exponents_wh}. Then,       
   %
     \begin{itemize}
    \item[1)] $\hat{\mathcal{R}}_0(\xh)\sim\hat{\mathcal{I}}_0(\xh)
                                      \sim|\xh|^{-\gh_0^\prime/\Dh^\prime}$ as
             $\xh\to\infty$ $(p\to p_c^-$ and $0<|h|\ll1)$ 
    \item[2)]
      $[\xh_r\hat{\mathcal{R}}_0(\xh)-\xh_i\hat{\mathcal{I}}_0(\xh)]
      \sim[\xh_r\hat{\mathcal{I}}_0(\xh)+\xh_i\hat{\mathcal{R}}_0(\xh)]
      \sim|\xh|^{(\Dh-\gh_0)/\Dh}$ as $\xh\to\infty$    
    \item[3)] $\dha_r\,^\prime=\dha_i\,^\prime=\Dh^\prime/\gh_0^\prime=\dha$ 
    \item[4)] $\delta_r=\delta_i=\Delta/(\Delta-\gamma)=\delta$ 
     \end{itemize}
   %
 \end{lemma}
%
\noindent \textbf{Proof}:
%
Let $0<h\ll1$, so that $g(p,h)$ and $\hat{g}(p,h)$ are analytic for
all $p\in[0,1]$ \cite{Golden:CMP-473}. Equations \eqref{eq:mh_Stieltjes_rep},
\eqref{eq:Crit_Exponents_wh}, \eqref{eq:Complex_ghat} and Lemma
\ref{lem:nonzero_gh_n} imply that  
%
\begin{align}\label{eq:Complex_w_asymp_infty}
  |h|^{-1/\dha_r^\prime}\sim w_r(p_c,h)
              \sim\hat{g}_r(p_c,h)
              \sim\lim_{p\to p_c^-}(p_c-p)^{-\gh_0^\prime}\hat{\mathcal{R}}_0(\xh),
              \\
   |h|^{-1/\dha_i^\prime}\sim w_i(p_c,h)
              \sim\hat{g}_i(p_c,h)
              \sim\lim_{p\to p_c^-}(p_c-p)^{-\gh_0^\prime}\hat{\mathcal{I}}_0(\xh). \notag            
\end{align}
%
The analyticity of $g(p,h)$ and $\hat{g}(p,h)$ implies that they are
bounded for all $p\in[0,1]$. Therefore, in order to cancel the diverging
$p$ dependent prefactors in equations \eqref{eq:Complex_w_asymp_infty}, we
must have
$\hat{\mathcal{R}}_0(\xh)\sim\hat{\mathcal{I}}_0(\xh)\sim|x|^{-\gh_0^\prime/\Dh^\prime}$
as $\xh\to\infty$ $(p\to p_c^-$ and $0<h\ll1)$. Equation
\eqref{eq:Complex_w_asymp_infty} then implies that
%
\begin{align}\label{eq:Complex_}
  |h|^{-1/\dha_r^\prime}&\sim(p_c-p)^{-\gh_0^\prime}|h|^{-\gh_0^\prime/\Dh^\prime}(p_c-p)^{-\Dh^\prime(-\gh_0^\prime/\Dh^\prime)}
               =|h|^{-\gh_0^\prime/\Dh^\prime},\\
   |h|^{-1/\dha_i^\prime}&\sim|h|^{-\gh_0^\prime/\Dh^\prime}. \notag             
\end{align}
%
Therefore, by Lemma \ref{lem:G_ghat_asymp_x_to_infty},
$\dha_r\,^\prime=\dha_i\,^\prime=\Dh^\prime/\gh_0^\prime=\dha\,^\prime$. 

Let $0<h\ll1$, so that $g(p,h)$ and $\hat{g}(p,h)$ are analytic for
all $p\in[0,1]$ \cite{Golden:CMP-473}. Equations
\eqref{eq:mh_Stieltjes_rep} and \eqref{eq:m_w_relation} imply that
shows that $m(p_c,h)\sim\lim_{p\to p_c^+}h\hat{g}(p,h)$. Therefore 
equations \eqref{eq:Crit_Exponents_mh} and \eqref{eq:Complex_ghat}
implies that  
%
\begin{align}\label{eq:Complex_ghat_m_infty}
   &|h|^{1/\delta_r}\sim m_r(p_c,0)%&=\lim_{p\to p_c^+}[h_r\hat{g}_r(p,h)-h_i\hat{g}_i(p,h)]\\
         =(p-p_c)^{\Dh-\gh_0}
           \lim_{p\to p_c^+}[\xh_r\hat{\mathcal{R}}_0(\xh)-\xh_i\hat{\mathcal{I}}_0(\xh)],
           \\
  &|h|^{1/\delta_i}\sim m_i(p,0)%&=\lim_{p\to p_c^+}[h_r\hat{g}_i(p,h)+h_i\hat{g}_r(p,h)]\notag\\
         =(p-p_c)^{\Dh-\gh_0}
            \lim_{p\to p_c^+}[\xh_r\hat{\mathcal{I}}_0(\xh)+\xh_i\hat{\mathcal{R}}_0(\xh)].
            \notag
\end{align}
%
The analyticity of $g(p,h)$ and $\hat{g}(p,h)$ implies that they are
bounded for all $p\in[0,1]$. Therefore, in order to cancel the diverging
$p$ dependent prefactors in equations \eqref{eq:Complex_ghat_m_infty}, we
must have
$[\xh_r\hat{\mathcal{R}}_0(\xh)-\xh_i\hat{\mathcal{I}}_0(\xh)]
 \sim[\xh_r\hat{\mathcal{I}}_0(\xh)+\xh_i\hat{\mathcal{R}}_0(\xh)]\sim|x|^{(\Dh-\gh_0)/\Dh}$
as $\xh\to\infty$ $(p\to p_c^+$ and $0<h\ll1)$. Therefore equation
\eqref{eq:Complex_ghat_m_infty}, and Lemmas
\ref{lem:asymp_Scaling_funs_x_to_0_p>pc} and
\ref{lem:G_ghat_asymp_x_to_infty} imply that 
$\delta_r=\delta_i=\Dh/(\Dh-\gh_0)=\Delta/(\Delta-\gamma)=\delta$ $\Box$.
%
%----------------------------------------------------------------------------------
\begin{lemma}\label{lem:s_t}
  If $\Delta=\Delta^\prime$, $\gamma=\gamma^\prime$, $\Dh=\Dh^\prime$, and $\gh_0=\gh_0^\prime$. Then
  %
     \begin{itemize}
    \item[1)] $s+t=\Delta$  
    \item[2)] $\delta=1/(1-1/\dha\,^\prime)$    
     \end{itemize}
   %  
 \end{lemma}
%
\noindent \textbf{Proof}:
%
Assume that the spectral properties of the measures, $d\ph(y)$ and
$y\,d\phi(y)$, have the symmetry $\Delta=\Delta^\prime$, $\gamma=\gamma^\prime$, $\Dh=\Dh^\prime$, and
$\gh_0=\gh_0^\prime$. By Lemma \ref{lem:Scaling_rel_t_s_gamman} we have $t=\Delta-\gamma$
and $s=\gh_0^\prime$, and by Lemma \ref{lem:G_ghat_asymp_x_to_infty} we
have $\delta=\Delta/(\Delta-\gamma)$ and $\dha\,^\prime=\Dh^\prime/\gh_0^\prime$. Lemmas
\ref{lem:asymp_Scaling_funs_x_to_0_p>pc}--\ref{lem:asymp_Scaling_funs_x_to_0_p<pc}
show that $\gamma=\gh_0$, $\Delta=\Dh$, $\gamma^\prime=\gh_0^\prime$, and
$\Delta^\prime=\Dh^\prime$. Therefore,
%
\begin{align*}
  &s+t=\gh_0^\prime+\Delta-\gamma=\gh_0+\Delta-\gamma=\Delta\\
  &\delta=\Delta/(\Delta-\gamma)=1/(1-\gamma/\Delta)=1/(1-\gh_0/\Dh)=1/(1-\gh_0^\prime/\Dh^\prime)=1/(1-1/\dha\,^\prime)
  \quad \Box.
\end{align*}
%----------------------------------------------------------------------------------

In this section we derived the (two--parameter) scaling relations
regarding the conductor/insulator critical transition
\eqref{eq:Cond-Insul_Crit_Beh_pc} and that of the
conductor/superconductor critical transition 
\eqref{eq:Cond-SuperCond_Crit_Beh_pc}. Assuming that the measures,
$d\ph(y)$ and $y\,d\phi(y)$, have the spectral symmetry property $\Delta=\Delta^\prime$,
$\gamma=\gamma^\prime$, $\Dh=\Dh^\prime$, and $\gh_0=\gh_0^\prime$, we also showed that there
are (two--parameter) scaling relations between these two sets of
critical exponents. There is no apparent mathematical necessity for
this spectral symmetry. Although, the relation $s+t=\Delta$ is consistent
with equation (4) in \cite{Efros:PSSB-303}, $s=t(\delta-1)$, which was
derived under a physical scaling hypothesis and leads to the two
dimensional duality relation $s=t$
\cite{Bergman:SSP-147,Clerc:AP-191}, as $\delta=2$ for $d=2$
\cite{Efros:PSSB-303}. The scaling relations are independent of the
path of the limit $h\to0$. Although the behavior of the system, as a
function of $p\in[0,1]$, is highly dependent on the location of
$h\in\mathcal{U}$ and is governed by equation
\eqref{eq:Diff_g_ghat_relation} (or formally by equation
\eqref{eq:Diff_g_ghat_relation_Integral}) or equivalently by the  
system of coupled partial differential equations
\eqref{eq:Complex_Diff_g_ghat_relation}.  

We have shown how the symmetries between the 
integral representations of $\sigma^*$ and $[\sigma^{-1}]^*$ may be used to
generalize the results of this section in terms of
$[\sigma^{-1}]^*$. This beautiful mathematical framework, regarding the
geometric critical transitions of percolating binary composites, will
be extended further to our statistical mechanics description of
electrically/thermally driven critical transitions of binary
dielectrics and metal/dielectric composites in section
\ref{sec:StatMech_of_Composites}. In section \ref{sec:Measure_Equiv}
we discuss some of the more subtle measure theoretic details regarding
the underlying symmmetries between $m(p,h)$ and $w(p,z(h))$. We will
show that this leads to a generalization of a result
\cite{Day:JPCM-96} which characterizes the measure $\varrho$ found in
equation \eqref{eq:BM_measure_relationship_E}, in terms of the
symmetries between the measures $d\ph(y)$ and $y\,d\phi(y)$.   
%
\subsection{Measure Equivalences in Transport} \label{sec:Measure_Equiv}
%
In section \ref{sec:Crit_Behav_of_Transport} we derived scaling
relations describing the critical transitions exhibited by binary
conductors. These scaling relations were found using the
relationships \eqref{eq:m_w_relation} and \eqref{eq:g_ghat_relation}
between $m(h,p)$ and $w(h,p)$, and $g(h,p)$ and $\hat{g}(h,p)$,
respectively, linking the critical behavior of conductor/insulator and
conductor/superconductor systems. In this section, we further explore
the consequences of this link. 

Using equations \eqref{eq:mh_Stieltjes_rep} and
\eqref{eq:m_w_relation} we arived at 
%
\begin{align}\label{eq:g_ghat_relation_subsection}
  g(p,h)+h\hat{g}(p,h)=1, \quad \forall \ p\in[0,1], \ h\in\mathcal{U}.
\end{align}
%
Equation \eqref{eq:g_ghat_relation_subsection} contains a profound formula
relating $g(p,h)$ and $\hat{g}(p,h)$, and suggests a deep relationship
between the measures $\phi(p)$ and $\ph(p)$. More specifically, it
implies that the measures behave in a way which maintains the equality
in \eqref{eq:g_ghat_relation_subsection} for all $p\in[0,1]$ when $h\in\mathcal{U}$
is held fixed. Moreover, for fixed $p\in[0,1]$, the measure transforms,
$g(p,h)$ and $\hat{g}(p,h)$, also behave in a way which maintains the
equality in \eqref{eq:g_ghat_relation_subsection} for all $h\in\mathcal{U}$. This
section is devoted to the analysis of this equation. 

The integral representaion of equation \eqref{eq:g_ghat_relation_subsection}
follows from equations \eqref{eq:mh_Stieltjes_rep}, and is given by
%
\begin{align}\label{eq:g_ghat_relation_integral}
  \int_0^\infty\frac{d\phi(y)}{1+hy}+h\int_0^\infty\frac{d\ph(y)}{1+hy}=1.
\end{align}
%
We wish to reexpress equation \eqref{eq:g_ghat_relation_integral} in a
more suggestive form by adding and subtracting the quantity
$h\int_0^{S(p)}y\,d\phi(y)/(1+hy)$, which is permissible if the modulus of
this quantity is finite for all $p\in[0,1]$ and $h\in\mathcal{U}$. The
affirmation of this fact directly follows from the techniques of Lemma
\ref{lem:h_diff_commutation}. In particular, set $h\in\mathcal{U}$,
and $0\ll S_h<\infty$ satisfying \eqref{eq:S1_asymp}, and write
$\Sigma_\phi:=[S_0(p),S_h]\cup(S_h,S(P)]$. In Lemma 
\ref{lem:h_diff_commutation} we showed that the mass $\phi_0$ of the
positive measure $\phi$ is uniformly bounded for all $p\in[0,1]$. Therefore,
%
\begin{align}\label{eq:L1(y_phi)_bound_finite_set}
 \left| \int_{S_0(p)}^{S_h}\frac{y\,d\phi(y)}{1+hy}\right|\leq
  \frac{S_h\,\phi([S_0(p),S_h])}{|1+hS_0(p)|}<\infty,
\end{align}
%
Moreover, assuming that $S(p)=\infty$,  
%
\begin{align}
 \left|\int_{S_h}^{S(p)}\frac{y\,d\phi(y)}{1+hy}\right|
     \sim\frac{1}{|h|}\int_{S_h}^{S(p)} d\phi(y)
     =\frac{\phi([S_h,S(p)])}{|h|}<\infty.
\end{align}
%
Therefore, the modulus of the quantity $h\int_0^\infty y\,d\phi(y)/(1+hy)$ is
finite for all $p\in[0,1]$ and $h\in\mathcal{U}$, and we may add and
subtract it in equation \eqref{eq:g_ghat_relation_integral}, 
yeilding   
%
\begin{align}\label{eq:Phi_transform}
   1&=\int_0^\infty\frac{d\phi(y)}{1+hy}+h\int_0^\infty\frac{d\ph(y)}{1+hy}\\
    &=\left[\int_0^\infty\frac{d\phi(y)}{1+hy}+h\int_0^\infty\frac{y\,d\phi(y)}{1+hy}\right]
    +h\left[\int_0^\infty\frac{d\ph(y)}{1+hy}-\int_0^\infty\frac{y\,d\phi(y)}{1+hy}\right]
    \notag\\
    &=\phi_0+h\int_0^\infty\frac{d\Phi_0(y)}{1+hy},\quad d\Phi_0(y):=d\ph(y)-y\,d\phi(y).\notag
\end{align}
%
As $\hat{g}(p,h)$ is an analytic function of $h$ for all $p\in[0,1]$
when $h\in\mathcal{U}$ \cite{Golden:CMP-473}, the above argument leading
to equation \eqref{eq:Phi_transform} shows that the transform
$h\int_0^\infty d\Phi_0(y)/(1+hy)$ of the signed measure \cite{Rudin:87}
$\Phi_0(dy)$ is defined for all $p\in[0,1]$ and $h\in\mathcal{U}$, and is
given by the second term in the second line in equation
\eqref{eq:Phi_transform} \cite{Rudin:87}.  

Equations \eqref{eq:phi_moments_F(s)} and \eqref{eq:Phi_transform}
demonstrate that equation \eqref{eq:g_ghat_relation_integral} may be
reexpressed as  
%
\begin{align}\label{eq:n=0_measure_equivalence}
 h \int_0^\infty\frac{d\Phi_0(y)}{1+hy}\equiv1-\phi_0(p)\equiv m(p,0)=
         \begin{cases}
         0, & \text{for all } p<p_c \\
         O(1)\in(0,1), & \text{for all } p>p_c
         \end{cases}         
\end{align}
%
for all $h\in\mathcal{U}$. Equation \eqref{eq:n=0_measure_equivalence}
generalizes the result \eqref{eq:Divergence_Rate_w(p,h)}, gives a
alternate representation of $m(p,0)=\lim_{h\to0}hw(p,h)$, and shows that
the transform of the measure $\Phi_0$, $h\int_0^\infty d\Phi_0(y)/(1+hy)$, is
independent of $h$ for all $p\in[0,1]$. We may relate this
representation of $m(p,0)$ to the measure $\varrho$ found in equation
\eqref{eq:BM_measure_relationship_E} using equation
\eqref{eq:mh_Stieltjes_rep} and the identity $y=\lambda/(1-\lambda)\iff\lambda=y/(1+y)$:      
%
\begin{align}\label{eq:Measure_equivalence_rho}
  d\Phi_0(y)&:=d\ph(y)-y\,d\phi(y)
        =(y+1)\left(\left[-d\alpha\left(\frac{1}{y+1}\right)\right]
                    -y\,d\mu\left(\frac{y}{y+1}\right)
              \right)\\
        &=\frac{1}{(1-\lambda)^2}[-(1-\lambda)\,d\alpha(1-\lambda)-\lambda\,d\mu(\lambda)]
        =\frac{\lambda\,d\varrho(\lambda)}{(1-\lambda)^2}=y(1+y)\,d\varrho\left(\frac{y}{1+y}\right).\notag
\end{align}
%
We may therefore express equation \eqref{eq:n=0_measure_equivalence}
in terms of $\varrho(d\lambda)$ as follows: 
%
\begin{align}\label{eq:n=0_measure_equivalence_rho_transform}
   h\int_0^\infty\frac{d\Phi_0(y)}{1+hy}
      =h\int_0^\infty\frac{y(1+y)d\varrho(\frac{y}{1+y})}{1+hy}
      =\int_0^1\frac{\lambda\,d\varrho(\lambda)}{(1-\lambda)^2/h+\lambda(1-\lambda)}\,.
\end{align}
%
Equations 
\eqref{eq:n=0_measure_equivalence}--\eqref{eq:n=0_measure_equivalence_rho_transform}
are general formulas holding for two component stationary random media
in the lattice and continuum settings \cite{Golden:PRL-3935}. 
%
\begin{remark}\label{rem:varrho_condidtions}
  Define the transform $\mathcal{D}(p,h;\varrho)$ of the measure $\varrho$ as
  \begin{align}\label{eq:D_varrho}
    \mathcal{D}(p,h;\varrho)=\int_0^1\frac{\lambda\,d\varrho(\lambda)}{(1-\lambda)^2/h+\lambda(1-\lambda)}\,.
  \end{align}
  Equations
  \eqref{eq:n=0_measure_equivalence}--\eqref{eq:n=0_measure_equivalence_rho_transform}
  show that $\mathcal{D}(p,h;\varrho)$ satisfies the following properties:
  \begin{itemize}
  \item[(1)] $\mathcal{D}(p,h;\varrho)$ is independent of $h$ for
    all $p\in[0,1]$.    
  \item[(2)] $0<\mathcal{D}(p,h;\varrho)<1$ for all $p>p_c$.
  \item[(3)] $\mathcal{D}(p,h;\varrho)\equiv m(p,0)$ for all $p\in[0,1]$.   
  \end{itemize}  
\end{remark}
%
\noindent The following lemma is the key result of this section.
%
\begin{lemma}\label{lem:Measure_consistentcy_condition}
  Let $\mathcal{D}(p,h;\varrho)$ be defined as in equation
  \eqref{eq:D_varrho}, where $h\in\mathcal{U}$ and $p\in[0,1]$. If
  $\mathcal{D}(p,h;\varrho)$ satisfies the properties of Remark
  \ref{rem:varrho_condidtions}, then  
  %
\begin{align}\label{eq:Measure_consistentcy_condition}
 &\varrho(d\lambda)=W_0(p)\delta_0(d\lambda)+W_1(p)(1-\lambda)\delta_1(d\lambda),\\
  \quad W_0(p)=&\int_0^1\frac{d\alpha(\lambda)}{1-\lambda}-1, \quad
  W_1(p)=m(p,0)=1-\int_0^1\frac{d\mu(\lambda)}{1-\lambda}, \notag  
\end{align}
%
where $\delta_{\lambda_0}(d\lambda)$ is the Dirac measure centered at $\lambda_0$.
%
\end{lemma}
%
\noindent \textbf{Proof}:
%
Let $\mathcal{D}(p,h;\varrho)$, defined in equation \eqref{eq:D_varrho},
satisfy the properties of Remark \ref{rem:varrho_condidtions}. The
measure $\varrho$ is independent of $h$ \cite{Golden:CMP-473}. If the
measure $\varrho$ is over continuous spectrum \cite{Reed-1980} then
$\mathcal{D}(p,h;\varrho)$ depends on $h$, contradicting property
$(1)$. Therefore the measure $\varrho$ is defined over pure point spectrum
$\sigma_{pp}$ \cite{Reed-1980}. Moreover, in order for properties $(1)$ and
$(3)$ to be satisfied we must have $\sigma_{pp}\equiv\{0,1\}$, so that the measure
$\varrho$ is of the form   
%
\begin{align*}
  \varrho(d\lambda)=W_0(p,\lambda)\delta_0(d\lambda)+W_1(p,\lambda)\delta_1(d\lambda),
\end{align*}
%
where the $W_j(p,\lambda)$, $j=0,1$, are functions of the volume fraction
$p$ and $\lambda\in[0,1]$ which are to be determined. In view of the numerator
of the integrand in equation \eqref{eq:D_varrho}, we may assume that
the function $W_0(p,\lambda)=W_0(p,0):=W_0(p)\not\equiv0$ is independent of
$\lambda$. In order for property $(2)$ to be satisfied we must have
$W_1(p,\lambda)\sim1-\lambda$ as $\lambda\to1$ (any other power of $1-\lambda$ would contradict
property $(2)$). Therefore, with out loss of generality, we may set  
$W_1(p,\lambda)=(1-\lambda)W_1(p)$, $\,W_1(p)\not\equiv0$. Property $(3)$ then yields     
%
\begin{align}
  m(p,0)&=W_0(p)\lim_{\lambda\to0}\left[\frac{\lambda}{(1-\lambda)^2/h+\lambda(1-\lambda)}\right]
        +W_1(p)\lim_{\lambda\to1}\left[\frac{\lambda(1-\lambda)}{(1-\lambda)^2/h+\lambda(1-\lambda)}\right]
        \notag\\
        &=W_1(p).
\end{align}
%

We have shown that
%
\begin{align}\label{eq:varrho_except_W0}
  \varrho(d\lambda)=W_0(p)\delta_0(d\lambda)+m(p,0)(1-\lambda)\delta_1(d\lambda), \quad W_0(p)\not\equiv0.
\end{align}
%
Plugging equation \eqref{eq:varrho_except_W0} into 
\eqref{eq:BM_measure_relationship_E} $(\lambda d\mu(\lambda)=(1-\lambda)[-d\alpha(1-\lambda)] - \lambda d\varrho(\lambda))$,
we are able determine $W_0(p)$ by using the definition of $F(s)$
\eqref{eq:Fs_Integral} and equation \eqref{eq:Fs_relationships_G}
$(1-F(s)=(1-1/s)(1-G(s))\iff F(s)-(1-1/s)G(s)=1/s)$: 
%
\begin{align}\label{eq:find_varrho_W0}
  F(s)&=\int_0^1\frac{d\mu(\lambda)}{s-\lambda}=\int_0^1\frac{1-\lambda}{\lambda}\frac{[-d\alpha(1-\lambda)]}{s-\lambda}
                           -\int_0^1\frac{d\varrho(\lambda)}{s-\lambda}\notag\\
      &=-\left(1-\frac{1}{s}\right)\int_0^1\frac{[-d\alpha(1-\lambda)]}{s-\lambda}
         +\frac{1}{s}\int_0^1\frac{[-d\alpha(1-\lambda)]}{\lambda} -\int_0^1\frac{d\varrho(\lambda)}{s-\lambda}
       \notag\\
      &=\left(1-\frac{1}{s}\right)G(s)+\frac{1}{s}\int_0^1\frac{d\alpha(\lambda)}{1-\lambda}
         -\frac{W_0(p)}{s}-m(p,0)\lim_{s\to1}\frac{1-\lambda}{s-\lambda}, \quad
         \forall \ |s|>1 \Rightarrow\notag\\
      1&=\int_0^1\frac{d\alpha(\lambda)}{1-\lambda}-W_0(p),  
\end{align}
%
where we have used $(1-\lambda)/(\lambda(s-\lambda))=1/(s\lambda)-(1-1/s)/(s-\lambda)$.
The formulas \eqref{eq:varrho_except_W0}--\eqref{eq:find_varrho_W0}
imply equation \eqref{eq:Measure_consistentcy_condition} and suggest
that $\lambda=1$ is a removable singularity under $\mu$ \emph{and} $\alpha$ $\Box$.
%-----------------------------------------------------------------------

For completeness, we mention that the form
\eqref{eq:Measure_consistentcy_condition} of the measure $\varrho(d\lambda)$ is 
consistent with equation
\eqref{eq:Diff_g_ghat_relation_Integral}. Indeed, if we define the
following signed measure,
$d\Phi_{n-1}(y):=y^{n-1}d\Phi_0(y)$ for $n\geq1$, then by equation
\eqref{eq:Measure_equivalence_rho} the formula
\eqref{eq:Diff_g_ghat_relation_Integral} is equivalent to         
%
\begin{align}\label{eq:Diff_g_ghat_relation_Integral_equivalence}  
   0\equiv\int_0^\infty\frac{d\Phi_{n-1}(y)}{(1+hy)^{n+1}}
    =\int_0^1\frac{\left(\frac{\lambda}{1-\lambda}\right)^{n-1}\frac{\lambda\,d\varrho(\lambda)}{(1-\lambda)^2}}
             {(1+hy)^{n+1}}
    =\int_0^1\frac{\lambda^n\,d\varrho(\lambda)}{(1-\lambda+h\lambda)^{n+1}}\,.
\end{align}
%
By Lemma \ref{lem:h_diff_commutation}, the manipulations leading to
the left most equality in equaiton
\eqref{eq:Diff_g_ghat_relation_Integral_equivalence} are 
valid.  Moreover, for $n=1$, equation \eqref{eq:Complex_Diff_g}
implies that
%
\begin{align}\label{eq:Complex_IE} 
  0\equiv\int_0^\infty\frac{d\Phi_1(y)}{|1+hy|^4}=\int_0^1\frac{\frac{\lambda}{1-\lambda}\frac{\lambda\,d\varrho(\lambda)}{(1-\lambda)^2}}
             {|1+hy|^4}
    =\int_0^1\frac{\lambda^2(1-\lambda)\,d\varrho(\lambda)}{(1-\lambda+h\lambda)^4}\,.
\end{align}
%
for all $h\in\mathcal{U}$ such that $h_i\neq0$ and $p\in[0,1]$. By Lemma
\ref{lem:h_diff_commutation}, the manipulations leading to 
the left most equality in \eqref{eq:Complex_IE} are also
valid. Therefore, the form \eqref{eq:Measure_consistentcy_condition} of
$\varrho(d\lambda)$ satisfies the conditions required by equations
\eqref{eq:n=0_measure_equivalence} \emph{and} 
\eqref{eq:Diff_g_ghat_relation_Integral_equivalence}--\eqref{eq:Complex_IE}.

Equation \eqref{eq:Measure_consistentcy_condition} generalizes a
result found in \cite{Day:JPCM-96} regarding random resistor networks
(In \cite{Day:JPCM-96} $W_1(p,\lambda)\equiv0$ for all $p\in[0,1]$). Here we have
shown that there is also a delta function component of the measure
$\varrho(d\lambda)$ at $\lambda=1$ present when $p>p_c$. This result is independent of,
and provides a proof of our hypothesis: the existence of a gap in the
spectrum about $\lambda=1$ $(h=0)$ that collapses \emph{precisely} at
$p=p_c$. The qualitative idea is as follows. The operator $\chi\Gamma\chi$ always
has a large null space leading to an essential delta function at
$\lambda=0$ for all $p\in[0,1]$. Although as $p\to1$ the characteristic function
$\chi\to I$, the identity operator, and $\chi\Gamma\chi\to\Gamma$, and a delta function at
$\lambda=1$ must form (the projection operator $\Gamma$ only has spectrum in the
set $\{0,1\}$). The proof of Lemma
\eqref{lem:Measure_consistentcy_condition} verifies that the delta
function at $\lambda=1$ forms \emph{precisely} at the percolation threshold,
and gives a characterization of the phase transition.  
%

% If in two-column mode, this environment will change to single-column format so that long equations can be displayed. 
% Use only when necessary.
%\begin{widetext}
%$$\mbox{put long equation here}$$
%\end{widetext}

% Figures should be put into the text as floats. 
% Use the graphics or graphicx packages (distributed with LaTeX2e).
% See the LaTeX Graphics Companion by Michel Goosens, Sebastian Rahtz, and Frank Mittelbach for examples. 
%
% Here is an example of the general form of a figure:
% Fill in the caption in the braces of the \caption{} command. 
% Put the label that you will use with \ref{} command in the braces of the \label{} command.
%
% \begin{figure}
% \includegraphics{}%
% \caption{\label{}}%
% \end{figure}

% Tables may be be put in the text as floats.
% Here is an example of the general form of a table:
% Fill in the caption in the braces of the \caption{} command. Put the label
% that you will use with \ref{} command in the braces of the \label{} command.
% Insert the column specifiers (l, r, c, d, etc.) in the empty braces of the
% \begin{tabular}{} command.
%
% \begin{table}
% \caption{\label{} }
% \begin{tabular}{}
% \end{tabular}
% \end{table}

% If you have acknowledgments, this puts in the proper section head.
%\begin{acknowledgments}
% Put your acknowledgments here.
%\end{acknowledgments}

% Create the reference section using BibTeX:
\bibliography{murphy}

\end{document}
%
% ****** End of file aiptemplate.tex ******
