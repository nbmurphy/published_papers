\documentclass[english,12pt]{ttuthes}
\usepackage{amsmath,amssymb,latexsym,amsthm,babel,latexsym}
\usepackage{yfonts}
\usepackage{graphics}
\usepackage{graphicx}
\usepackage{mathrsfs}

%\usepackage[dvips]{graphicx}  

%various other packages
%\usepackage{tree-dvips}
%\usepackage{ecltree}
%\usepackage{epic}
%\usepackage{qtree} <-- this is perhaps the better package for drawing trees
%\usepackage{lscape}

\newtheorem{corollary}{Corollary}[chapter]
\newtheorem{lemma}{Lemma}[chapter]
\newtheorem{theorem}{Theorem}[chapter]
\newtheorem{proposition}{Proposition}[chapter]
\newtheorem{definition}{Definition}[chapter]
\newtheorem{notations}{Notations}[chapter]
\newtheorem{notation}{Notation}[chapter]
\newtheorem{assumption}{Assumption}[chapter]
\newtheorem{remark}{Remark}[chapter]
\newtheorem{problem}{Problem}[chapter]
\newtheorem{conjecture}{Conjecture}[chapter]
\renewcommand{\thechapter}{\Roman{chapter}}
\renewcommand{\thesection}{\arabic{chapter}.\arabic{section}}
\renewcommand{\theequation}{\arabic{chapter}.\arabic{equation}}
\renewcommand{\thefigure}{\arabic{chapter}.\arabic{figure}}
\renewcommand{\theremark}{\arabic{chapter}.\arabic{remark}}
\renewcommand{\thelemma}{\arabic{chapter}.\arabic{lemma}}
\renewcommand{\thecorollary}{\arabic{chapter}.\arabic{corollary}}
\renewcommand{\thetheorem}{\arabic{chapter}.\arabic{theorem}}

%%
%% Other shortcuts
%%
\newcommand{\Hc}{\mathcal{H}}
\newcommand{\Fc}{\mathcal{F}}
\newcommand{\Wc}{\mathcal{W}}
\newcommand{\ph}{\hat{\phi}}
\newcommand{\gh}{\hat{\gamma}}
\newcommand{\Dh}{\hat{\Delta}}
\newcommand{\dha}{\hat{\delta}}
\newcommand{\qh}{\hat{q}}
\newcommand{\xh}{\hat{x}}
\newcommand{\HM}{\mathcal{H}_{\text{max}}}
\newcommand{\Hm}{\mathcal{H}_{\text{min}}}
\newcommand{\sech}{\rm \hspace{0.7mm}sech}
\newcommand{\I}{\mathrm{i}}
\newcommand{\hh}{\hat{h}}
\newcommand{\mh}{m_r}
\newcommand{\mt}{m_i}
\newcommand\beps{\mbox{\boldmath${\epsilon}$}}
\newcommand\bmu{\mbox{\boldmath${\mu}$}}
\newcommand\bsig{\mbox{\boldmath${\sigma}$}}


%% The following is my command for writing fractions in text; E.g., ...then, $\gf{1}{2}$ is...
\newcommand{\gf}[2]{\genfrac{}{}{0.5pt}{0}{#1}{#2}}

\begin{document}
\title{Phase Transitions in Binary Composite Media} 
\author{N. Benjamin Murphy}

%%
\maketitle

\frontmatter
\chapter{ACKNOWLEDGMENTS}


% To thank someone or write in another language 
% If you compile this file, the following will appear in Russian.
% \selectlanguage{russian}
% \"I\^\i-\dh\'o\~n\~n\^e\`e\'e
% \selectlanguage{english}
% In English

\tableofcontents

\chapter{ABSTRACT}
Connectedness transitions in disordered composite media are
characterized using techniques of statistical mechanics, percolation
theory, and random matrix theory. In particular, we utilize and extend
these techniques to study conduction/insulation phase transitions in
classical two--component conductors, liquid/solid phase transitions in
electrorheological (ER) suspensions, and microstructural and transport
transitions in sea ice and osteoporotic bone. It is the
random matrix (or operator) at the heart of the analytic continuation
method of homogenization theory which leads to such a broad range of
methods. 

In electrified binary composite media, spectral representations of the
system energy capture the complicated geometric interactions
\emph{exactly} and decouple all system parameters. This important
feature leads to a canonical ensemble statistical mechanics framework,
which is both physically transparent and 
mathematically tractable. The framework is applied to polarized binary
dielectric media, giving a novel characterization of the
thermo/electric phase transitions occurring in ER fluids with sphere
radii $r_s\lesssim1\mu m$. We also address a central unresolved issue in
electrorheology, the existence of a critical applied electric field
strength for ER fluids with sphere radii $r_s\sim10\mu m$. In this regime
thermal effects are negligible and the behavior of the system cannot
be described using classical statistical mechanics. We apply and
extend generalized techniques of statistical mechanics, and
demonstrate that the critical field strength may be characterized by
physically accessible poles of a logarithmic potential representation
of the entropy of microstructure.   

We characterize critical transport transitions in binary composite
media by the critical behavior of spectral representations of the 
effective parameter. Our numerical calculations of the underlying
spectral measure, for various composite microstructures, indicate that
the geometric connectedness of the system is manifested by spectral
resonances and gaps in the measure. These simulations validate a key
assumption made 18 years ago which lead to a rigorous critical theory
for binary composites, as the system percolates the spectral gaps
collapse. We extend this critical theory to the complex case,
recovering Baker's (two--parameter) scaling relations of the Ising
model, and numerically verify these relations for random resistor
networks (RRNs).   

We finally show that connectedness transitions in binary composites
may also be characterized using random matrix theory (RMT). We
numerically demonstrate that the geometric resonance statistics of
RRNs undergo a transition, similar to the Anderson transition of
mesoscopic/quantum conductors, as the volume fraction of open bonds
$p$ is varied. As the open bonds percolate and the correlation length
diverges, the geometric resonance statistics transition from
uncorrelated Poisson--like statistics towards highly correlated
Wigner--Dyson statistics. Numerical fitting to the plasma model of
RMT gives compelling evidence that the random matrix mentioned above is
in the one parameter universality class of q-deformed random matrix
ensembles with orthogonal symmetry. We also show that there is a
strong fluctuation of eigenfunctions at the critical point in RRNs, a
prominent hallmark of the Anderson transition. These fluctuations can
be represented by a set of inverse participation ratios (IPRs)
$P_q$. Numerical calculations indicate that for fixed $p$ the
distribution function of the IPRs has a scale invariant
form. Furthermore, we show that the scaling behavior of the
ensemble--averaged $\langle P_q\rangle$ is described by the fractal dimension
$D_q$.  

%comment out \listoffigures if you lack figures
\renewcommand{\baselinestretch}{1}\selectfont
\listoffigures \addcontentsline{toc}{chapter}{LIST OF FIGURES}
\renewcommand{\baselinestretch}{1.5}\selectfont

\mainmatter \setcounter{page}{1} \pagenumbering{arabic}
%
%%%%%%%%%%%%%%%%%%%%%%%%%%%%%%%%%%%%%%%%%%%%%%%%%%%%%%%%%%%%%%%%%%%%%%%%%%%%%%%%%%%%%%%%%%%%

% 
\chapter{INTRODUCTION}
\setcounter{equation}{0}
%
PUT A FIGURE HERE WITH AN ER FLUID, SEA ICE BRINE CHANNELS, TRABECULAR
BONE, A p=0.5 RANDOM RESISTOR NETWORK, AND A PERCOLATING ISING MODEL.
Composite materials have found a well rooted place in the sciences and
engineering technology due to their countless applications. Their
utility follows from the characteristic of combining
attributes of the constituent materials \cite{MILTON:2002:TC}. They
have even greater functionality when the composite has the ability to
transition between the various underlying and/or combined attributes
of the constituents. This is a key feature of high contrast disordered  
composites. Namely, the critical dependence of the effective transport
properties on system parameters, such as volume fraction, component
contrast ratio, applied field strength, etc. The behavior of such
media is particularly challenging to describe physically, and to
predict mathematically. 

For example, electrorheological (ER) fluids, which are suspensions of
spherical particles in a viscous dielectric fluid, undergo a
liquid--to--solid phase transition as an applied electric field
strength surpasses a critical value $E_c$
\cite{Halsey:S-761,Tao-EF,Wen:APL-2147,Wen:PF-1826}. Plastic or glass
dielectric spheres form clusters and then chains which coalesce into
columns with periodic lattice arrangements of the spheres as the field
increases. Metal spheres, which form necklaces that coalesce into
fractal net structures as the field increases, also undergo an
electrically-induced transition in the connectedness of the spheres
within a few milliseconds \cite{Wen:SoftMatter-200}. The resultant
rapid electrically induced transition in the viscosity has been
exploited in many engineering applications including clutches, brakes,
human prosthetics, micro--fluidic valves, and automotive suspension
systems \cite{Wen:SoftMatter-200} (and references therein).  

Another composite which displays complex critical behavior is sea ice,
consisting of pure ice with sub--millimeter brine inclusions, whose
volume fraction $\phi$, geometry, and connectedness vary significantly
with temperature $T$. The polar ice packs are both indicators and
agents of climate change. They also host extensive algal and bacterial
communities, which live in the brine inclusions and sustain life in
the polar oceans. Fluid flow through porous sea ice mediates a broad
range of processes such as the growth and decay of seasonal ice, the
evolution of surface melt ponds and ice pack reflectance, and biomass
build--up \cite{Golden:NAMS:2009}. In \cite{Golden:S-2238} the
\emph{on--off switch} for for fluid transport in sea ice was
identified. For brine volume fractions $\phi$ below about $5\%$, columnar
sea ice is effectively impermeable to fluid flow, while for $\phi$ above
$5\%$, it is increasingly permeable. This critical brine volume fraction
$\phi_c\approx5\%$ corresponds to a critical temperature $T_c\approx5^{\circ}$C for a
typical bulk sea ice salinity of $5$ parts per thousand, which is
known as the \emph{rule of fives}. Fluid flow is facilitated by
\emph{brine channels}--connected brine structures ranging in the scale
from a few centimeters for horizontal cross--sections to a meter or
more in the vertical direction. The critical behavior of fluid flow
through sea ice was postulated in \cite{Golden:S-2238} to result from a
temperature--driven transition in connectedness of the brine
microstructure, or a \emph{percolation threshold}. The $5\%$ critical
brine volume fraction in sea ice was predicted by noting the close
similarity of its microstructure to that of stealthy, radar absorbing
materials, and adapted for sea ice a continuum percolation model for
compressed powders \cite{Kusy:JAP-5301,Kusy:N-58} used in the design
of these materials.

Bone has a complex, porous microstructure whose characteristics depend
on the type of bone, such as cortical or trabecular, as well as the
age and health of the individual. The strength of bone and its ability
to resist fracture depend strongly on this porous microstructure, in
particular, on the \emph{quality} of the connectedness of the hard,
solid phase. For example, in dense cortical bone the pores can be
sparse and disconnected, yet exhibit increasing volume fraction and
connectivity with the onset of osteoporosis. Trabecular or spongy bone
displays a broad range of biconnected microstructures, ranging from a
solid bone matrix containing numerous connected pores, to sparse solid
fibers within a dominant, connected pore space. With the onset of
osteoporosis the trabecular bone can become more disconnected and
remaining connections can become more tenuous or fragile, making
injuries such as hip fractures more common. The development of
non--destructive methods to determine the state of osteoporotic bone
microstructure is crucial to the care and comfort of patients with
this crippling condition. There have been many studies of bone
structure and mechanics, and their dependence on aging, changes in
porosity, etc., for example \cite{}.

Lattice and continuum percolation models have been used to study a
broad range of materials including rocks
\cite{Bourbie:JGR-11524,Broadbent:PCPS-629}, semiconductors
\cite{Efros-84}, thin films \cite{Davis:OC-70}, glacial ice
\cite{Enting:1985:LSM}, polycrystalline metals \cite{Chen:PRL-035701},
radar absorbing coatings \cite{Kusy:N-58}, and carbon nanotube
composites \cite{Kyrylyuk:PNAS-8221}. In the simplest case of the two
dimensional square lattice \cite{Stauffer-92,Torquato:RHM-02}, the
bonds are open with probability $p$ and closed with probability
$1-p$. Connected sets of open bonds are called open clusters. The
average cluster size grows as $p$ increases, and there is a critical
probability $p_c$, $0<p_c<1$, called the \emph{percolation threshold},
where an infinite cluster of open bonds first appears. In $d=2$,
$p_c=1/2$, and in $d=3$, $p_c\approx0.25$. Now consider transport through
the associated \emph{random resistor network} (RRN), where the bonds
are assigned electrical conductivities $\sigma_1$ with probability $1-p$,
and $\sigma_2=1$ with probability $p$. Then the effective of bulk
conductivity $\sigma^*(p)$ exhibits critical behavior as $h=\sigma_1/\sigma_2\to0$. For
$p<p_c$, $\sigma^*(p)=0$, while $\sigma^*(p)>0$ for $p>p_c$, with
$\sigma^*(p)\sim(p-p_c)^t$, as $p\to p_c^+$, where $t$ is the critical exponent,
believed to be \emph{universal} for lattices, depending only on
dimension.

The critical behavior of the RRN is reminiscent of a phase transition
in statistical mechanics
\cite{Clerc:AP-191,Bergman:SSP-147,Efros:PSSB-303,Hong:PRB-4564}, like
that exhibited by an Ising ferromagnet
\cite{Thompson-1988,Christensen-2005} around its Curie point at a
critical temperature $T=T_c$, as the applied magnetic field $H\to0$. The
formulations and basic physics of these two classes of problems are
nevertheless quite different. In
\cite{Golden:JMP-5627,Golden:PRL-3935}, however, it was observed that
the \emph{analytic continuation method} for bounding effective
transport properties of composites
\cite{Bergman:PRL-1285,Milton:APL-300,Golden:CMP-473} provides a
mathematical link between them, through the Lee--Yang--Ruelle Theorem in
statistical mechanics
\cite{Yang:PR:404,Lee:PR:411,Baker-1990,Baker:PRL-990,Ruelle:PRL:303,Ruelle:AM:589}.
This theorem, which states that the zeros of the partition function
lie on the unit circle in an appropriate variable, yields a
logarithmic potential integral representation for the free energy of
an Ising model, and a Stieltjes representation for the magnetization
$M(H)$. In the analytic continuation method, $m=\sigma^*/\sigma_2$ is an
analytic function of $h$, taking the upper half plane to the upper
half plane. A Stieltjes integral representation for $F(s)=1-m(h)$,
where $s=1/(1-h)$, is used to obtain rigorous bounds on $\sigma^*$ given
microstructural information. This formula is based on a resolvent
representation for the electric field, involving a self--adjoint
random matrix (or operator). All the geometry of the composite is
incorporated through a spectral measure of the matrix. Through these
\emph{mathematical} parallels the two classes of problems can be
placed on an equal footing. This analytic parallel has been exploited
to show that the critical exponents of transport in both the lattice and
continuum percolation models obey the classical scaling relations of
statistical mechanics \cite{Golden:PRL-3935}. Here, we establish the
physics underlying the connection, extend the framework to the
quasi-static regime, and explore further the mathematics common to both
types of problems.

Random matrix theory was first introduced by Wigner and Dyson to 
describe the quantized energy levels of complex nuclei. This framework
is now pervasive in many areas of physics and pure and applied
mathematics, including analytic number theory
\cite{Mezzadri:2005:RMT_RH}, integrable systems and quantum chaos
\cite{Forrester:2010:RMT}, biological networks \cite{Luo_etal:PLA:06},
random graph theory \cite{Bandyopadhyay:PRE_SNSMP:07,Kuhn},
mesoscopic conductors \cite{Muttalib_etal_qRME}, and random growth
processes \cite{Forrester:2010:RMT} (see
\cite{Guhr,Brody_etal,Konig,Mirlin:PhysRep:259} for reviews). The
first random matrix ensembles (RMEs) to be studied were those with
probability density $P(\mathbf{H})\propto\exp[-\text{Tr}V(\mathbf{H})]$,
$V(\lambda)=\lambda^2/2$, known as Gaussian ensembles. The Wigner-Dyson (WD)
statistics derived from these ensembles are characterized by strong
eigenvalue correlations giving rise to the phenomenon of level
repulsion \cite{Mehta:2004:RMT}.

Many disordered systems exhibit WD statistics in a particular regime, 
but significantly deviate from these statistics toward Poisson--like
statistics as the disorder in the the system transitions 
\cite{Luo_etal:PLA:06,Muttalib_etal_qRME,Bandyopadhyay:PRE_SNSMP:07,Canali}
(and references therein). In the WD regime the rescaling of eigenvalues  
allows exact calculation of all relevant statics, in the infinite
volume limit, completely independent of any parameter
\cite{Mehta:2004:RMT}. In this limit the statistics are 
translationally invariant and independent of the detailed form of the
function $V$ \cite{Canali}. Indeed, for all $V$ such that
$V(\lambda)\gtrsim|\lambda|^\alpha$, $\alpha\geq1$, $\lambda\gg1$ the zero parameter WD statistics hold,
illustrating that the statistics truly are universal
\cite{Mehta:2004:RMT,Canali}. When $\alpha<1$ the translational invariance
symmetry of the scaled eigenvalues is broken and the system undergoes a
``phase transition'' \cite{Canali}. Functions of the form
$V(\lambda)=A\ln^2|\lambda|$, $A>0$ accurately describe the transition toward  
Poisson-like statistics \cite{Muttalib_etal_qRME,Canali}. Using
functions $V$ related to weight functions of q-deformed orthogonal   
polynomials, a one parameter generalization of the zero parameter WD
RMEs has allowed the entire course of this disorder driven transition
to be explained by a single model
\cite{Muttalib_etal_qRME}. Remarkably, Dyson's Brownian motion model
of the eigenvalues, with the Gaussian RMEs as equilibrium
distributions, has also been generalized to include these q-deformed
random matrix ensembles \cite{Blecken:JPA:31}.  

Powerful field-theoretic (FT) techniques have shown analytically that
this model describes exactly the local fluctuations of energy levels
in metals and is more than a simple phenomenological conjecture 
\cite{Efetov}. The theory has been extended to describe the strong 
fluctuations in eigenfunctions of disordered systems near a critical
point \cite{Kravtsov:PRL:1913,Mirlin:PhysRep:259}. The supersymmetry FT
techniques play a very important role in the unification of disordered
systems, quantum chaos, and all other models exhibiting this
transition \cite{Efetov}. Here we demonstrate that macroscopic
electric systems also exhibit such a transition. Our results illustrate
strikingly that macroscopic electric systems exhibit the same
universal fluctuations as mesoscopic electric systems of size
$10^{-6}m$ and systems nine orders of magnitude smaller.   
%
%

%%%%%%%%%%%%%%%%%%%%%%%%%%%%%%%%%%%%%%%%%%%%%%%%%%%%%%%%%%%%%%%%%% 
%
%
%write in ALL CAPS when naming the chapter
\chapter{THE ANALYTIC CONTINUATION METHOD}
\label{ch:The_Analytic_Cont_Meth}

%normal capitalization in section
\section{Overview}
\label{sec:Background_TACM}

% 
In this section we discuss the effective parameters for binary
dielectric composite media. Although, the framework applies to many
classical transport problems including electrical conductivity and
permittivity, magnetic permeability, elasticity, thermal conductivity,
etc. \cite{MILTON:2002:TC}. We now give a brief overview of the
analytic continuation method
\cite{Bergman:PRL-1285,Milton:APL-300,Golden:CMP-473}.   
 
Let $(\Omega,P)$ be a probability space and let $\beps(\vec{x},\omega)$ be
the local permittivity, a (spatially) stationary random field in
$\vec{x}\in \mathbb{R}^d$ and $\omega \in \Omega$, where $\Omega$ is the set of all
realizations of our random medium, and $P(d\omega)$ is the underlying
probability measure, compatible with stationarity
\cite{Golden:CMP-473}. We assume $\beps(\vec{x},\omega)$ takes the
values $\epsilon_1$ and $\epsilon_2$ and write $\beps(\vec{x},\omega) = \epsilon_1 \chi_1
(\vec{x},\omega) + \epsilon_2 \chi_2(\vec{x},\omega)$, where $\chi_j$ is the
characteristic function of medium $j=1,2$, which equals one for all $\omega
\in \Omega$ having medium $j$ at $\vec{x}$, and zero otherwise
\cite{Golden:CMP-473}  

Let $\vec{E}(\vec{x})$ and $\vec{D}(\vec{x})$ be the stationary random
electric and displacement fields, related by $\vec{D}= \beps
\vec{E}$, satisfying \cite{Golden:CMP-473} 
%
\begin{align}\label{eq:Maxwells_Equations_ED}  
	\vec{\nabla}\times\vec{E}=0, \quad
	\vec{\nabla}\cdot\vec{D}=0,\quad \quad	
	\vec{E}=\vec{E}_0+\vec{E}_f, \quad
	\langle\vec{E}\rangle=\vec{E}_0,
\end{align}
%
where $\langle \cdot \rangle$ denotes ensemble average over $\Omega$ or, by an ergodic
theorem \cite{Golden:CMP-473}, spatial average
over all of ${\mathbb{R}}^d$ and $\vec{E}_f$ is the fluctuating field
with mean zero about $\vec{E}_0$. We write $\vec{E}_0=E_0\vec{e}_k$,
where $\vec{e}_k$ is a unit vector, for some $k = 1, \ldots, d$. The
effective complex permittivity tensor $\beps^*$ is defined as 
%
\begin{equation}                                    \label{eq:eff_eps_def}
    \langle \vec{D} \rangle=  \beps ^* \langle \vec{E} \rangle.
    %\quad \text{and}  \quad 
    %\langle \vec{E} \rangle=  (\beps^*)^{-1} \langle \vec{D} \rangle,
\end{equation}
%
For simplicity we focus on one diagonal
coefficient $\epsilon^*:=\beps^*_{kk}$. Central to our studies is the,
system energy given by $\frac{1}{2}\;\langle\vec{D}\cdot\vec{E}\rangle$
\cite{Jackson-1999}. A key variational calculation \cite{Golden:CMP-473}
yields the energy constraint $\langle\vec{D}\cdot\vec{E}_f\rangle=0$. Therefore
the energy may be expressed as  
%
\begin{equation}                                    \label{eq:Reduced_Energy}
    \frac{1}{2}\;\langle\vec{D}\cdot\vec{E}\rangle
      =\frac{1}{2}\;\langle\vec{D}\rangle\cdot\vec{E}_0
        =  \frac{1}{2}\;\epsilon^* E_0^2.
\end{equation}
%
Due to the homogeneity of $\epsilon^*$, $\epsilon^*(a\sigma_1,a\sigma_2)=a\epsilon^*(\sigma_1,\sigma_2)$, it
depends only on the ratio $h = \epsilon_1/\epsilon_2$ and we define $m(h)
=\epsilon^*/\epsilon_2$. The function $m(h)$ is analytic off the negative real axis
in the $h$--plane, taking the upper half plane to the upper half plane
\cite{Golden:CMP-473}. For definiteness we assume that $|\epsilon_1|<|\epsilon_2|$
so that $0<|h|<1$. Metal inclusions are modeled by letting $|\epsilon_2|\to\infty$
or $|h|\to0$ \cite{Wen:SoftMatter-200,Jackson-1999}.    

The key step in the method is obtaining an integral representation for
$\epsilon^*$. It is more convenient to consider $F(s):=1-m(h)$, $s=1/(1-h)$,
which is analytic off $[0,1]$ in the $s$--plane
\cite{Bergman:PRC-377,Golden:CMP-473}.  Then \cite{Golden:CMP-473}
%
\begin{align} \label{eq:Fs_Integral}
	F(s)=1-m(h)
	=\int_0^1 \frac{d\mu(\lambda)}{s-\lambda}\;,
   \qquad m(h)=\frac{\epsilon^*}{\epsilon_2},
   \quad s = \frac{1}{1-\epsilon_1/\epsilon_2}\;,   
 \end{align}
%
where $\mu$ is a positive measure on $[0,1]$, and by definition
$0\leq|F(s)|\leq1$. Equation \eqref{eq:Fs_Integral} is a
general formula holding for two component stationary random media in
the lattice and continuum settings \cite{Golden:PRL-3935}. For later
physical considerations we also define the function $w(z)=\epsilon^*/\epsilon_1$,
$z=1/h$ which is analytic of the negative real axis in the $z$--plane, 
taking the upper half plane to the upper half plane
\cite{Golden:CMP-473}. Again, it is more convenient to consider
$G(t):=1-w(z)$, $t=1/(1-z)=1-s$, which is analytic off $[0,1]$ in the 
$t$--plane \cite{Bergman:PRC-377,Golden:CMP-473,Bergman:AP-78}. Then  
%
\begin{equation}\label{eq:Gt_Integral}
	G(t)=1-w(z) 
	=\int_0^1\frac{d\alpha(\lambda)}{t-\lambda}\;,
        \qquad w(z)=\frac{\epsilon^*}{\epsilon_1},
   \quad t=1-s\;,
\end{equation}
%
where $\alpha$ is a positive measure on $[0,1]$, and by definition
$-\infty<|G(t)|\leq0$.

Stieltjes transforms of other measures involving $[\beps^{-1}]^*$ may
be similarly found by writing the inverse permittivity as
$[\beps(\vec{x},\omega)]^{-1}=\chi_1(\vec{x},\omega)/\epsilon_1+\chi_2(\vec{x},\omega)/\epsilon_2$. For
simplicity we focus on one diagonal coefficient
$1/\epsilon^*=[\epsilon^{-1}]^*:=[\beps^{-1}]^*_{kk}$. By the homogeneity of
$[\epsilon^{-1}]^*$ we define the functions $\tilde{m}(h)=\epsilon_1/\epsilon^*$ and
$\tilde{w}(z)=\epsilon_2/\epsilon^*$. The functions $E(s):=1-\tilde{m}(h)$, which is 
analytic off $[0,1]$ in the $s$--plane, and $H(t):=1-\tilde{w}(z)$,
which is analytic off $[0,1]$ in the $t$--plane, have the following
integral representations \cite{Bergman:AP-78,Golden:CMP-473} 
%
\begin{align}\label{eq:Es_Integral}
  E(s)&=1-\tilde{m}(h)=\int_0^1\frac{d\eta(\lambda)}{s-\lambda}\;
\qquad \tilde{m}(h)=\frac{\epsilon_1}{\epsilon^*}, \\
\label{eq:Ht_Integral}
H(t)&=1-\tilde{w}(z)=\int_0^1\frac{d\tau(\lambda)}{t-\lambda}\;,
\qquad \tilde{w}(z)=\frac{\epsilon_2}{\epsilon^*}
\end{align}
% 
We have conformed to the notation in \cite{Bergman:AP-78}, which is
far from standardized. Due to the beautiful symmetries in these
functions \eqref{eq:Fs_Integral}--\eqref{eq:Ht_Integral}, without
loss of generality, the rest of this section is focused on $F(s)$ and
$\mu$.    

These Stieltjes transforms of $\mu$ etc. are examples of
Herglotz functions. As will be shown in section
\ref{subsec:Spec_Decomp_Energy} below, applying The Spectral Theorem
\cite{Reed-1980} to the energy constraint  $\langle\vec{D}\cdot\vec{E}_f \rangle=0$
yields 
% 
\begin{equation} \label{eq:Energy_Constraint}
	\langle E_f^2\rangle=E_0^2\int_0^1\frac{\lambda \; d\mu(\lambda)}{(s-\lambda)^2}\;, 
\end{equation}
% 
which in turn leads to an exact, detailed description of all other
energy contributions within the system in terms of Herglotz functions
involving $\mu$. A key feature of formulas
\eqref{eq:Reduced_Energy}--\eqref{eq:Energy_Constraint}
%\eqref{eq:Fs_Integral},\eqref{eq:Gt_Integral}, and
%\eqref{eq:Energy_Constraint}
is that the parameter information in $s$ 
and $E_0$ is {\it separated} from the geometry of the composite
encapsulated in $\mu$, through its moments $\mu_i=\int_0^1 \lambda^i d\mu(\lambda)$, which
depend on the correlation functions of the medium
\cite{Golden:CMP-473}. For example, $\mu_0=p_1$, the volume fraction of
$\epsilon_1$. The Stieltjes integral representation for $F(s)$ may be used to
obtain rigorous bounds on $\epsilon^*$ given microstructural information 
\cite{Bergman:PRL-1285,Milton:APL-300,Golden:CMP-473,Bergman:AP-78}. 

As will be shown in section \ref{sec:Resolv_Rep_E_D} below, these
Stieltjes formulas arise from the resolvent representation of the
electric field $\vec{E}$ and displacement field $\vec{D}$. For
example, $\vec{E} = s(s + \Gamma \chi_1)^{-1}\vec{E}_0$, is obtained via
manipulation of \eqref{eq:Maxwells_Equations_ED}, where $\Gamma=\vec{\nabla}(-\Delta)^{-1}\nabla\cdot$,
yielding $F(s)=\langle\chi_1[(s+\Gamma\chi_1)^{-1}\vec{e}_k]\cdot\vec{e}_k \rangle$. The operator
$\Gamma$ is a projection onto curl-free fields, based on convolution with
the free-space Green's function for $-\Delta$ \cite{Golden:CMP-473}. In the
Hilbert space $L^2(\Omega,P)$ with weight $\chi_1$ in the inner product,
$\Gamma\chi_1$ a bounded linear self--adjoint integro--differential operator
with simple spectrum \cite{Golden:CMP-473}. Moreover, the
underlying self--adjoint operator, $\mathbf{M}:=\chi_1\Gamma\chi_1$ is
composition of projection operators so that $\|\mathbf{M}\|\leq1$, where
$\|\cdot\|$ denotes the operator norm on $L^2(\Omega,P)$. Formula
\eqref{eq:Fs_Integral} involves a spectral representation of the
resolvent $(s+\Gamma\chi_1)^{-1}$, where $\mu$ is a spectral measure of the
family of projections of $-\Gamma\chi_1$, in the $\langle\vec{e}_k,\vec{e}_k\rangle$ state
\cite{Golden:CMP-473,Reed-1980}.       
%
\section{Resolvent Representations of Physical Fields}
\label{sec:Resolv_Rep_E_D}
%
In this section we derive two resolvent representations for the
electric field $\vec{E}$ and two resolvent representations for the
displacement field $\vec{D}$. The Spectral Theorem \cite{Reed-1980} and the
resolvent representations of the electric field lead to integral
representations of the effective parameter $\epsilon^*$ involving Stieltjes
transformations of the the measures $\mu$ and $\alpha$, introduced in section 
\ref{sec:Background_TACM}. Similarly, the two resolvent
representations of the displacement field lead to integral
representations of the inverse of the effective parameter $[\epsilon^*]^{-1}$
involving Stieltjes transformations of the measures $\eta$ and
$\tau$. Furthermore, we use the Stieltjes-Perron inversion formula
\cite{Henrici:1974:v2,Henrici:1974:v3} to identify the relationships
between the measures $\mu$ and $\alpha$, and $\eta$ and $\tau$. We now derive the
resolvent representations for the electric field.

Define the Hilbert space of stationary random fields by $\mathscr{H}_s\subset L^2(\Omega,P)$
where the probability measure $P(d\omega)$ is compatible with
stationarity \cite{Golden:CMP-473}. Furthermore, define the associated
Hilbert space of stationary curl free random fields  
%
\begin{align}\label{eq:curlfreeHilbert}
  \mathscr{H}_\times:=\{\vec{Y}(\omega)\in \mathscr{H}_s \ | \ \vec{\nabla} \times\vec{Y}=0 \text{ weakly and }
\langle\vec{Y}\rangle=0\}, 
\end{align}
%
where $\langle\cdot\rangle$ means ensemble average over $\Omega$, $\omega\in\Omega$, and
$\vec{Y}:\Omega\mapsto\mathbb{R}^d$. Consider the following variational problem: 
find $\vec{E}_f\in\mathscr{H}_\times$ such that   
%
\begin{align}\label{eq:Weak_Curl_Free_Variational_Form}
  \langle\beps(\vec{E}_0+\vec{E}_f)\cdot\vec{Y}\rangle=0 \quad  \forall \
  \vec{Y}\in\mathscr{H}_\times
\end{align}
%
where $\vec{E}_0$ is the external uniform electric field applied to
the system \cite{Golden:CMP-473}. Under the assumption that the
bilinear form
$a(\vec{u},\vec{v})=\vec{u}^{\;T}\beps(\vec{x},\omega)\vec{v}$, where
$\vec{x},\vec{u},\vec{v}\in\mathbb{R}^d$, is bounded and coercive, this
problem has a unique solution \cite{Golden:CMP-473} satisfying
\eqref{eq:Maxwells_Equations_ED}.    

As in section \ref{sec:Background_TACM}, for simplicity, we
focus on one diagonal coefficient of the effective permittivity
$\epsilon^*=\beps^*_{kk}$, defined by
%

\begin{align}\label{eq:Effective_Permittivity}
  \epsilon^*\vec{E}_0&:= \langle\vec{D}\rangle= \langle(\chi_1\epsilon_1+\chi_2\epsilon_2)\vec{E}\rangle
              =\langle\epsilon_2(h\chi_1+\chi_2)\vec{E}\rangle
              =\langle\epsilon_2(1-\chi_1/s)\vec{E}\rangle\notag \\             
              &=\epsilon_2(\vec{E}_0-\langle\chi_1\vec{E}\rangle/s)
\end{align}
%
where $h=\epsilon_1/\epsilon_2$, $s=1/(1-h)$, and in the last step we have used
$\langle\vec{E}\rangle=\vec{E}_0=E_0\vec{e}_k$, where $\vec{e}_k$ is a
unit vector. The same manipulations in equation 
\eqref{eq:Effective_Permittivity}, applied to $\vec{\nabla}\cdot\vec{D}=0$,
yields    
%
\begin{align} \label{eq:DivEquation}
  \vec{\nabla}\cdot(1-\chi_1/s)(\vec{E}_0+\vec{E}_f)=0.
\end{align}
%
We note that the following manipulations are formal in the
sense that the formula $\vec{\nabla}\cdot\vec{D}=0$ is ill defined, as
$\chi_1(\vec{x},\omega)$ hence $\epsilon(\vec{x},\omega)$ is not differentiable for all
$\vec{x}\in\mathbb{R}^d$. This formalism can be made rigorous using a
weak formulation \cite{Golden:CMP-473,Folland:95}. We have
$\vec{E}_f=-\vec{\nabla}\varphi$ weakly, as $\vec{E}_f\in\mathscr{H}_\times$ \cite{Griffiths-1999}. 
% This follows from the result that $\vec{Y}$ is curl free if
% and only if $\vec{Y}=\vec{\nabla}(\varphi+c)$ for some scalar field $\varphi$ and an
% arbitrary constant $c$ \cite{Griffiths-1999}.
Therefore, equation \eqref{eq:DivEquation} may be written  
%
\begin{align}\label{eq:Resolvent_Rep_Es}
  0&=\vec{\nabla}\cdot(\vec{E}_f-\chi_1\vec{E}/s)=-\Delta\varphi-\vec{\nabla}\cdot(\chi_1\vec{E}/s)\Rightarrow
  \notag\\
0&=\vec{E}_f+\vec{\nabla}\left[(-\Delta^{-1})\vec{\nabla}\cdot(\chi_1\vec{E}/s)\right]:=
\vec{E}_f+\Gamma\chi_1\vec{E}/s \Rightarrow\notag\\
s\vec{E}_0&=(s+\Gamma\chi_1)\vec{E}\Rightarrow\quad
\boxed{\vec{E}=s(s+\Gamma\chi_1)^{-1}\vec{E}_0}\;,
\end{align}
%
The boxed formula in equations \eqref{eq:Resolvent_Rep_Es} gives a
resolvent representation of the electric field \cite{Stone:64}. The
operator $-\Gamma=-\vec{\nabla}(-\Delta)^{-1}\vec{\nabla}\cdot$ is a projection onto curl-free
fields, based on convolution with the free-space Green's function for
$-\Delta$ \cite{Golden:CMP-473}. Indeed, if $\vec{Y}$ is curl free then
$\vec{Y}=-\vec{\nabla}\varphi$ and     

% 
\begin{align}\label{eq:proj_Gamma}
  -\Gamma\vec{Y}=-\vec{\nabla}(-\Delta)^{-1}\vec{\nabla}\cdot\vec{Y}
              =-\vec{\nabla}(-\Delta)^{-1}(-\Delta)\varphi
              =-\vec{\nabla}\varphi
              =\vec{Y}.
\end{align}
%
Therefore, equations \eqref{eq:Resolvent_Rep_Es} and
\eqref{eq:proj_Gamma} imply that
%
\begin{align}
  -\Gamma\vec{E}=-\Gamma\vec{E}_f=\vec{E}_f, \quad
  -\Gamma\chi_1\vec{E}=s\vec{E}_f, \quad
  -\Gamma\chi_2\vec{E}=t\vec{E}_f, \quad
  t:=1-s,
\end{align}
%
In summary
%
\begin{align}\label{eq:Herglotz_Energy_Rep_E}
  \epsilon^*=\epsilon_2(1-F(s)),\quad
  F(s):=\langle\chi_1(s+\Gamma\chi_1)^{-1}\vec{e}_k\cdot\vec{e}_k\rangle, \quad
  \langle\vec{D}\cdot\vec{E}\rangle=\epsilon^*E_0^2.
\end{align}
%

In the Hilbert space $L^2(\Omega,P)$ with weight $\chi_1$ in the
inner product, $\Gamma\chi_1$ is a bounded self--adjoint operator
\cite{Golden:CMP-473}. Therefore the resolvent $(s+\Gamma\chi_1)^{-1}$ is also self
adjoint with respect to this inner product \cite{Stone:64}. The
projection valued measure form of The Spectral Theorem states
\cite{Reed-1980} 
%
\begin{theorem}[The Spectral Theorem]\label{thm:Spectral_Theorem}
  Let $\mathscr{H}$ be a Hilbert space. There is a one-to-one
  correspondence between self-adjoint operators M and
  projection-valued measures $\{E_\Omega\}$ on $\mathscr{H}$, the
  correspondence being given by 
  %
  \begin{align}
    M=\int_{\mathbb{R}} \lambda dE_\lambda. 
  \end{align}
  %
  If $g(\cdot)$ is a real-valued Borel function on $\mathbb{R}$, then
  %
  \begin{align}
    g(M)=\int_{\mathbb{R}} g(\lambda) dE_\lambda.
  \end{align}
  Furthermore, if $f$ is a cyclic vector, then it is possible to
  represent $\mathscr{H}$ as $L^2(\mathbb{R}, d\mu_f)$ where $\mu_f$ is
  the measure satisfying
  %
  \begin{align}
    \int_{\mathbb{R}} g(\lambda)d\mu_f(\lambda)=(f,g(M)f)
  \end{align}
  %
  in such a way that $M$ becomes multiplication by $\lambda$.
  %
\end{theorem}

% 
The family of projection operators, $E_{\lambda}$, in Theorem
\ref{thm:Spectral_Theorem} is called the \emph{resolution of the
  identity}, $\int_{\mathbb{R}}E_{\lambda}=I$, where $I$ is the identity
operator on $\mathscr{H}$. Applying Theorem to $M=-\Gamma\chi_1$, with
$\mathscr{H}=\mathscr{H}_\times$, $f=\vec{e}_k$, $g(\lambda)=(s-\lambda)^{-1}$, and
$(\cdot,\cdot)=\langle\chi_1\cdot,\cdot\rangle$, where $\langle\cdot,\cdot\rangle$ is the $L^2(\Omega,P)$ inner product,
yields the Bergman--Milton representation of the effective dielectric
constant \eqref{eq:Fs_Integral}
\cite{Bergman:PRL-1285,Milton:APL-300,Golden:CMP-473}.

Another equivalent representation follows from factoring out $\epsilon_1$ in
equations \eqref{eq:Effective_Permittivity}--\eqref{eq:DivEquation}
instead of $\epsilon_2$. Defining $z = 1/h$ and $t=1/(1-z)=1-s$ and noting
the symmetry $\epsilon(\epsilon_1\chi_1,\epsilon_2\chi_2)=\epsilon(\epsilon_2\chi_2,\epsilon_1\chi_1)$ we arrive at  
%
\begin{align}\label{eq:Resolvent_Rep_Et}
   &\epsilon^*\vec{E}_0=\epsilon_1(\vec{E}_0-\langle\chi_2\vec{E}\rangle/t),\notag\\
    &\vec{\nabla}\cdot(1-\chi_2/t)(\vec{E}_0+\vec{E}_f)=0,\notag\\
    &\boxed{\vec{E}=t(t+\Gamma\chi_2)^{-1}\vec{E}_0},
\end{align}
%
and $\vec{E}_f=-\Gamma\chi_2\vec{E}/t$. The boxed formula in equation
\eqref{eq:Resolvent_Rep_Et} gives another equivalent resolvent
representation of the electric field. In summary
%
\begin{align}\label{eq:Herglotz_Energy_Rep_G}
  \epsilon^*=\epsilon_1(1-G(t)),\quad
  G(t):=\langle\chi_2(t+\Gamma\chi_2)^{-1}\vec{e}_k\cdot\vec{e}_k\rangle, \quad
  \langle\vec{D}\cdot\vec{E}\rangle=\epsilon^*E_0^2.
\end{align}
%
Applying Theorem \ref{thm:Spectral_Theorem} to $M=-\Gamma\chi_2$
with $\mathscr{H}=\mathscr{H}_\times$, $f=\vec{e}_k$,
$g(\lambda)=(s-\lambda)^{-1}$, and  
$(\cdot,\cdot)=\langle\chi_2\cdot,\cdot\rangle$, where $\langle\cdot,\cdot\rangle$ is the $L^2(\Omega,P)$
inner product, yields the Bergman--Milton representation of the
effective dielectric constant \eqref{eq:Gt_Integral}
\cite{Bergman:AP-78}. 

The equivalence of these two representations allows one to derive
relationships between the measures $\mu$ and $\alpha$ via The Stieltjes-Perron
Inversion Theorem \cite{Day:JPCM-96,Henrici:1974:v3,MILTON:2002:TC}. 
%
\begin{align}\label{eq:Stieltjes-Perron}
  \mu(y)=-\frac{1}{\pi}\lim_{\delta\downarrow0}\text{Im}F(y+\I\delta)\;, \quad
  y\in\text{supp}(\mu):=\Sigma_\mu 
\end{align}
%
To evoke this theorem directly we note that $t=1-s$ and write
%
\begin{align}\label{eq:Gts}
  G(t(s))=-\int_0^1\frac{d\alpha(\lambda)}{s-(1-\lambda)}
         :=-\int_0^1\frac{d\tilde{\alpha}(\lambda)}{s-\lambda}
         :=-\tilde{G}(s),
\end{align}
%
where $d\tilde{\alpha}(\lambda)=-d\alpha(1-\lambda)$. By equations
\eqref{eq:Herglotz_Energy_Rep_E}, \eqref{eq:Herglotz_Energy_Rep_G},
and $h=1-1/s$ we have  
%
\begin{align}\label{eq:Fs_relationships_G}
  1-F(s)=(1-1/s)(1+\tilde{G}(s))
\end{align}
%
Therefore, by setting $s=y+\I\delta$ for $y\in\Sigma_\mu\cap\Sigma_\alpha$ and $y\neq0$, where
$\Sigma_\mu:=\text{supp}(\mu)$, equations 
\eqref{eq:Fs_Integral} and
\eqref{eq:Stieltjes-Perron}--\eqref{eq:Fs_relationships_G} imply that
%
\begin{align}
  \mu(y)+\tilde{\alpha}(y)&=\frac{1}{\pi}\lim_{\delta\downarrow 0}\text{Im}
        \left[ \frac{-y+\I\delta}{y^2+\delta^2}
          \left(\text{Re}\,\tilde{G}(y+\I\delta)+\I\,\text{Im}\,\tilde{G}(y+\I\delta)
          \right)\right]\\
       &=\frac{1}{\pi}\lim_{\delta\downarrow 0}
        \left[\frac{\delta}{y^2+\delta^2}\text{Re}\,\tilde{G}(y+\I\delta) 
          -\frac{y}{y^2+\delta^2}\text{Im}\,\tilde{G}(y+\I\delta)
          \right]\notag \\
          &:= \frac{1}{y}\tilde{\alpha}(y)-\varrho(y),\notag
  \quad \varrho(y):=-\frac{1}{\pi}\lim_{\delta\downarrow 0}
  \frac{\delta}{y^2+\delta^2}\text{Re}\tilde{G}(y+\I\delta).\notag 
\end{align}
%
Thus,
%
\begin{align}\label{eq:BM_measure_relationship_E}
  y\mu(y)&=(1-y)[-\alpha(1-y)] - y\varrho(y), \\
  \varrho(y)&=-\frac{1}{\pi}\lim_{\delta\downarrow 0}\frac{\delta}{y^2+\delta^2}
  \int_0^1\frac{(y+\lambda-1)d\alpha(\lambda)}{(y+\lambda-1)^2+\delta^2}\,,\notag
\end{align}
%
where the minus sign in $[-\alpha(1-y)]$ is different from
\cite{Day:JPCM-96}, and accounts for reversal in the integration
order, $1-y:[0,1]\mapsto[1,0]$. Equation
\eqref{eq:BM_measure_relationship_E} is a general formula holding for  
two component stationary random media in the lattice and continuum
settings \cite{Golden:PRL-3935}. In Section \ref{sec:Measure_Equiv}
below we prove, for percolation models where the connectivity of
the system is determined by the volume fraction $p$ of defect
inclusions within an otherwise homogeneous medium, that the measure
$\varrho$ is given by $\varrho(dy)=W_0(p)\delta_0(dy)+m(p,0)(1-y)\delta_1(dy)$, where
$\delta_{y_0}(dy)$ is the Dirac measure centered at $y_0$, $W(p)\not\equiv0$,
and we have set $m(h):=m(p,h)$. 

We now derive the resolvent representations for the displacement field
$\vec{D}$, which lead to integral representations
\eqref{eq:Es_Integral} and \eqref{eq:Ht_Integral} for $E(s)$ and
$H(t)$ respectively. Again, consider the space of stationary random 
fields $\mathscr{H}_s\subset L^2(\Omega,P)$ and define the associated
Hilbert space of 
stationary divergence free random fields \cite{Golden:CMP-473}.   
%
\begin{align}\label{eq:DivfreeHilbert}
  \mathscr{H}_{\bullet}:=\{\vec{Y}(\omega)\in \mathscr{H}_s \ | \ \vec{\nabla}\cdot\vec{Y}=0 \text{ weakly and }
\langle\vec{Y}\rangle=0\}, 
\end{align}
%
Now consider the variational problem: find
$\vec{D}_f\in \mathscr{H}_{\bullet}$ such that    
%
\begin{align}
  \label{eq:Weak_Div_Free_Variational_Form}
  \langle\beps^{-1}(\vec{D}_0+\vec{D}_f)\cdot\vec{Y}\rangle=0 \quad  \forall \
  \vec{Y}\in\mathscr{H}_{\bullet}  
\end{align}
%
where $\vec{D}_0$ is the external uniform displacement field applied
to the system. Under the assumption that the bilinear form
$a(\vec{u},\vec{v})=\vec{u}^{T}[\beps(\vec{x},\omega)]^{-1}\vec{v}$ is
bounded and coercive, this problem has a unique solution
\cite{Golden:CMP-473} satisfying     
%
\begin{align}
  \label{eq:Homogenization_Equations_D}
   \vec{\nabla}\times\vec{E}=0\;, \quad
   \vec{\nabla}\cdot\vec{D}=0\;,\quad
   \vec{E}=\beps^{-1}\vec{D}\;,\quad   
   \vec{D}=\vec{D}_0+\vec{D}_f\;,\quad
   \langle\vec{D}\rangle=\vec{D}_0\;.\quad
\end{align}
%
By equation \eqref{eq:Weak_Div_Free_Variational_Form}, we have that
$\langle\vec{E}\cdot\vec{D}_f\rangle=0$. Therefore we have the following energy
constraint on the system $\langle\vec{E}\cdot\vec{D}\rangle =\langle\vec{E}\cdot\vec{D}_0\rangle$.

For simplicity, we again focus on one diagonal coefficient of the effective
inverse permittivity $1/\epsilon^*=1/\beps^*_{kk}$, defined by
%
\begin{align}\label{eq:Inverse_Effective_Permittivity}
  [\epsilon^*]^{-1}\vec{D}_0&:= \langle\vec{E}\rangle= \langle(\chi_1/\epsilon_1+\chi_2/\epsilon_2)\vec{D}\rangle
              =\langle(\epsilon_1)^{-1}(\chi_1+h\chi_2)\vec{D}\rangle=\langle(\epsilon_1)^{-1}(1-\chi_2/s)\vec{D}\rangle
              \notag\\             
              &=(\epsilon_1)^{-1}(\vec{D}_0-\langle\chi_2\vec{D}\rangle/s)
\end{align}
%
where $h=\epsilon_1/\epsilon_2$, $s=1/(1-h)$, and in the last step we have used
$\langle\vec{D}\rangle=\vec{D}_0=D_0\vec{d}_k$, where $\vec{d}_k$ is a unit
vector. The same manipulations in equation
\eqref{eq:Inverse_Effective_Permittivity}, applied to
$\vec{\nabla}\times\vec{E}=0$, yields   
%
\begin{align}\label{eq:CurlEquation}
  \vec{\nabla}\times(1-\chi_2/s)(\vec{D}_0+\vec{D}_f)=0.
\end{align}
%
From $\vec{D}_f\in\mathscr{H}_{\bullet}$ we have
$\vec{D}_f=\vec{\nabla}\times(\vec{A}+\vec{C})$ weakly, where
$\vec{\nabla}\times\vec{C}=0$. This follows from the result that $\vec{Y}$
is weakly divergence free if and only if  
$\vec{Y}=\vec{\nabla}\times(\vec{A}+\vec{C})$ weakly for some vector field
$\vec{A}$ and an arbitrary curl free vector
$\vec{C}$ \cite{Griffiths-1999,Folland:95}. The arbitrary vector
$\vec{C}$ can be chosen so that $\vec{\nabla}\cdot\vec{A}=0$ weakly
\cite{Griffiths-1999}. The vector $\vec{C}$ chosen in this manner
gives the Coulomb (or transverse) gauge of $\vec{D}_f$
\cite{Griffiths-1999,Jackson-1999}. Let
$\mathscr{C}_{\bullet}=\overline{\{\vec{f}\in\mathscr{H}_{\bullet}|\vec{\nabla}\cdot\vec{f}=0\}}$ be the 
\emph{closure} of the space of stationary divergence free random fields
of Coulomb gauge, consequently a Hilbert space. Using the identity
$\vec{\nabla}\times(\vec{\nabla}\times\vec{A})=\vec{\nabla}(\vec{\nabla}\cdot\vec{A})-\Delta\vec{A}$, for
$\vec{D}_f\in\mathscr{C}_{\bullet}$, equation \eqref{eq:CurlEquation} may be
written    
%
\begin{align}\label{eq:ResolventRepDs}
  0&=\vec{\nabla}\times(\vec{D}_f-\chi_2\vec{D}/s)=-\Delta\vec{A}-\vec{\nabla}\times(\chi_2\vec{D}/s)\Rightarrow
  \notag\\   
0&=\vec{D}_f-\vec{\nabla}\times\left[(-\Delta)^{-1}\vec{\nabla}\times(\chi_2\vec{D}/s)\right]:=
\vec{D}_f-\Upsilon\chi_2\vec{D}/s \Rightarrow\notag\\
s\vec{D}_0&=(s-\Upsilon\chi_2)\vec{D}\Rightarrow\quad
\boxed{\vec{D}=s(s-\Upsilon\chi_2)^{-1}\vec{D}_0}.
\end{align}
%
The boxed formula in equation \eqref{eq:ResolventRepDs} gives a
resolvent representation of the displacement field. On the Hilbert
space $\mathscr{C}_{\bullet}$, the operator $\Upsilon=\vec{\nabla}\times(-\Delta)^{-1}\vec{\nabla}\times$ is
a projector based on convolution with the free-space Green's function
for $-\Delta$ \cite{Golden:CMP-473}, and is hence bounded $\|\Upsilon\|\leq1$ in the
$\mathscr{H}_{\bullet}$ norm \cite{Folland:95}. Indeed, if $\vec{Y}$ 
is divergence free and of Coulomb gauge then 
$\vec{Y}=\vec{\nabla}\times\vec{A}$ for some vector field $\vec{A}$ with
$\vec{\nabla}\cdot\vec{A}=0$ \cite{Griffiths-1999,Jackson-1999}. Therefore, 
% 
\begin{align}\label{eq:proj_Upsilon}
  \Upsilon\vec{Y}&=\vec{\nabla}\times(-\Delta)^{-1}\vec{\nabla}\times\vec{Y}
              =\vec{\nabla}\times(-\Delta)^{-1}(-\Delta)\vec{A}=\vec{\nabla}\times\vec{A}
              =\vec{Y}.
\end{align}
%
Therefore, equations \eqref{eq:ResolventRepDs} and
\eqref{eq:proj_Upsilon} imply that
%
\begin{align}
  \Upsilon\vec{D}=\Upsilon\vec{D}_f=\vec{D}_f, \quad
  \Upsilon\chi_2\vec{D}=s\vec{D}_f, \quad
  \Upsilon\chi_1\vec{D}=t\vec{D}_f, \quad
  t:=1-s,
\end{align}
%
In summary
%
\begin{align}\label{eq:Herglotz_Energy_Rep_Ds}
  [\epsilon^{-1}]^*=(1-E(s))/\epsilon_1,\quad
  E(s)=\langle\chi_2(s-\Upsilon\chi_2)^{-1}\vec{d}_k\cdot\vec{d}_k\rangle, \quad
  \langle\vec{D}\cdot\vec{E}\rangle=D_0^2/\epsilon^*
\end{align}
%

In the Hilbert space $L^2(\Omega,P)$ with weight $\chi_2$ in the
inner product, $\Upsilon\chi_2$ is a bounded self--adjoint operator. If we
reduce the domain of the operator $M=\Upsilon\chi_2$ to the Hilbert space 
$\mathscr{C}_{\bullet}$ with weight $\chi_2$ in the inner product, then $\Upsilon\chi_2$
is a self--adjoint composition of projection operators and hence has
norm $\|\Upsilon\chi_2\|\leq1$. Therefore the resolvent $(s-\Upsilon\chi_2)^{-1}$ is also
self--adjoint with respect to this inner product \cite{Stone:64}.
Applying Theorem \ref{thm:Spectral_Theorem} to $M=\Upsilon\chi_2$, with
$\mathscr{H}=\mathscr{H}_\bullet$, $f=\vec{d}_k$, 
$g(\lambda)=(s-\lambda)^{-1}$, and $(\cdot,\cdot)=\langle\chi_2\cdot,\cdot\rangle$, where $\langle\cdot,\cdot\rangle$ is the
$L^2(\Omega,P)$ inner product, yields the Bergman--Milton representation of
the effective inverse dielectric constant \eqref{eq:Es_Integral}. 

Another equivalent representation follows from factoring out $\epsilon_2$ in
equations\eqref{eq:Inverse_Effective_Permittivity}--\eqref{eq:CurlEquation}  
instead of $\epsilon_1$. Again, defining $z = 1/h$ and $t=1/(1-z)=1-s$ and noting
the symmetry  $\epsilon(\chi_1/\epsilon_1,\chi_2/\epsilon_2)=\epsilon(\chi_2/\epsilon_2,\chi_1/\epsilon_1)$ we arrive at 
%
\begin{align}\label{eq:Resolvent_Rep_Dt}
   &[\epsilon^{-1}]^*\vec{D}_0=[\epsilon_2]^{-1}(\vec{D}_0-\langle\chi_1\vec{D}\rangle/t),\notag\\
    &\vec{\nabla}\cdot(1-\chi_1/t)(\vec{D}_0+\vec{D}_f)=0,\notag\\
    &\boxed{\vec{D}=t(t-\Upsilon\chi_1)^{-1}\vec{D}_0},
\end{align}
%
and $\vec{D}_f=\Upsilon\chi_1\vec{D}/t$. Equation
\eqref{eq:Resolvent_Rep_Dt} gives an equivalent resolvent
representation of the electric field. In summary
%
\begin{align}\label{eq:Herglotz_Energy_Rep_Dt}
  [\epsilon^*]^{-1}=(1-H(t))/\epsilon_2,\quad
  H(t):=\langle\chi_1(t-\Upsilon\chi_1)^{-1}\vec{d}_k\cdot\vec{d}_k\rangle, \quad
  \langle\vec{D}\cdot\vec{E}\rangle=D_0^2/\epsilon^*.
\end{align}
%
Applying Theorem \ref{thm:Spectral_Theorem} to $M=\Upsilon\chi_1$,
with $\mathscr{H}=\mathscr{H}_\bullet$, $f=\vec{d}_k$,
$g(\lambda)=(s-\lambda)^{-1}$, and  $(\cdot,\cdot)=\langle\chi_1\cdot,\cdot\rangle$, where $\langle\cdot,\cdot\rangle$ is the $L^2(\Omega,P)$
inner product, yields the Bergman--Milton representation of the
effective dielectric constant \eqref{eq:Ht_Integral}.

The equivalence of these two representations allows one to derive
relationships between the measures $\eta$ and $\tau$ via The Stieltjes-Perron
Inversion Theorem \eqref{eq:Stieltjes-Perron}
\cite{Day:JPCM-96,Henrici:1974:v3,MILTON:2002:TC}.
To evoke this theorem directly we note that $t=1-s$ and write
%
\begin{align}\label{eq:Hts}
  H(t(s))=-\int_0^1\frac{d\tau(\lambda)}{s-(1-\lambda)}
         =-\int_0^1\frac{d\tilde{\tau}(\lambda)}{s-\lambda}
         :=-\tilde{H}(s),
\end{align}
%
where $d\tilde{\tau}(\lambda)=d\tau(1-\lambda)$. By equations
\eqref{eq:Herglotz_Energy_Rep_Ds}, \eqref{eq:Herglotz_Energy_Rep_Dt},
and $h=1-1/s$ we have  
%
\begin{align}\label{eq:Es_relationships_D}
  1-E(s)=(1-1/s)(1+\tilde{H}(s)).
\end{align}
%
By symmetry we have the following result
%
\begin{align}\label{eq:BM_measure_relationship_D}
  y\eta(y)&=(1-y)[-\tau(1-y)] - y\varrho_0(y), \\
  \varrho_0(y)&=-\frac{1}{\pi}\lim_{\delta\downarrow 0}\frac{\delta}{y^2+\delta^2}
  \int_0^1\frac{(y+\lambda-1)d\tau(\lambda)}{(y+\lambda-1)^2+\delta^2}.\notag
\end{align}
%
Equation \eqref{eq:BM_measure_relationship_D} is a general formula
holding for two component stationary random media in the lattice and
continuum settings \cite{Golden:PRL-3935}. 
%
% \subsection{Generalized Representations of the Effective Permittivity}
% \label{sec:Gen_Rep_eps}
% %
% We now show how the ideas of this section can be extended to systems
% with internal sources. Specifically, we derive representations of the
% effective permittivity involving the operator $\Gamma$ for systems with
% total internal charge density $\rho_T=\rho_f+\rho_b\neq0$, where $\rho_f$ is the free
% charge density and $\rho_b$ is the bound charge density. The resultant
% representations of the electric field involve the operator
% $s(1+\Gamma)+2\Gamma\chi_1$. This operator is \emph{not} self--adjoint in the
% Hilbert space $\mathscr{H}_\times$ with inner product weighted by
% $\chi_1$. However, we show in section \ref{sec:Herglotz_Energy_Reps}
% below that the energy constraint $\langle\vec{D}\cdot\vec{E}_f\rangle=0$ leads to
% integral representations, in terms of the measures $\mu$ and $\alpha$,
% involving operators which are not self adjoint with respect to the
% underlying inner product. The problem of proving the existence of
% integral representations associated with the generalized effective
% permittivity is a topic of future research. 

% Consider the Hilbert space of stationary curl free random fields
% $\mathscr{H}_\times$ defined in equation \eqref{eq:curlfreeHilbert}. As
% before, we define the total electric field by
% $\vec{E}=\vec{E}_f+\vec{E}_0$, where $\vec{E}_f\in\mathscr{H}_\times$ and
% $\langle\vec{E}\rangle=\vec{E}_0=E_0\vec{e}_k$. The weak curl free variational
% problem \eqref{eq:Weak_Curl_Free_Variational_Form} and equation  
% \eqref{eq:Effective_Permittivity} implies that the effective
% permittivity may be defined through the system energy as follows   
% %
% \begin{align}
%   \langle\vec{D}\cdot\vec{E}\rangle=\langle\vec{D}\rangle\cdot\vec{E}_0
%                  :=\beps^*\vec{E}_0\cdot\vec{E}_0
%                  =\epsilon_2(\vec{E}_0-\langle\chi_1\vec{E}\rangle/s).
% \end{align}
% %
% By linearity of the material, we have the following
% identity
% %
% \begin{align}  \label{eq:fieldRelations}
%   \epsilon_0\vec{E}=\vec{D}-\vec{P}_0:=\epsilon\vec{E}-(\epsilon-\epsilon_0)\vec{E}\iff
%   \epsilon_2\vec{E}=\vec{D}-\vec{P}_2:=\epsilon\vec{E}-(\epsilon-\epsilon_2)\vec{E}.
% \end{align}
% %
% Therefore $\epsilon_2$ can be used in place of $\epsilon_0$, the permittivity of
% free space, in the definition of the total charge,
% $\epsilon_2\vec{\nabla}\cdot\vec{E}=\rho_T=\rho_f+\rho_b$, and therefore in the system polarization,
% $\vec{P}:=(\epsilon-\epsilon_2)\vec{E}=-\epsilon_2(\chi_1/s)\vec{E}$, without physical nor mathematical
% inconsistencies \cite{Jackson-1999,Griffiths-1999}.  The total, free,
% and bound charge densities are defined, respectively, as \cite{Griffiths-1999} 
% %
% \begin{align}
%   \epsilon_2\vec{\nabla}\cdot\vec{E}:=\rho_T, \quad
%   -\vec{\nabla}\cdot\vec{P}:=\rho_b, \quad
%   \vec{\nabla}\cdot\vec{D}:=\rho_f.
% \end{align}
% %

% The same steps leading to the resolvent representation of
% the electric field \eqref{eq:Resolvent_Rep_Es} when
% $\rho_f=0$, may also be done for internal sources $\rho_T=\rho_f+\rho_b\neq0$,
% resulting in 
% %
% \begin{align}
%   \frac{\epsilon_2}{s}(s+\Gamma\chi_1)\vec{E}&=\epsilon_2\vec{E}_0-\vec{\nabla}(\Delta^{-1})\rho_f
%   = \epsilon_2\vec{E}_0-\vec{\nabla}(\Delta^{-1})(\epsilon_2\vec{\nabla}\cdot\vec{E}+\vec{\nabla}\cdot\vec{P})\\
%   &= \frac{\epsilon_2}{s}\left(s\vec{E}_0-\left(s\Gamma-\Gamma\chi_1\right)\vec{E}\right)\notag
% \end{align}
%
%
%
\section{Pad\'{e} Approximants and Orthogonal Polynomials}
\label{sec:Orthogonal_Polynomials}
%       
Intimately entwined within the spectral theory of bounded linear
self--adjoint operators with simple spectrum, is the theory of
orthogonal polynomials. In section \ref{sec:Resolv_Rep_E_D} we derived
resolvent representations for the electric and displacement fields
involving bounded linear operators with simple spectrum that are
self--adjoint on a given Hilbert space. Without loss of generality,
here we focus on the operator $\mathbf{M}=\chi_1\Gamma\chi_1$, with Stieltjes
transform $F(s)$ of the spectral measure $\mu$, which is self--adjoint
on the Hilbert space $\mathscr{H}_\times\subset L^2(\Omega,P)$. In this section we
discuss the connection between this class of operators and orthogonal  
polynomials. We discuss how numerator and denominator orthogonal
polynomials provide exponentially convergent Pad\'{e} approximants of
$F(s)$. We also explore some physical consequences of the connection
with orthogonal polynomials. Namely, we show how the Stieltjes
transformations of the orthogonality measures of the generalized
numerator polynomials of order 0,1, and 2, are related to the Herglotz 
decomposition of  the system energy. 

In the finite lattice setting the operator $\mathbf{M}:=\mathbf{M}_n$
is given by a matrix of order $n$, say. We construct the associated
Jacobi matrix $\mathbf{J}_n$ of order $n$ corresponding 
to the denominator polynomials, which are orthogonal with respect to
$\mu$. In the general case, The Spectral Theorem
\ref{thm:Spectral_Theorem} generates a mapping, $\psi_\mu$, from the operator
$\mathbf{M}$ to the spectral measure: $\mathbf{M}\stackrel{\psi_\mu}{\mapsto}\mu$. We
discuss another (one--to--one) map, $\mu\stackrel{\psi}{\mapsto}\mathbf{J}$, from
$\mu$ to a bounded Jacobi matrix $\mathbf{J}$ of infinite order which
acts on $l_2$, and has $\mu$ as it's spectral measure. The construction
of $\psi$ turns out to be the classical problem of the construction of
the polynomials orthogonal to $\mu$. This construction demonstrates that
$\varphi=\psi^{-1}$ is the spectral map, and it extends to a unitary map
$l_2\stackrel{\mathbf{U}}{\mapsto}L^2(\mu)$. In the matrix case the map $\psi$
becomes a bijection.  

The connection between orthogonal polynomials and the operator
$\mathbf{M}$ is much deeper and far reaching than that given by The
Spectral Theorem or Pad\'{e} approximants of $F(s)$. The connection is
fundamental, and resides in infinite 
matrix representations of operators in this class. More specifically,
in the existence of an orthonormal basis of $\mathscr{H}_\times$, which
reduces this representation to Jacobi form. In
section \ref{sec:General_Considerations} we review the relevant
abstract aspects of this beautiful theory which lead to this
result, and provide the means for solving important problems in the
theory of continued fractions and the theory of moments. This
framework of orthogonal polynomials, Jacobi operators, and allied topics
provide important information regarding the spectral measure $\mu$, thus
the effective parameter $\epsilon^*$ and critical transitions in random binary
composite media.    

%%%%%%%%%%%%%%%%%%%%%%%%%%%%%%%%%%%%%%%%%%%%%%%%%%%%%%%%%%%%%%
\noindent \textbf{Pad\'{e} Approximants}
%%%%%%%%%%%%%%%%%%%%%%%%%%%%%%%%%%%%%%%%%%%%%%%%%%%%%%%%%

The analytic continuation method, outlined in section
\ref{sec:Background_TACM}, is very useful. It not only yields exact
representations of the system energy in terms of Stieltjes
transformations of $\mu$, but it can also be used to obtain rigorous bounds
on $\epsilon^*$ given microstructural information. How the geometric
information of the composite is incorporated into 
the measure $\mu$ can be found through a perturbation expansion of $\epsilon^*$
around a homogeneous medium $\epsilon_1=\epsilon_2$ or $s=\infty$. By expanding the
integrand of $F(s)$ in \eqref{eq:Fs_Integral} for $|s|>1$ in powers of
$1/s$, we obtain \cite{Golden:JMP-5627}
%
\begin{align}\label{eq:Fs_near_infinity}
  F(s)=\sum_{j=0}^\infty\frac{\mu_j}{s^{j+1}},
\end{align}
%
where the $\mu_j$ are the moments of the measure $\mu$:
%
\begin{align}\label{eq:moments_mu}
  \mu_j=\int_0^1\lambda^jd\mu(\lambda)
     =(-1)^j\int_\Omega P(d\omega)[\chi_1(\Gamma\chi_1)^j\vec{e}_k]\cdot\vec{e}_k,
     \quad j=0,1,\ldots
\end{align}
%
The second equality follows from the same expansion of $F(s)$ in
\eqref{eq:Herglotz_Energy_Rep_E} around $s=\infty$
\cite{Golden:JMP-5627}. Clearly from this second formula we have, for
any medium, 
%
\begin{align}
\mu_0=\langle\chi_1\rangle=p_1,  
\end{align}
%
the volume fraction of $\epsilon_1$, where $\langle \cdot \rangle$ denotes ensemble average
over $\Omega$ or spatial average over all of ${\mathbb{R}}^d$, and $d$ is
the dimensionality of the system. When the medium is isotropic it can
be shown that \cite{Golden:CMP-473,Bruno:JSP-365}  
%
\begin{align}
  \mu_1=\frac{p_1p_2}{d},
\end{align}
%
where $p_2=1-p_1$.

In general $\mu_n$ depends on the $(n+1)$--point correlation function of
the medium under consideration \cite{Golden:CMP-473,Golden:JMP-5627}. 
In principle, if all the $n$-point correlation functions are known
then the spectral measure may be uniquely determined and $F(s)$, hence
$\epsilon^*$, is exactly known \cite{Shohat:1963}. This shows the usefulness
of the representation \eqref{eq:Fs_near_infinity}: by analytic
continuation, local information at $s=\infty$ yields global information for
$s\in\mathbb{C}\text{\textbackslash}\Sigma_\mu$, where the support of $\mu$ is denoted
$\Sigma_\mu:=\text{supp}(\mu)\subset[0,1]$. In practice, complete 
information regarding the composite is unavailable and one often
resorts to approximations of $F(s)$. Taylor polynomials are clearly
not a good class of functions to approximate $F(s)$ as they do not
capture the singularities of this function. Rational functions of
polynomials are the simplest approximating functions with
singularities \cite{Assche:SAT:2006}.

The $[m,n]$ Pad\'{e} approximant
of $F(s)$ is the rational function of \emph{monic} polynomials
$Q_m/P_n$, with $Q_m$ of degree $\leq m$ and $P_n$ of degree $\leq n$. For
the Pad\'{e} approximation of $F(s)$ near infinity one takes $m=n-1$,
which leads to the following interpolation condition \cite{Assche:SAT:2006}:  
%
\begin{align}
  P_n(s)F(s)-Q_{n-1}(s)=O(s^{-n-1}), \quad |s|\to\infty.
\end{align}
%
The \emph{denominator polynomials} $P_n$ are orthogonal to the measure
$\mu$ \cite{Assche:SAT:2006}, i.e.,
%
\begin{align}
  \int_0^1\lambda^kP_n(\lambda)d\mu(\lambda)=0, \quad k=0,1,\ldots,n-1,
\end{align}
%
which we will normalize $\tilde{p}_n=P_n/\|P_n\|_{\mu}$, where $\|\cdot\|_{\mu}$
is the $L^2(\mu)$ norm. The corresponding numerator polynomials,
$\tilde{q}_{n-1}=Q_{n-1}/\|P_n\|_{\mu}$ are given by \cite{Assche:SAT:2006} 
%
\begin{align}
  \tilde{q}_{n-1}(s)=\int_0^1\frac{\tilde{p}_n(s)-\tilde{p}_n(\lambda)}{s-\lambda}d\mu(\lambda),
\end{align}
%
and the error may be written \cite{Assche:SAT:2006}
%
\begin{align}\label{eq:Pade_error}
  F(s)-\frac{\tilde{q}_{n-1}(s)}{\tilde{p}_n(s)}
      =\frac{1}{\tilde{p}^2_n(s)}\int_0^1\frac{\tilde{p}^2_n(\lambda)}{s-\lambda}d\mu(\lambda).
\end{align}
%
The integral on the right hand side of \eqref{eq:Pade_error} may be
bounded independent of $n$, so the convergence of the Pad\'{e}
approximants is completely determined by the asymptotic behavior of
$\tilde{p}_n$ \cite{Assche:SAT:2006}.  

In order to illuminate some important properties of orthogonal
polynomials, we give a brief sketch of the asymptotic behavior of
$|\tilde{p}_n(s)|^{1/n}$ when $s$ is in a compact set
$K\subset\mathbb{C}\text{\textbackslash}\Sigma_\mu$. By orthogonality, the zeros of
$\tilde{p}_n$ are simple and contained in $(0,1)$ \cite{Deift:2000:RMT},
denote them by $0<\lambda_{1,n}<\lambda_{2,n}<\cdots <\lambda_{n,n}<1$. Denote the leading
coefficient of $\tilde{p}_n$ by $\gamma_n=\|P_n\|_{\mu}^{-1}$ so that
%
\begin{align}
 \tilde{p}_n(s)= \gamma_n\prod_{j=1}^n(s-\lambda_{j,n}).
\end{align}
%
The asymptotic behavior thus requires knowing the behavior of $\gamma_n$ and
the asymptotic distribution of the zeros.

First, consider the normalized counting measure of the zeros
%
\begin{align}
  \nu_n(d\lambda)=\frac{1}{n}\sum_{j=1}^n\delta_{\lambda_{j,n}}(d\lambda),
\end{align}
%
which defines a sequence of probability measures supported on the
interval $(0,1)$, and describes the distribution of the zeros of
$\tilde{p}_n$. Helly's selection principle tells us
\cite{Assche:SAT:2006,Deift:2000:RMT} that there is a subsequence
$\{n_k\}$ that converges weakly to a probability measure $\nu$ supported
on the interval $[0,1]$. For the monic polynomial $P_n$ we have   
%
\begin{align}
  \frac{1}{n}\log|P_n(s)|=\frac{1}{n}\sum_{j=1}^n\log|s-\lambda_{j,n}|
                        =\int_0^1\log|s-\lambda_{j,n}|d\nu_n(\lambda),
\end{align}
%
so that, by weak convergence for $s\in K$,
%
\begin{align}\label{eq:Pn_asymptotics}
  \lim_{j\to\infty}|P_{n_j}(s)|^{1/n_j}=\exp\left(U_{\nu}(s;\nu)\right), \quad
  U_\nu(s;m):=\int_{\Sigma_\nu}\log|s-\lambda|dm(\lambda),
\end{align}
%
where $\Sigma_\mu\subseteq\Sigma_\nu\subseteq[0,1]$ and we use $\nu_n(d\lambda)$ and $d\nu_n(\lambda)$ interchangeably.

Second, just as the Chebyshev polynomials minimize the $L^\infty([-1,1])$
norm, the $P_n$ minimize the $L^2(\mu)$ norm
\cite{Deift:2000:RMT}. Indeed, the leading coefficient
$\gamma_n=\|P_n\|_{\mu}^{-1}$ solves the minimization problem: 
%
\begin{align}
  \gamma_n^{-2}=\text{inf}_{\pi_{n}(\lambda)=\lambda^n+\cdots}(\|\pi_n\|_{\mu}^2),
\end{align}
%
so that the minimum is attained at the monic orthogonal polynomial
$P_n$ \cite{Assche:SAT:2006}. This extremal problem for $\gamma_n$ may be
used to show that the asymptotic behavior of $\gamma_n^{1/n}$ and the
asymptotic distribution $\nu$ of the zeros are described by an
equilibrium problem for (logarithmic) potentials
\cite{Assche:SAT:2006}.

More specifically, if $\mu$ is sufficiently regular
\cite{Simon:IPI:07,Saff_Totik:97} and if we denote by 
$\mathcal{M}_1(\Sigma_\nu)$ the set of probability measures supported on
$\Sigma_\nu$, then the measure $\nu$ is the unique minimizer (the equilibrium
measure for $\Sigma_\nu$) of the quadratic functional  
%
\begin{align}\label{eq:Classic_OP_functional_min}
  \mathcal{E}_{\nu}[\nu]=\inf_{m\in\mathcal{M}_1(\Sigma_\nu)}\mathcal{E}_{\nu}[m],\quad
  \mathcal{E}_{\nu}[m]:=-\int_{\Sigma_\nu}U_\nu(\lambda;m)dm(\lambda),
\end{align}
%
were $U_\nu(\lambda;m)$ is defined in equation \eqref{eq:Pn_asymptotics}. This
standard variational problem of potential theory, which has the
electrostatic interpretation of $\nu$ being the equilibrium distribution
of a distribution $m$ of positive charges on a conductor $\Sigma_\nu$, is
equivalent to the relations \cite{Saff_Totik:97}:
%
\begin{align}\label{eq:Classical_OP_Euler_Lag_Eq}
  &-2U_\nu(\lambda;\nu)=-l_{\nu},\quad \lambda\in\Sigma_\nu,\\
  &-2U_\nu(\lambda;\nu)\geq-l_{\nu},\quad \lambda\in\mathbb{R}\text{\textbackslash}\Sigma_\nu.\notag
\end{align}
%
Formulas \eqref{eq:Classical_OP_Euler_Lag_Eq} are the Euler--Lagrange
equations for \eqref{eq:Classic_OP_functional_min}, the quantity
$\exp(l_{\nu}/2)$ is the logarithmic capacity of $\Sigma_\nu$, and
$-l_{\nu}/2$ is known as the Robin constant \cite{Pastur:JAT:06}.
It can be shown, on the potential curves
%
\begin{align}
  C_r=\left\{s\in\mathbb{C}\text{\textbackslash}\Sigma_\nu\left.\right\vert
      \left(\lim_{n\to\infty}\gamma_n^{1/n}\right)\exp(U_\nu(s;\nu))=r \right\}, 
\end{align}
%
with $r>1$, that
%
\begin{align}
  \lim_{n\to\infty}\left|F(s)-\frac{\tilde{q}_{n-1}(s)}{\tilde{p}_n(s)}\right|^{1/n}=\frac{1}{r^2},
\end{align}
%
demonstrating that the convergence is exponential
\cite{Assche:SAT:2006}. 

From equation \eqref{eq:Fs_Integral} we see that $F(s)$ is a linear
mapping of $\mathscr{B}_0$, the set of positive finite Borel measures on
$[0,1]$, to the complex plane. Let
%
\begin{align}\label{eq:Set_of_mu_Measures}
  \mathscr{B}(\mu_1,\mu_2,\ldots,\mu_n)=
     \left\{\mu\in\mathscr{B}_0 \left|\right. \; \mu_j=\int_0^1\lambda^jd\mu(\lambda), \ j=1,2,\ldots,n\right\}. 
\end{align}
%
The set of measures $\mathscr{B}(\mu_1,\mu_2,\ldots,\mu_n)$ is a compact, convex
subset of $\mathscr{B}_0$ with the topology of weak convergence
\cite{Golden:CMP-473}. One can show that the measure $\mu$ is a weak
limit of convex combinations of $n$--point measures, i.e., measures of
the form  \cite{Golden:CMP-473} 
%
\begin{align}\label{eq:mu_measure}
  \mu(d\lambda)=\sum_{j=1}^{n}m_j\delta_{\lambda_j}(d\lambda),
\end{align}
%
where $d\mu(\lambda)$ and $\mu(d\lambda)$ are used interchangeably and
%
\begin{align}\label{eq:mu_properties}
  m_j\geq0, \ 0\leq\lambda_1<\lambda_2<\cdots<\lambda_{n}<1, \ \mu_i=\sum_{j=1}^{n}m_j\lambda_j^i, \ i=0,1,\ldots,n-1.
\end{align}
%

A Pad\'{e} approximation of $F(s)$ thus gives an idea of the
singularities $\{\lambda_j\}$ of $F(s)$. The singularities of the Pad\'{e}
approximant are the zeros of $P_n$. In finite lattice systems the random
operator $\mathbf{M}=\chi_1\Gamma\chi_1$ may be represented by a sparse random
\emph{matrix} \cite{Golden:CMP-467}, of size $n$, say,
$\mathbf{M}:=\mathbf{M}_n$. In this finite lattice setting the discrete
measure $\mu$ is given exactly by equation \eqref{eq:mu_measure}, and
$F(s)$ may be written \cite{Golden:CMP-467}   
%
\begin{align}\label{eq:Discrete_Fs}
  F(s)=\sum_{j=1}^{n}\frac{m_j}{s-\lambda_j}, \quad
  m_j=\langle\vec{e}_k^{\;T}E_{\lambda_j}\vec{e}_k\rangle,\quad
  E_{\lambda_j}=\phi_j\phi_j^T, \quad \mathbf{M}_n\phi_j=\lambda_j\phi_j,
\end{align}
%
where $\phi_i^T\phi_j=\delta_{i,j}$, $\vec{e}_k=[1,1,\cdots,1]$, and $\langle\cdot\rangle$ is division
by $n$. The family of projection operators $E_{\lambda}$ is called the
\emph{resolution of the identity}, $\sum_{j=1}^nE_{\lambda_j}=I_n$, where $I_n$
is the identity operator on $\mathbb{R}^n$. In this finite lattice
setting the Pad\'{e} approximants of $F(s)$ are exact:        
%
\begin{align}\label{eq:Exact_Pade_Fs}
  F(s)\equiv p_1\frac{Q_{n-1}(s)}{P_n(s)},\quad
  P_n(s)=\prod_{j=1}^n(s-\lambda_j), \quad Q_{n-1}(s)=p_1^{-1}\sum_{j=1}^nm_j\prod_{j\neq l=1}^n(s-\lambda_l),
\end{align}
%
where the normalization by the volume fraction, $p_1=\sum_{j=1}^nm_j=\int_0^1d\mu(\lambda)$, makes
$Q_{n-1}(s)$ a monic polynomial. In this case, the singularities of
$F(s)$ (the eigenvalues of $\mathbf{M}_n$) are precisely the
zeros of $P_n(s)$. By multiplying
$F(s)$ in \eqref{eq:Discrete_Fs} by $s-\lambda_i$ and letting $s\to\lambda_i$
one can easily see from \eqref{eq:Exact_Pade_Fs} that the weights
(Christoffel numbers) $\{m_j\}_{j=1}^n$ are given by \cite{Assche:JCAM:1991:237}
%
\begin{align}\label{eq:Christoffel_nums_QP}
  m_j=p_1\frac{Q_{n-1}(\lambda_j)}{P_n^\prime(\lambda_j)},
\end{align}
where the prime denotes differentiation in the variable $s$. 

%%%%%%%%%%%%%%%%%%%%%%%%%%%%%%%%%%%%%%%%%%%%%%%%%%%%%%%%%%%%%%%%%%%%%%%%%%
\noindent\textbf{Denominator Polynomials}\newline
%%%%%%%%%%%%%%%%%%%%%%%%%%%%%%%%%%%%%%%%%%%%%%%%%%%%%%%%%%%%%%%%%%%%%%%%%%
%
It is important to note that the measure $\mu$ defined by
\eqref{eq:mu_measure}--\eqref{eq:Discrete_Fs}, for $n<\infty$, can only
generate polynomials orthogonal to $\mu$ up to order  
$n-1$. This can be easily seen \cite{Deift:2000:RMT} as the polynomial 
$\pi(s):=\prod_{j=1}^n(s-\lambda_j)=0$ in $L^2(\mu)$ so that $\{1,s,s^2,\ldots,s^{n-1}\}$ spans
$\{1,s,s^2,\ldots\}$ in $L^2(\mu)$. Conversely \cite{Deift:2000:RMT}, observe
that for some $i$ the set of polynomials $\{1,s,s^2,\ldots,s^i\}$ is linearly
dependent in $L^2(\mu)$ only if $n>i$. Indeed, if this set is dependent
for some $i\geq n$, then there  exists a polynomial
$\pi(s)=\sum_{j=1}^ic_js^j$, with not all $c_j=0$, such that $\pi(s)=0$ in
$L^2(\mu)$. But $\pi(s)$ has at most $i\geq n$ real zeros, a contradiction,
hence $n>i$. The Gram--Schmidt procedure can be carried out on the set
$\{1,s,s^2,\ldots\}$ as long as the $L^2(\mu)$ norm $\|\cdot\|_\mu$ is strictly
positive definite. If we denote $\mathbb{P}$ the set of real
polynomials on $[0,1]$ and $\mathbb{P}_i\subset\mathbb{P}$ the space of
polynomials of degree $\leq i$, then the above argument shows that
$\|\cdot\|_\mu$ is strictly positive definite on $\mathbb{P}_i$ only if 
$i<n.$  Moreover, in the general case, $n=\infty$, the $L^2(\mu)$ norm is
strictly positive definite on $\mathbb{P}_i$ if and only if the Hankel
determinants $D_j$ are strictly positive, $D_j>0$, for $j=1,2,\ldots,i$,
where   
%
\begin{align}\label{eq:Hankel_Det}
 D_j=
  \left|
    \begin{matrix}
      \mu_0   & \mu_1 & \mu_2   & \cdots & \mu_j  \\
      \mu_1   & \mu_2 & \mu_3   & \cdots & \mu_{j+1}\\
      \cdots     & \cdots   & \cdots     & \cdots & \cdots    \\
      \mu_j & \mu_{j+1} & \mu_{j+2} & \cdots & \mu_{2j}
    \end{matrix}
  \right|  
\end{align}
%
and the $\{\mu_j\}$ are the moments of the measure $\mu$ defined
in \eqref{eq:moments_mu}. This can be easily seen
\cite{Gautschi:2004:OP} by writing $\pi_j(\lambda)=\sum_{j=1}^ic_j\lambda^j$ so that  
%
\begin{align}
  \|\pi_j\|_\mu^2=\int_0^1d\mu(\lambda)\sum_{j,l=0}^ic_jc_l\lambda^{j+l}=\sum_{j,l=0}^ic_jc_l\mu_{j+l}.
\end{align}
%
The norm $\|\cdot\|_\mu$ is thus strictly positive definite if and only if 
the Hankel matrix $[\mu_{j+l}]_{j,l=0,1,2,\cdots,i}$ is strictly positive definite
if and only if $D_j>0$ for $j=1,2,\ldots,i$.

In the case of an infinite random binary medium ($n=\infty$), an infinite
sequence of orthogonal polynomials may be generated. The polynomials
$\tilde{p}_i=P_i/\|P_i\|_\mu$, orthonormal in $L^2(\mu)$ 
%
\begin{align}\label{eq:Orthonormal_P_mu}
  \langle \tilde{p}_i,\tilde{p}_j\rangle_\mu=\int_{\Sigma_\mu}\tilde{p}_i(\lambda)\tilde{p}_j(\lambda)d\mu(\lambda)=\delta_{i,j},
\end{align}
%
are given by \cite{Deift:2000:RMT,Szego:39} 
%
\begin{align}\label{eq:OP_mu_mom_def}
  \tilde{p}_i(s)=(D_{i-1}D_i)^{-1/2}
  \left|
    \begin{matrix}
      \mu_0   & \mu_1 & \mu_2   & \cdots & \mu_i  \\
      \mu_1   & \mu_2 & \mu_3   & \cdots & \mu_{i+1}\\
      \cdots     & \cdots   & \cdots     & \cdots & \cdots    \\
      \mu_{i-1} & \mu_i & \mu_{i+1} & \cdots & \mu_{2i-1}\\
      1     & s    & s^2    & \cdots & s^i
    \end{matrix}
  \right|,  
\end{align}
%
where, for $i\geq0$, $D_i$ is defined in equation
\eqref{eq:Hankel_Det}. The zeros of $\tilde{p}_{i-1}$ interlace those
of $\tilde{p}_i$ \cite{Ismail:2005}, and
$\tilde{p}_n^\prime(s)/\tilde{p}_n(s)=\sum_{j=1}^n(s-\lambda_{j,n})^{-1}$
\cite{Assche:JCAM:1991:237}.

The polynomials $\tilde{p}_i$ satisfy the following three term
recursion relation \cite{Gautschi:2004:OP} 
%
\begin{align}\label{eq:ON_3term_recursion}
  \sqrt{\beta_{j+1}}\tilde{p}_{j+1}(s)&=(s-\alpha_j)\tilde{p}_{j}(s)
                            -\sqrt{\beta_j}\tilde{p}_{j-1}(s),
                            \quad j=0,1,2,\ldots\\
  &\tilde{p}_{-1}(s)=0, \quad  \tilde{p}_{0}(s)=1/\sqrt{\beta_0},\notag 
\end{align}
%
where \cite{Gautschi:2004:OP,Deift:2000:RMT}
%
\begin{align}\label{eq:recursion_coeff}
  \alpha_j&=\frac{\langle\lambda P_j,P_j\rangle_\mu}{ \langle P_j,P_j\rangle_\mu}=\langle\lambda\tilde{p}_j,\tilde{p}_j\rangle_\mu,
    \quad j=0,1,2,\ldots \\
  \beta_j&=\frac{\langle P_j,P_j\rangle_\mu}{ \langle P_{j-1},P_{j-1}\rangle_\mu}= \langle\lambda\tilde{p}_{j-1},\tilde{p}_j\rangle_\mu^2,\notag
  \quad j=1,2,\ldots,  
\end{align}
%
and $\beta_0:=\langle P_0,P_0\rangle_\mu=\int_0^1d\mu(\lambda)=p_1$, hence
$\|P_n\|_\mu^2=\beta_n\beta_{n-1}\cdots\beta_1\beta_0$. Using the recursion coefficients, the
numerator polynomial in \eqref{eq:Christoffel_nums_QP}, $Q_{n-1}$, can be
replaced, giving the Christoffel numbers in terms of the $\tilde{p}_j$
and their derivatives \cite{Assche:JCAM:1991:237}
%
\begin{align}\label{eq:Christoffel_nums}
   m_i=p_1\frac{Q_{n-1}(\lambda_{i,n})}{P_n^\prime(\lambda_{i,n})}
     =-(\sqrt{\beta_{n+2}}\tilde{p}_{n+1}^\prime(\lambda_{i,n})\tilde{p}_{n+2}(\lambda_{i,n}))^{-1}
     =(\sqrt{\beta_{n+1}}\tilde{p}_{n+1}^\prime(\lambda_{i,n})\tilde{p}_n(\lambda_{i,n}))^{-1}
\end{align}
%

Denote the essentially self--adjoint
Jacobi operator $\mathbf{J}(\mu)$ 
\cite{Stone:64,Deift:2000:RMT,Gautschi:2004:OP} by
% 
 \begin{align}\label{eq:Infinite_Jacobi_Mn}
   \mathbf{J}(\mu)=
    \left[
   \begin{matrix}  
     \alpha_0        & \sqrt{\beta_1}&          &          &   &0 \\       
     \sqrt{\beta_1} & \alpha_1       & \sqrt{\beta_2}&          &   &  \\
               &\sqrt{\beta_2} & \alpha_2       & \sqrt{\beta_3}&   &  \\
               &          & \ddots       & \ddots        &\ddots  &  \\
      0        &          &         &          &    &      
   \end{matrix}
   \right]  
 \end{align}
%
and its $n \times n $ leading principal minor matrix by
%
\begin{align}\label{eq:Finite_Jacobi_Mn}
  \mathbf{J}_n=[\mathbf{J}(\mu)]_{[1:n,1:n]}. 
\end{align}
%
If one then lets $\vec{p}_n(s)
=[\tilde{p}_0(s),\tilde{p}_1(s),\ldots,\tilde{p}_{n-1}(s)]^T$ then equation  
\eqref{eq:ON_3term_recursion} may be expressed in matrix form as
\cite{Gautschi:2004:OP} 
%
\begin{align}\label{eq:Recursion_Matrix}
  \mathbf{J}_n\vec{p}_n(s)=s\vec{p}_n(s)
     -\sqrt{\beta_n}\tilde{p}_n(s)\vec{e}_n,
\end{align}
%
where $\vec{e}_n=[0,0,\ldots,1]$ is the $n$th standard basis vector in
$\mathbb{R}^n$. Therefore, the simple zeros of $\tilde{p}_n$,
$\{\lambda_{j,n}\}_{j=1}^n$, are the eigenvalues of the Jacobi
matrix $\mathbf{J}_n$, and the
$\{\vec{p}_n(\lambda_{i,n})\}_{i=1}^n$, are the
corresponding eigenvectors:   
%
\begin{align}\label{eq:Normalized_Eig_Jn}
  \mathbf{J}_n\vec{v}_i=\lambda_{i,n}\vec{v}_i, \quad
  \vec{v}_i^{\;T}\vec{v}_i=1, \quad
  \vec{v}_i=\frac{\vec{p}_n(\lambda_{i,n})}{\|\vec{p}_n(\lambda_{i,n})\|}, \quad
  \|\vec{p}_n(\lambda_{i,n})\|^2=\sum_{j=0}^{n-1}[\tilde{p}_j(\lambda_{i,n})]^2.
\end{align}
%
As $\Sigma_\mu\subset[0,1]$, the coefficients of the Jacobi matrix satisfy the
following bounds \cite{Gautschi:2004:OP}     
%
\begin{align}\label{eq:Bounded_Jacobi}
  0<\alpha_j<1,\quad 0<\beta_j\leq1.
\end{align}
%
Therefore, the matrix $\mathbf{J}(\mu)$ is a \emph{bounded} Jacobi
matrix, which in turn implies the measure of orthogonality $\mu$ is
unique \cite{Ismail:2005}. 

In the finite lattice setting, $\mathbf{M}=\mathbf{M}_n$ is a real symmetric,
sparse random matrix \cite{Golden:CMP-467} of size $n$, say, and the
discrete ($n$ dependent) spectral measure $\mu$ is given by equations
\eqref{eq:mu_measure}--\eqref{eq:Discrete_Fs}. In general, given a
real symmetric matrix 
$\mathbf{A}$ there is an orthogonal similarity transformation
$\mathbf{Q}^T\mathbf{A}\mathbf{Q}=\mathbf{T}$, where the orthogonal
matrix $\mathbf{Q}$ and the tridiagonal symmetric matrix $\mathbf{T}$,
having nonnegative off diagonal elements, are uniquely determined by
$\mathbf{A}$ and the first column of $\mathbf{Q}$
\cite{Parlett:1980}. We now construct a similarity transformation
which generates the Jacobi matrix, $\mathbf{J}_n$, associated with the
polynomials orthogonal to the measure $\mu$ \cite{Gautschi:2004:OP}.
%
\begin{theorem}\label{thm:Mn_Jacobi_Theorem}
%
Let $\mu(d\lambda)=\sum_{j=1}^nm_j\delta_{\lambda_j}(d\lambda)$ be the spectral measure of the
real symmetric matrix $\mathbf{M}_n$ of order $n$. Define the diagonal 
matrix $\mathbf{\Lambda}_n=\textrm{diag}[\lambda_j]_{j=1}^n$, and the vectors
$\vec{m}=[m_1,m_2,\ldots,m_n]^T$ and
$\sqrt{\vec{m}}=[\sqrt{m_1},\sqrt{m_2},\ldots,\sqrt{m_n}]^T$. Furthermore, 
let $\mathbf{J}_n(\mu)$ be the Jacobi matrix of order $n$ defined by
equations \eqref{eq:Infinite_Jacobi_Mn}--\eqref{eq:Normalized_Eig_Jn}
with eigenvalues $\{\lambda_{j,n}\}_{j=1}^n$ and eigenvectors
$\{\vec{p}_n(\lambda_{j,n})\}_{j=1}^n$, where 
$\vec{p}_n(s)=[\tilde{p}_0(s),\tilde{p}_1(s),\ldots,\tilde{p}_{n-1}(s)]^T$
and the $\{\tilde{p}_j\}_{j=0}^{n-1}$ are orthonormal with respect to
$\mu$. Denote the matrix of orthogonal eigenvectors by
$\mathbf{P}=[\vec{p}_n(\lambda_{1,n}),\vec{p}_n(\lambda_{2,n}),\ldots,\vec{p}_n(\lambda_{n,n})]$     
with norms $\xi_i=\|\vec{p}_n(\lambda_{i,n})\|$, and
$\mathbf{V}_n=[\vec{v}_1,\vec{v}_2,\ldots,\vec{v}_n]$ the corresponding 
matrix of orthonormal eigenvectors.

Then, there exists a unique orthogonal matrix $\mathbf{Q}_n$ and
a unique tridiagonal symmetric matrix $\mathbf{T}_n$ of order $n$, with
nonnegative off diagonal elements, given by
%
\begin{align}\label{eq:Spectral_Decomp_Jn}
  \left[
  \begin{matrix}
    1           &\vec{0}^{T}\\
    \vec{0}  &\mathbf{Q}_n^T
  \end{matrix}
  \right]
%
  \left[
  \begin{matrix}
    1                  &\sqrt{\vec{m}}^{\;T}\\
    \sqrt{\vec{m}}  &\mathbf{\Lambda}_n
  \end{matrix}
  \right]
%
  \left[
  \begin{matrix}
    1           &\vec{0}^{T}\\
    \vec{0}  &\mathbf{Q}_n
  \end{matrix}
  \right]
%  
  =
%
  \left[
  \begin{matrix}
    1           &\sqrt{\vec{m}}^{\;T}\mathbf{Q}_n\\
    \mathbf{Q}_n^T\sqrt{\vec{m}}  &\mathbf{Q}_n^T\mathbf{\Lambda}_n\mathbf{Q}_n
  \end{matrix}
  \right]
  % 
  =
%
  \left[
  \begin{matrix}
    1           &\sqrt{\beta_0}\vec{e}_1^{\;T}\\
    \sqrt{\beta_0}\vec{e}_1  & \mathbf{T}_n
  \end{matrix}
  \right],  
\end{align}
%
where $\vec{e}_1$ and $\vec{0}$ are the 1st standard basis vector
and null vector of $\mathbb{R}^n$, respectively, and $\beta_0=\beta_{0,n}=\|1\|_\mu^2$
depends on $n$ through the measure $\mu$. Furthermore,
$\mathbf{Q}_n\equiv\mathbf{V}_n^T$, $\mathbf{T}_n\equiv\mathbf{J}_n(\mu)$,
$m_i=\xi_i^{-2}$, and $\{\lambda_{j,n}\}\equiv\{\lambda_j\}$. 
%
\end{theorem}

% 
\begin{proof}
%  
Define the real symmetric matrix $\mathbf{A}$ of order $n+1$ by  
%
\begin{align}
  \mathbf{A}=
  \left[
  \begin{matrix}
    1                  &\sqrt{\vec{m}}^{\;T}\\
    \sqrt{\vec{m}}  &\mathbf{\Lambda}_n
  \end{matrix}
  \right].  
\end{align}
%
There exists a orthogonal similarity transformation
$\mathbf{Q}^T\mathbf{A}\mathbf{Q}=\mathbf{T}$, where the orthogonal
matrix $\mathbf{Q}$ and the tridiagonal symmetric matrix $\mathbf{T}$,
having nonnegative off diagonal elements, are uniquely determined by
$\mathbf{A}$ and the first column of $\mathbf{Q}$
\cite{Parlett:1980}. We require that the first column of $\mathbf{Q}$
is $\vec{e}_1$, the first standard basis vector of $\mathbb{R}^n$, i.e.,

% 
\begin{align}   
  \mathbf{Q}=
  \left[
  \begin{matrix}
    1           &\vec{0}^{T}\\
    \vec{0}  &\mathbf{Q}_n
  \end{matrix}
  \right],
\end{align}
%
where $\vec{0}$ is the null vector in $\mathbb{R}^n$. Therefore, the
orthogonal matrix $\mathbf{Q}_n$ of order $n$ and the 
tridiagonal matrix $\mathbf{T}$ of order $n+1$ are unique, and the
off diagonal elements of $\mathbf{T}$ are nonnegative
\cite{Parlett:1980}.

By construction, as $\mathbf{T}$ is tridiagonal,
%
\begin{align}
  \mathbf{T}=\mathbf{Q}^T\mathbf{A}\mathbf{Q}=
%
  \left[
    \begin{matrix}
    1           &\sqrt{\vec{m}}^{\;T}\mathbf{Q}_n\\
    \mathbf{Q}_n^T\sqrt{\vec{m}}  &\mathbf{Q}_n^T\mathbf{\Lambda}_n\mathbf{Q}_n
  \end{matrix}
  \right]
%  
  =
%
  \left[
  \begin{matrix}
    1           &c\vec{e}_1^{\;T}\\
    c\vec{e}_1  & \mathbf{T}_n
  \end{matrix}
  \right],  
\end{align}
%
for a unique constant $c\geq0$, where $\mathbf{T}_n$ is a unique
tridiagonal symmetric matrix with nonnegative off diagonal elements. Therefore, 
$\mathbf{T}_n=\mathbf{Q}_n^T\mathbf{\Lambda}_n\mathbf{Q}_n$, where
$\mathbf{Q}_n$ is a unique orthogonal matrix. By definition

$\mathbf{J}_n(\mu)=\mathbf{V}_n^T\mathbf{\Lambda}_n\mathbf{V}_n$ where, by the
ordering of the simple eigenvalues
$\lambda_{1,n}<\lambda_{2,n}<\cdots<\lambda_{n,n}$, $\mathbf{V}_n$ is a unique
orthogonal matrix, and $\mathbf{J}_n(\mu)$ is a tridiagonal symmetric
matrix with nonnegative off diagonal elements. Therefore we have that
$\mathbf{Q}_n\mathbf{T}_n\mathbf{Q}_n^T=\mathbf{V}_n\mathbf{J}_n(\mu)\mathbf{V}_n^T$
where $\mathbf{Q}_n$, $\mathbf{T}_n$, and $\mathbf{V}_n$ are
unique; which implies that $\mathbf{V}_n\equiv\mathbf{Q}_n^T$ and
$\mathbf{T}_n\equiv\mathbf{J}_n(\mu)$.   

We now show that
$\mathbf{V}_n\sqrt{\vec{m}}=\sqrt{\beta_0}\vec{e}_1$. Recalling that
$\tilde{p}_0=\beta_0^{-1/2}$, by orthonormality one obtains
$\beta_0^{1/2}\delta_{0,l}=\langle1,\tilde{p}_l\rangle_\mu=\sum_{j=1}^nm_j\tilde{p}_l(\lambda_{j,n})$, 
$l=0,1,\ldots,n-1$, or in matrix form 
%
\begin{align}\label{eq:Matrix_Orthonormality}
  \mathbf{P}\vec{m}=\beta_0^{1/2}\vec{e}_1,
\end{align}

% 
with $\mathbf{P}$ and $\vec{m}$ defined in the statement of the
theorem. Therefore if we denote $\Xi=\text{diag}[\xi_1,\xi_2,\ldots,\xi_n]$, with
$\xi_i=\|\vec{p}_n(\lambda_{i,n})\|$ defined in the statement of the theorem,
and $\vec{1}=[1,1,\ldots,1]^T\in\mathbb{R}^n$, then multiplying
\eqref{eq:Matrix_Orthonormality} from the left by $\mathbf{P}^T$ we
obtain 
%
\begin{align}\label{eq:Recover_Cristoffel_nums}
  \Xi^2\vec{m}=\beta_0^{1/2}\mathbf{P}^T\vec{e}_1=\beta_0^{1/2}\beta_0^{-1/2}\vec{1}=\vec{1}
\end{align}
%
Therefore, from \eqref{eq:Recover_Cristoffel_nums} we recover the
Christoffel numbers \eqref{eq:Christoffel_nums} of $\mathbf{M}_n$ from
the norms of the eigenvectors of $\mathbf{J}_n$:
$\vec{m}=\Xi^{-2}\vec{1}$. Or in coordinate form  
%
\begin{align}\label{eq:Coordinate_Cristoffel_nums}
  m_i=\xi_i^{-2}, \qquad \xi_i:=\|\vec{p}_n(\lambda_{i,n})\|, \quad i=1,2,\ldots,n.
\end{align}
%

Equivalent to \eqref{eq:Coordinate_Cristoffel_nums} is
$\sqrt{m_i}=\|\vec{p}_n(\lambda_{i,n})\|^{-1}$. Therefore, recalling that the first component
of the vectors $v_{i,1}=\beta_0^{-1/2}\|\vec{p}_n(\lambda_{i,n})\|^{-1}$, we have
$\sqrt{m_i}=\beta_0^{1/2}v_{i,1}$. Or in matrix form
$\sqrt{\vec{m}}^{\;T}=\sqrt{\beta_0}\vec{e}_1^{\;T}\mathbf{V}$, where
$\sqrt{\vec{m}}^{\;T}$ is defined in the statement of the
theorem. Therefore, by the orthogonality of the matrix $\mathbf{V}$,
transposition of this expression yields    
%
\begin{align}
  \mathbf{V}\sqrt{\vec{m}}=\sqrt{\beta_0}\vec{e}_1,
\end{align}
%
which concludes the proof of the theorem.
%
\end{proof}

The procedure leading to equation \eqref{eq:Spectral_Decomp_Jn} 
defines a \emph{Lanczos}--type algorithm. A stable variant of this
algorithm \cite{Gautschi:2004:OP} accurately produces $\mathbf{Q}$ and 
$\mathbf{T}$, given the matrix $\mathbf{A}$. The three term recursion
relation coefficients of the Jacobi operator associated with
$\mathbf{M}_n$ can thus be determined by this algorithm (see section
\ref{sec:Calc_Spec_Meas_Comp_Micro}). The matrix $\mathbf{T}$ is the
``extended'' Jacobi matrix on the right of
\eqref{eq:Spectral_Decomp_Jn}. The nonzero elements of the tridiagonal
sub-matrix $\mathbf{J}_n$ are precisely the recursion coefficients,
$\{\alpha_{j,n}\}_{j=0}^n$ and $\{\sqrt{\beta_{j,n}}\}_{j=1}^n$, with
$\beta_0:=\sum_{j=1}^nm_j=p_1$ \cite{Gautschi:2004:OP}.

We have shown that, given an infinite binary composite medium
described by the bounded self adjoint operator $\mathbf{M}=\chi_1\Gamma\chi_1$,
The Spectral Theorem \ref{thm:Spectral_Theorem} 
gives the existence of a positive spectral measure $\mu$ with support
$\Sigma_\mu\subseteq[0,1]$. The measure $\mu$ defines an infinite sequence of
orthogonal polynomials, which in turn defines a bounded Jacobi matrix
$\mathbf{J}$ of infinite order. There is, in fact, a one--to--one
mapping between bounded Jacobi matrices of infinite order and positive
measures with compact support. In the matrix case this map becomes a
bijection and the spectral measure for $\mathbf{J}_n$ is parameterized
by $2n$ real numbers:  
%
\begin{align}
  \left\{(\lambda_1<\lambda_2<\cdots<\lambda_n,\xi_1^{-1},\xi_2^{-1},\ldots,\xi_n^{-1})
      | \ \xi_i>0, \ \sum_{j=1}^n\left(\xi_j^{-1}\right)^2=p_1\right\}.
\end{align}
%
These results are summarized by Theorem \ref{thm:Spectral_Theorem_J}
below \cite{Deift:2000:RMT}. To characterize the mapping to $\mu$ in the
matrix case we define 
%
\begin{align}\label{eq:mu_Characterization}
  \mathscr{A}_{p_1}:=\left\{(\gamma_1<\gamma_2<\cdots<\gamma_n,\theta_1,\theta_2,\ldots,\theta_n)| \ \theta_i>0, \ \sum_{j=1}^n\theta_j^2=p_1\right\}.
\end{align}
%
%
\begin{theorem}[Spectral Theorem for Jacobi Matrices]
  \label{thm:Spectral_Theorem_J}
  Let $\mathcal{B}_J$ denote the set of bounded Jacobi matrices of

  infinite order and $\mathcal{B}_{J_n}$ denote the set of bounded Jacobi
  matrices of order $n$. Let $\mathscr{B}$ denote the set of positive
  finite Borel measures on $\mathbb{R}$ with compact support, and
  $\mathcal{M}_{p_1}:=\{\mu\in\mathscr{B}|\int d\mu=p_1\}$. There exists a one--to--one map
  $\psi=\varphi^{-1}$, where $\psi:\mathcal{M}_{p_1}\mapsto\mathcal{B}_J$, such that for all
  $\mu\in\mathcal{M}_{p_1}$ there exists $\mathbf{J}\in\mathcal{B}_J$ such that
  $\mu=\varphi(\mathbf{J})=\varphi\circ\psi(\mu)$. Furthermore, for all
  $\mathbf{J}\in\mathcal{B}_J$ there exists a unique $\mu\in\mathcal{M}_{p_1}$ 
  such that 
  %
  \begin{align}
    \langle(s-\mathbf{J})^{-1}\vec{e_0},\vec{e_0}\rangle=\int\frac{d\mu(\lambda)}{s-\lambda}
  \end{align}
  %
  Moreover, $\mathbf{J}\stackrel{\varphi}{\mapsto}d\mu$ is the spectral map in the
  sense that the map
  %
  \begin{align}
    l_2\ni \pi(\mathbf{J})\vec{e_0}\stackrel{\mathbf{U}}{\mapsto} \pi(s)\in L^2(\mu), \quad
    \pi  \text{ a polynomial},
  \end{align}
%
  extends to a unitary map and
  %
  \begin{align}
    (\mathbf{U}\mathbf{J}\mathbf{U}^{-1}f)(s)=sf(s),\quad
    \forall f\in L^2(\mu).
  \end{align}
  %
                      
  In the matrix case, the spectral map
  %
  $$
  \mathcal{B}_{J_n}\stackrel{\varphi}{\mapsto}
     \mu=\sum_{j=1}^n\xi_j^{-2}\delta_{\lambda_j}\mapsto(\lambda_1,\lambda_2,\cdots,\lambda_n,\xi_1^{-1},\xi_2^{-1},\ldots,\xi_n^{-1})\in\mathscr{A}_{p_1}     
  $$
  %
  is a bijection, where the weights $\xi_i=\|\vec{p}_n(\lambda_i)\|$ are
  defined in Theorem \ref{eq:Spectral_Decomp_Jn} and $\mathscr{A}_{p_1}$ is
  defined in \eqref{eq:mu_Characterization}. In particular, for fixed
  $\mathbf{\Lambda}_0=\text{diag}(\lambda_{10},\lambda_{20},\ldots,\lambda_{n0})$, the isospectral
  set $\mathscr{M}_{\Lambda_0}$, where
  %
  $$
  \mathscr{M}_{\Lambda_0}:=\{\mathbf{J}\in\mathcal{B}_J: 
                \text{spec }\mathbf{J}=(\lambda_{10},\lambda_{20},\ldots,\lambda_{n0})\},
  $$
  %
  is the positive n--tant of the sphere with radius $p_1$,
  $S_{p_1}^{n-1}$ in $\mathbb{R}^n$.     
%  
\end{theorem}

% 
For a proof of Theorem \ref{thm:Spectral_Theorem_J} see
\cite{Deift:2000:RMT}. In the general case, the spectral map
$\mathbf{J}\stackrel{\varphi}{\mapsto}\mu$ takes $\mathcal{B}_J$ into the
hyperplane $\mathcal{M}_{p_1}\subset\mathscr{B}$
\cite{Deift:2000:RMT}. Given $\mathbf{J}_0\in\mathcal{B}_J$, one
considers the isospectral set
$\mathscr{M}_0:=\{\mathbf{J}\in\mathcal{B}_J:  
                \text{spec }\mathbf{J}=\text{spec }\mathbf{J}_0\}$.
Under $\varphi$, $\mathscr{M}_0$ is injected into the hyperplane
$\mathcal{M}_{p_1}$ \cite{Deift:2000:RMT}. The geometry of this
injection, i.e., how $\varphi(\mathscr{M}_0)$ lies inside
$\mathcal{M}_{p_1}$, can be conveniently described in terms of basic
notations of real analysis \cite{Deift:2000:RMT,Deift:JFA:358}.
  

We now summarize the mapping structure associated with the effective
parameter $\epsilon^*$. Define a random geometry $(\Omega,P)$ and denote by
$\mathcal{B}_\times$, the class of bounded linear self--adjoint
transformations with domain $\mathscr{H}_\times\subset L^2(\Omega,P)$, unit norm,
and simple spectrum, where $\mathscr{H}_\times$ is defined in equation
\eqref{eq:curlfreeHilbert}. The resolvent representation of the  
electric field \eqref{eq:Resolvent_Rep_Es} defines a map
$\varphi_\omega:\mathscr{H}_\times\mapsto\mathcal{B}_\times\ni\mathbf{M}=\chi_1\Gamma\chi_1$. The Spectral
Theorem \ref{thm:Spectral_Theorem} 
subsequently defines a map
$\psi_\mu:\mathcal{B}_\times\mapsto\mathcal{M}_{p_1}\ni\mu=\langle E_\lambda\vec{e}_k,\vec{e}_k\rangle$,
where $\langle\cdot,\cdot\rangle$ is the $L^2(\Omega,P)$ inner product, $E_\lambda$ is the
resolution of the identity corresponding to $\mathbf{M}$, and
$\vec{e}_k$ is defined in section \ref{sec:Background_TACM}. The 
one--to--one map $\psi:\mathcal{M}_{p_1}\mapsto\mathcal{B}_J$, with $\psi=\varphi^{-1}$,
defines a bounded Jacobi matrix $\mathbf{J}(\mu)$ of infinite order, with
spectral measure $\mu$, which acts on $l_2$. The map $\varphi$ extends to a
unitary map $\mathbf{U}:l_2\mapsto\mathbb{P}\subset L^2(\mu)$, where $\mathbb{P}$
denotes the set of real polynomials on $[0,1]$. Pictorially, we have 
%
\begin{align*}
  (\mathscr{H}_\times\subset L^2(\Omega,P))\stackrel{\varphi_\omega}{\longrightarrow}(\mathbf{M}\in\mathcal{B}_\times)
  \stackrel{\psi_\mu}{\longrightarrow}(\mu\in\mathcal{M}_{p_1})
  \mathop{\longleftrightarrow}_{\varphi}^{\psi}(\mathbf{J}(\mu)\in\mathcal{B}_J)
  \mathop{\longleftrightarrow}_{\mathbf{U}^{-1}}^{\mathbf{U}}(\pi(s)\in L^2(\mu)).
\end{align*}
%
The construction of $\psi=\varphi^{-1}$ turns out to be the classical problem of
the construction of the polynomials orthogonal to the given measure
$\mu\in\mathcal{M}_{p_1}$ \cite{Deift:2000:RMT}. Even though each random
geometry $(\Omega,P)$ gives rise to a measure $\mu\in\mathcal{M}_{p_1}$ with
Stieltjes transformation $F(s)$ \eqref{eq:Fs_Integral}, hence a
bounded Jacobi matrix, it is known \cite{Golden:CMP-473} that not
every measure $\mu\in\mathscr{B}(\mu_1,\mu_2,\ldots,\mu_n)\subset\mathcal{M}_{p_1}$ gives a
function $m(h)=1-F(s)$ that is the effective (relative) dielectric
constant of a random medium, where $\mathscr{B}(\mu_1,\mu_2,\ldots,\mu_n)$ is
defined in \eqref{eq:Set_of_mu_Measures}.
  
%%%%%%%%%%%%%%%%%%%%%%%%%%%%%%%%%%%%%%%%%%%%%%%%%%%%%%%%%%%%%%%
\noindent \textbf{Generalized Numerator Polynomials}\newline
%%%%%%%%%%%%%%%%%%%%%%%%%%%%%%%%%%%%%%%%%%%%%%%%%%%%%%%%%%%%%%%
%
By equation \eqref{eq:Herglotz_Energy_Rep_E}, we see that there is
physical significance to the Stieltjes transformation, $F(s)$, of the
denominator polynomial orthogonality measure $\mu$. Here, we
demonstrate that the Stieltjes transformation of the orthogonality
measures, $\mu^{[i]}$, associated with the generalized numerator
polynomials, for $i=1,2$, also have physical
significance. Furthermore, we construct a closed form representation
of the measures $\mu^{[i]}$, for all $i=1,2,\ldots$, in terms of the
moments of $\mu$.   

The generalized monic numerator polynomials of order
$i\geq0$, where $i\in\mathbb{Z}$, are defined to be the solution of the
three--term recursion relation
\cite{Gautschi:2004:OP,Assche:JCAM:1991:237}     
%
\begin{align}\label{eq:NumeratorOP_recurrence_rel}
  P^{[i]}_{n+1}(s)&=(s-\alpha_{n+i})P^{[i]}_{n}(s)-\beta_{n+i}P^{[i]}_{n-1}(s),
                            \quad n=0,1,2,\ldots\\
  &P^{[i]}_{-1}(s)=0, \quad  P^{[i]}_{0}(s)=1.\notag
\end{align}
%
The monic denominator polynomials are given by $P_n=P^{[0]}_n$ and the
monic numerator polynomials of the Pad\'{e} approximants of $F(s)$
\eqref{eq:Pade_error} are given by $Q_{n-1}=P^{[1]}_{n-1}$
\cite{Gautschi:2004:OP,Assche:JCAM:1991:237}.

It is an elementary exercise to show that a sequence of orthogonal
polynomials satisfy a three--term recursion relation
\cite{Gautschi:2004:OP,Deift:2000:RMT,Ismail:2005}. The converse is
known as Favard's Theorem \cite{Gautschi:2004:OP}. It states that if
an (infinite) sequence of polynomials satisfy a three--term recurrence
relation, such as \eqref{eq:NumeratorOP_recurrence_rel}, with all the
$\beta_k$ positive and $\alpha_k\in\mathbb{R}$, then there exists a orthogonality
measure for this sequence (see Theorems 10.23 page 531, 10.25 page
539, and 10.27 page 545 in \cite{Stone:64}).

The Stieltjes transformation of the orthogonality measures, $\mu^{[i]}$, of
the $P_n^{[i]}$ can be expressed in terms of Cauchy integrals of $\mu$: 
%
\begin{align}\label{eq:Stieltjes_num_denom_meas_rel}
  F^{[i]}(s)=\int_{\mathbb{R}}\frac{d\mu^{[i]}(\lambda)}{s-\lambda}=\frac{1}{\beta_i}\frac{\rho_i(s)}{\rho_{i-1}(s)},
  \quad \rho_i(s)=\int_{\mathbb{R}}\frac{P_i(\lambda)d\mu(\lambda)}{s-\lambda}, \quad \rho_{-1}(s):=1,
\end{align}
%
(see \cite{Assche:JCAM:1991:237} equation 3.7).  If we denote
$P_i(s):=\sum_{j=0}^ia_js^j$ and $P_{i-1}(s):=\sum_{j=0}^{i-1}b_js^j$,
$a_i=b_{i-1}=1$, then $\rho_i(s)=\sum_{j=0}^ia_jF_j(s)$ and
$\rho_{i-1}(s)=\sum_{j=0}^{i-1}b_jF_j(s)$, where $F_j$ is defined in
\eqref{Eq:Energy_Basis_F} below. Therefore equation
\eqref{eq:Stieltjes_num_denom_meas_rel} may be written as 
%
\begin{align}\label{Eq:Energy_Basis_F}
   \beta_i\left[\sum_{j=0}^{i-1}b_jF_j(s)\right]F^{[i]}(s)=\sum_{j=0}^ia_jF_j(s),
   \quad
   F_j(s):=\int_0^1\frac{\lambda^j d\mu(\lambda)}{s-\lambda}.
\end{align}
%

The orthogonality measures, $\mu^{[i]}$, may be found using the spectral
resolution of the corresponding essentially self--adjoint Jacobi
operator defined by \eqref{eq:NumeratorOP_recurrence_rel} (see Theorem
10.23 page 531 in \cite{Stone:64}). By equation
\eqref{Eq:Energy_Basis_F}, the measures $\mu^{[i]}$ may also be
found in terms of the measures $\lambda^j\mu(d\lambda)$, for $j=0,1,2,\ldots,i$, via The
Stieltjes--Perron Inversion Theorem \eqref{eq:Stieltjes-Perron} (see
the formula after equation 3.7 in
\cite{Assche:JCAM:1991:237}). Moreover, as $\Sigma_\mu\subset[0,1]$, the
$\Sigma_{\mu^{[i]}}$ are compact intervals for all $i\geq0$
\cite{Chihara:1978}. Therefore, the measures $\mu^{[i]}$ are also
uniquely determined by their moments \cite{Shohat:1963}. Indeed, it is
known \cite{Ismail:2005} that the roots of $Q_{n-1}$,
$\{\lambda_{j,n-1}^{[1]}\}_{j=1}^{n-1}$, interlace the roots of $P_n$,
$\{\lambda_{i,n}\}_{j=1}^{n}$, i.e., 
% 
\begin{align}\label{eq:InterlaceQP}
  0<\lambda_{1,n}<\lambda_{1,n-1}^{[1]}<\lambda_{2,n}<\lambda_{2,n-1}^{[1]}<\cdots<\lambda_{n-1,n-1}^{[1]}<\lambda_{n,n}<1,
\end{align}
%
so that $\Sigma_{\mu^{[1]}}\subseteq\Sigma_\mu\subseteq[0,1]$. Moreover, it can be shown that for
all $i\in\mathbb{N}$, the support of $\mu^{[i]}$,
$\Sigma_{\mu^{[i]}}\subseteq\text{co(supp}(\mu))$, where $\text{co(supp}(\mu))$, defined  
in \cite{Chihara:1978}, is a compact interval. Therefore the moment
problem for $\mu^{[i]}$ is determined for all $i\geq0$ \cite{Shohat:1963}. 

We now construct a closed form solution of the measures $\mu^{[i]}$,
for all $i\geq1$, in terms of the moments of $\mu$. Applying the expansion
\eqref{eq:Fs_near_infinity} to the $F_j$ in equation
\eqref{Eq:Energy_Basis_F}, switching the order of the infinite and
finite sums, and setting $\mu_j^{[1]}:=\int_0^1\lambda^jd\mu^{[1]}(\lambda)$ we have 
%
\begin{align}\label{eq:Num_meas_mom_exp}
 \beta_i\left[\sum_{l=i-1}^\infty\frac{\langle\lambda^l,P_{i-1}\rangle_\mu}{s^{l+1}}\right]
     \left[\sum_{l=0}^\infty\frac{\mu_l^{[i]}}{s^{l+1}}\right]=\sum_{l=i}^\infty\frac{\langle\lambda^l,P_i\rangle_\mu}{s^{l+1}},
\end{align}
%
where $\langle\cdot,\cdot\rangle_\mu$ denotes the $L^2(\mu)$ inner product and the change in
the lower sum indices follows from orthogonality of the $P_j$ with
respect to this inner product. The moments, $\mu_j^{[i]}$, of the
measures $\mu^{[i]}$ may be found by equating the coefficients of powers
of $s^{-1}$ in equation \eqref{eq:Num_meas_mom_exp}. Doing so defines
an infinite (linear) system equations, $\mathbf{L}_i\vec{x}=\vec{b}$,
where $x_j=\mu_j^{[i]}$ and $b_j=\langle\lambda^{i+j},P_i\rangle_\mu/\beta_i$, for $j=0,1,2,\ldots$,
and $\mathbf{L}_i$ is a lower triangular Toeplitz convolution matrix: 
%
\begin{align}
  \left[
    \begin{matrix}
      \langle\lambda^{i-1},P_{i-1}\rangle_\mu & 0 & 0 & 0&\cdots\\
      \langle\lambda^i,P_{i-1}\rangle_\mu &\langle\lambda^{i-1},P_{i-1}\rangle_\mu & 0 & 0&\cdots\\
      \langle\lambda^{i+1},P_{i-1}\rangle_\mu &\langle\lambda^i,P_{i-1}\rangle_\mu &\langle\lambda^{i-1},P_{i-1}\rangle_\mu & 0&\cdots\\
     % \langle\lambda^{i+2},P_{i-1}\rangle_\mu &\langle\lambda^{i+1},P_{i-1}\rangle_\mu &\langle\lambda^i,P_{i-1}\rangle_\mu & \langle\lambda^{i-1},P_{i-1}\rangle_\mu&0&\cdots\\ 
      \vdots  &  \vdots & \vdots &\ddots &\cdots
    \end{matrix}
  \right]
  \left[
    \begin{matrix}
      \mu_0^{[i]} \\ \mu_1^{[i]} \\ \mu_2^{[i]} \\  \vdots  
    \end{matrix}
  \right]=
  \left[
    \begin{matrix}
       \langle\lambda^i,P_i\rangle_\mu/\beta_i \\  \langle\lambda^{i+1},P_i\rangle_\mu/\beta_i \\  \langle\lambda^{i+2},P_i\rangle_\mu/\beta_i\\  \vdots  
    \end{matrix}
  \right].          
\end{align}
%
Therefore the moments of the measure $\mu^{[i]}$ may be found by
back substitution, and are given recursively by  
%
\begin{align}  
  \mu_j^{[i]}=\frac{1}{\langle\lambda^{i-1},P_{i-1}\rangle_\mu}
        \left[\frac{ \langle\lambda^{i+j},P_i\rangle_\mu}{\beta_i}
               -\sum_{l=0}^{j-1}\langle\lambda^{(i+j-1)-l},P_{i-1}\rangle_\mu \; \mu_l^{[i]}
       \right]\quad  
\end{align}
%
for all $i=1,2,\ldots$ and $j=0,1,\ldots$, where empty sums are understood to
be zero. 

The physical significance of equations
\eqref{eq:Stieltjes_num_denom_meas_rel}-- \eqref{Eq:Energy_Basis_F}
lies in the Herglotz decomposition of the system energy. In
particular, in Theorem \ref{thm:Herglotz_Decomp_Energy} of section 
\ref{subsec:Spec_Decomp_Energy} below, we will show that the functions
$F_j(s)$, for $j=0,1,2$, are given explicitly in terms of partitions of
the system energy. Indeed, from equation \eqref{Eq:Energy_Basis_F} we
see that $F^{[0]}(s)=F(s)/p_1$, $F^{[1]}(s)$, and $F^{[2]}(s)$ are
given in terms of these functions, and in Theorem 
\ref{thm:Herglotz_Decomp_Energy} of section
\ref{subsec:Spec_Decomp_Energy} below we show that    
%
\begin{align}\label{eq:Energy_partition_basis}
  sF(s)=\langle\chi_1\vec{E}\cdot\vec{E}_0\rangle/E_0^2, \quad
  F_1(s)=\langle\chi_1\vec{E}_f\cdot\vec{E}_0\rangle/E_0^2, \quad
  F_2(s)=-\int_\infty^sds\langle\chi_1E_f^2\rangle/E_0^2,
\end{align}
%
where $\lim_{|s|\to\infty}F_2(s)=0$ and $\langle\cdot\rangle$ denotes ensemble average over $\Omega$
or spatial average over all of ${\mathbb{R}}^d$. There, we also show
that these three functions $F_j(s)$, for $j=0,1,2$, and their first
derivatives in $s$, serve as basis functions for a detailed
decomposition of the system energy in terms of Herglotz functions
involving $\mu$. 

By equation \eqref{eq:NumeratorOP_recurrence_rel},
with $j=0$, we have that $P_0(\lambda)=1$, $P_1(\lambda)=\lambda-\alpha_0$, and
$P_2(\lambda)=\lambda^2-(\alpha_0+\alpha_1)\lambda+\alpha_0\alpha_1-\beta_1$. Therefore, by equation
\eqref{eq:Energy_partition_basis} and equation
\eqref{Eq:Energy_Basis_F}, for $j=1,2$ and for
$s\in\mathbb{C}\backslash(\Sigma_\mu\cup\text{co(supp}(\mu)))$, we have 
% %
% \begin{align}\label{eq:Numerator_measure_ST}  
%   \beta_1F(s)F^{[1]}(s)&=F_1(s)-\alpha_0F(s),\\
%   \beta_2(F_1(s)-\alpha_0F(s))F^{[2]}(s)&=F_2(s)-(\alpha_0+\alpha_1)F_1(s)+(\alpha_0\alpha_1-\beta_1)F(s)\notag
% \end{align}
% %
% Therefore, for $s\in\mathbb{C}\backslash(\Sigma_\mu\cup\text{co(supp}(\mu)))$    
%
\begin{align}\label{eq:Num_Poly_Energy_Rep}
 F^{[1]}(s)&=
    \left(
      \frac{\langle\chi_1\vec{E}_f\cdot\vec{E}_0\rangle-(\alpha_0/s)\langle\chi_1\vec{E}\cdot\vec{E}_0\rangle}
           {(\beta_1/s)\langle\chi_1\vec{E}\cdot\vec{E}_0\rangle}
    \right),\\
 F^{[2]}(s)&=
    \frac{-\int_\infty^sds\langle\chi_1E_f^2\rangle-(\alpha_0+\alpha_1)\langle\chi_1\vec{E}_f\cdot\vec{E}_0\rangle
            +[(\alpha_0\alpha_1-\beta_1)/s]\langle\chi_1\vec{E}\cdot\vec{E}_0\rangle}
          {\beta_2(\langle\chi_1\vec{E}_f \cdot\vec{E}_0\rangle -(\alpha_0/s)\langle\chi_1\vec{E}\cdot\vec{E}_0\rangle)}.
          \notag
\end{align}
%
It is interesting to note that a possible instability occurs in $F^{[2]}(s)$
when its denominator, $\langle\chi_1[\vec{E}_f -(\alpha_0/s)\vec{E}] \cdot\vec{E}_0\rangle$,
becomes small. There can be no such instability for $F^{[1]}(s)$ as
$F(s)>0$ (uniformly) for all $s$ in any compact subset of
$\mathbb{C}\backslash\Sigma_\mu$ \cite{Golden:CMP-473}. Equations
\eqref{eq:Num_Poly_Energy_Rep} are general formulas holding for two
component stationary random media in the lattice and continuum
settings \cite{Golden:PRL-3935}. Clearly, for all $i\geq0$, the
$F^{[i]}(s)$ may be expressed in terms of the $F_j(s)$, for
$j=0,1,\ldots,i$, and therefore may be expressed in terms of the energy
basis \eqref{eq:Energy_partition_basis}. However, only for $i=0,1,2$
are the $F^{[i]}(s)$ given completely in terms of this basis. 
%
\subsection{General Considerations}\label{sec:General_Considerations}
%
A thoroughly satisfactory theory connecting linear transformations and
infinite matrices can be developed only when the transformation is also
bounded \cite{Stone:64}. The connection arises through the existence of a
coordinate system in a Hilbert space $\mathscr{H}$. More specifically,
one can show that there exists a complete orthonormal set
$\{\psi_j\}_{j=1}^\infty$ in $\mathscr{H}$ such that if $f\in\mathscr{H}$ then we 
may define a system of coordinates $f=(x_1,x_2,\ldots)$
via. $f=\sum_{j=1}^\infty x_j\psi_j$, where $x_i=\langle f,\psi_i\rangle$ and $\langle\cdot,\cdot\rangle$ is the
inner product of $\mathscr{H}$ \cite{Stone:64}. A transformation
$\mathbf{T}$ in $\mathscr{H}$ takes each element $(x_1,x_2,\ldots)$ of its
domain $\mathcal{D}$ into the corresponding element $(y_1,y_2,\ldots)$ of
its range $\mathcal{R}$. Thus $\mathbf{T}$ determines an infinite set of
equations 
%
\begin{align}
  y_i=f_i(x_1,x_2,\ldots), \quad i=1,2,\ldots,
\end{align}
%
where the functions $f_1,f_2,\ldots$ are functions of infinitely many
variables defined throughout $\mathcal{D}$. When the functions are
linear and homogeneous this equation takes a special form
%
\begin{align}\label{eq:Def_Lin_Transform}
  y_i=\sum_{j=1}^\infty a_{i,j}x_j \quad i=1,2,\ldots,
\end{align}
%
characterized by the infinite matrix $\mathbf{A}=\{a_{i,j}\}$, where
$\mathcal{D}$ must be restricted so that the formula
\eqref{eq:Def_Lin_Transform} is valid. The transformation $\mathbf{T}$
is evidently linear. Therefore, to an arbitrary infinite matrix
$\mathbf{A}$ there corresponds a linear transformation $\mathbf{T}$
defined in terms of the coordinate system introduced in $\mathscr{H}$
\cite{Stone:64}. 

For the class of bounded linear transformations $\mathcal{B}$, with
domain $\mathscr{H}$, the correspondence with infinite matrices may be
made precise. Indeed (see Theorem 3.5 page 93 in \cite{Stone:64}),
%
\begin{theorem}\label{thm:Bnd_Lin_Op_Matrix}
%
let $\{\psi_j\}_{j=1}^\infty$ be an arbitrary complete
orthonormal set in $\mathscr{H}$, and denote 
by $\mathcal{B}_M$ the class of all bounded matrices. Then to each
$\mathbf{T}\in\mathcal{B}$ there corresponds a matrix
$\mathbf{A}=\{a_{i,j}\}=\{(T\psi_i,\psi_j)\}$ in $\mathcal{B}_M$, where $(\cdot,\cdot)$
is the inner product on $\mathscr{H}$. Furthermore, each
$\mathbf{A}\in\mathcal{B}_M$ defines a transformation $\mathbf{T}$ in
$\mathcal{B}$. This correspondence between $\mathcal{B}$ and
$\mathcal{B}_M$ is a one--to--one correspondence
%
\end{theorem}
     
\textbf{Note} that in this section we use the symbols $F,G$ to denote
arbitrary real valued functions, not to be confused with the functions
defined in section \ref{sec:Background_TACM} associated with the
effective parameter $\epsilon^*$. When we consider the bounded linear
self--adjoint operator $\mathbf{M}=\chi_1\Gamma\chi_1$ and the Hilbert space
$\mathscr{H}_\times$, the correspondence discussed in the preceding
paragraph has far reaching consequences, through a general operational
calculus applicable to an arbitrary self--adjoint operator. This
calculus may be based upon the Radon-Stieltjes integral: if $E_\lambda$ is
the resolution of the identity corresponding to $\mathbf{M}$ and if
$F(\lambda)$ is a real valued function, we define a transformation
$\mathbf{T}(F)$ by means of the equations \cite{Stone:64} 
%
\begin{align}\label{eq:Def_Radon-Stieltjes_int}
  \left\langle\mathbf{T}(F)\vec{f},\vec{g}\right\rangle=
      \int_0^1F(\lambda)d\left\langle E_\lambda\vec{f},\vec{g}\right\rangle,
\end{align}
%
where $\vec{f},\vec{g}\in\mathscr{H}_\times$ and $\langle\cdot,\cdot\rangle$ is the inner
product on $\mathscr{H}_\times$. The integral
\eqref{eq:Def_Radon-Stieltjes_int} is a Radon-Stieltjes integral. We 
will be primarily concerned with the vector
$\vec{e}_k\in\mathscr{H}_\times$, where $\vec{E}_0=E_0\vec{e}_k$ is
defined in section \ref{sec:Background_TACM}, and transformations of
the form
%
\begin{align}\label{eq:Def_Radon-Stieltjes_int_ek}
  \|\mathbf{T}(F)\vec{f}\|^2=\int_0^1F(\lambda)d\mu(\lambda),
  \quad \mu(\lambda)=\|E_\lambda\vec{e}_k\|^2,
\end{align}
%
where $\mu$ is the spectral measure associated with $\mathbf{M}$ and
$\vec{e}_k$. Formula \eqref{eq:Def_Radon-Stieltjes_int_ek} is
essentially The Spectral Theorem \ref{thm:Spectral_Theorem}. The
difference in notation is justified as $E_\lambda$ is a self adjoint projection,
hence $\langle E_\lambda\vec{e}_k,\vec{e}_k\rangle=\|E_\lambda\vec{e}_k\|^2$ (see Theorem 2.25
page 70 in \cite{Stone:64}). The following paragraph summarizes some
key technical properties of the bounded self--adjoint operator
$\mathbf{M}$ on $\mathscr{H}_\times$.

Let $L^2(\mu):=L^2(\vec{e}_k)$ denote the Hilbert space
consisting of the class of all real-valued $\mu$--measurable functions
$F(\lambda)$ such that $|F(\lambda)|^2$ is $\mu$-integrable, with the inner product
%
\begin{align}
  \langle F,G\rangle_\mu=\int_0^1F(\lambda)G(\lambda)d\mu(\lambda).
\end{align}
%
If $\mathfrak{M}(\vec{e}_k)$ denotes the set of all
elements $f^*=\mathbf{M}(F)f$, where $F\in L^2(\vec{e}_k)$ and
$f\in\mathscr{H}_\times$, then the correspondence $f^*\sim F(\lambda)$ defines
an isometric isomorphism between $\mathfrak{M}(\vec{e}_k)$ and
$L^2(\vec{e}_k)$, and $\mathfrak{M}(\vec{e}_k)$ is a closed linear
manifold (see Theorem 6.2 page 226 in \cite{Stone:64}). Furthermore,
let  $E_\lambda$ be the resolution of the identity corresponding to the
bounded self--adjoint operator $\mathbf{M}$ on $\mathscr{H}_\times$
and $E$ be the projection of $\mathscr{H}_\times$ on
$\mathfrak{M}(\vec{e}_k)$. Then $\mathfrak{M}(\vec{e}_k)$ is the
closed linear manifold  determined by the set of elements
$E_\lambda\vec{e}_k$, $0\leq\lambda\leq1$; it reduces both $E_\lambda$ and $\mathbf{M}$ (see
Theorem 7.2 page 243 in \cite{Stone:64}). Consequently
$\mathfrak{M}(\vec{e}_k)$ is a Hilbert space. Finally, as $\mathbf{M}$ 
has simple spectrum, the isometric correspondence between
$\mathfrak{M}(\vec{e}_k)$ and $L^2(\vec{e}_k)$ carries $\mathbf{M}$
into a self--adjoint operator $\mathbf{M}_1$ in $L^2(\vec{e}_k)$. The
domain $\mathcal{D}_1$ of $\mathbf{M}_1$ comprises those and only
those elements $F\in L^2(\vec{e}_k)$ for which the integral
$\int_0^1\lambda^2|F(\lambda)|^2d\mu(\lambda)$ exists and $\mathbf{M}_1$ takes an element
$F\in\mathcal{D}_1$ in to the element $\lambda F(\lambda)$ (see Theorem 7.10 page 277
in \cite{Stone:64}). Using the properties of $\mathbf{M}$ summarized
in this paragraph we are now ready to state the main results of this
section.

\begin{theorem}[Stone]\label{thm:Matrix_Rep_M}
%
A complete orthonormal set $\{g_j\}$ in $\mathfrak{M}(\vec{e}_k)$
can be selected such that the matrix $\mathbf{A}(g)$ associated with
$\mathbf{M}$ by $\{g_j\}$ has the following properties:
%
\begin{itemize}
%  
\item[(1)] its elements are expressible in the form
%
\begin{align}\label{eq:Matrix_for_M}
  a_{jl}&=\langle\mathbf{M}g_j,g_l\rangle=\langle\lambda G_j,G_l\rangle_\mu=\int_0^1\lambda G_j(\lambda)G_l(\lambda)d\mu(\lambda),\\
  \mu(\lambda)&=\|E_\lambda\vec{e}_k\|^2,\notag  
\end{align}
%
where $\{G_j\}$ is a complete orthonormal set in $L^2(\vec{e}_k)$ such
that the following integrals exist: $\int_0^1\lambda^2|G_i(\lambda)|^2d\mu(\lambda)$.
%
\item[(2)]
%
The transformation $\mathbf{T}_1(\mathbf{A})$, with domain
$\{g_j\}$, associated with the matrix $\mathbf{A}$ by the set
$\{g_j\}$ is essentially self--adjoint, i.e.,
$\mathbf{T}_1^*(\mathbf{A})=\mathbf{T}_1^{**}(\mathbf{A})$.
\end{itemize}
\end{theorem}
The converse is also true (see Theorem 7.12 page 278 of
\cite{Stone:64}). By a more judicious choice of the element
$\vec{e}_k$ and the set $\{g_j\}$ the matrix $\mathbf{A}$ may be reduced
to Jacobi form \cite{Stone:64}. 
%
\begin{theorem}[Stone]\label{thm:Jacobi_Matrix_Rep_M}
  In Theorem \ref{thm:Matrix_Rep_M}, an element
  $\vec{f}\in\mathscr{H}_\times$, in lieu of $\vec{e}_k$, and the complete
  orthonormal set $\{g_j\}$ may be so chosen that the matrix
  $\mathbf{A}(g)$ associated with $\mathbf{M}$ by $\{g_j\}$ enjoys, in
  addition to the properties enumerated in Theorem
  \ref{thm:Matrix_Rep_M}, the special property that $G_l(\lambda)$ is a
  polynomial of degree $l$ for $l=0,1,2,\ldots,$. In consequence, the matrix
  $\mathbf{A}=\{a_{jl}\}$ is in Jacobi form
  \eqref{eq:Infinite_Jacobi_Mn}, where $\alpha_i\in\mathbb{R}$ and $\beta_i\neq0$.  
%
\end{theorem}
%

For the proof of Theorem \ref{thm:Jacobi_Matrix_Rep_M} see Theorem
7.13 page 282 in \cite{Stone:64}. The converse is also true (see   
Theorem 7.14 page 286 in \cite{Stone:64}). It is important to note
that since, in Theorem \ref{thm:Jacobi_Matrix_Rep_M}, the element
$\vec{f}\neq\vec{e}_k$ the measure of orthogonality for the polynomials
$G_i(\lambda)$ is $\vartheta(\lambda):=\|E_\lambda\vec{f}\|^2\neq\|E_\lambda\vec{e}_k\|^2:=\mu(\lambda)$. Where
$\text{supp}(\vartheta)\subseteq[0,1]$, therefore $0<\alpha_i<1$ and $0<\beta_i\leq1$, $i=1,2,\ldots,$
\cite{Gautschi:2004:OP}. This abstract theory provide the means for
solving important problems in the theory of continued fractions and
the theory of moments. Although these results have important
mathematical significance, the development of and the physical
significance of these allied topics are beyond the scope of this
thesis.
%
%
%
%
\section{Geometric Resonances and Residues of Fractal
  Animals} \label{sec:Fractal_Animals} 
%
\subsection{The Existence of a Spectral Gap for Finite Lattice
  Systems} \label{sec:Spectral_Gap}
%
\section{Calculation of the Spectral Measure for Composite
  Microstructures}
\label{sec:Calc_Spec_Meas_Comp_Micro}
%
\begin{itemize}
\item Display plots of eigenvalue densities and spectral
  functions of the 2-d and 3-d RRN with fractal animal resonances
  displayed as vertical lines here.
\item Also Display plots of the largest and smallest eigenvalue
  distribution and the average gap as a function of $p$. How does the
  distribution of largest and smallest eigenvalues compare to the
  Tracy-Widom distribution? Are there q-deformed results?
\item Also display eigenvalue densities and spectral functions
  associated with the following microstructures: pack ice, melt ponds,
  brine channels, trabecular bone\ldots  
\end{itemize}
%
\section{Herglotz Representations of the System Energy}
\label{sec:Herglotz_Energy_Reps}
%
In sections \ref{sec:Background_TACM} and \ref{sec:Resolv_Rep_E_D} we
introduced the effective permittivity $\epsilon^*$ and briefly discussed
its origins, properties, and connection to Herglotz functions
involving the spectral measure $\mu$. By definition, the effective
permittivity is the permittivity of a homogeneous dielectric medium
that stores the same amount of energy as the associated two component
system under consideration \cite{MILTON:2002:TC}. This definition
provides a natural way of modeling binary dielectric media using
statistical mechanics, as the system is modeled entirely by energetic 
contributions. A key feature of this, and related, Herglotz
representations of the system energy is the parameter separation
property. As will be shown in this thesis, this property has far
reaching consequences in the statistical mechanical description of
phase transitions in binary composite media. In this section we 
further explore the properties of these representations.

In section \ref{subsec:Spec_Decomp_Energy} we derive a detailed
decomposition of the system energy for general two component
stationary random media in the lattice and continuum settings, in terms
of Herglotz functions involving $\mu$. In section
\ref{sec:Extremal_Geometries} we give a detailed analysis of some important
extremal geometries in the continuum setting, and derive Herglotz
representations of the corresponding system energy. In section
\ref{subsec:Hierarchial_RRNs} we derive Herglotz representations of
the system energy for the corresponding extremal geometries
(topologies) in the lattice setting, and extend this representation to
hierarchical random resistor networks. 
%
\subsection{A General Spectral Decomposition of the System Energy}
\label{subsec:Spec_Decomp_Energy}
%
In this section we extend the utility of the mathematical and physical
properties of the effective permittivity by constructing a detailed
decomposition of the system energy, holding for general two component
stationary random media in the lattice and continuum settings. In
section \ref{sec:Resolv_Rep_E_D} we showed that the variational 
problems \eqref{eq:Weak_Curl_Free_Variational_Form} and
\eqref{eq:Weak_Div_Free_Variational_Form} provide reduced
representations of the system energy in terms of $\epsilon^*$ and
$(\epsilon^*)^{-1}$. In particular, we showed that
%
\begin{align}\label{eq:Energy_Constraint_E_D}
  \langle\vec{D}\cdot\vec{E}_f\rangle=\langle\vec{E}\cdot\vec{D}_f\rangle=0,
\end{align}
%
where $\vec{E}_f:=\vec{E}-\vec{E}_0$ and
$\vec{D}_f:=\vec{D}-\vec{D}_0$ are the fluctuations about the mean
electric field, $\langle\vec{E}\rangle:=\vec{E}_0$, and mean displacement field, 
$\langle\vec{D}\rangle:=\vec{D}_0$, respectively, so that $\langle\vec{E}_f\rangle=\langle\vec{D}_f\rangle=0$, where
$\langle\cdot\rangle$ denotes ensemble average over $\Omega$ or volume average over all
$\mathbb{R}^d$, and $d$ is the dimension of the system. A direct
consequence of these variational problems is that the system energy
per-unit-volume (puv) has the following reduced representations
\cite{Jackson-1999}  
%
\begin{align}\label{eq:Reduced_Energy_E_D}
  \frac{1}{2}\langle\vec{D}\cdot\vec{E}\rangle=\frac{1}{2}\langle\vec{D}\cdot\vec{E}_0\rangle
  =\frac{1}{2}\langle\vec{D}_0\cdot\vec{E}\rangle.
\end{align}
%

By the symmetries discussed in sections \ref{sec:Background_TACM} and
\ref{sec:Resolv_Rep_E_D} we, with out loss of generality, focus on one
diagonal coefficient of the effective permittivity
$\epsilon^*:=\beps^*_{kk}$ and the representation
$\langle\vec{D}\cdot\vec{E}_0\rangle:=\beps^*\vec{E}_0\cdot\vec{E}_0:=\epsilon^*E_0^2$, with
$\vec{E}_0=E_0\vec{e}_k$, $\vec{e}_k$ the $k$th standard basis vector
in $\mathbb{R}^d$. Denote by $\mathbf{R}_s=(s+\Gamma\chi_1)^{-1}$ the
resolvent of the operator $-\Gamma\chi_1$, 
which is self adjoint in the $L^2(\Omega,P)$ inner product weighted by
$\chi_1$ \cite{Golden:CMP-473}. Hence $\mathbf{R}_s$ is also self adjoint
with respect to this inner--product for $s\in\mathbb{C}\backslash[0,1]$
\cite{Stone:64}. The energy constraint
\eqref{eq:Energy_Constraint_E_D} yields 
% 
\begin{align}
 0=\langle\vec{D}\cdot\vec{E}_f\rangle&=\langle\epsilon_2(1-\chi_1/s)\vec{E}\cdot\vec{E}_f\rangle
 =\langle\epsilon_2(1-\chi_1/s)(\vec{E}_0\cdot\vec{E}_f+E_f^2)\rangle\\
 &=\epsilon_2\left(\langle E_f^2\rangle- \left(\langle\chi_1\vec{E}_0\cdot\vec{E}_f\rangle
     + \langle\chi_1E_f^2\rangle\right)/s\right).\notag
\end{align}
%
Therefore, by The Spectral Theorem \ref{thm:Spectral_Theorem}  
%
\begin{align}\label{eq:seed}
  \langle E_f^2\rangle&=\frac{E_0^2}{s}\left( \langle\chi_1(s\mathbf{R}_s\vec{e}_k\cdot\vec{e}_k-1)\rangle
    +\langle\chi_1\|(s\mathbf{R}_s-1)\vec{e}_k\|^2\rangle   \right) \\
    &= \frac{E_0^2}{s} \left(
      \int_0^1\left[ \frac{s}{s-\lambda}-1 \right] d\mu(\lambda)
      + \int_0^1\left[ \frac{s}{s-\lambda}-1 \right]^2 d\mu(\lambda)
      \right)\notag\\
    &=\frac{E_0^2}{s}\int_0^1 \frac{\lambda(s-\lambda)+\lambda^2}{(s-\lambda)^2} d\mu(\lambda)\notag\\
    &=E_0^2\int_0^1 \frac{\lambda d\mu(\lambda)}{(s-\lambda)^2}:=-E_0^2\frac{\partial F_1(s)}{\partial s}, \notag
\end{align}
%
where the Stieltjes transformation, $F_i(s)$, of the measure $\lambda^i\mu$ is
defined in equation \eqref{Eq:Energy_Basis_F}. Equation
\eqref{eq:seed} is a general formula holding for two component
stationary random media in the lattice and continuum settings
\cite{Golden:PRL-3935}.

The main theorem of this section follows immediately
from formula \eqref{eq:seed} and The Spectral Theorem
\ref{thm:Spectral_Theorem}, which shows that the $F_j(s)$, $j=0,1,2$ and
their first derivatives serve as basis functions for a detailed
decomposition of the system energy in terms of Herglotz functions
involving $\mu$, where $F_0(s):=F(s)$. 

% 
\begin{theorem}\label{thm:Herglotz_Decomp_Energy}
   Denote by $F_i(s)$, the Stieltjes transformation of the measure
   $\lambda^i\mu(d\lambda)$ (see equation \eqref{Eq:Energy_Basis_F}). Then the
   following are general formula holding for two component stationary  
   random media in the lattice and continuum settings: 
%
  \begin{align}\label{eq:Ed2_Energy_Partitions_Ef2}
  \frac{\langle\chi_1E_f^2\rangle}{E_0^2}=-\frac{\partial F_2(s)}{\partial s},   \quad
  \frac{\langle E_f^2\rangle}{E_0^2}=-\frac{\partial F_1(s)}{\partial s}, \quad
  % \frac{\langle\chi_2E_f^2\rangle}{E_0^2}=\frac{\partial F_2(s)}{\partial s}-\frac{\partial F_1(s)}{\partial s}
\end{align}  
%
  \begin{align}\label{eq:Ed2_Energy_Partitions_Ef*E0}
  \frac{ \langle\chi_1\vec{E}_f\cdot\vec{E}_0\rangle}{E_0^2}=F_1(s), \quad
  \frac{ \langle\vec{E}_f\cdot\vec{E}_0\rangle}{E_0^2}=0, \quad
 % \frac{ \langle\chi_2\vec{E}_f\cdot\vec{E}_0\rangle}{E_0^2}=-F_1(s),
  \end{align}
%
  \begin{align}\label{eq:Ed2_Energy_Partitions_E2}
   \frac{\langle\chi_1E^2\rangle}{E_0^2}=-s^2\frac{\partial F(s)}{\partial s}, \quad
   \frac{ \langle E^2\rangle}{E_0^2}=1-\frac{\partial F_1(s)}{\partial s}, \quad
  % \frac{ \langle\chi_2E^2\rangle}{E_0^2}=1-\frac{\partial F_1(s)}{\partial s}+s^2\frac{\partial F(s)}{\partial s},
  \end{align}
%
  \begin{align}\label{eq:Ed2_Energy_Partitions_E*E0}
    \frac{ \langle\chi_1\vec{E}\cdot\vec{E}_0\rangle}{E_0^2}=sF(s),\quad
    \frac{\langle\vec{E}\cdot\vec{E}_0\rangle}{E_0^2}=1,\quad
   % \frac{ \langle\chi_2\vec{E}\cdot\vec{E}_0\rangle}{E_0^2}=1-sF(s),
  \end{align}
%
  \begin{align}\label{eq:Ed2_Energy_Partitions_E*Ef}
    \frac{ \langle\chi_1\vec{E}\cdot\vec{E}_f\rangle}{E_0^2}=-s^2\frac{\partial F(s)}{\partial s}-sF(s),\quad
    \frac{\langle\vec{E}\cdot\vec{E}_f\rangle}{E_0^2}=-\frac{\partial F_1(s)}{\partial s},\quad
   % \frac{ \langle\chi_2\vec{E}\cdot\vec{E}_f\rangle}{E_0^2}=(s-1)\frac{\partial F_1(s)}{\partial s}.
  \end{align}
%
  where the formulas involving $\chi_2$ are given by the relation $\chi_2=1-\chi_1$.
\end{theorem}
%
\begin{proof}
  Using equation \eqref{eq:seed} and The
  Spectral Theorem \ref{thm:Spectral_Theorem} we have
%  
  \begin{align*}
  \frac{\langle\chi_1E_f^2\rangle}{E_0^2}&= \langle\chi_1(s\mathbf{R}_s-1)^2\vec{e}_k\cdot\vec{e}_k\rangle
             =\int_0^1 \left(\frac{s}{s-\lambda} -1\right)^2 d\mu(\lambda)
             =\int_0^1 \frac{\lambda^2d\mu(\lambda)}{(s-\lambda)^2}=-\frac{\partial F_2(s)}{\partial s}. \\
  \frac{\langle E_f^2\rangle}{E_0^2}&=  \langle\|(s\mathbf{R}_s-1)\vec{e}_k\|^2 \rangle
                  =\int_0^1\frac{\lambda d\mu(\lambda)}{(s-\lambda)^2}=-\frac{\partial F_1(s)}{\partial s}.\notag 
 % \langle\chi_2E_f^2\rangle/E_0^2&=  \langle(1-\chi_1)E_f^2 /E_0^2\rangle =\int_0^1\frac{\lambda(1-\lambda)d\mu(\lambda)}{(s-\lambda)^2}
\end{align*}
%
\begin{align*}
  \frac{\langle\chi_1\vec{E}_f\cdot\vec{E}_0\rangle}{E_0^2}&=
     \langle\chi_1(s\mathbf{R}_s-1)\vec{e}_k\cdot\vec{e}_k \rangle
     =\int_0^1\frac{\lambda d\mu(\lambda)}{s-\lambda}=F_1(s).\\
  \frac{\langle\vec{E}_f\cdot\vec{E}_0\rangle}{E_0^2}&= 0. \notag
 % \langle\chi_2\vec{E}_f\cdot\vec{E}_0\rangle/E_0^2&=- \langle\chi_1\vec{E}_f\cdot\vec{E}_0/E_0^2\rangle\notag 
\end{align*}
%
\begin{align*}
  \frac{\langle\chi_1E^2\rangle}{E_0^2}&= \langle\chi_1s^2R^2_s\vec{e}_k\cdot\vec{e}_k \rangle
           =s^2\int_0^1\frac{d\mu(\lambda)}{(s-\lambda)^2}=-s^2\frac{\partial F(s)}{\partial s}.\\
 \frac{\langle E^2\rangle}{E_0^2}&=  \frac{\langle(E_0^2+2\vec{E}_f\cdot\vec{E}_0+E_f^2)\rangle}{E_0^2} 
  =1+\int_0^1\frac{\lambda d\mu(\lambda)}{(s-\lambda)^2}=1-\frac{\partial F_1(s)}{\partial s}.\notag
 % \langle\chi_2E^2\rangle/E_0^2&=  \langle(1-\chi_1)E^2/E_0^2 \rangle
 % =1+\int_0^1\frac{\lambda d\mu(\lambda)}{(s-\lambda)^2}-s^2\int_0^1\frac{d\mu(\lambda)}{(s-\lambda)^2}\notag 
\end{align*}
%
\begin{align*}
  \frac{\langle\chi_1\vec{E}\cdot\vec{E}_0\rangle}{E_0^2}&=
      \langle\langle\chi_1s\mathbf{R}_s\vec{e}_k\cdot\vec{e}_k\rangle
       =s\int_0^1\frac{d\mu(\lambda)}{s-\lambda}=sF(s).\\
  \frac{\langle\vec{E}\cdot\vec{E}_0\rangle}{E_0^2}&= 1.\notag 
 % \langle\chi_2\vec{E}\cdot\vec{E}_0 /E_0^2\rangle&=
 %     \langle(1-\chi_1)\vec{E}\cdot\vec{E}_0\rangle/E_0^2
 % =1-s\int_0^1\frac{d\mu(\lambda)}{s-\lambda}\notag 
\end{align*}
%
\begin{align*}
  \frac{\langle\chi_1\vec{E}\cdot\vec{E}_f\rangle}{E_0^2}&=\frac{\langle\chi_1E^2\rangle-\langle\chi_1\vec{E}\cdot\vec{E}_0\rangle}{E_0^2}        
        =-s^2\frac{\partial F(s)}{\partial s}-sF(s).\\
  \frac{\langle\vec{E}\cdot\vec{E}_f\rangle}{E_0^2}&=\frac{\langle E^2\rangle-\langle\vec{E}\cdot\vec{E}_0\rangle}{E_0^2}
     =\int_0^1\frac{\lambda d\mu(\lambda)}{(s-\lambda)^2}=-\frac{\partial F_1(s)}{\partial s}.
 % \langle\chi_2\vec{E}\cdot\vec{E}_f\rangle/E_0^2&=\langle(1-\chi_2)\vec{E}\cdot\vec{E}_f\rangle/E_0^2
 % =(1-s)\int_0^1\frac{\lambda d\mu(\lambda)}{(s-\lambda)^2}.\notag 
\end{align*}
%
These formula hold for general two component stationary random media in
the lattice and continuum settings \cite{Golden:PRL-3935}.
\end{proof}
%
All the possible variants of energy per--unit--volume (puv) may be
easily found using Theorem \ref{thm:Herglotz_Decomp_Energy}; for
example $\langle\epsilon_1\chi_1\vec{E}\cdot\vec{E}_f\rangle=-\epsilon_1E_0^2(s^2F^\prime(s)+sF(s))$. If we
denote Theorem \ref{thm:Herglotz_Decomp_Energy} ``The 
$(\vec{E},F(s),\mu)$ Energy Theorem'', then, by the symmetries discussed in
sections \ref{sec:Background_TACM} and \ref{sec:Resolv_Rep_E_D}, the analogous
$(\vec{E},G(t),\alpha)$, $(\vec{D},E(s),\eta)$, and $(\vec{D},H(t),\tau)$ energy
theorems also hold.
%
\subsection{Extremal Geometries}
\label{sec:Extremal_Geometries}
%
This section is devoted to an analysis of some important extremal
geometries of binary composites. Extremal geometries are central to the
theory of binary composites as they are the building blocks of all
such composites \cite{MILTON:2002:TC}. We will see that, in these
special cases, the energy constraint, 
$\langle\vec{D}\cdot\vec{E}\rangle=\langle\vec{D}\cdot\vec{E}_0\rangle$, is a direct consequence
of the condition $\langle\vec{E}\rangle=\vec{E}_0$ and boundary conditions, where
in this section $\langle\cdot\rangle$ denotes volume average over all
$\mathbb{R}^d$. As the results of this section are valid for
multi--component media, we will use the notation $\langle\chi_i\rangle=p_i$ for the
volume fraction of material component $i$, where $\chi_i$ is the
characteristic function of material component $i$, $i=1,2,\cdots,n$. By the
symmetries discussed in sections \ref{sec:Background_TACM} and 
\ref{sec:Resolv_Rep_E_D} we, with out loss of generality, focus on the
representation $F(s)$, $s=1/(1-h)$, $h=\epsilon_1/\epsilon_2$. To simplify the
analysis in this section we assume that the constituents are linear
and ideal (perfect electrical insulators \cite{Reitz-1993}) so that
the bound charge distribution $\rho_b$ is directly proportional to
the free charge density $\rho_f$, $\rho_b\propto\rho_f=\vec{\nabla}\cdot\vec{D}=0$
\cite{Robertson-1993}. Therefore, the interior of each constituent may
be thought of as free space partitioned by the boundaries of the
constituents. By linearity of the material, we have the following
identities 
%
\begin{align}
  \label{eq:fieldRelations}
  \epsilon_0\vec{E}=\vec{D}-\vec{P}_0\equiv\epsilon\vec{E}-(\epsilon-\epsilon_0)\vec{E}\iff
  \epsilon_2\vec{E}=\vec{D}-\vec{P}_2\equiv\epsilon\vec{E}-(\epsilon-\epsilon_2)\vec{E}.
\end{align}
%
Therefore $\epsilon_2$ can be used in place of $\epsilon_0$ in the definition of the
total charge \cite{Jackson-1999,Griffiths-1999}
$\epsilon_2\vec{\nabla}\cdot\vec{E}=\rho_T=\rho_f+\rho_b$ without physical nor mathematical
inconsistencies, given the assumptions made about the system.   
% 
% 
%
\begin{figure}[h!]
\begin{center}
%\subfigure[Laminates parallel to the applied field]{
\includegraphics[scale=0.4]{parallelLaminants.eps}%}
%
\qquad
%\subfigure[Laminates perpendicular to the applied field]{
\includegraphics[scale=0.32]{perpendicularLaminants.eps}%}
\caption{\label{fig:Laminates} Extremal Geometries of the
  Effective Permittivity}
\end{center}
\end{figure}

%%%%%%%%%%%%%%%%%%%%%%%%%%%%%%%%%%%%%%%%%%%%%%%%%%%%%%%%%%%%%%%%%%%%%%%
\noindent\textbf{Homogeneous systems}\newline
%%%%%%%%%%%%%%%%%%%%%%%%%%%%%%%%%%%%%%%%%%%%%%%%%%%%%%%%%%%%%%%%%%%%%%%
%
Consider an infinite ideal homogeneous dielectric system of
permittivity $\epsilon\equiv\epsilon_2$ immersed in a uniform electric field. The
dielectric is assumed ideal, therefore there 
are no free nor bound charge densities, $\rho_f=\rho_b\equiv0$. The condition
$\langle\vec{E}\rangle=\vec{E}_0$ and symmetry implies
$\vec{E}\equiv\vec{E}_0$. Therefore the associated system energy
per--unit--volume is given by 
$\frac{1}{2}\langle\vec{D}\cdot\vec{E}\rangle\equiv\frac{1}{2}\epsilon_2E_0^2$. This implies that
there are no internal--internal nor external--internal interactions in
the system. This is a direct consequence of the condition
$\langle\vec{E}\rangle=\vec{E}_0$ and, from a statistical mechanics point of view,
requires that energy associated with boundary interactions be
considered part of the internal energy of the system (see section
\ref{sec:Information_Theory} below for details concerning the internal
energy of the system). This is equivalent to placing the energy of the
surface integral terms, which are usually thrown away in the
derivation of electric energy densities \cite{Griffiths-1999}, as a
part of the internal energy.     

%%%%%%%%%%%%%%%%%%%%%%%%%%%%%%%%%%%%%%%%%%%%%%%%%%%%%%%%%%%%%%%%%%%%%%%
\vspace{0.1in}\noindent\textbf{Laminates Parallel to the Applied Field}\newline
%%%%%%%%%%%%%%%%%%%%%%%%%%%%%%%%%%%%%%%%%%%%%%%%%%%%%%%%%%%%%%%%%%%%%%%
%
Consider an infinite ideal binary dielectric system immersed in a
uniform electric field, with laminate geometry parallel to the applied
field (see figure \ref{fig:Laminates}a). The dielectric constituents
are assumed ideal, therefore $\rho_f=\rho_b\equiv0$. This system is known
\cite{Scaife-1989} to have an effective permittivity
$\epsilon^*=p_1\epsilon_1+p_2\epsilon_2$, with $F(s)=1-\epsilon^*/\epsilon_2=p_1/s$. The formula for
$\epsilon^*$ is not limited to binary composite media, we therefore use the
following definitions $\epsilon:=\sum_{j=1}^n\chi_j\epsilon_j$ and
$\epsilon^*_\parallel:=\sum_{j=1}^np_j\epsilon_j$. The total electric field is curl free,
$\vec{\nabla}\times\vec{E}=0$, which causes the tangential component of the
electric field to be continuous across the contrast boundaries
\cite{Jackson-1999}, $(\vec{E}_i-\vec{E}_{i+1})\times\vec{n}=0$, where 
$\vec{n}$ is the unit normal to the contrast boundaries and
$\vec{E}(\vec{x})=\vec{E}_i$ when $\epsilon(\vec{x})=\epsilon_i$. This and
symmetry implies $\vec{E}_i=\vec{E}_{i+1}$ for all $i$, therefore
$\vec{E}_i=\vec{E}_j$ for all $i,j$. By focusing on $i=1$, the
condition $\langle\vec{E}\rangle=\vec{E}_0$ implies  
%
\begin{align*}
  \vec{E}_0=\left\langle\sum_{j=1}^n\chi_j\vec{E}_j\right\rangle
             =\vec{E}_1\sum_{j=1}^np_j=\vec{E}_1.
\end{align*}
%
Therefore, no surface charge densities are induced on the dielectric
contrast boundaries, $\sigma_{i,i+1}:=\epsilon_2(\vec{E}_i-\vec{E}_{i+1})\cdot\vec{n}=0$
\cite{Jackson-1999}. The identification $\vec{E}\equiv\vec{E}_0$ and
equation \eqref{eq:Resolvent_Rep_Es},
$\vec{E}=s\mathbf{R}_s\vec{E}_0$, implies $\langle\mathbf{R}_s\rangle \to\mathbf{I}_d/s$ in the 
infinite volume limit where $\mathbf{I}_d$ is the identity operator
in $\mathbf{R}^d$.

The identification $\vec{E}\equiv\vec{E}_0$ and
$\vec{D}=\epsilon\vec{E}$ yields the system energy per--unit--volume
%
\begin{align}
  \left\langle\frac{1}{2}\vec{D}\cdot\vec{E}\right\rangle
             &=\left\langle\frac{1}{2}\left(\sum_{j=1}^n\chi_j\epsilon_j\right)E_0^2\right\rangle
             =\frac{1}{2}\epsilon^*_\parallel E_0^2
             =\frac{1}{2}\epsilon_2E_0^2\left(1-\frac{p_1}{s}\right) 
\end{align}
%
in accordance with the general theory, where the last equality holds
only for two component composite media. As in the homogeneous
dielectric system, the condition $\langle\vec{E}\rangle=\vec{E}_0$ requires that
energy associated with boundary interactions be considered part of the
internal energy of the system.

%%%%%%%%%%%%%%%%%%%%%%%%%%%%%%%%%%%%%%%%%%%%%%%%%%%%%%%%%%%%%%%%%%%%%%%
\vspace{0.1in}\noindent\textbf{Laminates Perpendicular to the Applied
  Field}\newline 
%%%%%%%%%%%%%%%%%%%%%%%%%%%%%%%%%%%%%%%%%%%%%%%%%%%%%%%%%%%%%%%%%%%%%%%
%
Consider an infinite ideal binary dielectric system immersed in a
uniform electric field, in the direction $\vec{e}_k$, with laminate
geometry perpendicular to the applied field (see figure
\ref{fig:Laminates}b). The dielectric constituents are assumed ideal,
therefore $\rho_f=\rho_b\equiv0$. This system is known \cite{Scaife-1989} to
have effective permittivity $\epsilon^*=(p_1/\epsilon_1+p_2/\epsilon_2)^{-1}$, with 
$F(s)=1-\epsilon^*/\epsilon_2=p_1/(s-p_2)$. As with the parallel laminate geometry,
the results here are not limited to binary composites. We therefore
use the definitions $\epsilon:=\sum_{j=1}^n \chi_j\epsilon_j$ and  $(\epsilon^*_\perp)^{-1}:=\sum_{j=1}^n p_j/\epsilon_j$. 

The absence of free charges within the system causes the displacement
field to be divergence free $\vec{\nabla}\cdot\vec{D}=\rho_f\equiv0$. This in turn
causes the normal component of the displacement field to be continuous
\cite{Jackson-1999}, $(\epsilon_i\vec{E}_i-\epsilon_{i+1}\vec{E}_{i+1})\cdot\vec{n}=0$
for all $i$, where $\vec{n}:=\vec{e}_k$ is the unit normal to the
contrast boundaries and $\vec{E}(\vec{x})=\vec{E}_i$ when
$\epsilon(\vec{x})=\epsilon_i$, i.e., $\vec{E}=\sum_{j=1}^n\chi_j\vec{E}_j$. This and the
symmetry of the system, $\vec{E}_j=E_j\vec{e}_k$ for $j=0,1,\ldots,n$,
implies $\epsilon_i\vec{E}_i=\epsilon_{i+1}\vec{E}_{i+1}$ for all $i$, thus  
$\epsilon_i\vec{E}_i=\epsilon_{j}\vec{E}_{j}$ for all $i,j$. By focusing on $\vec{E}_1$, we
have that $\epsilon_i\vec{E}_i=\epsilon_1\vec{E}_1$ for all $i$. The condition
$\langle\vec{E}\rangle=\vec{E}_0$ then implies  
%
\begin{align}
  \label{eq:perpendicularInvarianceE}
  \vec{E}_0=\sum_{j=1}^np_j\vec{E}_j
   =\epsilon_1\vec{E}_1\sum_{j=1}^n\frac{p_j}{\epsilon_j}
    =\frac{\epsilon_1}{\epsilon^*_\perp}\vec{E}_1,
\end{align}
so that $\vec{E}_0=E_0\vec{e}_k$, which imposes a global continuity
equation for the displacement field:  
%
\begin{align}
  \label{eq:GlobalContinuityD}
  \epsilon_j\vec{E}_j=\epsilon^*_\perp\vec{E}_0, \quad j=1,2,\ldots,n.
\end{align}
%
By equation \eqref{eq:GlobalContinuityD} the orthonormality of the
$\chi_i$, $\chi_i\chi_j=\delta_{ij}\chi_i$, and the symmetry of the system,
$\vec{E}_j=E_j\vec{e}_k$ for $j=0,1,\ldots,n$, the system energy per--unit--volume is
given by   
%
\begin{align}\label{eq:Perp_Lam_Energy}
  \frac{1}{2}\langle\vec{D}\cdot\vec{E}\rangle&= \frac{1}{2}\langle\epsilon\vec{E}\cdot\vec{E}\rangle
             = \frac{1}{2}\left\langle
               \left(\sum_{j=1}^n\chi_j\epsilon_j\right)\left(\sum_{j=1}^n\chi_j\vec{E}_j\right)\cdot
               \left(\sum_{j=1}^n\chi_j\vec{E}_j\right)
                           \right\rangle  \\
             &=\frac{1}{2}\sum_{j=1}^np_j\epsilon_jE_j^2
             =\frac{1}{2}\sum_{j=1}^np_j\epsilon_j\left(\frac{\epsilon^*_\perp E_0}{\epsilon_j}\right)^2
             \notag\\
             &=\frac{1}{2}\epsilon^*_\perp E_0^2
             =\frac{1}{2}\epsilon_2E_0^2\left(1-\frac{p_1}{s-p_2}\right)\notag
\end{align}
%
in accordance with the general theory, where the last equality holds
only for two component composite media. By equation
\eqref{eq:GlobalContinuityD} the surface charge densities induced on
the dielectric contrast boundaries are given by 
%
\begin{align}\label{eq:Charge_Density}
  \sigma_{i,i+1}:&=\epsilon_2(\vec{E}_i-\vec{E}_{i+1})\cdot\vec{n}
       =E_0\epsilon_2\epsilon^*_\perp\left(\frac{1}{\epsilon_i}-\frac{1}{\epsilon_{i+1}}\right)\\
       &=\pm E_0\epsilon_2\epsilon^*_\perp\left(\frac{1}{\epsilon_1}-\frac{1}{\epsilon_2}\right)
       =\pm E_0\epsilon^*_\perp\frac{1-h}{h}=\pm\frac{E_0\epsilon^*_\perp}{s-1},\notag
\end{align}
%
where the equalities in the second line of equation
\eqref{eq:Charge_Density} hold only for two component 
composite media, and we have used $h=\epsilon_1/\epsilon_2=(s-1)/s$.

The following theorem illustrates that, for this special geometry, the
energy constraint $\langle\vec{D}\cdot\vec{E}_f\rangle=0$, the condition
$\langle\vec{E}\rangle=\vec{E}_0$, and equation \eqref{eq:GlobalContinuityD} are
equivalent statements.    

\begin{theorem}\label{thm:parallel_Laminate_Equivalence}
%
Consider an infinite ideal $n$-component dielectric system immersed
in a uniform electric field, in the direction $\vec{e}_k$, with laminate 
geometry perpendicular to the applied field. Denote $\epsilon:=\sum_{j=1}^n \chi_j\epsilon_j$,
$\epsilon^*_\parallel:=\sum_{j=1}^n\epsilon_jp_j$ and $(\epsilon^*_\perp)^{-1}:=\sum_{j=1}^n p_j/\epsilon_j$, where
$p_j=\langle\chi_j\rangle$ is the volume fraction of material component $j$,
$j=1,2,\cdots,n$, and $\chi_j$ is the corresponding characteristic
function. Then the boundary condition $\epsilon_jE_j=\epsilon_1E_1$ for $j=1,2,\cdots,n$,
the electric field definition $\vec{E}:=\vec{E}_f+\vec{E}_0$, the
symmetry of the problem $\vec{E}_j=E_j\vec{e}_k$ for $j=0,1,\ldots,n$, and
the orthonormality of the $\chi_i$, $\chi_i\chi_j=\delta_{ij}\chi_i$, imply that the
following statements are equivalent  
%
\begin{align*}
  \text{(1): } &\langle\vec{E}_f\rangle=0 \\
  \text{(2): } &\epsilon_j\vec{E}_j=\epsilon^*_\perp\vec{E}_0 \text{ for } j=1,2,\ldots,n \\
  \text{(3): } &\langle\vec{D}\cdot\vec{E}_f\rangle=0,
\end{align*}
%
where $\vec{E}(\vec{x})=\vec{E}_i$ when $\epsilon(\vec{x})=\epsilon_i$, i.e.,
$\vec{E}=\sum_{j=1}^n\chi_j\vec{E}_j$.
%
\end{theorem}
%
\begin{proof}
%
The property (1)$\Rightarrow$(2) has already been established by equation
\eqref{eq:perpendicularInvarianceE} and the boundary condition
$\epsilon_jE_j=\epsilon_1E_1$ for $j=1,2,\ldots,n$. Conversely (1)$\Leftarrow$(2), if 
$\epsilon_j\vec{E}_j=\epsilon^*_\perp\vec{E}_0$ for $j=1,2,\ldots,n$ then
$\langle\vec{E}_f\rangle:=\langle\vec{E}\rangle-\vec{E}_0=0$. Indeed, 
%
\begin{align}
  \langle\vec{E}_f\rangle
  =\left\langle\sum_{j=1}^n\chi_j\vec{E}_j\right\rangle-\vec{E}_0
  =\vec{E}_0\left(\epsilon^*_\perp\sum_{j=1}^n\frac{p_j}{\epsilon_j}-1\right)=0.\notag
\end{align}
%
The property (2)$\Rightarrow$(3) may be proved using the electric field
definition, $\vec{E}:=\vec{E}_f+\vec{E}_0$, the symmetry of the
problem, $\vec{E}_j=E_j\vec{e}_k$ for $j=0,1,\ldots,n$, and the
orthornormality of the $\chi_i$, $\chi_i\chi_j=\delta_{ij}\chi_i$. Indeed, if
$\epsilon_j\vec{E}_j=\epsilon^*_\perp\vec{E}_0$ for $j=1,2,\ldots,n$ then   
%
\begin{align}\label{eq:equivalence_2to3}
  \langle\vec{D}\cdot\vec{E}_f\rangle&= \langle\epsilon\vec{E}\cdot\vec{E}_f\rangle=
  \left\langle\left(\sum_{j=1}^n\chi_j\epsilon_j\right)\left(\sum_{j=1}^n\chi_j\vec{E}_j\right)\cdot
    \left(\sum_{j=1}^n\chi_j\vec{E}_j-\vec{E}_0\right)\right\rangle \notag\\
  &=\left\langle
    \left(\sum_{j=1}^n\epsilon_j\chi_j\right)\left(\sum_{j=1}^n\chi_j(E_j^2-E_jE_0)\right)
    \right\rangle\notag\\
  &=\left\langle\sum_{j=1}^n\chi_j(\epsilon_jE_j^2-\epsilon_jE_jE_0)\right\rangle\\
  &=\sum_{j=1}^np_j\left(
     \epsilon_j\left(\frac{\epsilon^*_\perp}{\epsilon_j}\right)^2-\epsilon_j\frac{\epsilon^*_\perp}{\epsilon_j}
           \right)E_0^2=0,\notag
\end{align}
%
where we have used $\sum_{j=1}^np_j=1$ in the last line. Conversely (2)$\Leftarrow$(3), the
electric field definition, the symmetry of the problem, and the
orthornormality of the $\chi_i$ yields equation
\eqref{eq:equivalence_2to3}. Therefore if $\langle\vec{D}\cdot\vec{E}_f\rangle=0$ then
the boundary condition, $\epsilon_jE_j=\epsilon_1E_1$ for $j=1,2,\ldots,n$, yields
%
\begin{align*}
  0&= \langle\vec{D}\cdot\vec{E}_f\rangle=\sum_{j=1}^np_j(\epsilon_jE_j^2-\epsilon_jE_jE_0)\\
   &=\sum_{j=1}^np_j\left(
             \epsilon_j\left(\frac{\epsilon_1E_1}{\epsilon_j}\right)^2-\epsilon_j\frac{\epsilon_1E_1}{\epsilon_j}E_0
            \right)\notag\\
   &=\epsilon_1E_1\left(\frac{\epsilon_1E_1}{\epsilon^*_\perp}-E_0\right)\notag.        
\end{align*}
%
This and the boundary condition then imply
$\epsilon_j\vec{E}_j=\epsilon^*_\perp\vec{E}_0$ for $j=1,2,\ldots,n$.
%
\end{proof}
%

The following theorem is the analogue of Theorem
\ref{thm:Herglotz_Decomp_Energy} for an infinite ideal $n$-component 
dielectric system immersed in a uniform electric field with laminate
geometry perpendicular to the applied field. 
%
\begin{theorem}
%  
Consider an infinite ideal $n$-component dielectric system immersed
in a uniform electric field with laminate geometry perpendicular to
the applied field. Denote $\epsilon:=\sum_{j=1}^n \chi_j\epsilon_j$,
$\epsilon^*_\parallel:=\sum_{j=1}^n\epsilon_jp_j$ and $(\epsilon^*_\perp)^{-1}:=\sum_{j=1}^n p_j/\epsilon_j$, where
$p_j=\langle\chi_j\rangle$ is the volume fraction of material component $j$,
$j=1,2,\cdots,n$, and $\chi_j$ is the corresponding characteristic
function. Furthermore denote the total electric field by
$\vec{E}=\vec{E}_0+\vec{E}_f$, where $\langle\vec{E}\rangle:=\vec{E}_0$ and
$\vec{E}(\vec{x})=\vec{E}_i$ when $\epsilon(\vec{x})=\epsilon_i$, i.e.,
$\vec{E}=\sum_{j=1}^n\chi_j\vec{E}_j$. Then 
\begin{align*}
  \langle\epsilon\vec{E}\cdot\vec{E}\rangle&=\epsilon^*_\perp E_0^2\\
  \langle\epsilon\vec{E}\cdot\vec{E}_0\rangle&=\epsilon^*_\perp E_0^2\\
  \langle\epsilon\vec{E}\cdot\vec{E}_f\rangle&=0\\
  \langle\epsilon\vec{E}_f\cdot\vec{E}_0\rangle&=(\epsilon^*_\perp-\epsilon^*_\parallel)E_0^2\\
  \langle\epsilon\vec{E}_f\cdot\vec{E}_f\rangle&=(\epsilon^*_\parallel-\epsilon^*_\perp)E_0^2\\
\end{align*}
%
\end{theorem}
%
\begin{proof}  
   The property $\langle\epsilon\vec{E}\cdot\vec{E}\rangle=\epsilon^*_\perp E_0^2$ has already been
   established in equation \eqref{eq:Perp_Lam_Energy}. By Theorem
   \ref{thm:parallel_Laminate_Equivalence} $\langle\epsilon\vec{E}\cdot\vec{E}_f\rangle=0$,
   therefore $\vec{E}=\vec{E}_0+\vec{E}_f$ implies
   $\langle\epsilon\vec{E}\cdot\vec{E}_0\rangle=\epsilon^*_\perp E_0^2$. Moreover,  
%  
  \begin{align*}
     \langle\epsilon\vec{E}_f\cdot\vec{E}_0\rangle&=\left\langle\left(\sum_{j=1}^n\chi_j\epsilon_j\right)
                           \left(\sum_{j=1}^n\chi_j\vec{E}_j-\vec{E}_0\right)\cdot\vec{E}_0
                         \right\rangle
                     =\left\langle\sum_{j=1}^n\chi_j\epsilon_j(E_j-E_0)E_0\right\rangle\\
                     &=\sum_{j=1}^np_j\epsilon_j\left(\frac{\epsilon^*_\perp}{\epsilon_j}-1\right)E_0^2
                     =(\epsilon^*_\perp-\epsilon^*_\parallel)E_0^2,
   \end{align*}
%
   where we have used $\sum_{j=1}^np_j=1$. By $ \langle\epsilon\vec{E}_f\cdot\vec{E}\rangle=0$
   and $\vec{E}=\vec{E}_0+\vec{E}_f$ we therefore have
   $\langle\epsilon\vec{E}_f\cdot\vec{E}_f\rangle=(\epsilon^*_\parallel-\epsilon^*_\perp)E_0^2$. 
\end{proof}
%
%
\subsection{Hierarchical Random Resistor Networks}
\label{subsec:Hierarchial_RRNs}
%
%%%%%%%%%%%%%%%%%%%%%%%%%%%%%%%%%%%%%%%%%%%%%%%%%%%%%%%%%%%%%%%%%%%%%%%%%%%%
%%%%%%%%%%%%%%%%%%%%%%%%%%%%%%%%%%%%%%%%%%%%%%%%%%%%%%%%%%%%%%%%%%%%%%%%%%%%%
\chapter{STATISTICAL MECHANICS OF BINARY COMPOSITES} 
\label{ch:Stat_Mech_Bin_Comps}
%
%%%%%%%%%%%%%%%%%%%%%%%%%%%%%%%%%%%%%%%%%%%%%%%%%%%%%%%%%%%%%%%%%%%%%%%%%%
%%%%%%%%%%%%%%%%%%%%%%%%%%%%%%%%%%%%%%%%%%%%%%%%%%%%%%%%%%%%%%%%%%%%%%%%%%
%
%\vspace{0.2in}
\noindent
\textbf{Thermally/Electrically Driven Phase Transitions}
%
%%%%%%%%%%%%%%%%%%%%%%%%%%%%%%%%%%%%%%%%%%%%%%%%%%%%%%%%%%%%%%%%%%%%%%%%%%
%%%%%%%%%%%%%%%%%%%%%%%%%%%%%%%%%%%%%%%%%%%%%%%%%%%%%%%%%%%%%%%%%%%%%%%%%%

Thermodynamics was originally a self contained theory of heat and
work, based firmly on experimental evidence. Statistical mechanics was
subsequently developed to provide a micro-physical foundation for this
empirical subject. Thermodynamics was founded as a science by
R. Clausius when he gave a kinematic description of thermodynamical
systems \cite{Berdichevsky-1997}. He postulated that every 
thermodynamical system may be characterized by some generalized
coordinates (or parameters), which may vary over a physically
realizable range. For instance, a solid body experiencing homogeneous
deformation may be characterized by the six independent components of
the (symmetric) deformation tensor $(e_1,e_2,\cdots,e_6)$
\cite{Berdichevsky-1997,Robertson-1993}. We will denote generalized 
coordinates by $y=(y_1,y_2,\ldots,y_n)$. A thermodynamic system is also
characterized by the empirical temperature $T$. The
$(n+1)$--dimensional state space of points with coordinates
$(y_1,y_2,\ldots,y_n,T)$ describes the thermodynamic state of the system
\cite{Berdichevsky-1997}. 

Clausius formulated two statements, which are now commonly known as
the first and second laws of thermodynamics \cite{Berdichevsky-1997}. The
first law postulates that every infinitesimal thermodynamic process may
be characterized by a work function which is a differential form
of the $\{dy_1,dy_2,\ldots,dy_n\}$, $\delta W:=\sum_{j=1}^nA_idy_i$, and a heat supply
$\delta Q$ which is a differential form of the $\{dy_1,dy_2,\ldots,dy_n,dT\}$, and that
the sum is the differential form of some function of state
$U(y_1,y_2,\ldots,y_n,T)$ called the internal energy
\cite{Berdichevsky-1997,Thompson-1988,Bobbio-2000,Robertson-1993},
%
\begin{align}\label{eq:Clasius_First_Law}
  dU=\delta W+\delta Q:=\sum_{j=1}^nA_jdy_j+TdS.
\end{align}
%
The term $TdS$ is given by the second law, which Clausius derived by 
arguing: for any cycle, the following equation holds
\cite{Berdichevsky-1997} 
%
\begin{align}\label{eq:Clasius_Entropy}
  \oint\frac{\delta Q}{T}=0 \iff \delta Q=TdS.
\end{align}
%

The existence of the state function $S(y_1,y_2,\ldots,y_n)$, which Clausius
called the entropy, is equivalent to the vanishing of the integral in
equation \eqref{eq:Clasius_Entropy} \cite{Berdichevsky-1997}. He called $T$
the absolute temperature. Neither $\delta Q$ nor $\delta W$ are exact differentials
\cite{Thompson-1988,Bobbio-2000,Berdichevsky-1997}. The symbols
$\delta Q$ and $\delta W$ are used to indicate the linear differential forms or
pfaffins of these functions \cite{Bobbio-2000}. Although, $dU$ as well
as $dS$ are exact \cite{Bobbio-2000}. More generally, the existence of
the state functions $T\geq0$ and $S$, related by $\delta Q=TdS$, may be 
established by means of a theorem on canonical presentation of
differential forms (\cite{Berdichevsky-1997} and references therein).

By the first and second laws of thermodynamics
\eqref{eq:Clasius_First_Law}--\eqref{eq:Clasius_Entropy} we have   
%
\begin{align}\label{eq:Constitutive_eqs}
  \frac{1}{T}=\frac{\partial S(U,y)}{\partial U}, \quad
  A_j=-T\;\frac{\partial S(U,y)}{\partial y_j}, \quad j=1,2,\ldots,n.
\end{align}
%
The two laws of thermodynamics are thus reduced to the statement that
there exists and entropy function, $S(U,y)$, such that the absolute
temperature, $T$, and generalized forces, $\{A_j\}_{j=1}^n$, are expressed in terms
of the entropy by the constitutive equations
\eqref{eq:Constitutive_eqs} \cite{Berdichevsky-1997}. The term
``entropy'' has since been generalized and used in science in many
different senses, and has been used in many areas of mathematics and
physics, including dynamical systems, random matrix theory, topology,
and information theory
\cite{Martin-1981,Abul-Magd:j-PLA:2007,Sethna-2006}. The ideas of
information theory and information entropy form a solid foundation for
statistical thermophysics \cite{Robertson-1993}. 

\section{Information Theory and The Canonical Ensemble}
\label{sec:Information_Theory} 
%
In statistical physics one is faced with the task of assigning
probabilities to events associated with complex many body systems,
based on a few significant bits of information. In practice, this
information is far from sufficient to obtain objective nor unique
probabilities. In order to develop a theory that describes macroscopic 
properties of a system, based on underlying microscopic properties
which are not precisely known, it is common to use a maximum entropy
principle. The prediction of macroscopic behavior
based on insufficient or incomplete data is part of information theory. 
%Here we extended this framework
%to systems where thermal fluctuations are not of primary importance.

An entropy function $S$ is a measure of the amount of uncertainty in a
statistical model \cite{Robertson-1993}. The idea behind entropy is
that one is not entitled to assume more knowledge, less uncertainty,
than that given by subsidiary conditions such as average values and
unity measure of the probability space. Any assignment of 
probabilities that satisfy these conditions but yield a value of $S$
other than its maximum is unjustified on the basis of known
data. Therefore, the common attitude is to use probability
measures which maximize entropy, thereby maximizing the
uncertainty of a system, subject to known information. 

The analysis done by Shannon [1948] provides a remarkably clear
quantitative measure of the uncertainty inherent in a set of
probabilities $\{f_\omega\}$ \cite{Robertson-1993,Balian:NCB:471}. In this
analysis he derived the following expression expression widely known
as Shannon's \emph{information entropy}
%
\begin{align} \label{eq:ShannonEntropy}
  S[\{f_\omega\}]=-k\sum_\omega f_\omega\ln{f_\omega}.
\end{align}
%
where $f_\omega$ is the probability of event $\omega$ and $k$ is an arbitrary
positive constant which sets the units of $S$. One can show that the
entropy is a strictly concave function \cite{Sethna-2006} with a
global minimum, $S=0$, attained when $f(\omega)=1$ for some $\omega\in\Omega$ (no
uncertainty) \cite{Robertson-1993}, and with a global maximum attained when
$f(\omega)=f(\omega^\prime)$ for all $\omega,\omega^\prime\in\Omega$ (no
information)\cite{Sethna-2006,Firas}. Therefore the entropy is
inherently positive $S\geq0$.    

A common method for maximizing functions with given constraints is the
method of Lagrange multipliers. Of course one always has the
constraint $\sum_\omega f_\omega=1$. When only the average of some quantity is
known, $U:=\langle U_\omega\rangle=\sum_\omega f_\omega U_\omega$, the resultant probability 
distribution is called the \emph{canonical ensemble}
\cite{Robertson-1993}.  The canonical ensemble is found by maximizing 
$S/k-\alpha\sum_\omega f_\omega-\beta\sum_\omega f_\omega U_\omega$ over probability distributions 
$\{f_\omega\}$, where $\alpha$ and $\beta$ are Lagrange multipliers. Regarding the
$\{f_\omega\}$ as independent variables, one arrives at at the following
probability distribution
\cite{Robertson-1993,Balian:NCB:471,Firas}     
%
\begin{align}	\label{eq:Cannonical_Ensemble}
	f_\omega = Z^{-1}\exp{(-\beta U_\omega)}, \quad
	Z  = \sum_\omega \exp(-\beta U_\omega).
\end{align}
%
The distribution, $\{f_\omega\}$, and it's normalization, $Z:= \exp(\alpha+1)$,
are widely known as the Gibbs--Boltzmann distribution and the
partition function respectively. The exponential nature of 
the canonical ensemble allows averages to be calculated via the
\emph{pressure function} $\ln{Z}=\alpha+1$; for example
\cite{Robertson-1993,Chandler-1987} 
%
\begin{align}
  U=\langle U_\omega\rangle=-\frac{\partial\ln{Z}}{\partial\beta}, \quad
  \text{Var}(U_\omega)=\langle U_\omega^2\rangle-\langle U_\omega\rangle^2=\frac{\partial^2\ln{Z}}{\partial\beta^2}.
\end{align}
%

The function $\Fc:=-\beta^{-1}\ln{Z}$ is widely known as the
Helmholtz free energy, or simply free energy. Using the free energy,
one may write the entropy in the following form \cite{Robertson-1993}   
% 
\begin{align}
  \label{eq:EnergyConservation}
  U=ST+\Fc,   
      %=-\beta^{-1}\sum_\omega f_\omega[-\beta U_\omega-\ln{Z}]
      %=\left. \frac{\partial(\beta \Fc)}{\partial\beta}\right|_{U}-\Fc,
      \quad T:=(k\beta)^{-1},
\end{align}
%
where, as we will see in equation \eqref{eq:FirstLaw} below, the
function $U$ is identified with the internal energy
\eqref{eq:Clasius_First_Law}. Equation \eqref{eq:EnergyConservation}
shows that the internal energy and free energy are Legendre
transformations of each other (see section
\ref{sec:LegendreTransformations} below). In statistical thermophysics
the absolute temperature is defined by $T=(k\beta)^{-1}$
\cite{Thompson-1988}. Under the information theoretic approach to
statistical mechanics the (universal \cite{Firas}) constant $k$ is
arbitrary. If one sets $k=1$ then $T$ has units of energy and $\beta$ is
the inverse temperature. If one sets $k$ to Boltzmann's constant then
$T$ has units of Kelvin. In statistical thermophysics the quantity $Q$
defines processes such as heat transfer and/or radiation
\cite{Bobbio-2000}. Although, in the information theoretic framework,
$Q$ encompasses all energetic processes in which information, $-S$, is
lost.

Quantum theory, indeed, identifies the $\{U_\omega\}$ as energy states of Hamiltonian
systems and the $\{f_\omega\}$ as the corresponding equilibrium distribution
\cite{Robertson-1993}. This identification has been generalized to
Hamiltonian systems with a continuum of energy states. The macroscopic
energy is given by the system Hamiltonian $\Hc$ and the
corresponding partition function is given by
$Z=\int_{\omega\in\Omega}P(d\omega)\exp{(-\beta\Hc(\omega))}$, where $\omega\in\Omega$ is the space of
all statistical configurations, $P(d\omega)$ is the reference measure of
the system when $\beta=0$, and $Z^{-1}P(d\omega)\exp{(-\beta\Hc(\omega))}$ is
the equilibrium probability (Gibbs) measure. To simplify notation we
will continue to use that of a discrete probability space as its
generalization is now clear.  

The relation \eqref{eq:EnergyConservation} was obtained without making
any assumptions regarding the nature of the system and is therefore a
fundamental relation of the information theoretic approach to
statistical mechanics. It is a statement of conservation of energy and
is therefore a constraint imposed on the system \cite{Bobbio-2000}. In
order to see this we look at the differential form of this equation
\cite{Robertson-1993}   
%
\begin{align}\label{eq:FirstLaw} 
 dU=TdS-\sum_\omega f_\omega dU_\omega:=\delta Q+\delta W,
\end{align}
%
which recovers the first law of Thermodynamics
\eqref{eq:Clasius_First_Law} and identifies $U=\langle U_\omega\rangle$ with the
internal energy. The term $\delta W=-\sum_\omega f_\omega dU_\omega$ represents the
differential of work done by the surroundings on the 
system, changing the characteristic energy states $\{U_\omega\}$
\cite{Robertson-1993}. Conservation of energy in Hamiltonian systems
then identifies $-\delta W$ as the work done by the system on the
surroundings \cite{Thompson-1988,Baker-1990}. Various work terms may
be identified by expanding the differential $dU_\omega$ in state variables,
$\{y_j\}_{j=1}^n$, which determine the characteristic levels
$\{U_\omega\}$, and by subsequently examining the physical relevance of the
generalized forces $A_j:=\langle\partial U_\omega/\partial y_j\rangle$ \cite{Robertson-1993}.
%
\begin{align}\label{eq:Generalized_First_Law}
  dU=TdS-\sum_{j=1}^nA_jdy_j, \quad  A_j:=\left\langle\frac{\partial U_\omega}{\partial y_j}\right\rangle.
\end{align}
%
Equations \eqref{eq:FirstLaw}--\eqref{eq:Generalized_First_Law}
recover the constitutive equations \eqref{eq:Constitutive_eqs}.
Equation \eqref{eq:Generalized_First_Law} gives the differential of
the internal energy in terms of the state functions $T$ and
$\{A_j\}_{j=1}^n$, and state variables $S$ and $\{y_j\}_{j=1}^n$:
$U=U(S,y)$, $T=T(S,y)$, and $A_j=A_j(S,y)$. A simple calculation,
using equations \eqref{eq:EnergyConservation} and
\eqref{eq:Generalized_First_Law}, shows that the differential of the
free energy is given in terms of the state functions $S$ and
$\{A_j\}_{j=1}^n$, and state variables $T$ and $\{y_j\}_{j=1}^n$:
$\Fc=\Fc(T,y)$, $S=S(T,y)$, $A_j=A_j(T,y)$
\cite{Robertson-1993}  
%
\begin{align}\label{eq:Helmholtz_first_law}
  d\Fc=-SdT-\sum_{j=1}^nA_jdy_j, \quad
  S=-\frac{\partial\Fc}{\partial T} \quad
  A_j=-\frac{\partial\Fc}{\partial y_j}.
\end{align}
%
Details regarding Legendre transformations are postponed until section
\ref{sec:LegendreTransformations} below.    

In the derivation of the first law \eqref{eq:FirstLaw} no assumptions
were made about the nature of the system nor the evolution to equilibrium. 
Therefore, it is valid for reversible, irreversible, quasi--static,
and even non--quasi--static evolutions during which the thermodynamic
state cannot be defined at all \cite{Bobbio-2000}. This is important when
one is studying systems with electromagnetic processes which are
generally irreversible \cite{Bobbio-2000}. For a detailed
discussion of thermodynamic state, reversibility and other related
concepts see \cite{Bobbio-2000,Thompson-1988,Robertson-1993}. 

If other subsidiary conditions are known, say the averages
$\langle f_n\rangle=c_n$ of functions $f_n(U_\omega)$ (again one always has the
constraint $f_0=1$, $c_0=1$), the Gibbs-Boltzmann distribution becomes
\cite{Robertson-1993,Balian:NCB:471} 
%
\begin{align}	\label{eq:General_Cannonical_Ensemble}
f_\omega=Z^{-1}\exp\left(-\sum_n\beta_n f_n(U_\omega)\right),\quad
Z=\sum_\omega\exp\left(-\sum_n\beta_n f_n(U_\omega)\right),
\end{align}
%
and the resultant value of $S$ is either reduced or left unchanged
\cite{Robertson-1993}. From equation
\eqref{eq:General_Cannonical_Ensemble} one has the analogue of
equation \eqref{eq:EnergyConservation} $S/k=\ln{Z}+\sum_n\beta_n\langle f_n\rangle$ showing
that the entropy and the pressure function are Legendre transforms of
one another \cite{Robertson-1993}. If we regard $S$ as a function of
the $\{\langle f_n\rangle\}$, we have the generalized constitutive equations
%
\begin{align}\label{eq:Gen_Constitutive_eqs}
  \beta_n=\frac{\partial(S/k)}{\partial\langle f_n\rangle}, \quad \langle f_n\rangle=-\frac{\partial\ln{Z}}{\partial\beta_n}
\end{align}
%
giving the $\{\beta_n\}$ in terms of the $\{c_n\}$
\cite{Robertson-1993}. The Helmholtz free energy contains the same
amount of information as the entropy. Indeed, every result which can
be calculated from one, can be calculated from the other. Moreover,
the symmetric matrices defined by
%
\begin{align}
  [\mathbf{B}_1]_{n,m}:=\frac{\partial^2(S/k)}{\partial\langle f_n\rangle\partial\langle f_m\rangle}, \quad
  [\mathbf{B}_2]_{l,j}:=-\frac{\partial^2\ln{Z}}{\partial\beta_l\partial\beta_j}, \quad
  \mathbf{B}_1=\mathbf{B}_2^{-1}
\end{align}
are inverses of one another \cite{Robertson-1993}. 

The second law is a \emph{maximum entropy principle}. It states
that the entropy $S$ will increase to a maximum value at equilibrium for
isolated systems (fixed total energy and mass) with fixed external
state variables (volume, electric field, etc.) and internal energy
$U$ \cite{Bobbio-2000,Robertson-1993,Thompson-1988}. Unconstrained
state variables, such as temperature, evolve to the equilibrium values
as the entropy becomes maximum \cite{Callen-1985}. The \emph{minimum
  internal energy principle}  is essentially a restatement of the
second law \cite{Callen-1985}. It states that the internal energy will
decrease to a minimum value at equilibrium for closed systems (only
energetic transfers) with fixed external state variables and entropy $S$.

Mathematically, the second law states that if $y_i$ is an
unconstrained variable of state which varies as a system approaches
equilibrium, then at equilibrium 
%
\begin{align}
  \left.\frac{\partial S}{\partial y_i}\right|_U=0, \quad
  \left.\frac{\partial^2S}{\partial y_i^2}\right|_U<0.
\end{align}
%
Although, from the properties of an exact differential
\cite{Robertson-1993}, Legendre transformations \cite{Bobbio-2000},
and the first law \eqref{eq:FirstLaw} we have that, at equilibrium,
%
\begin{align}
  \left.\frac{\partial U}{\partial y_i}\right|_S=-T\left.\frac{\partial S}{\partial y_i}\right|_U=0,
  \quad
  \left.\frac{\partial^2U}{\partial y_i^2}\right|_S=-T\left.\frac{\partial^2S}{\partial y_i^2}\right|_U>0,
\end{align}
%
showing that the internal energy is in fact at a minimum. Therefore,
we have shown that (\cite{Callen-1985} chapter 5)
%
\begin{align}\label{eq:min_U_Principle}
  U_0(S_0)=\inf_{\tilde{y}}U(S_0,\tilde{y})
\end{align}
%
where the minimization is with respect to the unconstrained
variables $\tilde{y}$. As the system approaches equilibrium, the
unconstrained variables take their equilibrium values and the internal
energy $U_0$ is a function only of the entropy $S_0$ \cite{Callen-1985}.     

Using the maximum entropy principle and the equivalent minimum
internal energy principle, one may argue the 
\emph{minimum Helmholtz free energy principle}. It 
states that, for closed systems with fixed external state variables and
temperature, the Helmholtz free energy is minimized at equilibrium
with respect to \emph{any unconstrained internal variables}. To see
this let $\tilde{y}$ be the set of unconstrained internal variables. By
definition of the Helmholtz free energy and the maximum entropy
principle $\Fc(T,\tilde{y})=\sup_S(U(S,\tilde{y})-TS)$. The
maximum occurs when the variable $T$ becomes the equilibrium
temperature, $T_0$, since $T=(\partial U/\partial S)_{\tilde{y}}$. By the minimum
internal energy principle \eqref{eq:min_U_Principle}, at equilibrium,
the Helmholtz free energy will be  
%
\begin{align}\label{eq:Min_Free_energy}
  \Fc_0(T_0)=\sup_{S_0}(U_0(S_0)-T_0S_0)
          =\sup_{S_0}(\inf_{\tilde{y}}(U(S_0,\tilde{y}))-T_0S_0)\\
          =\inf_{\tilde{y}}(\sup_{S_0}(U(S_0,\tilde{y})-T_0S_0))
          =\inf_{\tilde{y}}(\Fc_0(T_0,\tilde{y})),\notag
\end{align}
%
where we have assumed that the order of the extrema can be exchanged,
showing that the Helmholtz free energy is minimized at equilibrium.  
These physical arguments \cite{Callen-1985} can be made rigorous using 
concepts of measure theory, free entropy, specifications, and
large deviation theory under Gibbs measures \cite{Firas} (see section
\ref{subsec:LDT_Gibbs} below for details).

%%%%%%%%%%%%%%%%%%%%%%%%%%%%%%%%%%%%%%%%%%%%%%%%%%%%%%%%%%%%%%%%%%%%%%%%%%%%%%%
%
\section{Thermodynamic Potentials and Maxwell's Relations}
\label{sec:LegendreTransformations}
%
%%%%%%%%%%%%%%%%%%%%%%%%%%%%%%%%%%%%%%%%%%%%%%%%%%%%%%%%%%%%%%%%%%%%%%%%%%%%%%
%
In order to use the methods of statistical mechanics to uniquely
determine the macroscopic behaviors of a system, one must assume a
set of constitutive relations which define the state variables and the
state functions \cite{Bobbio-2000}. It is typically assumed that the
functions of state are invertible, at least locally. Therefore, the
functions of state and state variables may change roles.

The change of variables is accomplished through Legendre
transformations \cite{Bobbio-2000,Baker-1990,Robertson-1993}. Through
these transformations, various thermodynamic potentials determine the
corresponding state functions through constitutive equations like
those in equations \eqref{eq:Generalized_First_Law} and
\eqref{eq:Helmholtz_first_law}. Depending on which variables are 
chosen as state variables, one may use different thermodynamic
potentials that make calculations of certain functions of state much
easier as, in the new coordinate system, the state functions are
simply derivatives of the corresponding thermodynamic potential with
respect to the conjugate state variables \cite{Callen-1985}.

As a concrete example we now derive equation
\eqref{eq:Helmholtz_first_law}. By equations
\eqref{eq:EnergyConservation}, \eqref{eq:FirstLaw}, and
\eqref{eq:Generalized_First_Law}, the internal energy, $U=U(S,y)$, is
a function of the state variables $(y_1,\ldots,y_n)$ and the entropy $S$, with total
differential $dU=TdS-\sum_{j=1}^nA_jdy_j$. The free energy
$\Fc=\Fc(T,y)$, with total differential
$d\Fc=-SdT-\sum_{j=1}^nA_jdy_j$ is given by the Legendre transformation
$\Fc(T,y)=U-TS$. Indeed 
% 
\begin{align*}
  d\Fc=dU-TdS-SdT=TdS-\sum_{j=1}^nA_jdy_j-TdS-SdT=-SdT-\sum_{j=1}^nA_jdy_j.
\end{align*}
%
In this way we may define many ``thermodynamic potentials'' $U$,
$\Fc$, etc.

Often a thermodynamic system is completely
described by the absolute temperature $T$ and two state variables
$y_1$ and $y_2$, say, with conjugate state functions $A_1$ and
$A_2$. For example if the system of interest is a variable volume
vessel filled with a gas, and separated by a semi-permeable membrane 
\cite{Thompson-1988,Robertson-1993}, then $A_1:=P$ is the external  
pressure on the movable vessel wall, $y_1:=V$ is the volume of the
vessel, $A_2:=\mu$ is the chemical potential, and $y_2:=N$ is the number
of gas molecules to the left of the membrane, say. From section
\ref{sec:Background_TACM} we found that, in polarized binary
dielectric media, the natural state variables are the average electric
field $E_0:=y_1$ and the dielectric contrast parameter $h:=y_2$. The natural
conjugate state functions are the effective polarization $P^*:=A_1$
and the ``contrast potential'' $\Psi:=A_2$. As this is the primary system
of interest in this work, we will henceforth use this notation. We
will see in section \ref{sec:StatMech_of_Composites} below that the
parameter separation property of the system energy
\eqref{eq:Reduced_Energy}--\eqref{eq:Fs_Integral} gives an especially 
convenient representation of $P^*$. The contrast potential, $\Psi$, is
subsequently defined using Maxwell's relations.
 
Maxwell's relations are given by equating commuted mixed partial
derivatives of the various thermodynamic potentials, with respect to
the associated state variables. Although, we will see that the mixed
partial derivatives of the thermodynamic potentials do not commute in
general. We will also show that there are important consequences when
this commuting property fails to hold. The following formulas
summarize the Maxwell's relations for polarized binary dielectric media.  
%
\begin{align}\label{eq:Helmholtz_Free_Energy_Maxwell's_Relations}
  \Fc=U-TS, \qquad &d\Fc=-SdT-P^*dE_0-\Psi ds\\
  \quad -\frac{\partial^2\Fc}{\partial E_0\partial T}&=\frac{\partial S}{\partial E_0}
                                     =\frac{\partial P^*}{\partial T}
                                     \notag\\
  \quad -\frac{\partial^2\Fc}{\partial E_0\partial s}&=\frac{\partial\Psi}{\partial E_0}
                                     =\frac{\partial P^*}{\partial s}
                                     \notag\\
  \quad -\frac{\partial^2\Fc}{\partial T\partial s}&=\frac{\partial S}{\partial s}
                                     =\frac{\partial\Psi}{\partial T}
                                     \notag 
\end{align}  
%
%
\begin{align}\label{eq:Gibbs_Free_Energy_Maxwell's_Relations}  
  \mathcal{G}=U-TS+P^*E_0, \qquad &d\mathcal{G}=-SdT+E_0dP^*-\Psi ds \\
  \quad -\frac{\partial^2\mathcal{G}}{\partial s\partial P^*}&=-\frac{\partial E_0}{\partial s}
                           =\frac{\partial\Psi}{\partial P^*}
                           \notag\\
  \quad -\frac{\partial^2\mathcal{G}}{\partial T\partial P^*}&=-\frac{\partial E_0}{\partial T}
                           =\frac{\partial S}{\partial P^*}
                           \notag
 %  \quad -\frac{\partial^2\mathcal{G}}{\partial T\partial s}&=\frac{\partial S}{\partial s}
%                           =\frac{\partial\Psi}{\partial T}
%                           =-\frac{\partial^2\Fc}{\partial T\partial s}
%                           \notag 
\end{align}  
%
\begin{align}\label{eq:Grand_Potential_Maxwell's_Relations}  
  \Phi=U-TS+s\Psi, \qquad &d\Phi=-SdT-P^*dE_0+sd\Psi\\
  \quad -\frac{\partial^2\Phi}{\partial T\partial\Psi}&=-\frac{\partial s}{\partial T}
                                   =\frac{\partial S}{\partial\Psi},                              
                                   \notag
 %  &\quad -\frac{\partial^2\Phi}{\partial E_0\partial T}=\frac{\partial P^*}{\partial T}
%                                    =\frac{\partial S}{\partial E_0}
%                                    =-\left( =\frac{\partial^2\Fc}{\partial E_0\partial T}\right)^{-1}
%                                    \notag\\
%   &\quad -\frac{\partial^2\Phi}{\partial E_0\partial\Psi}=-\frac{\partial s}{\partial E_0}
%                                     =\frac{\partial P^*}{\partial\Psi}
%                                     =-\left(\frac{\partial^2\mathcal{G}}{\partial s\partial P^*}\right)^{-1}\notag
\end{align}  
%
% \begin{align*}  
%   d\mathcal{G}_2&=-SdT+E_0dP^*+sd\Psi\\
%   &\quad -\frac{\partial^2\mathcal{G}_2}{\partial T\partial P^*}=-\frac{\partial E_0}{\partial T}
%                             =\frac{\partial S}{\partial P^*}
%                             =-\frac{\partial^2\mathcal{G}}{\partial T\partial P^*}\\
%   &\quad -\frac{\partial^2\mathcal{G}_2}{\partial\Psi\partial T}=-\frac{\partial s}{\partial T}
%                           =\frac{\partial S}{\partial\Psi}
%                           =-\frac{\partial^2\Phi}{\partial T\partial\Psi}\\
%   &\quad \frac{\partial^2\mathcal{G}_2}{\partial P^*\partial\Psi}=\frac{\partial E_0}{\partial\Psi}
%                            =\frac{\partial s}{\partial P^*}
%                            =-\left(\frac{\partial^2\Fc}{\partial E_0\partial s}\right)^{-1}
% \end{align*}  
% %
% \begin{align*}  
%   dU&=TdS-P^*dE_0-\Psi ds\\
%   &\quad -\frac{\partial^2U}{\partial E_0\partial S}=-\frac{\partial T}{\partial E_0}
%                            =\frac{\partial P^*}{\partial S}
%                            =-\left(\frac{\partial^2\mathcal{G}}{\partial T\partial P^*}\right)^{-1}\\
%   &\quad -\frac{\partial^2U}{\partial E_0\partial s}=\frac{\partial\Psi}{\partial E_0}
%                            =\frac{\partial P^*}{\partial s}
%                            =-\frac{\partial^2\Fc}{\partial E_0\partial s}\\
%   &\quad -\frac{\partial^2U}{\partial S\partial s}=-\frac{\partial T}{\partial s}
%                            =\frac{\partial\Psi}{\partial S}
%                            =-\left(\frac{\partial^2\Phi}{\partial T\partial\Psi}\right)^{-1}
% \end{align*}
% %
% %
% \begin{align*}
%   d\mathcal{E}&=TdS+E_0dP^*-\Psi ds\\
%  &\quad \frac{\partial^2\mathcal{E}}{\partial S\partial P^*}=\frac{\partial T}{\partial P^*}
%                           =\frac{\partial E_0}{\partial S}
%                           =-\left(\frac{\partial^2\Fc}{\partial E_0\partial T}\right)^{-1}\\
%   &\quad -\frac{\partial^2\mathcal{E}}{\partial S\partial s}=-\frac{\partial T}{\partial s}
%                           =\frac{\partial\Psi}{\partial S}
%                           =-\left(\frac{\partial^2\Phi}{\partial T\partial\Psi}\right)^{-1}\\
%   &\quad -\frac{\partial^2\mathcal{E}}{\partial P^*\partial s}=-\frac{\partial E_0}{\partial s}
%                            =\frac{\partial\Psi}{\partial P^*}
%                            =-\frac{\partial^2\mathcal{G}}{\partial s\partial P^*}
% \end{align*}
% %
% \begin{align*}
%   dU_2&=TdS-P^*dE_0+sd\Psi\\
%   &\quad \frac{\partial^2U}{\partial S\partial\Psi}=\frac{\partial s}{\partial S}
%                           =\frac{\partial T}{\partial\Psi}
%                           =-\left(\frac{\partial^2\Fc}{\partial T\partial s}\right)^{-1}\\
%   &\quad -\frac{\partial^2U}{\partial E_0\partial S}=-\frac{\partial T}{\partial E_0}
%                            =\frac{\partial P^*}{\partial S}
%                            =-\left(\frac{\partial^2\mathcal{G}}{\partial T\partial P^*}\right)^{-1}\\
%   &\quad -\frac{\partial^2U}{\partial E_0\partial\Psi}=-\frac{\partial s}{\partial E_0}
%                           =\frac{\partial P^*}{\partial\Psi}
%                           =-\left(\frac{\partial^2\mathcal{G}}{\partial s\partial P^*}\right)^{-1}
% \end{align*}
% %
% \begin{align*}
%   dU_3&=TdS+E_0dP^*+sd\Psi\\
%   &\quad \frac{\partial^2U_3}{\partial S\partial P^*}=\frac{\partial E_0}{\partial S}
%                             =\frac{\partial T}{\partial P^*}
%                             =-\left(\frac{\partial^2\Fc}{\partial E_0\partial T}\right)^{-1}\\
%   &\quad \frac{\partial^2U_3}{\partial\Psi\partial S}=\frac{\partial s}{\partial S}
%                             =\frac{\partial T}{\partial\Psi}
%                             =-\left(\frac{\partial^2\Fc}{\partial T\partial s}\right)^{-1}\\
%   &\quad \frac{\partial^2U_3}{\partial P^*\partial\Psi}=\frac{\partial E_0}{\partial\Psi}
%                             =\frac{\partial s}{\partial P^*}
%                             =-\left(\frac{\partial^2\Fc}{\partial E_0\partial s}\right)^{-1},
% \end{align*}
% %
% where $\Fc$ is the Helmholtz free energy, $\mathcal{G}$ is the
% Gibbs free energy, and $\Phi$ is the Grand Potential. It is well known
% \cite{Baker-1990} that any of these thermodynamic potentials
% determines any other. For example  
% %
% \begin{align}\label{eq:Free_Energy_Relations}
%   U=\Fc+TS=\Fc-T\frac{\partial \Fc}{\partial T}\\
%   \mathcal{G}=\Fc+P^*E_0=\Fc-E_0\frac{\partial \Fc}{\partial E_0}\notag\\
%   \Fc=\mathcal{G}-P^*E_0=\mathcal{G}-P^*\frac{\partial \mathcal{G}}{\partial P^*}.\notag
% \end{align}
% %
where $\Fc$ is the Helmholtz free energy, $\mathcal{G}$ is the
Gibbs free energy, and $\Phi$ is the Grand potential
\cite{Robertson-1993}. There are eight thermodynamic potentials total,
each with three Maxwell's relations. However the identity
\cite{Robertson-1993} $\partial a/\partial b=(\partial b/\partial a)^{-1}$ makes many of these
relations redundant. It is true that any of these statistical
mechanics potentials determines any other. For example
\cite{Baker-1990}    
%
\begin{align}  \label{eq:Free_Energy_Relations}
  U&=\Fc+TS=\Fc-T\frac{\partial \Fc}{\partial T}\\
  \mathcal{G}&=\Fc+P^*E_0=\Fc-E_0\frac{\partial \Fc}{\partial E_0}\notag\\
  \Fc&=\mathcal{G}-P^*E_0=\mathcal{G}-P^*\frac{\partial \mathcal{G}}{\partial P^*}.\notag
\end{align}
%
However, equation \eqref{eq:Free_Energy_Relations} and the reported
Maxwell's relations above indicate why certain thermodynamic
potentials make calculations of certain state functions much more
convenient.  As we will see in section
\ref{sec:StatMech_of_Composites} below, Maxwell's relations provide a
great deal of information about thermally/electrically driven phase   
transitions occurring in binary dielectric systems. 
%
\section{Temperature and the Generalized Equipartition Theorem}
\label{sec:Can_Temp_Equipartition_Thm}
%
Within the canonical ensemble, the variables that naturally characterize
the macroscopic state of the system are the temperature, state
variables which describe intrinsic properties of the system such as
the particle number $N$ or the dielectric contrast parameter $h$, and state
variables which describe external work done on the system such as the
volume $V$ or the average electric field $E_0$. Energy fluctuations
are allowed since the system has been placed in direct thermal contact
with an external heat bath, or temperature reservoir, with fixed
temperature $T$. Consequently, the temperature naturally appears
in various canonical ensemble averages, whereby one can obtain useful
relations between various mechanical quantities and $T$.

The mathematical meaning of the temperature, within the canonical
ensemble, is given by The Generalized Equipartition Theorem of the
ensemble \cite{Uline:JCP:124301}. We assume that the system of interest
is a matrix/particle composite. We denote the linear momentum and
position vectors of particle $i$ by $\vec{p}_i=(p_1,\ldots,p_d)_i$ and
$\vec{r}_i=(r_1,\ldots,r_d)_i$, respectively, where $d$ is the dimension of
the system. The canonical ensemble partiton function $Z$ of this system
is given by \cite{Uline:JCP:124301} 
%
\begin{align}
  Z=\int_{\mathbb{R}^{2dN}}e^{-\beta\Hc}d\vec{p}_1\cdots d\vec{p}_Nd\vec{r}_1\cdots d\vec{r}_N,
\end{align}
%
where $\beta=1/kT$, $k$ is Boltzmann's constant, and $\Hc$ is the 
Hamiltonian of the system. We assume the Hamiltonian has the following
asymptotic behavior
%
\begin{align}\label{eq:Hamiltonian_class}
  \lim_{|\vec{p}_i|\to\infty}\exp(-\beta\Hc)=
  \lim_{|\vec{r}_i|\to\infty}\exp(-\beta\Hc)=0,
\end{align}
%
for all $i=1,\ldots,N$. Integration by parts then gives \cite{Uline:JCP:124301}
%
\begin{align}
  \left\langle p_m\frac{\partial\Hc}{\partial p_n}\right\rangle:=
   Z^{-1}\int_{\mathbb{R}^{2dN}}p_m (-kT)\frac{\partial e^{-\beta\Hc}}{\partial p_n}dp_1\cdots dp_{dN}dr_1\cdots dr_{dN}=kT\delta_{nm},
\end{align}
%
where $\delta_{nm}$ is the Kronecker delta. Similarly, using the $r_i$ in lieu
of the $p_i$, we have The Viral Theorem
$\langle r_m\frac{\partial\Hc}{\partial r_n}\rangle=kT\delta_{nm}$. Therefore  
%
\begin{align}\label{eq:equipartition_theorem}
  \langle\vec{p}_m\cdot\nabla_{\vec{p}_n}\Hc\rangle=\langle\vec{r}_m\cdot\nabla_{\vec{r}_n}\Hc\rangle=dkT\delta_{nm}
\end{align}
%
Equation \eqref{eq:equipartition_theorem} is called \emph{The
  Generalized Equipartition Theorem}, and indicates that each particle
is coupled to the external bath in an identical manner. When derived
within the canonical ensemble, the above relations are valid for any
system size (neglecting the energy of interaction between particles
within the system and particles within the heat bath)
\cite{Uline:JCP:124301}. Moreover, the explicit form of the
Hamiltonian is not needed, and the theorem is valid for any
Hamiltonian system satisfying \eqref{eq:Hamiltonian_class}. 

When one considers a more explicit form of the system Hamiltonian
%
\begin{align}\label{eq:Particle_Hamiltonian}
  \Hc=\sum_{j=1}^N\frac{|\vec{p}_j|^2}{2m_j}+U(\vec{r}_1,\ldots,\vec{r}_N),
\end{align}
%
where $m_j$ is the mass of particle $j$ and $U$ is the inter-particle
interaction potential satisfying $\lim_{|\vec{r}_j|\to\infty}U=\infty$ for all
$j=1,\ldots,N$, we obtain a well known consequence of this equivalent
coupling \cite{Uline:JCP:124301}   
%
\begin{align}\label{eq:principle_of_equipartition}
  \left\langle\frac{\vec{p}_i\cdot\vec{p}_j}{2m_j}\right\rangle=\frac{dkT}{2}\delta_{ij},
  \quad \Rightarrow \quad
  \sum_{j=1}^N\left\langle\frac{|\vec{p}_j|^2}{2m_j}\right\rangle=\frac{dNkT}{2},
\end{align}
%
which follows by direct substitution of formula
\eqref{eq:Particle_Hamiltonian} in equation
\eqref{eq:equipartition_theorem}. In other words, the average kinetic
energy of each degree of freedom is given by $dkT/2$, i.e., the
kinetic energy is equally partitioned between its degrees of
freedom. This secondary result is known as the principle of
equipartition. It is important to note that The Generalized Equipartition
Theorem \eqref{eq:equipartition_theorem} is the fundamental result
from which one determines how the translational degrees of freedom are
related to the temperature of the heat bath. The equal sharing of
kinetic energy is a secondary result that follows from taking the
kinetic energy to be a homogeneous quadratic function of the
generalized momenta, and is not necessary consequence of Hamiltonian
systems. In fact, The Generalized Equipartition Theorem
\eqref{eq:equipartition_theorem} holds in systems where the principle
of equipartition \eqref{eq:principle_of_equipartition} is no longer
valid \cite{Uline:JCP:124301}.

In sections \ref{sec:MicCanEns_Background} and
\ref{sec:MCE_Config_Temp} below we will continue our discussion on
temperature. We will find that the temperature of non-molecular
systems may also be described by a generalized equipartition theorem
using the micro--canonical ensemble. Recent generalizations of this
theorem have led to the notion of configurational temperature. The
configurational temperature may be used to characterize phase
transitions in ER fluids with sphere radii $r_s\gtrsim10\mu m$, where thermal
fluctuations are negligible.
%
\section{Asymptotic Analysis of the Phase Transitions in the Canonical Ensemble}
\label{sec:Asymp_Can_Ens_Phase_Trans}
%
In this section we explore how critical phase transitions may be
characterized by the Gibbs--Boltzmann--canonical--ensemble (GBCE)
probability measure. We show that for high temperatures, $T\gg1$
($\beta\ll1$), the system behaves according to the underlying reference
probability measure, and as the temperature vanishes, $T\to0$ ($\beta\to\infty$),
the probability distribution becomes highly localized about the
system's minimum energy state. In the canonical ensemble, a critical
phase transition is characterized by a sudden change in the behavior
of the probability distribution, largely deviating away from the
reference probability measure near a critical temperature $T_c$
towards the delta probability distribution, about the minimum energy
state.  

For simplicity, and in order to avoid mathematical details regarding
the commutability and convergence of important sums, we consider a
Hamiltonian system with a finite number of (distinct) energy states  
%
\begin{align}
  -\infty<\Hm:=\Hc_1<\Hc_2<\cdots<\Hc_M:=\HM<\infty.
\end{align}
%
The above assumptions allow us to put the underlying probability space
$\Omega$ into one--to--one correspondence with the following set of integers
$\Omega:=\{1,2,\ldots,M\}$, with $|\Omega|:=M$. The reference measure is thus the
uniform distribution over this set. The reference measure can be much
more complicated in general. For example, in ER fluids, the underlying
reference probability measure $P(d\omega)$ characterizes the modified
Poisson distribution of hard noninteracting spheres.  

When $|\beta\HM|\ll1$, we may expand $\exp(\-\beta\Hc_\omega)$ in a Taylor
series, and the partition function may be represented by
%
\begin{align}\label{eq:HighT_Z}
  Z=\sum_\omega e^{-\beta\Hc_\omega}:=|\Omega|(1+V_0(\{\Hc_\omega\})), \quad
  V_j(\{\Hc_\omega\}):= \langle\tilde{V}_j\rangle:=\sum_{n=1}^\infty\frac{(-1)^n\beta^n}{n!}\langle\Hc_\omega^{n+j}\rangle,
\end{align}
%
where $\langle\cdot\rangle$ denotes averaging with respect to the reference
probability measure. Therefore, the GBCE probability distribution is
given by 
%
\begin{align}\label{eq:HighT_fw}
  f_\omega&=|\Omega|^{-1}(1+\tilde{V}_0)(1+R(\{\Hc_\omega\})), \quad
  R(\{\Hc_\omega\}):=\sum_{j=1}^\infty[-V_0(\{\Hc_\omega\})]^j,\\
  f_\omega&\sim|\Omega|^{-1}+\beta(\Hc_\omega-\langle\Hc_\omega\rangle)
    +\frac{\beta^2}{2}[\Hc_\omega^2-\langle\Hc_\omega^2\rangle
       -2(\Hc_\omega\langle\Hc_\omega\rangle-\langle\Hc_\omega\rangle^2)]+O(\beta^3),
       \notag
\end{align}
%
where $\beta\ll1$ is chosen so that $|V_0|<1$. Equation \eqref{eq:HighT_fw}
shows that, to $O(\beta)$, the GBCE probability measure behaves like the
reference probability measure.

By equation \eqref{eq:HighT_fw} and
$\tilde{V}_n\equiv\Hc_\omega^n\tilde{V}_0$, the moments of
$\Hc_\omega$, $\langle\Hc_\omega^n\rangle_{\Hc}$, where 
$\langle\cdot\rangle_{\Hc}$ denotes the GBCE average, are given by   
%
\begin{align*}
  \langle\Hc_\omega^n\rangle_{\Hc}&=[\langle\Hc_\omega^n\rangle+V_n(\{\Hc_\omega\})]
                            [1+R(\{\Hc_\omega\})]\\
      &\sim\langle\Hc_\omega^n\rangle-\beta\text{Cov}(\Hc_\omega^n,\Hc_\omega)
          +\frac{\beta^2}{2}[\text{Cov}(\Hc_\omega^n,\Hc_\omega^2)
           -2\langle\Hc_\omega\rangle\text{Cov}(\Hc_\omega^n,\Hc_\omega)]+O(\beta^3).
%    \\  &\sim\langle\Hc_\omega^n\rangle-\beta(\langle\Hc_\omega^{n+1}\rangle
%                 -\langle\Hc_\omega\rangle\langle\Hc_\omega^n\rangle)\\
%    &+\frac{\beta^2}{2}[\langle\Hc_\omega^{n+2}\rangle-\langle\Hc_\omega^2\rangle\langle\Hc_\omega^n\rangle
%    -2(\langle\Hc_\omega^{n+1}\rangle\langle\Hc_\omega\rangle
%           -\langle\Hc_\omega\rangle^2\langle\Hc_\omega^n\rangle)]+O(\beta^3). 
\end{align*}
%
In particular, the GBCE variance of the Hamiltonian,
$\text{Var}(\Hc_\omega)_{\Hc}$, satisfies 
%
\begin{align}
  \text{Var}(\Hc_\omega)_{\Hc}\sim\text{Var}(\Hc_\omega)-
    \beta[\text{Cov}(\Hc_\omega^2,\Hc_\omega)
        -2\langle\Hc_\omega\rangle\text{Var}(\Hc_\omega)]+O(\beta^2),      
\end{align}
%
where $\text{Var}(\cdot)$ is the variance with respect to the reference
probability measure. We will see that in the opposite limit, $|\beta\Hm|\gg1$,
the GBCE variance of the Hamiltonian vanishes
exponentially. This indicates that, in this limit, the distribution
approaches the delta probability measure, about the minimum energy. By equation
\eqref{eq:HighT_Z}, the Helmholtz free energy and the entropy are
given by    
%
\begin{align}
 \Fc&=-\beta^{-1}\ln{Z}
             =-\beta^{-1}\ln|\Omega|-\beta^{-1}\ln(1+V_0)\\
             &\sim-\beta^{-1}\ln|\Omega|+\langle\Hc_\omega\rangle
                -\frac{\beta}{2}\text{Var}(\Hc_\omega)+O(\beta^2),\notag\\
 S&=(\langle\Hc_\omega\rangle_{\Hc}-\Fc)/T
   \sim k\ln|\Omega|-\frac{\beta}{2T}\text{Var}(\Hc_\omega)
   +O\left(\frac{\beta^2}{T}\right).\notag
\end{align}
%
Therefore when $|\beta\HM|\ll1$, to $O(\beta)$, the Helmholtz free energy
follows the average energy, $\langle\Hc_\omega\rangle$, and, to $O(\beta/T)$, the
entropy is at its global maximum $S=k\ln|\Omega|$.

In the opposite limit, when $|\beta\Hm|\gg1$, the partition function may be
represented by
%
\begin{align}\label{eq:LowT_Z}
  Z=\sum_\omega e^{-\beta\Hc_\omega}=e^{-\beta\Hm}\left(1+\sum_{\omega\neq1}e^{-\beta\Delta\Hc_\omega}\right)
   :=e^{-\beta\Hm}(1+Z_{\Delta}),   
\end{align}
%
where $\Delta\Hc_\omega:=\Hc_\omega-\Hm \geq0$ with equality only if
$\omega=1$. Therefore, the GBCE probability distribution is given by 
%
\begin{align}\label{eq:LowT_fw}
  f_\omega=e^{-\beta\Delta\Hc_\omega}\left(1+\sum_{j=1}^\infty(-Z_{\Delta})^j\right),
\end{align}
%
where $\beta\gg1$ is chosen so that $Z_{\Delta}\sim e^{-\beta\Delta\Hc_2}<1$. As
$\exp(-\beta\Delta\Hc_\omega)\ll1$ for all $\omega\neq1$, $Z_{\Delta}$ is exponentially
small when $|\beta\Hm|\gg1$. Therefore, $\exp(-\beta\Delta\Hc_1)\equiv1$ and
equation \eqref{eq:LowT_fw} shows that, to
$O(\exp(-\beta\Delta\Hc_2))$, the GBCE probability measure behaves like 
the delta probability measure about $\Hm$.

By equation \eqref{eq:LowT_fw} the moments of
$\Hc_\omega$, $\langle\Hc_\omega^n\rangle_{\Hc}$, are given by
%
\begin{align}
   \langle\Hc_\omega^n\rangle_{\Hc}&=\left(\Hm^n+\sum_{\omega\neq1}\Hc_\omega^ne^{-\beta\Delta\Hc_\omega}
                          \right)
                          \left(1+\sum_{j=1}^\infty(-Z_{\Delta})^j\right)\\
                          &\sim\Hm^n+(\Hc_2^n-\Hm^n)e^{-\beta\Delta\Hc_2}
                          +O\left(e^{-\beta\Delta\Hc_3}\right)\notag
\end{align}
%
In particular, the GBCE variance of the Hamiltonian,
$\text{Var}(\Hc_\omega)_{\Hc}$, satisfies
%
\begin{align}
  \text{Var}(\Hc_\omega)_{\Hc}
  \sim\Delta\Hc_2^2e^{-\beta\Delta\Hc_2}+O\left(e^{-\beta\Delta\Hc_3}\right).
\end{align}
%
This indicates that the GBCE variance of the Hamiltonian vanishes
exponentially in the limit $\beta\to\infty$, so that the GBCE probability measure
approaches the delta probability measure about the minimum energy
$\Hm$. By equation \eqref{eq:LowT_Z}, the Helmholtz free energy and
the entropy are given by
%
\begin{align}
  \Fc&=-\beta^{-1}\ln{Z}=\Hm-\beta^{-1}\ln(1+Z_{\Delta})
              =\Hm+O\left(\beta^{-1}e^{-\beta\Delta\Hc_2}\right)\\
  S&=(\langle\Hc_\omega\rangle_{\Hc}-\Fc)/T
    =O\left(\beta e^{-\beta\Delta\Hc_2}\right).\notag 
\end{align}
%
Therefore when $|\beta\Hm|\gg1$, to exponential order, the Helmholtz free
energy follows the minimum energy $\Hm$, and the entropy is at its
global minimum $S=0$. When the crossover between these two regimes,
$|\beta\HM|\ll1$ and $|\beta\Hm|\gg1$, is so dramatic that the Helmholtz free
energy looses its analytic properties at a critical temperature $T_c$,
in the infinite volume limit, the system is said to undergo a phase
transition (see figures 2.26--2.27 in \cite{Christensen-2005}
regarding the Ising model).
%
\section{Purturbation Analysis of the Canonical Ensemble}
\label{sec:Purturb_Can_Ens}
%
DO A GENERAL PURTURBATION ANALYSIS SECTION HERE, TO BE USED LATER FOR
$\mathcal{H}_\omega=E_0P^*_\omega(E_0,N,s)$
%
\section{Baker's Critical Theory of the Ising Model}
\label{sec:Magnetic_Systems}
%
In this section, for completeness, we begin by briefly discussing the
symmetries between transport in random two phase dielectric media and
transport in random two phase magnetic media. These symmetries
immediately yield integral representations for the effective
magnetic permeability, $\bmu^*$, and its inverse, $[\bmu^*]^{-1}$. The
majority of this section is devoted to an exploration of the analytic
structure of an important (simplified) statistical mechanics model of
ferro-magnetism: \emph{The Ising Model}. In section
\ref{sec:Crit_Behav_of_Transport} we will show that the critical 
behavior of transport in lattice and continuum percolation models 
mathematically parallels the percolation characteristics of the Ising
model, putting these two classes of problems on an equal footing. 

Following section \ref{sec:Background_TACM}, let $(\Omega,P)$ be a
probability space and let $\bmu(\vec{x},\omega)$ be the local permeability,
a (spatially) stationary random field in $\vec{x}\in\mathbb{R}^d$ and
$\omega\in\Omega$. We assume $\bmu(\vec{x},\omega)$ takes the values $\mu_1$ and $\mu_2$
and write $\bmu(\vec{x},\omega)=\mu_1\chi_1(\vec{x},\omega)+\mu_2\chi_2(\vec{x},\omega)$ and
$[\bmu]^{-1}(\vec{x},\omega)=\chi_1(\vec{x},\omega)/\mu_1+\chi_2(\vec{x},\omega)/\mu_2$. Let
$\vec{B}(\vec{x})$ and $\vec{H}(\vec{x})$ be the stationary random
magnetic induction and magnetic fields, related by
$\vec{B}=\bmu\vec{H}$, satisfying \cite{Golden:CMP-473}   
%
\begin{align}\label{eq:Maxwells_Equations_HB}  
	\vec{\nabla}\times\vec{H}=0, \quad
	\vec{\nabla}\cdot\vec{B}=0,\quad \quad	
	\vec{B}=\vec{B}_0+\vec{B}_f, \quad
	\langle\vec{B}\rangle=\vec{B}_0,
\end{align}
%
where $\langle\cdot\rangle$ denotes ensemble average over $\Omega$ or spatial average
over all of ${\mathbb{R}}^d$ and $\vec{B}_f$ is the fluctuating field
with mean zero about $\vec{B}_0$. We write $\vec{B}_0=B_0\vec{b}_k$,
where $\vec{b}_k$ is a unit vector, for some $k = 1, \ldots, d$. The dual
problem is similarly defined, focusing on
$\vec{H}=\vec{H}_0+\vec{H}_f$ with
$\langle\vec{H}\rangle=\vec{H}_0=H_0\vec{h}_k$. The effective complex 
permeability tensor, $\bmu^*$, and its inverse are defined as  
%
\begin{align}\label{eq:eff_mu_def}
    \langle\vec{B}\rangle=\bmu^*\langle\vec{H}\rangle, \quad 
    \langle\vec{H}\rangle=[\bmu^{-1}]^*\langle\vec{B}\rangle.
\end{align}
%
For simplicity we focus on one diagonal coefficient of the tensors
$\mu^*:=\bmu^*_{kk}$ and $[\mu^{-1}]^*:=[\bmu^{-1}]^*_{kk}$.

The mathematical symmetries,
$(\vec{E},\vec{D},\beps)\leftrightarrow(\vec{H},\vec{B},\bmu)$, 
between equations \eqref{eq:Maxwells_Equations_ED} and
\eqref{eq:Maxwells_Equations_HB} immediately yield the representations
$\mu^*=\mu_2(1-\langle\chi_1(s+\Gamma\chi_1)^{-1}\vec{h}_k\cdot\vec{h}_k\rangle)$ with integral
representation \eqref{eq:Fs_Integral},
$\mu^*=\mu_1(1-\langle\chi_2(t+\Gamma\chi_2)^{-1}\vec{h}_k\cdot\vec{h}_k\rangle)$ with integral 
representation \eqref{eq:Gt_Integral},
$[\mu^{-1}]^*=\mu_1^{-1}(1-\langle\chi_2(t-\Upsilon\chi_2)^{-1}\vec{b}_k\cdot\vec{b}_k\rangle)$ with integral  
representation \eqref{eq:Es_Integral}, and
$[\mu^{-1}]^*=\mu_2^{-1}(1-\langle\chi_1(t-\Upsilon\chi_1)^{-1}\vec{b}_k\cdot\vec{b}_k\rangle)$ with
integral representation \eqref{eq:Ht_Integral}, where
$s:=1/(1-\mu_1/\mu_2)$ and $t:=1-s$. The measures in the integral
representations, of course, depend on the respective vectors
$\vec{h}_k$ and $\vec{b}_k$ (see section \ref{sec:Resolv_Rep_E_D} for
details). 

Now consider a linear homogeneous magnetic system with fixed, or slowly
moving, boundaries so that Maxwell's equations reduce to
\cite{Griffiths-1999}
%
\begin{align}
  \vec{\nabla}\cdot\vec{B}=0, \quad  \vec{\nabla}\times\vec{H}=\vec{J}_f,
\end{align}
%
where $\vec{J}_f$ is the free current density. Here the magnetic
induction $\vec{B}$ and the magnetic field $\vec{H}$ are related by
$\vec{H}=\vec{B}/\mu_0-\vec{M}$, where $\mu_0$ is the permeability of free
space. The magnetic dipole moment density, $\vec{M}$, satisfies
$\vec{\nabla}\times\vec{M}=\vec{J}_b$, where $\vec{J}_b$ is the bound current
density. By linearity of the material we have $\vec{M}:=\chi_m\vec{H}$,
where $\chi_m$ is the magnetic susceptibility, so that
$\vec{B}=\mu_0(1+\chi_m)\vec{H}:=\mu\vec{H}$, where $\mu$ is the
magnetic permeability of the system. For such a material,
the bound current density is proportional the free current:
$\vec{J}_b=\vec{\nabla}\times(\chi_m\vec{H})=\chi_m\vec{J}_f $. For heterogeneous
anisotropic systems $\chi_m$, hence $\mu$, is a second rank tensor 
dependent on spatial coordinates. Although, for the homogeneous systems
studied in this section, both are constants multiplied by the identity
tensor and henceforth will be assumed constant
\cite{Jackson-1999,Griffiths-1999,Reitz-1993,Robertson-1993,Bobbio-2000}.  

The quantity of interest in the following analysis is the work
differential, $\delta W$, of equation \eqref{eq:FirstLaw}.  Once this is known,
the differential of the Helmholtz free energy may be found via equation
\eqref{eq:Helmholtz_first_law}, from which the functions of state
may be calculated. The system under consideration, $\Lambda$,
is the center section of a long solenoid completely filled with an
isotropic, perfectly electrical insulating, linear magnetizable
material, all in equilibrium with a heat bath at temperature $T$
\cite{Robertson-1993}. To ensure that the field produced by the
solenoid is uniform throughout, the system is restricted
longitudinally, though extends radially to the full size of the
solenoid \cite{Robertson-1993}. By the above assumptions, the material
contains no free current density and therefore contains no bound
current density, by their proportionality
\cite{Robertson-1993}. Therefore the system may be modeled as empty
space bound by a surface current density $\vec{K}_b=\vec{M}\times\vec{n}$
where $\vec{n}$ is the outward normal unit vector at the surface of
the solenoid. All interactions of the system with its surroundings are
those of the described system with the current sheet.

One way to establish a uniform field in the system is to generate a
constant current in the solenoid winding, thereby inducing a constant
magnetic field $\vec{H}$ in the system. This method, as well as an
alternative way of generating the applied field, is discussed in detail
in \cite{Robertson-1993}. The system with constant
$\vec{H}:=\vec{H}_0$ has a magnetism $\vec{M}=\chi_m\vec{H}_0$ and an
induction field $\vec{B}_0:=\mu_0\vec{H}_0$. Therefore, the total
induction field is
$\vec{B}=\mu_0(\vec{M}+\vec{H}_0)=\mu_0\vec{M}+\vec{B}_0$ and the magnetic
energy density in the system is given by
\cite{Robertson-1993,Griffiths-1999}  
%
\begin{align}\label{eq:Magnetic_energy_partition}
  \mathcal{W}:=\frac{B^2}{2\mu_0}=\frac{1}{2\mu_0}(B_0^2+2\mu_0\vec{B}_0\cdot\vec{M}+\mu_0^2M^2).
\end{align}
%

The question is: what is the proper work differential, $\delta W$, to be
inserted in the differential of the Helmholtz free energy
\eqref{eq:Helmholtz_first_law}? The $B_0^2$ term in
\eqref{eq:Magnetic_energy_partition} is the external field 
interacting with itself independent of the system, and is usually taken
to be part of the energy of the surroundings
\cite{Robertson-1993}. The third term is the magnetic dipole density
of the system interacting with itself independent of the external
field, and is usually taken to be part of the internal energy
\cite{Robertson-1993}. Therefore, the work done by the system on the
solenoid coil is either $\int_\Lambda\vec{M}\cdot
d\vec{B}_0:=\vec{\mathcal{M}}\cdot d\vec{B}_0$ or
$\int_\Lambda\vec{B}_0\cdot d\vec{M}:=\vec{B}_0\cdot d\vec{\mathcal{M}}$. An experiment,
explained in depth in \cite{Robertson-1993}, can be performed which
establishes that the term $\vec{B}_0\cdot d\vec{\mathcal{M}}$ can result
in no energy transfer from the coil to the system. Therefore this term
must be placed in the energy of the surroundings. The term
$\vec{\mathcal{M}}\cdot d\vec{B}_0$ represents work done \emph{by} the
system \emph{on} the mutual energy \cite{Robertson-1993}. Therefore,
as $\vec{B}_0$ has the effect of aligning $\vec{M}$ with $\vec{B}_0$
\cite{Thompson-1988,Jackson-1999,Griffiths-1999},
$\delta W=-\vec{\mathcal{M}}\cdot d\vec{B}_0:=-\mathcal{M}dB_0$ is the work done
\emph{on} the system \emph{by} the mutual energy
\cite{Robertson-1993}, and is to be inserted into equations
\eqref{eq:FirstLaw} and \eqref{eq:Helmholtz_first_law}. In the
remainder of this section we adhere to the notations of
\cite{Sethna-2006,Thompson-1988,Christensen-2005,Baker-1990,Firas}
and use Gaussian units \cite{Jackson-1999} $(\epsilon_0\equiv\mu_0\equiv1)$ so that the
uniform applied magnetic field
$\vec{H}:=\vec{H}_0=\vec{B}_0/\mu_0\equiv\vec{B}_0$, and we therefore use  
$\vec{H}$ in lieu of $\vec{B}_0$. In summary,
% 
\begin{align}\label{eq:Magnetic_First_Law}
  d\Fc=-SdT-\langle\mathcal{M}\rangle_{\Hc}dH, \qquad
  \langle\mathcal{M}\rangle_{\Hc}=-\frac{\partial\Fc}{\partial H},\qquad
  \chi:=\frac{\partial\langle\mathcal{M}\rangle_{\Hc}}{\partial H}=-\frac{\partial^2\Fc}{\partial H^2}\geq0,
\end{align} 
%
where $\langle\cdot\rangle_{\Hc}$ denotes the associated GBCE average. Under the
canonical ensemble, the magnetic susceptibility, $\chi$, gives the
sensitivity of the average magnetization,
$\langle\mathcal{M}\rangle_{\Hc}$, to changes in the external field, $H$,
at a fixed temperature $T$. 
%
\subsection{Magnetic Spin Systems--The Ising Model}
\label{sec:MagneticSpinSystems}
%
For magnetic systems, the microscopic objects are spins rather than
particles. This stems from the fact that magnetic monopoles do not
exist \cite{Griffiths-1999}. The simplest possible model of a spin
system is the Ising model \cite{Robertson-1993}. The Ising model has
had an enormous impact on statistical physics, and its importance
cannot be overstated \cite{Christensen-2005}. This celebrated model
predicts many aspects of the temperature driven phase transition that
occurs when a ferromagnetic material is heated above its Curie
temperature; the point where the material looses its spontaneous
ferro-magnetism and becomes paramagnetic
\cite{Griffiths-1999,Ruelle-1969,Firas,Robertson-1993,Christensen-2005}.

A general Ising lattice is a regular array of elements, each of which
can interact with an external field and with other elements in the
lattice. As a simplification, the elements (spins $\vec{s}_i$) may be
thought of as vectors occupying fixed points in space. The non-kinetic
part of the Hamiltonian is \cite{Robertson-1993}     
%
\begin{align} \label{eq:GeneralIsingStatement}
  \Hc=- \sum_j V_j(\vec{s}_j)
            -\sum_{(i,j)} J_{i,j}(\vec{s}_i,\vec{s}_j),
\end{align}
%
where the notation $(i,j)$ restricts the sum to run over all
distinct pairs of spins, $V_j(\vec{s}_j)$ is an external--internal
interaction term that may represent the effect of an external field or
its equivalent, and $J_{i,j}=J_{j,i}$ is the spin--spin interaction
energy \cite{Robertson-1993}.  The Hamiltonian
\eqref{eq:GeneralIsingStatement}, under the canonical ensemble, may be 
inferred by the maximum entropy principle subject to knowledge of the  
average field--spin interaction energy, $\langle-\sum_jV_j(\vec{s}_j)\rangle_{\Hc}$, and the average
spin--spin interaction energy,
$\langle-\sum_{(i,j)}J_{i,j}(\vec{s}_i,\vec{s}_j)\rangle_{\Hc}$. 

Consider a system, $\Lambda\subset\mathbb{Z}^d$, that consists of $N$ magnetic spins 
$\{\vec{s}_j\}_{j=1}^N$ in the presence of a uniform external magnetic field
$\vec{H}=H\vec{h_k}$, where the applied magnetic field strength $H$ is
not to be confused with $H(t)=1-\epsilon_2/\epsilon^*$
\eqref{eq:Ht_Integral}. Assuming that the magnitudes of the spins are
uniform, the spin variables may be rescaled to be unitless, with
$\|\vec{s}_i\|=1$, so that the interaction energy of the external field
with each spin is given by $-mH\vec{h}_k\cdot\vec{s}_i$, where $m$ is the
(constant) magnetic dipole moment of each spin
\cite{Griffiths-1999}. The classical Ising model simplifies things 
even further by requiring that the spins be parallel,
$\vec{h}_k\cdot\vec{s}_i:=s_i=1$, or anti--parallel,
$\vec{h}_k\cdot\vec{s}_i:=s_i=-1$, with the applied field, and only allows
for nearest neighbor spin--spin interactions; which we will henceforth
incorporate into the $(i,j)$ notation. The classic Ising model
Hamiltonian is given by
\cite{Christensen-2005,Chandler-1987,Robertson-1993,Thompson-1988}    
%
\begin{align}\label{eq:Ising_Model_Hamiltonian}
  \Hc_\omega=-HM_\omega -JG_\omega, \quad M_\omega =m\sum_js_j, \quad G_\omega=\sum_{(i,j)}G_{ij}, \quad G_{ij}=s_is_j,
\end{align}
%
where $M_\omega$ is the total magnetization of the spin
configuration $\omega\in\Omega:=\{-1,1\}^N$. The spin--spin interaction energy,
$-JG_\omega$, has been greatly simplified \cite{Griffiths-1999}, and is
given by a field independent geometric factor, $G_\omega$, multiplied by
the spin--spin interaction strength $J$. The interaction strength is
assumed to be independent of $H$ and $T$, and is positive for
ferromagnets, $J>0$.    

The magnetic susceptibility \emph{per spin}, $\chi$, gives the sensitivity of
the average magnetization \emph{per spin}, $M:=\langle M_\omega\rangle_{\Hc}/N$, to
changes in the external field at a fixed temperature, and is an example
of a response function \cite{Christensen-2005}. By direct
differentiation of the Gibbs factor,
$f_\omega:=Z^{-1}\exp(-\beta\Hc_\omega)$, one obtains 
\cite{Christensen-2005,Chandler-1987}  
%
\begin{align}
  \frac{\partial f_\omega}{\partial H}=\beta(M_\omega-M), \qquad
  \frac{\partial f_\omega}{\partial T}=\frac{\beta}{T}(\Hc_\omega-\langle\Hc_\omega\rangle_{\Hc})               
\end{align}
%
Therefore, the variance of the magnetization per spin is related to
the magnetic susceptibility per spin through
\cite{Christensen-2005,Chandler-1987} 
%
\begin{align}\label{eq:Ising_susceptibility}
  \chi(T,H):=\frac{\partial M}{\partial H}
        =-\frac{\partial^2\Fc}{\partial H^2}
        =\beta(\langle M_\omega^2\rangle_{\Hc}-M^2)
        :=\beta\text{Var}(M_\omega)_\Hc\geq0,
\end{align}
%
where $\text{Var}(\cdot)_\Hc$ denotes the GBCE variance. Similarly, the
heat capacity per spin, or specific heat, $c(T,H)$, gives 
the sensitivity of the average energy per spin, $\langle\Hc_\omega\rangle_{\Hc}/N$,
to changes in the temperature at fixed external field, and is related
to the variance of average energy through \cite{Christensen-2005}
%
\begin{align}\label{eq:Ising_specific_heat}
  c(T,H)=\frac{1}{N}\frac{\partial\langle\Hc_\omega\rangle_{\Hc}}{\partial T}
        =\frac{\beta}{NT}(\langle\Hc_\omega^2\rangle_{\Hc}-\langle\Hc_\omega\rangle^2_{\Hc})
        :=\frac{\beta}{NT}\text{Var}(\Hc_\omega)_\Hc.
\end{align}
%
The specific heat is another example of a response function. The
relationships \eqref{eq:Ising_susceptibility} and
\eqref{eq:Ising_specific_heat} between the response functions $\chi(T,H)$
and $c(T,H)$ and the variance of the total magnetization, $M_\omega$, and
energy, $\Hc_\omega$, respectively, are examples of
a general \emph{fluctuation--dissipation--theorem}
\cite{Chandler-1987}. 

The key question is: does the short--ranged nearest--neighbor
interaction among the spins in the Ising model give rise to
spontaneous magnetization, not induced by an external field? More
precisely, in the infinite volume limit \cite{Ruelle-1969,Firas}, does
there exist a critical temperature $T_c$ such that, for $T<T_c$, we
have that $\lim_{H\to0}|M(T,H)|>0$ and $\lim_{H\to0}|M(T_c,H)|=0$? To
explore this question, for now, we set the external field to zero
$H=0$.  

A competition exists between the interaction energy, $J$, tending to 
align the orientation of the spins, and the thermal energy,
$\beta^{-1}=kT$, tending to randomize the spin orientations. This 
competition is encapsulated in the minimization
\eqref{eq:Min_Free_energy} of the Helmholtz free energy
$\Fc=\langle\Hc_\omega\rangle_{\Hc}-ST$ \cite{Christensen-2005}. The entropic
contribution of the free energy is proportional to the temperature, and
the ratio of the two terms is
$\langle\Hc_\omega\rangle_{\Hc}/ST\propto J/(kT)$. Thus, for relatively high
temperatures, $J/(kT)\ll1$, the free energy is minimized by maximizing the
entropy, $S$. This is achieved by randomizing the orientation of the
spins, resulting in a paramagnetic material
\cite{Christensen-2005}. However, for relatively low temperatures,
$J/(kT)\gg1$, the free energy is minimized by minimizing the average
energy, $\langle\Hc_\omega\rangle_{\Hc}$. This is achieved by aligning
the spins, resulting in a ferromagnetic material
\cite{Christensen-2005}.  

This reasoning suggests that there may very well be a phase transition
from a disordered paramagnetic high--temperature phase to an ordered
ferromagnetic low--temperature phase \cite{Christensen-2005}. The
transition between the two regimes takes place when the magnetic and
thermal energies are comparable, $J/(kT)\sim1$. This is indeed the case
and, in fact, the spontaneous magnetization is a second order phase
transition \cite{Ruelle-1969,Firas,Robertson-1993,Christensen-2005}. We  
therefore anticipate a continuous but abrupt increase in the average 
magnetization per spin, hence in the average energy per spin, at a
critical temperature $T_c$. We also anticipate that, along the
critical isotherm $T=T_c$, the magnetization will vanish in the limit
$H\to0$. In other words, we anticipate that, in the infinite volume
limit, there exist positive critical exponents $\beta$, $\gamma$, $\delta$, and
$\alpha$ such that \cite{Christensen-2005,Baker-1990} 
%
\begin{align}\label{eq:Ising_Crit_Exponents}
  &M(T,0)\sim(T_c-T)^\beta, \hspace{0.11in} \text{as  }  T\to T_c^-,\\
  &\chi(T,0)\sim(T-T_c)^{-\gamma},  \hspace{0.1in} \text{as  }  T\to T_c^+, \notag\\
  &M(T_c,H)\sim H^{1/ \delta }, \hspace{0.35in} \text{as  }  H \to 0^+,\notag\\
  &c(T,0)\sim(T-T_c)^{-\alpha}, \hspace{0.12in} \text{as  }  T\to T_c^+.\notag
\end{align}
%

On may also analogously define positive critical exponents and $\gamma^\prime$,
and $\alpha^\prime$ for $T\to T_c^-$. Just as in percolation theory 
\cite{Christensen-2005,Stauffer-92}, we can also introduce a
correlation length, $\xi(T,H)$, which, loosely speaking, sets the
typical linear scale of the largest cluster of aligned spins at
temperature $T$ and external field $H$ \cite{Christensen-2005}. As the
temperature decreases, larger and larger clusters of aligned spins
appear \cite{Christensen-2005,Ruelle-1969}. We therefore anticipate
that there exists a positive critical exponent, $\nu$, such that 
%
\begin{align}
  \xi(T,0)\sim(T-T_c)^{-\nu},  \hspace{0.12in} \text{as  }  T\to T_c^+. 
\end{align}
%
The correlation length is mathematically defined through the
spin--spin correlation function
%
\begin{align}
  g(\vec{r}_i,\vec{r}_j)=\langle(s_i-\langle s_i\rangle_{\Hc})(s_j-\langle s_j\rangle_{\Hc})\rangle_{\Hc}
                     =\langle s_is_j\rangle_{\Hc}-\langle s_i\rangle_{\Hc}\langle s_j\rangle_{\Hc},
\end{align}
%
which describes the correlations in the fluctuations of the spins
$s_i$ and $s_j$, at positions $\vec{r}_i$ and $\vec{r}_j$, around
their average values $\langle s_i\rangle_{\Hc}$ and
$\langle s_j\rangle_{\Hc}$ \cite{Christensen-2005}. By the definition of
the magnetization, $M_\omega=\sum_is_i$, and by The Fluctuation Dissipation Theorem
\eqref{eq:Ising_susceptibility}, the spin--spin correlation function
in the Ising model is related to the susceptibility, hence the free
energy, in the following way \cite{Christensen-2005}
%
\begin{align}\label{eq:Correlation_sum_rule}
  \chi=\beta(\langle M_\omega^2\rangle_{\Hc}-M^2)
   =\beta\sum_{l,j=1}^Ng(\vec{r}_l,\vec{r}_j)
   =N\beta\sum_{j=1}^Ng(\vec{r}_i,\vec{r}_j)
   \approx N\beta\int_\Lambda g(\vec{r}_i,\vec{r}_j)d\vec{r}_j,
\end{align}
%
where, in the second to last step, we exploit the translation
invariance of the system, 
$g(\vec{r}_l,\vec{r}_j)=g(\vec{r}_l+\vec{r},\vec{r}_j+\vec{r})$, and
choose and arbitrary site $\vec{r}_i$ as a reference position
\cite{Christensen-2005}. By the last step in equation
\eqref{eq:Correlation_sum_rule}, and by equation
\eqref{eq:Ising_Crit_Exponents}, the correlation function cannot decay
exponentially fast with distance, $r=|\vec{r}_i-\vec{r}_j|$, when
$T\to T_c$ for $H=0$, and therefore decays like a power law at the
critical point  
%
\begin{align}
  g(\vec{r}_i,\vec{r}_j)\sim r^{-(d-2+\eta)},  \hspace{0.12in}
  \text{ for}\quad  (T,H)=(T_c,0),
\end{align}
%
where $d$ is the dimension and $\eta$ is a critical exponent
\cite{Christensen-2005}.

In the section \ref{sec:LYRB_Crit_Theory} we review a rigorous
mathematical theory, which characterizes the temperature driven phase 
transition of the Ising model in terms of a Stieltjes function
representation of the total magnetization $M$. This representation is
a direct consequence of The Lee--Yang--Ruelle Theorem, reviewed in
section \ref{sec:RuelleIsing}, which states that the zeros of the
Ising model partition function lie on the unit circle in an
appropriate complex variable. We will see in section
\ref{sec:Crit_Behav_of_Transport} that the associated
Lee--Yang--Ruelle--Baker critical theory of the Ising model can be 
adapted to transport problems in random binary composite media,
providing a rigorous mathematical theory which describes percolation
driven phase transitions therein.   

\subsection{Abstract Lattice Systems and The Lee--Yang--Ruelle Theorem}
\label{sec:RuelleIsing}
%
In 1952 T. D. Lee and C. N. Yang showed that the root distribution of
the Ising model partition function completely determines the
associated equation of state \cite{Yang:PR:404}. They also showed that
the problem of a ferromagnetic Ising model is mathematically identical
with that of a ``lattice gas'' \cite{Lee:PR:411,Thompson-1988}. More
specifically, they proved that the equation of state of the condensed
phase, as well as the gas phase, can be correctly obtained from a
knowledge of the distribution of partition function roots
\cite{Yang:PR:404,Lee:PR:411}. Moreover, they also demonstrated that
the properties of the system, in relation to phase transitions, are
determined by the behavior of the roots near the positive real
axis. They did so by proving that roots of the partition function, a
polynomial in an appropriate complex variable, lie on the unit circle
\cite{Lee:PR:411,Ruelle-1969}. Consequently, the temperature driven
phase transition (spontaneous magnetization) may be characterized by
the pinching of the real axis by the roots of the partition function;
an important characterization of the phase transition.

In 1971 D. Ruelle extended The Lee--Yang Theorem and, indeed, proved
that there exists a gap in the partition function zeros about the
positive real axis for high temperatures $T$
\cite{Ruelle:PRL:303}. Moreover, he proved that the gap collapses as
$T$ decreases to a critical temperature $T_c>0$. G. A. Baker subsequently
used The Lee-Yang--Ruelle Theorem to prove that the magnetization has
a Stieltjes integral representation
\cite{Baker:PRB:1184,Baker:PR:434,Baker:PRL-990,Gaunt:PRB-1184,Baker-1990}. He
showed that this integral representation provides detailed information
regarding the percolation aspects of the phase transition, and yields
(two--parameter) critical exponent scaling relations.  

In this section we outline an abstract Hilbert space
framework of the Ising model and prove The Lee--Yang Theorem. In
section \ref{sec:LYRB_Crit_Theory} we outline Baker's critical
theory of the Ising model, and sketch how the Stieltjes integral
representation of the magnetization leads to (two--parameter) scaling
relations for the critical exponents
\eqref{eq:Ising_Crit_Exponents}. This beautiful theory is utilized and
extended in section \ref{sec:Crit_Behav_of_Transport}, proving that
the Lee--Yang--Ruelle--Baker critical theory applies to classical 
transport problems such as electrical conductivity and permittivity,
and magnetic permeability. Thereby recovering Baker's two--parameter
scaling relations and Baker's inequalities for the critical exponents
for transport, both in the (real) static and (complex) quasi--static
cases.   

The following abstract Banach space characterization of lattice
systems, and the proof of The Lee--Yang Theorem, is due to D. Ruelle 
\cite{Ruelle-1969}. Consider a $d$ dimensional lattice
$\Lambda\subset\mathbb{Z}^d$. Associated with a lattice site $x\in\Lambda$ is an
occupation number, or equivalently a lattice site state number,
$n_x=0,\ldots,M$. Let there be $V$ sites and consider the 
set of all possible configurations of the system within $\Lambda:=\{x_j\}_{j=1}^V$,
$x_j\in\mathbb{Z}^d$. There are $(M+1)^V$ such configurations
parameterized by the $V$--tuples $(n_{x_1},\ldots,n_{x_V})$ where
$\sum_{j=1}^Vn_{x_j}:=n$. The set of ``admissible functions'' $\{f_\alpha\}$ on
$x\in\Lambda$, $\mathscr{H}_x$,  defines a Hilbert space of dimension $M+1$
\cite{Ruelle-1969}. A natural definition of the Hilbert space on $\Lambda$
is  given by $\mathscr{H}_\Lambda:= \bigotimes_{x\in\Lambda}\mathscr{H}_x$ of dimension
$(M+1)^V$. Each $f_\alpha\in\mathscr{H}_\Lambda$ is a self adjoint operator,
bounded below, with spectrum consisting of isolated eigen--values of
finite multiplicity \cite{Ruelle-1969}. 

Typically a potential energy acts on the particles
which is independent of momentum. In this case the Hamiltonian is
given by $\Hc_\Lambda=\sum_{j=1}^n|\vec{p}_j|^2/2m + U_\Lambda(x_1,\ldots,x_n)$,
where $\vec{p}_j$ and $m$ are the momentum and mass of
particle $j$, respectively. In systems such as this, the statistical
weight due to kinetics trivially cancels in all calculations
\cite{Thompson-1988,Ruelle-1969}. Therefore, we may make the
identification $\Hc_\Lambda:= U_\Lambda$, where $U_\Lambda$ summarizes all
interactions of the system with the environment and within the system
itself. 

To further develop the binary structure of the classical ferromagnetic
Ising model, consider a lattice gas with site state number $M=1$ so
that each site $x\in\Lambda$ is in state zero or one, e.g.,
corresponding to spin down and spin up 
respectively. The canonical ensemble is a measure,
$Z_\Lambda^{-1}\exp{(-\beta\Hc_\Lambda)}d\tau$, on the set of all
possible configurations of the system within $\Lambda$, where $\tau$ is the
measure corresponding to the Bernoulli random variable on $\Lambda$. Since the
mutual site interaction locations are uniquely 
determined by the boundary of the system $\partial\Lambda$ and the state one
particles, the potential energy may be written
\cite{Ruelle-1969,Firas}
%
\begin{align}
  U_\Phi (X):= U_\Lambda(X):=\sum_{Y\subset X}\Phi(Y), \quad X:=\{x\in\Lambda|n_x=1\},
\end{align}
%
where $\Phi(\emptyset):=0$.
% 
\begin{theorem}
  Define $\|\Phi\|:=\sum_{0\in X}|\Phi(X)|/N(X)$ where $N(X)$ is the cardinality of
  $X$ \emph{\cite{Ruelle-1969}}.
  \begin{itemize}
  \item[\it{(i)}] The set $\mathscr{B}$ of potentials
    $\Phi:X\to\mathbb{R}$ such that $\|\Phi\|<\infty$ is a Banach Space.
    \item[\it{(ii)}] The set $\mathscr{B}_0:=\{\Phi\in\mathscr{B} \ | \ \Phi(X)\neq0
      \text{ only for a finite number of sets } X\ni0\}$, is dense in
      $\mathscr{B}$          
  \end{itemize}   
\end{theorem}
%
\noindent A potential $\Phi\in\mathscr{B}_0$ is said to have \emph{finite
  range}. 
%
\begin{definition}
  A potential $\Phi\in\mathscr{B}^k\subset\mathscr{B}$ satisfying
$\Phi(X)=0$ unless $N(X)=k$, $k\geq1$ is known as a \emph{k body potential}.  
\end{definition}
%
\noindent In the infinite volume limit, \emph{translational
  invariance} is an important property of the system, i.e., for all
$a\in\mathbb{Z}^d$, $U_\Phi(X+a)=U_\Phi(X)$, or equivalently
$\Phi(X+a)=\Phi(X)$. Translational invariance implies $\mathscr{B}^1$ is one
dimensional, in fact it is easy to see that $\mathscr{B}^1$ is
isometrically isomorphic to $\mathbb{R}$. 
% 
\begin{definition}
  \label{def:Pair_Interaction}
  A potential $\Phi\in\mathscr{B}$ is a \emph{pair interaction} if
  $N(X)>2$ implies $\Phi(X)=0$, i.e., $\Phi:=(\Phi^1,\Phi^2)\in\mathscr{B}^1\oplus\mathscr{B}^2$. 
\end{definition}
%
\noindent Let $\Phi$ be a pair interaction with finite range, i.e., 
$\Phi\in\mathscr{B}_0^1\oplus\mathscr{B}_0^2$. Under this decomposition of
$\mathscr{B}$ we have $U_\Phi(X)=N(X)\Phi^1(X)+ U_{\Phi^2}(X)$
\cite{Ruelle-1969}. Therefore, 
%
\begin{align}
  \label{eq:RuellIsingHamiltonian}
  U_\Phi(X)&=N(X)\Phi^1+\sum_{\{x,y\}\subset X}\Phi^2(\{x,y\})\\\notag
       &=N(X)\Phi^1
         +\frac{1}{2}\sum_{x\in X}
         \left[\sum_{\overset{y\in \Lambda}{y\neq x}}-\sum_{y\in\Lambda\backslash X}\right]\Phi^2(\{x,y\})
         \\\notag
       & := N(X)(\Phi^1+C_{\Phi^2})-\frac{1}{2}\sum_{x\in X}\sum_{y\in X^c}\Phi^2(\{x,y\}).        
         \notag
\end{align}
%
Note that translational invariance of the system implies
$\Phi^2\{(x,y)\}=\Phi^2(|x-y|))$ so that $\Phi^2\{(x,y)\}=\Phi^2\{(y,x)\}$. 

To clarify the connection of the Hamiltonian
\eqref{eq:RuellIsingHamiltonian} with spin systems and, in particular,
the classical Ising model \eqref{eq:Ising_Model_Hamiltonian}, it is
necessary to remark on the two terms in the last line of equation
\eqref{eq:RuellIsingHamiltonian}. The $1/2$ factor accounts for the
double counting performed in the sum decomposition. The term
$N(X)(\Phi^1+C_{\Phi^2})$ is, apart from the negligible constant
$\frac{V}{2}(\Phi^1+C_{\Phi^2})$, the interaction of the spins with the
applied magnetic field \cite{Ruelle-1969}. The last term is the mutual
interactions of the states $n_x=0$ and $n_x=1$, or ``spin up--spin
down'' ($\{\uparrow\downarrow\}$) interactions. In the classical ferromagnetic Ising
model Hamiltonian  \eqref{eq:GeneralIsingStatement}, the spin up--spin
down interaction term is positive as $J>0$. Therefore, we have that
$\Phi^2\leq0$.    

With this abstraction of the Ising model, the partition function takes
a useful form:
%
\begin{align}\label{eq:Ruelle_Ising_PartitionFunction}
  Z_\Lambda(\Phi)&=\sum_{X\subset\Lambda}\exp{\left(-\beta U_\Phi(X)\right)}\\
  &=\sum_{X\subset\Lambda}\exp{\left(-\beta N(X)(\Phi^1+C_{\Phi^2})\right)}
          \exp{\left(\frac{\beta}{2}\sum_{x\in X}\sum_{y\in X^c}\Phi^2(\{x,y\})\right)}
  \notag\\
       &:=\sum_{X\subset\Lambda}z^{N(X)}\prod_{x\in X}\prod_{y\in X^c}e^{\frac{\beta}{2}\Phi(\{x,y\})}
       :=\sum_{X\subset\Lambda}z^{N(X)}\prod_{x\in X}\prod_{y\in X^c}A_{x,y}\notag\\
       &:=\mathcal{P}^n(z),\notag
\end{align}
%
where the variable $z$ is not to be confused with the analyticity
variable of $w(z)$ \eqref{eq:Gt_Integral} and $\tilde{w}(z)$
\eqref{eq:Ht_Integral}. The Lee--Yang Theorem is a direct consequence
of the form of the partition function
\eqref{eq:Ruelle_Ising_PartitionFunction} and the following theorem
characterizing multinomials \cite{Ruelle-1969}: 
%
\begin{theorem}
  \label{thm:GenLeeYang}
  Let $\{A_{ij}\}_{i\neq j}$ be a family of real numbers such that
  $A_{ij}=A_{ji}$ which satisfy $-1\leq A_{ij}\leq1$ for $i,j=1,\ldots,n$. We
  define a polynomial $\mathcal{P}_n$ of n variables by
  %
  \begin{align*}
    \mathcal{P}_n(z_1,\ldots,z_n):=\sum_Sz^S\prod_{i\in S}\prod_{j\in S^c}A_{ij},
  \end{align*}
  %
  where the summation runs over all subsets $S:=\{i_1,\ldots,i_s\}$ of the
  set $\{1,\ldots,n\}$, $z^S:=\prod_{m=1}^sz_{i_m}$, and $S^c=\{j_1,\ldots,j_{n-s}\}$ is
  the complement of $S$ in $\{1,\ldots,n\}$. Then
  $\mathcal{P}_n(z_1,\ldots,z_n)=0$ and $|z_m|\geq1$, for $m=1,\ldots ,n-1$,
  implies $|z_n|\leq1$. 
\end{theorem}
%
\begin{corollary}(The Lee--Yang Theorem)
  \label{thm:LeeYang}
  Using the notations of Theorem \ref{thm:GenLeeYang}, we define the
  polynomial in \eqref{eq:Ruelle_Ising_PartitionFunction},
  $\mathcal{P}^n(z)$ of degree $n$, by 
  %
  \begin{align*}
    \mathcal{P}^n(z):=\mathcal{P}_n(z,\ldots,z)=\sum_Sz^{N(S)}\prod_{i\in S}\prod_{j\in S^c}A_{ij}
  \end{align*}
  %
  where $N(S)$ is the number of elements in $S$. Then the zeros of
  $\mathcal{P}^n$ all lie on the unit circle $\{z:|z| =1\}$
\end{corollary}
%
\begin{proof}
 As $\Phi^2\leq0$, the real polynomial coefficients
 $A_{x,y}:=\exp{(\beta\Phi^2\{x,y\}/2)}$  satisfy $0<A_{x,y}\leq1$. As was
 mentioned above, translational invariance implies that
 $\Phi^2(\{x,y\})=\Phi^2(\{y,x\})$, thus $A_{x,y}=A_{y,x}$. Therefore, applying
 Theorem \ref{thm:GenLeeYang} to $\mathcal{P}^n(z)=0$, implies that
 $|z|\leq1$ and $|z|\geq1$, thus $|z|=1$.     
\end{proof}
%
\begin{remark}
\label{rem:LeeYang}  
It should be noted that in Theorem \ref{thm:LeeYang} no assumptions
are made about
%
\begin{itemize}
  \item the range of the interaction $\Phi^2$,
  \item the dimensionality of the lattice,
  \item or the size and structure of the lattice.
\end{itemize}
%
Furthermore, the coefficients $A_{x,y}$ do not explicitly dependent on
$\Phi^1$, hence the magnetic field $H$. This implies the distribution of
zeros of the partition function is not explicitly dependent on
$H$. Rather, dependent only on the nature of the mutual spin interactions.   
\end{remark}

The following theorem and its corollary provides an extension of The
Lee--Yang Theorem \cite{Ruelle:PRL:303}.  
%%
%% Ruelle Theorem
%%
\begin{theorem}(Ruelle)
  \label{thm:Ruelle}
  Let $\mathcal{P}$ be a complex polynomial of several variables, which
  is of degree one with respect to each, \textit{i.e.}, let $\Lambda$ be a
  finite set and 
%  
  \begin{equation*}
    P(z_{\Lambda}) = \sum_{X \subset \Lambda} c_{X} \; z^{X},
  \end{equation*}
%  
  where $z_{\Lambda} = (z_{x})_{x \in \Lambda }$, $z^{X} = \prod_{x \in X} z_{x}$, and
  $c_{X} \in \mathbb{C}$. Let $\Lambda_{\alpha}$ be a finite covering of $\Lambda$, and for
  every $x \in \Lambda_{\alpha}$ let $M_{\alpha,x}$ be a closed subset of
  $\mathbb{C}$ such that $0 \not\in M_{\alpha,x} $. \\
  Assume that, for each $\alpha$,
%  
   \begin{equation*}
    P_{\alpha}(z_{\Lambda_{\alpha}}) = \sum_{X \subset \Lambda_{\alpha}} c_{\alpha,X} \; z^{X} \neq 0, 
    \text{ when }  z_{x} \not\in M_{\alpha,x}, \text{ all } \ x\in\Lambda.
  \end{equation*}
%  
  Then the polynomial
%  
   \begin{equation*}
    P(z_{\Lambda}) = \sum_{X \subset \Lambda}  z^{X} \prod_{\alpha}c_{\alpha,(\Lambda_{\alpha}\bigcap X)} \neq 0, 
     \text{ when } z_{x} \not\in -\prod_{\alpha} (-M_{\alpha,x}), \text{ all }  x\in\Lambda.
  \end{equation*}
%
\end{theorem}
\begin{corollary}(Ruelle)
  \label{cor:RuelleLeeYangExtension}  
  \begin{itemize}
    \item[\it{(i)}](The Lee--Yang Theorem)
      Let $\Lambda_\alpha$ be the two--point subsets of $\Lambda$: $\Lambda_\alpha=\{x,y\}$ and let 
      $c_{\alpha,X}=A_{xy}$ when $X=\{x\}$ or $X=\{y\}$, and $c_{\alpha,X}=1$ when
      $X=\emptyset$ or $X=\{x,y\}$. For real $A_{xy}$ satisfying $-1\leq A_{xy}\leq1$ we
      may make the identification $M_{\alpha,x}=\{\xi\in\mathbb{C}:|\xi|\geq1\}$; hence
      %
      \begin{align*}
        Q(z)=\sum_{X\subset\Lambda}z^{N(X)}\prod_{x\in X}\prod_{y\in X^c}A_{xy}\notag
      \end{align*}
      %
      does not vanish when $|z|<1$. By symmetry, $Q(z)$ does not vanish
      when $|z|>1$, hence the zeros of $Q$ have absolute value 1; this
      is The Lee--Yang Theorem.
      \item[\it{(ii)}](Zero Free Region)
        Let $\Phi$ be a real function on the $d$-tuples of integers mod
        $m$, the ``periodic lattice'' $\mathbb{Z}_m^d$, with
        $\Phi(x)=\Phi(-x)$, and take $\Lambda=\mathbb{Z}_m^d$. Let $\Lambda_\alpha$ be the
        two--point subsets of $\Lambda:\Lambda_\alpha=\{x,y\}$, and write
        $c_{\alpha,X}=\exp{[-\beta\Phi(x-y)]}$ when $X=\{x,y\}$ and $c_{\alpha,X}=1$ when
        $X=\emptyset$, $\{x\}$, or $\{y\}$. Take $M_{\alpha,x}=\Delta_{xy}^\beta$ where
  %      
  \begin{align*}
    \Delta_{xy}^\beta=\left\{
    %
    \begin{array}{ll}
       \{\xi\in\mathbb{C}:|\xi+1|\leq(1-e^{\beta\Phi(x-y)})^{1/2}\} & \text{ for }
    \Phi(x-y)\leq0,\\
    %
    \{\xi\in\mathbb{C}:|\xi e^{-\beta\Phi(x-y)}+1\leq (1-e^{-\beta\Phi(x-y)})^{1/2}\} &\text{ for }
    \Phi(x-y)\geq0.
    \end{array}
    %
    \right.
  \end{align*}
  %
   Then
  % 
   \begin{align*}
     Q(z)=\sum_{X\subset\Lambda}z^{N(X)}\exp{[-\beta\sum_{\{x,y\}\subset X}\Phi(x-y)]}\notag
   \end{align*}
  % 
   can vanish only when
   \begin{align*}
     z\in\Gamma^\beta=-\prod_{y\in\mathbb{Z}_m^d}(-\Delta_{0y}^\beta).
   \end{align*}
  For small $\beta$, $\Gamma^\beta$ does not intersect the positive real axis.
\end{itemize}
\end{corollary}
%

Corollary \ref{cor:RuelleLeeYangExtension} extends The Lee--Yang
Theorem and rigorously proves the existence and uniqueness of a phase
transition (at $z=1$), by establishing the existence of a gap in the
zeros of the partition function about $z=1$ for large $T$ which
vanishes for a critical temperature $T_c>0$. More specifically, The
Lee--Yang--Ruelle Theorem establishes the existence of an angle
$\theta_0(T)$, defined from the positive real $z$ axis, such that
$\theta_0(T)>0$ for $T>T_c$ and $\lim_{T\to T_c}\theta_0(T)=0$. Consequently, the zeros of the
partition function, $z_i:=\exp(\I\theta_{i})$, are contained in the set
$\mathcal{Z}_T$, where  
%
\begin{align}
  \mathcal{Z}_T
   :=\{z\in\mathbb{C}\;|\; z=\exp(\I\theta) \text{ and } \theta\not\in[-\theta_{0}(T),\theta_{0}(T)]\}.
\end{align}
%

Using some useful symmetries of the multinomial in Theorem
\eqref{thm:GenLeeYang}, D. Ruelle was able to completely characterize 
the Lee--Yang polynomials \cite{Ruelle:AM:589}. The existence of a
second order phase transition in the Ising model may be established by
other means \cite{Ruelle-1969,Firas}, although The Lee--Yang--Ruelle
Theorem has many deep and far reaching consequences. It has led to
Lee--Yang--Ruelle--Baker critical theory of the Ising model, which 
provides important details regarding the percolation aspects of the
temperature driven phase transition, and may be adapted to percolation
driven phase transitions in transport problems of binary composite
media. We now outline Lee--Yang--Ruelle--Baker critical theory of the
Ising model.      
%
\subsection{Lee--Yang--Ruelle--Baker critical theory
of the Ising model}\label{sec:LYRB_Crit_Theory}
%
A traditional viewpoint is that, in the GBCE, a phase transition
is characterized by regions of phase space where the Helmholtz free
energy loses its analyticity
\cite{Christensen-2005,Robertson-1993,Baker-1990}. One can clearly see   
from the definition of the Helmholtz free energy that this can happen
when zeros of the partition function cross physically realizable
values. Corollary \ref{cor:RuelleLeeYangExtension} demonstrates that,
for the Ising model, this can occur only at $z=1$, and that the phase
transition is characterized by the pinching of the real axis by the
partition function zeros.

In the classical Ising model $z=\exp(-2\beta mH)$ \cite{Baker-1990},
therefore the zeros of the partition function lie on the imaginary
axis in the $H$-plane with a gap in the interval $[\I0,\I\theta_H]$, where
$\theta_H\to0$ as $T\to T_c^+$. Therefore the phase transition can occur only
at $H=0$. In the $J$-plane the situation is somewhat more complex, as
was pointed out independently by Fisher and Jones. For the
two-dimensional Ising model in zero magnetic field, Fisher has shown
that these zeros must lie on two circles in the complex $v =
\tanh(2\beta J)$ plane \cite{Golden:JMP-5627}.    

Using regular assemblies, the leading polynomial coefficient of the
classical Ising model partition function (on a general lattice) can be
determined \cite{Baker-1990}. The partition function, $Z$, and Helmholtz
free energy per spin, $f:=\Fc/N$, are given by
\cite{Baker-1990}       
%
\begin{align}
  Z =e^{\beta N\zeta}\prod_{j=1}^{N}(z-z_j),\quad
  f=-\zeta -\beta^{-1} \int\ln|z-z_i|d\nu(z),\quad
      \zeta=mH+\frac{qJ}{2}
\end{align}
%
where $\nu(dz)=N^{-1}\sum_{j=1}^{N}\delta_{z_i}(dz)$ is the normalized counting
measure of the partition function roots and $q$ is the number of
nearest neighbor sites of the given lattice. As $N\to\infty$, the average
root density converges to an analytic function in $z$, both inside and
outside the unit circle \cite{Yang:PR:404,Ruelle-1969}. Furthermore,
in this limit, the distribution of zeros, $\mathcal{Z}_T$, is a subset
of the unit circle in the $z$ plane \cite{Lee:PR:411,Wintner:MP:1},
and is given by the analytic density function, $g(\theta)$, on the unit
circle \cite{Baker-1990}. By Helly's selection principle, the
symmetries of complex roots of a real valued polynomial, and The
Lee--Yang--Ruelle Theorem, in the infinite volume limit we have
\cite{Baker-1990}    
%
\begin{align}\label{eq:inf_vol_Free_Energy}
   f &=-\zeta - \beta^{-1} \int_{\theta_0(T)}^{2\pi - \theta_0(T)} \ln|z-e^{\I \theta}| dg(\theta)
  =  -\zeta -\beta^{-1} \int_{\theta_{0}(T)}^{\pi} \ln (1 + z^2 -2z \cos\theta) dg(\theta).
\end{align}
%
Therefore, the system equilibrium is given by the minimization of a
logarithmic potential over measures $g(d\theta)=g(\theta)d\theta$.

Define the following unitless variables $h:=\beta mH$ and
$\tau:=\tanh{h}$. Using equation \eqref{eq:inf_vol_Free_Energy}, 
G. A. Baker showed that the magnetization intensity per spin, defined
as $M/m=-\partial(\beta f)/\partial h$, is given by  
%
\begin{align}\label{Ising_Stieltjes_Fun}
  \frac{M}{m} =\tau(1+(1-\tau^2)G(\tau^2)), \quad G(\tau^2)=\int_0^\infty\frac{d\psi(y)}{1+\tau^2y}\;,
\end{align}
%
where $G$ is a Stieltjes (or Herglotz) function of $\tau^2$ and $\psi(dy)$
is a non--negative definite measure, defined by manipulation of
$g(d\theta)$, which is supported in $[0,S(T)]$
\cite{Baker:PR:434,Baker:PRL-990,Baker-1990}. The integral
representation \eqref{Ising_Stieltjes_Fun} immediately leads to the
inequalities
%
\begin{align}\label{eq:Gtau_inneq}
  G\geq0, \qquad \frac{\partial G}{\partial u}\leq0, \qquad \frac{\partial^2G}{\partial u^2}\geq0,
\end{align}
%
where $u:=\tau^2$. The last equation in \eqref{eq:Gtau_inneq} is the GHS
inequality, which is an important tool in the study of the Ising model
\cite{Golden:JMP-5627}. 

Define a critical exponent, $\Delta$, for the gap, $\theta_0(T)$, in the
distribution of the Lee--Yang--Ruelle zeros. The manipulations in
\cite{Baker:PR:434,Baker:PRL-990} show that the upper limit of
integration, $S(T)$, not to be confused with the entropy, diverges as
$T\to T_c^+$ with critical exponent $-2\Delta$. It also can be shown that
moments $\psi_n$ of the measure $\psi(dy)$ scale like $\partial^{2n}f/\partial H^{2n}$,
and therefore also diverge as $T\to T_c^+$ with critical exponents
$\gamma_n$, $n\geq1$ \cite{Baker:PR:434,Baker:PRL-990,Baker-1990}. In summary  
%
\begin{align}
  &\theta_0(T)\sim(T-T_c)^\Delta, \hspace{0.25in} \text{as  } T\to T_c^+,\\
  &S(T)\sim(T-T_c)^{-2\Delta}, \hspace{0.13in} \text{as  }  T\to T_c^+,\notag\\
  &\psi_n\sim(T-T_c)^{-\gamma_n}, \hspace{0.33in} \text{as  }  T\to T_c^+,\notag
\end{align}
%
with $\gamma:=\gamma_1$. We may also define a critical exponent, $\Delta^\prime$, for
$T\to T_c^-$.

The integral representation \eqref{Ising_Stieltjes_Fun} of the
magnetic intensity per spin was used to obtain the following
(two--parameter) scaling relations for the critical exponents $\beta$,
$\gamma$, $\delta$, and $\Delta$, and \emph{Baker's inequalities} for the critical
exponents $\gamma_n$ of the higher field derivatives of the free energy per
spin $f$, or equivalently, of the moments $\psi_n$ of $\psi(dy)$
\cite{Baker:PR:434,Baker:PRL-990,Baker-1990}   
%
\begin{align}
  \beta=\Delta-\gamma, \qquad \delta=\frac{\Delta}{\Delta-\gamma}, \qquad \gamma_{n+1}-2\gamma_n+\gamma_{n-1}\geq0.
\end{align}
%
Moreover, the sequence $\gamma_n$ is actually linear in $n$,
%
\begin{align}
  \gamma_n=\gamma+2n\Delta,\quad n\geq0
\end{align}
%
with constant gap $\gamma_{i+1}-\gamma_i=2\Delta$
\cite{Baker:PR:434,Baker:PRL-990,Baker-1990}. These results are
referred to as ``two-parameter scaling'' because  all the
thermodynamic critical indices can be derived from just two,
\textit{e.g.} $\Delta $ and $\gamma $. Other sets of basic indices are possible
as well \cite{Baker-1990}. These homogeneity ideas also lead to the
following exponent relations and equalities \cite{Baker-1990}
% 
\begin{align} \label{eq:Tpm_Exponent_Relations}  
  \alpha^{\prime } &+ 2\beta + \gamma = 2, \hspace{0.36in} (\text{the Rushbroke relation})
\\
  2 &- \alpha^{\prime } \leq \beta(\delta + 1), \hspace{0.19in}
   (\text{the Griffiths' thermodynamic inequality})\notag
\end{align}
%
\section{Critical Behavior of Transport in Lattice and Continuum
  Percolation Models}
\label{sec:Crit_Behav_of_Transport}
%
In 1997 K. M. Golden proved that, in the static limit,
Lee--Yang--Ruelle--Baker critical theory may be adapted to
characterize percolation driven critical transitions in transport  
\cite{Golden:PRL-3935}. This deep and far reaching result puts these
two classes of seemingly unrelated problems on an equal mathematical
footing. He did so by considering percolation models, where the
connectedness of the system is determined by the volume fraction $p$
of defect inclusions in an otherwise homogeneous medium. He
demonstrated that the function $m(h):=m(p,h)$ \eqref{eq:Fs_Integral}
plays the role of the magnetization per spin $M(T,H)$ in the Ising
model. Here, the volume fraction $p$ mimics the temperature $T$
while the contrast ratio $h$ mimics the applied magnetic field
$H$. More specifically, the critical insulator/conductor behavior in
transport arises when $h=0$, as $p\to p_c$ \cite{Golden:PRL-3935}, and
in the Ising model the analogous critical behavior arises when $H=0$,
as $T\to T_c$ \cite{Christensen-2005}. Using these mathematical
parallels, K. M. Golden showed that the critical exponents of
transport satisfy Baker's inequalities, Baker's (two--parameter) scaling
relations, etc.  

Here, in a novel unified approach we reproduce Golden's static
results for the insulator/conductor system, and produce the analogus
static results for the conductor/superconductor system, finding the
(two--parameter) scaling relations of each system. Moreover, we extend
the results pertaining to each system to the quasi--static limit,
where $h$ becomes complex $(h\in\mathbb{C})$
\cite{Jackson-1999,Efros:PSSB-303}. Under a physically consistent
\cite{Efros:PSSB-303} symmetry assumption, we link these two sets of
scaling relations so that, under this assumption, the scaling
relations of both systems are determined by only (two) parameters. We
also provide a general proof of the fundamental assumption underlying
Golden's critical theory: the existence of a spectral gap which
collapses as $p\to p_c$. The proof thereof characterizes the phase
transition via the appearance of a delta function in the spectrum
\emph{precisely} at the percolation threshold.          

The most natural formulation of this problem is in terms of the
conduction problem in the continuum $\mathbb{R}^d$, which includes the
lattice $\mathbb{Z}^d$ as a special case
\cite{Golden:JMP-5627,Golden:CMP-473}. Although, the underlying
symmetries in the effective parameter problem of electrical
conductivity and permittivity, magnetic permeability, and thermal 
conductivity, immediately generalize the results of this section to
all of these systems \cite{MILTON:2002:TC}. Let $\sigma_j$ denote the
complex conductivity of material component $j=1,2$ of the binary
composite \cite{Efros:PSSB-303}, and $\sigma^*(p,h)$ denote the effective
complex conductivity. In the limit $h:=\sigma_1/\sigma_2\to0$, the composite may  
be interpreted as a conductor/superconductor system ($\sigma_2\to\infty$ while
$0<|\sigma_1|<\infty$), or a conductor/insulator system ($\sigma_1\to0$ while
$0<|\sigma_2|<\infty$). 

The relationship \eqref{eq:Fs_relationships_G} between the different 
representations of the effective complex conductivity,
$\sigma^*(p,h):=\sigma_2\,m(p,h)=\sigma_1w(p,z(h))$
\eqref{eq:Fs_Integral}--\eqref{eq:Gt_Integral}, is exploited to
illuminate many of the symmetries in this framework. We show how 
symmetries between the integral representations of $\sigma^*$ and
$[\sigma^{-1}]^*$ may be used to immediately generalize the results of
this section in terms of $[\sigma^{-1}]^*$. Some of the more subtle measure
theoretic details regarding the underlying symmmetries between
$m(p,h)$ and $w(p,z(h))$ are discussed in section
\ref{sec:Measure_Equiv}. This leads to a generalization of a result
\cite{Day:JPCM-96} which determines the measure $\varrho(d\lambda)$ introduced in
equation \eqref{eq:BM_measure_relationship_E}, and characterizes the
phase transition by the appearance of a delta function component in
$\varrho(d\lambda)$ \emph{precisely} at the percolation threshold. In section
\ref{sec:StatMech_of_Composites} we will demonstrate that this
critical theory of transport may be extended further, characterizing 
the GBCE statistical mechanics description of (electrically/thermally)
driven phase transitions in binary composite media.           
%
\subsection{Formulation} \label{sec:Crit_Theory_Formulation} 
%
As we have already discussed the effective permittivity problem in
section \ref{sec:Background_TACM} and the effective permeability
problem in section \ref{sec:Magnetic_Systems}, we now give an extremely
brief formulation of the effective conductivity problem
\cite{Golden:JMP-5627,Golden:CMP-473,Golden:PRL-3935}. Let the local
conductivity be defined
as $\bsig(\vec{x},\omega):=\sigma_1\chi_1(\vec{x},\omega)+\sigma_2\chi_2(\vec{x},\omega)$ 
and its inverse be defined as
$[\bsig^{-1}](\vec{x},\omega):=\chi_1(\vec{x},\omega)/\sigma_1+\chi_2(\vec{x},\omega)/\sigma_2$,
which are two--valued stationary random fields in
$\vec{x}\in\mathbb{R}^d$ and $\omega\in\Omega$. Let $\vec{E}(\vec{x},\omega)$ and
$\vec{J}(\vec{x},\omega)$ be stationary random electric and current fields,
which are related by $\vec{J}=\bsig\vec{E}$ and satisfy 
%
\begin{align}\label{eq:Maxwells_Equations_EJ}  
  	\vec{\nabla}\times\vec{E}=0, \quad
	\vec{\nabla}\cdot\vec{J}=0,\qquad 	
	\vec{E}=\vec{E}_0+\vec{E}_f, \quad
	\langle\vec{E}\rangle=\vec{E}_0:=E_0\vec{e}_k.
\end{align}
%
For the random resistor network (RRN), the differential equations
become difference equations (Kirchoff's laws)
\cite{Golden:CMP-467,Golden:JMP-5627}.

The effective complex conductivity tensor, $\bsig^*$, and 
$[\bsig^{-1}]^*$ are defined via the averages
$\langle\vec{J}\rangle=\bsig^*\langle\vec{E}\rangle$ and $\langle\vec{E}\rangle=[\bsig^{-1}]^*\langle\vec{J}\rangle$,
respectively. Without loss of  
generality, we focus on one diagonal component of these
symmetric tensors: $\sigma^*:=\bsig^*_{kk}$ and
$[\sigma^{-1}]^*:=[\bsig^{-1}]^*_{kk}$. Due to the homogeneity of the
functions $\sigma^*$ and $[\sigma^{-1}]^*$, we consider the dimensionless functions
$m(h)=\sigma^*/\sigma_2$, $w(z)=\sigma^*/\sigma_1$, $\tilde{m}(h)=\sigma_1[\sigma^{-1}]^*$, and
$\tilde{w}(z)=\sigma_2[\sigma^{-1}]^*$, defined in
\eqref{eq:Fs_Integral}--\eqref{eq:Ht_Integral}, where $h=\sigma_1/\sigma_2$ and
$z=1/h$. To simplify the presentation of this framework we focus on
the variable $h$, and make the following definitions: $w(h):=w(z(h))$
and $\tilde{w}(h):=\tilde{w}(z(h))$. We assume that $|h|<1$,
i.e. $0<|\sigma_1|<|\sigma_2|<\infty$, and we further restrict $h$ in the complex
plane to the set  
%
\begin{align}\label{eq:h_Domain}
  \mathcal{U}:=\{h:=h_r+\I h_i\in\mathbb{C}: |h|<1 \text{ and } h\not\in(-1,0]\},
\end{align}
%
where $m(h)$, $w(h)$, $\tilde{m}(h)$, and $\tilde{w}(h)$ are analytic
functions of $h$ \cite{Golden:CMP-473}.  

The associated Stieltjes (or Herglotz) functions are given by $F(s)=1-m(h)$,
$G(t)=1-w(z)$, $E(s)=1-\tilde{m}(h)$, and $H(t)=1-\tilde{w}(z)$, where
$s=1/(1-h)$ and $t=1-s$. In order to illuminate the symmetries between
these functions we focus on the variable $s$, and define
$G(s):=G(t(s))$ and $H(s):=H(t(s))$. Using $h$ and $s$ in lieu of $z$
and $t$, respectively, these functions have the following integral
representations 
% 
\begin{align}\label{eq:Herglotz_Funs_sed_LYRB}
  F(s)&=\langle\chi_1(s+\Gamma\chi_1)^{-1}\vec{e}_k\cdot\vec{e}_k\rangle=\int_{\lambda_0}^{\lambda_1}\frac{d\mu(\lambda)}{s-\lambda}\,,\\
  E(s)&=\langle\chi_2(s-\Upsilon\chi_2)^{-1}\vec{d}_k\cdot\vec{d}_k\rangle=\int_{\tilde{\lambda}_0}^{\tilde{\lambda}_1}\frac{d\eta(\lambda)}{s-\lambda}\,,
   \notag \\
  G(s)&=\langle\chi_2(1-s+\Gamma\chi_2)^{-1}\vec{e}_k\cdot\vec{e}_k\rangle=-\int_{1-\hat{\lambda}_1}^{1-\hat{\lambda}_0}\frac{[-d\alpha(1-\lambda)]}{s-\lambda}\,,
   \notag \\
  H(s)&=\langle\chi_1(1-s-\Upsilon\chi_1)^{-1}\vec{d}_k\cdot\vec{d}_k\rangle=-\int_{1-\check{\lambda}_1}^{1-\check{\lambda}_0}\frac{[-d\tau(1-\lambda)]}{s-\lambda}\,.
  \notag
\end{align}
%
Here $\mu$, $\eta$, $\alpha$, and $\tau$ are bounded positive measures which
depend only on the geometry of the medium
\cite{Golden:CMP-473,Bergman:AP-78}, and are supported on
$\Sigma_\mu,\Sigma_\eta,\Sigma_\alpha,\Sigma_\tau\subseteq[0,1]$, respectively, where $\lambda_0:=\inf(\Sigma_\mu)\in[0,1)$, 
$\lambda_1:=\sup(\Sigma_\mu)\in(0,1]$, $\tilde{\lambda}_0:=\inf(\Sigma_\eta)\in[0,1)$, $\tilde{\lambda}_1:=\sup(\Sigma_\eta)\in(0,1]$,
$\hat{\lambda}_0:=\inf(\Sigma_\alpha)\in[0,1)$, $\hat{\lambda}_1:=\sup(\Sigma_\alpha)\in(0,1]$,
$\check{\lambda}_0:=\inf(\Sigma_\tau)\in[0,1)$, and  $\check{\lambda}_1:=\sup(\Sigma_\tau)\in(0,1]$ (see section
\ref{sec:Resolv_Rep_E_D} for details).

By the positivity of these measures, the following inequalities hold
for all $p\in[0,1]$: 
%
\begin{align}\label{eq:Herglotz_Inneq}
  \frac{\partial^{2n}F}{\partial s^{2n}}>0, \qquad
  \frac{\partial^{2n-1}F}{\partial s^{2n-1}}<0,
    \qquad \qquad
  \frac{\partial^{2n}E}{\partial s^{2n}}>0,\qquad
  \frac{\partial^{2n-1}E}{\partial s^{2n-1}}<0,
    \\ 
  \frac{\partial^{2n}G}{\partial s^{2n}}<0, \qquad
  \frac{\partial^{2n-1}G}{\partial s^{2n-1}}>0,
    \qquad \qquad
  \frac{\partial^{2n}H}{\partial s^{2n}}<0, \qquad
  \frac{\partial^{2n-1}H}{\partial s^{2n-1}}>0,
  \notag
\end{align}
%
where $n\geq0$ and $h\in\mathcal{U}$. Equation
\eqref{eq:Herglotz_Inneq} is the analogue of equation
\eqref{eq:Gtau_inneq} in the Ising model. The formula $\partial^2F/\partial s^2>0$
in \eqref{eq:Herglotz_Inneq}, for example, is a macroscopic version of
the fact that the effective resistance of a finite network is a
concave downward function of the resistances of the individual network
elements \cite{Golden:JMP-5627}. When $h\in\mathcal{U}$ such that $h_i\neq0$,
equations \eqref{eq:Herglotz_Inneq} become
%
\begin{align}\label{eq:Herglotz_NonZero}
  \left|\frac{\partial^nF}{\partial s^n}\right|>0, \quad
  \left|\frac{\partial^nE}{\partial s^n}\right|>0, \quad
  \left|\frac{\partial^nG}{\partial s^n}\right|>0, \quad
  \left|\frac{\partial^nH}{\partial s^n}\right|>0, \quad p\in[0,1].
\end{align}
%

The formulas in equation \eqref{eq:Herglotz_Funs_sed_LYRB} are, up to
sign and reflection of the spectrum about $\lambda=1/2$, Stieltjes
transforms of the measures $\mu$, $\eta$, $\alpha$, and $\tau$. These can be
converted into Stieltjes functions \cite{Baker-1990} of $h$ via the
change of variables $s=1/(1-h)$ and $\lambda(y)=y/(y+1)\iff y(\lambda)=\lambda/(1-\lambda)$ so
that, for example,  
%
\begin{align}\label{eq:var_subs_Fs}
  F(s)&=\int_{S_0}^{S}\frac{d\mu(\frac{y}{y+1})}
                {\frac{1}{1-h}-\frac{y}{y+1}}
                :=(1-h)\int_{S_0}^{S}\frac{(y+1)d\mu(\frac{y}{y+1})}{1+hy}
                %:=(1-h)\int_{S_0}^{S}\frac{d\phi(y)}{1+hy}
                \,,  \notag\\
  G(s)&=-\int_{\hat{S}_0}^{\hat{S}}\frac{[-d\alpha(\frac{1}{y+1})]}
                {\frac{1}{1-h}-\frac{y}{y+1}}
                :=(h-1)\int_{\hat{S}_0}^{\hat{S}}\frac{(y+1)[-d\alpha(\frac{1}{y+1})]}{1+hy}
                %:=(h-1)\int_{\hat{S}_0}^{\hat{S}}\frac{d\ph(y)}{1+hy}
                \,.               
\end{align}    
%
Here $S_0:=\lambda_0/(1-\lambda_0)$, $S:=\lambda_1/(1-\lambda_1)$,
$\hat{S}_0:=(1-\hat{\lambda}_1)/\hat{\lambda}_1$, and $\hat{S}:=(1-\hat{\lambda}_0)/\hat{\lambda}_0$,
so that $\lim_{\lambda_0\to0}S_0=0$, $\lim_{\lambda_1\to1}S=\infty$,
$\lim_{\hat{\lambda}_1\to1}\hat{S}_0=0$, $\lim_{\hat{\lambda}_0\to0}\hat{S}=\infty$, and
$S\in\mathbb{R}^+$ is not to be confused with entropy. Therefore, by
equation \eqref{eq:var_subs_Fs} and the underlying symmetries in
equations \eqref{eq:Herglotz_Funs_sed_LYRB}, the Stieltjes function
representations of the formulas in equation
\eqref{eq:Herglotz_Funs_sed_LYRB} are given by         
% 
\begin{align}\label{eq:mh_Stieltjes_rep} 
    m(h)&=1+(h-1)g(h),\quad
    g(h):=\int_0^\infty\frac{d\phi(y)}{1+hy}\,, \quad
    d\phi(y):=(y+1)d\mu\left(\frac{y}{y+1}\right),\notag \\
%     
    \tilde{m}(h)&=1+(h-1)\tilde{g}(h), \quad
    \tilde{g}(h):=\int_0^\infty\frac{d\tilde{\phi}(y)}{1+hy}\,,\quad
    d\tilde{\phi}(y):=(y+1)d\eta\left(\frac{y}{y+1}\right),\notag \\
%    
     w(h)&=1-(h-1)\hat{g}(h),\quad
     \hat{g}(h):=\int_0^\infty\frac{d\ph(y)}{1+hy}\,, \quad
     d\ph(y):=(y+1)\left[-d\alpha\left(\frac{1}{y+1}\right)\right],\notag \\
%     
    \tilde{w}(h)&=1-(h-1)\check{g}(h),
      \quad \check{g}(h):=\int_0^\infty\frac{d\check{\phi}(y)}{1+hy}\,,\quad
      d\check{\phi}(y):=(y+1)\left[-d\tau\left(\frac{1}{y+1}\right)\right].
\end{align}
%
As $\mu$, $\eta$, $\alpha$, and $\tau$ are bounded positive measures on
$[0,1]$, $\phi$, $\tilde{\phi}$, $\ph$, and $\check{\phi}$ are positive
measures on $[0,\infty]$, and are also bounded if the supremum of the
support of these measures is finite. Equations
\eqref{eq:mh_Stieltjes_rep} are general formula holding for two
component stationary random media in lattice and continuum settings
\cite{Golden:PRL-3935}, and should be compared to equation
\eqref{Ising_Stieltjes_Fun} regarding the Ising model.       

By equation \eqref{eq:mh_Stieltjes_rep}, the moments $\phi_n$ of $\phi$
satisfy  
%
\begin{align}\label{eq:phi_moments}
  \phi_n=\int_0^\infty y^nd\phi(y)
    =\int_0^\infty y^n(y+1)d\mu\left(\frac{y}{y+1}\right)
    =\int_0^1\frac{\lambda^nd\mu(\lambda)}{(1-\lambda)^{n+1}}\,.
\end{align}
%
A partial fraction expansion of $\lambda^n/(1-\lambda)^{n+1}$ then shows that
%
\begin{align}\label{eq:phi_moments_F(s)}
  \frac{(-1)^n}{n!}\lim_{s\to1}\frac{\partial^nF(s)}{\partial s^n}=\int_0^1\frac{d\mu(\lambda)}{(1-\lambda)^{n+1}}
                                =\sum_{j=0}^n{n \choose j} \phi_j\,,
\end{align}
%
demonstrating that $\phi_n$ depends on $\int_0^1d\mu(\lambda)/(1-\lambda)^{n+1}$
(and) all the lower moments of $\phi$: $\phi_j$, $j=0,1,\ldots,n-1$. Moreover,
equation \eqref{eq:phi_moments} suggests that the moments $\phi_n$ become
singular as $\sup\{\Sigma_\mu\}\to1$. However
we will show that this is only true for the moments of order $j\geq1$,
and that $\lambda=1$ is a removable simple singularity under $\mu$. Theorem
\ref{thm:Herglotz_Decomp_Energy} of section
\ref{subsec:Spec_Decomp_Energy} further identifies the first two
moments of $\phi$ with energy components: 
%
\begin{align}\label{eq:phi_energy_relations}
  \phi_0=\lim_{s\to1}\frac{\langle\chi_1\vec{E}\cdot\vec{E}_0\rangle}{E_0^2},   \quad
  \phi_1=\lim_{s\to1}\frac{\langle E_f^2\rangle}{E_0^2}.
\end{align}
%

Similarly, the moments $\ph_n$ of $\ph$ satisfy
%
\begin{align}\label{eq:phi_hat_moments}
  \ph_n%&=\int_0^\infty y^nd\ph(y)
      %=\int_0^\infty y^n(y+1)d\alpha\left(\frac{1}{y+1}\right)
      &=\int_0^1\frac{\lambda^n[-d\alpha(1-\lambda)]}{(1-\lambda)^{n+1}}
      =\int_0^1\frac{(1-\lambda)^nd\alpha(\lambda)}{\lambda^{n+1}}
      =\sum_{j=0}^n(-1)^j {n \choose j} \int_0^1\frac{d\alpha(\lambda)}{\lambda^{n+1-j}}\notag\\
      &=\sum_{j=0}^n\frac{(-1)^{n+1}}{(n-j)!}{n \choose j}
             \lim_{s\to1}\frac{\partial^{n-j}G(s)}{\partial s^{n-j}}.
\end{align}
%
Equation \eqref{eq:phi_hat_moments} suggests, and we will show, that
the moments $\ph_n$ become singular as $\inf\{\Sigma_\alpha\}\to0$ for all
$n\geq0$. Theorem \ref{thm:Herglotz_Decomp_Energy} of section
\ref{subsec:Spec_Decomp_Energy} similarly identifies the first two
moments of $\ph$ with energy components. By the symmetries in
equations \eqref{eq:Herglotz_Funs_sed_LYRB} and
\eqref{eq:mh_Stieltjes_rep}, equations 
\eqref{eq:phi_moments}--\eqref{eq:phi_moments_F(s)} hold for
$\tilde{\phi}$ with $E(s)$ and $\eta$ in lieu of $F(s)$ and $\mu$, and equation
\eqref{eq:phi_hat_moments} holds for $\check{\phi}$ with $H(s)$ and $\tau$ in lieu
of $G(s)$ and $\alpha$. Furthermore, Theorem \ref{thm:Herglotz_Decomp_Energy}
identifies the first two moments of $\tilde{\phi}$ and $\check{\phi}$ with
energy components. In order to make connections to $F(s)$ and $G(s)$ in the
representation of equations \eqref{eq:phi_moments_F(s)} and
\eqref{eq:phi_hat_moments}, we have asumed that $F(s)$ and $G(s)$ may
be differentiated under the intergral sign with respect to $s$. This
is warrented by Lemma \ref{lem:h_diff_commutation} below. 

For percolation models such as the RRN
\cite{Stauffer-92,Torquato:RHM-02}, the connectedness of the system is
determined by the volume fraction $p$ of type two inclusions in an
otherwise homogeneous type one medium. The average cluster size of
these  inclusions grows as $p$ increases, and there is a critical
volume fraction $p_c$, $\;0<p_c<1$, called the \emph{percolation
  threshold}, where an infinite cluster of the inclusions first 
appears. Consider transport through a RRN \cite{Golden:PRL-3935} where 
bonds are assigned electrical conductivities $\sigma_2$ with probability
$p$, and $\sigma_1$ with probability $1-p$. As $h\to0$ ($\sigma_1\to0$ and
$0<|\sigma_2|<\infty$), the effective conductivity $\sigma^*(p,h):=\sigma_2\,m(p,h)$ and the
effective inverse conductivity
$[\sigma^{-1}]^*(p,h):=\sigma_2^{-1}\tilde{w}(p,h)$ undergo a
conductor/insulator critical transition. While, as $h\to0$ ($\sigma_2\to\infty$ and
$0<|\sigma_1|<\infty$), the effective conductivity $\sigma^*(p,h):=\sigma_1w(p,h)$ and
inverse effective conductivity
$[\sigma^{-1}]^*(p,h):=\sigma_1^{-1}\tilde{m}(p,h)$ undergo a 
conductor/superconductor critical transition:      
%
\begin{align}\label{eq:Cond-Insul_Crit_Beh_pc}
  &|\sigma^*(p,0)|:=|\sigma_2\,m(p,0)|=\left\{
    \begin{array}{ll}
      0, &       \text{for } p<p_c\\
      0<|\sigma_1|<|\sigma^*(p)|<|\sigma_2|, & \text{for } p>p_c
    \end{array}
    \right. ,
\\
  &|[\sigma^{-1}]^*(p,0)|:=|\sigma_2^{-1}\tilde{w}(p,0)|=\left\{
    \begin{array}{ll}
      \infty, &       \text{for } p<p_c\\
     |\sigma_2|^{-1}<|[\sigma^{-1}]^*(p)|<|\sigma_1|^{-1}, & \text{for } p>p_c
    \end{array}
    \right. ,\notag\\
\label{eq:Cond-SuperCond_Crit_Beh_pc}
  &|\sigma^*(p,0)|:=|\sigma_1w(p,0)|=\left\{
    \begin{array}{ll}
      0<|\sigma^*(p)|<\infty, &       \text{for } p<p_c\\
      \infty, & \text{for } p>p_c
    \end{array}
    \right. ,
\\
  &|[\sigma^{-1}]^*(p,0)|:=|\sigma_1^{-1}\tilde{m}(p,0)|=\left\{
    \begin{array}{ll}
      0<|[\sigma^{-1}]^*(p)|<\infty, &       \text{for } p<p_c\\
      0, & \text{for } p>p_c
    \end{array}
    \right. .\notag
  \end{align}
%

We will focus on the conductor/superconductor critical transition of
the effective conductivity $\sigma^*(p,h)=\sigma_1w(p,h)$ and the
conductor/insulator critical transition of the the effective
conductivity $\sigma^*(p,h)=\sigma_2\,m(p,h)$. It is clear from equations
\eqref{eq:mh_Stieltjes_rep} and 
\eqref{eq:Cond-Insul_Crit_Beh_pc}--\eqref{eq:Cond-SuperCond_Crit_Beh_pc},
that the corresponding results immediately generalize to
$[\sigma^{-1}]^*(p,h)=\sigma_2^{-1}\tilde{w}(p,h)$ and
$[\sigma^{-1}]^*(p,h)=\sigma_1^{-1}\tilde{m}(p,h)$, respectively, with $p\mapsto1-p$. 

The critical behavior of binary conductors is made more precise
through the definition of the following critical exponents. For
$h\in\mathbb{R}\cap\mathcal{U}$, as $h\to0$ the effective conductivity
$\sigma^*(p,h)=\sigma_2\,m(p,h)$ exhibits critical conductor/insulator behavior
near the percolation threshold $p_c$, $\sigma^*(p,0)\sim(p-p_c)^t$ as
$p\to p_c^+$, moreover at $p=p_c$,
$\sigma^*(p_c,h)\sim h^{1/\delta}$ as $h\to0$. We assume the existence of the
critical exponents $t$ and $\delta$, as well as $\gamma$, defined via a
conductive susceptibility $\chi(p,0):=\partial m(p,0)/\partial h\sim(p-p_c)^{-\gamma}$ as
$p\to p_c^+$. Furthermore, for $p>p_c$, we assume that there is a gap 
$\theta_\mu\sim(p-p_c)^\Delta$ in the support of $\mu$ around $h=0$ or $s=1$ which
collapses as $p\to p_c^+$, or that any spectrum in this region does not
affect power law behavior \cite{Golden:PRL-3935}. Therefore, for our
percolation models with $p>p_c$, the support of $\phi$ is contained in
the compact interval $[0,S(p)]\subset\subset\mathbb{R}^+$, where $S(p)\sim(p-p_c)^{-\Delta}$ as
$p\to p_c^+$. As the moments of $\phi$ become singular as $\theta_\mu\to0$ 
\eqref{eq:phi_moments}, we also assume that there exist critical
exponents $\gamma_n$ such that $\phi_n(p)\sim(p-p_c)^{-\gamma_n}$ as $p\to p_c^+$,
$n\geq0$. When $h\in\mathcal{U}$ such that $h_i\neq0$, we
assume the existence of critical exponents $t_r$, $\delta_r$, $t_i$ and
$\delta_i$ corresponding to $m_r(p,h):=\text{Re}(m(p,h))$ and
$m_i(p,h):=\text{Im}(m(p,h))$. The critical exponents, $\gamma_n$ and $\Delta$,
associated with the measure $\phi$ are independent of $h$ and are thus
unaffected. 

In summary:  
%
\begin{eqnarray}\label{eq:Crit_Exponents_mh}
  &m(p,0)\sim(p-p_c)^t,  &\text{as  } p \to p_c^+,\\
  &m_r(p,0)\sim(p-p_c)^{t_r},  &\text{as  } p \to p_c^+,\notag\\
  &m_i(p,0)\sim(p-p_c)^{t_i},  &\text{as  } p \to p_c^+,\notag\\
  &m(p_c,h)\sim h^{1/ \delta },  &\text{as } h \to 0, \notag\\
  &m_r(p_c,h)\sim h^{1/ \delta_r },  &\text{as } |h| \to 0, \notag\\
  &m_i(p_c,h)\sim h^{1/ \delta_i },  &\text{as } |h| \to 0, \notag\\
  &\chi(p,0)\sim(p-p_c)^{-\gamma},  &\text{as }  p\to p_c^+,\notag\\
  &\phi_n\sim(p-p_c)^{-\gamma_n},  &\text{as }  p\to p_c^+. \notag\\
  &\theta_\mu(p)\sim(p-p_c)^\Delta,  &\text{as }  p\to p_c^+,\notag\\
  &S(p)\sim(p-p_c)^{-\Delta},  &\text{as } p \to p_c^+.\notag
\end{eqnarray} 
%
We also assume the existence of critical exponents associated with the
left hand limit $p\to p_c^-$: $\gamma^\prime$, $\gamma^\prime_n$, and $\Delta^\prime$. The
conductivity critical exponent, $t$, is believed to be
\emph{universal} for lattices, depending only on dimension
\cite{Golden:PRL-3935}. The critical exponents $\gamma$, $\delta$, $\Delta$, and
$\gamma_n$ for transport are different from those defined in section
\ref{sec:Magnetic_Systems} for the Ising model.

For $h\in\mathbb{R}\cap\mathcal{U}$, as $h\to0$ the effective conductivity
$\sigma^*(p,h)=\sigma_1w(p,h)$ exhibits critical conductor/superconductor
behavior near $p_c$, $\sigma^*(p,0)\sim(p-p_c)^{-s}$ as $p\to p_c^-$, and at
$p=p_c$, $\sigma^*(p_c,h)\sim h^{-1/\dha}$ as $h\to0$, where the superconductor
critical exponent $s$ is not to be confused with the contrast
parameter. We assume the existence of the critical exponents $s$ and
$\dha$, as well as $\gh$, defined via a conductive susceptibility
$\hat{\chi}(p):=\partial w(p,0)/\partial h\sim(p-p_c)^{-\gh}$ as $p\to p_c^-$. Furthermore,
for $p<p_c$, we assume that there is a gap $\theta_\alpha\sim(p-p_c)^{\Dh^\prime}$ in the
support of $[-d\alpha(1-\lambda)]$ around $h=0$ or $s=1$ which collapses as
$p\to p_c^-$, so that the support of $\ph$ is contained in the compact
interval $[0,\hat{S}(p)]\subset\subset\mathbb{R}^+$, where
$\hat{S}(p)\sim(p-p_c)^{-\Delta}$ as $p\to p_c^+$. As the moments of $\ph$ become
singular as $\theta_\alpha\to0$ \eqref{eq:phi_hat_moments}, we also assume that
there exist critical exponents $\gh_n^\prime$ such that
$\ph_n(p)\sim(p-p_c)^{-\gh_n^\prime}$ as $p\to p_c^-$, $n\geq0$. When
$h\in\mathcal{U}$ such that $h_i\neq0$, we assume the existence of critical
exponents $s_r$, $s_i$, $\dha_r$, and $\dha_i$ corresponding to
$w_r(p,h):=\text{Re}(w(p,h))$ and $w_i(p,h):=\text{Im}(w(p,h))$. The
critical exponents, $\gamma_n^\prime$ and $\Delta^\prime$, associated with the measure
$\ph$ are independent of $h$ and are thus unaffected.

In summary:
%
\begin{eqnarray}\label{eq:Crit_Exponents_wh}
  &\hat{w}(p,0)\sim(p-p_c)^s,  &\text{as } p \to p_c^-,\\
  &\hat{w}_r(p,0)\sim(p-p_c)^{s_r}, &\text{as  } p \to p_c^-,\notag\\
  &\hat{w}_i(p,0)\sim(p-p_c)^{s_i}, &\text{as  } p \to p_c^-,\notag\\                     
  &\hat{w}(p_c,h)\sim h^{1/ \dha }, &\text{as } h \to 0, \notag\\
 &\hat{w}_r(p_c,h)\sim h^{1/ \dha_r },&\text{as } |h| \to 0, \notag\\
 &\hat{w}_i(p_c,h)\sim h^{1/ \dha_i }, &\text{as } |h| \to 0, \notag\\            
 &\hat{\chi}(p,0)\sim(p-p_c)^{-\gh^\prime}, &\text{as }  p\to p_c^-, \notag\\             
  &\ph_n\sim(p-p_c)^{-\gh_n^\prime}, &\text{as }  p\to p_c^-, \notag\\
  &\theta_\alpha(p)\sim(p-p_c)^{\Dh^\prime},  &\text{as }  p\to p_c^-,\notag\\
  &\hat{S}(p)\sim(p-p_c)^{-\Dh^\prime}, &\text{as } p \to p_c^-.\notag
\end{eqnarray} 
%
We also assume the existence of critical exponents associated with the
right hand limit $p\to p_c^+$: $\gh$, $\gh_n$, and $\Dh$. To be more precise,
when we assume the existence of a critical exponent, we assume the
existence, for example, of the following limit \cite{Baker-1990}:
% 
\begin{align}
  \gh^\prime:=\limsup_{p\to p_c^-}\left( \frac{-\ln \hat{\chi}(p,0)}{\ln(p-p_c)}  \right).
\end{align}
%

We now briefly discuss the gaps $\theta_\mu$ (for $p>p_c$)  and $\theta_\alpha$ (for
$p<p_c$). While, in the infinite volume limit, the spectra
actually extends all the way to $h=0$, the part close to $h=0$
corresponds to very large, but very rare connected regions of the
insulating (superconducting) phase, Lifshitz phenomenon, and is
believed to give exponentially small contributions to $\sigma^*$, and not
affect power law behavior \cite{Golden:PRL-3935}. Bruno
\cite{Bruno:PRSLA-353} has proven the existence of a spectral gap in
matrix/particle systems with polygonal inclusions, and studied how the
gap vanishes as the inclusions touch (like $p\to p_c$). In section
\ref{sec:Spectral_Gap} we proved the existence of spectral gaps for
finite lattice systems. Furthermore, in section
\ref{sec:Calc_Spec_Meas_Comp_Micro} we numerically demonstrated the
existence of spectral gaps for random resistor networks and correlated
their collapse with the percolation threshold. For the actual model we
expect behavior similar to the lattice case.
%
\subsection{Baker's Critical Theory for Transport in Binary Composite
  Media}
%
Baker's critical theory characterizes phase transitions via the
asymptotic behaviors of underlying Stieltjes functions, near a critical 
point. This powerfull method has been very successfull in the Ising
model, precisely characterizing spontaneous magnetization
\cite{Baker-1990}. We will show that this method has far reaching
utility in the characterization of phase transitions in transport,
exhibited by a wide variety of binary composite media.   
%
\begin{definition}  \label{def:stieltjes}
  A function f(z) is said to be a \emph{Stieltjes function} (or
  \emph{series of Stieltjes}), if 
  %
  \begin{align} \label{eq:stieltjes}
    f(z)=\int_0^\infty\frac{d\phi(u)}{1+uz}
    =\sum_{j =0}^\infty(-z)^j\int_0^\infty u^jd\phi(u)
    :=\sum_{j =0}^\infty(-z)^j\phi_j,
  \end{align}
  %
  where $\phi(u)$ is a bounded, non-decreasing function, taking on
  infinitely many values, and all the moments $\phi_j$ of $\phi$ are
  finite.  
\end{definition}
%
By hypothesis, for $p<p_c$ the measure $\ph$ is compactly supported,
hence bounded with bounded moments of all orders. Therefore the
function $\hat{g}(h):=\hat{g}(p,h)$ is a Stieltjes function for
$p<p_c$. For all $h\in\mathcal{U}$ the Stieltjes function $\hat{g}(p,h)$
is analytic and has a convergent series representation
\eqref{eq:stieltjes} for all $h\in\mathcal{U}$ such that
$|h|\hat{S}(p)<1$ \cite{Golden:PRL-3935,Golden:CMP-473}. Similarly for
$p>p_c$, $g(h):=g(p,h)$ is a Stieltjes function and is analytic
with a convergent series representation \eqref{eq:stieltjes} for all
$h\in\mathcal{U}$ such that $|h|S(p)<1$. The following theorem
characterizes Stieltjes functions \cite{Baker-1990}.  
% 
\begin{theorem} \label{thm:stieltjes_Characterization}
   Let $D(i,j)$ denote the following determinant
    \begin{align} \label{eq:Detf} 
     D(i,j) = \left|
                 \begin{matrix}
                   \phi_i&\phi_{i+1}&\cdots&\phi_{i+j}\\ 
                   \vdots&\vdots&\ddots&\vdots\\
                   \phi_{i+j}&\phi_{i+j+1}&\cdots&\phi_{i+2j}                            
                   \end{matrix}
              \right| ,    
   \end{align}
   where $D(i,j)=D(j,i)$. The $\phi_n$ form a series of Stieltjes if and
   only if $\mathcal{D}(i,j) \geq 0$ for all $i,j =0,1,2,\ldots$

 \end{theorem}
%
Baker's inequalities for the sequence $\gamma_n$ immediately follows from
Theorem \ref{thm:stieltjes_Characterization} and equations
\eqref{eq:Crit_Exponents_mh}--\eqref{eq:Crit_Exponents_wh}. Indeed,
$\phi_n\sim(p-p_c)^{-\gamma_n}$ and Theorem \ref{thm:stieltjes_Characterization}
with $i=n$ and $j=1$ imply, for $|p-p_c|\ll1$, that    
%
\begin{align}
  (p-p_c)^{-\gamma_n - \gamma_{n+2}}-(p-p_c)^{-2\gamma_{n+1}} &\geq  0
  \notag \\
%  
  \iff (p-p_c)^{-\gamma_n - \gamma_{n+2} + 2\gamma_{n+1} }&\geq1
  \notag \\
%  
  \iff-\gamma_n - \gamma_{n+2} + 2\gamma_{n+1} &\leq 0
  \notag\\
%
  \label{eq:CondBakerIneq_m}
  \iff   \gamma_{n+1}-2\gamma_n+\gamma_{n-1}&\geq  0.
\end{align}
% 
The sequence of inequalities \eqref{eq:CondBakerIneq_m} are
\emph{Baker's inequalities} for transport, corresponding to $m(p,h)$,
and they imply that the sequence $\gamma_n$ increases at least linearly
with $n$.  The symmetries in equations \eqref{eq:mh_Stieltjes_rep} and
\eqref{eq:Crit_Exponents_mh}--\eqref{eq:Crit_Exponents_wh} imply that
Baker's inequalities also hold for the sequences $\gamma_n^\prime$, $\gh_n$, and
$\gh_n^\prime$. 

The key results of this section are the two--parameter scaling
relations between the critical exponents in the conductor/insulator
system,
defined in equations \eqref{eq:Crit_Exponents_mh},
and that of the conductor/superconductor system.
defined in equations \eqref{eq:Crit_Exponents_wh}.
By equations  \eqref{eq:BM_measure_relationship_E} we know that the
measures $\mu$ and $\alpha$ are related, therefore the measures $\phi$ and $\ph$
are related. Moreover, by equations \eqref{eq:Herglotz_Energy_Rep_E}
and \eqref{eq:Herglotz_Energy_Rep_G} we know that $m(p,h)$ and
$w(p,h)$ are related, therefore the Stieltjes functions $g(p,h)$ and
$\hat{g}(p,h)$ are related. We therefore anticipate that these sets of
critical exponents 
%defined in equations \eqref{eq:Crit_Exponents_mh} and  
% \eqref{eq:Crit_Exponents_wh}
are also related. This is indeed the case, and the resultant
relationship between the insulation critical exponent $t$ and the
superconduction critical exponent $s$ is in agreement with the seminal
paper by A. L. Efros and B. I. Shklovskii \cite{Efros:PSSB-303}.
These results are summarized in Theorem \ref{thm:Crit_Theory_m_w}
below. 
%
% The key results of this section are the two--parameter scaling
% relations between the critical exponents defined in equations
% \eqref{eq:Crit_Exponents_mh}--\eqref{eq:Crit_Exponents_wh}. These
% results are summarized in Theorem \ref{thm:Crit_Theory_m_w} below.
%
%
\begin{theorem} \label{thm:Crit_Theory_m_w}
  Let $t$, $t_r$, $t_i$, $\delta$, $\delta_r$, $\delta_i$, $\gamma$, $\gamma_n$, $\Delta$, $\gamma_n^\prime$,
  and $\Delta^\prime$ be   defined as in equations \eqref{eq:Crit_Exponents_mh},
  and $s$, $s_r$, $s_i$, $\dha$, $\dha_r$, $\dha_i$, $\gh$, $\gh_n^\prime$,
  $\Dh^\prime$, $\gh_n$, and $\Dh$ be defined as in equations
  \eqref{eq:Crit_Exponents_wh}. Then the following scaling relations
  hold:
%  
  \begin{align*}   
   &1) \gamma_1=\gamma, \ \gamma_1^\prime=\gamma^\prime, \ \gh_1=\gh, \text{ and } \gh_1^\prime=\gh^\prime\\
   &2) \ \gamma_0^\prime=0, \ \gamma_0<0, \ \gamma_n^\prime>0 \text{ and } \gamma_n>0 \text{ for } n\geq1\\
   &3) \ \gh_n^\prime>0 \text{ for } n\geq0\\
   &4) \ \gamma_1=\gh_0 \text{ and } \Delta=\Dh\\
   &5) \ \gamma_1^\prime=\gh_0^\prime \text{ and } \Delta^\prime=\Dh^\prime \\
   &6) \ \gamma_n=\gamma+\Delta(n-1) \text{ for } n\geq1 \\
   &7) \ \gh_n^\prime=\gh_0^\prime+\Dh^\prime n=\gh+\Dh^\prime(n-1) \text{ for } n\geq0 \\
   &8) \ t=\Delta-\gamma \\
   &9) \ s=\gh_0^\prime=\gh-\Dh^\prime \\
   &10) \ \delta=\frac{\Delta}{\Delta-\gamma} \\
   &11) \ \dha\;^\prime=\frac{\Dh^\prime}{\gh_0^\prime}=\frac{\Dh^\prime}{\gh-\Dh^\prime} \\
   &12) \ t_r=t_i=t \\
   &13) \ s_r=s_i=s \\
   &14) \ \delta_r=\delta_i=\delta \text{ and } \dha_r=\dha_i=\dha \\
   &15) \text{ If } \Delta=\Delta^\prime \text{ and } \gamma=\gamma^\prime, \text{ then } t+s=\Delta \text{
     and }  \delta=1/(1-1/\dha\,^\prime)
  \end{align*}
%  
\end{theorem}
%

Theorem \ref{thm:Crit_Theory_m_w} will be proven via a sequence of
lemmas as we collect some important properties of $m(p,h)$, $g(p,h)$,
$w(p,h)$, and $\hat{g}(p,h)$, and how they are related. From equations
\eqref{eq:Herglotz_Energy_Rep_E} and \eqref{eq:Herglotz_Energy_Rep_G}
we have   
%
\begin{align}\label{eq:m_w_relation}
  m(p,h)=hw(p,h), \quad \forall \ p\in[0,1], \ h\in\mathcal{U}.
\end{align}
%
Using equations \eqref{eq:mh_Stieltjes_rep}, minor algebraic
manipulation in equation \eqref{eq:m_w_relation} implies that 
%
\begin{align}\label{eq:g_ghat_relation}
  g(p,h)+h\hat{g}(p,h)=1, \quad \forall \ p\in[0,1], \ h\in\mathcal{U}.
\end{align}
%
For all $h\in\mathcal{U}$ and $p\in[0,1]$, the functions $g(p,h)$ and
$\hat{g}(p,h)$ are analytic in $h$ \cite{Golden:CMP-473}, and
therefore have bounded $h$ derivatives of all orders
\cite{Rudin:87}. An inductive argument applied to equation
\eqref{eq:g_ghat_relation} yields  
%
\begin{align}\label{eq:Diff_g_ghat_relation}
  \frac{\partial^ng}{\partial h^n}+n\frac{\partial^{n-1}\hat{g}}{\partial h^{n-1}}+h\frac{\partial^n\hat{g}}{\partial h^n}=0,
  \quad \forall \ p\in[0,1], \ h\in\mathcal{U}, \ n\geq1.
\end{align}
%
In the complex quasi--static case, where $h\in\mathcal{U}$ such that
$h_i\neq0$, the complex representation of equation
\eqref{eq:Diff_g_ghat_relation} is        
%
\begin{align}\label{eq:Complex_Diff_g_ghat_relation}
  &\frac{\partial^ng_r}{\partial h^n}+n\frac{\partial^{n-1}\hat{g}_r}{\partial h^{n-1}}
  +h_r\frac{\partial^n\hat{g}_r}{\partial h^n}-h_i\frac{\partial^n\hat{g}_i}{\partial h^n}=0,
  \quad n\geq1, \\
%  
  &\frac{\partial^ng_i}{\partial h^n}+n\frac{\partial^{n-1}\hat{g}_i}{\partial h^{n-1}}
  +h_r\frac{\partial^n\hat{g}_i}{\partial h^n}+h_i\frac{\partial^n\hat{g}_r}{\partial h^n}=0,
  \quad n\geq1, \notag
\end{align}
%
which holds for all $p\in[0,1]$ and $h\in\mathcal{U}$, where we have set, for
$n\geq0$, 
%
\begin{align*}
  \frac{\partial^ng_r}{\partial h^n}:=\text{Re}\frac{\partial^ng}{\partial h^n}, \qquad
  \frac{\partial^ng_i}{\partial h^n}:=\text{Im}\frac{\partial^ng}{\partial h^n},
  \\
  \frac{\partial^n\hat{g}_r}{\partial h^n}:=\text{Re}\frac{\partial^n\hat{g}}{\partial h^n}, \qquad
  \frac{\partial^n\hat{g}_i}{\partial h^n}:=\text{Im}\frac{\partial^n\hat{g}}{\partial h^n}.
\end{align*}
Equations
\eqref{eq:m_w_relation}--\eqref{eq:Complex_Diff_g_ghat_relation} are
general formulas holding for two component stationary random media in
the lattice and continuum settings \cite{Golden:PRL-3935}.

The integral representation \eqref{eq:mh_Stieltjes_rep} of equations
\eqref{eq:g_ghat_relation}--\eqref{eq:Complex_Diff_g_ghat_relation}
follows from the formulas in the following lemma.
%---------------------------------------------------------------------------------
\begin{lemma}\label{lem:h_diff_commutation}  
  %
  For all $h\in\mathcal{U}$ and $p\in[0,1]$, the Stieltjes functions
  $g(p,h)$ and $\hat{g}(p,h)$ may be differentiated under the integral
  sign: 
  %
  \begin{align}\label{eq:Diff_g}
    %\frac{\partial^nF(s)}{\partial s^n}&=\frac{\partial^n}{\partial s^n}\int_0^1\frac{d\mu(\lambda)}{s-\lambda}
    %                 =(-1)^nn!\int_0^1\frac{d\mu(\lambda)}{(s-\lambda)^{n+1}}
    %    \iff\\
    \frac{\partial^ng(p,h)}{\partial h^n}&=\frac{\partial^n}{\partial h^n}\int_0^\infty\frac{d\phi(y)}{1+hy}
                     =(-1)^nn!\int_0^\infty\frac{y^nd\phi(y)}{(1+hy)^{n+1}}\sim\phi_n\,.
         \\
    %\frac{\partial^nG(s)}{\partial s^n}&=-\frac{\partial^n}{\partial s^n}\int_0^1\frac{d\alpha(1-\lambda)}{s-\lambda}
    %                 =(-1)^{n+1}n!\int_0^1\frac{d\alpha(1-\lambda)}{(s-\lambda)^{n+1}}
    %    \iff\notag\\
    \frac{\partial^n\hat{g}(p,h)}{\partial h^n}&=\frac{\partial^n}{\partial h^n}\int_0^\infty\frac{d\ph(y)}{1+hy}
                     =(-1)^nn!\int_0^\infty\frac{y^nd\ph(y)}{(1+hy)^{n+1}}\sim\ph_n,
           \notag           
  \end{align}
  %
  where $n\geq1$ and the asymptotics in equation \eqref{eq:Diff_g} hold when
  $0<|h|\ll1$ and $|p-p_c|\ll1$. Moreover, for all $h\in\mathcal{U}$ and
  $p\in[0,1]$, and all $j\leq n+1$,
  %
  \begin{align}\label{eq:Complex_Diff_g_bounds}
   \int_0^\infty\frac{y^{n+j}d\phi(y)}{|1+hy|^{2(n+1)}},
   \int_0^\infty\frac{y^{n+j}d\ph(y)}{|1+hy|^{2(n+1)}}<\infty.
  \end{align}
  %
\end{lemma}
%
\noindent \textbf{Proof}:
%
For general $p\in[0,1]$, the support of $\phi$ and $\ph$ are given by
$\Sigma_\phi:=[S_0(p),S(p)]$ and $\Sigma_{\ph}:=[\hat{S}_0(p),\hat{S}(p)]$,
respectively, and are defined in terms of the support of $\mu$ and $\alpha$,
respectively, below equation \eqref{eq:var_subs_Fs}. By hypothesis, the
extremum of these sets satisfy
$\lim_{p\to p_c^+}S_0(p)=\lim_{p\to p_c^-}\hat{S}_0(p)=0$ and
$\lim_{p\to p_c^+}S(p)=\lim_{p\to p_c^-}\hat{S}(p)=\infty$. For every
$h\in\mathcal{U}$, it is clear that there exist real, strictly positive
numbers $S_h$ and $\hat{S}_h$ such that 
%
\begin{align}\label{eq:S1_asymp}
  1\ll|h|S_h<\infty, \quad 1\ll|h|\hat{S}_h<\infty.
\end{align}
%
We show that the formulas in equation \eqref{eq:Diff_g} hold for all
$h\in\mathcal{U}$ and $n\in\mathbb{N}$ by showing that the function
$y^n/(1+hy)^{n+1}$ is a member of the sets $L^1(\phi)$ and $L^1(\ph)$ for
all $h\in\mathcal{U}$ and $n\in\mathbb{N}$ (see Theorem 2.27 in
\cite{Folland:95}).   

We first show that $y^n/(1+hy)^{n+1}\in L^1(\phi)$ for all $h\in\mathcal{U}$
and all $n\in\mathbb{N}$. Set $h\in\mathcal{U}$, $n\in\mathbb{N}$, and
$0\ll S_h<\infty$ satisfying \eqref{eq:S1_asymp}, and write
$\Sigma_\phi:=[S_0(p),S_h]\cup(S_h,S(P)]$. For all $p\in[0,1]$, equations
\eqref{eq:phi_moments_F(s)} and \eqref{eq:Cond-Insul_Crit_Beh_pc}
imply that $0<\lim_{h\to0}|m(p,h)|=1-\phi_0<1$, which implies that the mass
$\phi_0$ of $\phi$ is uniformly bounded for all $p\in[0,1]$. Therefore,
%
\begin{align}\label{eq:L1(phi)_bound_finite_set}
  \int_{S_0(p)}^{S_h}\left|\frac{y^n}{(1+hy)^{n+1}}\right|d\phi(y)\leq
  \frac{S_h^n\,\phi([S_0(p),S_h])}{|1+hS_0(p)|^{n+1}}<\infty,
\end{align}
%
where $\phi([S_0(p),S_h])$ is the \emph{bounded} $\phi$ measure of the set
$[S_0(p),S_h]$. If $S(p)<\infty$ we simply set $S_h\equiv S(p)$ and we are
done. Otherwise set $S(p)\equiv\infty$. In terms of the support of $\mu$, we have
$\lambda_1(p):=S(p)/(1+S(p))\equiv1$ and $\lambda_h:=S_h/(1+S_h)\gg0$. Equations
\eqref{eq:mh_Stieltjes_rep} and \eqref{eq:S1_asymp} imply
%
\begin{align}\label{eq:L1(phi)_bound_infinite_set}
   \int_{S_h}^{S(p)}\left|\frac{y^n}{(1+hy)^{n+1}}\right|d\phi(y)
      &\sim\frac{1}{|h|^{n+1}}\int_{S_h}^{S(p)}\frac{d\phi(y)}{y}
      =\frac{1}{|h|^{n+1}}\int_{S_h}^{S(p)}
                 \frac{1+y}{y}d\mu\left(\frac{y}{1+y}\right)\notag\\
      &=\frac{1}{|h|^{n+1}}\int_{\lambda_h}^{\lambda_1(p)}\frac{d\mu(\lambda)}{\lambda}
      \leq \frac{1}{|h|^{n+1}}\,\frac{\mu_0}{\lambda_h}<\infty
\end{align}
%
for all $h\in\mathcal{U}$ and all $n\in\mathbb{N}$, where
$\mu_0=1-p$. Therefore, $y^n/(1+hy)^{n+1}\in L^1(\phi)$ for all
$h\in\mathcal{U}$ and all $n\geq0$. 

We now show that $y^n/(1+hy)^{n+1}\in L^1(\ph)$ for all $h\in\mathcal{U}$
and all $n\geq0$. Set $h\in\mathcal{U}$ and $0\ll\hat{S}_h<\infty$ satisfying
\eqref{eq:S1_asymp}, and write
$\Sigma_{\ph}:=[\hat{S}_0(p),\hat{S}_h]\cup(\hat{S}_h,\hat{S}(P)]$. We have
%
\begin{align}\label{eq:L1(ph)_bound_finite_set}
  \int_{\hat{S}_0(p)}^{\hat{S}_h}\left|\frac{y^n}{(1+hy)^{n+1}}\right|d\phi(y)\leq
  \frac{\hat{S}_h^n\,\ph([\hat{S}_0(p),\hat{S}_h])}{|1+h\hat{S}_0(p)|^{n+1}}<\infty.
\end{align}
%
The boundedness of equation \eqref{eq:L1(ph)_bound_finite_set} follows
from equations \eqref{eq:var_subs_Fs}--\eqref{eq:mh_Stieltjes_rep},
showing that the $\ph$ measure of the interval
$[\hat{S}_0(p),\hat{S}_h]$ is bounded. More specifically, in terms of
the support of $\alpha$, we have
$\hat{S}_0(p)=(1-\hat{\lambda}_1(p))/\hat{\lambda}_1(p) 
\iff\hat{\lambda}_1(p)=1-\hat{S}_0(p)/(1+\hat{S}_0(p))$ and
$\hat{S}_h:=(1-\hat{\lambda}_h)/\hat{\lambda}_h\gg0
\iff\hat{\lambda}_h=1-\hat{S}_h/(1+\hat{S}_h)\ll1$. Then equation
\eqref{eq:var_subs_Fs}--\eqref{eq:mh_Stieltjes_rep} imply that   
%
\begin{align}\label{eq:ph_measure_of_finite_set_bound}
  \ph([\hat{S}_0(p),\hat{S}_h])&=\int_{\hat{S}_0(p)}^{\hat{S}_h}d\ph(y)
         =\int_{\hat{S}_0(p)}^{\hat{S}_h}(1+y)\left[-d\alpha\left(\frac{1}{1+y}\right)\right]
         =\int_{1-\hat{\lambda}_1(p)}^{1-\hat{\lambda}_h}\frac{[-d\alpha(1-\lambda)]}{1-\lambda}\notag\\
         &=\int_{\hat{\lambda}_h}^{\hat{\lambda}_1(p)}\frac{d\alpha(\lambda)}{1-\lambda}
         \leq\frac{\alpha_0}{1-\hat{\lambda}_h}<\infty,
\end{align}
%
where $\alpha_0=p$. If $\hat{S}(p)<\infty$ we simply set $\hat{S}_h\equiv S(p)$ and we
are done. Otherwise set $\hat{S}(p)\equiv\infty$, where
$\hat{S}(p)=(1-\hat{\lambda}_0(p))/\hat{\lambda}_0(p)
\iff\hat{\lambda}_0(p)=1-\hat{S}(p)/(1+\hat{S}(p))\equiv0$. Therefore, similar to
equation \eqref{eq:L1(phi)_bound_infinite_set}, we have   
%
\begin{align}\label{eq:L1(ph)_bound_infinite_set}
   \int_{\hat{S}_h}^{\hat{S}(p)}\left|\frac{y^n}{(1+hy)^{n+1}}\right|d\ph(y)
      %&\sim\frac{1}{|h|^{n+1}}\int_{\hat{S}_h}^{\hat{S}(p)}
      %           \frac{1+y}{y}d\alpha\left(\frac{1}{1+y}\right)
      %            \notag\\
      &\sim\frac{1}{|h|^{n+1}}\int_{1-\hat{\lambda}_h}^{1-\hat{\lambda}_0(p)}\frac{[-d\alpha(1-\lambda)]}{\lambda}        
      =\frac{1}{|h|^{n+1}}\int_{\hat{\lambda}_0(p)}^{\hat{\lambda}_h}\frac{d\alpha(\lambda)}{1-\lambda}
      \notag\\
      &\leq \frac{1}{|h|^{n+1}}\frac{\alpha_0}{1-\hat{\lambda}_h}<\infty.
\end{align}
%
Therefore, $y^n/(1+hy)^{n+1}\in L^1(\ph)$ for all $n\geq0$ and all
$h\in\mathcal{U}$.

The asymptotic behaviors in equation \eqref{eq:Diff_g} follow from
equations \eqref{eq:phi_moments}--\eqref{eq:phi_moments_F(s)}, Baker's
inequalities \eqref{eq:CondBakerIneq_m}, and equation
\eqref{eq:Diff_mh_Fs} ($g(p,h)=sF(p,s)$ and $\hat{g}(p,h)=-s\,G(p,s)$)
which implies that 
%
\begin{align}\label{eq:Diff_mh_Fs}
  \lim_{h\to0}\frac{\partial^ng(p,h)}{\partial h^n}
         =\sum_{j=0}^nc_j\lim_{s\to1}\frac{\partial^jF(p,s)}{\partial s^j}\sim\phi_n\,,\\
  \lim_{h\to0}\frac{\partial^n\hat{g}(p,h)}{\partial h^n}
         =\sum_{j=0}^nb_j\lim_{s\to1}\frac{\partial^jG(p,s)}{\partial s^j}\sim\ph_n\,,  \notag     
\end{align}
where $c_j,b_j\in\mathbb{Z}$.

The proof that the function $y^{n+j}/|1+hy|^{2(n+1)}$ is in the sets
$L^1(\phi)$ and $L^1(\ph)$. Is very similar to the proof of equation
\eqref{eq:Diff_g}. As before, splitting up the supports $\Sigma_\phi$ and
$\Sigma_{\ph}$ into unions bounded and unbounded intervals, the bounds
analogous to equations \eqref{eq:L1(phi)_bound_finite_set} and
\eqref{eq:L1(ph)_bound_finite_set} follow as before. Noting that
$(1+y)/y^i=(1-\lambda)^{i-1}/\lambda^i$ one may easily find the following bounds
using the same procedure as in equations
\eqref{eq:L1(phi)_bound_infinite_set} and
\eqref{eq:L1(ph)_bound_infinite_set}:
%
\begin{align}
  \int_0^\infty\frac{y^{n+j}d\phi(y)}{|1+hy|^{2(n+1)}}
     &\leq\frac{\mu_0}{|h|^{2(n+1)}}\,\frac{(1-\lambda_h)^{n+1-j}}{\lambda_h^{n+2-j}}\\
  \int_0^\infty\frac{y^{n+j}d\ph(y)}{|1+hy|^{2(n+1)}}
     &\leq\frac{\alpha_0}{|h|^{2(n+1)}}\,\frac{(\lambda_h)^{n+1-j}}{1-\lambda_h^{n+2-j}},  
\end{align}
%
for all $h\in\mathcal{U}$ and $j\leq n+1$ $\Box$.
%-----------------------------------------------------------------------

Equations \eqref{eq:Diff_g_ghat_relation} and \eqref{eq:Diff_g} imply
that   
%
\begin{align}\label{eq:Diff_g_ghat_relation_Integral}
  \int_0^{S(p)}\frac{y^nd\phi(y)}{(1+hy)^{n+1}}=\int_0^{\hat{S}(p)}\frac{y^{n-1}d\ph(y)}{(1+hy)^n} 
                                -h \int_0^{\hat{S}(p)}\frac{y^nd\ph(y)}{(1+hy)^{n+1}}
  \,,                              
\end{align}
%
which holds for all $n\geq1$, $p\in[0,1]$, and $h\in\mathcal{U}$. The
integral representations of equations
\eqref{eq:Complex_Diff_g_ghat_relation} may be obtained by equation
\eqref{eq:Diff_g} as follows:  
%
\begin{align}\label{eq:Complex_Diff_g}
  \frac{\partial^ng}{\partial h^n}
   &=(-1)^nn!\int_0^{S(p)}\frac{y^nd\phi(y)}{|1+hy|^{2(n+1)}}(1+\bar{h}y)^{n+1}\\
   &=(-1)^nn!\sum_{j=0}^{n+1}{n+1 \choose j}\bar{h}^j
                 \int_0^{S(p)}\frac{y^{n+j}d\phi(y)}{|1+hy|^{2(n+1)}}\,.
 \notag
\end{align}
%

In Lemma \ref{lem:h_diff_commutation} we showed that the function
$y^n/(1+hy)^{n+1}\in L^1(\phi),L^1(\ph)$ for all $n\geq0$, $h\in\mathcal{U}$, and
$p\in[0,1]$. Therefore, both sides of equation
\eqref{eq:Diff_g_ghat_relation_Integral} are bounded for all
$n\geq0$, $h\in\mathcal{U}$, and $p\in[0,1]$. However, on can easily see that the
bounds given in Lemma \ref{lem:h_diff_commutation} are violated as
$h\to0$. We now show that equation
\eqref{eq:Diff_g_ghat_relation_Integral} is defined as $h\to0$.

Let $h$ By the proof of Lemma \ref{lem:h_diff_commutation} we have
$y^i/(1+hy)^j\in L^1(\ph)$ for all $j>i$ and $h\in\mathcal{U}$. Indeed, the
boundedness of the $\ph$ integral of the modulus of this function over
the compact interval $[0,S_1]$, where $S_1<\infty$ is arbitrary and fixed,
directly follows from equation
\eqref{eq:L1(ph)_bound_finite_set}  
%
%----------------------------------------------------------------
\begin{lemma}\label{lem:nonzero_gamma1_etc}
  $\gamma_1=\gamma$, $\gamma_1^\prime=\gamma^\prime$, $\gh_1=\gh$, and $\gh_1^\prime=\gh^\prime$
\end{lemma}
%
\noindent \textbf{Proof}:
%
Set $0<p-p_c\ll1$, by equation \eqref{eq:mh_Stieltjes_rep}
$(g(h)=sF(s))$, \eqref{eq:phi_moments_F(s)}, and
\eqref{eq:Crit_Exponents_mh} we have    
%
\begin{align}\label{eq:gamma1_gamma}
  (p-p_c)^{-\gamma}\sim\chi(p,0)
          :=\frac{\partial m(p,0)}{\partial h}
          =\lim_{s\to1}\left(F(p,s)+\frac{\partial F(p,s)}{\partial s}\right)
          =-\phi_1\sim(p-p_c)^{-\gamma_1},
\end{align}
%
hence $\gamma_1=\gamma$. Similarly for $0<p_c-p\ll1$, we have $\gamma_1^\prime=\gamma^\prime$. By
equation \eqref{eq:gamma1_gamma}, and the symmetries between $m$ and
$w$ \eqref{eq:mh_Stieltjes_rep} and the critical exponent definitions 
\eqref{eq:Crit_Exponents_mh}--\eqref{eq:Crit_Exponents_wh}, we also
have that $\gh_1=\gh$ and $\gh_1^\prime=\gh^\prime$ $\Box$.   
%----------------------------------------------------------------

Equation \eqref{eq:m_w_relation} is consistent with, and provides a
link between equations \eqref{eq:Cond-Insul_Crit_Beh_pc} and
\eqref{eq:Cond-SuperCond_Crit_Beh_pc}. We will see that the
fundamental asymmetry  between $m(p,h)$ and $w(p,h)$ ($\gamma_0^\prime=0$ and
$\gh_0^\prime>0$), given by Theorem \ref{thm:Crit_Theory_m_w}.2-3, is a
direct and essential consequence of equation \eqref{eq:m_w_relation},
and has deep and far reaching implications.      
%
%----------------------------------------------------------------
\begin{lemma}\label{lem:zero_gamma0}
  %
  Let the sequences $\gamma_n$ and $\gamma_n^\prime$, $n\geq0$, be defined as in
  equation \eqref{eq:Crit_Exponents_mh}. Then
  %
  \begin{align*}
    &1) \quad \gamma_0^\prime=0, \ \gamma_0<0, \ \gamma_n^\prime>0,   \text{ and } \ \gamma_n>0 \
        \text{ for } \ n\geq1 \\
    &2) \quad 0<\lim_{h\to0}\langle\chi_1\vec{E}\cdot\vec{E}_0\rangle/E_0^2<1 \
         \text{ for all } \ p\in[0,1]
  \end{align*}
  %
\end{lemma}
%
\noindent \textbf{Proof}:
%
By equation \eqref{eq:Cond-SuperCond_Crit_Beh_pc} $|w(p,0)|$ is  
bounded for all $p<p_c$. Thus for all $p<p_c$, equations
\eqref{eq:phi_moments_F(s)}, \eqref{eq:Crit_Exponents_mh}, 
and \eqref{eq:m_w_relation} imply that
%
\begin{align*}
  0=\lim_{h\to0}hw(p,h)=\lim_{h\to0}m(p,h)=\lim_{s\to1}(1-F(p,s))=1-\phi_0(p)\sim1-(p_c-p)^{-\gamma_0^\prime},
\end{align*}
%
for $0<p_c-p\ll1$, which is consistent with equation
\eqref{eq:Cond-Insul_Crit_Beh_pc}. Therefore, $\gamma_0^\prime=0$ and $\phi$ is a
probability measure for all $p<p_c$. The strict positivity of the
$\gamma_n^\prime$, for $n\geq1$, follows from Baker's inequalities
\eqref{eq:CondBakerIneq_m}. Therefore, from equation
\eqref{eq:gamma1_gamma} we have that
%
\begin{align}\label{eq:div_phi1}
  \infty=\lim_{p\to p_c^-}\phi_1(p)=-\lim_{p\to p_c^-}\frac{\partial m(p,0)}{\partial h}.
\end{align}
%
For $p>p_c$, equations \eqref{eq:phi_moments_F(s)} and
\eqref{eq:Cond-Insul_Crit_Beh_pc} imply that
$0<\lim_{h\to0}|m(p,h)|=1-\phi_0<1$. Therefore, $(p-p_c)^{-\gamma_0}\sim\phi_0<1$ for
all $0<p-p_c\ll1$, hence $\gamma_0<0$. The strict positivity of $\gamma_1$ follows
from equation \eqref{eq:div_phi1}, and the strict positivity of the
$\gamma_n$ for $n\geq2$ follows from Baker's inequalities
\eqref{eq:CondBakerIneq_m}. Equation \eqref{eq:phi_energy_relations}
and the inequality $0<\lim_{h\to0}|m(p,h)|=1-\phi_0<1$ implies that
$0<\lim_{h\to0}\langle\chi_1\vec{E}\cdot\vec{E}_0\rangle/E_0^2<1$ for all $p\in[0,1]$ $\Box$.    
%    
%------------------------------------------------------
%
%-------------------------------------------------------
\begin{lemma}\label{lem:nonzero_gh_n}
  %
  Let the sequence $\gh_n^\prime$, $n\geq0$, be defined as in equation
  \eqref{eq:Crit_Exponents_wh}. Then
  %
  \begin{align*}
  &1) \quad \gh_n^\prime>0 \ \text{ for all } \ n\geq0\\
  &2) \quad \lim_{h\to0}\langle E_f^2\rangle=\infty \ \text{ for all } \ p>p_c.
  \end{align*}
  %
\end{lemma}
%
\noindent \textbf{Proof}:
%
By equation \eqref{eq:Cond-Insul_Crit_Beh_pc} we have
$0<\lim_{h\to0}|m(p,h)|<1$, for all $p>p_c$. Therefore equation
\eqref{eq:m_w_relation} implies that
$\lim_{h\to0}w(p,h)=\lim_{h\to0}m(p,h)/h=\infty$, for all $p>p_c$, which is
consistent with equation
\eqref{eq:Cond-SuperCond_Crit_Beh_pc}. More specifically, equations
\eqref{eq:Cond-Insul_Crit_Beh_pc} and \eqref{eq:m_w_relation} imply
that $0<\lim_{h\to0}|m(p,h)|=\lim_{h\to0}|hw(p,h)|:=L(p)<1$, where
$\lim_{p\to p_c^+}L(p)=0$. Therefore, by 
equation \eqref{eq:mh_Stieltjes_rep}, we have
%
\begin{align}\label{eq:Divergence_Rate_w(p,h)}
  &\lim_{h\to0}|hw(p,h)|=\lim_{h\to0}|h\hat{g}(p,h)|\in(0,1), 
                        \text{ for all } p>p_c, 
 \\
  &\lim_{h\to0}|hw(p,h)|=\lim_{h\to0}|h\hat{g}(p_c,h)|=0
         \text{ for all } p<p_c . \notag                                       
\end{align}
%
As will be shown below, equation \eqref{eq:Divergence_Rate_w(p,h)} has
very important consequences. By equations \eqref{eq:phi_hat_moments},
\eqref{eq:Cond-SuperCond_Crit_Beh_pc}, and
\eqref{eq:Crit_Exponents_wh} we have,
%
\begin{align*}
  \infty=\lim_{p\to p_c^-}\lim_{h\to0}w(p,h)
   =\lim_{p\to p_c^-}\lim_{s\to1}(1-G(p,s))
   =1+\lim_{p\to p_c^-}\ph_0(p)
   \sim1+\lim_{p\to p_c^-}(p_c-p)^{-\gh_0^\prime},
\end{align*}
%
hence $\gh_0^\prime>0$. Baker's inequalities \eqref{eq:CondBakerIneq_m}
then imply that $\gh_n^\prime>0$ for all $n\geq0$. Equation
\eqref{eq:phi_energy_relations} and  $\gh_0^\prime>0$ implies that 
%
\begin{align*}
  \lim_{h\to0}\langle E_f^2\rangle=\infty \quad \forall \ p>p_c, \quad \Box.
\end{align*}  
%   
%----------------------------------------------------------------------

The asymptotic behavior of $\hat{g}(p,h)$ in equation 
\eqref{eq:Diff_g} and Lemma \ref{lem:nonzero_gh_n} motivates the
following fundamental homogenization assumption of this section
\cite{Baker-1990}:   
%
\begin{remark}\label{rem:homogenization_w}
Near the critical point $(p,h)=(p_c,0)$, the asymptotic behavior of
the Stieltjes function $\hat{g}(p,h)$ is determined primarily by the
mass $\ph_0(p)$ of the measure $\ph$ and the rate of collapse of the
spectral gap $\theta_\alpha$.  
\end{remark}
%
\noindent By remark \ref{rem:homogenization_w}, and in light of Lemmas
\ref{lem:nonzero_gamma1_etc}--\ref{lem:nonzero_gh_n}, we make the
following changes in variables 
%
\begin{align}\label{eq:variable_change_w}
  &\qh:=y(p_c-p)^{\Dh^\prime}, \quad \hat{Q}(p):=\hat{S}(p)(p_c-p)^{\Dh^\prime},
      %\quad x:=h(p-p_c)^{\Dh^\prime},\\
      \quad d\hat{\pi}(\qh):=(p_c-p)^{\gh_0^\prime} \;d\ph(y),
  \\
  \label{eq:variable_change_m}
   &q:=y(p-p_c)^\Delta, \quad Q(p):=S(p)(p-p_c)^\Delta,
      %\quad x:=h(p-p_c)^{\Dh},\\
      \quad d\pi(q):=(p-p_c)^\gamma \,y\,d\ph(y), 
\end{align}
%
so that $\hat{Q}(p),Q(p)\sim1$ and the masses $\hat{\pi}_0$ and $\pi_0$ of
the measures $\hat{\pi}$ and $\pi$, respectively, satisfy $\hat{\pi}_0,\pi_0\sim1$
as $p\to p_c$. Equations
\eqref{eq:variable_change_w}--\eqref{eq:variable_change_m} define the   
following scaling functions $G_{n-1}(x)$, $\hat{G}_n(\xh)$, $\mathcal{G}_{n-1,j}(x)$,
and $\hat{\mathcal{G}}_{n,j}(\xh)$ as follows.

For $h\in\mathcal{U}\cap\mathbb{R}$, equation \eqref{eq:Diff_g} of Lemma
\ref{lem:h_diff_commutation} and equations
\eqref{eq:variable_change_w}--\eqref{eq:variable_change_m} imply, for 
$n\geq0$, that       
%
\begin{align}\label{eq:Scaling_fun_Def}
  &\frac{\partial^ng}{\partial h^n}\propto(p-p_c)^{-(\gamma+\Delta(n-1))} G_{n-1}(x),
      %&=\int_0^{S(p)}\frac{y^n d\phi(y)}{(1+hy)^{n+1}}
      %=(p_c-p)^{-(\gamma+\Delta n)}\int_0^{Q(p)}
      %   \frac{q^n d\pi(q)}{(1+xq)^{(n+1)}}\\
%     
&&
  \frac{\partial^n\hat{g}}{\partial h^n}\propto(p_c-p)^{-(\gh_0^\prime+\Dh^\prime n)} \hat{G}_n(\xh), 
      %&=\int_0^{\hat{S}(p)}\frac{y^nd\ph(y)}{(1+hy)^{n+1}}
      %=(p_c-p)^{-(\gh_0^\prime+\Dh^\prime n)}\int_0^{\hat{Q}(p)}
      %   \frac{\qh^nd\hat{\pi}(\qh)}{(1+xq)^{(n+1)}}\\
\\ 
  &G_{n-1}(x):=\int_0^{Q(p)}\frac{q^{n-1}d\pi(q)}{(1+xq)^{n+1}},
&&
  \hat{G}_n(\xh):=\int_0^{\hat{Q}(p)}\frac{\qh^nd\hat{\pi}(\qh)}{(1+\xh \qh)^{n+1}},
\notag\\  
  &x:=h(p-p_c)^{-\Delta}, \quad 0<p-p_c\ll1,
  &&
  \xh:=h(p_c-p)^{-\Dh^\prime}, \quad 0<p_c-p\ll1. \notag
\end{align}
%
Analogous formulas are defined for the opposite limits, involving
$\Dh$, $\gh_0$, $\Delta^\prime$, and $\gamma^\prime$. 

For $h\in\mathcal{U}$ such that $h_i\neq0$, we analogously define scaling
functions $\mathcal{R}_{n-1}(x)$, $\mathcal{I}_{n-1}(x)$,
$\hat{\mathcal{R}}_{n}(\xh)$, and $\hat{\mathcal{I}}_{n}(\xh)$ as
follows. Using equations \eqref{eq:Complex_Diff_g} (which follows from
Remark \ref{lem:h_diff_commutation}) and
\eqref{eq:variable_change_w}--\eqref{eq:variable_change_m} we have,
for $0<p-p_c\ll1$,  
%
\begin{align}\label{eq:Complex_Scaling_fun_Def}
\frac{\partial^ng}{\partial h^n}   
   &=(-1)^nn!\sum_{j=0}^{n+1}{n+1 \choose j}\bar{h}^j
                 \int_0^{S(p)}\frac{y^{n+j}d\phi(y)}{|1+hy|^{2(n+1)}}\\
   &:=(-1)^nn!\sum_{j=0}^{n+1}{n+1 \choose j}[\bar{x}(p-p_c)^\Delta]^j
                 (p-p_c)^{-(\gamma+\Delta(n-1+j))}\mathcal{G}_{n-1,j}(x)\notag\\
   &:=(-1)^nn!(p-p_c)^{-(\gamma+\Delta(n-1))}\mathcal{K}_{n-1}(x)\notag\\
   &:=(-1)^nn!(p-p_c)^{-(\gamma+\Delta(n-1))}
      \left[\mathcal{R}_{n-1}(x)+\I\,\mathcal{I}_{n-1}(x)\right],
   \text{ and similarly}, \notag\\
\frac{\partial^n\hat{g}}{\partial h^n}
     &:=(-1)^nn!(p-p_c)^{-(\gh_0+\Dh n)}
       \left[\hat{\mathcal{R}}_{n}(\xh)+\I\,\hat{\mathcal{I}}_{n}(\xh)\right],\notag
\end{align}
%
where $x$ and $\xh$ are defined in equation \eqref{eq:Scaling_fun_Def}
and  
%
\begin{align}\label{eq:Complex_Scaling_fun_Def_Integrals}
 &\mathcal{G}_{n-1,j}(x):=\int_0^{Q(p)}\frac{q^{n-1+j}d\pi(q)}{|1+xq|^{2(n+1)}}
 &&
 &\hat{\mathcal{G}}_{n,j}(\xh):=\int_0^{\hat{Q}(p)}\frac{\qh^{n+j}d\hat{\pi}(\qh)}{|1+\xh\qh|^{2(n+1)}}
 \\
 &\mathcal{K}_{n-1}(x):=\sum_{j=0}^{n+1}{n+1 \choose j}\bar{x}^j
                       \mathcal{G}_{n-1,j}(x)
 &&
 &\hat{\mathcal{K}}_n(\xh):=\sum_{j=0}^{n+1}{n+1 \choose j}\bar{\xh}^j
                       \hat{\mathcal{G}}_{n,j}(\xh)
 \notag\\
 &\mathcal{R}_{n-1}(x):=\text{Re}(\mathcal{K}_{n-1}(x)),
 &&
 &\hat{\mathcal{R}}_n(\xh):=\text{Re}(\hat{\mathcal{K}}_n(\xh)),
   \notag\\   
 &\mathcal{I}_{n-1}(\xh):=\text{Im}(\mathcal{K}_{n-1}(x)),
 &&
 &\hat{\mathcal{I}}_n(\xh):=\text{Im}(\hat{\mathcal{K}}_n(\xh)).
 \notag
\end{align}
%
Analogous formulas are defined for the opposite limit, $0<p_c-p\ll1$
involving $\Dh^\prime$, $\gh^\prime_0$, $\Delta^\prime$, and $\gamma^\prime$. 

From equations \eqref{eq:Herglotz_Inneq}--\eqref{eq:Herglotz_NonZero}
we have, for $h\in\mathcal{U}$, $p\in[0,1]$, and $n\geq0$,
%
\begin{align}\label{eq:Non-negative_Gn_Ghn}
   G_{n-1}(x)&>0, \qquad
  \mathcal{G}_{n-1,j}(x)>0,\\
%
  \hat{G}_n(\xh)&>0, \qquad
  \hat{\mathcal{G}}_{n,j}(\xh)>0. \notag
\end{align}
%
We asume that, for all $p<p_c$, $w(p,0)$ is analytic, and for all
$p>p_c$, $m(p,0)$ is analytic (or that $h$ derivatives of $g(p,h)$ and
$\hat{g}(p,h)$ of all orders are bounded at $h=0$ for $p>p_c$ and
$p<p_c$ respectively
(\cite{Golden:CMP-473,Golden:CMP-467,Golden:SIAM89} and refrences 
therein)). Therefore we have, for $n\geq0$,   
% 
\begin{align}\label{eq:Bounded_Gn_h}
  &\lim_{h\to0}G_{n-1}(x)<\infty, \qquad
  \lim_{h\to0}\mathcal{G}_{n-1,j}(x)<\infty,  \quad
  p>p_c,\\
%
  &\lim_{h\to0}\hat{G}_n(\xh)<\infty, \qquad
  \lim_{h\to0}\hat{\mathcal{G}}_{n,j}(\xh)<\infty,  \quad
  p<p_c. \notag
\end{align}
%
%-----------------------------------------------------------------------------------
 \begin{lemma}\label{lem:asymp_Scaling_funs_x_to_0_p>pc}
   Let $\hat{G}_n(\xh)$, $G_{n-1}(x)$, and the associated critical
   exponents be defined as in equation \eqref{eq:Scaling_fun_Def}, for
   $p>p_c$. Then  
   %
     \begin{itemize}
    \item[1)] $G_{n-1}(x)\sim1$ as $x\to0$ $(h\to0$ and $0<p-p_c\ll1)$ for all $n\geq1$
    \item[2)] $[\hat{G}_{n-1}(\xh)-\xh\hat{G}_n(\xh)]\sim1
      %\iff \xh^n\hat{G}_n(\xh)\sim\hat{G}_0(\xh)
      $ as $\xh\to0$ $(h\to0$ 
      and $0<p-p_c\ll1)$ for all $n\geq1$  
    \item[3)] $\gamma=\gh_0$  
    \item[4)] $\Delta=\Dh$    
     \end{itemize}
   %
 \end{lemma}
%
\noindent \textbf{Proof}:
%
Let $h\in\mathcal{U}\cap\mathbb{R}$, so that $g(p,h)$ and $\hat{g}(p,h)$ are
real analytic \cite{Golden:CMP-473}, and $p>p_c$ so that, by equation
\eqref{eq:Bounded_Gn_h}, all $h$ derivatives of $g(p,h)$ are bounded
for $h=0$. Therefore, equations
\eqref{eq:Diff_g_ghat_relation_Integral},    
\eqref{eq:Scaling_fun_Def}, and
\eqref{eq:Non-negative_Gn_Ghn}--\eqref{eq:Bounded_Gn_h} imply that,
for all $n\geq1$, $0<p-p_c\ll1$, and $0<h\ll1$,   
%
\begin{align}\label{eq:Matching_Condition_Gn_Gnhat_p>pc}
  (0,\infty)\ni(p-p_c)^{-(\gamma+\Delta(n-1))}G_{n-1}(x)
       =(p-p_c)^{-(\gh+\Dh(n-1))}[\hat{G}_{n-1}(\xh)-\xh\hat{G}_n(\xh)].
\end{align}
%
Equations \eqref{eq:Non-negative_Gn_Ghn}--\eqref{eq:Bounded_Gn_h}
imply that $G_{n-1}(x)\sim1$ as $x\to0$, for all $n\geq1$. Equation
\eqref{eq:Matching_Condition_Gn_Gnhat_p>pc} then implies that 
$[\hat{G}_{n-1}(\xh)-\xh\hat{G}_n(\xh)]\sim1$ as $\xh\to0$,
for all $n\geq1$, a competition in sign between two diverging
terms. %Or equivalently,  $\xh^n\hat{G}_n(\xh)\sim\hat{G}_0(\xh)$
%for all $n\geq1$.
Therefore, 
%
\begin{align}
  \gamma+\Delta(n-1)=\gh_0+\Dh(n-1), \quad n\geq1.
\end{align}
%
Which in turn, implies that $\gamma=\gh_0$ and $\Delta=\Dh$ $\Box$.
%
%-------------------------------------------------------
%
%-------------------------------------------------------
 \begin{lemma}\label{lem:asymp_Scaling_funs_x_to_0_p<pc}
   Let $\hat{G}_n(\xh)$, $G_{n-1}(x)$, and the associated critical
   exponents be defined as in equation \eqref{eq:Scaling_fun_Def}, for
   $p<p_c$. Then
   %
     \begin{itemize}
    \item[1)] $\hat{G}_{n-1}(\xh)\sim1$ as $\xh\to0$ $(h\to0$ and $0<p_c-p\ll1)$
      for all $n\geq1$ 
    \item[2)] $G_{n-1}(x)\sim1$ as $x\to0$ $(h\to0$ and $0<p_c-p\ll1$ for all $n\geq1$
    \item[3)] $\gamma^\prime=\gh_0^\prime$  
    \item[4)] $\Delta^\prime=\Dh^\prime$    
     \end{itemize}
   %
 \end{lemma}
%
\noindent \textbf{Proof}:
%
Let $h\in\mathcal{U}\cap\mathbb{R}$, so that $g(p,h)$ and $\hat{g}(p,h)$ are
real analytic \cite{Golden:CMP-473}. Moreover let $p<p_c$ so that, by
equation \eqref{eq:Bounded_Gn_h}, all $h$ derivatives of
$\hat{g}(p,h)$ are bounded for $h=0$. Thus by equation
\eqref{eq:Diff_g} we have   
%
\begin{align*}
  \lim_{h\to0}h \int_0^{S(p)}\frac{y^nd\ph(y)}{(1+hy)^{n+1}}=0.
\end{align*}
%
Therefore, equations \eqref{eq:Diff_g_ghat_relation_Integral}, 
\eqref{eq:Scaling_fun_Def}, and
\eqref{eq:Non-negative_Gn_Ghn}--\eqref{eq:Bounded_Gn_h} imply that,
for all $n\geq1$, $0<p_c-p\ll1$, and $0<h\ll1$,  
%
\begin{align}\label{eq:Matching_Condition_Gn_Gnhat_p<pc}
  (0,\infty)\ni(p_c-p)^{-(\gh^\prime+\Dh^\prime(n-1))}\hat{G}_{n-1}(\xh)
       \sim(p_c-p)^{-(\gamma^\prime+\Delta^\prime(n-1))}G_{n-1}(x).
\end{align}
%
Equations \eqref{eq:Non-negative_Gn_Ghn}--\eqref{eq:Bounded_Gn_h}
imply that $\hat{G}_{n-1}(\xh)\sim1$ as $\xh\to0$ for all $n\geq1$. Equation 
\eqref{eq:Matching_Condition_Gn_Gnhat_p<pc} then implies that
$G_{n-1}(x)\sim1$ as $x\to0$ for all $n\geq1$. Therefore, 
%
\begin{align*}
  \gamma^\prime+\Delta^\prime(n-1)=\gh_0^\prime+\Dh^\prime(n-1), \quad n\geq1.
\end{align*}
%
Which in turn, implies that $\gamma^\prime=\gh_0^\prime$ and $\Delta^\prime=\Dh^\prime$ $\Box$.
%
%-------------------------------------------------------
%
%-------------------------------------------------------
 \begin{lemma}\label{lem:Scaling_rel_t_s_gamman}
   Let $\hat{G}_n(\xh)$, $G_{n-1}(x)$, and the associated critical
   exponents be defined as in equation
   \eqref{eq:Scaling_fun_Def}. Then   
   %
     \begin{itemize}
    \item[1)] $\gamma_n= \gamma+\Delta(n-1)$ for all $n\geq1$ 
    \item[2)] $\gh_n^\prime=\gh_0^\prime+\Dh^\prime n=\gh^\prime+\Dh^\prime(n-1)$ for all $n\geq0$ 
    \item[3)] $t=\Delta-\gamma$
    \item[4)] $s=\gh_0^\prime=\gh^\prime-\Dh^\prime$  
     \end{itemize}
   %
 \end{lemma}
%
\noindent \textbf{Proof}:
%
Let $0<p-p_c\ll1$. By equations  \eqref{eq:Diff_g} of
Remark \ref{lem:h_diff_commutation}, \eqref{eq:Crit_Exponents_mh}, 
\eqref{eq:Scaling_fun_Def}, and Lemma
\ref{lem:asymp_Scaling_funs_x_to_0_p>pc} we have, for all $n\geq1$,
%
\begin{align*}
  (p-p_c)^{-\gamma_n}&\sim\phi_n
             \sim\lim_{h\to0}\frac{\partial^ng(p,h)}{\partial h^n}
             \sim(p-p_c)^{-(\gamma+\Delta(n-1))}\lim_{x\to0}G_{n-1}(x)\\
             &\sim(p-p_c)^{-(\gamma+\Delta(n-1))}.\notag 
\end{align*}
%
Therefore $\gamma_n=\gamma+\Delta(n-1)$ for all $n\geq1$, with constant gap
$\gamma_i-\gamma_{i-1}=\Delta$, which is consistent with the absence of multifractal
behavior for the bulk conductivity \cite{Stauffer-92}.

Let $0<p_c-p\ll1$. Similarly, by equations \eqref{eq:Diff_g} of
Remark \ref{lem:h_diff_commutation}, \eqref{eq:Crit_Exponents_wh}, 
\eqref{eq:Scaling_fun_Def}, and Lemma
\ref{lem:asymp_Scaling_funs_x_to_0_p<pc} and we have, for all $n\geq1$, 
%
\begin{align*}
  (p_c-p)^{-\gh_n}\sim\ph_n
             \sim\lim_{h\to0}\frac{\partial^n\hat{g}(p,h)}{\partial h^n}
             \propto(p_c-p)^{-(\gh_0^\prime+\Dh^\prime n)}\lim_{\xh\to0}\hat{G}_n(\xh)
             \sim(p_c-p)^{-(\gh_0^\prime+\Dh^\prime n)}. 
\end{align*}
%
Therefore, by Lemma \ref{lem:nonzero_gamma1_etc}, we have
$\gh_n=\gh_0^\prime+\Dh^\prime n=\gh^\prime+\Dh^\prime(n-1)$ for all $n\geq0$, with constant
gap $\gh^\prime_i-\gh^\prime_{i-1}=\Dh$, which is consistent with the absence of
multifractal behavior for the bulk conductivity \cite{Stauffer-92}.

Let $0<p-p_c\ll1$. Equations \eqref{eq:mh_Stieltjes_rep},
\eqref{eq:Crit_Exponents_mh}, \eqref{eq:g_ghat_relation},
\eqref{eq:Divergence_Rate_w(p,h)}, and \eqref{eq:Scaling_fun_Def} yield
%
\begin{align}\label{eq:t_calculation}
  (p-p_c)^t&\sim\lim_{h\to0}m(p,h)
        =1-\lim_{h\to0}g(p,h)
        =\lim_{h\to0}h\hat{g}(p,h)
        =(p-p_c)^{\Dh-\gh_0 }\lim_{\xh\to0}\xh\hat{G}_0(\xh)\notag\\
        &\sim(p-p_c)^{\Dh-\gh_0}.
\end{align}
%
Therefore, by Lemma \ref{lem:asymp_Scaling_funs_x_to_0_p>pc} we have
$t=\Dh-\gh_0=\Delta-\gamma$.

Let $0<p_c-p\ll1$. By equations \eqref{eq:mh_Stieltjes_rep},
\eqref{eq:Crit_Exponents_wh}, \eqref{eq:Scaling_fun_Def}, and Lemmas
\ref{lem:nonzero_gh_n} and \ref{lem:asymp_Scaling_funs_x_to_0_p<pc},
%
\begin{align*}
  (p_c-p)^{-s}\sim\lim_{h\to0}w(p,h)
           \sim\lim_{h\to0}\hat{g}(p,h)
           =(p_c-p)^{-\gh_0^\prime}\lim_{\xh\to0}\hat{G}_0(\xh)
           \sim(p_c-p)^{-\gh_0^\prime}. 
\end{align*}
%
Therefore, by Lemma \ref{lem:Scaling_rel_t_s_gamman}.2, we have
$s=\gh_0^\prime=\gh^\prime-\Dh^\prime$ $\Box$. 
%---------------------------------------------------------------------------
%
%-------------------------------------------------------
 \begin{lemma}\label{lem:G_ghat_asymp_x_to_infty}
   Let $\hat{G}_n(\xh)$, $G_{n-1}(x)$, and the associated critical
   exponents be defined as in equation \eqref{eq:Scaling_fun_Def}, for
   $p>p_c$ and $p<p_c$. Then for all $n\geq1$ 
   %
     \begin{itemize}
    \item[1)] $G_{n-1}(x)\sim[\hat{G}_{n-1}(\xh)-\xh\hat{G}_n(\xh)]\sim
      x^{-(\gamma+\Delta(n-1))/\Delta}$ as $x\to\infty$ $(p\to p_c^+$ and 
      $0<h\ll1)$
    \item[2)] $G_{n-1}(x)\sim[\hat{G}_{n-1}(\xh)-\xh\hat{G}_n(\xh)]\sim
      x^{-(\gamma^\prime+\Delta^\prime(n-1))/\Delta^\prime}$ as $x\to\infty$ $(p\to p_c^-$ and $0<h\ll1)$           
    \item[3)] $\delta=\Delta/(\Delta-\gamma)$
    \item[4)] $\dha\,^\prime=\Dh^\prime/\gh_0^\prime=\Dh^\prime/(\gh^\prime-\Dh^\prime)$  
     \end{itemize}
   %
 \end{lemma}
%
\noindent \textbf{Proof}:
%
Let $0<h\ll1$, so that $g(p,h)$ and $\hat{g}(p,h)$ are analytic for
all $p\in[0,1]$ \cite{Golden:CMP-473}. The analyticity of $g(p,h)$ and
$\hat{g}(p,h)$ implies that all orders of $h$ derivatives of these
functions are bounded as $p\to p_c$, from the left or the
right. Therefore, equation \eqref{eq:Matching_Condition_Gn_Gnhat_p>pc}
holds for $0<p-p_c\ll1$, and 
%
\begin{align}\label{eq:x_infty_p<pc}
  (0,\infty)\ni(p_c-p)^{-(\gamma^\prime+\Delta^\prime(n-1))}G_{n-1}(x)
       =(p_c-p)^{-(\gh^\prime+\Dh^\prime(n-1))}[\hat{G}_{n-1}(\xh)-\xh\hat{G}_n(\xh)]
\end{align}
%
holds for $0<p_c-p\ll1$. Moreover, in order to cancel the diverging $p$
dependent 
prefactors in equations \eqref{eq:Matching_Condition_Gn_Gnhat_p>pc}
and \eqref{eq:x_infty_p<pc} we must have, for all $n\geq1$,  
%
\begin{align}\label{eq:Asymp_Gn_Ghn_x_to_infty}
  &G_{n-1}(x)\sim x^{-(\gamma+\Delta(n-1))/\Delta}, %\quad
  &&
  &[\hat{G}_{n-1}(\xh)-\xh\hat{G}_n(\xh)]\sim\xh^{-(\gh+\Dh(n-1))/\Dh}, \quad
      \text{as } p\to p_c^+,
\\
  &G_{n-1}(x)\sim x^{-(\gamma^\prime+\Delta^\prime(n-1))/\Delta^\prime}, %\quad
  &&
  &[\hat{G}_{n-1}(\xh)-\xh\hat{G}_n(\xh)]\sim\xh^{-(\gh^\prime+\Dh^\prime(n-1))/\Dh^\prime}, \quad
      \text{as }   p\to p_c^-.    \notag
\end{align}
%
Lemma \ref{lem:G_ghat_asymp_x_to_infty}.1-2 follows from equation
\eqref{eq:Asymp_Gn_Ghn_x_to_infty} and Lemmas
\ref{lem:asymp_Scaling_funs_x_to_0_p>pc}--\ref{lem:asymp_Scaling_funs_x_to_0_p<pc}.

Now by equations \eqref{eq:mh_Stieltjes_rep}, \eqref{eq:Crit_Exponents_mh},
\eqref{eq:m_w_relation}, \eqref{eq:Scaling_fun_Def}, and
\eqref{eq:Asymp_Gn_Ghn_x_to_infty} for $n=1$, we have
%
\begin{align}
  h^{1/\delta}&\sim\lim_{p\to p_c^+}m(p,h)
      \sim\lim_{p\to p_c^+}h\hat{g}(p,h)
      =h\lim_{p\to p_c^+}(p-p_c)^{-\gh_0 }\hat{G}_0(\xh)\\
      &\sim h(p-p_c)^{-\gh_0 }h^{-\gh_0/\Dh}(p-p_c)^{-\Dh(-\gh_0/\Dh) }
      =h^{(\Dh-\gh_0)/\Dh}, \text{ as } h\to0.\notag
\end{align}
%
for $0<h\ll1$. Therefore by Lemma
\ref{lem:asymp_Scaling_funs_x_to_0_p<pc}, we have 
$\delta=\Dh/(\Dh-\gh_0)=\Delta/(\Delta-\gamma)$. Similarly by equations
\eqref{eq:mh_Stieltjes_rep}, \eqref{eq:Crit_Exponents_wh},
\eqref{eq:Scaling_fun_Def}, and \eqref{eq:Asymp_Gn_Ghn_x_to_infty}
for $n=1$, and Lemma \ref{lem:nonzero_gh_n} 
%
\begin{align}
   h^{-1/{\dha^\prime}}\sim\lim_{p\to p_c^-}w(p,h)
      \sim\lim_{p\to p_c^-}\hat{g}(p,h)
      =\lim_{p\to p_c^-}(p-p_c)^{-\gh_0^\prime}\hat{G}_0(\xh)      
      =h^{-\gh_0^\prime/\Dh\,^\prime},
\end{align}
%
for $0<h\ll1$. Therefore, by Lemma \ref{lem:Scaling_rel_t_s_gamman} we have 
$\dha\,^\prime=\Dh^\prime/\gh_0^\prime=\Dh^\prime/(\gh^\prime-\Dh^\prime)$ $\Box$. 
%-------------------------------------------------------
%
%-----------------------------------------------------------------------------------
 \begin{lemma}\label{lem:Complex_s_t}
   Let $h\in\mathcal{U}$ such that $h_i\neq0$, and $\hat{\mathcal{G}}_{n,j}(\xh)$,
   $\hat{\mathcal{R}}_n(\xh)$, $\hat{\mathcal{I}}_n(\xh)$, and the
   associated critical exponents be defined as in equations
   \eqref{eq:Complex_Scaling_fun_Def}--\eqref{eq:Complex_Scaling_fun_Def_Integrals} 
   for $p>p_c$ and $p<p_c$. Furthermore, let $s_r$, $s_i$, $t_r$, and
   $t_i$ be defined as in equations
   \eqref{eq:Crit_Exponents_mh}--\eqref{eq:Crit_Exponents_wh}. Then,       
   %
     \begin{itemize}
    \item[1)] $\hat{\mathcal{R}}_0(\xh)\sim\hat{\mathcal{I}}_0(\xh)\sim1$ as
      $\xh\to0$ $(h\to0$ and $0<p_c-p\ll1)$
    \item[2)]
      $\lim_{\xh\to0}[\xh_r\hat{\mathcal{R}}_0(\xh)-\xh_i\hat{\mathcal{I}}_0(\xh)]
      \sim\lim_{\xh\to0}[\xh_r\hat{\mathcal{I}}_0(\xh)+\xh_i\hat{\mathcal{R}}_0(\xh)]\sim1$
      for $0<p-p_c\ll1$  
    \item[3)] $s_r=s_i=\gh_0^\prime=s$ 
    \item[4)] $t_r=t_i=\Delta-\gamma=t$ 
     \end{itemize}
   %
 \end{lemma}
%
\noindent \textbf{Proof}:
%
Let $0<p_c-p\ll1$, $h\in\mathcal{U}$ such that $h_i\neq0$, and $0<|h|\ll1$. By
equation
\eqref{eq:Complex_Scaling_fun_Def}--\eqref{eq:Complex_Scaling_fun_Def_Integrals}, 
for $n=0$, we have  
%
\begin{align}
  \hat{g}(p,h)=\int_0^{\hat{S}(p)}\frac{d\ph(y)}{|1+hy|^2}
                +\bar{h}\int_0^{\hat{S}(p)}\frac{y\,d\ph(y)}{|1+hy|^2}
              =(p-p_c)^{-\gh_0^\prime}[\hat{\mathcal{G}}_{0,0}(\xh)
                +\bar{\xh}\hat{\mathcal{G}}_{0,1}(\xh)],
\end{align}
%
so that
%
\begin{align}\label{eq:Complex_ghat}
  \hat{g}_r&=(p_c-p)^{-\gh_0^\prime}\hat{\mathcal{R}}_0(\xh)
          =(p_c-p)^{-\gh_0^\prime}[\hat{\mathcal{G}}_{0,0}(\xh)
                +\xh_r\hat{\mathcal{G}}_{0,1}(\xh)]\\
  \hat{g}_i&=(p_c-p)^{-\gh_0^\prime}\hat{\mathcal{I}}_0(\xh)
          =-(p_c-p)^{-\gh_0^\prime}\xh_i\hat{\mathcal{G}}_{0,1}(\xh).
          \notag
\end{align}
%
Equations \eqref{eq:Non-negative_Gn_Ghn}--\eqref{eq:Bounded_Gn_h}
imply that $\hat{\mathcal{R}}_0(\xh)\sim\hat{\mathcal{I}}_0(\xh)\sim1$ as
$\xh\to0$ $(h\to0$ and $0<p_c-p\ll1)$. Therefore, equations
\eqref{eq:mh_Stieltjes_rep}, \eqref{eq:Crit_Exponents_wh},
\eqref{eq:Complex_ghat} and Lemma \ref{lem:nonzero_gh_n} imply that 
%
\begin{align}\label{eq:Complex_w_asymp}
  (p_c-p)^{-s_r}\sim w_r(p,0)
              \sim\hat{g}_r(p,0)
              \sim(p_c-p)^{-\gh_0^\prime}\lim_{\xh\to0}\hat{\mathcal{R}}_0(\xh)
              \sim(p_c-p)^{-\gh_0^\prime},\\
   (p_c-p)^{-s_i}\sim w_i(p,0)
              \sim\hat{g}_i(p,0)
              \sim(p_c-p)^{-\gh_0^\prime}\lim_{\xh\to0}\hat{\mathcal{I}}_0(\xh)
              \sim(p_c-p)^{-\gh_0^\prime}. \notag            
\end{align}
%
Equation \eqref{eq:Complex_w_asymp} and Lemma
\ref{lem:Scaling_rel_t_s_gamman} imply that $s_r=s_i=\gh_0^\prime=s$, which
generalizes the result involving $s$ in Lemma
\ref{lem:Scaling_rel_t_s_gamman}. It's worth noting that these scaling
relations are independent of the path of the limit $h\to0$. The above
proof does not preclude that either $s_r=0$ or $s_i=0$, but not
both. Although, this is not physically consistent
\cite{Efros:PSSB-303}.  

Let $0<p-p_c\ll1$ and $0<|h|\ll1$. Equation \eqref{eq:t_calculation} shows
that we have $m(p,0)=\lim_{h\to0}h\hat{g}(p,h)$. Therefore equation
\eqref{eq:Complex_ghat}, for $p>p_c$, implies that 
%
\begin{align}\label{eq:Complex_ghat_m}
  m_r(p,0)&\sim\lim_{h\to0}[h_r\hat{g}_r(p,h)-h_i\hat{g}_i(p,h)]
         =(p-p_c)^{\Dh-\gh_0}
           \lim_{\xh\to0}[\xh_r\hat{\mathcal{R}}_0(\xh)-\xh_i\hat{\mathcal{I}}_0(\xh)],
           \notag\\
  m_i(p,0)&\sim\lim_{h\to0}[h_r\hat{g}_i(p,h)+h_i\hat{g}_r(p,h)]
         =(p-p_c)^{\Dh-\gh_0}
            \lim_{\xh\to0}[\xh_r\hat{\mathcal{I}}_0(\xh)+\xh_i\hat{\mathcal{R}}_0(\xh)].
\end{align}
%
By equation \eqref{eq:Divergence_Rate_w(p,h)}, 
$\lim_{\xh\to0}[\xh_r\hat{\mathcal{R}}_0(\xh)-\xh_i\hat{\mathcal{I}}_0(\xh)]\sim
\lim_{\xh\to0}[\xh_r\hat{\mathcal{I}}_0(\xh)-\xh_i\hat{\mathcal{R}}_0(\xh)]\sim1$ 
for $0<p-p_c\ll1$. Therefore, equations \eqref{eq:Crit_Exponents_mh} and
\eqref{eq:Complex_ghat_m} imply that
%
\begin{align}\label{eq:Complex_m_asymp}
  (p-p_c)^{t_r}\sim m_r(p,0)\sim(p-p_c)^{\Dh-\gh_0}, \quad (p-p_c)^{t_i}\sim m_i(p,0)\sim(p-p_c)^{\Dh-\gh_0}
\end{align}
%
Equation \eqref{eq:Complex_m_asymp} and Lemmas
\ref{lem:asymp_Scaling_funs_x_to_0_p>pc} and
\ref{lem:Scaling_rel_t_s_gamman} imply that
$t_r=t_i=\Dh-\gh_0=\Delta-\gamma=t$, which generalizes the result involving $t$
in Lemma \ref{lem:Scaling_rel_t_s_gamman}. It's worth noting that
these scaling relations are independent of the path of the limit
$h\to0$. The above proof does not preclude that either $t_r=0$ or
$t_i=0$, but not both. Although, this is not physically consistent
\cite{Efros:PSSB-303} $\Box$.   
%-----------------------------------------------------------------------------------
%
%-----------------------------------------------------------------------------------
 \begin{lemma} \label{lem:Complex_delta}
   Let $h\in\mathcal{U}$ such that $h_i\neq0$, and $\hat{\mathcal{G}}_{n,j}(\xh)$,
   $\hat{\mathcal{R}}_n(\xh)$, $\hat{\mathcal{I}}_n(\xh)$, and the
   associated critical exponents be defined as in equations
   \eqref{eq:Complex_Scaling_fun_Def}--\eqref{eq:Complex_Scaling_fun_Def_Integrals} 
   for $p>p_c$ and $p<p_c$. Furthermore, let $\dha_r$, $\dha_i$, $\delta_r$, and
   $\delta_i$ be defined as in equations
   \eqref{eq:Crit_Exponents_mh}--\eqref{eq:Crit_Exponents_wh}. Then,       
   %
     \begin{itemize}
    \item[1)] $\hat{\mathcal{R}}_0(\xh)\sim\hat{\mathcal{I}}_0(\xh)
                                      \sim|\xh|^{-\gh_0^\prime/\Dh^\prime}$ as
             $\xh\to\infty$ $(p\to p_c^-$ and $0<|h|\ll1)$ 
    \item[2)]
      $[\xh_r\hat{\mathcal{R}}_0(\xh)-\xh_i\hat{\mathcal{I}}_0(\xh)]
      \sim[\xh_r\hat{\mathcal{I}}_0(\xh)+\xh_i\hat{\mathcal{R}}_0(\xh)]
      \sim|\xh|^{(\Dh-\gh_0)/\Dh}$ as $\xh\to\infty$    
    \item[3)] $\dha_r\,^\prime=\dha_i\,^\prime=\Dh^\prime/\gh_0^\prime=\dha$ 
    \item[4)] $\delta_r=\delta_i=\Delta/(\Delta-\gamma)=\delta$ 
     \end{itemize}
   %
 \end{lemma}
%
\noindent \textbf{Proof}:
%
Let $0<h\ll1$, so that $g(p,h)$ and $\hat{g}(p,h)$ are analytic for
all $p\in[0,1]$ \cite{Golden:CMP-473}. Equations \eqref{eq:mh_Stieltjes_rep},
\eqref{eq:Crit_Exponents_wh}, \eqref{eq:Complex_ghat} and Lemma
\ref{lem:nonzero_gh_n} imply that  
%
\begin{align}\label{eq:Complex_w_asymp_infty}
  |h|^{-1/\dha_r^\prime}\sim w_r(p_c,h)
              \sim\hat{g}_r(p_c,h)
              \sim\lim_{p\to p_c^-}(p_c-p)^{-\gh_0^\prime}\hat{\mathcal{R}}_0(\xh),
              \\
   |h|^{-1/\dha_i^\prime}\sim w_i(p_c,h)
              \sim\hat{g}_i(p_c,h)
              \sim\lim_{p\to p_c^-}(p_c-p)^{-\gh_0^\prime}\hat{\mathcal{I}}_0(\xh). \notag            
\end{align}
%
The analyticity of $g(p,h)$ and $\hat{g}(p,h)$ implies that they are
bounded for all $p\in[0,1]$. Therefore, in order to cancel the diverging
$p$ dependent prefactors in equations \eqref{eq:Complex_w_asymp_infty}, we
must have
$\hat{\mathcal{R}}_0(\xh)\sim\hat{\mathcal{I}}_0(\xh)\sim|x|^{-\gh_0^\prime/\Dh^\prime}$
as $\xh\to\infty$ $(p\to p_c^-$ and $0<h\ll1)$. Equation
\eqref{eq:Complex_w_asymp_infty} then implies that
%
\begin{align}\label{eq:Complex_}
  |h|^{-1/\dha_r^\prime}&\sim(p_c-p)^{-\gh_0^\prime}|h|^{-\gh_0^\prime/\Dh^\prime}(p_c-p)^{-\Dh^\prime(-\gh_0^\prime/\Dh^\prime)}
               =|h|^{-\gh_0^\prime/\Dh^\prime},\\
   |h|^{-1/\dha_i^\prime}&\sim|h|^{-\gh_0^\prime/\Dh^\prime}. \notag             
\end{align}
%
Therefore, by Lemma \ref{lem:G_ghat_asymp_x_to_infty},
$\dha_r\,^\prime=\dha_i\,^\prime=\Dh^\prime/\gh_0^\prime=\dha\,^\prime$. 

Let $0<h\ll1$, so that $g(p,h)$ and $\hat{g}(p,h)$ are analytic for
all $p\in[0,1]$ \cite{Golden:CMP-473}. Equations
\eqref{eq:mh_Stieltjes_rep} and \eqref{eq:m_w_relation} imply that
shows that $m(p_c,h)\sim\lim_{p\to p_c^+}h\hat{g}(p,h)$. Therefore 
equations \eqref{eq:Crit_Exponents_mh} and \eqref{eq:Complex_ghat}
implies that  
%
\begin{align}\label{eq:Complex_ghat_m_infty}
   &|h|^{1/\delta_r}\sim m_r(p_c,0)%&=\lim_{p\to p_c^+}[h_r\hat{g}_r(p,h)-h_i\hat{g}_i(p,h)]\\
         =(p-p_c)^{\Dh-\gh_0}
           \lim_{p\to p_c^+}[\xh_r\hat{\mathcal{R}}_0(\xh)-\xh_i\hat{\mathcal{I}}_0(\xh)],
           \\
  &|h|^{1/\delta_i}\sim m_i(p,0)%&=\lim_{p\to p_c^+}[h_r\hat{g}_i(p,h)+h_i\hat{g}_r(p,h)]\notag\\
         =(p-p_c)^{\Dh-\gh_0}
            \lim_{p\to p_c^+}[\xh_r\hat{\mathcal{I}}_0(\xh)+\xh_i\hat{\mathcal{R}}_0(\xh)].
            \notag
\end{align}
%
The analyticity of $g(p,h)$ and $\hat{g}(p,h)$ implies that they are
bounded for all $p\in[0,1]$. Therefore, in order to cancel the diverging
$p$ dependent prefactors in equations \eqref{eq:Complex_ghat_m_infty}, we
must have
$[\xh_r\hat{\mathcal{R}}_0(\xh)-\xh_i\hat{\mathcal{I}}_0(\xh)]
 \sim[\xh_r\hat{\mathcal{I}}_0(\xh)+\xh_i\hat{\mathcal{R}}_0(\xh)]\sim|x|^{(\Dh-\gh_0)/\Dh}$
as $\xh\to\infty$ $(p\to p_c^+$ and $0<h\ll1)$. Therefore equation
\eqref{eq:Complex_ghat_m_infty}, and Lemmas
\ref{lem:asymp_Scaling_funs_x_to_0_p>pc} and
\ref{lem:G_ghat_asymp_x_to_infty} imply that 
$\delta_r=\delta_i=\Dh/(\Dh-\gh_0)=\Delta/(\Delta-\gamma)=\delta$ $\Box$.
%
%----------------------------------------------------------------------------------
\begin{lemma}\label{lem:s_t}
  If $\Delta=\Delta^\prime$, $\gamma=\gamma^\prime$, $\Dh=\Dh^\prime$, and $\gh_0=\gh_0^\prime$. Then
  %
     \begin{itemize}
    \item[1)] $s+t=\Delta$  
    \item[2)] $\delta=1/(1-1/\dha\,^\prime)$    
     \end{itemize}
   %  
 \end{lemma}
%
\noindent \textbf{Proof}:
%
Assume that the spectral properties of the measures, $d\ph(y)$ and
$y\,d\phi(y)$, have the symmetry $\Delta=\Delta^\prime$, $\gamma=\gamma^\prime$, $\Dh=\Dh^\prime$, and
$\gh_0=\gh_0^\prime$. By Lemma \ref{lem:Scaling_rel_t_s_gamman} we have $t=\Delta-\gamma$
and $s=\gh_0^\prime$, and by Lemma \ref{lem:G_ghat_asymp_x_to_infty} we
have $\delta=\Delta/(\Delta-\gamma)$ and $\dha\,^\prime=\Dh^\prime/\gh_0^\prime$. Lemmas
\ref{lem:asymp_Scaling_funs_x_to_0_p>pc}--\ref{lem:asymp_Scaling_funs_x_to_0_p<pc}
show that $\gamma=\gh_0$, $\Delta=\Dh$, $\gamma^\prime=\gh_0^\prime$, and
$\Delta^\prime=\Dh^\prime$. Therefore,
%
\begin{align*}
  &s+t=\gh_0^\prime+\Delta-\gamma=\gh_0+\Delta-\gamma=\Delta\\
  &\delta=\Delta/(\Delta-\gamma)=1/(1-\gamma/\Delta)=1/(1-\gh_0/\Dh)=1/(1-\gh_0^\prime/\Dh^\prime)=1/(1-1/\dha\,^\prime)
  \quad \Box.
\end{align*}
%----------------------------------------------------------------------------------

In this section we derived the (two--parameter) scaling relations
regarding the conductor/insulator critical transition
\eqref{eq:Cond-Insul_Crit_Beh_pc} and that of the
conductor/superconductor critical transition 
\eqref{eq:Cond-SuperCond_Crit_Beh_pc}. Assuming that the measures,
$d\ph(y)$ and $y\,d\phi(y)$, have the spectral symmetry property $\Delta=\Delta^\prime$,
$\gamma=\gamma^\prime$, $\Dh=\Dh^\prime$, and $\gh_0=\gh_0^\prime$, we also showed that there
are (two--parameter) scaling relations between these two sets of
critical exponents. There is no apparent mathematical necessity for
this spectral symmetry. Although, the relation $s+t=\Delta$ is consistent
with equation (4) in \cite{Efros:PSSB-303}, $s=t(\delta-1)$, which was
derived under a physical scaling hypothesis and leads to the two
dimensional duality relation $s=t$
\cite{Bergman:SSP-147,Clerc:AP-191}, as $\delta=2$ for $d=2$
\cite{Efros:PSSB-303}. The scaling relations are independent of the
path of the limit $h\to0$. Although the behavior of the system, as a
function of $p\in[0,1]$, is highly dependent on the location of
$h\in\mathcal{U}$ and is governed by equation
\eqref{eq:Diff_g_ghat_relation} (or formally by equation
\eqref{eq:Diff_g_ghat_relation_Integral}) or equivalently by the  
system of coupled partial differential equations
\eqref{eq:Complex_Diff_g_ghat_relation}.  

We have shown how the symmetries between the 
integral representations of $\sigma^*$ and $[\sigma^{-1}]^*$ may be used to
generalize the results of this section in terms of
$[\sigma^{-1}]^*$. This beautiful mathematical framework, regarding the
geometric critical transitions of percolating binary composites, will
be extended further to our statistical mechanics description of
electrically/thermally driven critical transitions of binary
dielectrics and metal/dielectric composites in section
\ref{sec:StatMech_of_Composites}. In section \ref{sec:Measure_Equiv}
we discuss some of the more subtle measure theoretic details regarding
the underlying symmmetries between $m(p,h)$ and $w(p,z(h))$. We will
show that this leads to a generalization of a result
\cite{Day:JPCM-96} which characterizes the measure $\varrho$ found in
equation \eqref{eq:BM_measure_relationship_E}, in terms of the
symmetries between the measures $d\ph(y)$ and $y\,d\phi(y)$.   
%
\subsection{Measure Equivalences in Transport} \label{sec:Measure_Equiv}
%
In section \ref{sec:Crit_Behav_of_Transport} we derived scaling
relations describing the critical transitions exhibited by binary
conductors. These scaling relations were found using the
relationships \eqref{eq:m_w_relation} and \eqref{eq:g_ghat_relation}
between $m(h,p)$ and $w(h,p)$, and $g(h,p)$ and $\hat{g}(h,p)$,
respectively, linking the critical behavior of conductor/insulator and
conductor/superconductor systems. In this section, we further explore
the consequences of this link. 

Using equations \eqref{eq:mh_Stieltjes_rep} and
\eqref{eq:m_w_relation} we arived at 
%
\begin{align}\label{eq:g_ghat_relation_subsection}
  g(p,h)+h\hat{g}(p,h)=1, \quad \forall \ p\in[0,1], \ h\in\mathcal{U}.
\end{align}
%
Equation \eqref{eq:g_ghat_relation_subsection} contains a profound formula
relating $g(p,h)$ and $\hat{g}(p,h)$, and suggests a deep relationship
between the measures $\phi(p)$ and $\ph(p)$. More specifically, it
implies that the measures behave in a way which maintains the equality
in \eqref{eq:g_ghat_relation_subsection} for all $p\in[0,1]$ when $h\in\mathcal{U}$
is held fixed. Moreover, for fixed $p\in[0,1]$, the measure transforms,
$g(p,h)$ and $\hat{g}(p,h)$, also behave in a way which maintains the
equality in \eqref{eq:g_ghat_relation_subsection} for all $h\in\mathcal{U}$. This
section is devoted to the analysis of this equation. 

The integral representaion of equation \eqref{eq:g_ghat_relation_subsection}
follows from equations \eqref{eq:mh_Stieltjes_rep}, and is given by
%
\begin{align}\label{eq:g_ghat_relation_integral}
  \int_0^\infty\frac{d\phi(y)}{1+hy}+h\int_0^\infty\frac{d\ph(y)}{1+hy}=1.
\end{align}
%
We wish to reexpress equation \eqref{eq:g_ghat_relation_integral} in a
more suggestive form by adding and subtracting the quantity
$h\int_0^{S(p)}y\,d\phi(y)/(1+hy)$, which is permissible if the modulus of
this quantity is finite for all $p\in[0,1]$ and $h\in\mathcal{U}$. The
affirmation of this fact directly follows from the techniques of Lemma
\ref{lem:h_diff_commutation}. In particular, set $h\in\mathcal{U}$,
and $0\ll S_h<\infty$ satisfying \eqref{eq:S1_asymp}, and write
$\Sigma_\phi:=[S_0(p),S_h]\cup(S_h,S(P)]$. In Lemma 
\ref{lem:h_diff_commutation} we showed that the mass $\phi_0$ of the
positive measure $\phi$ is uniformly bounded for all $p\in[0,1]$. Therefore,
%
\begin{align}\label{eq:L1(y_phi)_bound_finite_set}
 \left| \int_{S_0(p)}^{S_h}\frac{y\,d\phi(y)}{1+hy}\right|\leq
  \frac{S_h\,\phi([S_0(p),S_h])}{|1+hS_0(p)|}<\infty,
\end{align}
%
Moreover, assuming that $S(p)=\infty$,  
%
\begin{align}
 \left|\int_{S_h}^{S(p)}\frac{y\,d\phi(y)}{1+hy}\right|
     \sim\frac{1}{|h|}\int_{S_h}^{S(p)} d\phi(y)
     =\frac{\phi([S_h,S(p)])}{|h|}<\infty.
\end{align}
%
Therefore, the modulus of the quantity $h\int_0^\infty y\,d\phi(y)/(1+hy)$ is
finite for all $p\in[0,1]$ and $h\in\mathcal{U}$, and we may add and
subtract it in equation \eqref{eq:g_ghat_relation_integral}, 
yeilding   
%
\begin{align}\label{eq:Phi_transform}
   1&=\int_0^\infty\frac{d\phi(y)}{1+hy}+h\int_0^\infty\frac{d\ph(y)}{1+hy}\\
    &=\left[\int_0^\infty\frac{d\phi(y)}{1+hy}+h\int_0^\infty\frac{y\,d\phi(y)}{1+hy}\right]
    +h\left[\int_0^\infty\frac{d\ph(y)}{1+hy}-\int_0^\infty\frac{y\,d\phi(y)}{1+hy}\right]
    \notag\\
    &=\phi_0+h\int_0^\infty\frac{d\Phi_0(y)}{1+hy},\quad d\Phi_0(y):=d\ph(y)-y\,d\phi(y).\notag
\end{align}
%
As $\hat{g}(p,h)$ is an analytic function of $h$ for all $p\in[0,1]$
when $h\in\mathcal{U}$ \cite{Golden:CMP-473}, the above argument leading
to equation \eqref{eq:Phi_transform} shows that the transform
$h\int_0^\infty d\Phi_0(y)/(1+hy)$ of the signed measure \cite{Rudin:87}
$\Phi_0(dy)$ is defined for all $p\in[0,1]$ and $h\in\mathcal{U}$, and is
given by the second term in the second line in equation
\eqref{eq:Phi_transform} \cite{Rudin:87}.  

Equations \eqref{eq:phi_moments_F(s)} and \eqref{eq:Phi_transform}
demonstrate that equation \eqref{eq:g_ghat_relation_integral} may be
reexpressed as  
%
\begin{align}\label{eq:n=0_measure_equivalence}
 h \int_0^\infty\frac{d\Phi_0(y)}{1+hy}\equiv1-\phi_0(p)\equiv m(p,0)=
         \begin{cases}
         0, & \text{for all } p<p_c \\
         O(1)\in(0,1), & \text{for all } p>p_c
         \end{cases}         
\end{align}
%
for all $h\in\mathcal{U}$. Equation \eqref{eq:n=0_measure_equivalence}
generalizes the result \eqref{eq:Divergence_Rate_w(p,h)}, gives a
alternate representation of $m(p,0)=\lim_{h\to0}hw(p,h)$, and shows that
the transform of the measure $\Phi_0$, $h\int_0^\infty d\Phi_0(y)/(1+hy)$, is
independent of $h$ for all $p\in[0,1]$. We may relate this
representation of $m(p,0)$ to the measure $\varrho$ found in equation
\eqref{eq:BM_measure_relationship_E} using equation
\eqref{eq:mh_Stieltjes_rep} and the identity $y=\lambda/(1-\lambda)\iff\lambda=y/(1+y)$:      
%
\begin{align}\label{eq:Measure_equivalence_rho}
  d\Phi_0(y)&:=d\ph(y)-y\,d\phi(y)
        =(y+1)\left(\left[-d\alpha\left(\frac{1}{y+1}\right)\right]
                    -y\,d\mu\left(\frac{y}{y+1}\right)
              \right)\\
        &=\frac{1}{(1-\lambda)^2}[-(1-\lambda)\,d\alpha(1-\lambda)-\lambda\,d\mu(\lambda)]
        =\frac{\lambda\,d\varrho(\lambda)}{(1-\lambda)^2}=y(1+y)\,d\varrho\left(\frac{y}{1+y}\right).\notag
\end{align}
%
We may therefore express equation \eqref{eq:n=0_measure_equivalence}
in terms of $\varrho(d\lambda)$ as follows: 
%
\begin{align}\label{eq:n=0_measure_equivalence_rho_transform}
   h\int_0^\infty\frac{d\Phi_0(y)}{1+hy}
      =h\int_0^\infty\frac{y(1+y)d\varrho(\frac{y}{1+y})}{1+hy}
      =\int_0^1\frac{\lambda\,d\varrho(\lambda)}{(1-\lambda)^2/h+\lambda(1-\lambda)}\,.
\end{align}
%
Equations 
\eqref{eq:n=0_measure_equivalence}--\eqref{eq:n=0_measure_equivalence_rho_transform}
are general formulas holding for two component stationary random media
in the lattice and continuum settings \cite{Golden:PRL-3935}. 
%
\begin{remark}\label{rem:varrho_condidtions}
  Define the transform $\mathcal{D}(p,h;\varrho)$ of the measure $\varrho$ as
  \begin{align}\label{eq:D_varrho}
    \mathcal{D}(p,h;\varrho)=\int_0^1\frac{\lambda\,d\varrho(\lambda)}{(1-\lambda)^2/h+\lambda(1-\lambda)}\,.
  \end{align}
  Equations
  \eqref{eq:n=0_measure_equivalence}--\eqref{eq:n=0_measure_equivalence_rho_transform}
  show that $\mathcal{D}(p,h;\varrho)$ satisfies the following properties:
  \begin{itemize}
  \item[(1)] $\mathcal{D}(p,h;\varrho)$ is independent of $h$ for
    all $p\in[0,1]$.    
  \item[(2)] $0<\mathcal{D}(p,h;\varrho)<1$ for all $p>p_c$.
  \item[(3)] $\mathcal{D}(p,h;\varrho)\equiv m(p,0)$ for all $p\in[0,1]$.   
  \end{itemize}  
\end{remark}
%
\noindent The following lemma is the key result of this section.
%
\begin{lemma}\label{lem:Measure_consistentcy_condition}
  Let $\mathcal{D}(p,h;\varrho)$ be defined as in equation
  \eqref{eq:D_varrho}, where $h\in\mathcal{U}$ and $p\in[0,1]$. If
  $\mathcal{D}(p,h;\varrho)$ satisfies the properties of Remark
  \ref{rem:varrho_condidtions}, then  
  %
\begin{align}\label{eq:Measure_consistentcy_condition}
 &\varrho(d\lambda)=W_0(p)\delta_0(d\lambda)+W_1(p)(1-\lambda)\delta_1(d\lambda),\\
  \quad W_0(p)=&\int_0^1\frac{d\alpha(\lambda)}{1-\lambda}-1, \quad
  W_1(p)=m(p,0)=1-\int_0^1\frac{d\mu(\lambda)}{1-\lambda}, \notag  
\end{align}
%
where $\delta_{\lambda_0}(d\lambda)$ is the Dirac measure centered at $\lambda_0$.
%
\end{lemma}
%
\noindent \textbf{Proof}:
%
Let $\mathcal{D}(p,h;\varrho)$, defined in equation \eqref{eq:D_varrho},
satisfy the properties of Remark \ref{rem:varrho_condidtions}. The
measure $\varrho$ is independent of $h$ \cite{Golden:CMP-473}. If the
measure $\varrho$ is over continuous spectrum \cite{Reed-1980} then
$\mathcal{D}(p,h;\varrho)$ depends on $h$, contradicting property
$(1)$. Therefore the measure $\varrho$ is defined over pure point spectrum
$\sigma_{pp}$ \cite{Reed-1980}. Moreover, in order for properties $(1)$ and
$(3)$ to be satisfied we must have $\sigma_{pp}\equiv\{0,1\}$, so that the measure
$\varrho$ is of the form   
%
\begin{align*}
  \varrho(d\lambda)=W_0(p,\lambda)\delta_0(d\lambda)+W_1(p,\lambda)\delta_1(d\lambda),
\end{align*}
%
where the $W_j(p,\lambda)$, $j=0,1$, are functions of the volume fraction
$p$ and $\lambda\in[0,1]$ which are to be determined. In view of the numerator
of the integrand in equation \eqref{eq:D_varrho}, we may assume that
the function $W_0(p,\lambda)=W_0(p,0):=W_0(p)\not\equiv0$ is independent of
$\lambda$. In order for property $(2)$ to be satisfied we must have
$W_1(p,\lambda)\sim1-\lambda$ as $\lambda\to1$ (any other power of $1-\lambda$ would contradict
property $(2)$). Therefore, with out loss of generality, we may set  
$W_1(p,\lambda)=(1-\lambda)W_1(p)$, $\,W_1(p)\not\equiv0$. Property $(3)$ then yields     
%
\begin{align}
  m(p,0)&=W_0(p)\lim_{\lambda\to0}\left[\frac{\lambda}{(1-\lambda)^2/h+\lambda(1-\lambda)}\right]
        +W_1(p)\lim_{\lambda\to1}\left[\frac{\lambda(1-\lambda)}{(1-\lambda)^2/h+\lambda(1-\lambda)}\right]
        \notag\\
        &=W_1(p).
\end{align}
%

We have shown that
%
\begin{align}\label{eq:varrho_except_W0}
  \varrho(d\lambda)=W_0(p)\delta_0(d\lambda)+m(p,0)(1-\lambda)\delta_1(d\lambda), \quad W_0(p)\not\equiv0.
\end{align}
%
Plugging equation \eqref{eq:varrho_except_W0} into 
\eqref{eq:BM_measure_relationship_E} $(\lambda d\mu(\lambda)=(1-\lambda)[-d\alpha(1-\lambda)] - \lambda d\varrho(\lambda))$,
we are able determine $W_0(p)$ by using the definition of $F(s)$
\eqref{eq:Fs_Integral} and equation \eqref{eq:Fs_relationships_G}
$(1-F(s)=(1-1/s)(1-G(s))\iff F(s)-(1-1/s)G(s)=1/s)$: 
%
\begin{align}\label{eq:find_varrho_W0}
  F(s)&=\int_0^1\frac{d\mu(\lambda)}{s-\lambda}=\int_0^1\frac{1-\lambda}{\lambda}\frac{[-d\alpha(1-\lambda)]}{s-\lambda}
                           -\int_0^1\frac{d\varrho(\lambda)}{s-\lambda}\notag\\
      &=-\left(1-\frac{1}{s}\right)\int_0^1\frac{[-d\alpha(1-\lambda)]}{s-\lambda}
         +\frac{1}{s}\int_0^1\frac{[-d\alpha(1-\lambda)]}{\lambda} -\int_0^1\frac{d\varrho(\lambda)}{s-\lambda}
       \notag\\
      &=\left(1-\frac{1}{s}\right)G(s)+\frac{1}{s}\int_0^1\frac{d\alpha(\lambda)}{1-\lambda}
         -\frac{W_0(p)}{s}-m(p,0)\lim_{s\to1}\frac{1-\lambda}{s-\lambda}, \quad
         \forall \ |s|>1 \Rightarrow\notag\\
      1&=\int_0^1\frac{d\alpha(\lambda)}{1-\lambda}-W_0(p),  
\end{align}
%
where we have used $(1-\lambda)/(\lambda(s-\lambda))=1/(s\lambda)-(1-1/s)/(s-\lambda)$.
The formulas \eqref{eq:varrho_except_W0}--\eqref{eq:find_varrho_W0}
imply equation \eqref{eq:Measure_consistentcy_condition} and suggest
that $\lambda=1$ is a removable singularity under $\mu$ \emph{and} $\alpha$ $\Box$.
%-----------------------------------------------------------------------

For completeness, we mention that the form
\eqref{eq:Measure_consistentcy_condition} of the measure $\varrho(d\lambda)$ is 
consistent with equation
\eqref{eq:Diff_g_ghat_relation_Integral}. Indeed, if we define the
following signed measure,
$d\Phi_{n-1}(y):=y^{n-1}d\Phi_0(y)$ for $n\geq1$, then by equation
\eqref{eq:Measure_equivalence_rho} the formula
\eqref{eq:Diff_g_ghat_relation_Integral} is equivalent to         
%
\begin{align}\label{eq:Diff_g_ghat_relation_Integral_equivalence}  
   0\equiv\int_0^\infty\frac{d\Phi_{n-1}(y)}{(1+hy)^{n+1}}
    =\int_0^1\frac{\left(\frac{\lambda}{1-\lambda}\right)^{n-1}\frac{\lambda\,d\varrho(\lambda)}{(1-\lambda)^2}}
             {(1+hy)^{n+1}}
    =\int_0^1\frac{\lambda^n\,d\varrho(\lambda)}{(1-\lambda+h\lambda)^{n+1}}\,.
\end{align}
%
By Lemma \ref{lem:h_diff_commutation}, the manipulations leading to
the left most equality in equaiton
\eqref{eq:Diff_g_ghat_relation_Integral_equivalence} are 
valid.  Moreover, for $n=1$, equation \eqref{eq:Complex_Diff_g}
implies that
%
\begin{align}\label{eq:Complex_IE} 
  0\equiv\int_0^\infty\frac{d\Phi_1(y)}{|1+hy|^4}=\int_0^1\frac{\frac{\lambda}{1-\lambda}\frac{\lambda\,d\varrho(\lambda)}{(1-\lambda)^2}}
             {|1+hy|^4}
    =\int_0^1\frac{\lambda^2(1-\lambda)\,d\varrho(\lambda)}{(1-\lambda+h\lambda)^4}\,.
\end{align}
%
for all $h\in\mathcal{U}$ such that $h_i\neq0$ and $p\in[0,1]$. By Lemma
\ref{lem:h_diff_commutation}, the manipulations leading to 
the left most equality in \eqref{eq:Complex_IE} are also
valid. Therefore, the form \eqref{eq:Measure_consistentcy_condition} of
$\varrho(d\lambda)$ satisfies the conditions required by equations
\eqref{eq:n=0_measure_equivalence} \emph{and} 
\eqref{eq:Diff_g_ghat_relation_Integral_equivalence}--\eqref{eq:Complex_IE}.

Equation \eqref{eq:Measure_consistentcy_condition} generalizes a
result found in \cite{Day:JPCM-96} regarding random resistor networks
(In \cite{Day:JPCM-96} $W_1(p,\lambda)\equiv0$ for all $p\in[0,1]$). Here we have
shown that there is also a delta function component of the measure
$\varrho(d\lambda)$ at $\lambda=1$ present when $p>p_c$. This result is independent of,
and provides a proof of our hypothesis: the existence of a gap in the
spectrum about $\lambda=1$ $(h=0)$ that collapses \emph{precisely} at
$p=p_c$. The qualitative idea is as follows. The operator $\chi\Gamma\chi$ always
has a large null space leading to an essential delta function at
$\lambda=0$ for all $p\in[0,1]$. Although as $p\to1$ the characteristic function
$\chi\to I$, the identity operator, and $\chi\Gamma\chi\to\Gamma$, and a delta function at
$\lambda=1$ must form (the projection operator $\Gamma$ only has spectrum in the
set $\{0,1\}$). The proof of Lemma
\eqref{lem:Measure_consistentcy_condition} verifies that the delta
function at $\lambda=1$ forms \emph{precisely} at the percolation threshold,
and gives a characterization of the phase transition.  
%
\subsection{Numerical Results}
\label{subsec:Num_Res_Crit_Behav_of_Transport}
\begin{itemize}
\item Numerical evaluation of the critical exponents for the RRN in
  1-d, 2-d, and 3-d.
\item Numerical Confirmation of the scaling relations.  
\end{itemize}
%
\section{Statistical Mechanics of Homogenization for Composite
  Materials}
\label{sec:StatMech_of_Composites}
%
\subsection{Introduction} \label{sec:Intro_Stat_Mech_Comp}
%
% In section \ref{sec:Background_TACM} we obtained an exact
% representation of the energy associated with an infinite binary
% dielectric medium, $\frac{1}{2}\langle\vec{D}\cdot\vec{E}\rangle$, in terms of the effective
% permittivity $\epsilon^*$. We also showed in section
% \ref{sec:Herglotz_Energy_Reps} that the intrinsic energy constraint on
% the system, $\langle\vec{D}\cdot\vec{E}_f\rangle=0$, leads to a detailed decomposition
% of the the system energy in terms of Herglotz functions involving the
% spectral measure $\mu$, and its associated measures $\alpha$, $\varphi$, and $\tau$
% \eqref{eq:Fs_Integral}, \eqref{eq:Gt_Integral},
% \eqref{eq:Es_Integral}, \eqref{eq:Ht_Integral}, respectively. These
% representations capture all the  complicated geometric interactions
% \emph{exactly}, conserve energy, and separate all system
% parameters. These are important features in a statistical mechanics
% model, as the system is modeled entirely by energetic contributions
% \cite{Bobbio-2000,Robertson-1993}. 

% These elegant representations of electric energy lie at the heart of
% our statistical mechanics model of binary dielectric systems. Well
% known analytic properties of $\epsilon^*$
% \cite{Golden:CMP-473,MILTON:2002:TC} are adopted by the electric
% component of the system Hamiltonian and the electric work term, both
% of which are derived here from first principles in physics. Once the
% electric work term is identified, the first law of thermodynamics
% \eqref{eq:Generalized_First_Law} for binary dielectric media and
% Maxwell's relations
% \eqref{eq:Helmholtz_Free_Energy_Maxwell's_Relations} provide detailed
% physically consistent results which have deep connections with
% orthogonal polynomial theory, logarithmic potential theory, and random
% matrix theory. The parameter separation property of The Analytic
% Continuation Method yields a statistical mechanics framework, both
% mathematically tractable and physically transparent, which parallels
% the Ising model. The parallels are so transparent that some results
% regarding binary dielectric media may be simply sited from well known
% results of the Ising model. We also demonstrate that the critical
% theory developed in section \ref{sec:Crit_Behav_of_Transport} may be
% extended to our statistical mechanics framework for binary dielectric
% media developed here.        

% By the linearity of the governing equations \cite{MILTON:2002:TC}, the
% Hamiltonian associated with a binary dielectric system is the
% sum of pure electric energies, pure elastic energies, pure thermal
% energies, and mutual coupling energies \cite{Robertson-1993}. In order
% to model binary dielectric media  using methods of statistical
% mechanics, we must decompose the system Hamiltonian into these various
% contributions and identify the associated work terms, to be inserted
% into the first law of thermodynamics \eqref{eq:FirstLaw}. When
% multiple energetic sources are considered which exhibit coupling, such
% as the electric/elastic phenomenon of electrostriction exhibited by
% all dielectrics, this decomposition can be highly nontrivial
% \cite{Bobbio-2000}. In this thesis we focus solely on electric aspects
% of the system and will therefore refer to the electric component of
% the Hamiltonian by simply, the Hamiltonian. 

% Here we consider a finite binary dielectric system
% $\mathcal{V}\subset\mathbb{R}^d$ with volume $V=|\mathcal{V}|$ satisfying
% $1\ll V<\infty$. From section \ref{sec:Background_TACM} we have
% \cite{Golden:CMP-473} that $\langle\vec{D}\cdot\vec{E}_f\rangle=o(V^{-a})$ and
% $\langle\vec{E}_f\rangle=o(V^{-b})$, as $V\to\infty$, for some positive constants
% $a,b>0$, where $\langle\cdot\rangle:=\frac{1}{V}\int_\mathcal{V}dV\cdot$ denotes spatial
% average over the region $\mathcal{V}$ and $dV$ is the Lebesgue measure
% restricted to $\mathcal{V}$. To simplify notation and to streamline
% the analysis we will henceforth drop the $o(V^{-a})$ and $o(V^{-b})$
% notation associated with setting these quantities to zero. The details 
% regarding these perturbations will be subsequently discussed.

% For this large but finite system, effective quantities
% depend weakly on boundary conditions, and on the geometric
% configuration $\omega\in\Omega$ through the spectral measure $\mu_\omega$,
% $G(t;\mu_\omega):=G_\omega(s):=1-\epsilon^*_\omega/\epsilon_1$ \eqref{eq:Gt_Integral}, where $t=1-s$
% and $\Omega$ denotes the space of all geometric configurations accessible
% to the micro-structure of the system with volume $V$. The space of
% geometric configurations is determined by the underlying reference
% measure $P(d\omega)$ defined in section \ref{sec:Background_TACM}. For
% example, in ER fluids $P(d\omega)$ characterizes the modified Poisson
% distribution of hard noninteracting spheres. We use the representation
% of energy involving $G_\omega(s)$ \eqref{eq:Gt_Integral} so that metal
% inclusions within a dielectric matrix may be obtained by letting
% $\epsilon_2\to\infty$ ($h\to0$ or $s\to1$) with $0<\epsilon_1<\infty$ held fixed
% \cite{Jackson-1999}. This representation is most natural, physically, as
% $G(s)<0$ implies that $\epsilon^*_\omega=\epsilon_1(1+|G_\omega(s)|):=\epsilon_1(1+\chi^*_\omega(s))$ and
% $\frac{1}{2}\langle\vec{D}\cdot\vec{E}\rangle=\frac{1}{2}\epsilon^*_\omega E_0^2$, to $o(V^{-a})$. Therefore,
% $|G_\omega(s)|:=\chi^*_\omega(s)$ plays the role of the electric susceptibility of
% a homogeneous dielectric system of permittivity $\epsilon^*_\omega$
% \cite{Jackson-1999,Griffiths-1999}.    
% %
% \subsection{A Derivation of the Hamiltonian and Electric Work Term}
% \label{sec:Ham_Work}
% %
% In this section, we follow the suggestive analysis given in section
% \ref{sec:Magnetic_Systems} to provide a physical derivation of the
% system Hamiltonian and electric work term associated with binary
% dielectric media \cite{Robertson-1993}. By decomposing the electric
% field into its volume average $\langle\vec{E}\rangle=\vec{E}_0$ and the
% fluctuation about this average $\vec{E}_f$,
% $\vec{E}=\vec{E}_0+\vec{E}_f$ with $\langle\vec{E}_f\rangle=0$, the energy density
% may be written as \eqref{eq:Reduced_Energy} 
% %
% \begin{align}
%   \label{eq:energy_partition}
%   \left\langle\frac{1}{2}\epsilon
%        \left(E_0^2+2\vec{E}_f\cdot\vec{E}_0+E_f^2\right)\right\rangle_V:=
%      \Wc_0+\Wc_{int}+\Wc_f
%                  =\Wc_0+\frac{1}{2}\Wc_{int},
% \end{align}
% %
% where we have used
% $\langle\vec{D}\cdot\vec{E}_f\rangle=\frac{1}{2}\Wc_{int}+\Wc_f=0$. Using
% the resolvent representation of the electric field
% \eqref{eq:Resolvent_Rep_Es}, recalling that $\langle\vec{E}_f\rangle=0$, and that
% $\epsilon=\epsilon_1(1-\chi_2/(1-s))$ we find 
% %
% \begin{align}
%   \label{eq:Physical_Summary}
%   \Wc_0+\frac{1}{2}\Wc_{int}
%   &=\frac{1}{2}\epsilon_1E_0^2
%         \left[1-\frac{p_2}{1-s}-\left(G_\omega(s)-\frac{p_2}{1-s}\right)
%        \right]
% \end{align}
% %
% where $p_2=\langle\chi_2\rangle$ is the volume fraction of type two material (see
% section \ref{sec:Resolv_Rep_E_D} for details). 

% Each term in equation \eqref{eq:energy_partition} must be analyzed in
% order to correctly obtain the Hamiltonian and the electric work term,
% to be inserted in the first law of thermodynamics \eqref{eq:FirstLaw}
% for binary dielectrics. By the linearity of Maxwell's equations, the
% Hamiltonian is the sum of Coulomb potential
% energy terms representing the potential energy of all charged
% particles associated with the system \cite{Robertson-1993}. These
% terms include charged particles both within the system and in the
% surroundings. 

% The mutual energies of the 
% particles within the system, in the absence of an external field, are
% macroscopically regarded as part of the internal energy
% \cite{Robertson-1993}. Furthermore, by macroscopic electric neutrality of
% the system, $\rho_f=\rho_b=0$ (see section \ref{sec:Extremal_Geometries} for
% details), the volume average of all electric fields generated by such
% charges must be zero. More specifically \cite{Bobbio-2000}, in the
% lack of an external field there still exists a nonzero microscopic
% field energy density. The question is, how is this energy accounted
% for in the macroscopic continuum description? This energy cannot be
% viewed as a mere shift in the zero level of internal energy, because
% it is dependent on inter-particle distances and is therefore, in 
% general, density and temperature dependent. Thus, this energy
% effectively contributes to the intrinsic energy of the system, and must
% be included in the thermodynamic internal energy of the system 
% \cite{Bobbio-2000}. The mutual energies of the external particles are
% not included in the Hamiltonian as these interactions are no more of
% interest than that of the heat bath in thermal systems. 

% The Hamiltonian includes all Coulomb terms associated with the energy
% stored within the system and the interaction energy of the external
% field with the dielectric body. The electric work done on the binary
% dielectric is part of the mutual energies of the external/internal 
% interactions. Where do these external/internal energies belong? The
% surprising answer is that the microscopic Hamiltonian places the
% entire energy within the system \cite{Robertson-1993}. Therefore the
% macroscopic treatment must also do so in order to be compatible
% \cite{Robertson-1993}.   

% The term $\Wc_0$ represents the mutual energy density of the
% external charges in the presence of a two component dielectric
% composite of laminates parallel to the applied field (see section
% \ref{sec:Extremal_Geometries}). The analysis done in section
% \ref{sec:Extremal_Geometries} shows that this interaction is
% independent of the surface charge distribution within the dielectric
% body and does not induce charge densities within the dielectric
% body. Therefore this term is not included in the Hamiltonian nor in
% the first law of thermodynamics.

% The term $\Wc_f$ is analogous to the pair-wise self
% interaction term of the Ising model. It represents the mutual
% self--energy density of the dielectric in its state of polarization,
% induced by the external electric field. Therefore this term is
% included in the system Hamiltonian.

% The term $\Wc_{int}$ represents the external charges
% interacting with, and polarizing, a homogeneous dielectric with
% ``susceptibility'' $G_\omega(s)-p_2/(1-s)$. The above discussion of the term
% $\Wc_0$ illuminates why the contribution $p_2/(1-s)$ is removed
% from this external/internal interaction term. One may be inclined to
% partition this interaction energy between the the system and
% surroundings, but the previous discussion has already indicated that it
% must be regarded as part of the system energy to be compatible with
% the microscopic description  \cite{Robertson-1993}. This interaction
% polarizes the dielectric body changing the characteristic energy
% levels of the system. Therefore, it is this term that is to be inserted
% into the first law of thermodynamics for binary dielectrics, and is to
% be included in the Hamiltonian.

% In summary, the system Hamiltonian is determined by
% $\Wc_{int}+\Wc_f=\frac{1}{2}\Wc_{int}$ and the electric work term,
% $W_{int}$, to be inserted into the first law for binary dielectrics is
% determined by $\Wc_{int}$. In order to make natural connections to
% physics, we make the following definition
% % 
% \begin{align}
%   \label{eq:physical_quantities}
%   P^*_\omega(E_0,N,s)=\frac{\epsilon_1}{N}\chi^*_\omega(s)E_0,
% \end{align}
% %
% were $N$ is the sphere number density. With this definition,
% $\chi^*_\omega:=|G_\omega|$ and $P^*_\omega$ play the role of the effective
% permeability and effective polarization of the binary dielectric
% system with configuration $\omega\in\Omega$, respectively. When the system
% inclusions consist of $N_p$ identical particles with single particle
% volume $V_p$ (e.g. spheres for the ER fluid), the volume fraction of
% particles is given by $p_2=N_pV_p/V$. Therefore, the number density
% $(N=N_p/V)$ may be expressed as $N=p_2/V_p$, introducing a length
% scale into the problem.

% With the definition \eqref{eq:physical_quantities}, the Hamiltonian,
% $\Hc_\omega$, and the associated electric work term, $\delta W^e$, up to a
% constant shift in the zero level, are given by \cite{Murphy_Stat_Mech}  
% %
% \begin{align}\label{eq:System_Hamiltonian}
%   \Hc_\omega&=-\frac{\epsilon_1 E_0^2}{2N} \chi^*_\omega(s)
% 	:= -\frac{1}{2}\mathcal{P}^*_\omega(E_0,N,s)E_0\\
%   \delta W^e&=-\langle \mathcal{P}^*_\omega(E_0,N,s)\rangle_{\Hc}\;dE_0
% 	:=-\mathcal{P}^*(T,E_0,N,s)\;dE_0\;.    
% \notag       
% \end{align}
% %
% where we have included the work done by the battery
% \cite{Griffiths-1999}, $-\mathcal{P}^*_\omega E_0$, and $\langle\cdot\rangle_\Hc$
% means Gibbs-Boltzmann canonical ensemble averaging. Therefore, the
% Helmholtz free energy of the system, including the constant shift in
% the zero level, is given by 
% % 
% \begin{align}
%   \Fc:=\frac{\epsilon_1E_0^2}{2N}\frac{p_2}{1-s} -\beta^{-1}\ln{Z}, \quad
%   Z:=\sum_{\omega\in\Omega}\exp(-\beta\Hc_\omega)
% \end{align}
% %
% and, by the parameter separation property of the Hamiltonian, we have  
% %
% \begin{align}
% 	P^*(T,E_0,N,s)&=-\frac{\partial \mathcal{F}}{\partial E_0}. \notag
% \end{align}
% %
 
% The analysis done in this section demonstrates the competition between
% the work done on the system and geometric affects. The energy
% constraint on the system,
% $\langle\vec{D}\cdot\vec{E}_f\rangle=\frac{1}{2}\Wc_{int}+\Wc_f=0$, demonstrates that
% \emph{exactly half} of the external/internal interaction energy is
% counteracted by the internal interactions of the system. Surprisingly,
% this constraint is independent of all parameter values. This suggests
% that, for very large systems, the charge density induced on the
% contrast boundaries by the applied field arranges itself in a way
% which approximately conserves the equality
% $\Wc_f+\frac{1}{2}\Wc_{int}=0$ for every configuration $\omega\in\Omega$ and every
% value of the external field strength and contrast parameter. This
% integral constraint seems to impose some sort of energy conservation
% on the system. This conservation is mathematically summarized by
% section \ref{sec:Herglotz_Energy_Reps}. 

% \subsection{The First Law and Stability}
% %
% Regardless of the physical arguments leading to
% \eqref{eq:System_Hamiltonian}, this equation and the first law
% \eqref{eq:FirstLaw} define a statistical mechanics model of binary
% dielectrics. In this section we explore the properties of the model,
% summarized in the following theorem. 
% %
% \begin{theorem} \label{thm:Stability_Consistency}	
% 	The electric component of the system Hamiltonian and the
%         associated electric work term, given in equation
%         \eqref{eq:System_Hamiltonian}, yield a statistical mechanics
%         model of the purely electric aspects of ER fluids with the
%         following properties.  
% %
% 	\begin{itemize}
% 		\item[(1)] In the infinite volume limit, $V\to\infty$, the
%                   model captures all complicated geometric
%                   interactions exactly.  
% 		\item[(2)] Electric components of state functions are
%                   determined by the Gibbs--Boltzmann statistics of
%                   Herglotz functions involving spectral measures
%                   $\{\mu_\omega\}_{\omega\in\Omega}$. For example the system entropy
%                   $S=S_0-|S^e|$, and the electric entropy $S^e$, which
%                   is zero for $E_0=0$, is determined by the variance
%                   of the ``effective susceptibility''
%                   $\chi^*_\omega(s)=|G_\omega(s)|$.  
%                 \item[(3)] Under the assumption that $S_0$ is bounded
%                   for some values of $\;T$, $N$, and $s$, the
%                   positivity of the entropy, $S>0$, is equivalent
%                   to $$0\leq\int_0^{E_0} dE_0^\prime (E_0^\prime)^3
%                   \emph{Var}(\chi^*_\omega(s))_\Hc<(2N^2kT^2/\epsilon_1^2)S_0.$$
%                   Therefore the Gibbs-Boltzmann variance of
%                   $\Hc^e_\omega$ vanishes as $E_0\to\infty$. Specifically,
%                   there exists a constant $\delta>0$ such that, for $E_0 \gg
%                   1$, $$\emph{Var}(\chi^*_\omega(s))_\Hc=o(E_0^{-(4+\delta)})$$
%                   and $$\emph{Var}(\Hc^e_\omega)_\Hc=o(E_0^{-\delta}).$$
%                   Furthermore, under the assumption that $S_0$ is
%                   bounded as $T\to0$, the Gibbs-Boltzmann variance of
%                   $\Hc^e_\omega$ vanishes as $T\to0$, for all
%                   $E_0>0$.  
% 		\item[(4)] The entropy decreases with $E_0$, $\partial S/\partial
%                   E_0<0$, a necessary condition for solidification in
%                   ER fluids.
%                   \item[(5)] Consistent with experiments, the
%                   Gibbs--Boltzmann ensemble averaged effective
%                   permittivity  $\langle\epsilon^*_\omega\rangle_\Hc$ increases with
%                   $E_0$, $$\partial \langle\epsilon^*_\omega\rangle_\Hc /\partial E_0>0,$$ and
%                   levels off as $E_0\to\infty$, $$\partial \langle\epsilon^*_\omega\rangle_\Hc /\partial
%                   E_0=O(E_0^{-(3+\delta)}).$$
%                 \item[(6)] The model is locally stable in $E_0$, $T$,
%                   and $s$.			 
% 		\item[(7)] If the signs in equation
%                   \eqref{eq:System_Hamiltonian} are reversed then the
%                   associated results follow from those stated above
%                   under the mapping $\epsilon_1\to-\epsilon_1$. The entropy is
%                   invariant under this mapping and property (3)
%                   continues to hold. Although, $\langle\epsilon^*_\omega\rangle_\Hc$
%                   decreases with $E_0$, inconsistent with experiments,
%                   and the system becomes unstable as $E_0\to\infty$. 
% 	\end{itemize}
% %
% \end{theorem}

% \noindent \textit{Proof of theorem \ref{thm:First_Law_Consequences}}. 
% By the discussion in section
% \ref{sec:The_Analytic_Continuation_Method}, in the infinite volume
% limit, the model captures all complicated geometric interactions
% \emph{exactly} through the electric component of the Hamiltonian
% \eqref{eq:System_Hamiltonian} (see the ergodic theorem in
% \cite{Golden:CMP-473} for details). Well known analytic properties
% \cite{Golden:CMP-473} of the electric work term
% \eqref{eq:System_Hamiltonian} and a Legendre transformation
% \cite{Robertson,Bobbio} of the first law \eqref{eq:FirstLaw} yield the
% differential of Helmholtz free energy for ER fluids
% \cite{Murphy_Stat_Mech}  
% %
% \begin{align}
%   \label{eq:Differential_of_Helmholtz_Free_Energy}
%     d\mathcal{F}&=-SdT-\mathcal{P}^*dE_0-\Psi ds-\Phi dN.
% \end{align}
% %
% The entropy $S$ and functions of state $\mathcal{P}^*$, $\Psi$, and $\Phi$,
% corresponding to independent state variables $E_0$, $s$, and $N$
% respectively, satisfy Maxwell's relations, including
% \cite{Murphy_Stat_Mech}
% % 
% \begin{align} \label{eq:Maxwells_Relations}
% 	\frac{\partial S}{\partial E_0}=\frac{\partial \mathcal{P}^*}{\partial T}\,,\quad\quad
% 	\frac{\partial \Psi}{\partial E_0}=\frac{\partial \mathcal{P}^*}{\partial s}\,,\quad\quad
% 	\frac{\partial \Phi}{\partial E_0}=\frac{\partial \mathcal{P}^*}{\partial N}\,.
% \end{align}
% %
% These Maxwell's relations decouple the electric component of all state
% functions. Furthermore, these electric components of state functions
% are determined by Gibbs--Boltzmann statistics of Herglotz functions
% involving spectral measures $\{\mu_\omega\}_{\omega\in\Omega}$ \cite{Murphy_Stat_Mech}. For
% example, the first of Maxwell's relations in equation
% \eqref{eq:Maxwells_Relations} and formula \eqref{eq:pdf_derivative}
% yield the following representation of the entropy associated with an
% ER fluid 
% %
% \begin{align}	\label{eq:Entropy_ER_Fluid}
%   S(T,E_0,N,s)-S_0(T,N,s)&\equiv S^e(T,E_0,N,s)\\
% 	&=\int_0^{E_0}dE_0^\prime\frac{\epsilon_1 E_0^\prime}{N}\frac{\partial }{\partial T}\langle \chi^*_\omega(s)\rangle_{\Hc}
% \notag\\
%     &=\int_0^{E_0}dE_0^\prime\frac{\epsilon_1 E_0^\prime}{N}\left(
%                     \frac{-\epsilon_1 (E_0^\prime)^2}{2NkT^2}
%                  \right)\text{Var}(\chi^*_\omega(s))_{\Hc},
% \notag 
% \end{align}
% %
% so that the ``electric entropy'' is non-positive $S^e(T,E_0,N,s)\leq0$,
% i.e. we have $S=S_0-|S^e|$.

% For $s\in(1+s_1,s_2)$, where $0<s_1,s_2<\infty$, the constraint
% $\epsilon_1<\epsilon^*_\omega<\epsilon_2$ implies that $\chi^*_\omega(s)$ is uniformly bounded for all
% $\omega\in\Omega$ \cite{Golden:CMP-473}.  Therefore,
% $\text{Var}(\chi^*_\omega(s))_{\Hc}$ and $|S^e|$ are bounded for all
% $0\leq E_0<\infty$. Assuming that $S_0$ is bounded for some values of $T$,
% $N$, and $s$, the positivity of the entropy \cite{Sethna,Firas},
% $S>0$, implies the integral in equation \eqref{eq:Entropy_ER_Fluid} is
% bounded for every $E_0>0$ and therefore converges in the large $E_0$
% limit
% % 
% \begin{align} \label{eq:Bounded_Integral}
% 0\leq\int_0^{E_0}dE_0^\prime (E_0^\prime)^3 \text{Var}(\chi^*_\omega(s))_{\Hc}< (2N^2kT^2/\epsilon_1^2)S_0(T,N,s)<\infty.
% \end{align}
% %
% This implies that the variance of the Hamiltonian vanishes in the
% large $E_0$ limit 
% %
% \begin{align}	\label{eq:VarG_Asymptotics}
% \text{Var}(\chi^*_\omega(s))_{\Hc}&=o\left(E_0^{-(4+\delta)}\right), \quad
% 	\text{Var}(\Hc_\omega^e)_{\Hc}=o(E_0^{-\delta}), \quad E_0\gg1 
% \end{align}
% %
% for some $\delta>0$. Furthermore, under the assumption that $S_0$ is also
% bounded as $T\to0$, equation \eqref{eq:Bounded_Integral} also shows that
% the Gibbs-Boltzmann variance of $\Hc^e_\omega$ vanishes as $T\to0$,
% for all $E_0>0$. Therefore if we identify $T$ with the absolute
% temperature and characterize the ER variation with the vanishing of
% $\text{Var}(\Hc_\omega^e)_{\Hc}$, then the model may also
% capture the temperature induced ER variation observed in
% \cite{Tao:APL-1844}. Using the second law of thermodynamics, discussed
% in section \ref{sec:The_2nd_Law_for_ER_fluids}, the result
% \eqref{eq:VarG_Asymptotics} can be physically understood from an
% entropic point of view.

% Equation \eqref{eq:Entropy_ER_Fluid} implies $\partial S/\partial E_0<0$, a
% necessary condition for solidification in ER fluids. Equation
% \eqref{eq:VarG_Asymptotics} implies the Gibbs--Boltzmann ensemble
% average of the effective permittivity,
% $\langle\epsilon^*_\omega\rangle_\Hc\equiv\epsilon_1(1+\langle\chi^*_\omega(s)\rangle_\Hc)$, increases with
% $E_0$ and levels off as $E_0\to\infty$,
% % 
% \begin{align}	\label{eq:increasing_Eff_permittivity}
% 	\left.\frac{\partial \langle\chi^*_\omega(s)\rangle_\Hc}{\partial E_0}\right|_{T}=
% 	\frac{\beta\epsilon_1 E_0}{N}\text{Var}(\chi^*_\omega(s))_\Hc=+O(E_0^{-(3+\delta)})\geq0,
% \end{align}
% %
% consistent with experiments \cite{Wen_etal}. By equation
% \eqref{eq:G_g_Substitution}, $\langle g_\omega(h)\rangle_\Hc$ also increases
% with $E_0$. This and equations \eqref{eq:VarG_Asymptotics} and
% \eqref{eq:Reduced_Energy} demonstrate that, as the critical parameter
% $\epsilon_1 E_0^2/2NkT\to\infty$, the system settles into non-varying energy states
% which minimize $-\frac{1}{2}\langle \vec{D}\cdot\vec{E}\rangle$. The rate at which
% this happens and the details of the phase transition that occurs as
% $E_0\to E_c$ are given by the Lee--Yang--Baker critical theory for
% transport, outlined in section \ref{sec:LYB_Theory_of_Transport} (see
% equation \eqref{eq:G_g_Equivalence_ER_Fluids}). 

% For simplicity we assume that $ds=dN=0$, as the dielectric contrast
% and sphere number density are fixed in an ER fluid. For thermodynamic
% stability, the Helmholtz free energy must be, locally, a concave
% function of its intensive parameters \cite{Robertson}. By the
% parameter separation property of the Hamiltonian
% \eqref{eq:System_Hamiltonian} and equations
% \eqref{eq:Differential_of_Helmholtz_Free_Energy} and
% \eqref{eq:pdf_derivative} we have
% %
% \begin{align}	\label{eq:Free_Energy_Derrivatives_E0}
% 	-\left.\frac{\partial \mathcal{F}}{\partial E_0}\right|_{T}
% 		&=\mathcal{P}^*(T,E_0,N,s)								   
% 		=\frac{\epsilon_1 E_0}{N} \langle\chi^*_\omega(s)\rangle_\Hc\geq0,\\
% 	-\left.\frac{\partial^2 \mathcal{F}}{\partial E_0^2}\right|_{T}
% 		&=\frac{\partial \mathcal{P}^*}{\partial E_0}
% 		=\frac{\epsilon_1}{N}\left(\langle\chi^*_\omega(s)\rangle_\Hc
% 		+ \frac{\beta\epsilon_1E_0^2}{N}\text{Var}(\chi^*_\omega(s))_\Hc\right)\geq0. 
% \notag
% \end{align}
% %
% Equation \eqref{eq:Free_Energy_Derivatives_E0} shows that our
% statistical mechanics model of ER fluids is locally stable in
% $E_0$. For $E_0\gg1$, the stability condition reduces to the classical
% condition $\langle\chi^*_\omega\rangle_\Hc\geq0$, the positivity of the effective
% susceptibility \cite{Robertson}.  

% Analogous to the Ising model \cite{Baker-1990}, the critical behavior
% of the average polarization $\mathcal{P}^*$, hence the phase
% transition in $E_0$, is determined by Lee--Yang--Baker critical theory
% for transport, outlined in section
% \ref{sec:LYB_Theory_of_Transport}. Assuming a Fubini theorem
% \cite{Reed-1980}, $\langle \chi^*_\omega(s)\rangle_\Hc=\langle |G_\omega(s)|\rangle_\Hc$ has the
% integral representation \eqref{eq:Gs_Integral} with measure $d\mu(\lambda)\equiv\langle
% d\mu_\omega(\lambda)\rangle_\Hc$, or equivalently by equation
% \eqref{eq:G_g_Substitution},  $\langle g_\omega(h)\rangle_\Hc$ has the integral
% representation \eqref{eq:Goldens_Stieltjes_Representation_of_m} with
% measure $d\eta(\lambda)\equiv\langle d\eta_\omega(y)\rangle_\Hc$ 
% %
% \begin{align}	\label{eq:G_g_Equivalence_ER_Fluids}
% 	\left\langle \int_0^1\frac{d\mu_\omega(\lambda)}{s-(1-\lambda)}\right\rangle_\Hc
% 		=\int_0^1 \frac{ \langle d\mu_\omega(\lambda) \rangle_\Hc}{s-(1-\lambda)}\; , 
% \quad
% 	\left\langle \int_0^1\frac{d\eta_\omega(y)}{1+hy}\right\rangle_\Hc
% 		=\int_0^1 \frac{ \langle d\eta_\omega(y) \rangle_\Hc}{1+hy}.
% \end{align}
% %

% % SEE THE FOLLOWING REFERENCES TO RELATE THE CRITICAL INDICES FOR
% % SPECIFIC HEAT etc:
% % J . PHYS. C ( P R O C . PHYS. S O C . ) , 1968, SER. 2, VOL. 1.
% % Critical behaviour of the Ising model above and below the
% % critical temperature J. W. ESSAMT and D. I,. HUNTERIS

% % PRL 1967 Vol 155 Num 2 Ising-Model Critical Indices below the Critical
% % Temperature G.A. BAKER


% By equation \eqref{eq:increasing_Eff_permittivity}, as $E_0$
% increases, the system will tend to configurations which maximize $\langle
% \chi^*_\omega(s)\rangle_\Hc=\langle |G_\omega(s)|\rangle_\Hc$ (or equivalently $\langle
% g_\omega(h)\rangle_\Hc$) i.e. the system will tend to percolating
% configurations with vanishing gaps in the spectral measure $\mu$ (or
% equivalently $\eta$). Therefore, by section
% \ref{sec:LYB_Theory_of_Transport}, these measures determine the
% critical behavior in ER fluids, under the identification of $p_c$ with
% $E_c$, and provide detailed information regarding phase transitions.

% Under the assumption that differentiation in $T$ commutes with $E_0$
% integration in \eqref{eq:Entropy_ER_Fluid}, the parameter separation
% property of the Hamiltonian \eqref{eq:System_Hamiltonian} and
% equations \eqref{eq:Differential_of_Helmholtz_Free_Energy} and
% \eqref{eq:pdf_derivative} imply that 
% %
% \begin{align} 	\label{eq:Free_Energy_Derrivatives_T}
% 	-\left.\frac{\partial \mathcal{F}}{\partial T}\right|_{E_0}
%         &=S(T,E_0,N,s)=S_0(T,N,s)+S_e(T,E_0,N,s)\geq0, 
% \end{align}
% %
% and
% \begin{align}     \label{eq:free_energy_concavity}
% %
% -\left.\frac{\partial^2 \mathcal{F}}{\partial T^2}\right|_{E_0}
% 	&=\frac{\partial S}{\partial T}
% 	=\frac{\partial S_0}{\partial T}
% 	+\int_0^{E_0} dE_0^\prime \left[\frac{\epsilon_1 E_0^\prime}{N}\right]
% 				  \left[ \frac{-\epsilon_1 (E_0^\prime)^2}{2NkT^2}\right]
% 				  \left[ \frac{-2}{T}\right] 
% 		\text{Var}(|G_\omega|)_\Hc\notag\\
% 		&+\int_0^{E_0} dE_0^\prime \left[\frac{\epsilon_1 E_0^\prime}{N}\right]
% 		\left[ \frac{-\epsilon_1 (E_0^\prime)^2}{2NkT^2}\right]^2
% 		\left\langle|G_\omega|(|G_\omega|-\langle|G_\omega| \rangle_\Hc)^2\right\rangle_\Hc 
% \notag\\
% %
% &=\frac{\partial S_0}{\partial T}
% 		+ \int_0^{E_0} dE_0^\prime \left[ \frac{\epsilon_1^2 (E_0^\prime)^3}{N^2kT^3}
% 		- \frac{\epsilon_1^3 (E_0^\prime)^5}{4N^3k^2T^4}\langle|G_\omega| \rangle_\Hc\right] 
% 			\text{Var}(|G_\omega|)_\Hc\notag\\
% 			&+\int_0^{E_0} dE_0^\prime \left[ \frac{\epsilon_1^3 (E_0^\prime)^5}{4N^3k^2T^4}\right]
% 		(\left\langle|G_\omega|^3\right\rangle_\Hc
% 		-\left\langle|G_\omega|^2\right\rangle_\Hc\langle|G_\omega|\rangle_\Hc)\geq0.  % \notag 
% \end{align}
% %
% When $E_0 = 0$ the stability condition reduces to $\partial S_0/\partial T \geq 0$,
% which implies the positivity of a heat capacity \cite{Robertson}. This
% and equation \eqref{eq:Free_Energy_Derivatives_T} implies that the
% system is locally stable in $T$ for all $E_0\geq0$.  A similar
% calculation shows \cite{Murphy_Stat_Mech} that the system is locally
% stable in $s$ for all $E_0\geq0$. Equations \eqref{eq:VarG_Asymptotics}
% and \eqref{eq:free_energy_concavity} demonstrate that, as $E_0\to\infty$,
% there are three successive regimes determining the local concavity of
% $\mathcal{F}$ in $T$. \emph{This might be used to predict the three
%   successive structural regimes in ER fluids, with increasing $E_0$,
%   clusters, chains, and columns \cite{Wen_etal}}.

% We have shown that the sign convention used in equation
% \eqref{eq:System_Hamiltonian} is sufficient to insure a physically
% consistent statistical mechanics model for ER fluids which is locally
% stable in all state variables. We now show that this convention is
% also necessary for physical consistency and local stability in $E_0$,
% and possibly in $T$. If the signs in equation
% \eqref{eq:System_Hamiltonian} are both changed, all results follow
% from previous results under the mapping $\epsilon_1\mapsto-\epsilon_1$. The entropy
% \eqref{eq:Entropy_ER_Fluid} is invariant under this mapping, therefore
% equation \eqref{eq:VarG_Asymptotics} still holds and the entropy
% decreases with $E_0$ and levels off as $E_0\to\infty$. Equation
% \eqref{eq:Free_Energy_Derivatives_E0} is not invariant under this
% mapping. Therefore the effective dielectric constant decreases with
% $E_0$, inconsistent with experiments \cite{Wen_etal}. Equation
% \eqref{eq:increasing_Eff_permittivity} is not invariant. This and
% equation \eqref{eq:VarG_Asymptotics} implies that, for $E_0\gg1$, the
% system is locally unstable in $E_0$. Therefore the sign convention
% used in equation \eqref{eq:System_Hamiltonian} is both necessary and
% sufficient for the physical consistency of the statistical mechanics
% model and the local stability in $E_0$. Equation
% \eqref{eq:free_energy_concavity} is not invariant and the local
% stability of the system in $T$ is determined by competition of
% integral magnitudes. Equation  \eqref{eq:VarG_Asymptotics} suggests
% that, as $E_0\to\infty$, the system may become locally unstable in $T$
% although a more detailed analysis must be performed to be
% certain. $\blacksquare$   


% \vspace{8ex}

% The first law of thermodynamics for ER fluids
% \eqref{eq:Differential_of_Helmholtz_Free_Energy} is a fundamental tool
% in statistical mechanics, leading to invaluable information regarding
% state functions and phase transitions. The second law of
% thermodynamics is also a fundamental tool in statistical mechanics,
% giving invaluable information regarding phase transitions and
% equilibrium configurations of the system. We conclude this article
% with a brief discussion of the second law of thermodynamics for ER
% fluids.  
% %

% \subsection{The Second Law of Thermodynamics for ER Fluids}	\label{sec:The_2nd_Law_for_ER_fluids} %%%%%%%%%%%%%%%%%%%%%%%%%%%%%%%%%%%%%%%%%%%%%%%%%%%%%%%%%%%%
% %

% For a given system volume V, let $\mathcal{M}(\mu_V)$ be the space of
% all spectral measures $\{\mu_\omega\}_{\omega\in\Omega}$ corresponding to geometric
% configurations $\omega$ accessible to the microstructure. These geometric
% configurations are determined by the reference measure $P(d\omega)$ defined
% in section \ref{sec:The_Analytic_Continuation_Method}. Theorem
% \ref{thm:First_Law_Consequences} demonstrates that all electric
% components of extensive internal variables depend on
% $\{\mu_\omega\}_{\omega\in\Omega}$. The following is an analogue for the principle of
% minimum Helmholtz free energy given in \cite{Firas}, determining phase
% transitions and the equilibrium configurations of the system (see also
% \eqref{eq:EnergyConservation}) 
% %
% \begin{align}	\label{eq:Helmholtz_Variational_Form}
% \mathcal{F}_\infty\equiv\lim_{V\to\infty}(-\beta V)^{-1}\ln{Z}
%   =\inf_{\mu\in\mathcal{M}(\mu_\infty)}\{-S[\mu] \;T - \frac{1}{2}\mathcal{P}^*[\mu] \;E_0\}
% \end{align}   
% %
% where $S[\mu_\omega]=S(T,E_0,N,s;\mu_\omega)$ and
% $\mathcal{P}^*[\mu_\omega]=\mathcal{P}^*(T,E_0,N,s;\mu_\omega)$ are given in
% equations \eqref{eq:System_Hamiltonian} and
% \eqref{eq:Entropy_ER_Fluid}, respectively. In
% \eqref{eq:Helmholtz_Variational_Form} we have held $T$, $N$ and $s$
% constant, neglected energies associated with geometric deformation,
% and included the work done by the battery $-\mathcal{P}^*[\mu]\;E_0$
% \cite{Wen_etal}. 

% The second law of thermodynamics provides a physical explanation as to
% why the variance of $\Hc^e_\omega$ vanishes as $E_0\to\infty$. As $E_0$
% increases quasistatically, work is performed on the ER fluid and, by
% the second law and equations \eqref{eq:Helmholtz_Variational_Form} and
% \eqref{eq:Free_Energy_Derivatives_E0}, the entropy $S$ and the
% polarization $\mathcal{P}^*$, hence $\epsilon^*$, increase in order to
% minimize $\mathcal{F}_\infty$. The entropy $S=S_0-|S^e|$ can increase by
% increasing $S_0$ and by decreasing $|S^e|$. Equation
% \eqref{eq:Entropy_ER_Fluid} implies $\partial S/\partial E_0<0$ hence $|S^e|$
% decreases with $E_0$, as $S_0$ is only dependent of the electric field
% through the geometric configuration $\omega\in\Omega$. By equation
% \eqref{eq:Entropy_ER_Fluid} the only way to decrease $|S^e|$ as $E_0$
% increases, is to decrease $\text{Var}(\Hc^e_\omega)$ like
% $o(E_0^{-\delta})$ for some $\delta>0$. By the principle of minimum internal
% energy, the decrease of $|S^e|$ and the increase of $S_0$ and
% $\mathcal{P}^*$ cause the system to settle into a non-varying minimum
% energy state. This argument also suggests that the mixing entropy
% \cite{Robertson} is contained within $S_0$ and excluded from
% $S^e$. The variational form of the Helmholtz free energy
% \eqref{eq:Helmholtz_Variational_Form} and a systematic, mathematically
% rigorous, analysis of the topics discussed in this section are central
% to our future work regarding ER fluids.  





% \begin{itemize}
% \item Discuss the limitations of the canonical ensemble in ER fluids
%   to sphere sizes $r_s\lesssim1\mu m$. 
% \end{itemize}
% %
% \subsection{The System Hamiltonian}
% \label{subsec:TACM_Hamiltonian}
% %
% We conclude this section with a final physical remark. The 
% energy constraint, $\langle\vec{D}\cdot\vec{E}_f\rangle=0$, requires that
% \emph{exactly half} of the external/internal interaction energy is
% counteracted by the interactions within the system, independent of
% all parameter values and the geometry of the configuration $\omega\in\Omega$. This
% suggests that the resultant charge density that forms on the contrast
% boundaries, as a consequence of the applied field, does so in a way
% that conserves this equality for every configuration
% $\omega\in\Omega$ and parameter values. This integral 
% constraint seems to impose some sort of energy conservation on the
% system. This conservation is mathematically summarized by the integral
% equations given in this section.
% %
% \begin{itemize}
% \item Give my physical derivation of the Hamiltonian here.
% \item In ER fluids, $\epsilon_1$ is the permittivity of the liquid, $\epsilon_2$ is
%   the permittivity of the spherical inclusions, and metal spheres are 
% modeled by letting $\epsilon_2\to\infty$ or $h\to0$ \cite{Wen:SoftMatter-200,Jackson-1999}.  
% \end{itemize}
%
\subsection{Maxwell's Relations}
\label{subsec:TACM_Maxwells_Rel}
\begin{itemize}
\item Do general setting $\Hc=\Hc(T,H_0(E_0,N),s)$ leading to
  commutators found during AK trip as Maxwell's relations.
\item Specify the Hamiltonian
  $\Hc=\Hc_0G_\omega(s)+\Hc_f(\omega)$. Do commutator
  analysis in the main text for $G_\omega(s)=g(s)G_\omega$ as a special case
  first linking the behavior of $s$, $T$, $N$ and $E_0$ (State as a
  theorem). 
\item Subsequently show the beautiful results in the general case and
  present the commutator analysis, using this full Hamiltonian, as a
  theorem in the appendix.   
\item  Cite the appendix and state the results
  $\text{Cov}(F,F^\prime)=\sum_i\text{Cov}(F,\Hc_f(\omega))_i$
  in the main text.
\end{itemize} 
%
\subsection{Perturbation and Asymptotic Analysis}
\label{subsec:TACM_Pert_Asymp_Analysis}
\begin{itemize}
\item Do Perturbation and Asymptotic Analysis for
  $\Hc=\Hc_0G_\omega(s)$ here.
\end{itemize}
%
\subsection{Mean Field Theory}
\label{subsec:TACM_MFT}
Is there a MFT ($\beta\Hc_0\to\infty$) similar to Dyson's approach for
binary composites?
%
\subsection{Numerical Results}
\label{subsec:TACM_Numerics}
\begin{itemize}
\item Present the phase transition of the RRN in $s$ and $E_0$,
  similar to the dipole model that I've already done.
\end{itemize}
%
%
\section{A General Dipole Model of ER Fluids}
\label{subsec:GDM}
%
\subsection{Introduction}
\label{subsec:GDM_Intro}
%
\subsection{Perturbation about Aligned Spins}
\label{subsec:GDM_Pert}
%
\subsection{The Percolation Phase Transition}
\label{subsec:GDM_Percolation}
%
\subsection{Mean Field Theory}
\label{subsec:GDM_MFT}
%
\subsection{Numerical Results}
\label{subsubsec:GDM_Numerics}
%
\subsection{Large Deviation Theory Under Gibbs Measures}
\label{subsec:LDT_Gibbs}
%
Indeed, denote
$\mathcal{M}(\mathcal{X})$ the space of finite signed measures on a
measurable space $(\mathcal{X},\mathscr{B})$ and
$\mathcal{M}_1(\mathcal{X})$ denote the subspace of probability
measures, where $\mathscr{B}$ denotes the set of Borel measurable
functions on the vector space $\mathcal{X}$. Let
$b\mathscr{B}\subset\mathscr{B}$ be the space of bounded measurable
functions \cite{Firas}.
%
\begin{definition}
  For $\nu,\lambda\in\mathcal{M}_1(\mathcal{X})$, the entropy of $\nu$ relative to
  $\lambda$, $H(\nu|\lambda)$ is defined by
  %
  \begin{align}\label{eq:Relative_Entropy}
   H(\nu|\lambda)=\left\{
        %
        \begin{array}{ll}
           \int\phi\log{\phi}d\lambda & \text{if } \nu\ll\lambda \text{ and }
                                         \phi=\frac{d\nu}{d\lambda} \\
           \infty          &  \text{otherwise.}                  
        \end{array}
        %
        \right.
  \end{align}
  %
\end{definition}
%
The integral in \eqref{eq:Relative_Entropy} is well--defined, as
$x\log{x}\geq-1/e$. The relative entropy is non--negative, $H(\nu|\lambda)\geq0$, and
$H(\nu|\lambda)=0$ if and only if $\nu=\lambda$ \cite{Firas}. Moreover $H(\nu|\lambda)$ is
strictly convex and has the variation representation \cite{Firas}
%
\begin{align}
  H(\nu|\lambda)=\sup_{g\in b\mathscr{B}}\{E^\nu[g]-\log{E^\lambda[e^g]}\},
\end{align}
%
where $E^\nu[\cdot]$ denotes the expectation with respect to the probability
measure $\nu$, i.e., $Z_\beta=E^\lambda[e^{-\beta\Hc}]$ is the partition
function at inverse temperature $\beta$. The MEP may be precisely stated
in the setting of Sanov's Theorem for empirical measures
\cite{Firas}. The MEP may be precisely stated as follows

% 
\noindent\textbf{The Maximum Entropy Principle} \textit{Let $\mathcal{S}$ be a
Polish space and suppose that the set $C\subset\mathcal{M}_1(\mathcal{S})$
is closed convex, and satisfies}
%
\begin{align}
  \inf_{\nu\in C}H(\nu|\lambda)=\inf_{\nu\in C^\circ}H(\nu|\lambda)<\infty,
\end{align}
%
\textit{where $C^\circ$ is the interior of $C$. Then, there is a unique
$\tilde{\nu}\in C$ that minimizes $H(\cdot|\lambda)$ over $C$}.
%
\begin{remark}
  The ``entropy'' in the name of the principle is thermodynamic
  entropy $-H(\nu|\lambda)$. Hence, minimizing $H(\cdot|\lambda)$ corresponds to
  maximizing thermodynamic entropy $H(\nu|\lambda)$.
\end{remark}
%
Use of the specific entropy $h(\nu|\lambda)$ one has the 
Briefly sum up the minimum Helmholtz principle (\cite{Firas}
page 81--92)?
The idea behind large deviation theory under Gibbs measures is as
follows. When the system is acted upon by a significant external
stimulus described by $\beta\Hc$, such as an applied electric
field or a heat bath, the behavior of the system, determined by the
probability measure $P(d\omega)f(\omega)$, deviates substantially from that
determined by the reference measure $P(d\omega)$ which governs the
stochastic properties of the system in the absence of such a stimulus  
\cite{Firas}. One can show that the global minimum of the entropy
$S=0$ is attained when $f=1$ for some $\omega\in\Omega$ (no uncertainty)
\cite{Sethna-2006,Firas}. Therefore the entropy is inherently positive, and
strictly positive for ER fluids $S>0$, an important property which
will be utilized below.  

%

%%%%%%%%%%%%%%%%%%%%%%%%%%%%%%%%%%%%%%%%%%%%%%%%%%%%%%%%%%%%%%%%%%%%%%%%%%
%%%%%%%%%%%%%%%%%%%%%%%%%%%%%%%%%%%%%%%%%%%%%%%%%%%%%%%%%%%%%%%%%%%%%%%%%%

\textbf{Electrically Driven Phase Transitions: The Microcanonical
  Ensemble}

%%%%%%%%%%%%%%%%%%%%%%%%%%%%%%%%%%%%%%%%%%%%%%%%%%%%%%%%%%%%%%%%%%%%%%%%%%
%%%%%%%%%%%%%%%%%%%%%%%%%%%%%%%%%%%%%%%%%%%%%%%%%%%%%%%%%%%%%%%%%%%%%%%%%%
\section{Background}\label{sec:MicCanEns_Background}
%
\begin{itemize}
\item Discuss the limitations of the canonical ensemble for ER fluids
  with $r_s\gtrsim10\mu m$.  
\item Give background of the microcanonical ensemble in terms of
  Hamiltonian dynamical systems.  
\item Define temperature as in Uline etal \cite{Uline:JCP:124301} and
  briefly discuss the lack of equipartitioning in the microcanonical
  ensemble.
\item Briefly introduce the configurational temperature, referring to
  section %\ref{subsec:MCE_Config_Temp}
  for a more detailed study of
  temperature in ER fluids.  
\item Review M. Campisi's theory using volume entropy.  
\item Derive the Polynomial representation of phase space volume (or
  surface area)  \cite{Uline:JCP:124301,Berdichevsky-1997} for
  $\Hc_\omega=\Hc_0G_\omega(s)$
\end{itemize}
%

\section{The Dipole Model for ER Fluids Revisited}
\begin{itemize}
\item Numerical results
\end{itemize}
%
\section{Critical Behavior of Transport in Lattice Percolation
  Models}
\begin{itemize}
\item Numerical results
\end{itemize}
%
\section{Configurational Temperature of Microstructure with Constraints}
\label{sec:MCE_Config_Temp}
\begin{itemize}
\item Give background on configurational temperature from H. H. Rugh,
  L. Lue and D. J. Evens, Y. Han and D. G. Grier, B. D. Butler etal,
  G. Rickayzen and J. G. Powles, J. Delhommelle and D. J. Evans,
  K. P. Travis and C. Braga, C. Baig and B. J. Edwards, N. Rathore et
  al, and O. G. Jepps et al.
\item Extend L. Lue and D. J. Evans configurational temperature with
  constraints for ER fluids and give supporting numerical simulations.    
\end{itemize}
%
\chapter{RANDOM MATRIX THEORY OF HOMOGENIZATION
             FOR COMPOSITE MATERIALS}
\label{ch:RMT_of_Composites}
%
\section{Background}
\label{sec:RMT_Background}
These ensembles, and hence 
the WD statistics, are independent of spatial dimensionality $d$ of a
physical system \cite{Canali}. They are also independent of the
coordinate basis as they are, by definition, invariant under
similarity transformations \cite{Mehta:2004:RMT}. This limits the applicability of these
ensembles to regimes of a physical system where all normalizable
linear combinations of eigenstates have similar properties and
dimensionality is in some sense irrelevant \cite{Canali}.

: $V(\lambda;q)\sim|\lambda|$ for $q=1^-$ and
$V(\lambda;q)\sim\ln^2|\lambda|$  for $q\ll1$ \cite{Muttalib_etal_qRME}.


In recent years a broad
range of mathematical techniques have been developed to study phase
transitions exhibited by such composites, revealing features which
are virtually ubiquitous in disordered systems.

%
\begin{itemize}
\item Briefly review Eigenvalue statistics: Balian's approach  
\item Brief review of Mehta's orthogonal polynomial method and its
q-deformed generalization, and the limits to the GOE here.
\item Briefly review Eigenvector statistics.
\end{itemize}

%
\section{Eigenvalue Statistics of Homogenization for Composites}
\label{sec:Eval_Stats_of_Composites}
%
\begin{itemize}
\item Introduction to the plasma model isomorphism of RMT
\item Variational principles: Pastur's results for $V(\lambda)\gtrsim(1+\epsilon)\ln|\lambda|$
\item Numerical results: Canali's fitting to the 1-d, 2-d, and 3-d RRN
\end{itemize}
%
\section{Eigenvector Statistics of Homogenization for Composites}
\label{sec:Evec_Stats_of_Composites}
%







%normal capitalization in subsection
%\subsection{}

%Spacing before and after equations.  I used the following \vspace
%command to correct the double spacing; use this also %with tables.  I
%sometimes had to use \vspace{-.15in} with tables 
%\vspace{-.1in}
%\[
% a_1 \geq a_2 \geq \cdots \geq a_k \geq 1.
% \vspace{-.1in}
%\]




 
%Bibliography goes on a new page
\newpage

%Use \nocite{*} to include references not directly referred to in the text
\nocite{*}
\backmatter
\vspace{\baselineskip}
\renewcommand{\baselinestretch}{1.0}\selectfont
\bibliographystyle{plain}

%It is help to keep the bibliography in a separate file.  In this case, the bibliography is kept in the file named %thesisBib.bib
\bibliography{murphy}
\vspace{-\baselineskip}






\newpage
\begin{appendix}
\addcontentsline{toc}{chapter}{APPENDIX}
 \setcounter{equation}{0}

\begin{center}APPENDIX \\C++ CODE\end{center}

\newpage

\end{appendix}

\end{document}
